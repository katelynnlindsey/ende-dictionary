

\section*{1.1 Sky}
\begin{entrylist}
\entry{ddapall}\headword{ddapall}{\pos{Noun}} {\definition{sky}}
\entry{kakakän}\headword{kakakän}{\pos{Noun}} {\definition{(outer) space}}
\entry{kakänkakän}\headword{kakänkakän}{\pos{Noun}} {\definition{air}}
\end{entrylist}

\section*{1.1.1 Sun}
\begin{entrylist}
\entry{bädab}\headword{bädab}{\pos{Intransitive S verb}} {\definition{to dawn, break}}
\entry{gazen}\headword{gazen}{\pos{Transitive S verb}} {\definition{to rise, come out; shine}}
\entry{gllae}\headword{gllae}{\pos{Intransitive S verb}} {\definition{to shine}}
\entry{kälakäle}\headword{kälakäle}{\pos{Intransitive S verb}} {\definition{to set (of the sun)}}
\entry{pänae}\headword{pänae}{\pos{Noun}} {\definition{zenith (position of the sun at high noon)}}
\entry{yäbäd}\headword{yäbäd}{\pos{Noun}} {\definition{sun}}
\end{entrylist}

\section*{1.1.1.1 Moon}
\begin{entrylist}
\entry{bädab}\headword{bädab}{\pos{Intransitive S verb}} {\definition{to shine brightly (of the moon)}}
\entry{ddänddäm}\headword{ddänddäm}{\pos{Intransitive S verb}} {\definition{to set (of the moon)}}
\entry{gazen}\headword{gazen}{\pos{Transitive S verb}} {\definition{to rise, come out; shine}}
\entry{kok}\headword{kok}{\pos{Noun}} {\definition{moon}}
\entry{kok apte}\headword{kok apte}{\pos{Noun}} {\definition{half moon}}
\entry{kok indrang}\headword{kok indrang}{\pos{Noun}} {\definition{moonlight}}
\entry{kok kuddäll}\headword{kok kuddäll}{\pos{Noun}} {\definition{new moon}}
\entry{kok kätt pälläk}\headword{kok kätt pälläk}{\pos{Noun}} {\definition{crescent moon}}
\entry{kok pllayang}\headword{kok pllayang}{\pos{Noun}} {\definition{full moon}}
\entry{kok pälläk}\headword{kok pälläk}{\pos{Noun}} {\definition{first quarter moon}}
\entry{kok sisor}\headword{kok sisor}{\pos{Noun}} {\definition{first light (the end of a new moon)}}
\entry{kokta}\headword{kokta}{\pos{Noun}} {\definition{moonlight}}
\end{entrylist}

\section*{1.1.1.2 Star}
\begin{entrylist}
\entry{bädab piro}\headword{bädab piro}{\pos{Noun}} {\definition{morning star}}
\entry{piro}\headword{piro}{\pos{Noun}} {\definition{star}}
\entry{piro kanas}\headword{piro kanas}{\pos{Noun}} {\definition{type of game where the first person to see a star in the sky wins}}
\entry{pɨnypɨny}\headword{pɨnypɨny}{\pos{Transitive S verb}} {\definition{many stars appearing}}
\end{entrylist}

\section*{1.1.2 Air}
\begin{entrylist}
\entry{tuk}\headword{tuk}{\pos{Noun}} {\definition{air}}
\end{entrylist}

\section*{1.1.2.1 Blow air}
\begin{entrylist}
\entry{popllem}\headword{popllem}{\pos{Intransitive S verb}} {\definition{to flap}}
\entry{pädoe}\headword{pädoe}{\pos{Transitive S verb}} {\definition{to blow}}
\end{entrylist}

\section*{1.1.3 Weather}
\begin{entrylist}
\entry{bawa}\headword{bawa}{\pos{Noun}} {\definition{rain shower}}
\entry{bawa minyminy}\headword{bawa minyminy}{\pos{Noun}} {\definition{season characterized by hunting and fishing in light rain (tenth season; corresponds to July)}}
\entry{gameny}\headword{gameny}{\pos{Intransitive verb}} {\definition{to become cloudy, when dark rain clouds are forming}}
\entry{nyaraman}\headword{nyaraman}{\pos{Noun}} {\definition{fine day}}
\entry{obergaban}\headword{obergaban}{\pos{Noun}} {\definition{clear day}}
\entry{yogoll}\headword{yogoll}{\pos{Noun}} {\definition{rain}}
\entry{yogoll kutt}\headword{yogoll kutt}{\pos{Noun}} {\definition{black cloud}}
\entry{yäbäd}\headword{yäbäd}{\pos{Noun}} {\definition{weather}}
\entry{yäbäd bäng}\headword{yäbäd bäng}{\pos{Noun}} {\definition{dry, hot season characterized by burning grass (twelfth season; corresponds to September)}}
\entry{yäbäd kuttakuttang}\headword{yäbäd kuttakuttang}{\pos{Modifier}} {\definition{really hot}}
\end{entrylist}

\section*{1.1.3.1 Wind}
\begin{entrylist}
\entry{amiyeamiye}\headword{amiyeamiye}{\pos{Adverb}} {\definition{upwind, against the wind}}
\entry{kiwale wel}\headword{kiwale wel}{\pos{Noun}} {\definition{type of wind}}
\entry{kllomokllomoll}\headword{kllomokllomoll}{\pos{Adverb}} {\definition{downwind, leeward}}
\entry{kämag}\headword{kämag}{\pos{Noun}} {\definition{west wind; windy storm from the west}}
\entry{popllem}\headword{popllem}{\pos{Intransitive S verb}} {\definition{to flap}}
\entry{pädoe}\headword{pädoe}{\pos{Transitive S verb}} {\definition{to blow}}
\entry{singosingol}\headword{singosingol}{\pos{Adverb}} {\definition{upwind, windward}}
\entry{wel}\headword{wel}{\pos{Noun}} {\definition{wind}}
\end{entrylist}

\section*{1.1.3.2 Cloud}
\begin{entrylist}
\entry{ddapall käkan}\headword{ddapall käkan}{\pos{Noun}} {\definition{cloud}}
\entry{ddapall källa}\headword{ddapall källa}{\pos{Noun}} {\definition{cloud}}
\entry{ibe}\headword{ibe}{\pos{Noun}} {\definition{mist; fog}}
\entry{kut}\headword{kut}{\pos{Noun}} {\definition{raincloud}}
\entry{säresäremang}\headword{säresäremang}{\pos{Modifier}} {\definition{cloudy, dim}}
\end{entrylist}

\section*{1.1.3.3 Rain}
\begin{entrylist}
\entry{bawa sarasaram}\headword{bawa sarasaram}{\pos{Noun}} {\definition{drizzle}}
\entry{daindaim}\headword{daindaim}{\pos{Noun}} {\definition{drizzle}}
\entry{däbae}\headword{däbae}{\pos{Intransitive S verb}} {\definition{to drizzle}}
\entry{ketrop}\headword{ketrop}{\pos{Transitive A verb}} {\definition{to stop the rain}}
\entry{kut}\headword{kut}{\pos{Noun}} {\definition{raincloud}}
\entry{läläm}\headword{läläm}{\pos{Noun}} {\definition{muddy spot}}
\entry{mame}\headword{mame}{\pos{Intransitive S verb}} {\definition{to fall, rain}}
\end{entrylist}

\section*{1.1.3.5 Storm}
\begin{entrylist}
\entry{kämag}\headword{kämag}{\pos{Noun}} {\definition{west wind; windy storm from the west}}
\end{entrylist}

\section*{1.1.3.6 Lightning, thunder}
\begin{entrylist}
\entry{ddäddäll}\headword{ddäddäll}{\pos{Noun}} {\definition{thunder}}
\entry{gwara}\headword{gwara}{\pos{Noun}} {\definition{lightning}}
\end{entrylist}

\section*{1.1.3.7 Flood}
\begin{entrylist}
\entry{bob}\headword{bob}{\pos{Noun}} {\definition{flood}}
\end{entrylist}

\section*{1.1.3.8 Drought}
\begin{entrylist}
\entry{walle gutu}\headword{walle gutu}{\pos{Noun}} {\definition{dry riverbed}}
\entry{yäbäd bäng}\headword{yäbäd bäng}{\pos{Noun}} {\definition{drought}}
\end{entrylist}

\section*{1.2.1 Land}
\begin{entrylist}
\entry{ekaklle}\headword{ekaklle}{\pos{Noun}} {\definition{land}}
\entry{ekaklle}\headword{ekaklle}{\pos{Noun}} {\definition{ground}}
\entry{tutu}\headword{tutu}{\pos{Noun}} {\definition{land}}
\end{entrylist}

\section*{1.2.1.1 Mountain}
\begin{entrylist}
\entry{durgu}\headword{durgu}{\pos{Noun}} {\definition{cliff}}
\entry{tutu}\headword{tutu}{\pos{Noun}} {\definition{mountain, hill}}
\end{entrylist}

\section*{1.2.1.3 Plain, plateau}
\begin{entrylist}
\entry{mattmett}\headword{mattmett}{\pos{Noun}} {\definition{plain, flat area}}
\entry{poddpodd}\headword{poddpodd}{\pos{Noun}} {\definition{plain, field, clearing}}
\end{entrylist}

\section*{1.2.1.4 Valley}
\begin{entrylist}
\entry{kopek}\headword{kopek}{\pos{Noun}} {\definition{valley}}
\entry{kup}\headword{kup}{\pos{Noun}} {\definition{valley}}
\end{entrylist}

\section*{1.2.1.6 Forest, grassland, desert}
\begin{entrylist}
\entry{ap}\headword{ap}{\pos{Noun}} {\definition{grassland, savannah}}
\entry{awe}\headword{awe}{\pos{Noun}} {\definition{savannah}}
\entry{goro}\headword{goro}{\pos{Modifier}} {\definition{lush, overgrown, wild}}
\entry{goro wälläng}\headword{goro wälläng}{\pos{Noun}} {\definition{jungle}}
\entry{pob}\headword{pob}{\pos{Noun}} {\definition{savanna}}
\entry{pudd}\headword{pudd}{\pos{Noun}} {\definition{place with flattened grass where wallabies sleep}}
\entry{pɨnyapɨnye}\headword{pɨnyapɨnye}{\pos{Noun}} {\definition{area with burnt grass}}
\entry{sabana}\headword{sabana}{\pos{Noun}} {\definition{savanna}}
\entry{surum}\headword{surum}{\pos{Noun}} {\definition{sand}}
\entry{wälläng}\headword{wälläng}{\pos{Noun}} {\definition{backwoods, hinterland, (Australia) bush (any rural, undeveloped landscape)}}
\end{entrylist}

\section*{1.2.1.7 Earthquake}
\begin{entrylist}
\entry{ruriruri}\headword{ruriruri}{\pos{Noun}} {\definition{earthquake}}
\end{entrylist}

\section*{1.2.2.1 Soil, dirt}
\begin{entrylist}
\entry{itbonmäll}\headword{itbonmäll}{\pos{Noun}} {\definition{dirt}}
\entry{läläm}\headword{läläm}{\pos{Noun}} {\definition{muddy spot}}
\entry{pale}\headword{pale}{\pos{Noun}} {\definition{type of white clay used for painting babies}}
\entry{pollgo suwe}\headword{pollgo suwe}{\pos{Noun}} {\definition{dirt left on skin after coming out of the water}}
\entry{surum}\headword{surum}{\pos{Noun}} {\definition{sand}}
\entry{täpe}\headword{täpe}{\pos{Noun}} {\definition{mud}}
\entry{wärpir}\headword{wärpir}{\pos{Noun}} {\definition{slippery mud found by the river}}
\entry{yorkoll}\headword{yorkoll}{\pos{Noun}} {\definition{dirt}}
\end{entrylist}

\section*{1.2.2.2 Rock}
\begin{entrylist}
\entry{boser}\headword{boser}{\pos{Noun}} {\definition{rock found in creek}}
\entry{dädär}\headword{dädär}{\pos{Noun}} {\definition{stone, rock}}
\entry{dädär käp}\headword{dädär käp}{\pos{Noun}} {\definition{stone}}
\entry{dɨdɨr}\headword{dɨdɨr}{\pos{Noun}} {\definition{rock}}
\entry{maza}\headword{maza}{\pos{Noun}} {\definition{reef}}
\entry{ttägäll käp}\headword{ttägäll käp}{\pos{Noun}} {\definition{stone (esp. one used for cooking in mumus)}}
\end{entrylist}

\section*{1.2.2.3 Metal}
\begin{entrylist}
\entry{auri}\headword{auri}{\pos{Noun}} {\definition{metal}}
\end{entrylist}

\section*{1.2.2.4 Mineral}
\begin{entrylist}
\entry{pänpän}\headword{pänpän}{\pos{Noun}} {\definition{calcium hydroxide, lime}}
\entry{surum}\headword{surum}{\pos{Noun}} {\definition{sand}}
\end{entrylist}

\section*{1.2.3.1 Liquid}
\begin{entrylist}
\entry{lɨklɨk}\headword{lɨklɨk}{\pos{Intransitive S verb}} {\definition{to melt}}
\end{entrylist}

\section*{1.2.3.2 Oil}
\begin{entrylist}
\entry{gi}\headword{gi}{\pos{Noun}} {\definition{grease, fat}}
\entry{owel}\headword{owel}{\pos{Noun}} {\definition{oil}}
\end{entrylist}

\section*{1.2.3.3 Gas}
\begin{entrylist}
\entry{ddol}\headword{ddol}{\pos{Noun}} {\definition{foam, bubbles, gas}}
\entry{käk}\headword{käk}{\pos{Noun}} {\definition{bubble}}
\end{entrylist}

\section*{1.3 Water}
\begin{entrylist}
\entry{bawa pokallmäll}\headword{bawa pokallmäll}{\pos{Noun}} {\definition{wave}}
\entry{bobag}\headword{bobag}{\pos{Noun}} {\definition{flooded}}
\entry{darkukiny}\headword{darkukiny}{\pos{Noun}} {\definition{type of grass}}
\entry{ddage llätt}\headword{ddage llätt}{\pos{Noun}} {\definition{river source}}
\entry{gall}\headword{gall}{\pos{Noun}} {\definition{canoe, boat}}
\entry{go}\headword{go}{\pos{Noun}} {\definition{drain}}
\entry{gägäb ine}\headword{gägäb ine}{\pos{Noun}} {\definition{dew}}
\entry{ine}\headword{ine}{\pos{Noun}} {\definition{water; liquid}}
\entry{kantärpie}\headword{kantärpie}{\pos{Noun}} {\definition{log stuck in the water}}
\entry{kuddäb}\headword{kuddäb}{\pos{Noun}} {\definition{raft}}
\entry{käza burala}\headword{käza burala}{\pos{Noun}} {\definition{water lily}}
\entry{mällänggäbe}\headword{mällänggäbe}{\pos{Noun}} {\definition{type of plant that grows along the riverside}}
\entry{pätpät}\headword{pätpät}{\pos{Transitive S verb}} {\definition{to dry}}
\entry{saper ine}\headword{saper ine}{\pos{Noun}} {\definition{clean water}}
\entry{tame}\headword{tame}{\pos{Noun}} {\definition{wave (of water)}}
\entry{totkoll}\headword{totkoll}{\pos{Noun}} {\definition{puddle}}
\entry{trongoe}\headword{trongoe}{\pos{Transitive S verb}} {\definition{to check a bucket or container for water}}
\entry{täränga}\headword{täränga}{\pos{Modifier}} {\definition{low, scant, shallow}}
\entry{tɨn}\headword{tɨn}{\pos{Noun}} {\definition{steam}}
\entry{walle menae}\headword{walle menae}{\pos{Noun}} {\definition{bank, shore, water's edge}}
\end{entrylist}

\section*{1.3.1 Bodies of water}
\begin{entrylist}
\entry{gallgall}\headword{gallgall}{\pos{Noun}} {\definition{bank, coast, shore}}
\entry{källäm}\headword{källäm}{\pos{Noun}} {\definition{pond; lagoon}}
\entry{walle}\headword{walle}{\pos{Noun}} {\definition{body of water}}
\end{entrylist}

\section*{1.3.1.1 Ocean, lake}
\begin{entrylist}
\entry{bem}\headword{bem}{\pos{Noun}} {\definition{sea, ocean}}
\entry{maza}\headword{maza}{\pos{Noun}} {\definition{reef}}
\entry{orbam}\headword{orbam}{\pos{Noun}} {\definition{marsh}}
\end{entrylist}

\section*{1.3.1.2 Swamp}
\begin{entrylist}
\entry{karama}\headword{karama}{\pos{Noun}} {\definition{swamp}}
\entry{läläm}\headword{läläm}{\pos{Noun}} {\definition{muddy spot}}
\entry{orbam}\headword{orbam}{\pos{Noun}} {\definition{marsh}}
\entry{pu}\headword{pu}{\pos{Noun}} {\definition{floating grass or island in swamp}}
\entry{pu}\headword{pu}{\pos{Noun}} {\definition{swamp garden}}
\entry{tawa}\headword{tawa}{\pos{Noun}} {\definition{swamp}}
\entry{ttalme}\headword{ttalme}{\pos{Noun}} {\definition{type of floating grass that grass, deer, and wallaby eat}}
\entry{zagu}\headword{zagu}{\pos{Noun}} {\definition{type of sugarcane-like plant that grows in the swamp}}
\end{entrylist}

\section*{1.3.1.3 River}
\begin{entrylist}
\entry{bun}\headword{bun}{\pos{Noun}} {\definition{mouth (of a river)}}
\entry{ddage}\headword{ddage}{\pos{Noun}} {\definition{stream, tributary}}
\entry{guem}\headword{guem}{\pos{Noun}} {\definition{channel; deep part of river}}
\entry{ine källa}\headword{ine källa}{\pos{Noun}} {\definition{mud, clay}}
\entry{mutt}\headword{mutt}{\pos{Noun}} {\definition{river source}}
\entry{toengg}\headword{toengg}{\pos{Noun}} {\definition{confluence, junction (of a river)}}
\entry{walle ddage}\headword{walle ddage}{\pos{Noun}} {\definition{river, stream}}
\entry{walle gutu}\headword{walle gutu}{\pos{Noun}} {\definition{dry riverbed}}
\entry{walle mäg}\headword{walle mäg}{\pos{Noun}} {\definition{creek; river source}}
\end{entrylist}

\section*{1.3.1.4 Spring, well}
\begin{entrylist}
\entry{ine bib}\headword{ine bib}{\pos{Noun}} {\definition{spring (water source)}}
\entry{ine kup}\headword{ine kup}{\pos{Noun}} {\definition{well (water source)}}
\entry{ine ma}\headword{ine ma}{\pos{Noun}} {\definition{well (water source)}}
\entry{kup}\headword{kup}{\pos{Noun}} {\definition{well}}
\entry{tatuma}\headword{tatuma}{\pos{Noun}} {\definition{washing place, outdoor bathing area}}
\entry{yatän}\headword{yatän}{\pos{Transitive S verb}} {\definition{to fetch (water)}}
\end{entrylist}

\section*{1.3.1.5 Island, shore}
\begin{entrylist}
\entry{kui}\headword{kui}{\pos{Noun}} {\definition{island}}
\end{entrylist}

\section*{1.3.2.1 Flow}
\begin{entrylist}
\entry{waowaem}\headword{waowaem}{\pos{Noun}} {\definition{current}}
\entry{wawaem}\headword{wawaem}{\pos{Noun}} {\definition{current}}
\entry{wawaem}\headword{wawaem}{\pos{Intransitive A verb}} {\definition{to flow}}
\end{entrylist}

\section*{1.3.2.2 Pour}
\begin{entrylist}
\entry{gollab}\headword{gollab}{\pos{Intransitive S verb}} {\definition{to pour}}
\entry{zämae}\headword{zämae}{\pos{Transitive S verb}} {\definition{to pour, put, transfer}}
\end{entrylist}

\section*{1.3.2.6 Tide}
\begin{entrylist}
\entry{bob}\headword{bob}{\pos{Noun}} {\definition{flood}}
\entry{käkan}\headword{käkan}{\pos{Noun}} {\definition{tide}}
\end{entrylist}

\section*{1.3.3 Wet}
\begin{entrylist}
\entry{bebe}\headword{bebe}{\pos{Intransitive S verb}} {\definition{to leak, excrete}}
\entry{käpät}\headword{käpät}{\pos{Noun}} {\definition{moisture}}
\entry{käpätang}\headword{käpätang}{\pos{Modifier}} {\definition{wet}}
\end{entrylist}

\section*{1.3.3.1 Dry}
\begin{entrylist}
\entry{baddbedd}\headword{baddbedd}{\pos{Intransitive S verb}} {\definition{to dry up}}
\entry{dorko}\headword{dorko}{\pos{Modifier}} {\definition{dry}}
\entry{dädär}\headword{dädär}{\pos{Modifier}} {\definition{dry}}
\entry{däroledärole}\headword{däroledärole}{\pos{Modifier}} {\definition{dry}}
\entry{päräll}\headword{päräll}{\pos{Modifier}} {\definition{dry}}
\end{entrylist}

\section*{1.3.4 Be in water}
\begin{entrylist}
\entry{gllae}\headword{gllae}{\pos{Transitive S verb}} {\definition{to float, drift; paddle}}
\entry{kämbäg}\headword{kämbäg}{\pos{Intransitive S verb}} {\definition{to baptize}}
\entry{peyam}\headword{peyam}{\pos{Intransitive S verb}} {\definition{to come out from water, surface}}
\end{entrylist}

\section*{1.3.6 Water quality}
\begin{entrylist}
\entry{bib}\headword{bib}{\pos{Noun}} {\definition{spring water}}
\entry{bɨk}\headword{bɨk}{\pos{Noun}} {\definition{poisoned creek}}
\end{entrylist}

\section*{1.4.1 Dead things}
\begin{entrylist}
\entry{kuddäll}\headword{kuddäll}{\pos{Modifier}} {\definition{dead}}
\end{entrylist}

\section*{1.4.2 Spirits of things}
\begin{entrylist}
\entry{idd}\headword{idd}{\pos{Noun}} {\definition{ghost}}
\end{entrylist}

\section*{1.5 Plant}
\begin{entrylist}
\entry{apgllu}\headword{apgllu}{\pos{Noun}} {\definition{small plant with a root that makes a maroon pigment for dyeing grass skirts when mixed with ash (e.g. Acacia ash or coconut ash). When not mixed with ash, it makes a yellow pigment.}}
\entry{bädma}\headword{bädma}{\pos{Noun}} {\definition{type of medicinal plant}}
\entry{bädmaol}\headword{bädmaol}{\pos{Noun}} {\definition{small sago flower}}
\entry{dadäp}\headword{dadäp}{\pos{Noun}} {\definition{plant part}}
\entry{dara}\headword{dara}{\pos{Noun}} {\definition{vine type}}
\entry{däm}\headword{däm}{\pos{Noun}} {\definition{plant}}
\entry{däränggedärängge}\headword{däränggedärängge}{\pos{Noun}} {\definition{large wild orchid}}
\entry{gamo}\headword{gamo}{\pos{Noun}} {\definition{plant type}}
\entry{gamu}\headword{gamu}{\pos{Noun}} {\definition{type of ginger with flat leaves; used as medicine for centipede bites and as bait for catching flying foxes; chew it first and the flying fox will eat it and become lethargic}}
\entry{gugall}\headword{gugall}{\pos{Noun}} {\definition{type of plant with red, yellow, and white flowers and fruit with small, round seeds; children use the fruit in bamboo blow guns}}
\entry{kausar gllall}\headword{kausar gllall}{\pos{Noun}} {\definition{pandanus type}}
\entry{kire}\headword{kire}{\pos{Modifier}} {\definition{unripe; raw, fresh}}
\entry{kitar}\headword{kitar}{\pos{Noun}} {\definition{floating grass}}
\entry{komotupi}\headword{komotupi}{\pos{Noun}} {\definition{long ginger}}
\entry{kukoll}\headword{kukoll}{\pos{Modifier}} {\definition{healthy, fertile, vibrant}}
\entry{kupi käp}\headword{kupi käp}{\pos{Noun}} {\definition{type of fruit (~4 cm in diameter) that children shoot at birds}}
\entry{kämtupi}\headword{kämtupi}{\pos{Noun}} {\definition{type of ginger}}
\entry{käza bädma}\headword{käza bädma}{\pos{Noun}} {\definition{type of plant that only the crocodile clan wears when going hunting for crocodiles; also a traditional medicine}}
\entry{mägda koll}\headword{mägda koll}{\pos{Noun}} {\definition{plant term}}
\entry{mänmänpitepite}\headword{mänmänpitepite}{\pos{Noun}} {\definition{type of plant that grows by the river; eaten by wallabies}}
\entry{tentatente}\headword{tentatente}{\pos{Noun}} {\definition{category of parasitic plants}}
\entry{ttaba}\headword{ttaba}{\pos{Noun}} {\definition{plant type}}
\entry{ttäbe}\headword{ttäbe}{\pos{Noun}} {\definition{a strong smelling plant whose bark, called ttäbe kollop, was traditionally worn around the neck to give fragrance and perfume smell. It was sometimes chewed and rubbed around the body and head to stop headache. The orignal purpose was also for protection from evil spirits.}}
\entry{upeupe}\headword{upeupe}{\pos{Noun}} {\definition{type of plant with a single stem and edible, tall fruit near the base of stem}}
\entry{wadär}\headword{wadär}{\pos{Noun}} {\definition{type of grass; cane}}
\entry{widwid}\headword{widwid}{\pos{Noun}} {\definition{type of plant with big leaves}}
\entry{woboll}\headword{woboll}{\pos{Noun}} {\definition{type of plant}}
\entry{wälläng gllall}\headword{wälläng gllall}{\pos{Noun}} {\definition{pandanus type}}
\entry{yid}\headword{yid}{\pos{Noun}} {\definition{liquid extracted from a plant}}
\end{entrylist}

\section*{1.5.1 Tree}
\begin{entrylist}
\entry{adräl}\headword{adräl}{\pos{Noun}} {\definition{type of tree}}
\entry{aduwi}\headword{aduwi}{\pos{Noun}} {\definition{type of tree}}
\entry{aebaeb}\headword{aebaeb}{\pos{Noun}} {\definition{type of tree}}
\entry{aetruaetru}\headword{aetruaetru}{\pos{Noun}} {\definition{type of tree with edible blue fruits and white flowers, found in the bush and along big creeks}}
\entry{aitar}\headword{aitar}{\pos{Noun}} {\definition{type of palm tree that grows along creeks with pools; shoots and soft trunk are edible; similar to dumar}}
\entry{allko wallägnewallägnen}\headword{allko wallägnewallägnen}{\pos{Noun}} {\definition{type of sago}}
\entry{amba}\headword{amba}{\pos{Noun}} {\definition{type of tree found in the grassland}}
\entry{angkäpäll}\headword{angkäpäll}{\pos{Noun}} {\definition{type of big tree found in the bush with white flowers and fruit, which cassowary eat when they fall}}
\entry{asiasi}\headword{asiasi}{\pos{Noun}} {\definition{type of large tree that grows in the bush and along big creeks, with white flowers and big red fruits eaten by cassowary and children; the trunk is used to make canoes}}
\entry{awäll}\headword{awäll}{\pos{Noun}} {\definition{type of tree that grows in the bush (where yam gardens are made) with white and blue flowers and dark, edible fruit}}
\entry{badar}\headword{badar}{\pos{Noun}} {\definition{type of tree that grows in the grassland with white flowers}}
\entry{band}\headword{band}{\pos{Noun}} {\definition{type of tree}}
\entry{bebe}\headword{bebe}{\pos{Noun}} {\definition{type of pandanus with long fruit (~2 feet)}}
\entry{beyat}\headword{beyat}{\pos{Noun}} {\definition{type of tree that grows in the bush with wood used for house sticks}}
\entry{big}\headword{big}{\pos{Noun}} {\definition{type of very large tree that grows in the bush with wood used for firewood, especially when camping.}}
\entry{bikme}\headword{bikme}{\pos{Noun}} {\definition{type of palm tree with hanging, poisonous yellow and green fruits that can be eaten after being buried by the creek for up to 2 years and then cooked on the fire}}
\entry{bisel}\headword{bisel}{\pos{Noun}} {\definition{type of sago that grows tall and wide}}
\entry{biwiz}\headword{biwiz}{\pos{Noun}} {\definition{type of large tree that grows in the bush with purple flowers and wood used for kundu drums and canoes}}
\entry{bllolla}\headword{bllolla}{\pos{Noun}} {\definition{type of tree}}
\entry{bob}\headword{bob}{\pos{Noun}} {\definition{type of tree}}
\entry{boddo}\headword{boddo}{\pos{Noun}} {\definition{type of large tree with a big trunk, fruit that deer eat, and aerial prop roots}}
\entry{boe}\headword{boe}{\pos{Noun}} {\definition{type of cultivated tree with edible indigo fruit and white flowers that attract birds and butterflies}}
\entry{bollga}\headword{bollga}{\pos{Noun}} {\definition{type of sago}}
\entry{bomall}\headword{bomall}{\pos{Noun}} {\definition{type of tree that grows in the bush with bark used to make sago baskets}}
\entry{bombom}\headword{bombom}{\pos{Noun}} {\definition{type of tree with big leaves and soft wood that floats and is carved by children}}
\entry{bunkombunkom}\headword{bunkombunkom}{\pos{Noun}} {\definition{type of tall, big tree that grows along creek with wood used for canoes}}
\entry{buz}\headword{buz}{\pos{Noun}} {\definition{type of cultivated tree with fist-sized, edible green fruit and light purple flowers}}
\entry{bäbnge}\headword{bäbnge}{\pos{Noun}} {\definition{type of palm with coconuts with a light yellow and green exocarp}}
\entry{bäd}\headword{bäd}{\pos{Noun}} {\definition{type of large tree that grows in the grassland with wood used for firewood and bark used for building walls}}
\entry{bädde}\headword{bädde}{\pos{Noun}} {\definition{type of large tree found by rivers}}
\entry{bägallem}\headword{bägallem}{\pos{Noun}} {\definition{type of tree that grows in the grassland with wood used for firewood and yellow fruit}}
\entry{bägem}\headword{bägem}{\pos{Noun}} {\definition{type of tree with pink flowers and fist-sized fruit with a pit}}
\entry{bägäm}\headword{bägäm}{\pos{Noun}} {\definition{type of tree found by creeks with bark used for weaving bags or strong rope}}
\entry{bäkän}\headword{bäkän}{\pos{Noun}} {\definition{type of cultivated plant with leaves eaten with sago}}
\entry{bälltoe}\headword{bälltoe}{\pos{Noun}} {\definition{type of tree}}
\entry{bälläg}\headword{bälläg}{\pos{Noun}} {\definition{Areca palm}}
\entry{bämäng}\headword{bämäng}{\pos{Noun}} {\definition{type of tree}}
\entry{bändam}\headword{bändam}{\pos{Noun}} {\definition{type of tree}}
\entry{bänzibänzi}\headword{bänzibänzi}{\pos{Noun}} {\definition{type of sago}}
\entry{bät}\headword{bät}{\pos{Noun}} {\definition{type of tree}}
\entry{bätäny}\headword{bätäny}{\pos{Noun}} {\definition{type of tree}}
\entry{bɨd}\headword{bɨd}{\pos{Noun}} {\definition{gum tree}}
\entry{ddadd}\headword{ddadd}{\pos{Noun}} {\definition{type of large tree with white flowers and blue fruit; wood used for making canoes}}
\entry{ddage}\headword{ddage}{\pos{Noun}} {\definition{branch}}
\entry{ddallwe}\headword{ddallwe}{\pos{Noun}} {\definition{type of tree}}
\entry{ddoga}\headword{ddoga}{\pos{Noun}} {\definition{type of tree}}
\entry{dem}\headword{dem}{\pos{Noun}} {\definition{type of sago used for paints}}
\entry{digodigol}\headword{digodigol}{\pos{Noun}} {\definition{type of tree}}
\entry{digol}\headword{digol}{\pos{Noun}} {\definition{type of big, tall tree that grows near gardens in the bush with white flowers, green fruits, and special, valuable red wood that is used for ax handles}}
\entry{dimes}\headword{dimes}{\pos{Noun}} {\definition{type of cultivated tree with sour, mango-like fruit}}
\entry{dini}\headword{dini}{\pos{Noun}} {\definition{type of small tree that grows in the bush with white flowers and red fruit that cassowaries like to eat}}
\entry{dinidini}\headword{dinidini}{\pos{Noun}} {\definition{type of small tree that grows in the bush with red fruit}}
\entry{dirindi}\headword{dirindi}{\pos{Noun}} {\definition{type of tree}}
\entry{diromdirom}\headword{diromdirom}{\pos{Noun}} {\definition{type of small tree with round, flat, edible fruit that are green and red when ripe}}
\entry{dogma}\headword{dogma}{\pos{Noun}} {\definition{type of tree that grows in the bush with wood that smells like matches and is good for house posts}}
\entry{domäll}\headword{domäll}{\pos{Noun}} {\definition{type of pandanus}}
\entry{dowa}\headword{dowa}{\pos{Noun}} {\definition{type of tree that grows in the grassland along creeks with wood used for firewood}}
\entry{dradre}\headword{dradre}{\pos{Noun}} {\definition{type of tree that grows in swamp with edible, currant-sized blue fruit}}
\entry{dubllodubllom}\headword{dubllodubllom}{\pos{Noun}} {\definition{type of tree that grows in the bush with yellow fruit that have a hard seed, in which there is an edible nut}}
\entry{dum}\headword{dum}{\pos{Noun}} {\definition{type of big tree that grows in bush with strong wood used for making canoes and paddles, sap used to paint bowstrings or to patch holes, yellow flowers, and green fruit; planted near house for shade}}
\entry{dumbi}\headword{dumbi}{\pos{Noun}} {\definition{type of red tree}}
\entry{dur}\headword{dur}{\pos{Noun}} {\definition{type of medium-sized bamboo that grows along creeks; used for dancing; young plants used for cooking sago}}
\entry{duwel}\headword{duwel}{\pos{Noun}} {\definition{type of tall tree that grows in the bush near yam gardens}}
\entry{däba}\headword{däba}{\pos{Noun}} {\definition{type of tree that grows in the grassland with leaves used to wrap sago and durable wood used for kundu drums, house posts, and formerly, bridges}}
\entry{dämar}\headword{dämar}{\pos{Noun}} {\definition{type of palm tree (~2 m) that used to be cooked and eaten; also fed to pigs}}
\entry{dändäräm}\headword{dändäräm}{\pos{Noun}} {\definition{type of small tree with white and purple flowers and many hard, marble-sized, green fruit that children play with}}
\entry{dänyäk}\headword{dänyäk}{\pos{Noun}} {\definition{small plant that grows in the grassland with purple and white flowers and blue fruit that children like to eat}}
\entry{därmir}\headword{därmir}{\pos{Noun}} {\definition{type of tree that is used to treat sores}}
\entry{därängbun}\headword{därängbun}{\pos{Noun}} {\definition{type of pandanus with a curved fruit shaped like a dog's head}}
\entry{därängge}\headword{därängge}{\pos{Noun}} {\definition{small orchid with blue, yellow, white, purple flowers}}
\entry{gae nge}\headword{gae nge}{\pos{Noun}} {\definition{type of palm with coconuts with a red or green exocarp; the husk is chewed and the young coconut water is drunk}}
\entry{gaimbi}\headword{gaimbi}{\pos{Noun}} {\definition{type of fruit tree}}
\entry{gangan}\headword{gangan}{\pos{Noun}} {\definition{type of tree}}
\entry{gaora}\headword{gaora}{\pos{Noun}} {\definition{type of sago}}
\entry{gaugau}\headword{gaugau}{\pos{Noun}} {\definition{type of big tree that grows in the bush along creeks with wood used for canoes}}
\entry{geawe}\headword{geawe}{\pos{Noun}} {\definition{type of big tree that grows along creeks with bright purple flowers and wood used for canoes}}
\entry{ger}\headword{ger}{\pos{Noun}} {\definition{type of big tree that grows in the bush}}
\entry{giragirag}\headword{giragirag}{\pos{Noun}} {\definition{type of tree}}
\entry{god}\headword{god}{\pos{Noun}} {\definition{type of cultivated fruit tree similar to dimes tree}}
\entry{gogo}\headword{gogo}{\pos{Noun}} {\definition{varieties of palms with coconuts with a dark green exocarp}}
\entry{golgol}\headword{golgol}{\pos{Noun}} {\definition{type of tree that grows in the bush with a straight trunk and wood used for house sticks}}
\entry{gollolla}\headword{gollolla}{\pos{Noun}} {\definition{type of palm}}
\entry{gonzagonzar}\headword{gonzagonzar}{\pos{Noun}} {\definition{type of tree}}
\entry{goral}\headword{goral}{\pos{Noun}} {\definition{type of tree}}
\entry{guli}\headword{guli}{\pos{Noun}} {\definition{type of tree}}
\entry{gullem suwetar}\headword{gullem suwetar}{\pos{Noun}} {\definition{type of tree that is used as medicine for snake bites and to repel snakes}}
\entry{gullme}\headword{gullme}{\pos{Noun}} {\definition{type of flowering tree that grows on the riverside in swamps with wood used for firewood and long, hanging red flowers that smell nice}}
\entry{guwaba}\headword{guwaba}{\pos{Noun}} {\definition{guava tree; water steeped with its leaves is used to wash sores}}
\entry{guziguzi}\headword{guziguzi}{\pos{Noun}} {\definition{type of tree}}
\entry{gwaga}\headword{gwaga}{\pos{Noun}} {\definition{type of big tree that grows in the bush with big, edible fruit that are yellow outside, red inside, and stain lips brown; when ripe, it will open and drop the heart-shaped seed}}
\entry{gwälläd}\headword{gwälläd}{\pos{Noun}} {\definition{type of tree}}
\entry{gäbgäb}\headword{gäbgäb}{\pos{Noun}} {\definition{type of tree}}
\entry{gäl}\headword{gäl}{\pos{Noun}} {\definition{type of tree that grows in the bush with white flowers, brown seeds, and fruit with a yellow pericarp and green exocarp}}
\entry{gäleb}\headword{gäleb}{\pos{Noun}} {\definition{type of tree}}
\entry{gällall}\headword{gällall}{\pos{Noun}} {\definition{type of pandanus}}
\entry{gällatater}\headword{gällatater}{\pos{Noun}} {\definition{type of tree}}
\entry{gälle}\headword{gälle}{\pos{Noun}} {\definition{type of big tree that grows in the bush near big creeks with wood used for canoes, white and blue flowers, and edible red fruit}}
\entry{iddnge}\headword{iddnge}{\pos{Noun}} {\definition{type of coconut palm}}
\entry{iddoiddob}\headword{iddoiddob}{\pos{Noun}} {\definition{type of tree}}
\entry{indre}\headword{indre}{\pos{Noun}} {\definition{type of tree that grows in the grassland with edible brown-yellow fruit and good, long-burning firewood}}
\entry{inmol}\headword{inmol}{\pos{Noun}} {\definition{type of small tree that grows in the bush along creeks; used to weave grass skirts after being pounded}}
\entry{ip}\headword{ip}{\pos{Noun}} {\definition{type of tree that grows in the grassland with bark that is chewed and sap used as an adhesive or poured on a spear to strengthen it}}
\entry{irwe}\headword{irwe}{\pos{Noun}} {\definition{type of cultivated tree with white flowers and juicy, red and white fruit with two seeds; used to treat cough}}
\entry{itaita}\headword{itaita}{\pos{Noun}} {\definition{type of big tree that grows in the bush with red fruit and a straight trunk used for timber}}
\entry{kaekae}\headword{kaekae}{\pos{Noun}} {\definition{type of small plant with blue and purple flowers and black fruit that cassowaries eat}}
\entry{kaembre}\headword{kaembre}{\pos{Noun}} {\definition{type of tree}}
\entry{kaepse}\headword{kaepse}{\pos{Noun}} {\definition{type of tree that grows in the bush with white flowers and big, yellow fruit that cassowaries and deer eat}}
\entry{kalemtoe}\headword{kalemtoe}{\pos{Noun}} {\definition{type of tree that grows in the bush with small, long, thin edible nuts that must be broken with stone; nuts are edible when steamed and are also eaten by cassowaries and doves; flesh is used for cream; similar to toe tree}}
\entry{kamuka}\headword{kamuka}{\pos{Noun}} {\definition{type of cultivated thorny citrus tree with small white flowers and softball-size fruit with thick green skin}}
\entry{kamäkamät}\headword{kamäkamät}{\pos{Noun}} {\definition{type of big tree that grows in the bush with yellow flowers and big yellow fruit that are collected}}
\entry{kang}\headword{kang}{\pos{Noun}} {\definition{sucker (additional unwanted shoot that grows by the base of a tree)}}
\entry{kanoe}\headword{kanoe}{\pos{Noun}} {\definition{type of tree with fruits that cassowary, pigs, and deer eat and bark that is put in the water to kill fish}}
\entry{kapang}\headword{kapang}{\pos{Noun}} {\definition{Acacia}}
\entry{kapräl}\headword{kapräl}{\pos{Noun}} {\definition{type of tree}}
\entry{ketmar}\headword{ketmar}{\pos{Noun}} {\definition{type of tree that grows in old gardens; after being skinned and soaked, it is weaved into skirts}}
\entry{kinpop}\headword{kinpop}{\pos{Noun}} {\definition{type of tree that grows in the bush with white flowers and wood used for house sticks and rafters}}
\entry{kito}\headword{kito}{\pos{Noun}} {\definition{type of black palm; in this immature stage, the shoots eaten as medicine and used to make baskets Used to build house and mat in the bush}}
\entry{kobe}\headword{kobe}{\pos{Noun}} {\definition{type of tree that grows along creeks (~ 3 m) with white and red or blue and purple flowers and edible red fruit with black or white stripes and 4-6 seeds inside}}
\entry{kobädd}\headword{kobädd}{\pos{Noun}} {\definition{type of tree}}
\entry{koeme}\headword{koeme}{\pos{Noun}} {\definition{type of tree that grows along creeks with edible, round red fruit}}
\entry{koemekoeme}\headword{koemekoeme}{\pos{Noun}} {\definition{type of tree that grows along creeks with inedible, small red fruit}}
\entry{koenbäll}\headword{koenbäll}{\pos{Noun}} {\definition{type of tree that grows in the bush (especially in old gardens) with hanging green fruit and liquid used to treat sores}}
\entry{kokall}\headword{kokall}{\pos{Noun}} {\definition{type of palm with fruit that are smaller than coconuts}}
\entry{kokallkokall}\headword{kokallkokall}{\pos{Noun}} {\definition{type of tree}}
\entry{kokne}\headword{kokne}{\pos{Noun}} {\definition{type of tree that grows in the grassland with blue flowers and edible blue fruit}}
\entry{kokpe}\headword{kokpe}{\pos{Noun}} {\definition{type of tree with yellow flowers and leaves that are used to wrap food for cooking on the fire}}
\entry{koktakokta}\headword{koktakokta}{\pos{Noun}} {\definition{when dead trees or tree branches become dry and shine in the night}}
\entry{kol}\headword{kol}{\pos{Noun}} {\definition{sago pith}}
\entry{kolem}\headword{kolem}{\pos{Noun}} {\definition{type of palm with coconuts that come in red or green varieties}}
\entry{kollko}\headword{kollko}{\pos{Noun}} {\definition{breadfruit}}
\entry{kollkokollko}\headword{kollkokollko}{\pos{Noun}} {\definition{type of tree that grows in bush with edible red fruit with an edible, big round nut in the seed}}
\entry{kollong}\headword{kollong}{\pos{Noun}} {\definition{type of tree that grows in the bush; used for grass skirts}}
\entry{kolwa}\headword{kolwa}{\pos{Noun}} {\definition{type of tree that grows in grassland with a trunk that has a diameter of ~9 cm, but very strong and used for spears and weapons}}
\entry{koplle}\headword{koplle}{\pos{Noun}} {\definition{type of big tree that grows in the bush with fruit that cassowaries eat}}
\entry{kubllu}\headword{kubllu}{\pos{Noun}} {\definition{type of tree that grows in the bush and savannah with bark used for string; similar to kapang}}
\entry{kud}\headword{kud}{\pos{Noun}} {\definition{type of pandanus with fat triangular fruit}}
\entry{kukpi}\headword{kukpi}{\pos{Noun}} {\definition{type of tree that grows in the bush with big fruit that children play with}}
\entry{kungge}\headword{kungge}{\pos{Noun}} {\definition{type of tree}}
\entry{kunob}\headword{kunob}{\pos{Noun}} {\definition{type of tree that grows around creeks with light wood used for house sticks}}
\entry{kunu}\headword{kunu}{\pos{Noun}} {\definition{type of short tree that grows in the grassland with poisonous bark for stunning fish}}
\entry{kutt llo}\headword{kutt llo}{\pos{Noun}} {\definition{type of tree that grows in the bush with bark that is scraped and rubbed on sores}}
\entry{kuyu}\headword{kuyu}{\pos{Noun}} {\definition{type of tree that grows in the bush or along creeks with soft wood that rots easily}}
\entry{kwakall}\headword{kwakall}{\pos{Noun}} {\definition{type of tree}}
\entry{kwallang}\headword{kwallang}{\pos{Noun}} {\definition{type of bush used for posts}}
\entry{kwantta}\headword{kwantta}{\pos{Noun}} {\definition{type of tree that grows in the grassland (~9 m) with green and yellow leaves used as bow pigment and hard fruit used to play a hockey-like game; also used for posts; liquid is extracted from the bark and given to dogs when they show signs of sickness}}
\entry{kwata}\headword{kwata}{\pos{Noun}} {\definition{type of tree that grows in the bush; used for house sticks}}
\entry{kwätäs}\headword{kwätäs}{\pos{Noun}} {\definition{type of tree}}
\entry{käbädral}\headword{käbädral}{\pos{Noun}} {\definition{type of tree that grows in the bush with sturdy wood}}
\entry{käbäll}\headword{käbäll}{\pos{Noun}} {\definition{type of tree that grows in the bush with soft wood used for canoe paddles}}
\entry{kädgal}\headword{kädgal}{\pos{Noun}} {\definition{type of tree that grows in the bush}}
\entry{kädkäd}\headword{kädkäd}{\pos{Transitive S verb}} {\definition{to remove bark, debark}}
\entry{käg}\headword{käg}{\pos{Noun}} {\definition{type of palm with wood used for flooring}}
\entry{käkpäl}\headword{käkpäl}{\pos{Noun}} {\definition{type of tree that grows in the bush; burned to fertilize the ground}}
\entry{käkäpyo}\headword{käkäpyo}{\pos{Noun}} {\definition{type of tree that grows in the grassland and savannah with flowers that wallaby and deer eat}}
\entry{källayoyo}\headword{källayoyo}{\pos{Noun}} {\definition{type of tree that grows in the bush with leaves used as toilet paper}}
\entry{kämsir}\headword{kämsir}{\pos{Noun}} {\definition{type of tree the grows in the bush with fruit that is black outside and green and red inside and is eaten by cassowaries; similar to sir}}
\entry{käpkumett}\headword{käpkumett}{\pos{Noun}} {\definition{type of tree}}
\entry{käpom}\headword{käpom}{\pos{Noun}} {\definition{type of big tree that grows in the bush with white flowers and edible white fruit; used as medicine for cough}}
\entry{kättlla}\headword{kättlla}{\pos{Noun}} {\definition{type of tree that grows in the bush with white flowers}}
\entry{käzapig}\headword{käzapig}{\pos{Noun}} {\definition{type of tree with big, sour, black fruit and white and blue flowers}}
\entry{lepade}\headword{lepade}{\pos{Noun}} {\definition{type of cultivated tree with purple and white flowers and edible black fruit that kids like to eat}}
\entry{lesna}\headword{lesna}{\pos{Noun}} {\definition{type of tree}}
\entry{linge}\headword{linge}{\pos{Noun}} {\definition{type of palm tree with blue flowers}}
\entry{llakällakätt}\headword{llakällakätt}{\pos{Noun}} {\definition{type of tree that grows in the bush; used for house sticks}}
\entry{llapu}\headword{llapu}{\pos{Noun}} {\definition{type of big tree that grows in the bush with red fruit and white flowers}}
\entry{llapuyurwe}\headword{llapuyurwe}{\pos{Noun}} {\definition{type of tree with seedless, edible red fruit}}
\entry{llo}\headword{llo}{\pos{Noun}} {\definition{tree}}
\entry{llo ddage}\headword{llo ddage}{\pos{Noun}} {\definition{tree branch}}
\entry{llo ngoeang}\headword{llo ngoeang}{\pos{Noun}} {\definition{forked tree branch}}
\entry{llo patt}\headword{llo patt}{\pos{Noun}} {\definition{fallen tree}}
\entry{llo ttam}\headword{llo ttam}{\pos{Noun}} {\definition{tree leaf}}
\entry{llowawi}\headword{llowawi}{\pos{Noun}} {\definition{type of tree}}
\entry{lonsis}\headword{lonsis}{\pos{Noun}} {\definition{lemon plant}}
\entry{ma ttängäm}\headword{ma ttängäm}{\pos{Noun}} {\definition{type of tree}}
\entry{mab}\headword{mab}{\pos{Noun}} {\definition{pandanus}}
\entry{madmed}\headword{madmed}{\pos{Noun}} {\definition{type of big tree that grows in the bush}}
\entry{maemae}\headword{maemae}{\pos{Noun}} {\definition{type of big tree that grows in places where yam gardens are planted; used for making canoes and paddles}}
\entry{mai}\headword{mai}{\pos{Noun}} {\definition{type of sago}}
\entry{maiwa}\headword{maiwa}{\pos{Noun}} {\definition{type of pandanus with a long, smooth fruit cooked in mumu}}
\entry{mandri}\headword{mandri}{\pos{Noun}} {\definition{cultivated lemon tree}}
\entry{manggo}\headword{manggo}{\pos{Noun}} {\definition{mango tree}}
\entry{mare}\headword{mare}{\pos{Noun}} {\definition{type of pandanus}}
\entry{matamata}\headword{matamata}{\pos{Noun}} {\definition{type of tree that grows along swamps with white and blue flowers and edible fruit (black outside, red inside) that ripen during January and February}}
\entry{mekae}\headword{mekae}{\pos{Noun}} {\definition{type of tree with white flowers and edible nuts that children eat}}
\entry{minggore manggo}\headword{minggore manggo}{\pos{Noun}} {\definition{type of tree}}
\entry{mintor}\headword{mintor}{\pos{Noun}} {\definition{type of tree with yellow flowers that bloom in June and July and a root is used as a kind of hockey stick}}
\entry{miriwa}\headword{miriwa}{\pos{Noun}} {\definition{type of pandanus}}
\entry{mismis}\headword{mismis}{\pos{Noun}} {\definition{type of tree used for firewood}}
\entry{moepo}\headword{moepo}{\pos{Noun}} {\definition{type of tree that grows in the bush with white flowers, inedible red fruit, and poisonous seeds and bark; an indicator that the soil is fertile and good for making a garden}}
\entry{moepotatae}\headword{moepotatae}{\pos{Noun}} {\definition{type of tree}}
\entry{mokoll}\headword{mokoll}{\pos{Noun}} {\definition{type of tree that grows in the bush with thick bark, white flowers, and small green fruit}}
\entry{moksir}\headword{moksir}{\pos{Noun}} {\definition{type of tree}}
\entry{moll}\headword{moll}{\pos{Noun}} {\definition{type of tree that grows around Malam with white flowers}}
\entry{momea gäl}\headword{momea gäl}{\pos{Noun}} {\definition{type of tree}}
\entry{mompara}\headword{mompara}{\pos{Noun}} {\definition{type of tree that grows near swamps and creeks with white flowers and yellow and brown fruit that ripen in April and May and are eaten by deer}}
\entry{mondo}\headword{mondo}{\pos{Noun}} {\definition{type of tree that grows in the bush with green or purple fruits that animals eat}}
\entry{mopmop}\headword{mopmop}{\pos{Noun}} {\definition{type of tree that grows in the grassland (~3 m) with white flowers and red fruit that are eaten to treat cough}}
\entry{mupni}\headword{mupni}{\pos{Noun}} {\definition{type of cultivated mango tree with fruit with yellow skin and a white interior that is juiced}}
\entry{mäga}\headword{mäga}{\pos{Noun}} {\definition{type of sago}}
\entry{mägäll}\headword{mägäll}{\pos{Noun}} {\definition{type of tree with bark used to tie spears, white flowers, edible leaves, and finger-sized, edible red fruit with edible seeds}}
\entry{mällakutang}\headword{mällakutang}{\pos{Noun}} {\definition{type of tree that grows in the bush with white flowers, black bark, and wood used for house sticks}}
\entry{mällät}\headword{mällät}{\pos{Noun}} {\definition{type of tree that grows in the grassland with fruit used to ignite fires and yellow flowers with edible nectar}}
\entry{män}\headword{män}{\pos{Noun}} {\definition{type of big tree cultivated in the bush with white flowers and blue and purple fruit that cassowary eat; tobacco is planted in the soil near this tree after it is burned}}
\entry{mäta}\headword{mäta}{\pos{Noun}} {\definition{type of tree with a red stem, young green leaves, and bark used as rope when old; big trees are used to light fire; liquid is extracted and drunk to treat cough}}
\entry{mätar onyang}\headword{mätar onyang}{\pos{Noun}} {\definition{type of tree}}
\entry{mätämätär}\headword{mätämätär}{\pos{Noun}} {\definition{type of tree}}
\entry{ngatt}\headword{ngatt}{\pos{Noun}} {\definition{type of tree}}
\entry{nge}\headword{nge}{\pos{Noun}} {\definition{coconut palm}}
\entry{nora}\headword{nora}{\pos{Noun}} {\definition{type of big introduced tree with red flowers that attract birds}}
\entry{nyeny}\headword{nyeny}{\pos{Noun}} {\definition{type of large tree that grows along the river and in swamps with white flowers, brown fruit, and white bark used for making mumu and topping a roof}}
\entry{nängga}\headword{nängga}{\pos{Noun}} {\definition{type of pandanus with bunches of fruit with strong shells; they fall one at a time}}
\entry{oboll}\headword{oboll}{\pos{Noun}} {\definition{type of small tree that grows in the grassland (~2 m) with white flowers and red fruit with a large brown seed; fruit is eaten by birds; wood is used for sturdy posts}}
\entry{olmopäga}\headword{olmopäga}{\pos{Noun}} {\definition{type of sago}}
\entry{omawe}\headword{omawe}{\pos{Noun}} {\definition{type of tree}}
\entry{opa}\headword{opa}{\pos{Noun}} {\definition{type of tree}}
\entry{opa ttangtte}\headword{opa ttangtte}{\pos{Noun}} {\definition{type of tree}}
\entry{pall tawe}\headword{pall tawe}{\pos{Noun}} {\definition{type of large palm that grows in the grassland with white flowers and coconuts with a red exocarp; small sticks are good for houses, and bark is used for walling}}
\entry{panda}\headword{panda}{\pos{Noun}} {\definition{type of tree}}
\entry{patt}\headword{patt}{\pos{Noun}} {\definition{tree trunk (fallen)}}
\entry{pattlle}\headword{pattlle}{\pos{Noun}} {\definition{type of small bamboo that grows in the bush and along creeks; used to cook sago and make flutes; may be burned for fertile land for planting}}
\entry{pewäl}\headword{pewäl}{\pos{Noun}} {\definition{type of palm}}
\entry{pig}\headword{pig}{\pos{Noun}} {\definition{type of cultivated tree with edible fruit}}
\entry{pimbyom}\headword{pimbyom}{\pos{Noun}} {\definition{small pieces of bark}}
\entry{pin}\headword{pin}{\pos{Noun}} {\definition{type of tree with white or red flowers and composite fruit that birds eat}}
\entry{poma}\headword{poma}{\pos{Noun}} {\definition{type of pandanus with leaves that women weave into mats}}
\entry{pomila}\headword{pomila}{\pos{Noun}} {\definition{type of citrus tree with big, edible fruit}}
\entry{pondo}\headword{pondo}{\pos{Noun}} {\definition{type of tree that grows in the swamp and old gardens with yellow flowers and pod-shaped fruit with inedible seeds}}
\entry{ponong}\headword{ponong}{\pos{Noun}} {\definition{type of tree that grows in grassland with white flowers and small green fruit and wood used for posts}}
\entry{potkam}\headword{potkam}{\pos{Noun}} {\definition{type of cultivated tree that also grows in the savanna; treats cough and aches}}
\entry{potopoto}\headword{potopoto}{\pos{Noun}} {\definition{type of tree that grows near swamps and creeks with yellow and white flowers and fruit that falls on the ground; animals eat it}}
\entry{pällämpälläm yurwe}\headword{pällämpälläm yurwe}{\pos{Noun}} {\definition{type of tree with white flowers and edible white fruit.}}
\entry{pättäl}\headword{pättäl}{\pos{Noun}} {\definition{type of tree that grows near swamps with yellow flowers and bark used for rope to tie firewood}}
\entry{sakar}\headword{sakar}{\pos{Noun}} {\definition{type of edible pandanus}}
\entry{sana}\headword{sana}{\pos{Noun}} {\definition{sago}}
\entry{sanasana}\headword{sanasana}{\pos{Noun}} {\definition{type of edible sago}}
\entry{sawasap}\headword{sawasap}{\pos{Noun}} {\definition{type of cultivated tree with edible yellow or green fruit}}
\entry{sem}\headword{sem}{\pos{Noun}} {\definition{type of tree that grows in the bush; used for making rope}}
\entry{sigip}\headword{sigip}{\pos{Noun}} {\definition{type of palm with fruit hanging from long pedicels; people chew the fruit like betelnut}}
\entry{siporo}\headword{siporo}{\pos{Noun}} {\definition{type of cultivated tree with thorns and sour yellow fruit}}
\entry{sir}\headword{sir}{\pos{Noun}} {\definition{type of tree that grows in the bush with white flowers and edible black fruit with one seed inside; cassowary and pigs eat the fruit}}
\entry{sisi}\headword{sisi}{\pos{Noun}} {\definition{type of pandanus with a trunk used for wood}}
\entry{so}\headword{so}{\pos{Noun}} {\definition{type of black palm; in this mature stage (~3 m), the wood is used for house flooring and containers for squeezing sago, and the pith is eaten}}
\entry{sosoga}\headword{sosoga}{\pos{Noun}} {\definition{type of sago}}
\entry{surusuru}\headword{surusuru}{\pos{Noun}} {\definition{type of tree that grows in the bush with white flowers and strong wood used for house posts and firewood; it was traditionally lit and used as a light at night because it burns slowly}}
\entry{säkar}\headword{säkar}{\pos{Noun}} {\definition{type of big palm tree that grows near big rivers with white flowers and hanging red fruit that cassowaries eat}}
\entry{sɨmell källamokott}\headword{sɨmell källamokott}{\pos{Noun}} {\definition{type of spiky tree that grows in the bush with white flowers, yellow fruit, and hardwood used for firewood}}
\entry{sɨmellkom}\headword{sɨmellkom}{\pos{Noun}} {\definition{type of sago}}
\entry{tae}\headword{tae}{\pos{Noun}} {\definition{type of tall tree that grows by rivers with white flowers, green fruit, and sturdy wood that can be used as a bridge}}
\entry{taemataema}\headword{taemataema}{\pos{Noun}} {\definition{type of tree that grows near the swamp with long yellow flowers and leaves that are used to treat fungal skin infections}}
\entry{tall}\headword{tall}{\pos{Noun}} {\definition{type of tree with wood used for posts and easily-peeling bark used for walling; also lit as a torch}}
\entry{tan}\headword{tan}{\pos{Noun}} {\definition{type of short plant with white flowers, brown fruit, and branches used as a broom}}
\entry{tanteny}\headword{tanteny}{\pos{Noun}} {\definition{type of tree}}
\entry{tarme koeme}\headword{tarme koeme}{\pos{Noun}} {\definition{type of tree}}
\entry{tawe}\headword{tawe}{\pos{Noun}} {\definition{type of slim, tall palm with coconuts that come in red or green varieties, bunches of yellow fruit that birds eat, and fronds used for camp flooring}}
\entry{tep}\headword{tep}{\pos{Noun}} {\definition{tree sap (used in making drums and arrows)}}
\entry{toe}\headword{toe}{\pos{Noun}} {\definition{type of tree that grows in the bush with white flowers, black fruit, small edible nuts that doves and cassowaries eat, and pith that is steamed or extracted}}
\entry{tokop}\headword{tokop}{\pos{Noun}} {\definition{type of tree that grows in the bush; used as house sticks and medicine}}
\entry{tonggo}\headword{tonggo}{\pos{Noun}} {\definition{type of small bamboo that grows in the bush along creeks with a red interior; sharp when split and used as a cutting tool}}
\entry{topotopoll}\headword{topotopoll}{\pos{Noun}} {\definition{type of tree that grows in the bush with white flowers; used as a yam stick}}
\entry{tot}\headword{tot}{\pos{Noun}} {\definition{type of tree that gows along creeks with white and blue flowers and bark used to weave bags or sago baskets}}
\entry{ttalamttalam}\headword{ttalamttalam}{\pos{Noun}} {\definition{type of tree}}
\entry{ttall ip}\headword{ttall ip}{\pos{Noun}} {\definition{type of tree that grows in the bush, grassland, and along creeks with indigo flowers, bark that is chewed, and liquid used as glue for spears}}
\entry{ttall nge}\headword{ttall nge}{\pos{Noun}} {\definition{type of palm with yellow leaves and coconuts with a yellow exocarp}}
\entry{ttall ttoe}\headword{ttall ttoe}{\pos{Noun}} {\definition{type of tree}}
\entry{ttek}\headword{ttek}{\pos{Noun}} {\definition{type of tree that grows in the grassland with white flowers and sap used as glue for spears}}
\entry{ttoep}\headword{ttoep}{\pos{Noun}} {\definition{type of tree}}
\entry{ttongttong}\headword{ttongttong}{\pos{Noun}} {\definition{type of tree}}
\entry{ttottoem}\headword{ttottoem}{\pos{Noun}} {\definition{type of tree that grows inthe swamp and along creeks with a mango-like, inedible fruit that is yellow when ripe}}
\entry{ttäk}\headword{ttäk}{\pos{Noun}} {\definition{type of tree that grows on floating grass (~3 m) with white or yellow flowers, green fruit, and a big trunk used to by children to float or for carfts}}
\entry{ttäle}\headword{ttäle}{\pos{Noun}} {\definition{type of tree}}
\entry{ttällma tränymägäll}\headword{ttällma tränymägäll}{\pos{Noun}} {\definition{tree type}}
\entry{ttämbe}\headword{ttämbe}{\pos{Noun}} {\definition{type of big tree that grows in the bush with blue, purple, and white flowers and red fruit}}
\entry{ttämbe role}\headword{ttämbe role}{\pos{Noun}} {\definition{type of tree}}
\entry{ttän}\headword{ttän}{\pos{Noun}} {\definition{type of tree that grows in the grassland (~60 m) with yellow flowers, small green fruit, and wood used for house posts}}
\entry{ttättawe}\headword{ttättawe}{\pos{Noun}} {\definition{type of tree}}
\entry{ttɨp}\headword{ttɨp}{\pos{Noun}} {\definition{type of sago}}
\entry{tugul}\headword{tugul}{\pos{Noun}} {\definition{type of big tree that grows in the bush with green, leaflike flowers and straight wood used for house sticks}}
\entry{tuwetuwe}\headword{tuwetuwe}{\pos{Noun}} {\definition{type of small tree that grows in the bush with white flowers and edible red fruit}}
\entry{tuwi}\headword{tuwi}{\pos{Noun}} {\definition{type of tree with white flowers, yellow fruit, and wood used for posts}}
\entry{tuwok}\headword{tuwok}{\pos{Noun}} {\definition{type of tree}}
\entry{täb}\headword{täb}{\pos{Noun}} {\definition{type of tree}}
\entry{täbe}\headword{täbe}{\pos{Noun}} {\definition{type of big tree that grows in the bush with white flowers and drupes that fall with an edible seed inside}}
\entry{täbäll pud}\headword{täbäll pud}{\pos{Noun}} {\definition{type of tree}}
\entry{täkla}\headword{täkla}{\pos{Noun}} {\definition{tree type}}
\entry{täl}\headword{täl}{\pos{Noun}} {\definition{type of large bamboo that grows anywhere; used for bows, bow strings, axe handles, spears, and clothing pegs}}
\entry{täpäll}\headword{täpäll}{\pos{Noun}} {\definition{type of pandanus used in a traditional hairstyle and woven together to make a big mat}}
\entry{täral}\headword{täral}{\pos{Noun}} {\definition{type of tree that grows in the grassland with white flowers, brown fruit, and wood used for house posts}}
\entry{täral pällämpälläm}\headword{täral pällämpälläm}{\pos{Noun}} {\definition{type of tree}}
\entry{tätkea}\headword{tätkea}{\pos{Noun}} {\definition{type of sago}}
\entry{tɨt}\headword{tɨt}{\pos{Noun}} {\definition{type of tree found in the bush}}
\entry{ugeuge}\headword{ugeuge}{\pos{Noun}} {\definition{type of tree}}
\entry{ullegäll}\headword{ullegäll}{\pos{Noun}} {\definition{type of tree that grows in the grassland with white flowers, black fruits, and red nuts}}
\entry{upiye}\headword{upiye}{\pos{Noun}} {\definition{type of tree used to make kwib charcoal}}
\entry{upoupoll}\headword{upoupoll}{\pos{Noun}} {\definition{type of tree}}
\entry{upye}\headword{upye}{\pos{Noun}} {\definition{type of tree with white flowers and black fruit that produces a black pigment}}
\entry{uriar}\headword{uriar}{\pos{Noun}} {\definition{type of palm with coconuts with a purple exocarp}}
\entry{uttang ttatta}\headword{uttang ttatta}{\pos{Noun}} {\definition{type of sago}}
\entry{wabeyawabeya}\headword{wabeyawabeya}{\pos{Noun}} {\definition{type of tree}}
\entry{waetwaet}\headword{waetwaet}{\pos{Noun}} {\definition{type of tree}}
\entry{wap}\headword{wap}{\pos{Noun}} {\definition{stick}}
\entry{waramawarama}\headword{waramawarama}{\pos{Noun}} {\definition{type of tree that grows in the bush with white flowers and edible yellow fruit}}
\entry{wasar}\headword{wasar}{\pos{Noun}} {\definition{type of edible palm}}
\entry{wawa}\headword{wawa}{\pos{Noun}} {\definition{type of tree that grows in the bush and along creeks white flowers and blue fruit}}
\entry{wib}\headword{wib}{\pos{Noun}} {\definition{type of tree}}
\entry{wibell}\headword{wibell}{\pos{Noun}} {\definition{type of tree}}
\entry{wilwil}\headword{wilwil}{\pos{Noun}} {\definition{type of tree that grows in the bush}}
\entry{wipell}\headword{wipell}{\pos{Noun}} {\definition{type of tall palm that grows along the riverside}}
\entry{wiswis}\headword{wiswis}{\pos{Noun}} {\definition{type of tree that grows in the bush with white flowers and edible orange fruit}}
\entry{wiyowe}\headword{wiyowe}{\pos{Noun}} {\definition{type of large palm that grows in the bush or along creeks with flowers that start from the top and spread downwards and white fruit that hangs like coconut}}
\entry{wizarab}\headword{wizarab}{\pos{Noun}} {\definition{type of pandanus with red fruit}}
\entry{wädwäd}\headword{wädwäd}{\pos{Noun}} {\definition{type of tree}}
\entry{wädɨwädɨg}\headword{wädɨwädɨg}{\pos{Noun}} {\definition{type of tree}}
\entry{wägba}\headword{wägba}{\pos{Noun}} {\definition{type of tree that grows in the bush with white flowers, bark used as medicine, and strong wood used for posts; helps make dogs' noses more sensitive}}
\entry{wälep}\headword{wälep}{\pos{Noun}} {\definition{type of tree that grows in the bush with blue flowers and small blue fruit}}
\entry{wällegäll}\headword{wällegäll}{\pos{Noun}} {\definition{type of tree with fruit that are black and edible when ripe, leaves used to roll cigarettes, and roots used to treat toothache or asthma}}
\entry{wällwäll}\headword{wällwäll}{\pos{Noun}} {\definition{type of tree}}
\entry{wälsa}\headword{wälsa}{\pos{Noun}} {\definition{type of tree that grows in the bush}}
\entry{wängän}\headword{wängän}{\pos{Noun}} {\definition{type of tree}}
\entry{wäno}\headword{wäno}{\pos{Noun}} {\definition{type of tree that grows in the grassland and along creeks with white flowers, small brown fruit, and bark used on rooves}}
\entry{yaber}\headword{yaber}{\pos{Noun}} {\definition{type of tree that grows in the bush with white flowers and poisonous bark used to catch fish}}
\entry{yagäl}\headword{yagäl}{\pos{Noun}} {\definition{type of tree that grows in the grassland with leaves used to sand bows and spears}}
\entry{yante}\headword{yante}{\pos{Noun}} {\definition{type of large tree that grows in the grassland with white flowers and wood used for house sticks}}
\entry{yarte}\headword{yarte}{\pos{Noun}} {\definition{type of tree with young wood used for house sticks}}
\entry{yobeg}\headword{yobeg}{\pos{Noun}} {\definition{type of cultivated shrub with white and yellow flowers and long leaves used to tie yam shoots to yam sticks}}
\entry{yorko}\headword{yorko}{\pos{Noun}} {\definition{type of large cane found in the bush}}
\entry{yubud}\headword{yubud}{\pos{Noun}} {\definition{type of tree}}
\entry{yuddädda}\headword{yuddädda}{\pos{Noun}} {\definition{type of palm with branches used for armbands}}
\entry{yure}\headword{yure}{\pos{Noun}} {\definition{type of sago}}
\entry{yuru}\headword{yuru}{\pos{Noun}} {\definition{type of pandanus}}
\entry{yurwe}\headword{yurwe}{\pos{Noun}} {\definition{type of tree}}
\entry{yäbäyäbäd}\headword{yäbäyäbäd}{\pos{Noun}} {\definition{type of tree that grows in the bush with white flowers and red fruit}}
\entry{yäbäyäbäl}\headword{yäbäyäbäl}{\pos{Noun}} {\definition{type of tree}}
\entry{yämak}\headword{yämak}{\pos{Noun}} {\definition{type of big tree found in the bush and by the river}}
\entry{yärmuyärmu}\headword{yärmuyärmu}{\pos{Noun}} {\definition{type of tree}}
\entry{yäru}\headword{yäru}{\pos{Noun}} {\definition{type of small tree with thorns}}
\entry{zib}\headword{zib}{\pos{Noun}} {\definition{type of big tree that grows in the bush}}
\end{entrylist}

\section*{1.5.2 Bush, shrub}
\begin{entrylist}
\entry{bllablla}\headword{bllablla}{\pos{Noun}} {\definition{type of cordyline with big leaves, red fruit, and leaves that are used to fan fire and tied around the waist and chest for dancing}}
\entry{daendae}\headword{daendae}{\pos{Noun}} {\definition{flowering plant that is said to be ancestral to the area, with many types; flowers used as adornment when dancing}}
\entry{dänäk}\headword{dänäk}{\pos{Noun}} {\definition{type of small bush with reddish fruit that is black when ripe}}
\entry{mompel}\headword{mompel}{\pos{Noun}} {\definition{aibika}}
\entry{piya}\headword{piya}{\pos{Noun}} {\definition{type of flowering plant}}
\entry{pollon}\headword{pollon}{\pos{Noun}} {\definition{type of small bush}}
\entry{pällonpällon}\headword{pällonpällon}{\pos{}} {\definition{}}
\entry{rata}\headword{rata}{\pos{Noun}} {\definition{type of flowering plant with red bracts and nectar that attracts insects}}
\entry{sapiri}\headword{sapiri}{\pos{Noun}} {\definition{type of flowering plant}}
\end{entrylist}

\section*{1.5.3 Grass, herb, vine}
\begin{entrylist}
\entry{apapun}\headword{apapun}{\pos{Noun}} {\definition{type of short grass that disperses seeds by attaching to animal fur}}
\entry{baob}\headword{baob}{\pos{Noun}} {\definition{water lily}}
\entry{bogel}\headword{bogel}{\pos{Noun}} {\definition{seaweed}}
\entry{bolod}\headword{bolod}{\pos{Noun}} {\definition{sugarcane}}
\entry{bolwod}\headword{bolwod}{\pos{Noun}} {\definition{tall plant, type of pitpit}}
\entry{burara}\headword{burara}{\pos{Noun}} {\definition{water lily}}
\entry{bälwod}\headword{bälwod}{\pos{Noun}} {\definition{Also called pitpit (pidgin word). Tall grass used for pakos (spear). Dry on fire, straighten it, push arrow on, and tie with tulip bark (~2m).}}
\entry{bärkebärke}\headword{bärkebärke}{\pos{Noun}} {\definition{type of algae that can be red or green like a parrot}}
\entry{daga}\headword{daga}{\pos{Noun}} {\definition{betel}}
\entry{dangne}\headword{dangne}{\pos{Noun}} {\definition{crawling vine that grows in the grassland with purple flowers; used to make rope}}
\entry{daradara}\headword{daradara}{\pos{Noun}} {\definition{type of vine with yellow fruit that are red when ripe}}
\entry{ddongddong}\headword{ddongddong}{\pos{Noun}} {\definition{thick cluster of short grass}}
\entry{dit}\headword{dit}{\pos{Noun}} {\definition{type of cane used for building houses, bows, and canoes}}
\entry{esam}\headword{esam}{\pos{Noun}} {\definition{lemongrass}}
\entry{gonz}\headword{gonz}{\pos{Noun}} {\definition{reed}}
\entry{ita}\headword{ita}{\pos{Noun}} {\definition{type of sedge}}
\entry{kabag}\headword{kabag}{\pos{Noun}} {\definition{type of thick grass that grows in swamps}}
\entry{kapalla}\headword{kapalla}{\pos{Noun}} {\definition{floating grass}}
\entry{korolläm}\headword{korolläm}{\pos{Noun}} {\definition{type of vine that grows beside creeks. Leaves are used to wrap things.}}
\entry{kukiny}\headword{kukiny}{\pos{Noun}} {\definition{type of long-leaf grass}}
\entry{kuku}\headword{kuku}{\pos{Noun}} {\definition{type of grass}}
\entry{kullkull}\headword{kullkull}{\pos{Noun}} {\definition{grassfire; burnt grass}}
\entry{kättapun}\headword{kättapun}{\pos{Noun}} {\definition{type of reed}}
\entry{malonäbe}\headword{malonäbe}{\pos{Noun}} {\definition{type of reed that is too soft for weaving}}
\entry{mama}\headword{mama}{\pos{Noun}} {\definition{grass pile}}
\entry{masaka}\headword{masaka}{\pos{Noun}} {\definition{single stem plant with inedible fruit}}
\entry{mormor}\headword{mormor}{\pos{Noun}} {\definition{type of small herb with white flowers and ovate leaves}}
\entry{mällam}\headword{mällam}{\pos{Noun}} {\definition{type of plant}}
\entry{mätka}\headword{mätka}{\pos{Noun}} {\definition{type of tall grass with a single stem and bunches of up to 12 small, edible fruit that grow near the base}}
\entry{ngämengäme}\headword{ngämengäme}{\pos{Noun}} {\definition{type of vine with sweet, edible fruit that are yellow when ripe}}
\entry{ngämral}\headword{ngämral}{\pos{Noun}} {\definition{type of vine that grows in the bush and irritates skin}}
\entry{perälla}\headword{perälla}{\pos{Noun}} {\definition{type of vine}}
\entry{pitpit}\headword{pitpit}{\pos{Noun}} {\definition{sugarcane}}
\entry{piyupiyu}\headword{piyupiyu}{\pos{Noun}} {\definition{large vine that grows from tree to tree in the bush}}
\entry{pllutt}\headword{pllutt}{\pos{Noun}} {\definition{type of vine with fruit that are yellow when ripe; children eat them}}
\entry{pu}\headword{pu}{\pos{Noun}} {\definition{floating grass or island in swamp}}
\entry{puder}\headword{puder}{\pos{Noun}} {\definition{type of long grass}}
\entry{pänbäll}\headword{pänbäll}{\pos{Noun}} {\definition{poisonous vine or root (used in fishing to stun fish)}}
\entry{pɨnyapɨnye}\headword{pɨnyapɨnye}{\pos{Noun}} {\definition{area with burnt grass}}
\entry{rolkutt}\headword{rolkutt}{\pos{Noun}} {\definition{crawling grass}}
\entry{torok}\headword{torok}{\pos{Noun}} {\definition{type of cane used for building houses, bows, and canoes}}
\entry{towall}\headword{towall}{\pos{Noun}} {\definition{grass}}
\entry{ttalme}\headword{ttalme}{\pos{Noun}} {\definition{type of floating grass that grass, deer, and wallaby eat}}
\entry{ttattel}\headword{ttattel}{\pos{Noun}} {\definition{type of thorny vine}}
\entry{ttope}\headword{ttope}{\pos{Noun}} {\definition{reed}}
\entry{ttäle}\headword{ttäle}{\pos{Noun}} {\definition{tendril}}
\entry{tältäl}\headword{tältäl}{\pos{Noun}} {\definition{type of grass}}
\entry{wandana}\headword{wandana}{\pos{Noun}} {\definition{grass in the garden}}
\entry{wätaote}\headword{wätaote}{\pos{Noun}} {\definition{type of large vine that grows in bush; used to tie fence posts together}}
\entry{yaedidib}\headword{yaedidib}{\pos{Noun}} {\definition{type of long, narrow grass (~0.3 m)}}
\entry{yoko}\headword{yoko}{\pos{Noun}} {\definition{type of cane used for building houses, bows, and canoes}}
\entry{yäg}\headword{yäg}{\pos{Noun}} {\definition{bush rope}}
\entry{zagu}\headword{zagu}{\pos{Noun}} {\definition{type of sugarcane-like plant that grows in the swamp}}
\end{entrylist}

\section*{1.5.4 Moss, fungus, algae}
\begin{entrylist}
\entry{bamearoro}\headword{bamearoro}{\pos{Noun}} {\definition{type of mushroom}}
\entry{bob lläkäm}\headword{bob lläkäm}{\pos{Noun}} {\definition{type of mushroom}}
\entry{bogel}\headword{bogel}{\pos{Noun}} {\definition{seaweed}}
\entry{ddällpoyampoyam}\headword{ddällpoyampoyam}{\pos{Noun}} {\definition{type of mushroom}}
\entry{kanken}\headword{kanken}{\pos{Noun}} {\definition{type of mushroom}}
\entry{kapangmändär}\headword{kapangmändär}{\pos{Noun}} {\definition{type of mushroom}}
\entry{kosrom}\headword{kosrom}{\pos{Noun}} {\definition{type of large mushroom that grows in the winter}}
\entry{kwangka lläkäm}\headword{kwangka lläkäm}{\pos{Noun}} {\definition{type of inedible mushroom}}
\entry{lizom}\headword{lizom}{\pos{Noun}} {\definition{type of mushroom}}
\entry{lläkäm}\headword{lläkäm}{\pos{Noun}} {\definition{mushroom}}
\entry{mitoem}\headword{mitoem}{\pos{Noun}} {\definition{type of mushroom}}
\entry{ngamtep}\headword{ngamtep}{\pos{Noun}} {\definition{type of mushroom}}
\entry{pallkeakeya}\headword{pallkeakeya}{\pos{Noun}} {\definition{type of red mushroom that grows in the grassland}}
\entry{pupulläkäm}\headword{pupulläkäm}{\pos{Noun}} {\definition{type of mushroom}}
\entry{sanalläkäm}\headword{sanalläkäm}{\pos{Noun}} {\definition{type of mushroom}}
\entry{tae lläkäm}\headword{tae lläkäm}{\pos{Noun}} {\definition{type of mushroom}}
\entry{tomäll}\headword{tomäll}{\pos{Noun}} {\definition{wart; fungal skin infection}}
\entry{täbe lläkäm}\headword{täbe lläkäm}{\pos{Noun}} {\definition{type of mushroom}}
\end{entrylist}

\section*{1.5.5 Parts of a plant}
\begin{entrylist}
\entry{am}\headword{am}{\pos{Noun}} {\definition{internode (section of bamboo or sugarcane, separated by nodes)}}
\entry{bomo}\headword{bomo}{\pos{Noun}} {\definition{aerial root (e.g. of pandanus)}}
\entry{buata}\headword{buata}{\pos{Noun}} {\definition{betel nut, areca nut (fruit of Areca catechu)}}
\entry{bugu}\headword{bugu}{\pos{Noun}} {\definition{sheath, base, midrib (of a palm leaf)}}
\entry{bänäm}\headword{bänäm}{\pos{Noun}} {\definition{thorns on sago leaves}}
\entry{ddage}\headword{ddage}{\pos{Noun}} {\definition{branch}}
\entry{ddäll}\headword{ddäll}{\pos{Noun}} {\definition{part of the sago trunk closest to the leaves before the shoot}}
\entry{dor}\headword{dor}{\pos{Noun}} {\definition{stalk}}
\entry{däg}\headword{däg}{\pos{Noun}} {\definition{hand, group, bunch, set}}
\entry{dɨp}\headword{dɨp}{\pos{Noun}} {\definition{shoot (of a plant)}}
\entry{gollob}\headword{gollob}{\pos{Noun}} {\definition{outer layer, hull, shell (e.g. of a turtle, egg)}}
\entry{kakoll kutt}\headword{kakoll kutt}{\pos{Noun}} {\definition{endocarp of coconut}}
\entry{kip}\headword{kip}{\pos{Noun}} {\definition{top of a plant}}
\entry{koko}\headword{koko}{\pos{Noun}} {\definition{shoot (of a plant)}}
\entry{kol}\headword{kol}{\pos{Noun}} {\definition{sago pith}}
\entry{kutt}\headword{kutt}{\pos{Noun}} {\definition{seed, core}}
\entry{käkäm}\headword{käkäm}{\pos{Noun}} {\definition{young leaf}}
\entry{käp}\headword{käp}{\pos{Noun}} {\definition{fruit}}
\entry{llo ttoe}\headword{llo ttoe}{\pos{Noun}} {\definition{tree bark}}
\entry{llo tubu}\headword{llo tubu}{\pos{Noun}} {\definition{tree stump}}
\entry{llupi}\headword{llupi}{\pos{Noun}} {\definition{branch}}
\entry{maiya}\headword{maiya}{\pos{Noun}} {\definition{bulbil (fruit of yam that is planted)}}
\entry{mit}\headword{mit}{\pos{Noun}} {\definition{base (of a plant)}}
\entry{ollondd}\headword{ollondd}{\pos{Noun}} {\definition{root}}
\entry{patt}\headword{patt}{\pos{Noun}} {\definition{tree trunk (fallen)}}
\entry{popo}\headword{popo}{\pos{Noun}} {\definition{flower}}
\entry{pot}\headword{pot}{\pos{Noun}} {\definition{tip, end, point; base (of yam)}}
\entry{pudd}\headword{pudd}{\pos{Noun}} {\definition{taro shoot}}
\entry{pätt}\headword{pätt}{\pos{Noun}} {\definition{trunk of a plant in the ground; a single plant}}
\entry{pättkäp}\headword{pättkäp}{\pos{Noun}} {\definition{node (ring or line on bamboo or sugarcane that separates internodes)}}
\entry{sanateya}\headword{sanateya}{\pos{Noun}} {\definition{sago pith}}
\entry{tep}\headword{tep}{\pos{Noun}} {\definition{tree sap (used in making drums and arrows)}}
\entry{ttam}\headword{ttam}{\pos{Noun}} {\definition{leaf}}
\entry{ttattep}\headword{ttattep}{\pos{Noun}} {\definition{mature leaf}}
\entry{ttäle}\headword{ttäle}{\pos{Noun}} {\definition{tendril}}
\entry{ttättäp}\headword{ttättäp}{\pos{Noun}} {\definition{young leaf}}
\entry{täkäll}\headword{täkäll}{\pos{Noun}} {\definition{thorn}}
\entry{täma}\headword{täma}{\pos{Noun}} {\definition{husk, exocarp/mesocarp (of coconut)}}
\entry{tän}\headword{tän}{\pos{Noun}} {\definition{stem}}
\entry{täp}\headword{täp}{\pos{Noun}} {\definition{sago shoot}}
\entry{tärpa}\headword{tärpa}{\pos{Noun}} {\definition{yam skin}}
\entry{utt}\headword{utt}{\pos{Noun}} {\definition{shoot (of a plant)}}
\entry{wap}\headword{wap}{\pos{Noun}} {\definition{stick}}
\entry{wäl}\headword{wäl}{\pos{Noun}} {\definition{main surface-level stem of a plant with rhizomes (e.g. sweet potato, lemongrass)}}
\entry{wänkäm molle}\headword{wänkäm molle}{\pos{Noun}} {\definition{soft part of a shoot or sucker (e.g. of taro, banana, or sago)}}
\end{entrylist}

\section*{1.5.6 Growth of plants}
\begin{entrylist}
\entry{blab}\headword{blab}{\pos{Intransitive S verb}} {\definition{to mature, reach puberty}}
\entry{kang}\headword{kang}{\pos{Noun}} {\definition{sucker (additional unwanted shoot that grows by the base of a tree)}}
\entry{kire}\headword{kire}{\pos{Modifier}} {\definition{unripe; raw, fresh}}
\entry{ngällngäll}\headword{ngällngäll}{\pos{Transitive S verb}} {\definition{to produce, yield, bear (fruit)}}
\entry{o}\headword{o}{\pos{Modifier}} {\definition{ripe}}
\end{entrylist}

\section*{1.6 Animal}
\begin{entrylist}
\entry{ddäddäg}\headword{ddäddäg}{\pos{Noun}} {\definition{edible animal, game, meat}}
\entry{gulin}\headword{gulin}{\pos{Noun}} {\definition{crab}}
\entry{llune}\headword{llune}{\pos{Modifier}} {\definition{wild}}
\end{entrylist}

\section*{1.6.1 Types of animals}
\begin{entrylist}
\entry{bitän}\headword{bitän}{\pos{Noun}} {\definition{type of animal}}
\entry{däräng olleolle}\headword{däräng olleolle}{\pos{Noun}} {\definition{type of possum-like animal with spotted skin}}
\entry{girag}\headword{girag}{\pos{Noun}} {\definition{long-nosed echymipera}}
\entry{kubull}\headword{kubull}{\pos{Noun}} {\definition{bush wallaby, dusky pademelon}}
\entry{källäng}\headword{källäng}{\pos{Noun}} {\definition{type of animal}}
\entry{mäd kubull}\headword{mäd kubull}{\pos{Noun}} {\definition{black bush wallaby}}
\entry{ngem}\headword{ngem}{\pos{Noun}} {\definition{virginia oppossum}}
\entry{pinzopinzo}\headword{pinzopinzo}{\pos{Noun}} {\definition{type of small insect that lives in the ground}}
\entry{tiklem}\headword{tiklem}{\pos{Noun}} {\definition{type of animal}}
\end{entrylist}

\section*{1.6.1.1 Mammal}
\begin{entrylist}
\entry{kemol}\headword{kemol}{\pos{Noun}} {\definition{camel}}
\end{entrylist}

\section*{1.6.1.1.3 Hoofed animals}
\begin{entrylist}
\entry{ddia}\headword{ddia}{\pos{Noun}} {\definition{deer}}
\entry{dongki}\headword{dongki}{\pos{Noun}} {\definition{donkey}}
\entry{sämell}\headword{sämell}{\pos{Noun}} {\definition{pig}}
\end{entrylist}

\section*{1.6.1.1.4 Rodent}
\begin{entrylist}
\entry{bittott}\headword{bittott}{\pos{Noun}} {\definition{grey squirrel}}
\entry{katt}\headword{katt}{\pos{Noun}} {\definition{type of medium-sized rodent that lives in the bush}}
\entry{kiklem}\headword{kiklem}{\pos{Noun}} {\definition{type of small edible rodent}}
\entry{kitapatt}\headword{kitapatt}{\pos{Noun}} {\definition{type of bandicoot-like animal}}
\entry{mäkat}\headword{mäkat}{\pos{Noun}} {\definition{rat}}
\entry{zogam}\headword{zogam}{\pos{Noun}} {\definition{rat}}
\end{entrylist}

\section*{1.6.1.1.5 Marsupial}
\begin{entrylist}
\entry{baet}\headword{baet}{\pos{Noun}} {\definition{cuscus}}
\entry{ddäma}\headword{ddäma}{\pos{Noun}} {\definition{pouch of a marsupial}}
\entry{dinggel}\headword{dinggel}{\pos{Noun}} {\definition{sugar glider}}
\entry{kämlla}\headword{kämlla}{\pos{Noun}} {\definition{short-beaked echidna}}
\entry{maigag}\headword{maigag}{\pos{Noun}} {\definition{northern brown bandicoot}}
\entry{ngatengate}\headword{ngatengate}{\pos{Noun}} {\definition{gliding possum, sugar glider}}
\entry{pall kubull}\headword{pall kubull}{\pos{Noun}} {\definition{red-legged pademelon}}
\entry{puku kubull}\headword{puku kubull}{\pos{Noun}} {\definition{type of bush wallaby with white tail}}
\entry{ttall}\headword{ttall}{\pos{Noun}} {\definition{agile wallaby, sandy wallaby}}
\entry{ttän maigag}\headword{ttän maigag}{\pos{Noun}} {\definition{type of bandicoot}}
\end{entrylist}

\section*{1.6.1.1.8 Bat}
\begin{entrylist}
\entry{titi}\headword{titi}{\pos{Noun}} {\definition{bat}}
\entry{topoll}\headword{topoll}{\pos{Noun}} {\definition{flying fox}}
\end{entrylist}

\section*{1.6.1.2 Bird}
\begin{entrylist}
\entry{aeb}\headword{aeb}{\pos{Noun}} {\definition{black-billed/yellow-legged brushturkey}}
\entry{arabuni}\headword{arabuni}{\pos{Noun}} {\definition{red backed buttonquail}}
\entry{asa bume}\headword{asa bume}{\pos{Noun}} {\definition{type of bird}}
\entry{awe}\headword{awe}{\pos{Noun}} {\definition{cassowary (used when hunting)}}
\entry{bette}\headword{bette}{\pos{Noun}} {\definition{crimson finch}}
\entry{bibol}\headword{bibol}{\pos{Noun}} {\definition{type of bird}}
\entry{biboz}\headword{biboz}{\pos{Noun}} {\definition{fairywren (emperor, white-shouldered)}}
\entry{boga}\headword{boga}{\pos{Noun}} {\definition{type of bird}}
\entry{bonydre}\headword{bonydre}{\pos{Noun}} {\definition{goshawk (grey-headed, brown); collared sparrowhawk}}
\entry{buindre}\headword{buindre}{\pos{Noun}} {\definition{collared sparrowhawk. This bird kills chickens.}}
\entry{bullull}\headword{bullull}{\pos{Noun}} {\definition{Papuan frogmouth}}
\entry{bur}\headword{bur}{\pos{Noun}} {\definition{type of bird}}
\entry{bärke}\headword{bärke}{\pos{Noun}} {\definition{Papuan eclectus}}
\entry{bäräbäräl}\headword{bäräbäräl}{\pos{Noun}} {\definition{type of bird}}
\entry{del}\headword{del}{\pos{Noun}} {\definition{coconut lorikeet}}
\entry{dibie}\headword{dibie}{\pos{Noun}} {\definition{spectacled longbill}}
\entry{diboz}\headword{diboz}{\pos{Noun}} {\definition{type of bird}}
\entry{dirom}\headword{dirom}{\pos{Noun}} {\definition{southern cassowary}}
\entry{dorllog}\headword{dorllog}{\pos{Noun}} {\definition{rufous-bellied kookaburra}}
\entry{du kyakya}\headword{du kyakya}{\pos{Noun}} {\definition{hook-billed kingfisher}}
\entry{dugo}\headword{dugo}{\pos{Noun}} {\definition{type of bird}}
\entry{däbi}\headword{däbi}{\pos{Noun}} {\definition{green-backed honeyeater}}
\entry{dängam}\headword{dängam}{\pos{Noun}} {\definition{Blyth's hornbil}}
\entry{därollog}\headword{därollog}{\pos{Noun}} {\definition{brolga}}
\entry{gaopi}\headword{gaopi}{\pos{Noun}} {\definition{Australian pelican}}
\entry{geagell}\headword{geagell}{\pos{Noun}} {\definition{Lewin's rail}}
\entry{giegier}\headword{giegier}{\pos{Noun}} {\definition{white-browed crake}}
\entry{gilib}\headword{gilib}{\pos{Noun}} {\definition{type of bird}}
\entry{giwi}\headword{giwi}{\pos{Noun}} {\definition{fruit dove (coroneted, orange-bellied, pink-spotted, orange-fronted)}}
\entry{gllogllo}\headword{gllogllo}{\pos{Noun}} {\definition{marbled frogmouth}}
\entry{gonggo}\headword{gonggo}{\pos{Noun}} {\definition{piping bellbird/crested pitohui}}
\entry{grawa}\headword{grawa}{\pos{Noun}} {\definition{Australasian darter}}
\entry{guboll}\headword{guboll}{\pos{Noun}} {\definition{New Guinean magpie}}
\entry{gäboll}\headword{gäboll}{\pos{Noun}} {\definition{magpie-lark}}
\entry{inpiak}\headword{inpiak}{\pos{Noun}} {\definition{whistling kite}}
\entry{inuinu}\headword{inuinu}{\pos{Noun}} {\definition{white-faced robin}}
\entry{ioläm}\headword{ioläm}{\pos{Noun}} {\definition{type of bird}}
\entry{iya}\headword{iya}{\pos{Noun}} {\definition{Australian masked owl}}
\entry{iyeiyem}\headword{iyeiyem}{\pos{Noun}} {\definition{common emerald dove}}
\entry{kakayam}\headword{kakayam}{\pos{Noun}} {\definition{greater bird-of-paradise}}
\entry{kallamatt}\headword{kallamatt}{\pos{Noun}} {\definition{Oriental dollarbird}}
\entry{kallmo}\headword{kallmo}{\pos{Noun}} {\definition{butcherbird (black-backed, hooded)}}
\entry{kaonggall}\headword{kaonggall}{\pos{Noun}} {\definition{yellow-faced myna}}
\entry{katt}\headword{katt}{\pos{Noun}} {\definition{Meyer's friarbird}}
\entry{kek}\headword{kek}{\pos{Noun}} {\definition{orange-footed scrubfowl}}
\entry{keyadaola}\headword{keyadaola}{\pos{Noun}} {\definition{type of bird}}
\entry{kikiem}\headword{kikiem}{\pos{Noun}} {\definition{type of bird}}
\entry{kiyaddadda}\headword{kiyaddadda}{\pos{Noun}} {\definition{paradise kingfisher (common, buff-breasted, little)}}
\entry{kokopasi}\headword{kokopasi}{\pos{Noun}} {\definition{shrikethrush (Arafura, rufous)}}
\entry{kopllalle}\headword{kopllalle}{\pos{Noun}} {\definition{oriole (brown, olive-backed, green)}}
\entry{kormas}\headword{kormas}{\pos{Noun}} {\definition{type of bird}}
\entry{kudädäri}\headword{kudädäri}{\pos{Noun}} {\definition{Zoe's imperial pigeon}}
\entry{kurkur}\headword{kurkur}{\pos{Noun}} {\definition{type of bird}}
\entry{kus}\headword{kus}{\pos{Noun}} {\definition{type of bird}}
\entry{kwangka}\headword{kwangka}{\pos{Noun}} {\definition{Torresian crow}}
\entry{kwarakwara}\headword{kwarakwara}{\pos{Noun}} {\definition{eastern hooded pitta}}
\entry{kär pipiem}\headword{kär pipiem}{\pos{Noun}} {\definition{purple-tailed imperial pigeon}}
\entry{kättekätte}\headword{kättekätte}{\pos{Noun}} {\definition{red-cheeked parrot}}
\entry{käza wirwir}\headword{käza wirwir}{\pos{Noun}} {\definition{frilled monarch}}
\entry{mamos}\headword{mamos}{\pos{Noun}} {\definition{comb-crested jacana}}
\entry{mend}\headword{mend}{\pos{Noun}} {\definition{type of bird}}
\entry{miroli}\headword{miroli}{\pos{Noun}} {\definition{black-capped lory}}
\entry{mise}\headword{mise}{\pos{Noun}} {\definition{common cicadabird}}
\entry{miwiwi}\headword{miwiwi}{\pos{Noun}} {\definition{dabbling duck (pacific black duck, grey teal)}}
\entry{mok}\headword{mok}{\pos{Noun}} {\definition{friarbird (noisy, little, helmeted)}}
\entry{mätta pirpir}\headword{mätta pirpir}{\pos{Noun}} {\definition{rainbow bee-eater}}
\entry{ngallngall}\headword{ngallngall}{\pos{Noun}} {\definition{catbird (spotted, ochre-breasted)}}
\entry{ngokngok}\headword{ngokngok}{\pos{Noun}} {\definition{boobook (Australian, barking [barking owl])}}
\entry{nurae}\headword{nurae}{\pos{Noun}} {\definition{type of large bird}}
\entry{nyinggulgul}\headword{nyinggulgul}{\pos{Noun}} {\definition{type of bird}}
\entry{nyonga}\headword{nyonga}{\pos{Noun}} {\definition{Triton cockatoo}}
\entry{obosasa}\headword{obosasa}{\pos{Noun}} {\definition{Australian rufous fantail}}
\entry{odoolo}\headword{odoolo}{\pos{Noun}} {\definition{type of bird}}
\entry{ogog}\headword{ogog}{\pos{Noun}} {\definition{grey-crowned babbler}}
\entry{otal}\headword{otal}{\pos{Intransitive S verb}} {\definition{to perch}}
\entry{pa}\headword{pa}{\pos{Noun}} {\definition{bird}}
\entry{patiti}\headword{patiti}{\pos{Noun}} {\definition{type of bird}}
\entry{pauro}\headword{pauro}{\pos{Noun}} {\definition{chicken}}
\entry{pewälewäle}\headword{pewälewäle}{\pos{Noun}} {\definition{green pygmy goose}}
\entry{pidor}\headword{pidor}{\pos{Noun}} {\definition{white-bellied sea-eagle}}
\entry{pintta}\headword{pintta}{\pos{Noun}} {\definition{palm cockatoo}}
\entry{piny}\headword{piny}{\pos{Noun}} {\definition{kingfisher (azure, little)}}
\entry{pinya dorollog}\headword{pinya dorollog}{\pos{Noun}} {\definition{sacred kingfisher}}
\entry{pitratra}\headword{pitratra}{\pos{Noun}} {\definition{masked lapwing}}
\entry{poses}\headword{poses}{\pos{Noun}} {\definition{martin (fairy, tree)}}
\entry{pugupugu}\headword{pugupugu}{\pos{Noun}} {\definition{collared imperial pigeon}}
\entry{puzi}\headword{puzi}{\pos{Noun}} {\definition{type of bird with crown}}
\entry{päräl}\headword{päräl}{\pos{Noun}} {\definition{radjah shelduck}}
\entry{sanga}\headword{sanga}{\pos{Noun}} {\definition{black-necked stork}}
\entry{sawiya}\headword{sawiya}{\pos{Noun}} {\definition{little egret}}
\entry{siklakla}\headword{siklakla}{\pos{Noun}} {\definition{golden-headed cisticola}}
\entry{slaslak}\headword{slaslak}{\pos{Noun}} {\definition{red-winged parrot}}
\entry{taewa}\headword{taewa}{\pos{Noun}} {\definition{type of bird}}
\entry{tarambobo}\headword{tarambobo}{\pos{Noun}} {\definition{hooded butcherbird}}
\entry{tarasoso}\headword{tarasoso}{\pos{Noun}} {\definition{type of bird}}
\entry{tarme}\headword{tarme}{\pos{Noun}} {\definition{kookaburra (blue-winged; spangled)}}
\entry{tatanggli}\headword{tatanggli}{\pos{Noun}} {\definition{willie wagtail}}
\entry{tawa aeb}\headword{tawa aeb}{\pos{Noun}} {\definition{Australasian swamphen}}
\entry{teweya}\headword{teweya}{\pos{Noun}} {\definition{brush cuckoo}}
\entry{tikuku}\headword{tikuku}{\pos{Noun}} {\definition{black-backed bittern}}
\entry{titi}\headword{titi}{\pos{Noun}} {\definition{brown honeyeater}}
\entry{torep}\headword{torep}{\pos{Noun}} {\definition{brown quail}}
\entry{toronggogo}\headword{toronggogo}{\pos{Noun}} {\definition{bar-shouldered dove}}
\entry{toto nyäknyäk}\headword{toto nyäknyäk}{\pos{Noun}} {\definition{Australian owlet-nightjar}}
\entry{ttall källa}\headword{ttall källa}{\pos{Noun}} {\definition{brahminy kite}}
\entry{ttangttang}\headword{ttangttang}{\pos{Noun}} {\definition{type of bird}}
\entry{ttoa}\headword{ttoa}{\pos{Noun}} {\definition{bird type}}
\entry{ttoettoe}\headword{ttoettoe}{\pos{Noun}} {\definition{blue-faced honeyeater}}
\entry{ttowa}\headword{ttowa}{\pos{Noun}} {\definition{Pacific koel}}
\entry{ttullong}\headword{ttullong}{\pos{Noun}} {\definition{large-tailed nightjar}}
\entry{ttäb}\headword{ttäb}{\pos{Noun}} {\definition{Pinon's imperial pigeon}}
\entry{ttätt käp}\headword{ttätt käp}{\pos{Noun}} {\definition{graceful honeyeater}}
\entry{turwe}\headword{turwe}{\pos{Noun}} {\definition{shining bronze cuckoo}}
\entry{täbie}\headword{täbie}{\pos{Noun}} {\definition{black sunbird}}
\entry{täny}\headword{täny}{\pos{Noun}} {\definition{lesser fig-parrot}}
\entry{uwo}\headword{uwo}{\pos{Noun}} {\definition{magnificent riflebird}}
\entry{wak}\headword{wak}{\pos{Noun}} {\definition{Papuan pitta}}
\entry{waso}\headword{waso}{\pos{Noun}} {\definition{eastern cattle egret}}
\entry{welwel}\headword{welwel}{\pos{Noun}} {\definition{type of bird}}
\entry{wiyasara}\headword{wiyasara}{\pos{Noun}} {\definition{silver gull}}
\entry{woddowoddo}\headword{woddowoddo}{\pos{Noun}} {\definition{rusty pitohui}}
\entry{wäkär}\headword{wäkär}{\pos{Noun}} {\definition{type of bird}}
\entry{wäkɨs}\headword{wäkɨs}{\pos{Noun}} {\definition{type of bird}}
\entry{wälläng ttäp}\headword{wälläng ttäp}{\pos{Noun}} {\definition{Papuan eagle}}
\entry{wällängakäbu}\headword{wällängakäbu}{\pos{Noun}} {\definition{wompoo fruit dove}}
\entry{yal}\headword{yal}{\pos{Noun}} {\definition{yellow-billed kingfisher}}
\entry{zo}\headword{zo}{\pos{Noun}} {\definition{fawn-breasted bowerbird}}
\entry{zäbo}\headword{zäbo}{\pos{Noun}} {\definition{yellow-streaked lory}}
\end{entrylist}

\section*{1.6.1.3 Reptile}
\begin{entrylist}
\entry{mos}\headword{mos}{\pos{Noun}} {\definition{type of goanna}}
\entry{turllo}\headword{turllo}{\pos{Noun}} {\definition{type of lizard}}
\entry{täme}\headword{täme}{\pos{Noun}} {\definition{monitor lizard, goanna}}
\end{entrylist}

\section*{1.6.1.3.1 Snake}
\begin{entrylist}
\entry{bätte}\headword{bätte}{\pos{Noun}} {\definition{type of snake}}
\entry{ddämoemkäp kuibiag}\headword{ddämoemkäp kuibiag}{\pos{Noun}} {\definition{type of python}}
\entry{däen}\headword{däen}{\pos{Noun}} {\definition{type of snake}}
\entry{därba}\headword{därba}{\pos{Noun}} {\definition{type of snake}}
\entry{gazibra}\headword{gazibra}{\pos{Noun}} {\definition{type of edible water snake with skin like sandpaper}}
\entry{gullem}\headword{gullem}{\pos{Noun}} {\definition{snake}}
\entry{ine mallmell}\headword{ine mallmell}{\pos{Noun}} {\definition{type of snake}}
\entry{kuibiag}\headword{kuibiag}{\pos{Noun}} {\definition{Papuan black snake}}
\entry{kwakasru}\headword{kwakasru}{\pos{Noun}} {\definition{type of snake}}
\entry{kɨllɨll}\headword{kɨllɨll}{\pos{Noun}} {\definition{type of snake}}
\entry{lla diben}\headword{lla diben}{\pos{Noun}} {\definition{type of snake}}
\entry{llälläp}\headword{llälläp}{\pos{Noun}} {\definition{type of small snake that catches frogs}}
\entry{mimidämäll}\headword{mimidämäll}{\pos{Noun}} {\definition{type of snake}}
\entry{moza}\headword{moza}{\pos{Noun}} {\definition{type of venomous snake}}
\entry{mällkakallamatt}\headword{mällkakallamatt}{\pos{Noun}} {\definition{type of venomous snake}}
\entry{mällät käp}\headword{mällät käp}{\pos{Noun}} {\definition{D'Albertis python}}
\entry{mätkakallamatt}\headword{mätkakallamatt}{\pos{Noun}} {\definition{type of snake}}
\entry{nge tärmir}\headword{nge tärmir}{\pos{Noun}} {\definition{type of snake}}
\entry{paklle}\headword{paklle}{\pos{Noun}} {\definition{type of snake}}
\entry{pongo}\headword{pongo}{\pos{Noun}} {\definition{scent}}
\entry{tigi}\headword{tigi}{\pos{Noun}} {\definition{type of snake}}
\entry{towallpipi}\headword{towallpipi}{\pos{Noun}} {\definition{type of venomous snake}}
\entry{ttoep}\headword{ttoep}{\pos{Noun}} {\definition{type of snake}}
\entry{ttälläp}\headword{ttälläp}{\pos{Noun}} {\definition{type of venomous snake}}
\entry{wadär gullem}\headword{wadär gullem}{\pos{Noun}} {\definition{Papuan python}}
\entry{waya gullem}\headword{waya gullem}{\pos{Noun}} {\definition{type of venomous snake}}
\end{entrylist}

\section*{1.6.1.3.2 Lizard}
\begin{entrylist}
\entry{boko}\headword{boko}{\pos{Noun}} {\definition{type of lizard}}
\entry{dongkäral}\headword{dongkäral}{\pos{Noun}} {\definition{type of lizard}}
\entry{kämgag}\headword{kämgag}{\pos{Noun}} {\definition{type of lizard}}
\entry{llo patipati}\headword{llo patipati}{\pos{Noun}} {\definition{tree goanna}}
\entry{mambag}\headword{mambag}{\pos{Noun}} {\definition{type of goanna}}
\entry{nongg}\headword{nongg}{\pos{Noun}} {\definition{type of gecko}}
\entry{onggall}\headword{onggall}{\pos{Noun}} {\definition{frilled lizard}}
\entry{pipllo}\headword{pipllo}{\pos{Noun}} {\definition{lizard}}
\entry{puikmetutu}\headword{puikmetutu}{\pos{Noun}} {\definition{type of gecko}}
\end{entrylist}

\section*{1.6.1.3.3 Turtle}
\begin{entrylist}
\entry{atata kottllam}\headword{atata kottllam}{\pos{Noun}} {\definition{type of turtle}}
\entry{gamo}\headword{gamo}{\pos{Noun}} {\definition{type of large sea turtle}}
\entry{gogo kottllam}\headword{gogo kottllam}{\pos{Noun}} {\definition{type of turtle with yellow scales on neck}}
\entry{gollob}\headword{gollob}{\pos{Noun}} {\definition{outer layer, hull, shell (e.g. of a turtle, egg)}}
\entry{kottllam}\headword{kottllam}{\pos{Noun}} {\definition{turtle, tortoise}}
\entry{pall kottllam}\headword{pall kottllam}{\pos{Noun}} {\definition{red-bellied short-necked turtle}}
\entry{paro kottllam}\headword{paro kottllam}{\pos{Noun}} {\definition{type of turtle with red scales on neck}}
\entry{ulle kottllam}\headword{ulle kottllam}{\pos{Noun}} {\definition{type of big turtle}}
\entry{uwo kottllam}\headword{uwo kottllam}{\pos{Noun}} {\definition{type of turtle}}
\entry{waro}\headword{waro}{\pos{Noun}} {\definition{type of turtle}}
\end{entrylist}

\section*{1.6.1.3.4 Crocodile}
\begin{entrylist}
\entry{käza}\headword{käza}{\pos{Noun}} {\definition{Hall's New Guinea crocodile}}
\end{entrylist}

\section*{1.6.1.4 Amphibian}
\begin{entrylist}
\entry{nge pollgo}\headword{nge pollgo}{\pos{Noun}} {\definition{coconut frog, green}}
\entry{pollgo}\headword{pollgo}{\pos{Noun}} {\definition{frog}}
\end{entrylist}

\section*{1.6.1.5 Fish}
\begin{entrylist}
\entry{bargae}\headword{bargae}{\pos{Noun}} {\definition{type of fish}}
\entry{boge}\headword{boge}{\pos{Noun}} {\definition{mudfish}}
\entry{bunkuttang}\headword{bunkuttang}{\pos{Noun}} {\definition{catfish}}
\entry{dompak}\headword{dompak}{\pos{Noun}} {\definition{eel}}
\entry{inbunatt}\headword{inbunatt}{\pos{Noun}} {\definition{mallet fish (big, lives in creek, lean)}}
\entry{kalläg}\headword{kalläg}{\pos{Noun}} {\definition{type of edible, fatty fish with big scales; found in the swamp, similar to barramundi}}
\entry{kapkap}\headword{kapkap}{\pos{Noun}} {\definition{mudskipper}}
\entry{kollba}\headword{kollba}{\pos{Noun}} {\definition{fish}}
\entry{mozaya}\headword{mozaya}{\pos{Noun}} {\definition{type of large, edible fish found in the swamp}}
\entry{mänyän}\headword{mänyän}{\pos{Noun}} {\definition{type of fish}}
\entry{od}\headword{od}{\pos{Noun}} {\definition{type of fatty freshwater fish}}
\entry{ttape}\headword{ttape}{\pos{Noun}} {\definition{type of small, flat fish}}
\entry{tärko}\headword{tärko}{\pos{Noun}} {\definition{type of small fish}}
\entry{wod}\headword{wod}{\pos{Noun}} {\definition{type of fatty fish}}
\entry{zire}\headword{zire}{\pos{Noun}} {\definition{barramundi}}
\end{entrylist}

\section*{1.6.1.7 Insect}
\begin{entrylist}
\entry{allko}\headword{allko}{\pos{Noun}} {\definition{fly (insect)}}
\entry{allko kukut}\headword{allko kukut}{\pos{Noun}} {\definition{blue fly}}
\entry{ankom}\headword{ankom}{\pos{Noun}} {\definition{ant}}
\entry{apapi}\headword{apapi}{\pos{Noun}} {\definition{butterfly}}
\entry{ballma}\headword{ballma}{\pos{Noun}} {\definition{type of biting bee found in trees}}
\entry{bisbis}\headword{bisbis}{\pos{Noun}} {\definition{type of stingless bee}}
\entry{bodobodom}\headword{bodobodom}{\pos{Noun}} {\definition{type of biting ant}}
\entry{budar}\headword{budar}{\pos{Noun}} {\definition{grub, larva}}
\entry{budombudom}\headword{budombudom}{\pos{Noun}} {\definition{red ants}}
\entry{bumrel}\headword{bumrel}{\pos{Noun}} {\definition{beetle}}
\entry{bällkäp}\headword{bällkäp}{\pos{Noun}} {\definition{ant pupae}}
\entry{bänz}\headword{bänz}{\pos{Noun}} {\definition{mosquito}}
\entry{bänz}\headword{bänz}{\pos{Noun}} {\definition{type of biting bee found in trees}}
\entry{bänäm}\headword{bänäm}{\pos{Noun}} {\definition{type of very small insect that lives on bandicoots}}
\entry{ddällgoe}\headword{ddällgoe}{\pos{Transitive S verb}} {\definition{to disturb a beehive}}
\entry{ddängall}\headword{ddängall}{\pos{Noun}} {\definition{type of stinging bee found in trees}}
\entry{deodeo}\headword{deodeo}{\pos{Noun}} {\definition{termite}}
\entry{duny}\headword{duny}{\pos{Noun}} {\definition{beetle}}
\entry{gongg}\headword{gongg}{\pos{Intransitive S verb}} {\definition{to disturb a bees nest and then to feel the bites}}
\entry{gonzagonzar budar}\headword{gonzagonzar budar}{\pos{Noun}} {\definition{type of grub}}
\entry{kaemne}\headword{kaemne}{\pos{Noun}} {\definition{bee}}
\entry{kok}\headword{kok}{\pos{Noun}} {\definition{grasshopper}}
\entry{konykony}\headword{konykony}{\pos{Noun}} {\definition{type of stinging insect that lives in the ground}}
\entry{kud dämadämar}\headword{kud dämadämar}{\pos{Noun}} {\definition{dragonfly}}
\entry{käban}\headword{käban}{\pos{Noun}} {\definition{louse}}
\entry{käza allko}\headword{käza allko}{\pos{Noun}} {\definition{type of fly that antagonizes other flies}}
\entry{lla winy}\headword{lla winy}{\pos{Noun}} {\definition{type of biting bee that lives in trees}}
\entry{miny}\headword{miny}{\pos{Noun}} {\definition{ant egg/larvae (small)}}
\entry{ngoetrangoetra}\headword{ngoetrangoetra}{\pos{Noun}} {\definition{type of ant that doesn't bite}}
\entry{nätnät}\headword{nätnät}{\pos{Noun}} {\definition{insects}}
\entry{omgälgäl}\headword{omgälgäl}{\pos{Noun}} {\definition{red ants}}
\entry{pid}\headword{pid}{\pos{Noun}} {\definition{horsefly}}
\entry{pidroll}\headword{pidroll}{\pos{Noun}} {\definition{black palm weevil}}
\entry{pip}\headword{pip}{\pos{Noun}} {\definition{red bee}}
\entry{pit}\headword{pit}{\pos{Noun}} {\definition{small insect that lives in the swamp and eats taro and sweet potato}}
\entry{pite}\headword{pite}{\pos{Noun}} {\definition{large winged ankom}}
\entry{rol}\headword{rol}{\pos{Noun}} {\definition{type of hairy caterpillar}}
\entry{sis}\headword{sis}{\pos{Noun}} {\definition{type of flying ant that come out from their flooded anthills in November}}
\entry{sispull}\headword{sispull}{\pos{Noun}} {\definition{maggot}}
\entry{tamllägtamlläg}\headword{tamllägtamlläg}{\pos{Noun}} {\definition{type of small caterpillar}}
\entry{tatruk}\headword{tatruk}{\pos{Noun}} {\definition{type of poisonous ant}}
\entry{tintromol}\headword{tintromol}{\pos{Noun}} {\definition{type of black or red ants}}
\entry{ttägäll}\headword{ttägäll}{\pos{Noun}} {\definition{termite mound, anthill (made by termites; ants may also live inside)}}
\entry{täbäll pud}\headword{täbäll pud}{\pos{Noun}} {\definition{type of biting bee found in trees}}
\entry{tät}\headword{tät}{\pos{Noun}} {\definition{type of insect}}
\entry{tätämall}\headword{tätämall}{\pos{Noun}} {\definition{type of small insect that glow green}}
\entry{ubony}\headword{ubony}{\pos{Noun}} {\definition{type of black bee}}
\entry{yoto}\headword{yoto}{\pos{Noun}} {\definition{type of biting bee found in trees}}
\entry{ziz}\headword{ziz}{\pos{Noun}} {\definition{insect}}
\end{entrylist}

\section*{1.6.1.8 Spider}
\begin{entrylist}
\entry{kungge}\headword{kungge}{\pos{Noun}} {\definition{spider}}
\end{entrylist}

\section*{1.6.1.9 Small animals}
\begin{entrylist}
\entry{big budar}\headword{big budar}{\pos{Noun}} {\definition{type of grub}}
\entry{dangnebudar}\headword{dangnebudar}{\pos{Noun}} {\definition{type of grub}}
\entry{dinggoll}\headword{dinggoll}{\pos{Noun}} {\definition{opossum}}
\entry{gara}\headword{gara}{\pos{Noun}} {\definition{aquatic leech}}
\entry{guzi}\headword{guzi}{\pos{Noun}} {\definition{yabby (type of crayfish)}}
\entry{kapang budar}\headword{kapang budar}{\pos{Noun}} {\definition{type of grub}}
\entry{kidwe}\headword{kidwe}{\pos{Noun}} {\definition{millipede}}
\entry{kokäl}\headword{kokäl}{\pos{Noun}} {\definition{small mudcrab}}
\entry{komo}\headword{komo}{\pos{Noun}} {\definition{centipede}}
\entry{komo}\headword{komo}{\pos{Noun}} {\definition{scorpion}}
\entry{kupull}\headword{kupull}{\pos{Noun}} {\definition{earthworm, worm}}
\entry{kämsir budar}\headword{kämsir budar}{\pos{Noun}} {\definition{type of grub}}
\entry{känär}\headword{känär}{\pos{Noun}} {\definition{type of edible grub found in the bush}}
\entry{kätt}\headword{kätt}{\pos{Noun}} {\definition{bivalve; shell of a mollusc}}
\entry{mande gara}\headword{mande gara}{\pos{Noun}} {\definition{water leech}}
\entry{molemoleg}\headword{molemoleg}{\pos{Noun}} {\definition{type of grub}}
\entry{mällätbudar}\headword{mällätbudar}{\pos{Noun}} {\definition{type of grub}}
\entry{pittbudar}\headword{pittbudar}{\pos{Noun}} {\definition{type of edible grub found in the bush and grassland}}
\entry{piyupiyu budar}\headword{piyupiyu budar}{\pos{Noun}} {\definition{type of grub}}
\entry{poddem}\headword{poddem}{\pos{Noun}} {\definition{young or small mammal (e.g. deer, wallaby)}}
\entry{potne kätt}\headword{potne kätt}{\pos{Noun}} {\definition{bivalve type}}
\entry{sambuag}\headword{sambuag}{\pos{Noun}} {\definition{lobster; prawn}}
\entry{sana budar}\headword{sana budar}{\pos{Noun}} {\definition{sago grub (larva of black palm weevil)}}
\entry{tikle}\headword{tikle}{\pos{Noun}} {\definition{type of small louse}}
\entry{tongle}\headword{tongle}{\pos{Noun}} {\definition{leech}}
\entry{täbe budar}\headword{täbe budar}{\pos{Noun}} {\definition{type of grub}}
\end{entrylist}

\section*{1.6.2 Parts of an animal}
\begin{entrylist}
\entry{ali}\headword{ali}{\pos{Noun}} {\definition{conch shell}}
\entry{bun}\headword{bun}{\pos{Noun}} {\definition{head}}
\entry{ddäddäg ttoe}\headword{ddäddäg ttoe}{\pos{Noun}} {\definition{leather, hide, animal skin}}
\entry{ddäg}\headword{ddäg}{\pos{Noun}} {\definition{back}}
\entry{ddäkop}\headword{ddäkop}{\pos{Noun}} {\definition{kidney}}
\entry{ddäll}\headword{ddäll}{\pos{Noun}} {\definition{chest}}
\entry{ddäll kutt}\headword{ddäll kutt}{\pos{Noun}} {\definition{sternum, breastbone}}
\entry{ddäma}\headword{ddäma}{\pos{Noun}} {\definition{pouch of a marsupial}}
\entry{dägmar}\headword{dägmar}{\pos{Noun}} {\definition{tongue}}
\entry{dägäldägäl}\headword{dägäldägäl}{\pos{Noun}} {\definition{intestines}}
\entry{gageb}\headword{gageb}{\pos{Noun}} {\definition{hind leg}}
\entry{gi}\headword{gi}{\pos{Noun}} {\definition{grease, fat}}
\entry{gogodd}\headword{gogodd}{\pos{Noun}} {\definition{spleen}}
\entry{ikop kom}\headword{ikop kom}{\pos{Noun}} {\definition{eyelash}}
\entry{ikop ku}\headword{ikop ku}{\pos{Noun}} {\definition{pupil}}
\entry{ikop ttoe}\headword{ikop ttoe}{\pos{Noun}} {\definition{eyelid}}
\entry{inkäm tär}\headword{inkäm tär}{\pos{Noun}} {\definition{throat}}
\entry{kep kutt}\headword{kep kutt}{\pos{Noun}} {\definition{hip bone}}
\entry{kote}\headword{kote}{\pos{Noun}} {\definition{nape}}
\entry{kote kutt}\headword{kote kutt}{\pos{Noun}} {\definition{cervical vertebrae}}
\entry{kwaena kutt}\headword{kwaena kutt}{\pos{Noun}} {\definition{hip bone}}
\entry{llan}\headword{llan}{\pos{Noun}} {\definition{ear}}
\entry{lläpät}\headword{lläpät}{\pos{Noun}} {\definition{digit}}
\entry{matta}\headword{matta}{\pos{Noun}} {\definition{shoulder}}
\entry{mulkul}\headword{mulkul}{\pos{Noun}} {\definition{brain}}
\entry{mälläng}\headword{mälläng}{\pos{Noun}} {\definition{nose}}
\entry{ngoi}\headword{ngoi}{\pos{Noun}} {\definition{tooth}}
\entry{pot}\headword{pot}{\pos{Noun}} {\definition{vagina of a mammal}}
\entry{putt}\headword{putt}{\pos{Noun}} {\definition{hoof}}
\entry{pätt}\headword{pätt}{\pos{Noun}} {\definition{body (of a person or animal)}}
\entry{sära}\headword{sära}{\pos{Noun}} {\definition{tail}}
\entry{sära mit}\headword{sära mit}{\pos{Noun}} {\definition{sacrum (bone)}}
\entry{tanyib}\headword{tanyib}{\pos{Noun}} {\definition{radioulna (fused radius and ulna, or arm bone) of a flying fox}}
\entry{tikop}\headword{tikop}{\pos{Noun}} {\definition{heart}}
\entry{ttoe}\headword{ttoe}{\pos{Noun}} {\definition{skin (of a person or animal)}}
\entry{ttäle lläpät}\headword{ttäle lläpät}{\pos{Noun}} {\definition{claw}}
\entry{ttäle pätt}\headword{ttäle pätt}{\pos{Noun}} {\definition{hind leg, hind limb}}
\entry{täkäll}\headword{täkäll}{\pos{Noun}} {\definition{horn}}
\entry{täkälltäkäll kutt}\headword{täkälltäkäll kutt}{\pos{Noun}} {\definition{spine}}
\entry{tätän kutt}\headword{tätän kutt}{\pos{Noun}} {\definition{rib (bone)}}
\entry{utt}\headword{utt}{\pos{Noun}} {\definition{conch shell}}
\end{entrylist}

\section*{1.6.2.1 Parts of a bird}
\begin{entrylist}
\entry{bod}\headword{bod}{\pos{Noun}} {\definition{beak}}
\entry{ddamba}\headword{ddamba}{\pos{Noun}} {\definition{wing}}
\entry{ikop}\headword{ikop}{\pos{Noun}} {\definition{eye}}
\entry{kom}\headword{kom}{\pos{Noun}} {\definition{feather}}
\entry{mas}\headword{mas}{\pos{Noun}} {\definition{small bone found in cassowaries}}
\entry{pa kom}\headword{pa kom}{\pos{Noun}} {\definition{feather}}
\entry{pite}\headword{pite}{\pos{Noun}} {\definition{tailfeather}}
\entry{pullkom}\headword{pullkom}{\pos{Noun}} {\definition{tailfeather}}
\end{entrylist}

\section*{1.6.2.2 Parts of a reptile}
\begin{entrylist}
\entry{bod}\headword{bod}{\pos{Noun}} {\definition{mouth}}
\entry{bäll kutt}\headword{bäll kutt}{\pos{Noun}} {\definition{femur}}
\entry{ddogollop}\headword{ddogollop}{\pos{Noun}} {\definition{reptile scale}}
\entry{ikop käp}\headword{ikop käp}{\pos{Noun}} {\definition{eyeball}}
\entry{intot}\headword{intot}{\pos{Noun}} {\definition{brow bone}}
\entry{mälläng sobasoba}\headword{mälläng sobasoba}{\pos{Noun}} {\definition{snout}}
\entry{sära pakall}\headword{sära pakall}{\pos{Noun}} {\definition{caudal fin}}
\entry{sära pot}\headword{sära pot}{\pos{Noun}} {\definition{tip of tail}}
\entry{ttang lläbe}\headword{ttang lläbe}{\pos{Noun}} {\definition{talon, claw}}
\entry{ttang pe}\headword{ttang pe}{\pos{Noun}} {\definition{upper part of foreleg (of a reptile)}}
\entry{ttäle pe}\headword{ttäle pe}{\pos{Noun}} {\definition{hind foot (of a reptile) with four toes}}
\entry{yo}\headword{yo}{\pos{Noun}} {\definition{liver}}
\end{entrylist}

\section*{1.6.2.3 Parts of a fish}
\begin{entrylist}
\entry{bod}\headword{bod}{\pos{Noun}} {\definition{mouth}}
\entry{ddamba}\headword{ddamba}{\pos{Noun}} {\definition{pectoral fin}}
\entry{dänräp}\headword{dänräp}{\pos{Noun}} {\definition{fish scale}}
\entry{ikop}\headword{ikop}{\pos{Noun}} {\definition{eye}}
\entry{kollba täkäll}\headword{kollba täkäll}{\pos{Noun}} {\definition{fish fin}}
\entry{kumiye käp}\headword{kumiye käp}{\pos{Noun}} {\definition{part of a fish}}
\entry{llan kollkoll}\headword{llan kollkoll}{\pos{Noun}} {\definition{gill}}
\entry{llan mit}\headword{llan mit}{\pos{Noun}} {\definition{operculum (gill cover)}}
\entry{sära}\headword{sära}{\pos{Noun}} {\definition{tail}}
\entry{sära pakall}\headword{sära pakall}{\pos{Noun}} {\definition{caudal fin}}
\entry{täkäll}\headword{täkäll}{\pos{Noun}} {\definition{fin}}
\entry{täkälluit}\headword{täkälluit}{\pos{Noun}} {\definition{radial cartilage}}
\end{entrylist}

\section*{1.6.2.4 Parts of an insect}
\begin{entrylist}
\entry{bod}\headword{bod}{\pos{Noun}} {\definition{mouth}}
\entry{ddamba}\headword{ddamba}{\pos{Noun}} {\definition{wing}}
\entry{ddamba mit}\headword{ddamba mit}{\pos{Noun}} {\definition{metathorax}}
\entry{ikop}\headword{ikop}{\pos{Noun}} {\definition{eye}}
\entry{kum}\headword{kum}{\pos{Noun}} {\definition{abdomen of an insect}}
\entry{kumpit}\headword{kumpit}{\pos{Noun}} {\definition{stinger (of an insect)}}
\entry{mälläng kom}\headword{mälläng kom}{\pos{Noun}} {\definition{antenna}}
\entry{sära pättkäp}\headword{sära pättkäp}{\pos{Noun}} {\definition{mesothorax}}
\entry{toma}\headword{toma}{\pos{Noun}} {\definition{wing}}
\entry{toma mit}\headword{toma mit}{\pos{Noun}} {\definition{metathorax}}
\end{entrylist}

\section*{1.6.2.5 Parts of small animals}
\begin{entrylist}
\entry{kätt}\headword{kätt}{\pos{Noun}} {\definition{bivalve; shell of a mollusc}}
\end{entrylist}

\section*{1.6.3 Animal life cycle}
\begin{entrylist}
\entry{bällkäp}\headword{bällkäp}{\pos{Noun}} {\definition{ant pupae}}
\entry{miny}\headword{miny}{\pos{Noun}} {\definition{ant egg/larvae (small)}}
\entry{peyom}\headword{peyom}{\pos{Modifier}} {\definition{mating}}
\end{entrylist}

\section*{1.6.3.1 Egg}
\begin{entrylist}
\entry{dirom käp}\headword{dirom käp}{\pos{Noun}} {\definition{cassowary egg}}
\entry{käp}\headword{käp}{\pos{Noun}} {\definition{egg}}
\entry{pänddäg}\headword{pänddäg}{\pos{Intransitive S verb}} {\definition{to hatch}}
\end{entrylist}

\section*{1.6.4 Animal actions}
\begin{entrylist}
\entry{peyom}\headword{peyom}{\pos{Modifier}} {\definition{mating}}
\end{entrylist}

\section*{1.6.4.1 Animal movement}
\begin{entrylist}
\entry{otal}\headword{otal}{\pos{Intransitive S verb}} {\definition{to perch}}
\entry{papiye}\headword{papiye}{\pos{Noun}} {\definition{animal tracks}}
\entry{paplläg}\headword{paplläg}{\pos{Intransitive S verb}} {\definition{to fly}}
\entry{pänyae}\headword{pänyae}{\pos{Intransitive S verb}} {\definition{to hop}}
\entry{yayo}\headword{yayo}{\pos{Verb}} {\definition{to slither}}
\end{entrylist}

\section*{1.6.4.3 Animal sounds}
\begin{entrylist}
\entry{bäbrem}\headword{bäbrem}{\pos{Ideophone}} {\definition{sound made by cassowaries}}
\entry{keam}\headword{keam}{\pos{Ideophone}} {\definition{sound made by deer}}
\entry{kunglle}\headword{kunglle}{\pos{Ideophone}} {\definition{sound made by flying foxes}}
\entry{ngok}\headword{ngok}{\pos{Ideophone}} {\definition{oink, snort}}
\entry{ngänglam}\headword{ngänglam}{\pos{Intransitive S verb}} {\definition{to buzz}}
\entry{pollpoll}\headword{pollpoll}{\pos{Transitive S verb}} {\definition{to bark (at)}}
\entry{wawaem}\headword{wawaem}{\pos{Noun}} {\definition{hiss}}
\end{entrylist}

\section*{1.6.5 Animal home}
\begin{entrylist}
\entry{däkna}\headword{däkna}{\pos{Noun}} {\definition{small black termite mound that burns for a long time; after a woman gives birth, it is heated in the fire, wrapped in bark and cloth, placed under a mat, and used to warm the woman's stomach}}
\entry{kulläb}\headword{kulläb}{\pos{Noun}} {\definition{large black termite mound}}
\entry{pera}\headword{pera}{\pos{Noun}} {\definition{bird perch}}
\entry{pudd}\headword{pudd}{\pos{Noun}} {\definition{place with flattened grass where wallabies sleep}}
\entry{ttägäll}\headword{ttägäll}{\pos{Noun}} {\definition{termite mound, anthill (made by termites; ants may also live inside)}}
\end{entrylist}

\section*{1.7 Nature, environment}
\begin{entrylist}
\entry{enma}\headword{enma}{\pos{Noun}} {\definition{nature}}
\entry{matu}\headword{matu}{\pos{Noun}} {\definition{ground}}
\end{entrylist}

\section*{2.1 Body}
\begin{entrylist}
\entry{amteamte}\headword{amteamte}{\pos{Noun}} {\definition{body part}}
\entry{ddäb}\headword{ddäb}{\pos{Noun}} {\definition{anus}}
\entry{ddäll kom}\headword{ddäll kom}{\pos{Noun}} {\definition{chest hair}}
\entry{ddäma}\headword{ddäma}{\pos{Noun}} {\definition{uterus}}
\entry{dompa}\headword{dompa}{\pos{Noun}} {\definition{penis (slang)}}
\entry{gabän}\headword{gabän}{\pos{Noun}} {\definition{wrist}}
\entry{ikop kom}\headword{ikop kom}{\pos{Noun}} {\definition{eyelash}}
\entry{ikop ku}\headword{ikop ku}{\pos{Noun}} {\definition{pupil}}
\entry{ikop ttoe}\headword{ikop ttoe}{\pos{Noun}} {\definition{eyelid}}
\entry{inkäm}\headword{inkäm}{\pos{Noun}} {\definition{neck}}
\entry{inkäm tär}\headword{inkäm tär}{\pos{Noun}} {\definition{throat}}
\entry{intot kom}\headword{intot kom}{\pos{Noun}} {\definition{eyebrow}}
\entry{kep kutt}\headword{kep kutt}{\pos{Noun}} {\definition{hip bone}}
\entry{kote}\headword{kote}{\pos{Noun}} {\definition{nape}}
\entry{kottma}\headword{kottma}{\pos{Noun}} {\definition{body part}}
\entry{kum}\headword{kum}{\pos{Noun}} {\definition{buttocks, butt}}
\entry{matta}\headword{matta}{\pos{Noun}} {\definition{shoulder}}
\entry{mälläng kobllokobllom}\headword{mälläng kobllokobllom}{\pos{Noun}} {\definition{part of the body}}
\entry{mänänggmit}\headword{mänänggmit}{\pos{Noun}} {\definition{body part}}
\entry{ngam koll}\headword{ngam koll}{\pos{Noun}} {\definition{breast}}
\entry{ngängäll käm}\headword{ngängäll käm}{\pos{Noun}} {\definition{body part}}
\entry{pongpongäll}\headword{pongpongäll}{\pos{Noun}} {\definition{body part}}
\entry{pätt}\headword{pätt}{\pos{Noun}} {\definition{body (of a person or animal)}}
\entry{suwe gäd}\headword{suwe gäd}{\pos{Noun}} {\definition{body part}}
\entry{tikop}\headword{tikop}{\pos{Noun}} {\definition{heart}}
\entry{ttäle}\headword{ttäle}{\pos{Noun}} {\definition{leg}}
\entry{ttälle}\headword{ttälle}{\pos{Noun}} {\definition{leg}}
\entry{täko}\headword{täko}{\pos{Noun}} {\definition{any body part}}
\entry{täkälltäkäll kutt}\headword{täkälltäkäll kutt}{\pos{Noun}} {\definition{spine}}
\entry{ume ttäp}\headword{ume ttäp}{\pos{Noun}} {\definition{mouth}}
\entry{wiyebubug}\headword{wiyebubug}{\pos{Noun}} {\definition{body part}}
\end{entrylist}

\section*{2.1.1 Head}
\begin{entrylist}
\entry{bun}\headword{bun}{\pos{Noun}} {\definition{head}}
\entry{bun kom}\headword{bun kom}{\pos{Noun}} {\definition{hair (on head)}}
\entry{bun kutt}\headword{bun kutt}{\pos{Noun}} {\definition{skull}}
\entry{bunkutt}\headword{bunkutt}{\pos{Noun}} {\definition{head}}
\entry{ingoll}\headword{ingoll}{\pos{Noun}} {\definition{face}}
\entry{mittapa}\headword{mittapa}{\pos{Noun}} {\definition{extremities of face (i.e. temples, cheek, chin)}}
\entry{mulkul}\headword{mulkul}{\pos{Noun}} {\definition{brain}}
\entry{teb}\headword{teb}{\pos{Noun}} {\definition{mandible, jawbone}}
\entry{ttatt kutt}\headword{ttatt kutt}{\pos{Noun}} {\definition{mandible, jawbone}}
\entry{tumku}\headword{tumku}{\pos{Noun}} {\definition{back of head}}
\entry{yätt}\headword{yätt}{\pos{Noun}} {\definition{forehead}}
\end{entrylist}

\section*{2.1.1.1 Eye}
\begin{entrylist}
\entry{daem}\headword{daem}{\pos{Intransitive A verb}} {\definition{to blink}}
\entry{ikop}\headword{ikop}{\pos{Noun}} {\definition{eye}}
\end{entrylist}

\section*{2.1.1.2 Ear}
\begin{entrylist}
\entry{llan}\headword{llan}{\pos{Noun}} {\definition{ear}}
\entry{llan ik}\headword{llan ik}{\pos{Noun}} {\definition{eardrum}}
\entry{llan kom}\headword{llan kom}{\pos{Noun}} {\definition{ear hair}}
\entry{llan källa}\headword{llan källa}{\pos{Noun}} {\definition{earwax}}
\end{entrylist}

\section*{2.1.1.3 Nose}
\begin{entrylist}
\entry{mollemolle}\headword{mollemolle}{\pos{Transitive A verb}} {\definition{to sniff, smell}}
\entry{mälläng}\headword{mälläng}{\pos{Noun}} {\definition{nose}}
\entry{mälläng ik}\headword{mälläng ik}{\pos{Noun}} {\definition{nostril}}
\entry{mälläng kom}\headword{mälläng kom}{\pos{Noun}} {\definition{nose hair}}
\end{entrylist}

\section*{2.1.1.4 Mouth}
\begin{entrylist}
\entry{bod}\headword{bod}{\pos{Noun}} {\definition{mouth}}
\entry{bod ttoe}\headword{bod ttoe}{\pos{Noun}} {\definition{lip}}
\entry{dägmar}\headword{dägmar}{\pos{Noun}} {\definition{tongue}}
\entry{ekamatär}\headword{ekamatär}{\pos{Noun}} {\definition{throat}}
\entry{gallab}\headword{gallab}{\pos{Transitive S verb}} {\definition{to open (one's mouth)}}
\entry{inkätt}\headword{inkätt}{\pos{Noun}} {\definition{neck; throat}}
\entry{mirmir}\headword{mirmir}{\pos{Transitive S verb}} {\definition{to lick}}
\entry{ttatt}\headword{ttatt}{\pos{Noun}} {\definition{jaw, chin}}
\entry{umettäp}\headword{umettäp}{\pos{Noun}} {\definition{mouth}}
\end{entrylist}

\section*{2.1.1.5 Tooth}
\begin{entrylist}
\entry{ngoi}\headword{ngoi}{\pos{Noun}} {\definition{tooth}}
\end{entrylist}

\section*{2.1.2 Torso}
\begin{entrylist}
\entry{ddäg}\headword{ddäg}{\pos{Noun}} {\definition{back}}
\entry{ddäll}\headword{ddäll}{\pos{Noun}} {\definition{chest}}
\entry{ddäll kutt}\headword{ddäll kutt}{\pos{Noun}} {\definition{sternum, breastbone}}
\entry{gablle}\headword{gablle}{\pos{Noun}} {\definition{hip; waist}}
\entry{kep}\headword{kep}{\pos{Noun}} {\definition{hip}}
\entry{kowa}\headword{kowa}{\pos{Noun}} {\definition{upper back}}
\entry{källän}\headword{källän}{\pos{Noun}} {\definition{waist}}
\entry{ngam}\headword{ngam}{\pos{Noun}} {\definition{breast}}
\entry{poder}\headword{poder}{\pos{Noun}} {\definition{scapula, shoulder blade}}
\entry{saomasaoma}\headword{saomasaoma}{\pos{Noun}} {\definition{armpit}}
\entry{ttatta}\headword{ttatta}{\pos{Noun}} {\definition{lower back}}
\entry{wänkäm}\headword{wänkäm}{\pos{Noun}} {\definition{belly button, navel}}
\end{entrylist}

\section*{2.1.3.1 Arm}
\begin{entrylist}
\entry{olmapänyik}\headword{olmapänyik}{\pos{Noun}} {\definition{armpit}}
\entry{saomasaoma}\headword{saomasaoma}{\pos{Noun}} {\definition{armpit}}
\entry{ttang}\headword{ttang}{\pos{Noun}} {\definition{hand}}
\entry{ttang koll}\headword{ttang koll}{\pos{Noun}} {\definition{palm}}
\entry{ttang kum}\headword{ttang kum}{\pos{Noun}} {\definition{elbow}}
\entry{ttang pättkäp}\headword{ttang pättkäp}{\pos{Noun}} {\definition{tibia}}
\end{entrylist}

\section*{2.1.3.2 Leg}
\begin{entrylist}
\entry{anggog}\headword{anggog}{\pos{Noun}} {\definition{thigh}}
\entry{bäll}\headword{bäll}{\pos{Noun}} {\definition{thigh}}
\entry{do}\headword{do}{\pos{Noun}} {\definition{femur}}
\entry{mankäp}\headword{mankäp}{\pos{Noun}} {\definition{calf (back of leg)}}
\entry{nying}\headword{nying}{\pos{Noun}} {\definition{foot}}
\entry{ttäle koll}\headword{ttäle koll}{\pos{Noun}} {\definition{instep}}
\entry{ttäle pättkäp}\headword{ttäle pättkäp}{\pos{Noun}} {\definition{tibia}}
\entry{tubu}\headword{tubu}{\pos{Noun}} {\definition{knee}}
\entry{tubukätt}\headword{tubukätt}{\pos{Noun}} {\definition{kneecap}}
\end{entrylist}

\section*{2.1.3.3 Finger, toe}
\begin{entrylist}
\entry{källatolma}\headword{källatolma}{\pos{Noun}} {\definition{middle finger}}
\entry{llimba}\headword{llimba}{\pos{Noun}} {\definition{fingernail}}
\entry{lläbe}\headword{lläbe}{\pos{Noun}} {\definition{nail, claw (of a finger or toe)}}
\entry{lläpät}\headword{lläpät}{\pos{Noun}} {\definition{digit}}
\entry{mända}\headword{mända}{\pos{Noun}} {\definition{thumb; big toe}}
\entry{mätkin}\headword{mätkin}{\pos{Noun}} {\definition{ring finger}}
\entry{nying lläbe}\headword{nying lläbe}{\pos{Noun}} {\definition{toenail}}
\entry{nying lläpät}\headword{nying lläpät}{\pos{Noun}} {\definition{toe}}
\entry{ttang lläbe}\headword{ttang lläbe}{\pos{Noun}} {\definition{talon, claw}}
\entry{ttang lläpät}\headword{ttang lläpät}{\pos{Noun}} {\definition{finger}}
\entry{tupi}\headword{tupi}{\pos{Noun}} {\definition{pointer finger, index finger}}
\entry{tärangesa}\headword{tärangesa}{\pos{Noun}} {\definition{pinky, little finger}}
\end{entrylist}

\section*{2.1.4 Skin}
\begin{entrylist}
\entry{bun dänräp}\headword{bun dänräp}{\pos{Noun}} {\definition{dandruff}}
\entry{dänräp}\headword{dänräp}{\pos{Noun}} {\definition{scab}}
\entry{piyepiye}\headword{piyepiye}{\pos{Noun}} {\definition{blister}}
\entry{päd}\headword{päd}{\pos{Noun}} {\definition{scar}}
\entry{tokop}\headword{tokop}{\pos{Noun}} {\definition{lump on skin}}
\entry{ttoe}\headword{ttoe}{\pos{Noun}} {\definition{skin (of a person or animal)}}
\end{entrylist}

\section*{2.1.5 Hair}
\begin{entrylist}
\entry{bun kom}\headword{bun kom}{\pos{Noun}} {\definition{hair (on head)}}
\entry{intot kom}\headword{intot kom}{\pos{Noun}} {\definition{eyebrow}}
\entry{kom}\headword{kom}{\pos{Noun}} {\definition{hair; hair-like thing}}
\entry{llan kom}\headword{llan kom}{\pos{Noun}} {\definition{ear hair}}
\entry{mälläng kom}\headword{mälläng kom}{\pos{Noun}} {\definition{nose hair}}
\entry{pälkom}\headword{pälkom}{\pos{Noun}} {\definition{body hair}}
\entry{pätär}\headword{pätär}{\pos{Noun}} {\definition{white hair}}
\entry{ttatt kom}\headword{ttatt kom}{\pos{Noun}} {\definition{beard}}
\end{entrylist}

\section*{2.1.6 Bone, joint}
\begin{entrylist}
\entry{bun kutt}\headword{bun kutt}{\pos{Noun}} {\definition{skull}}
\entry{bägäbägäl}\headword{bägäbägäl}{\pos{Noun}} {\definition{Achilles/calcaneal tendon (on the back of ankle)}}
\entry{bäll kutt}\headword{bäll kutt}{\pos{Noun}} {\definition{femur}}
\entry{ddäg kutt}\headword{ddäg kutt}{\pos{Noun}} {\definition{backbone, spine}}
\entry{ddäll kutt}\headword{ddäll kutt}{\pos{Noun}} {\definition{sternum, breastbone}}
\entry{do}\headword{do}{\pos{Noun}} {\definition{femur}}
\entry{intot}\headword{intot}{\pos{Noun}} {\definition{brow bone}}
\entry{kalkmo}\headword{kalkmo}{\pos{Noun}} {\definition{joints}}
\entry{kote kutt}\headword{kote kutt}{\pos{Noun}} {\definition{cervical vertebrae}}
\entry{kutt}\headword{kutt}{\pos{Noun}} {\definition{bone}}
\entry{kwaena kutt}\headword{kwaena kutt}{\pos{Noun}} {\definition{hip bone}}
\entry{poder}\headword{poder}{\pos{Noun}} {\definition{scapula, shoulder blade}}
\entry{sära mit}\headword{sära mit}{\pos{Noun}} {\definition{sacrum (bone)}}
\entry{tanyib}\headword{tanyib}{\pos{Noun}} {\definition{radioulna (fused radius and ulna, or arm bone) of a flying fox}}
\entry{teb}\headword{teb}{\pos{Noun}} {\definition{mandible, jawbone}}
\entry{ttang pättkäp}\headword{ttang pättkäp}{\pos{Noun}} {\definition{tibia}}
\entry{ttatt kutt}\headword{ttatt kutt}{\pos{Noun}} {\definition{mandible, jawbone}}
\entry{ttäle pättkäp}\headword{ttäle pättkäp}{\pos{Noun}} {\definition{tibia}}
\entry{tubukätt}\headword{tubukätt}{\pos{Noun}} {\definition{kneecap}}
\entry{tätän}\headword{tätän}{\pos{Noun}} {\definition{rib, side, flank}}
\entry{tätän kutt}\headword{tätän kutt}{\pos{Noun}} {\definition{rib (bone)}}
\end{entrylist}

\section*{2.1.8 Internal organs}
\begin{entrylist}
\entry{ddokop}\headword{ddokop}{\pos{Noun}} {\definition{kidney}}
\entry{ddäkop}\headword{ddäkop}{\pos{Noun}} {\definition{kidney}}
\entry{dupi}\headword{dupi}{\pos{Noun}} {\definition{stomach}}
\entry{dägäldägäl}\headword{dägäldägäl}{\pos{Noun}} {\definition{intestines}}
\entry{gogodd}\headword{gogodd}{\pos{Noun}} {\definition{spleen}}
\entry{gogodd}\headword{gogodd}{\pos{Noun}} {\definition{bladder}}
\entry{käll}\headword{käll}{\pos{Noun}} {\definition{spleen}}
\entry{källa}\headword{källa}{\pos{Noun}} {\definition{intestines, bowels, guts, innards (of an animal)}}
\entry{käm}\headword{käm}{\pos{Noun}} {\definition{stomach}}
\entry{petron}\headword{petron}{\pos{Noun}} {\definition{blood vessel; vein; artery}}
\entry{pllulli}\headword{pllulli}{\pos{Noun}} {\definition{lung}}
\entry{pällulle}\headword{pällulle}{\pos{Noun}} {\definition{lung}}
\entry{ttäbottäbo}\headword{ttäbottäbo}{\pos{Noun}} {\definition{rectum}}
\entry{yo}\headword{yo}{\pos{Noun}} {\definition{liver}}
\end{entrylist}

\section*{2.1.8.1 Heart}
\begin{entrylist}
\entry{guwo}\headword{guwo}{\pos{Noun}} {\definition{heart}}
\end{entrylist}

\section*{2.1.8.2 Stomach}
\begin{entrylist}
\entry{wunkäm}\headword{wunkäm}{\pos{Noun}} {\definition{anus}}
\end{entrylist}

\section*{2.1.8.3 Male organs}
\begin{entrylist}
\entry{kott}\headword{kott}{\pos{Noun}} {\definition{testicles}}
\entry{kämätt}\headword{kämätt}{\pos{Noun}} {\definition{testicle}}
\entry{suwe}\headword{suwe}{\pos{Noun}} {\definition{(euphemistic) testicles}}
\entry{wa}\headword{wa}{\pos{Noun}} {\definition{penis}}
\entry{wa piye}\headword{wa piye}{\pos{Noun}} {\definition{sperm}}
\end{entrylist}

\section*{2.1.8.4 Female organs}
\begin{entrylist}
\entry{ddäma}\headword{ddäma}{\pos{Noun}} {\definition{uterus}}
\entry{dirom gäz}\headword{dirom gäz}{\pos{Transitive S verb}} {\definition{to menstruate for the first time}}
\entry{giddollag}\headword{giddollag}{\pos{Modifier}} {\definition{menstruating, on one's period}}
\entry{käm}\headword{käm}{\pos{Noun}} {\definition{womb}}
\entry{llɨg dum}\headword{llɨg dum}{\pos{Noun}} {\definition{womb, uterus}}
\entry{ngam}\headword{ngam}{\pos{Noun}} {\definition{breast}}
\entry{ngam indäb}\headword{ngam indäb}{\pos{Noun}} {\definition{nipple, teat}}
\entry{täk}\headword{täk}{\pos{Noun}} {\definition{clitoris}}
\entry{yowa}\headword{yowa}{\pos{Noun}} {\definition{vagina}}
\entry{zawatt}\headword{zawatt}{\pos{Noun}} {\definition{vagina}}
\end{entrylist}

\section*{2.2 Body functions}
\begin{entrylist}
\entry{kallat}\headword{kallat}{\pos{}} {\definition{burp}}
\entry{kutt tataem}\headword{kutt tataem}{\pos{Intransitive A verb}} {\definition{to shiver}}
\entry{källayoyo}\headword{källayoyo}{\pos{Noun}} {\definition{type of tree that grows in the bush with leaves used as toilet paper}}
\entry{mam}\headword{mam}{\pos{Intransitive A verb}} {\definition{to bleed}}
\entry{mänymäny}\headword{mänymäny}{\pos{Verb}} {\definition{to vomit}}
\entry{ngänyngäny}\headword{ngänyngäny}{\pos{Transitive S verb}} {\definition{to swallow}}
\entry{susu}\headword{susu}{\pos{Noun}} {\definition{breastmilk}}
\end{entrylist}

\section*{2.2.1 Breathe, breath}
\begin{entrylist}
\entry{amtet}\headword{amtet}{\pos{Noun}} {\definition{breath}}
\entry{pllulli}\headword{pllulli}{\pos{Noun}} {\definition{lung}}
\end{entrylist}

\section*{2.2.2 Cough, sneeze}
\begin{entrylist}
\entry{ansi}\headword{ansi}{\pos{Intransitive A verb}} {\definition{to sneeze}}
\entry{kumiye}\headword{kumiye}{\pos{Noun}} {\definition{cough}}
\end{entrylist}

\section*{2.2.3 Spit, saliva}
\begin{entrylist}
\entry{bällma}\headword{bällma}{\pos{Noun}} {\definition{spit, saliva}}
\entry{bällma käkan}\headword{bällma käkan}{\pos{Noun}} {\definition{spit, saliva}}
\entry{mirmir}\headword{mirmir}{\pos{Transitive S verb}} {\definition{to lick}}
\end{entrylist}

\section*{2.2.4 Mucus}
\begin{entrylist}
\entry{kumiye käp}\headword{kumiye käp}{\pos{Noun}} {\definition{phlegm}}
\end{entrylist}

\section*{2.2.5 Bleed, blood}
\begin{entrylist}
\entry{mam}\headword{mam}{\pos{Noun}} {\definition{blood}}
\entry{petron}\headword{petron}{\pos{Noun}} {\definition{blood vessel; vein; artery}}
\end{entrylist}

\section*{2.2.6 Sweat}
\begin{entrylist}
\entry{memram}\headword{memram}{\pos{Noun}} {\definition{sweat}}
\end{entrylist}

\section*{2.2.7 Urinate, urine}
\begin{entrylist}
\entry{suwe}\headword{suwe}{\pos{Noun}} {\definition{urine, pee}}
\entry{suwe tɨngg}\headword{suwe tɨngg}{\pos{Transitive S verb}} {\definition{to urinate on}}
\entry{ttätt}\headword{ttätt}{\pos{Intransitive S verb}} {\definition{to urinate}}
\entry{tängg}\headword{tängg}{\pos{Transitive S verb}} {\definition{to urinate on, pee on}}
\end{entrylist}

\section*{2.2.8 Defecate, feces}
\begin{entrylist}
\entry{källa}\headword{källa}{\pos{Noun}} {\definition{feces, poop, waste}}
\entry{källama kup}\headword{källama kup}{\pos{Noun}} {\definition{toilet (lit. hole for pooping)}}
\entry{toelet}\headword{toelet}{\pos{Noun}} {\definition{toilet}}
\end{entrylist}

\section*{2.3 Sense, perceive}
\begin{entrylist}
\entry{dändär}\headword{dändär}{\pos{Transitive S verb}} {\definition{to sense, feel}}
\entry{gany}\headword{gany}{\pos{Transitive S verb}} {\definition{to focus, tune}}
\entry{molle dändär}\headword{molle dändär}{\pos{Transitive S verb}} {\definition{to smell, perceive a smell}}
\entry{sämongg}\headword{sämongg}{\pos{Intransitive S verb}} {\definition{to feel}}
\end{entrylist}

\section*{2.3.1 See}
\begin{entrylist}
\entry{ikop}\headword{ikop}{\pos{Transitive A verb}} {\definition{to see}}
\end{entrylist}

\section*{2.3.1.1 Look}
\begin{entrylist}
\entry{ikoikop}\headword{ikoikop}{\pos{Verb}} {\definition{to watch}}
\entry{ikop}\headword{ikop}{\pos{Transitive/Intransitive A verb}} {\definition{to look}}
\entry{ikop ddäddäg}\headword{ikop ddäddäg}{\pos{Transitive S verb}} {\definition{to look, watch}}
\entry{ngällae}\headword{ngällae}{\pos{Intransitive S verb}} {\definition{to look back}}
\end{entrylist}

\section*{2.3.1.2 Watch}
\begin{entrylist}
\entry{ikop ddäddäg}\headword{ikop ddäddäg}{\pos{Transitive S verb}} {\definition{to look, watch}}
\entry{lla ikoikopang}\headword{lla ikoikopang}{\pos{Noun}} {\definition{audience}}
\entry{täbab}\headword{täbab}{\pos{Transitive S verb}} {\definition{to watch, look after, patrol}}
\end{entrylist}

\section*{2.3.1.5 Visible}
\begin{entrylist}
\entry{pälläm}\headword{pälläm}{\pos{Modifier}} {\definition{visible}}
\end{entrylist}

\section*{2.3.1.5.1 Appear}
\begin{entrylist}
\entry{peyam}\headword{peyam}{\pos{Intransitive S verb}} {\definition{to pop up, reappear, resurface}}
\entry{pängän}\headword{pängän}{\pos{Intransitive S verb}} {\definition{to disappear}}
\entry{ttam}\headword{ttam}{\pos{Intransitive S verb}} {\definition{to appear}}
\end{entrylist}

\section*{2.3.1.7 Reflect, mirror}
\begin{entrylist}
\entry{anykeanyke}\headword{anykeanyke}{\pos{Noun}} {\definition{picture, image, reflection}}
\end{entrylist}

\section*{2.3.2 Hear}
\begin{entrylist}
\entry{dändär}\headword{dändär}{\pos{Transitive S verb}} {\definition{to hear, listen}}
\end{entrylist}

\section*{2.3.2.1 Listen}
\begin{entrylist}
\entry{dändär}\headword{dändär}{\pos{Transitive S verb}} {\definition{to hear, listen}}
\entry{llan gonggo}\headword{llan gonggo}{\pos{Transitive S verb}} {\definition{to turn one's ear, listen intently}}
\entry{llandräg}\headword{llandräg}{\pos{Intransitive A verb}} {\definition{to listen}}
\end{entrylist}

\section*{2.3.2.2 Sound}
\begin{entrylist}
\entry{amamärang}\headword{amamärang}{\pos{Intransitive A verb}} {\definition{to echo}}
\entry{arälle}\headword{arälle}{\pos{Noun}} {\definition{sound of the wallaby}}
\entry{eka}\headword{eka}{\pos{Noun}} {\definition{sound, song, call}}
\entry{ekaeka}\headword{ekaeka}{\pos{Intransitive A verb}} {\definition{to make noise}}
\entry{gogäle}\headword{gogäle}{\pos{Noun}} {\definition{noise}}
\entry{longgo}\headword{longgo}{\pos{Noun}} {\definition{noise}}
\entry{tatäraem}\headword{tatäraem}{\pos{Noun}} {\definition{noise}}
\end{entrylist}

\section*{2.3.2.3 Types of sounds}
\begin{entrylist}
\entry{llällam}\headword{llällam}{\pos{Ideophone}} {\definition{rustling (of plants); roaring (of water)}}
\entry{mämrem}\headword{mämrem}{\pos{Intransitive A verb}} {\definition{to growl}}
\entry{ngänglam}\headword{ngänglam}{\pos{Intransitive S verb}} {\definition{to buzz}}
\entry{seseyam}\headword{seseyam}{\pos{Ideophone}} {\definition{swish, sound of legs moving}}
\entry{ttang pllallem}\headword{ttang pllallem}{\pos{Intransitive A verb}} {\definition{to clap}}
\entry{ttällallem}\headword{ttällallem}{\pos{Intransitive A verb}} {\definition{to make a light noise}}
\end{entrylist}

\section*{2.3.2.4 Loud}
\begin{entrylist}
\entry{bällae}\headword{bällae}{\pos{Modifier}} {\definition{loud}}
\entry{ddoddollem}\headword{ddoddollem}{\pos{Transitive/Intransitive A verb}} {\definition{to make noise}}
\entry{gogäle}\headword{gogäle}{\pos{Noun}} {\definition{noise}}
\entry{tuyem}\headword{tuyem}{\pos{Intransitive A verb}} {\definition{to make a loud noise}}
\entry{waglla}\headword{waglla}{\pos{Noun}} {\definition{bullroarer}}
\end{entrylist}

\section*{2.3.2.5 Quiet}
\begin{entrylist}
\entry{känyär}\headword{känyär}{\pos{Modifier}} {\definition{quiet}}
\entry{känyärtto}\headword{känyärtto}{\pos{Modifier}} {\definition{silent, quiet}}
\entry{mätaru}\headword{mätaru}{\pos{Modifier}} {\definition{calm, peaceful, quiet}}
\entry{trangg}\headword{trangg}{\pos{Transitive S verb}} {\definition{to stop}}
\end{entrylist}

\section*{2.3.3 Taste}
\begin{entrylist}
\entry{dɨl}\headword{dɨl}{\pos{Modifier}} {\definition{bitter; sour}}
\entry{erängg}\headword{erängg}{\pos{Transitive S verb}} {\definition{to test, try; taste}}
\entry{koepang}\headword{koepang}{\pos{Modifier}} {\definition{sour}}
\entry{mirmir}\headword{mirmir}{\pos{Transitive S verb}} {\definition{to lick}}
\entry{moko}\headword{moko}{\pos{Noun}} {\definition{taste, flavor}}
\entry{mokoang}\headword{mokoang}{\pos{Modifier}} {\definition{tasty, delicious, flavorful; sweet}}
\entry{tomowang}\headword{tomowang}{\pos{Modifier}} {\definition{sour; bitter}}
\end{entrylist}

\section*{2.3.4 Smell}
\begin{entrylist}
\entry{molle}\headword{molle}{\pos{Noun}} {\definition{scent, odor, smell}}
\entry{molle dungg}\headword{molle dungg}{\pos{Transitive S verb}} {\definition{to smell, take a whiff}}
\entry{molle dändär}\headword{molle dändär}{\pos{Transitive S verb}} {\definition{to smell, perceive a smell}}
\entry{molleang}\headword{molleang}{\pos{Modifier}} {\definition{scented}}
\entry{mollemeny}\headword{mollemeny}{\pos{Modifier}} {\definition{odorless}}
\entry{mollemolle}\headword{mollemolle}{\pos{Transitive A verb}} {\definition{to sniff, smell}}
\entry{mälläng seam}\headword{mälläng seam}{\pos{Verb}} {\definition{to wrinkle up your nose and breathe out, reaction to a bad smell.}}
\entry{tänggag}\headword{tänggag}{\pos{Transitive S verb}} {\definition{to make a dog more sensitive to smells by rubbing lemongrass on their nose}}
\end{entrylist}

\section*{2.4 Body condition}
\begin{entrylist}
\entry{sämongg}\headword{sämongg}{\pos{Intransitive S verb}} {\definition{to feel}}
\end{entrylist}

\section*{2.4.1 Strong}
\begin{entrylist}
\entry{llokott}\headword{llokott}{\pos{Modifier}} {\definition{strong}}
\end{entrylist}

\section*{2.4.2 Weak}
\begin{entrylist}
\entry{dämbag}\headword{dämbag}{\pos{Noun}} {\definition{lazy person, weak person}}
\entry{mängallmeny}\headword{mängallmeny}{\pos{Modifier}} {\definition{weak}}
\end{entrylist}

\section*{2.4.4 Tired}
\begin{entrylist}
\entry{llowam}\headword{llowam}{\pos{Noun}} {\definition{fatigue, tiredness}}
\entry{llowamang}\headword{llowamang}{\pos{Modifier}} {\definition{tired}}
\entry{pumi}\headword{pumi}{\pos{Modifier}} {\definition{exhausting, tiring, strenuous}}
\end{entrylist}

\section*{2.4.5 Rest}
\begin{entrylist}
\entry{ellollo}\headword{ellollo}{\pos{Intransitive S verb}} {\definition{to put down load and rest}}
\entry{sipel}\headword{sipel}{\pos{Intransitive A verb}} {\definition{to rest}}
\end{entrylist}

\section*{2.5 Healthy}
\begin{entrylist}
\entry{damong}\headword{damong}{\pos{Modifier}} {\definition{healthy, well}}
\entry{käm}\headword{käm}{\pos{Transitive S verb}} {\definition{to heal}}
\entry{pämpllätt}\headword{pämpllätt}{\pos{Intransitive S verb}} {\definition{to become healthy; grow}}
\end{entrylist}

\section*{2.5.1 Sick}
\begin{entrylist}
\entry{itrell}\headword{itrell}{\pos{Noun}} {\definition{disease, illness, sickness}}
\entry{itrellang}\headword{itrellang}{\pos{Modifier}} {\definition{ill, sick}}
\entry{kumie}\headword{kumie}{\pos{Noun}} {\definition{cough}}
\entry{piro ttängattänge}\headword{piro ttängattänge}{\pos{Transitive S verb}} {\definition{to become unconscious}}
\entry{pungg}\headword{pungg}{\pos{Intransitive S verb}} {\definition{to get sick, be in pain}}
\entry{päkam}\headword{päkam}{\pos{Verb}} {\definition{to bend down, keel over (due to sickness)}}
\entry{ttattleang}\headword{ttattleang}{\pos{Modifier}} {\definition{sore, in pain; sick, ill}}
\entry{ttattllong}\headword{ttattllong}{\pos{Modifier}} {\definition{ill, sick}}
\end{entrylist}

\section*{2.5.2 Disease}
\begin{entrylist}
\entry{eiz}\headword{eiz}{\pos{Noun}} {\definition{HIV; AIDS}}
\entry{giro}\headword{giro}{\pos{Noun}} {\definition{chicken pox}}
\entry{itrell}\headword{itrell}{\pos{Noun}} {\definition{disease, illness, sickness}}
\entry{lepresi}\headword{lepresi}{\pos{Noun}} {\definition{leprosy}}
\entry{pentae}\headword{pentae}{\pos{Intransitive S verb}} {\definition{to spread, be transmitted}}
\entry{pentae}\headword{pentae}{\pos{Intransitive S verb}} {\definition{to transfer, transmit, spread, pass on, pass down}}
\entry{tibi}\headword{tibi}{\pos{Noun}} {\definition{tuberculosis}}
\entry{zem}\headword{zem}{\pos{Noun}} {\definition{germ}}
\end{entrylist}

\section*{2.5.2.2 Skin disease}
\begin{entrylist}
\entry{mäka}\headword{mäka}{\pos{Noun}} {\definition{plantar wart (on the soles of feet; caused by a virus)}}
\entry{piye}\headword{piye}{\pos{Noun}} {\definition{pus}}
\entry{tokop}\headword{tokop}{\pos{Noun}} {\definition{lump on skin}}
\entry{tomäll}\headword{tomäll}{\pos{Noun}} {\definition{wart; fungal skin infection}}
\entry{ute}\headword{ute}{\pos{Noun}} {\definition{sore, blister; wound}}
\entry{wän}\headword{wän}{\pos{Noun}} {\definition{boil (on skin)}}
\end{entrylist}

\section*{2.5.3 Injure}
\begin{entrylist}
\entry{ute}\headword{ute}{\pos{Noun}} {\definition{sore, blister; wound}}
\end{entrylist}

\section*{2.5.3.2 Poison}
\begin{entrylist}
\entry{bikme}\headword{bikme}{\pos{Noun}} {\definition{type of palm tree with hanging, poisonous yellow and green fruits that can be eaten after being buried by the creek for up to 2 years and then cooked on the fire}}
\entry{budombudom}\headword{budombudom}{\pos{Noun}} {\definition{red ants}}
\entry{dompak}\headword{dompak}{\pos{Noun}} {\definition{eel}}
\entry{komo}\headword{komo}{\pos{Noun}} {\definition{centipede}}
\entry{komo mogmog}\headword{komo mogmog}{\pos{Noun}} {\definition{type of black scorpion}}
\entry{komo takle}\headword{komo takle}{\pos{Noun}} {\definition{type of scorpion}}
\entry{konykony}\headword{konykony}{\pos{Noun}} {\definition{type of stinging insect that lives in the ground}}
\entry{kuibiag}\headword{kuibiag}{\pos{Noun}} {\definition{Papuan black snake}}
\entry{kumpit}\headword{kumpit}{\pos{Noun}} {\definition{stinger (of an insect)}}
\entry{kunu}\headword{kunu}{\pos{Noun}} {\definition{type of short tree that grows in the grassland with poisonous bark for stunning fish}}
\entry{kämo}\headword{kämo}{\pos{Noun}} {\definition{centipede}}
\entry{moepo}\headword{moepo}{\pos{Noun}} {\definition{type of tree that grows in the bush with white flowers, inedible red fruit, and poisonous seeds and bark; an indicator that the soil is fertile and good for making a garden}}
\entry{omgälgäl}\headword{omgälgäl}{\pos{Noun}} {\definition{red ants}}
\entry{pip}\headword{pip}{\pos{Noun}} {\definition{red bee}}
\entry{pänbäll}\headword{pänbäll}{\pos{Noun}} {\definition{poisonous vine or root (used in fishing to stun fish)}}
\entry{sili}\headword{sili}{\pos{Noun}} {\definition{chili}}
\entry{tamllägtamlläg}\headword{tamllägtamlläg}{\pos{Noun}} {\definition{type of small caterpillar}}
\entry{tatruk}\headword{tatruk}{\pos{Noun}} {\definition{type of poisonous ant}}
\entry{tintromol}\headword{tintromol}{\pos{Noun}} {\definition{type of black or red ants}}
\entry{yaber}\headword{yaber}{\pos{Noun}} {\definition{type of tree that grows in the bush with white flowers and poisonous bark used to catch fish}}
\end{entrylist}

\section*{2.5.4.1 Blind}
\begin{entrylist}
\entry{ikop songgorag}\headword{ikop songgorag}{\pos{Modifier}} {\definition{blind}}
\entry{kätang}\headword{kätang}{\pos{Modifier}} {\definition{blind}}
\end{entrylist}

\section*{2.5.4.3 Deaf}
\begin{entrylist}
\entry{ttomoll}\headword{ttomoll}{\pos{Modifier}} {\definition{deaf}}
\end{entrylist}

\section*{2.5.6.1 Pain}
\begin{entrylist}
\entry{arle}\headword{arle}{\pos{Noun}} {\definition{scream}}
\entry{ddäddäg}\headword{ddäddäg}{\pos{Transitive S verb}} {\definition{to pain, ache, hurt}}
\entry{gongg}\headword{gongg}{\pos{Intransitive S verb}} {\definition{to disturb a bees nest and then to feel the bites}}
\entry{itrell}\headword{itrell}{\pos{Intransitive S verb}} {\definition{to get hurt}}
\entry{kakep}\headword{kakep}{\pos{Noun}} {\definition{pain}}
\entry{käklläp}\headword{käklläp}{\pos{Transitive S verb}} {\definition{to sting, be painful}}
\entry{pameny}\headword{pameny}{\pos{Intransitive S verb}} {\definition{to scream}}
\entry{pungg}\headword{pungg}{\pos{Intransitive S verb}} {\definition{to get sick, be in pain}}
\entry{ttattle}\headword{ttattle}{\pos{Noun}} {\definition{pain}}
\entry{ttattleang}\headword{ttattleang}{\pos{Modifier}} {\definition{sore, in pain; sick, ill}}
\entry{tätäp}\headword{tätäp}{\pos{Noun}} {\definition{pain}}
\end{entrylist}

\section*{2.5.6.2 Fever}
\begin{entrylist}
\entry{ugri}\headword{ugri}{\pos{Noun}} {\definition{fever}}
\end{entrylist}

\section*{2.5.6.3 Swell}
\begin{entrylist}
\entry{piyepiye}\headword{piyepiye}{\pos{Noun}} {\definition{blister}}
\end{entrylist}

\section*{2.5.6.4 Lose consciousness}
\begin{entrylist}
\entry{ikop särem}\headword{ikop särem}{\pos{Noun}} {\definition{fainting, losing consciousness}}
\end{entrylist}

\section*{2.5.6.5 Dazed, confused}
\begin{entrylist}
\entry{ikikib}\headword{ikikib}{\pos{Noun}} {\definition{dizziness}}
\entry{konkon}\headword{konkon}{\pos{Modifier}} {\definition{intoxicated, intoxicating, consciousness-altering, drunk}}
\end{entrylist}

\section*{2.5.7.1 Doctor, nurse}
\begin{entrylist}
\entry{dektta}\headword{dektta}{\pos{Noun}} {\definition{doctor}}
\entry{nes}\headword{nes}{\pos{Noun}} {\definition{nurse}}
\entry{ute mälanenang}\headword{ute mälanenang}{\pos{Noun}} {\definition{healthcare worker}}
\end{entrylist}

\section*{2.5.7.2 Medicine}
\begin{entrylist}
\entry{meresin}\headword{meresin}{\pos{Noun}} {\definition{medicine}}
\entry{mullamulla}\headword{mullamulla}{\pos{Noun}} {\definition{medicine}}
\entry{zuwoe}\headword{zuwoe}{\pos{Transitive S verb}} {\definition{to pierce; to inject, administer a shot}}
\end{entrylist}

\section*{2.5.7.3 Medicinal plants}
\begin{entrylist}
\entry{gamo}\headword{gamo}{\pos{Noun}} {\definition{plant type}}
\entry{komotupi}\headword{komotupi}{\pos{Noun}} {\definition{long ginger}}
\end{entrylist}

\section*{2.5.7.4 Hospital}
\begin{entrylist}
\entry{mälamäle}\headword{mälamäle}{\pos{Transitive S verb}} {\definition{to dress (a wound)}}
\entry{ospel}\headword{ospel}{\pos{Noun}} {\definition{hospital; aid post}}
\entry{pallam}\headword{pallam}{\pos{Transitive S verb}} {\definition{to cut open}}
\entry{pam}\headword{pam}{\pos{Noun}} {\definition{injection, shot, vaccine}}
\entry{piro ttängattänge}\headword{piro ttängattänge}{\pos{Transitive S verb}} {\definition{to become unconscious}}
\entry{tät}\headword{tät}{\pos{Noun}} {\definition{stretcher}}
\entry{uteute ma}\headword{uteute ma}{\pos{Noun}} {\definition{aid post}}
\end{entrylist}

\section*{2.5.7.5 Traditional medicine}
\begin{entrylist}
\entry{ankom}\headword{ankom}{\pos{Noun}} {\definition{ant}}
\entry{badar}\headword{badar}{\pos{Noun}} {\definition{type of tree that grows in the grassland with white flowers}}
\entry{boe}\headword{boe}{\pos{Noun}} {\definition{type of cultivated tree with edible indigo fruit and white flowers that attract birds and butterflies}}
\entry{bädma}\headword{bädma}{\pos{Noun}} {\definition{type of medicinal plant}}
\entry{dara}\headword{dara}{\pos{Noun}} {\definition{type of traditional medicine}}
\entry{dowa}\headword{dowa}{\pos{Noun}} {\definition{type of tree that grows in the grassland along creeks with wood used for firewood}}
\entry{dumbi}\headword{dumbi}{\pos{Noun}} {\definition{type of red tree}}
\entry{däba}\headword{däba}{\pos{Noun}} {\definition{type of tree that grows in the grassland with leaves used to wrap sago and durable wood used for kundu drums, house posts, and formerly, bridges}}
\entry{däkna}\headword{däkna}{\pos{Noun}} {\definition{small black termite mound that burns for a long time; after a woman gives birth, it is heated in the fire, wrapped in bark and cloth, placed under a mat, and used to warm the woman's stomach}}
\entry{dämar}\headword{dämar}{\pos{Noun}} {\definition{type of palm tree (~2 m) that used to be cooked and eaten; also fed to pigs}}
\entry{esam}\headword{esam}{\pos{Noun}} {\definition{lemongrass}}
\entry{gamo}\headword{gamo}{\pos{Noun}} {\definition{plant type}}
\entry{gamu}\headword{gamu}{\pos{Noun}} {\definition{type of ginger with flat leaves; used as medicine for centipede bites and as bait for catching flying foxes; chew it first and the flying fox will eat it and become lethargic}}
\entry{gonagone}\headword{gonagone}{\pos{Transitive S verb}} {\definition{to heat}}
\entry{gullem suwetar}\headword{gullem suwetar}{\pos{Noun}} {\definition{type of tree that is used as medicine for snake bites and to repel snakes}}
\entry{guwaba}\headword{guwaba}{\pos{Noun}} {\definition{guava tree; water steeped with its leaves is used to wash sores}}
\entry{irwe}\headword{irwe}{\pos{Noun}} {\definition{type of cultivated tree with white flowers and juicy, red and white fruit with two seeds; used to treat cough}}
\entry{kapang}\headword{kapang}{\pos{Noun}} {\definition{Acacia}}
\entry{kapang bile}\headword{kapang bile}{\pos{Noun}} {\definition{type of medicine}}
\entry{kito}\headword{kito}{\pos{Noun}} {\definition{type of black palm; in this immature stage, the shoots eaten as medicine and used to make baskets Used to build house and mat in the bush}}
\entry{koenbäll}\headword{koenbäll}{\pos{Noun}} {\definition{type of tree that grows in the bush (especially in old gardens) with hanging green fruit and liquid used to treat sores}}
\entry{kokne}\headword{kokne}{\pos{Noun}} {\definition{type of tree that grows in the grassland with blue flowers and edible blue fruit}}
\entry{kollko}\headword{kollko}{\pos{Noun}} {\definition{breadfruit}}
\entry{komo}\headword{komo}{\pos{Noun}} {\definition{ginger}}
\entry{komotupi}\headword{komotupi}{\pos{Noun}} {\definition{long ginger}}
\entry{kutt llo}\headword{kutt llo}{\pos{Noun}} {\definition{type of tree that grows in the bush with bark that is scraped and rubbed on sores}}
\entry{kädgal}\headword{kädgal}{\pos{Noun}} {\definition{type of tree that grows in the bush}}
\entry{käpom}\headword{käpom}{\pos{Noun}} {\definition{type of big tree that grows in the bush with white flowers and edible white fruit; used as medicine for cough}}
\entry{käza bädma}\headword{käza bädma}{\pos{Noun}} {\definition{type of plant that only the crocodile clan wears when going hunting for crocodiles; also a traditional medicine}}
\entry{lläkäm}\headword{lläkäm}{\pos{Noun}} {\definition{mushroom}}
\entry{mameat}\headword{mameat}{\pos{Noun}} {\definition{papaya, pawpaw}}
\entry{mamkiel}\headword{mamkiel}{\pos{Noun}} {\definition{type of native banana}}
\entry{manggo}\headword{manggo}{\pos{Noun}} {\definition{mango tree}}
\entry{mintor}\headword{mintor}{\pos{Noun}} {\definition{type of tree with yellow flowers that bloom in June and July and a root is used as a kind of hockey stick}}
\entry{moll}\headword{moll}{\pos{Noun}} {\definition{type of tree that grows around Malam with white flowers}}
\entry{mopmop}\headword{mopmop}{\pos{Noun}} {\definition{type of tree that grows in the grassland (~3 m) with white flowers and red fruit that are eaten to treat cough}}
\entry{mätemäte}\headword{mätemäte}{\pos{Noun}} {\definition{type of introduced banana}}
\entry{nge dɨdɨr}\headword{nge dɨdɨr}{\pos{Noun}} {\definition{fully dry coconut}}
\entry{pin}\headword{pin}{\pos{Noun}} {\definition{type of tree with white or red flowers and composite fruit that birds eat}}
\entry{potkam}\headword{potkam}{\pos{Noun}} {\definition{type of cultivated tree that also grows in the savanna; treats cough and aches}}
\entry{potopoto}\headword{potopoto}{\pos{Noun}} {\definition{type of tree that grows near swamps and creeks with yellow and white flowers and fruit that falls on the ground; animals eat it}}
\entry{sele}\headword{sele}{\pos{Noun}} {\definition{chili}}
\entry{sem}\headword{sem}{\pos{Noun}} {\definition{type of tree that grows in the bush; used for making rope}}
\entry{siporo}\headword{siporo}{\pos{Noun}} {\definition{type of cultivated tree with thorns and sour yellow fruit}}
\entry{taemataema}\headword{taemataema}{\pos{Noun}} {\definition{type of tree that grows near the swamp with long yellow flowers and leaves that are used to treat fungal skin infections}}
\entry{tanteny}\headword{tanteny}{\pos{Noun}} {\definition{type of tree}}
\entry{tibekllop}\headword{tibekllop}{\pos{Noun}} {\definition{type of medicine}}
\entry{tokop}\headword{tokop}{\pos{Noun}} {\definition{type of tree that grows in the bush; used as house sticks and medicine}}
\entry{ttäbe}\headword{ttäbe}{\pos{Noun}} {\definition{a strong smelling plant whose bark, called ttäbe kollop, was traditionally worn around the neck to give fragrance and perfume smell. It was sometimes chewed and rubbed around the body and head to stop headache. The orignal purpose was also for protection from evil spirits.}}
\entry{ttän}\headword{ttän}{\pos{Noun}} {\definition{type of tree that grows in the grassland (~60 m) with yellow flowers, small green fruit, and wood used for house posts}}
\entry{ullegäll}\headword{ullegäll}{\pos{Noun}} {\definition{type of tree that grows in the grassland with white flowers, black fruits, and red nuts}}
\entry{upeupe}\headword{upeupe}{\pos{Noun}} {\definition{type of plant with a single stem and edible, tall fruit near the base of stem}}
\entry{winy}\headword{winy}{\pos{Noun}} {\definition{honey}}
\entry{wägba}\headword{wägba}{\pos{Noun}} {\definition{type of tree that grows in the bush with white flowers, bark used as medicine, and strong wood used for posts; helps make dogs' noses more sensitive}}
\entry{wällegäll}\headword{wällegäll}{\pos{Noun}} {\definition{type of tree with fruit that are black and edible when ripe, leaves used to roll cigarettes, and roots used to treat toothache or asthma}}
\entry{yarte}\headword{yarte}{\pos{Noun}} {\definition{type of tree with young wood used for house sticks}}
\end{entrylist}

\section*{2.5.8 Mental illness}
\begin{entrylist}
\entry{konkon}\headword{konkon}{\pos{Modifier}} {\definition{crazy, mad, insane, mentally ill}}
\end{entrylist}

\section*{2.6 Life}
\begin{entrylist}
\entry{giddoll}\headword{giddoll}{\pos{Noun}} {\definition{life}}
\entry{ttam}\headword{ttam}{\pos{Noun}} {\definition{life}}
\entry{ttam giddoll}\headword{ttam giddoll}{\pos{Noun}} {\definition{life}}
\end{entrylist}

\section*{2.6.1 Marriage}
\begin{entrylist}
\entry{burag}\headword{burag}{\pos{Noun}} {\definition{bride price (given to the bride's family by the groom)}}
\entry{gal}\headword{gal}{\pos{Noun}} {\definition{food offering}}
\entry{gullbe peyang}\headword{gullbe peyang}{\pos{Modifier}} {\definition{married (of a female)}}
\entry{gullbog}\headword{gullbog}{\pos{Modifier}} {\definition{married (of a female)}}
\entry{gungg}\headword{gungg}{\pos{Transitive S verb}} {\definition{to marry a widow}}
\entry{ittma}\headword{ittma}{\pos{Noun}} {\definition{husband's house}}
\entry{izig}\headword{izig}{\pos{Kinship noun}} {\definition{co-wife (another woman married to the same husband)}}
\entry{izigag}\headword{izigag}{\pos{Noun}} {\definition{polygynous marriage}}
\entry{llädäd}\headword{llädäd}{\pos{Transitive S verb}} {\definition{to marry}}
\entry{maret}\headword{maret}{\pos{Noun}} {\definition{marriage}}
\entry{mällawang}\headword{mällawang}{\pos{Modifier}} {\definition{married (of a male)}}
\entry{ngetae}\headword{ngetae}{\pos{Transitive S verb}} {\definition{to get engaged}}
\entry{nop mu}\headword{nop mu}{\pos{Noun}} {\definition{inter-tribe payment}}
\end{entrylist}

\section*{2.6.1.4 Divorce}
\begin{entrylist}
\entry{ddaebän}\headword{ddaebän}{\pos{Transitive S verb}} {\definition{to divorce}}
\entry{ttaempäg}\headword{ttaempäg}{\pos{Transitive S verb}} {\definition{to seperate, divorce}}
\end{entrylist}

\section*{2.6.2 Sexual relations}
\begin{entrylist}
\entry{eiz}\headword{eiz}{\pos{Noun}} {\definition{HIV; AIDS}}
\entry{kok to}\headword{kok to}{\pos{Noun}} {\definition{orgy}}
\entry{liglig}\headword{liglig}{\pos{Intransitive S verb}} {\definition{to have sex, copulate}}
\entry{liglig suang}\headword{liglig suang}{\pos{Modifier}} {\definition{having a lot of sex}}
\entry{liglig ttoen}\headword{liglig ttoen}{\pos{Noun}} {\definition{sexual relations}}
\end{entrylist}

\section*{2.6.3 Birth}
\begin{entrylist}
\entry{ddäma mu}\headword{ddäma mu}{\pos{Noun}} {\definition{birth payment made to the maternal uncle}}
\entry{dum}\headword{dum}{\pos{Noun}} {\definition{placenta}}
\entry{gogodd pänddäg}\headword{gogodd pänddäg}{\pos{}} {\definition{water breaking}}
\entry{komlla ttuim mängallmeny}\headword{komlla ttuim mängallmeny}{\pos{}} {\definition{two thighs weak}}
\entry{kudäd}\headword{kudäd}{\pos{Modifier}} {\definition{barren}}
\entry{käm}\headword{käm}{\pos{Noun}} {\definition{womb}}
\entry{kämang}\headword{kämang}{\pos{Modifier}} {\definition{pregnant}}
\entry{käza da nägagan}\headword{käza da nägagan}{\pos{Phrase}} {\definition{to miscarry}}
\entry{lla päpätt sämpallmeny alla}\headword{lla päpätt sämpallmeny alla}{\pos{}} {\definition{false labor}}
\entry{llɨg dum}\headword{llɨg dum}{\pos{Noun}} {\definition{placenta}}
\entry{llɨg mapät}\headword{llɨg mapät}{\pos{Noun}} {\definition{infant}}
\entry{llɨgmeny}\headword{llɨgmeny}{\pos{Modifier}} {\definition{childless}}
\entry{llɨgmeny agallo}\headword{llɨgmeny agallo}{\pos{}} {\definition{to plan for not having children}}
\entry{llɨgmeny e wätäba}\headword{llɨgmeny e wätäba}{\pos{}} {\definition{to plan for not having children}}
\entry{talme}\headword{talme}{\pos{Transitive/Intransitive A verb}} {\definition{to give birth (to)}}
\entry{zaze}\headword{zaze}{\pos{Intransitive S verb}} {\definition{to give birth, lay}}
\entry{zeg}\headword{zeg}{\pos{Intransitive S verb}} {\definition{to be born}}
\end{entrylist}

\section*{2.6.3.1 Pregnancy}
\begin{entrylist}
\entry{aulämän}\headword{aulämän}{\pos{Noun}} {\definition{conception}}
\entry{kok a nganzig}\headword{kok a nganzig}{\pos{Phrase}} {\definition{to go a month without menstruating}}
\entry{komene}\headword{komene}{\pos{Noun}} {\definition{postpartum period during which a woman warms herself by the fire}}
\entry{llɨg dum}\headword{llɨg dum}{\pos{Noun}} {\definition{womb, uterus}}
\entry{wa piye}\headword{wa piye}{\pos{Noun}} {\definition{sperm}}
\end{entrylist}

\section*{2.6.3.2 Fetus}
\begin{entrylist}
\entry{wänkäm tärpan}\headword{wänkäm tärpan}{\pos{Noun}} {\definition{umbilical cord}}
\end{entrylist}

\section*{2.6.3.4 Labor and birth pains}
\begin{entrylist}
\entry{pänddäg}\headword{pänddäg}{\pos{Intransitive S verb}} {\definition{to break water}}
\end{entrylist}

\section*{2.6.3.8 Fertile, infertile}
\begin{entrylist}
\entry{kotol}\headword{kotol}{\pos{Noun}} {\definition{traditional practice of sterilizing women after having 6–7 kids; the placenta is buried and a coconut is planted on top}}
\end{entrylist}

\section*{2.6.4 Stage of life}
\begin{entrylist}
\entry{blab}\headword{blab}{\pos{Intransitive S verb}} {\definition{to mature, reach puberty}}
\entry{kallänggag}\headword{kallänggag}{\pos{Noun}} {\definition{stage of life, when a baby is still lying down.}}
\entry{melem duag}\headword{melem duag}{\pos{Noun}} {\definition{stage of life when you're able to do all the work}}
\entry{pänayang}\headword{pänayang}{\pos{Noun}} {\definition{stage of life when a child begins to turn over}}
\entry{saem}\headword{saem}{\pos{Verb}} {\definition{to babble}}
\entry{sisor llɨg}\headword{sisor llɨg}{\pos{Noun}} {\definition{newborn, infant}}
\end{entrylist}

\section*{2.6.4.1 Baby}
\begin{entrylist}
\entry{bebi}\headword{bebi}{\pos{Noun}} {\definition{baby}}
\entry{beibi}\headword{beibi}{\pos{Noun}} {\definition{baby}}
\entry{llɨg mapät}\headword{llɨg mapät}{\pos{Noun}} {\definition{infant}}
\entry{mama}\headword{mama}{\pos{Noun}} {\definition{food (baby talk word)}}
\entry{susu}\headword{susu}{\pos{Noun}} {\definition{breastmilk}}
\entry{tata}\headword{tata}{\pos{Noun}} {\definition{meat}}
\end{entrylist}

\section*{2.6.4.1.1 Care for a baby}
\begin{entrylist}
\entry{anu}\headword{anu}{\pos{Verb}} {\definition{sleep (command given to babies)}}
\entry{susu}\headword{susu}{\pos{Noun}} {\definition{breastmilk}}
\end{entrylist}

\section*{2.6.4.2 Child}
\begin{entrylist}
\entry{ergodag}\headword{ergodag}{\pos{Noun}} {\definition{toddler}}
\entry{lla llɨg}\headword{lla llɨg}{\pos{Noun}} {\definition{boy}}
\entry{llɨg}\headword{llɨg}{\pos{Noun}} {\definition{boy}}
\entry{män}\headword{män}{\pos{Noun}} {\definition{girl}}
\end{entrylist}

\section*{2.6.4.2.1 Rear a child}
\begin{entrylist}
\entry{mändmänd}\headword{mändmänd}{\pos{Transitive S verb}} {\definition{to raise, rear}}
\entry{ulle}\headword{ulle}{\pos{Transitive A verb}} {\definition{to raise, rear}}
\end{entrylist}

\section*{2.6.4.4 Adult}
\begin{entrylist}
\entry{batt}\headword{batt}{\pos{Modifier}} {\definition{adult, mature}}
\end{entrylist}

\section*{2.6.4.5 Old person}
\begin{entrylist}
\entry{ause}\headword{ause}{\pos{Noun}} {\definition{old woman}}
\entry{auseause}\headword{auseause}{\pos{Modifier}} {\definition{nonsingular form of ause}}
\entry{llamda}\headword{llamda}{\pos{Noun}} {\definition{old man}}
\entry{llamäg}\headword{llamäg}{\pos{Noun}} {\definition{old man}}
\entry{masar}\headword{masar}{\pos{Modifier}} {\definition{ancestral, old (of one's grandparents' time)}}
\entry{mosenmosen}\headword{mosenmosen}{\pos{Noun}} {\definition{elders, seniors}}
\entry{mällause}\headword{mällause}{\pos{Noun}} {\definition{old woman}}
\end{entrylist}

\section*{2.6.4.6 Grow, get bigger}
\begin{entrylist}
\entry{blab}\headword{blab}{\pos{Intransitive S verb}} {\definition{to mature, reach puberty}}
\entry{bämblläd}\headword{bämblläd}{\pos{Transitive S verb}} {\definition{to expand, rise}}
\entry{dirom gäz}\headword{dirom gäz}{\pos{Transitive S verb}} {\definition{to menstruate for the first time}}
\entry{giddollag}\headword{giddollag}{\pos{Modifier}} {\definition{menstruating, on one's period}}
\entry{moepang}\headword{moepang}{\pos{Modifier}} {\definition{matured}}
\entry{päddab}\headword{päddab}{\pos{Intransitive S verb}} {\definition{to grow}}
\entry{päddpädd}\headword{päddpädd}{\pos{Intransitive S verb}} {\definition{to grow}}
\entry{pämpllätt}\headword{pämpllätt}{\pos{Intransitive S verb}} {\definition{to become healthy; grow}}
\entry{pänyanz}\headword{pänyanz}{\pos{Verb}} {\definition{to grow}}
\entry{pɨddab}\headword{pɨddab}{\pos{Intransitive S verb}} {\definition{to grow}}
\entry{ulle}\headword{ulle}{\pos{Intransitive A verb}} {\definition{to become big, grow up}}
\end{entrylist}

\section*{2.6.4.7 Initiation}
\begin{entrylist}
\entry{kamo}\headword{kamo}{\pos{Noun}} {\definition{reciprocal term for the young man and the older man that takes him through initiation}}
\entry{konymad}\headword{konymad}{\pos{Noun}} {\definition{man who steals a woman's things during her initiation ceremony}}
\entry{kädkäd}\headword{kädkäd}{\pos{Noun}} {\definition{type of initiation that one must do before you get something from someone}}
\entry{mulmul}\headword{mulmul}{\pos{Noun}} {\definition{rite of passage}}
\entry{waglla}\headword{waglla}{\pos{Noun}} {\definition{bullroarer}}
\end{entrylist}

\section*{2.6.4.8 Peer group}
\begin{entrylist}
\entry{lla bombllo}\headword{lla bombllo}{\pos{Noun}} {\definition{generation}}
\end{entrylist}

\section*{2.6.5.1 Man}
\begin{entrylist}
\entry{lla}\headword{lla}{\pos{Noun}} {\definition{man, male}}
\entry{llamda}\headword{llamda}{\pos{Noun}} {\definition{old man}}
\entry{llamäg}\headword{llamäg}{\pos{Noun}} {\definition{old man}}
\entry{mista}\headword{mista}{\pos{Noun}} {\definition{mister}}
\end{entrylist}

\section*{2.6.5.2 Woman}
\begin{entrylist}
\entry{ause}\headword{ause}{\pos{Noun}} {\definition{old woman}}
\entry{dirom gäz}\headword{dirom gäz}{\pos{Transitive S verb}} {\definition{to menstruate for the first time}}
\entry{giddollag}\headword{giddollag}{\pos{Modifier}} {\definition{menstruating, on one's period}}
\entry{mälla}\headword{mälla}{\pos{Noun}} {\definition{woman, female}}
\entry{mällause}\headword{mällause}{\pos{Noun}} {\definition{old woman}}
\entry{män duwar}\headword{män duwar}{\pos{Noun}} {\definition{young girl}}
\end{entrylist}

\section*{2.6.6 Die}
\begin{entrylist}
\entry{anyke}\headword{anyke}{\pos{Noun}} {\definition{spirit}}
\entry{auma}\headword{auma}{\pos{Noun}} {\definition{grave}}
\entry{dodro}\headword{dodro}{\pos{Intransitive S verb}} {\definition{to die}}
\entry{gagäll}\headword{gagäll}{\pos{Modifier}} {\definition{dead}}
\entry{idd}\headword{idd}{\pos{Noun}} {\definition{ghost}}
\entry{idd ma}\headword{idd ma}{\pos{Noun}} {\definition{afterlife}}
\entry{kuddäll}\headword{kuddäll}{\pos{Noun}} {\definition{death}}
\entry{kuddäll}\headword{kuddäll}{\pos{Intransitive A verb}} {\definition{to die}}
\entry{plengg}\headword{plengg}{\pos{Intransitive S verb}} {\definition{to die}}
\entry{täräb}\headword{täräb}{\pos{Noun}} {\definition{a dead person's belongings (put outside at a funeral until the feast ends)}}
\end{entrylist}

\section*{2.6.6.1 Kill}
\begin{entrylist}
\entry{ddungg}\headword{ddungg}{\pos{Transitive S verb}} {\definition{to decapitate}}
\entry{ddänggaddängge}\headword{ddänggaddängge}{\pos{Transitive S verb}} {\definition{to crucify}}
\entry{gäz}\headword{gäz}{\pos{Transitive S verb}} {\definition{to kill}}
\entry{kallɨngg}\headword{kallɨngg}{\pos{Transitive S verb}} {\definition{to kill}}
\entry{lla gugu}\headword{lla gugu}{\pos{Noun}} {\definition{restoring peace after a murder by trading a young girl in the victim's place}}
\entry{tärpam}\headword{tärpam}{\pos{Transitive S verb}} {\definition{to crucify}}
\end{entrylist}

\section*{2.6.6.3 Funeral}
\begin{entrylist}
\entry{au}\headword{au}{\pos{Transitive A verb}} {\definition{to bury}}
\entry{llollo}\headword{llollo}{\pos{Verb}} {\definition{to pay respects}}
\entry{täräb}\headword{täräb}{\pos{Noun}} {\definition{a dead person's belongings (put outside at a funeral until the feast ends)}}
\entry{täräb ma}\headword{täräb ma}{\pos{Noun}} {\definition{funeral home}}
\entry{tärämpmeny}\headword{tärämpmeny}{\pos{Noun}} {\definition{sleeping with a dead person's belongings}}
\end{entrylist}

\section*{2.6.6.4 Mourn}
\begin{entrylist}
\entry{llollo}\headword{llollo}{\pos{Verb}} {\definition{to pay respects}}
\entry{ngänaurur}\headword{ngänaurur}{\pos{Intransitive A verb}} {\definition{to mourn}}
\end{entrylist}

\section*{2.6.6.5 Bury}
\begin{entrylist}
\entry{au}\headword{au}{\pos{Noun}} {\definition{burial}}
\entry{au}\headword{au}{\pos{Transitive A verb}} {\definition{to bury}}
\entry{wɨndwɨnd}\headword{wɨndwɨnd}{\pos{Transitive S verb}} {\definition{to cover, bury}}
\end{entrylist}

\section*{2.6.6.8 Life after death}
\begin{entrylist}
\entry{yu ttängäm}\headword{yu ttängäm}{\pos{Noun}} {\definition{hell}}
\end{entrylist}

\section*{3.1 Soul, spirit}
\begin{entrylist}
\entry{anyke}\headword{anyke}{\pos{Noun}} {\definition{spirit}}
\entry{idd}\headword{idd}{\pos{Noun}} {\definition{ghost}}
\entry{kollmos ttam}\headword{kollmos ttam}{\pos{Noun}} {\definition{soul}}
\end{entrylist}

\section*{3.1.1 Personality}
\begin{entrylist}
\entry{bällämbällämmeny}\headword{bällämbällämmeny}{\pos{Modifier}} {\definition{thoughtless, apathetic, uncaring}}
\end{entrylist}

\section*{3.2 Think}
\begin{entrylist}
\entry{bällämbäll}\headword{bällämbäll}{\pos{Noun}} {\definition{thought}}
\entry{ngonongg}\headword{ngonongg}{\pos{Intransitive S verb}} {\definition{to think}}
\entry{näkäpngon}\headword{näkäpngon}{\pos{Noun}} {\definition{thought, knowledge}}
\end{entrylist}

\section*{3.2.1 Mind}
\begin{entrylist}
\entry{kame}\headword{kame}{\pos{Noun}} {\definition{ignorance, non-knowing; incomprehension, non-understanding}}
\entry{näkäp}\headword{näkäp}{\pos{Noun}} {\definition{mind, mindset, consciousness; thoughts}}
\entry{umllang}\headword{umllang}{\pos{Noun}} {\definition{knowledge, knowing, awareness}}
\end{entrylist}

\section*{3.2.1.1 Think about}
\begin{entrylist}
\entry{bälämbäl}\headword{bälämbäl}{\pos{Transitive S verb}} {\definition{to remember, think of}}
\entry{ngonongg}\headword{ngonongg}{\pos{Intransitive S verb}} {\definition{to think about, think of, recall, remember}}
\entry{ngänongg}\headword{ngänongg}{\pos{Intransitive S verb}} {\definition{to think about, think of}}
\end{entrylist}

\section*{3.2.1.3 Intelligent}
\begin{entrylist}
\entry{bun pallkamatt}\headword{bun pallkamatt}{\pos{Modifier}} {\definition{smart}}
\entry{kinekineang}\headword{kinekineang}{\pos{Modifier}} {\definition{smart}}
\entry{näkäpang}\headword{näkäpang}{\pos{Modifier}} {\definition{smart}}
\entry{umull}\headword{umull}{\pos{Noun}} {\definition{wisdom}}
\end{entrylist}

\section*{3.2.1.4 Stupid}
\begin{entrylist}
\entry{konkon}\headword{konkon}{\pos{Modifier}} {\definition{stupid, ignorant, foolish}}
\entry{näkäpmeny}\headword{näkäpmeny}{\pos{Modifier}} {\definition{foolish}}
\end{entrylist}

\section*{3.2.2 Learn}
\begin{entrylist}
\entry{umllang}\headword{umllang}{\pos{Intransitive A verb}} {\definition{to know; come to know, learn}}
\end{entrylist}

\section*{3.2.2.1 Study}
\begin{entrylist}
\entry{skulang}\headword{skulang}{\pos{Noun}} {\definition{student}}
\end{entrylist}

\section*{3.2.2.2 Check}
\begin{entrylist}
\entry{ngätae}\headword{ngätae}{\pos{Transitive S verb}} {\definition{to check}}
\entry{pänaemeny}\headword{pänaemeny}{\pos{Transitive S verb}} {\definition{to check}}
\end{entrylist}

\section*{3.2.2.4 Guess}
\begin{entrylist}
\entry{paengg}\headword{paengg}{\pos{Transitive S verb}} {\definition{to guess}}
\end{entrylist}

\section*{3.2.2.5 Solve}
\begin{entrylist}
\entry{särämbae}\headword{särämbae}{\pos{Transitive S verb}} {\definition{to fix, solve, resolve}}
\end{entrylist}

\section*{3.2.3 Know}
\begin{entrylist}
\entry{näkäpngon}\headword{näkäpngon}{\pos{Noun}} {\definition{thought, knowledge}}
\end{entrylist}

\section*{3.2.3.1 Known, unknown}
\begin{entrylist}
\entry{kame}\headword{kame}{\pos{Noun}} {\definition{ignorance, non-knowing; incomprehension, non-understanding}}
\entry{umllang}\headword{umllang}{\pos{Noun}} {\definition{knowledge, knowing, awareness}}
\end{entrylist}

\section*{3.2.3.3 Secret}
\begin{entrylist}
\entry{gwell}\headword{gwell}{\pos{Modifier}} {\definition{secret}}
\entry{känyär}\headword{känyär}{\pos{Modifier}} {\definition{secret, secretly}}
\entry{känyär eka}\headword{känyär eka}{\pos{Noun}} {\definition{secret}}
\entry{känyär ttoen}\headword{känyär ttoen}{\pos{Noun}} {\definition{secret}}
\entry{su}\headword{su}{\pos{Modifier}} {\definition{secret}}
\end{entrylist}

\section*{3.2.4 Understand}
\begin{entrylist}
\entry{ngänam}\headword{ngänam}{\pos{Transitive S verb}} {\definition{to understand, recognize}}
\end{entrylist}

\section*{3.2.4.1 Misunderstand}
\begin{entrylist}
\entry{kame}\headword{kame}{\pos{Noun}} {\definition{ignorance, non-knowing; incomprehension, non-understanding}}
\end{entrylist}

\section*{3.2.5.1 Believe}
\begin{entrylist}
\entry{imomdae}\headword{imomdae}{\pos{Transitive A verb}} {\definition{to believe}}
\end{entrylist}

\section*{3.2.5.4.1 Disagree}
\begin{entrylist}
\entry{eka laem}\headword{eka laem}{\pos{Intransitive S verb}} {\definition{to argue}}
\entry{käkllätt}\headword{käkllätt}{\pos{Transitive S verb}} {\definition{to fight, argue}}
\entry{nyärmeny}\headword{nyärmeny}{\pos{Noun}} {\definition{little argument between children}}
\end{entrylist}

\section*{3.2.5.8 Change your mind}
\begin{entrylist}
\entry{näkäp de pänae}\headword{näkäp de pänae}{\pos{Phrase}} {\definition{to change one's mind}}
\end{entrylist}

\section*{3.2.5.9 Approve of something}
\begin{entrylist}
\entry{daudau}\headword{daudau}{\pos{Intransitive A verb}} {\definition{to nod}}
\end{entrylist}

\section*{3.2.6 Remember}
\begin{entrylist}
\entry{bälämbäl}\headword{bälämbäl}{\pos{Transitive S verb}} {\definition{to miss, long for}}
\entry{bälämbäl}\headword{bälämbäl}{\pos{Transitive S verb}} {\definition{to remember, think of}}
\entry{gämoe}\headword{gämoe}{\pos{Transitive S verb}} {\definition{to miss, feel longing for}}
\entry{mitmit}\headword{mitmit}{\pos{Transitive A verb}} {\definition{to miss, long for}}
\entry{ngonongg}\headword{ngonongg}{\pos{Intransitive S verb}} {\definition{to think about, think of, recall, remember}}
\end{entrylist}

\section*{3.2.6.1 Forget}
\begin{entrylist}
\entry{ngllongg}\headword{ngllongg}{\pos{Transitive S verb}} {\definition{to forget}}
\entry{wangam}\headword{wangam}{\pos{Transitive S verb}} {\definition{to forget}}
\entry{yänkllollang}\headword{yänkllollang}{\pos{Modifier}} {\definition{forgetful}}
\end{entrylist}

\section*{3.2.6.2 Recognize}
\begin{entrylist}
\entry{ngänam}\headword{ngänam}{\pos{Transitive S verb}} {\definition{to understand, recognize}}
\end{entrylist}

\section*{3.2.7.3 Predict}
\begin{entrylist}
\entry{ballɨngg}\headword{ballɨngg}{\pos{Transitive S verb}} {\definition{to predict}}
\end{entrylist}

\section*{3.3 Want}
\begin{entrylist}
\entry{moko}\headword{moko}{\pos{Noun}} {\definition{desire, want; love}}
\end{entrylist}

\section*{3.3.1.2 Choose}
\begin{entrylist}
\entry{llätmäll}\headword{llätmäll}{\pos{Transitive S verb}} {\definition{to choose, select}}
\end{entrylist}

\section*{3.3.2.1 Agree to do something}
\begin{entrylist}
\entry{adako}\headword{adako}{\pos{TAM particle}} {\definition{~ ako}}
\end{entrylist}

\section*{3.3.2.4 Willing}
\begin{entrylist}
\entry{tälpe}\headword{tälpe}{\pos{Intransitive S verb}} {\definition{to volunteer}}
\end{entrylist}

\section*{3.3.3 Influence}
\begin{entrylist}
\entry{dändärek}\headword{dändärek}{\pos{Transitive S verb}} {\definition{to control, influence, rule, govern}}
\end{entrylist}

\section*{3.3.3.2 Advise}
\begin{entrylist}
\entry{ngallmeny}\headword{ngallmeny}{\pos{Transitive S verb}} {\definition{to advise}}
\end{entrylist}

\section*{3.3.3.6 Control}
\begin{entrylist}
\entry{dändärek}\headword{dändärek}{\pos{Transitive S verb}} {\definition{to control, influence, rule, govern}}
\end{entrylist}

\section*{3.3.3.7 Warn}
\begin{entrylist}
\entry{aläm}\headword{aläm}{\pos{Transitive A verb}} {\definition{to warn, caution}}
\end{entrylist}

\section*{3.3.3.8 Threaten}
\begin{entrylist}
\entry{mättae}\headword{mättae}{\pos{Transitive S verb}} {\definition{to threaten, raise fists}}
\end{entrylist}

\section*{3.3.4.2 Refuse permission}
\begin{entrylist}
\entry{aeb}\headword{aeb}{\pos{Transitive verb}} {\definition{to deny}}
\entry{nyärpae}\headword{nyärpae}{\pos{Intransitive S verb}} {\definition{to refuse (someone in the dative)}}
\end{entrylist}

\section*{3.3.5.1 Accept}
\begin{entrylist}
\entry{malam}\headword{malam}{\pos{Transitive S verb}} {\definition{to accept}}
\end{entrylist}

\section*{3.3.5.2 Reject}
\begin{entrylist}
\entry{yämbäg}\headword{yämbäg}{\pos{Transitive S verb}} {\definition{to disown, repudiate}}
\end{entrylist}

\section*{3.4.1 Feel good}
\begin{entrylist}
\entry{sämongg}\headword{sämongg}{\pos{Intransitive S verb}} {\definition{to feel}}
\end{entrylist}

\section*{3.4.1.1 Like, love}
\begin{entrylist}
\entry{moko}\headword{moko}{\pos{Noun}} {\definition{desire, want; love}}
\end{entrylist}

\section*{3.4.1.1.1 Enjoy doing something}
\begin{entrylist}
\entry{käm}\headword{käm}{\pos{Noun}} {\definition{love, enjoyment}}
\end{entrylist}

\section*{3.4.1.2 Happy}
\begin{entrylist}
\entry{kili}\headword{kili}{\pos{Modifier}} {\definition{happy}}
\entry{kiliang}\headword{kiliang}{\pos{Modifier}} {\definition{happy}}
\end{entrylist}

\section*{3.4.1.2.2 Calm}
\begin{entrylist}
\entry{mikuttmeny}\headword{mikuttmeny}{\pos{Modifier}} {\definition{calm}}
\entry{mätaru}\headword{mätaru}{\pos{Modifier}} {\definition{calm, peaceful, quiet}}
\entry{wi}\headword{wi}{\pos{Intransitive S verb}} {\definition{to settle}}
\end{entrylist}

\section*{3.4.1.3 Surprise}
\begin{entrylist}
\entry{läpon}\headword{läpon}{\pos{Intransitive S verb}} {\definition{to be amazed, be in awe}}
\entry{poper kwingg}\headword{poper kwingg}{\pos{Transitive S verb}} {\definition{to surprise}}
\end{entrylist}

\section*{3.4.1.4 Interested}
\begin{entrylist}
\entry{käklläp}\headword{käklläp}{\pos{Transitive S verb}} {\definition{to be amused}}
\end{entrylist}

\section*{3.4.1.4.4 Attract}
\begin{entrylist}
\entry{manglle}\headword{manglle}{\pos{Transitive A verb}} {\definition{to lure}}
\end{entrylist}

\section*{3.4.2 Feel bad}
\begin{entrylist}
\entry{sämongg}\headword{sämongg}{\pos{Intransitive S verb}} {\definition{to feel}}
\end{entrylist}

\section*{3.4.2.1 Sad}
\begin{entrylist}
\entry{kilimeny}\headword{kilimeny}{\pos{Modifier}} {\definition{unhappy, sad, annoyed}}
\entry{kote ttäkam}\headword{kote ttäkam}{\pos{Intransitive S verb}} {\definition{to hang one's head}}
\entry{llama}\headword{llama}{\pos{Modifier}} {\definition{unsatisfied}}
\end{entrylist}

\section*{3.4.2.1.1 Dislike}
\begin{entrylist}
\entry{llowam}\headword{llowam}{\pos{Noun}} {\definition{disdain, hate}}
\entry{llowamang}\headword{llowamang}{\pos{Modifier}} {\definition{unpreferable, unpleasant, tiresome (to someone in the dative)}}
\end{entrylist}

\section*{3.4.2.1.2 Hate, detest}
\begin{entrylist}
\entry{llowam}\headword{llowam}{\pos{Noun}} {\definition{disdain, hate}}
\entry{llowamang}\headword{llowamang}{\pos{Modifier}} {\definition{unpreferable, unpleasant, tiresome (to someone in the dative)}}
\entry{mikuttang}\headword{mikuttang}{\pos{Transitive A verb}} {\definition{to hate}}
\end{entrylist}

\section*{3.4.2.1.5 Lonely}
\begin{entrylist}
\entry{bälämbäl}\headword{bälämbäl}{\pos{Transitive S verb}} {\definition{to miss, long for}}
\entry{gämoe}\headword{gämoe}{\pos{Transitive S verb}} {\definition{to miss, feel longing for}}
\entry{mitmit}\headword{mitmit}{\pos{Transitive A verb}} {\definition{to miss, long for}}
\end{entrylist}

\section*{3.4.2.1.6 Upset}
\begin{entrylist}
\entry{konkon}\headword{konkon}{\pos{Modifier}} {\definition{crazy, mad, insane, mentally ill}}
\entry{llowamang}\headword{llowamang}{\pos{Modifier}} {\definition{upset, annoyed}}
\end{entrylist}

\section*{3.4.2.1.7 Shock}
\begin{entrylist}
\entry{läpon}\headword{läpon}{\pos{Intransitive S verb}} {\definition{to be amazed, be in awe}}
\entry{plenz}\headword{plenz}{\pos{Intransitive S verb}} {\definition{to be in shock}}
\end{entrylist}

\section*{3.4.2.2 Sorry}
\begin{entrylist}
\entry{kandärmang}\headword{kandärmang}{\pos{Modifier}} {\definition{sorry, apologetic, regretful}}
\entry{sari}\headword{sari}{\pos{Interjection}} {\definition{sorry}}
\end{entrylist}

\section*{3.4.2.2.1 Ashamed}
\begin{entrylist}
\entry{lel}\headword{lel}{\pos{Noun}} {\definition{shame}}
\entry{lelang}\headword{lelang}{\pos{Modifier}} {\definition{shameful}}
\end{entrylist}

\section*{3.4.2.3 Angry}
\begin{entrylist}
\entry{ikop kutt nängazmeny}\headword{ikop kutt nängazmeny}{\pos{Phrase}} {\definition{stare at someone intently when angry with them.}}
\entry{mikutt}\headword{mikutt}{\pos{Modifier}} {\definition{angry, mad}}
\entry{mikuttang}\headword{mikuttang}{\pos{Modifier}} {\definition{angry, short-tempered}}
\entry{tum}\headword{tum}{\pos{Modifier}} {\definition{angry}}
\end{entrylist}

\section*{3.4.2.3.1 Annoyed}
\begin{entrylist}
\entry{llowamang}\headword{llowamang}{\pos{Modifier}} {\definition{upset, annoyed}}
\end{entrylist}

\section*{3.4.2.4 Afraid}
\begin{entrylist}
\entry{arle}\headword{arle}{\pos{Noun}} {\definition{scream}}
\entry{gäglib}\headword{gäglib}{\pos{Transitive S verb}} {\definition{to chase, scare away/off (wild animals)}}
\entry{lel}\headword{lel}{\pos{Noun}} {\definition{fear}}
\entry{pameny}\headword{pameny}{\pos{Intransitive S verb}} {\definition{to scream}}
\entry{poper}\headword{poper}{\pos{Modifier}} {\definition{afraid, startled, scared}}
\entry{tikopmeny}\headword{tikopmeny}{\pos{Modifier}} {\definition{startled, scared}}
\end{entrylist}

\section*{3.4.2.4.1 Worried}
\begin{entrylist}
\entry{ddänddäm}\headword{ddänddäm}{\pos{Intransitive S verb}} {\definition{to worry}}
\entry{näkäp ttomoe}\headword{näkäp ttomoe}{\pos{Intransitive S verb}} {\definition{to worry}}
\end{entrylist}

\section*{3.4.2.5 Confused}
\begin{entrylist}
\entry{bendoe}\headword{bendoe}{\pos{Transitive S verb}} {\definition{to confuse, mix up}}
\entry{ttanttem}\headword{ttanttem}{\pos{Intransitive S verb}} {\definition{to be confused}}
\end{entrylist}

\section*{3.5 Communication}
\begin{entrylist}
\entry{ngange}\headword{ngange}{\pos{Transitive S verb}} {\definition{to communicate, deliver a message}}
\end{entrylist}

\section*{3.5.1 Say}
\begin{entrylist}
\entry{eka}\headword{eka}{\pos{Transitive/Intransitive A verb}} {\definition{to say, speak, talk, tell}}
\entry{lläntäg}\headword{lläntäg}{\pos{Transitive S verb}} {\definition{to tell}}
\entry{llɨtɨt}\headword{llɨtɨt}{\pos{Transitive S verb}} {\definition{to tell, report, say}}
\entry{panypeny}\headword{panypeny}{\pos{Transitive S verb}} {\definition{to speak, talk, say}}
\entry{umllang}\headword{umllang}{\pos{Transitive A verb}} {\definition{to tell, inform}}
\end{entrylist}

\section*{3.5.1.1 Voice}
\begin{entrylist}
\entry{buddog}\headword{buddog}{\pos{Modifier}} {\definition{deep (of a voice)}}
\entry{inkäm}\headword{inkäm}{\pos{Noun}} {\definition{voice}}
\entry{kälsrong}\headword{kälsrong}{\pos{Modifier}} {\definition{nice singing voice}}
\entry{llällamang}\headword{llällamang}{\pos{Modifier}} {\definition{noisy}}
\entry{zowag}\headword{zowag}{\pos{Modifier}} {\definition{hoarse}}
\end{entrylist}

\section*{3.5.1.1.1 Shout}
\begin{entrylist}
\entry{allka}\headword{allka}{\pos{Transitive/Intransitive A verb}} {\definition{to shout (at)}}
\entry{arle}\headword{arle}{\pos{Noun}} {\definition{scream}}
\entry{erany}\headword{erany}{\pos{Noun}} {\definition{scream}}
\entry{olle}\headword{olle}{\pos{Intransitive A verb}} {\definition{to shout, yell, call out}}
\entry{pameny}\headword{pameny}{\pos{Intransitive S verb}} {\definition{to scream}}
\entry{uk}\headword{uk}{\pos{Intransitive A verb}} {\definition{to shout, yell, cry out}}
\entry{waeyo}\headword{waeyo}{\pos{Interjection}} {\definition{shout made in distress}}
\end{entrylist}

\section*{3.5.1.1.2 Speak quietly}
\begin{entrylist}
\entry{nineyem}\headword{nineyem}{\pos{Noun}} {\definition{whisper}}
\entry{pänyik}\headword{pänyik}{\pos{Intransitive A verb}} {\definition{to whisper}}
\end{entrylist}

\section*{3.5.1.1.8 Speak poorly}
\begin{entrylist}
\entry{ddollombog}\headword{ddollombog}{\pos{Transitive S verb}} {\definition{to misspeak, speak with mistakes}}
\end{entrylist}

\section*{3.5.1.2 Talk about a subject}
\begin{entrylist}
\entry{eka tameny}\headword{eka tameny}{\pos{Intransitive S verb}} {\definition{to discuss, converse}}
\entry{panypeny}\headword{panypeny}{\pos{Transitive S verb}} {\definition{to speak, talk, say}}
\entry{tameny}\headword{tameny}{\pos{Transitive S verb}} {\definition{to dicuss, converse}}
\end{entrylist}

\section*{3.5.1.2.1 Announce}
\begin{entrylist}
\entry{kawa}\headword{kawa}{\pos{Noun}} {\definition{announcement, notice; plea}}
\end{entrylist}

\section*{3.5.1.2.3 Explain}
\begin{entrylist}
\entry{ddel}\headword{ddel}{\pos{Transitive S verb}} {\definition{to explain}}
\entry{llanded}\headword{llanded}{\pos{Transitive S verb}} {\definition{to clarify, make clear, explain}}
\end{entrylist}

\section*{3.5.1.2.6 Repeat}
\begin{entrylist}
\entry{ngäsangngäsang}\headword{ngäsangngäsang}{\pos{Adverb}} {\definition{repeatedly}}
\entry{täli}\headword{täli}{\pos{Transitive S verb}} {\definition{to repeat}}
\end{entrylist}

\section*{3.5.1.3 True}
\begin{entrylist}
\entry{imomdae}\headword{imomdae}{\pos{Noun}} {\definition{truth}}
\end{entrylist}

\section*{3.5.1.3.2 Tell a lie}
\begin{entrylist}
\entry{kuki}\headword{kuki}{\pos{Modifier}} {\definition{false, deceptive}}
\entry{kuki}\headword{kuki}{\pos{Transitive A verb}} {\definition{to deceive, trick, lie to}}
\entry{kuki eka}\headword{kuki eka}{\pos{Noun}} {\definition{lie}}
\end{entrylist}

\section*{3.5.1.3.5 Real}
\begin{entrylist}
\entry{imomdae}\headword{imomdae}{\pos{Modifier}} {\definition{true, real, actual}}
\end{entrylist}

\section*{3.5.1.4 Speak with others}
\begin{entrylist}
\entry{indurgoeg}\headword{indurgoeg}{\pos{Transitive S verb}} {\definition{to yell}}
\end{entrylist}

\section*{3.5.1.4.1 Call}
\begin{entrylist}
\entry{olle}\headword{olle}{\pos{Transitive A verb}} {\definition{to call (over), summon}}
\entry{trungg}\headword{trungg}{\pos{Transitive S verb}} {\definition{to invite, call over, summon}}
\entry{ttam}\headword{ttam}{\pos{Transitive S verb}} {\definition{to call, name}}
\end{entrylist}

\section*{3.5.1.4.3 Greet}
\begin{entrylist}
\entry{kilikili}\headword{kilikili}{\pos{Transitive A verb}} {\definition{to greet}}
\entry{mer ag}\headword{mer ag}{\pos{Phrase}} {\definition{good morning}}
\entry{sebor}\headword{sebor}{\pos{Transitive A verb}} {\definition{to greet}}
\entry{ttaempäg}\headword{ttaempäg}{\pos{Transitive S verb}} {\definition{to shake hands with}}
\end{entrylist}

\section*{3.5.1.5 Ask}
\begin{entrylist}
\entry{ngonoe}\headword{ngonoe}{\pos{Transitive S verb}} {\definition{to ask}}
\entry{waswes}\headword{waswes}{\pos{Transitive S verb}} {\definition{to ask, beg}}
\end{entrylist}

\section*{3.5.1.5.1 Answer}
\begin{entrylist}
\entry{eka mu}\headword{eka mu}{\pos{Noun}} {\definition{answer, reply}}
\entry{ngämingg}\headword{ngämingg}{\pos{Intransitive S verb}} {\definition{to answer}}
\end{entrylist}

\section*{3.5.1.7 Praise}
\begin{entrylist}
\entry{kili}\headword{kili}{\pos{Transitive A verb}} {\definition{to praise}}
\end{entrylist}

\section*{3.5.1.7.1 Thank}
\begin{entrylist}
\entry{eso}\headword{eso}{\pos{Interjection}} {\definition{thank you}}
\end{entrylist}

\section*{3.5.1.8.1 Blame}
\begin{entrylist}
\entry{malmal}\headword{malmal}{\pos{Transitive S verb}} {\definition{to accuse, blame}}
\entry{mitamitang}\headword{mitamitang}{\pos{Transitive A verb}} {\definition{to blame}}
\end{entrylist}

\section*{3.5.1.8.2 Insult}
\begin{entrylist}
\entry{inggoemeny}\headword{inggoemeny}{\pos{Intransitive S verb}} {\definition{to blaspheme; insult, mock}}
\end{entrylist}

\section*{3.5.1.8.3 Mock}
\begin{entrylist}
\entry{inggoemeny}\headword{inggoemeny}{\pos{Intransitive S verb}} {\definition{to blaspheme; insult, mock}}
\end{entrylist}

\section*{3.5.1.8.4 Gossip}
\begin{entrylist}
\entry{lulu}\headword{lulu}{\pos{Transitive S verb}} {\definition{to gossip about, tell rumors about}}
\entry{tulgoe}\headword{tulgoe}{\pos{Transitive S verb}} {\definition{to gossip about}}
\end{entrylist}

\section*{3.5.1.9 Promise}
\begin{entrylist}
\entry{tab}\headword{tab}{\pos{Noun}} {\definition{promise, oath; engagement}}
\end{entrylist}

\section*{3.5.2.2 News, message}
\begin{entrylist}
\entry{eka}\headword{eka}{\pos{Noun}} {\definition{word, message, news}}
\end{entrylist}

\section*{3.5.2.4 Admit}
\begin{entrylist}
\entry{ttam}\headword{ttam}{\pos{Transitive S verb}} {\definition{to confess}}
\end{entrylist}

\section*{3.5.3 Language}
\begin{entrylist}
\entry{eka}\headword{eka}{\pos{Noun}} {\definition{language}}
\entry{panypeny}\headword{panypeny}{\pos{Transitive S verb}} {\definition{to speak, talk, say}}
\entry{pänae}\headword{pänae}{\pos{Transitive S verb}} {\definition{to translate}}
\end{entrylist}

\section*{3.5.3.1 Word}
\begin{entrylist}
\entry{eka}\headword{eka}{\pos{Noun}} {\definition{word, message, news}}
\end{entrylist}

\section*{3.5.4 Story}
\begin{entrylist}
\entry{eka}\headword{eka}{\pos{Noun}} {\definition{story}}
\entry{llɨtɨt}\headword{llɨtɨt}{\pos{Transitive S verb}} {\definition{to tell, report, say}}
\entry{stori}\headword{stori}{\pos{Noun}} {\definition{story}}
\end{entrylist}

\section*{3.5.4.1 Fable, myth}
\begin{entrylist}
\entry{pepeb}\headword{pepeb}{\pos{Noun}} {\definition{folktale, legend}}
\end{entrylist}

\section*{3.5.4.6 Verbal tradition}
\begin{entrylist}
\entry{Lamlam}\headword{Lamlam}{\pos{Proper noun}} {\definition{name of a female ancestor (sister of Moli)}}
\entry{inggoemeny}\headword{inggoemeny}{\pos{Intransitive S verb}} {\definition{to joke}}
\entry{mabun}\headword{mabun}{\pos{Modifier}} {\definition{sacred}}
\entry{pepeb}\headword{pepeb}{\pos{Noun}} {\definition{folktale, legend}}
\entry{rokaroka}\headword{rokaroka}{\pos{Noun}} {\definition{riddle}}
\entry{rokaroka}\headword{rokaroka}{\pos{Noun}} {\definition{fable}}
\entry{ttoen}\headword{ttoen}{\pos{Noun}} {\definition{story}}
\end{entrylist}

\section*{3.5.5.1 Obscenity}
\begin{entrylist}
\entry{inggoemeny}\headword{inggoemeny}{\pos{Intransitive S verb}} {\definition{to blaspheme; insult, mock}}
\end{entrylist}

\section*{3.5.6 Sign, symbol}
\begin{entrylist}
\entry{ikop sära}\headword{ikop sära}{\pos{Noun}} {\definition{wink; eyebrow raise}}
\entry{saen}\headword{saen}{\pos{Noun}} {\definition{sign}}
\entry{toro}\headword{toro}{\pos{Noun}} {\definition{a symbol of a slain animal, used to display and inform people of the type of animal killed. It was also a hunter's pride to compete or show other hunter' of his skill. Feathers, offcut tail, or animal fur were displayed on a small stick or pitpit or obäll tree stick or stem. In other instances pandanus leaves were symbols for pig, grass for wallaby.}}
\end{entrylist}

\section*{3.5.6.1 Gesture}
\begin{entrylist}
\entry{ttang käpän}\headword{ttang käpän}{\pos{Verb}} {\definition{to snap one's fingers}}
\entry{ttang pllallem}\headword{ttang pllallem}{\pos{Intransitive A verb}} {\definition{to clap}}
\entry{wanwen}\headword{wanwen}{\pos{Transitive S verb}} {\definition{to shake, swing}}
\end{entrylist}

\section*{3.5.6.2 Point at}
\begin{entrylist}
\entry{tongg}\headword{tongg}{\pos{Transitive S verb}} {\definition{to point}}
\end{entrylist}

\section*{3.5.6.4 Laugh}
\begin{entrylist}
\entry{tongoe}\headword{tongoe}{\pos{Transitive S verb}} {\definition{to laugh (at)}}
\end{entrylist}

\section*{3.5.6.5 Cry, tear}
\begin{entrylist}
\entry{gädagäde}\headword{gädagäde}{\pos{Transitive S verb}} {\definition{to throw a tantrum}}
\entry{ngänaeka}\headword{ngänaeka}{\pos{Transitive A verb}} {\definition{to cry (about, for)}}
\entry{ngänaeka pollongg}\headword{ngänaeka pollongg}{\pos{Intransitive S verb}} {\definition{to burst into tears}}
\entry{ngänaeka täräm}\headword{ngänaeka täräm}{\pos{Noun}} {\definition{tear}}
\entry{ngänaekangänaeka}\headword{ngänaekangänaeka}{\pos{Intransitive A verb}} {\definition{nonsingular form of ngänaeka}}
\end{entrylist}

\section*{3.5.7 Reading and writing}
\begin{entrylist}
\entry{peba}\headword{peba}{\pos{Noun}} {\definition{paper}}
\end{entrylist}

\section*{3.5.7.1 Write}
\begin{entrylist}
\entry{biro}\headword{biro}{\pos{Noun}} {\definition{pen (writing implement)}}
\entry{dadäräb}\headword{dadäräb}{\pos{Transitive S verb}} {\definition{to write}}
\entry{ddägaddäge}\headword{ddägaddäge}{\pos{Intransitive S verb}} {\definition{to write}}
\entry{pen}\headword{pen}{\pos{Noun}} {\definition{pen}}
\end{entrylist}

\section*{3.5.7.2 Written material}
\begin{entrylist}
\entry{buk}\headword{buk}{\pos{Noun}} {\definition{book}}
\end{entrylist}

\section*{3.5.7.3 Read}
\begin{entrylist}
\entry{ttängattänge}\headword{ttängattänge}{\pos{Transitive S verb}} {\definition{to read, recite}}
\end{entrylist}

\section*{3.5.8.1 Meaning}
\begin{entrylist}
\entry{eka midd}\headword{eka midd}{\pos{Noun}} {\definition{meaning}}
\entry{midd}\headword{midd}{\pos{Noun}} {\definition{meaning; core, essence}}
\end{entrylist}

\section*{3.5.8.4 Show, indicate}
\begin{entrylist}
\entry{taempäg}\headword{taempäg}{\pos{Transitive S verb}} {\definition{to show, indicate, reveal}}
\end{entrylist}

\section*{3.5.9.6 Communication devices}
\begin{entrylist}
\entry{ali}\headword{ali}{\pos{Noun}} {\definition{conch shell}}
\entry{utt}\headword{utt}{\pos{Noun}} {\definition{conch shell}}
\entry{uttutt}\headword{uttutt}{\pos{Noun}} {\definition{small conch shell}}
\end{entrylist}

\section*{3.6 Teach}
\begin{entrylist}
\entry{bun lla}\headword{bun lla}{\pos{Noun}} {\definition{instructor, leader}}
\entry{tameny}\headword{tameny}{\pos{Transitive S verb}} {\definition{to teach}}
\end{entrylist}

\section*{3.6.1 Show, explain}
\begin{entrylist}
\entry{indrang}\headword{indrang}{\pos{Transitive A verb}} {\definition{to show, reveal, illuminate, bring to light}}
\end{entrylist}

\section*{3.6.2 School}
\begin{entrylist}
\entry{ai skul}\headword{ai skul}{\pos{Noun}} {\definition{secondary school, high school}}
\entry{buk}\headword{buk}{\pos{Noun}} {\definition{book}}
\entry{elementri skul}\headword{elementri skul}{\pos{Noun}} {\definition{elementary school}}
\entry{greid}\headword{greid}{\pos{Noun}} {\definition{grade}}
\entry{hedmasta}\headword{hedmasta}{\pos{Noun}} {\definition{headmaster}}
\entry{kame tameny ma skul}\headword{kame tameny ma skul}{\pos{}} {\definition{open distance school}}
\entry{klas}\headword{klas}{\pos{Noun}} {\definition{class}}
\entry{käme näkäp me ddäganen}\headword{käme näkäp me ddäganen}{\pos{Verb}} {\definition{to memorize}}
\entry{oknen ma}\headword{oknen ma}{\pos{Noun}} {\definition{duster, eraser}}
\entry{peba}\headword{peba}{\pos{Noun}} {\definition{paper}}
\entry{polen}\headword{polen}{\pos{Noun}} {\definition{assembly}}
\entry{ponen ma}\headword{ponen ma}{\pos{Noun}} {\definition{pencil sharpener}}
\entry{praemari skul}\headword{praemari skul}{\pos{Noun}} {\definition{primary school}}
\entry{skul}\headword{skul}{\pos{Noun}} {\definition{school; education}}
\entry{skulang}\headword{skulang}{\pos{Noun}} {\definition{student}}
\entry{tameny}\headword{tameny}{\pos{Transitive S verb}} {\definition{to teach}}
\entry{tamenyang}\headword{tamenyang}{\pos{Noun}} {\definition{teacher}}
\entry{tongoe ma ngätt}\headword{tongoe ma ngätt}{\pos{Noun}} {\definition{playground}}
\entry{ttaepnenttaepnen ma skul}\headword{ttaepnenttaepnen ma skul}{\pos{Noun}} {\definition{technical school}}
\entry{tuk skul}\headword{tuk skul}{\pos{Noun}} {\definition{technical university, university}}
\end{entrylist}

\section*{3.6.5 Correct}
\begin{entrylist}
\entry{imomdae}\headword{imomdae}{\pos{Modifier}} {\definition{correct, right}}
\entry{ttättle}\headword{ttättle}{\pos{Modifier}} {\definition{correct, proper}}
\end{entrylist}

\section*{3.6.7 Test}
\begin{entrylist}
\entry{soroe}\headword{soroe}{\pos{Transitive S verb}} {\definition{to challenge, try, test; tempt}}
\entry{soroe}\headword{soroe}{\pos{Noun}} {\definition{test, exam}}
\end{entrylist}

\section*{4.1 Relationships}
\begin{entrylist}
\entry{kamo}\headword{kamo}{\pos{Noun}} {\definition{reciprocal term for the young man and the older man that takes him through initiation}}
\entry{konymad}\headword{konymad}{\pos{Noun}} {\definition{man who steals a woman's things during her initiation ceremony}}
\entry{llɨg ikopang}\headword{llɨg ikopang}{\pos{Noun}} {\definition{boyfriend}}
\entry{masar}\headword{masar}{\pos{Kinship noun}} {\definition{grandfather (one's parent's father; reciprocal); ancestor}}
\entry{noponda}\headword{noponda}{\pos{Kinship noun}} {\definition{one's father's exchange brother (still owes him)}}
\entry{päzäg}\headword{päzäg}{\pos{Kinship noun}} {\definition{brother-in-law (man's wife's brother or man's sister's husband; reciprocal)}}
\entry{päzäg}\headword{päzäg}{\pos{Kinship noun}} {\definition{sister-in-law (man's wife's sister)}}
\entry{tikop}\headword{tikop}{\pos{Noun}} {\definition{beloved, sweetheart, dear}}
\end{entrylist}

\section*{4.1.1 Friend}
\begin{entrylist}
\entry{irwema}\headword{irwema}{\pos{Noun}} {\definition{two female friends who share a twin fruit from the irwe tree}}
\entry{kaeg}\headword{kaeg}{\pos{Noun}} {\definition{close male friend of the same age who went through initiation at the same time}}
\entry{kamo}\headword{kamo}{\pos{Noun}} {\definition{reciprocal term for the young man and the older man that takes him through initiation}}
\entry{kapera}\headword{kapera}{\pos{Noun}} {\definition{male friend or partner who comes from out of town}}
\entry{kokallma}\headword{kokallma}{\pos{Noun}} {\definition{two friends who share a twin fruit from the kokall tree (palm tree in the bush)}}
\entry{kämany}\headword{kämany}{\pos{Noun}} {\definition{type of friendship}}
\entry{mägällma}\headword{mägällma}{\pos{Noun}} {\definition{friends who share a twin fruit from the mägäll tree}}
\entry{nag ttoen}\headword{nag ttoen}{\pos{Noun}} {\definition{friendship}}
\entry{nagnag}\headword{nagnag}{\pos{Noun}} {\definition{nonsingular form of nag}}
\entry{omad}\headword{omad}{\pos{Noun}} {\definition{type of friend}}
\entry{pana}\headword{pana}{\pos{Noun}} {\definition{type of relationship}}
\entry{spallma}\headword{spallma}{\pos{Noun}} {\definition{two friends who split a twin coconut frond}}
\entry{tarko}\headword{tarko}{\pos{Noun}} {\definition{female friends who have shared a joined fruit or vegetable}}
\entry{upma}\headword{upma}{\pos{Noun}} {\definition{two friends who share a twin banana fruit}}
\end{entrylist}

\section*{4.1.4.1 Social class}
\begin{entrylist}
\entry{ulle binang}\headword{ulle binang}{\pos{Noun}} {\definition{master, owner, ruler, important person}}
\entry{zizag}\headword{zizag}{\pos{Noun}} {\definition{owner, master, lord}}
\end{entrylist}

\section*{4.1.5 Unity}
\begin{entrylist}
\entry{ngetae}\headword{ngetae}{\pos{Transitive S verb}} {\definition{to unite, join}}
\end{entrylist}

\section*{4.1.6.3 Alone}
\begin{entrylist}
\entry{känyär}\headword{känyär}{\pos{Modifier}} {\definition{alone, by oneself (follows a genitive noun)}}
\end{entrylist}

\section*{4.1.6.4 Independent person}
\begin{entrylist}
\entry{ddäddäbeabag}\headword{ddäddäbeabag}{\pos{Modifier}} {\definition{independent, resourceful}}
\end{entrylist}

\section*{4.1.8 Show affection}
\begin{entrylist}
\entry{ngongop}\headword{ngongop}{\pos{Transitive S verb}} {\definition{to hug, embrace}}
\entry{ume ddäddäl}\headword{ume ddäddäl}{\pos{Transitive S verb}} {\definition{to kiss}}
\end{entrylist}

\section*{4.1.9 Kinship}
\begin{entrylist}
\entry{kakak}\headword{kakak}{\pos{Noun}} {\definition{nonsingular form of kak}}
\entry{lla bombllo}\headword{lla bombllo}{\pos{Noun}} {\definition{generation}}
\entry{masamasar}\headword{masamasar}{\pos{Kinship noun}} {\definition{forefathers, ancestors}}
\end{entrylist}

\section*{4.1.9.1.1 Grandfather, grandmother}
\begin{entrylist}
\entry{auseause}\headword{auseause}{\pos{Noun}} {\definition{nonsingular form of ause}}
\entry{kak}\headword{kak}{\pos{Kinship noun}} {\definition{grandmother (one's parent's mother; reciprocal)}}
\entry{masar}\headword{masar}{\pos{Kinship noun}} {\definition{grandfather (one's parent's father; reciprocal); ancestor}}
\end{entrylist}

\section*{4.1.9.1.2 Father, mother}
\begin{entrylist}
\entry{baba}\headword{baba}{\pos{Kinship noun}} {\definition{father}}
\entry{mama}\headword{mama}{\pos{Kinship noun}} {\definition{mother}}
\entry{mami}\headword{mami}{\pos{Kinship noun}} {\definition{mommy, mummy}}
\entry{mäda}\headword{mäda}{\pos{Kinship noun}} {\definition{father}}
\entry{mäg}\headword{mäg}{\pos{Kinship noun}} {\definition{mother}}
\entry{yae}\headword{yae}{\pos{Kinship noun}} {\definition{mother}}
\entry{yaya}\headword{yaya}{\pos{Kinship noun}} {\definition{father}}
\end{entrylist}

\section*{4.1.9.1.3 Brother, sister}
\begin{entrylist}
\entry{dada}\headword{dada}{\pos{Kinship noun}} {\definition{older sibling of the same sex (man's older brother or woman's older sister)}}
\entry{mang}\headword{mang}{\pos{Kinship noun}} {\definition{brother (of a woman)}}
\entry{mangmang}\headword{mangmang}{\pos{Kinship noun}} {\definition{nonsingular form of mang}}
\entry{mosen}\headword{mosen}{\pos{Kinship noun}} {\definition{older sibling of the same-sex (man's older brother or woman's older sister)}}
\entry{mänyan}\headword{mänyan}{\pos{Kinship noun}} {\definition{younger sibling of the same-sex (man's younger brother or woman's younger sister)}}
\entry{mänyan}\headword{mänyan}{\pos{Kinship noun}} {\definition{co-sibling-in-law (man's wife's younger sister's husband or woman's husband's younger brother's wife)}}
\end{entrylist}

\section*{4.1.9.1.4 Son, daughter}
\begin{entrylist}
\entry{llɨg}\headword{llɨg}{\pos{Kinship noun}} {\definition{son}}
\entry{llɨg}\headword{llɨg}{\pos{Kinship noun}} {\definition{child}}
\end{entrylist}

\section*{4.1.9.1.5 Grandson, granddaughter}
\begin{entrylist}
\entry{kok}\headword{kok}{\pos{Kinship noun}} {\definition{grandchild (one's child's child)}}
\entry{kokok}\headword{kokok}{\pos{Kinship noun}} {\definition{nonsingular form of kok}}
\end{entrylist}

\section*{4.1.9.1.6 Uncle, aunt}
\begin{entrylist}
\entry{ddäma mu}\headword{ddäma mu}{\pos{Noun}} {\definition{birth payment made to the maternal uncle}}
\entry{masar}\headword{masar}{\pos{Kinship noun}} {\definition{uncle-in-law (woman's husband's mother's younger brother; reciprocal)}}
\entry{meyang}\headword{meyang}{\pos{Kinship noun}} {\definition{uncle (one's father's younger brother)}}
\entry{mädaulle}\headword{mädaulle}{\pos{Kinship noun}} {\definition{aunt (one's father's elder sister)}}
\entry{mädaulle}\headword{mädaulle}{\pos{Kinship noun}} {\definition{uncle (one's father's elder brother)}}
\entry{mälla yae}\headword{mälla yae}{\pos{Kinship noun}} {\definition{aunt (one's mother's elder sister)}}
\entry{mällpa}\headword{mällpa}{\pos{Kinship noun}} {\definition{aunt (one's mother's sister)}}
\entry{pope}\headword{pope}{\pos{Kinship noun}} {\definition{uncle (one's mother's brother)}}
\end{entrylist}

\section*{4.1.9.1.7 Cousin}
\begin{entrylist}
\entry{yäkäl}\headword{yäkäl}{\pos{Kinship noun}} {\definition{cousin}}
\end{entrylist}

\section*{4.1.9.1.9 Birth order}
\begin{entrylist}
\entry{dada}\headword{dada}{\pos{Kinship noun}} {\definition{older sibling of the same sex (man's older brother or woman's older sister)}}
\entry{mosen}\headword{mosen}{\pos{Kinship noun}} {\definition{older sibling of the same-sex (man's older brother or woman's older sister)}}
\entry{mosen}\headword{mosen}{\pos{Noun}} {\definition{eldest, firstborn}}
\entry{mänyan}\headword{mänyan}{\pos{Kinship noun}} {\definition{younger sibling of the same-sex (man's younger brother or woman's younger sister)}}
\entry{sära pot}\headword{sära pot}{\pos{Noun}} {\definition{lastborn, youngest}}
\entry{särapipi}\headword{särapipi}{\pos{Noun}} {\definition{lastborn, youngest}}
\end{entrylist}

\section*{4.1.9.2 Related by marriage}
\begin{entrylist}
\entry{mäg ulle}\headword{mäg ulle}{\pos{Kinship noun}} {\definition{aunt (one's father's elder brother's wife)}}
\entry{sens}\headword{sens}{\pos{Transitive A verb}} {\definition{to exchange (in marriage)}}
\entry{yäkälnda}\headword{yäkälnda}{\pos{Kinship noun}} {\definition{uncle (one's mother's sister's husband)}}
\end{entrylist}

\section*{4.1.9.2.1 Husband, wife}
\begin{entrylist}
\entry{mälla}\headword{mälla}{\pos{Kinship noun}} {\definition{wife}}
\end{entrylist}

\section*{4.1.9.2.2 In-law}
\begin{entrylist}
\entry{erang}\headword{erang}{\pos{Kinship noun}} {\definition{exchange sibling; exchange brother (man's wife's brother who marries the man's sister; reciprocal); exchange sister (woman's husband's sister who marries the woman's brother; reciprocal)}}
\entry{erngazeg}\headword{erngazeg}{\pos{Kinship noun}} {\definition{exchange cousin (one's parent's exchange sibling's child)}}
\entry{erngazenda}\headword{erngazenda}{\pos{Kinship noun}} {\definition{exchange uncle (one's parent's exchange brother)}}
\entry{erngazmäg}\headword{erngazmäg}{\pos{Kinship noun}} {\definition{exchange aunt (one's parent's exchange sister)}}
\entry{inbo}\headword{inbo}{\pos{Kinship noun}} {\definition{brother-in-law (woman's husband's brother)}}
\entry{inbo}\headword{inbo}{\pos{Kinship noun}} {\definition{sister-in-law (man's elder brother's wife or woman's husband's younger sister; reciprocal)}}
\entry{izig}\headword{izig}{\pos{Kinship noun}} {\definition{co-sister-in-law (woman's husband's brother's wife)}}
\entry{kak}\headword{kak}{\pos{Kinship noun}} {\definition{mother-in-law (woman's husband's mother; reciprocal)}}
\entry{kak}\headword{kak}{\pos{Kinship noun}} {\definition{daughter-in-law (woman's son's wife; reciprocal)}}
\entry{kobeam}\headword{kobeam}{\pos{Kinship noun}} {\definition{co-brother-in-law (man's wife's sister's husband)}}
\entry{kok}\headword{kok}{\pos{Kinship noun}} {\definition{daughter-in-law (one's son's wife)}}
\entry{masar}\headword{masar}{\pos{Kinship noun}} {\definition{father-in-law (woman's husband's father; reciprocal)}}
\entry{masar}\headword{masar}{\pos{Kinship noun}} {\definition{daughter-in-law (man's son's wife; reciprocal)}}
\entry{masar}\headword{masar}{\pos{Kinship noun}} {\definition{uncle-in-law (woman's husband's mother's younger brother; reciprocal)}}
\entry{meyang}\headword{meyang}{\pos{Kinship noun}} {\definition{brother-in-law (woman's husband's younger brother)}}
\entry{mädaulle}\headword{mädaulle}{\pos{Kinship noun}} {\definition{brother-in-law (woman's husband's elder brother)}}
\entry{mädaulle}\headword{mädaulle}{\pos{Kinship noun}} {\definition{sister-in-law (woman's husband's elder sister or woman's younger brother's wife; reciprocal)}}
\entry{mänang}\headword{mänang}{\pos{Kinship noun}} {\definition{father-in-law (man's wife's father; reciprocal)}}
\entry{mänang}\headword{mänang}{\pos{Kinship noun}} {\definition{mother-in-law (man's wife's mother; reciprocal)}}
\entry{mänang}\headword{mänang}{\pos{Kinship noun}} {\definition{son-in-law (one's daughter's husband; reciprocal)}}
\entry{mänyan}\headword{mänyan}{\pos{Kinship noun}} {\definition{co-sibling-in-law (man's wife's younger sister's husband or woman's husband's younger brother's wife)}}
\entry{nyamällatt}\headword{nyamällatt}{\pos{Kinship noun}} {\definition{woman's exchange sibling's child}}
\entry{päzäg}\headword{päzäg}{\pos{Kinship noun}} {\definition{brother-in-law (man's wife's brother or man's sister's husband; reciprocal)}}
\entry{päzäg}\headword{päzäg}{\pos{Kinship noun}} {\definition{sister-in-law (man's wife's sister)}}
\entry{päzäpäzäg}\headword{päzäpäzäg}{\pos{Kinship noun}} {\definition{in-laws}}
\end{entrylist}

\section*{4.1.9.3 Widow, widower}
\begin{entrylist}
\entry{gungg}\headword{gungg}{\pos{Transitive S verb}} {\definition{to marry a widow}}
\entry{mik}\headword{mik}{\pos{Noun}} {\definition{widow, widower}}
\entry{pakätt}\headword{pakätt}{\pos{Noun}} {\definition{widow's robe}}
\entry{sära}\headword{sära}{\pos{Noun}} {\definition{uncut grass skirt}}
\end{entrylist}

\section*{4.1.9.4 Orphan}
\begin{entrylist}
\entry{mädameny}\headword{mädameny}{\pos{Modifier}} {\definition{fatherless}}
\entry{mägmeny}\headword{mägmeny}{\pos{Modifier}} {\definition{motherless}}
\end{entrylist}

\section*{4.1.9.6 Adopt}
\begin{entrylist}
\entry{ddaebän}\headword{ddaebän}{\pos{Transitive S verb}} {\definition{to adopt}}
\entry{ngangem}\headword{ngangem}{\pos{Transitive S verb}} {\definition{to adopt}}
\end{entrylist}

\section*{4.1.9.7 Non-relative}
\begin{entrylist}
\entry{ttoenglla}\headword{ttoenglla}{\pos{Noun}} {\definition{unrelated to one's clan or tribe}}
\end{entrylist}

\section*{4.1.9.8 Family, clan}
\begin{entrylist}
\entry{llabun}\headword{llabun}{\pos{Noun}} {\definition{relative, kinsman, clansman}}
\entry{nane}\headword{nane}{\pos{Kinship noun}} {\definition{aunt (one's parent's younger sister)}}
\entry{pemli}\headword{pemli}{\pos{Noun}} {\definition{family}}
\entry{tän}\headword{tän}{\pos{Noun}} {\definition{clan}}
\entry{zaze}\headword{zaze}{\pos{Noun}} {\definition{generation}}
\end{entrylist}

\section*{4.1.9.9 Race}
\begin{entrylist}
\entry{dundu kllamen}\headword{dundu kllamen}{\pos{Noun}} {\definition{type of game involving a race}}
\entry{tän}\headword{tän}{\pos{Noun}} {\definition{tribe, nation, people}}
\end{entrylist}

\section*{4.2.1.1 Invite}
\begin{entrylist}
\entry{trungg}\headword{trungg}{\pos{Transitive S verb}} {\definition{to invite, call over, summon}}
\end{entrylist}

\section*{4.2.1.2 Encounter}
\begin{entrylist}
\entry{ngarängg}\headword{ngarängg}{\pos{Transitive S verb}} {\definition{to encounter, meet, run into}}
\end{entrylist}

\section*{4.2.1.3 Meet together}
\begin{entrylist}
\entry{ngämen}\headword{ngämen}{\pos{Transitive S verb}} {\definition{to reach, catch up to}}
\entry{ttängkamäll}\headword{ttängkamäll}{\pos{Transitive S verb}} {\definition{to meet, reach}}
\end{entrylist}

\section*{4.2.1.4 Visit}
\begin{entrylist}
\entry{ngatae}\headword{ngatae}{\pos{Transitive S verb}} {\definition{to visit}}
\end{entrylist}

\section*{4.2.1.4.1 Welcome, receive}
\begin{entrylist}
\entry{ballɨngg}\headword{ballɨngg}{\pos{Transitive S verb}} {\definition{to welcome, greet}}
\entry{kalmokalmoe}\headword{kalmokalmoe}{\pos{Modifier}} {\definition{welcoming}}
\entry{källakällae}\headword{källakällae}{\pos{Modifier}} {\definition{hospitable}}
\entry{mipdab}\headword{mipdab}{\pos{Transitive A verb}} {\definition{to accommodate; offer food to a visitor}}
\entry{okokol}\headword{okokol}{\pos{Transitive A verb}} {\definition{to welcome}}
\entry{seborsebor}\headword{seborsebor}{\pos{Transitive A verb}} {\definition{to welcome}}
\end{entrylist}

\section*{4.2.1.4.2 Show hospitality}
\begin{entrylist}
\entry{täbädd}\headword{täbädd}{\pos{Noun}} {\definition{guest, visitor, stranger}}
\end{entrylist}

\section*{4.2.1.7 Crowd, group}
\begin{entrylist}
\entry{gul}\headword{gul}{\pos{Noun}} {\definition{crowd, group, mob; school (of fish)}}
\entry{kullum}\headword{kullum}{\pos{Noun}} {\definition{group}}
\entry{lla gul}\headword{lla gul}{\pos{Noun}} {\definition{crowd}}
\entry{ngoi}\headword{ngoi}{\pos{Transitive A verb}} {\definition{to crowd, surround}}
\entry{ttämattäme}\headword{ttämattäme}{\pos{Intransitive S verb}} {\definition{to be crowded}}
\entry{wändäg}\headword{wändäg}{\pos{Transitive S verb}} {\definition{to crowd}}
\end{entrylist}

\section*{4.2.1.8 Organization}
\begin{entrylist}
\entry{menizment}\headword{menizment}{\pos{Noun}} {\definition{management}}
\end{entrylist}

\section*{4.2.1.8.3 Belong to an organization}
\begin{entrylist}
\entry{memba}\headword{memba}{\pos{Noun}} {\definition{member}}
\end{entrylist}

\section*{4.2.2.1 Ceremony}
\begin{entrylist}
\entry{konymad}\headword{konymad}{\pos{Noun}} {\definition{man who steals a woman's things during her initiation ceremony}}
\entry{waglla}\headword{waglla}{\pos{Noun}} {\definition{bullroarer}}
\end{entrylist}

\section*{4.2.2.2 Festival, show}
\begin{entrylist}
\entry{ingong}\headword{ingong}{\pos{Noun}} {\definition{sing-sing}}
\entry{para}\headword{para}{\pos{Noun}} {\definition{event where harvest and hunting bounty are compared and gifted for bragging rights}}
\end{entrylist}

\section*{4.2.2.3 Celebrate}
\begin{entrylist}
\entry{ballɨngg}\headword{ballɨngg}{\pos{Transitive S verb}} {\definition{to welcome, greet}}
\entry{kilikili}\headword{kilikili}{\pos{Transitive A verb}} {\definition{to rejoice, celebrate}}
\entry{ttang pllallem}\headword{ttang pllallem}{\pos{Intransitive A verb}} {\definition{to clap}}
\end{entrylist}

\section*{4.2.3 Music}
\begin{entrylist}
\entry{mamamemett}\headword{mamamemett}{\pos{Modifier}} {\definition{fast beat}}
\entry{tatratatraema}\headword{tatratatraema}{\pos{Noun}} {\definition{music}}
\end{entrylist}

\section*{4.2.3.3 Sing}
\begin{entrylist}
\entry{bandra}\headword{bandra}{\pos{Noun}} {\definition{song}}
\entry{ingong}\headword{ingong}{\pos{Noun}} {\definition{sing-sing}}
\entry{llɨtɨt}\headword{llɨtɨt}{\pos{Transitive S verb}} {\definition{to sing}}
\entry{pomer}\headword{pomer}{\pos{Noun}} {\definition{whistle}}
\entry{pättol}\headword{pättol}{\pos{Transitive S verb}} {\definition{to start singing}}
\entry{waewae}\headword{waewae}{\pos{Noun}} {\definition{song sung while beating sago}}
\end{entrylist}

\section*{4.2.3.5 Musical instrument}
\begin{entrylist}
\entry{alläp}\headword{alläp}{\pos{Noun}} {\definition{kundu drum}}
\entry{borale}\headword{borale}{\pos{Noun}} {\definition{traditional bamboo flute used to scare wallabies}}
\entry{dape}\headword{dape}{\pos{Noun}} {\definition{drum head}}
\entry{darombe}\headword{darombe}{\pos{Noun}} {\definition{mouth harp. quarter-moon-shaped bamboo flute with honey inside; takes a day to make}}
\entry{gora}\headword{gora}{\pos{Noun}} {\definition{rattle}}
\entry{inkätt}\headword{inkätt}{\pos{Noun}} {\definition{hollow of drum}}
\entry{kuib}\headword{kuib}{\pos{Noun}} {\definition{type of drum}}
\entry{patepate}\headword{patepate}{\pos{Noun}} {\definition{bamboo sticks used for percussion}}
\entry{patt}\headword{patt}{\pos{Noun}} {\definition{drum shell}}
\entry{ttatt kutt}\headword{ttatt kutt}{\pos{Noun}} {\definition{drum rim}}
\entry{ttongo}\headword{ttongo}{\pos{Noun}} {\definition{drum handle}}
\entry{waglla}\headword{waglla}{\pos{Noun}} {\definition{bullroarer}}
\entry{winyteya}\headword{winyteya}{\pos{Noun}} {\definition{honeycomb}}
\end{entrylist}

\section*{4.2.4 Dance}
\begin{entrylist}
\entry{bonzro}\headword{bonzro}{\pos{Noun}} {\definition{dancing on the side playfully}}
\entry{ddällombog}\headword{ddällombog}{\pos{Transitive S verb}} {\definition{to miss}}
\entry{därdärag}\headword{därdärag}{\pos{Adverb}} {\definition{in pairs}}
\entry{gora}\headword{gora}{\pos{Noun}} {\definition{rattle}}
\entry{ingong}\headword{ingong}{\pos{Noun}} {\definition{dance}}
\entry{inungoe}\headword{inungoe}{\pos{Transitive S verb}} {\definition{to shake (when dancing)}}
\entry{komotupi molle molle}\headword{komotupi molle molle}{\pos{Noun}} {\definition{type of dancing game that involves ginger}}
\entry{kämag}\headword{kämag}{\pos{Noun}} {\definition{round dance with singers and kundu drum in the middle}}
\entry{lla ikoikopang}\headword{lla ikoikopang}{\pos{Noun}} {\definition{audience}}
\entry{mängalmängal}\headword{mängalmängal}{\pos{Modifier}} {\definition{quick}}
\entry{nyäng}\headword{nyäng}{\pos{Intransitive S verb}} {\definition{to dance}}
\entry{peraenggag}\headword{peraenggag}{\pos{Noun}} {\definition{type of dance with two groups}}
\entry{sagol}\headword{sagol}{\pos{Noun}} {\definition{men's dance}}
\entry{saomasaoma}\headword{saomasaoma}{\pos{Noun}} {\definition{dancing band}}
\end{entrylist}

\section*{4.2.5 Drama}
\begin{entrylist}
\entry{nyanyu}\headword{nyanyu}{\pos{Transitive S verb}} {\definition{to act, dramatize}}
\end{entrylist}

\section*{4.2.6.1 Game}
\begin{entrylist}
\entry{bikme tutu}\headword{bikme tutu}{\pos{Noun}} {\definition{type of string game}}
\entry{bol}\headword{bol}{\pos{Noun}} {\definition{ball}}
\entry{bunkälle bunkälle}\headword{bunkälle bunkälle}{\pos{Noun}} {\definition{type of game where the players tie hair and hide}}
\entry{dundu kllamen}\headword{dundu kllamen}{\pos{Noun}} {\definition{type of game involving a race}}
\entry{dändäräm käp}\headword{dändäräm käp}{\pos{Noun}} {\definition{type of marble game}}
\entry{idaida}\headword{idaida}{\pos{Noun}} {\definition{type of game played outside in open space}}
\entry{imonzimonz}\headword{imonzimonz}{\pos{Noun}} {\definition{tag (game)}}
\entry{komlle}\headword{komlle}{\pos{Noun}} {\definition{type of game played with string}}
\entry{komotupi molle molle}\headword{komotupi molle molle}{\pos{Noun}} {\definition{type of dancing game that involves ginger}}
\entry{komälle}\headword{komälle}{\pos{Noun}} {\definition{type of game played with string}}
\entry{koplle täkmäl täkmäl}\headword{koplle täkmäl täkmäl}{\pos{Noun}} {\definition{type of game involving throwing koplle fruit}}
\entry{kurikuri}\headword{kurikuri}{\pos{Noun}} {\definition{game involving spinning tree fruit on one's hand}}
\entry{kätt tongoe}\headword{kätt tongoe}{\pos{Noun}} {\definition{type of game played with shells}}
\entry{labelabet}\headword{labelabet}{\pos{Noun}} {\definition{type of game involving planting a stick in the middle of a circle}}
\entry{llupi ttäganen ttägnen}\headword{llupi ttäganen ttägnen}{\pos{Noun}} {\definition{type of game involving hiding a tree branch in the water}}
\entry{mangkimangki}\headword{mangkimangki}{\pos{Noun}} {\definition{type of game involving chasing}}
\entry{nying tongoe}\headword{nying tongoe}{\pos{Noun}} {\definition{type of game involving kicking}}
\entry{patarapatara}\headword{patarapatara}{\pos{Noun}} {\definition{type of hand game where players make an L shape with their fingers}}
\entry{piro kanas}\headword{piro kanas}{\pos{Noun}} {\definition{type of game where the first person to see a star in the sky wins}}
\entry{tintromoltintromol}\headword{tintromoltintromol}{\pos{Noun}} {\definition{type of rhyming game}}
\entry{togotogol}\headword{togotogol}{\pos{Noun}} {\definition{hide-and-seek}}
\entry{tongoe}\headword{tongoe}{\pos{Noun}} {\definition{game; sport}}
\entry{wadär nyongkoe}\headword{wadär nyongkoe}{\pos{Noun}} {\definition{type of game involving pulling cane}}
\entry{wanttawantta}\headword{wanttawantta}{\pos{Noun}} {\definition{type of game like capture the flag but with a stick planted in the middle of a ring instead of a flag}}
\entry{wume nanyu}\headword{wume nanyu}{\pos{Noun}} {\definition{type of string game, imitation game}}
\end{entrylist}

\section*{4.2.6.2 Sports}
\begin{entrylist}
\entry{bol}\headword{bol}{\pos{Noun}} {\definition{ball}}
\entry{paspas}\headword{paspas}{\pos{Noun}} {\definition{type of game where players try to keep a ball off the ground}}
\entry{soka}\headword{soka}{\pos{Noun}} {\definition{soccer}}
\entry{tongoe}\headword{tongoe}{\pos{Noun}} {\definition{game; sport}}
\end{entrylist}

\section*{4.2.6.2.1 Football, soccer}
\begin{entrylist}
\entry{gol}\headword{gol}{\pos{Noun}} {\definition{goal}}
\end{entrylist}

\section*{4.2.7 Play, fun}
\begin{entrylist}
\entry{tongoe}\headword{tongoe}{\pos{Transitive S verb}} {\definition{to play}}
\end{entrylist}

\section*{4.2.8 Humor}
\begin{entrylist}
\entry{tongoeang}\headword{tongoeang}{\pos{Modifier}} {\definition{funny, humorous}}
\end{entrylist}

\section*{4.2.8.1 Serious}
\begin{entrylist}
\entry{binang}\headword{binang}{\pos{Modifier}} {\definition{serious}}
\end{entrylist}

\section*{4.2.9 Holiday}
\begin{entrylist}
\entry{Kuddäll Opap}\headword{Kuddäll Opap}{\pos{Proper noun}} {\definition{Passover}}
\entry{krismas}\headword{krismas}{\pos{Noun}} {\definition{Christmas}}
\end{entrylist}

\section*{4.3 Behavior}
\begin{entrylist}
\entry{nyanyu}\headword{nyanyu}{\pos{Noun}} {\definition{action}}
\end{entrylist}

\section*{4.3.1.1 Bad, immoral}
\begin{entrylist}
\entry{kotkot}\headword{kotkot}{\pos{Modifier}} {\definition{dirty, unclean}}
\end{entrylist}

\section*{4.3.2 Admire someone}
\begin{entrylist}
\entry{sɨngesɨnge}\headword{sɨngesɨnge}{\pos{Noun}} {\definition{admiration}}
\end{entrylist}

\section*{4.3.2.3 Proud}
\begin{entrylist}
\entry{ddonddo}\headword{ddonddo}{\pos{Modifier}} {\definition{proud}}
\end{entrylist}

\section*{4.3.3 Love}
\begin{entrylist}
\entry{bälämbäl}\headword{bälämbäl}{\pos{Transitive S verb}} {\definition{to miss, long for}}
\entry{gämoe}\headword{gämoe}{\pos{Transitive S verb}} {\definition{to miss, feel longing for}}
\entry{mitmit}\headword{mitmit}{\pos{Transitive A verb}} {\definition{to miss, long for}}
\entry{moko}\headword{moko}{\pos{Noun}} {\definition{desire, want; love}}
\entry{tikop}\headword{tikop}{\pos{Noun}} {\definition{beloved, sweetheart, dear}}
\end{entrylist}

\section*{4.3.3.3 Abandon}
\begin{entrylist}
\entry{kuddäll}\headword{kuddäll}{\pos{Modifier}} {\definition{abandoned}}
\entry{tämpeyam}\headword{tämpeyam}{\pos{Transitive S verb}} {\definition{to abandon, give up}}
\entry{yämbäg}\headword{yämbäg}{\pos{Transitive S verb}} {\definition{to disown, repudiate}}
\end{entrylist}

\section*{4.3.4.2 Help}
\begin{entrylist}
\entry{ngämingg}\headword{ngämingg}{\pos{Intransitive S verb}} {\definition{to help}}
\entry{ngäminggag}\headword{ngäminggag}{\pos{Modifier}} {\definition{helpful, supportive}}
\end{entrylist}

\section*{4.3.4.4.1 Selfish}
\begin{entrylist}
\entry{ttonggmeny}\headword{ttonggmeny}{\pos{Modifier}} {\definition{selfish, greedy}}
\end{entrylist}

\section*{4.3.4.5 Share with}
\begin{entrylist}
\entry{kädab}\headword{kädab}{\pos{Transitive S verb}} {\definition{to share, split, portion}}
\entry{nyänye}\headword{nyänye}{\pos{Transitive S verb}} {\definition{to share, split}}
\entry{singoll}\headword{singoll}{\pos{Transitive A verb}} {\definition{to give, provide, share}}
\end{entrylist}

\section*{4.3.4.5.1 Provide for, support}
\begin{entrylist}
\entry{singoll}\headword{singoll}{\pos{Transitive A verb}} {\definition{to give, provide, share}}
\entry{wänänang}\headword{wänänang}{\pos{Modifier}} {\definition{provider, bringing back food for one's family}}
\end{entrylist}

\section*{4.3.4.5.2 Care for}
\begin{entrylist}
\entry{täbab}\headword{täbab}{\pos{Transitive S verb}} {\definition{to watch, look after, patrol}}
\end{entrylist}

\section*{4.3.5.3 Reliable}
\begin{entrylist}
\entry{dangkam}\headword{dangkam}{\pos{Intransitive S verb}} {\definition{to rely on}}
\end{entrylist}

\section*{4.3.5.5 Deceive}
\begin{entrylist}
\entry{kuki}\headword{kuki}{\pos{Transitive A verb}} {\definition{to deceive, trick, lie to}}
\entry{trik}\headword{trik}{\pos{Noun}} {\definition{trick}}
\end{entrylist}

\section*{4.3.6.2 Tidy}
\begin{entrylist}
\entry{tontobabag}\headword{tontobabag}{\pos{Modifier}} {\definition{neat, tidy}}
\end{entrylist}

\section*{4.3.6.4 Mistake}
\begin{entrylist}
\entry{gomoe}\headword{gomoe}{\pos{Transitive S verb}} {\definition{to make a mistake}}
\end{entrylist}

\section*{4.3.7.2 Crazy}
\begin{entrylist}
\entry{konkon}\headword{konkon}{\pos{Modifier}} {\definition{crazy, mad, insane, mentally ill}}
\end{entrylist}

\section*{4.3.9.1 Custom}
\begin{entrylist}
\entry{burag}\headword{burag}{\pos{Noun}} {\definition{bride price (given to the bride's family by the groom)}}
\entry{kastom}\headword{kastom}{\pos{Noun}} {\definition{custom}}
\entry{ngalen}\headword{ngalen}{\pos{Noun}} {\definition{way, habit, manner, custom}}
\entry{pentae}\headword{pentae}{\pos{Intransitive S verb}} {\definition{to transfer, transmit, spread, pass on, pass down}}
\end{entrylist}

\section*{4.4.2 Trouble}
\begin{entrylist}
\entry{ambag}\headword{ambag}{\pos{Intransitive A verb}} {\definition{to cause trouble}}
\entry{buddog}\headword{buddog}{\pos{Transitive A verb}} {\definition{to trouble, bother}}
\entry{ikoll}\headword{ikoll}{\pos{Noun}} {\definition{incident, problem, trouble}}
\entry{tärabol}\headword{tärabol}{\pos{Noun}} {\definition{trouble}}
\end{entrylist}

\section*{4.4.2.1 Problem}
\begin{entrylist}
\entry{buddo}\headword{buddo}{\pos{Noun}} {\definition{problem, issue}}
\end{entrylist}

\section*{4.4.2.3 Accident}
\begin{entrylist}
\entry{nälan}\headword{nälan}{\pos{Adverb}} {\definition{accidentally}}
\end{entrylist}

\section*{4.4.2.5 Separate, alone}
\begin{entrylist}
\entry{sapang}\headword{sapang}{\pos{Modifier}} {\definition{separate, apart, different; own, personal}}
\entry{ttaempäg}\headword{ttaempäg}{\pos{Transitive S verb}} {\definition{to seperate, divorce}}
\end{entrylist}

\section*{4.4.2.6 Suffer}
\begin{entrylist}
\entry{orwa}\headword{orwa}{\pos{Noun}} {\definition{suffering}}
\end{entrylist}

\section*{4.4.3.1 Brave}
\begin{entrylist}
\entry{lelmeny}\headword{lelmeny}{\pos{Modifier}} {\definition{brave, fearless, bold}}
\end{entrylist}

\section*{4.4.3.4 Caution}
\begin{entrylist}
\entry{tonang}\headword{tonang}{\pos{Modifier}} {\definition{careful, cautious}}
\end{entrylist}

\section*{4.4.3.5 Solve a problem}
\begin{entrylist}
\entry{särämbae}\headword{särämbae}{\pos{Transitive S verb}} {\definition{to fix, solve, resolve}}
\end{entrylist}

\section*{4.4.3.7 Survive}
\begin{entrylist}
\entry{källmakällme}\headword{källmakällme}{\pos{Intransitive S verb}} {\definition{to survive}}
\end{entrylist}

\section*{4.4.4.4 Save from trouble}
\begin{entrylist}
\entry{ttam}\headword{ttam}{\pos{Transitive A verb}} {\definition{to save someone's life}}
\end{entrylist}

\section*{4.4.4.5 Protect}
\begin{entrylist}
\entry{anben}\headword{anben}{\pos{Verb}} {\definition{to guard}}
\entry{pänggmeny}\headword{pänggmeny}{\pos{Transitive S verb}} {\definition{to protect, look after, take care of}}
\entry{tanter}\headword{tanter}{\pos{Transitive S verb}} {\definition{to guard in one's sleep, sleep with}}
\entry{tarakoll}\headword{tarakoll}{\pos{Noun}} {\definition{protective outer wall built by ancestors}}
\end{entrylist}

\section*{4.4.4.6 Free from bondage}
\begin{entrylist}
\entry{inttemängg}\headword{inttemängg}{\pos{Intransitive S verb}} {\definition{to leave, see off, release, set free}}
\end{entrylist}

\section*{4.5.1 Person in authority}
\begin{entrylist}
\entry{bun lla}\headword{bun lla}{\pos{Noun}} {\definition{instructor, leader}}
\entry{ulle binang}\headword{ulle binang}{\pos{Noun}} {\definition{master, owner, ruler, important person}}
\entry{zizag}\headword{zizag}{\pos{Noun}} {\definition{owner, master, lord}}
\end{entrylist}

\section*{4.5.3.1 Lead}
\begin{entrylist}
\entry{bun lla}\headword{bun lla}{\pos{Noun}} {\definition{instructor, leader}}
\entry{därängg}\headword{därängg}{\pos{Transitive S verb}} {\definition{to lead}}
\entry{därängkänan}\headword{därängkänan}{\pos{Noun}} {\definition{leader}}
\entry{lida}\headword{lida}{\pos{Noun}} {\definition{leader}}
\entry{täram}\headword{täram}{\pos{Transitive S verb}} {\definition{to lead, take, carry, collect}}
\end{entrylist}

\section*{4.5.3.2 Command}
\begin{entrylist}
\entry{indugoeg}\headword{indugoeg}{\pos{Transitive S verb}} {\definition{to command}}
\entry{umllang}\headword{umllang}{\pos{Transitive A verb}} {\definition{to tell, inform}}
\end{entrylist}

\section*{4.5.3.3 Discipline, train}
\begin{entrylist}
\entry{metmäll}\headword{metmäll}{\pos{Transitive S verb}} {\definition{to beat, flog, hit}}
\entry{pirik}\headword{pirik}{\pos{Noun}} {\definition{baton, stick}}
\end{entrylist}

\section*{4.5.4.1 Obey}
\begin{entrylist}
\entry{malam}\headword{malam}{\pos{Transitive S verb}} {\definition{to obey, follow}}
\end{entrylist}

\section*{4.5.4.5 Follow, be a disciple}
\begin{entrylist}
\entry{kollmäll}\headword{kollmäll}{\pos{Transitive S verb}} {\definition{to follow}}
\entry{kollmällang}\headword{kollmällang}{\pos{Noun}} {\definition{follower, disciple}}
\end{entrylist}

\section*{4.5.5 Honor}
\begin{entrylist}
\entry{nyanyem}\headword{nyanyem}{\pos{Transitive S verb}} {\definition{to respect}}
\end{entrylist}

\section*{4.6 Government}
\begin{entrylist}
\entry{gabmantt}\headword{gabmantt}{\pos{Noun}} {\definition{government}}
\entry{kowatt ikopang}\headword{kowatt ikopang}{\pos{Noun}} {\definition{law and order}}
\entry{mamos bo bun}\headword{mamos bo bun}{\pos{Noun}} {\definition{term in government}}
\end{entrylist}

\section*{4.6.1 Ruler}
\begin{entrylist}
\entry{bun lla}\headword{bun lla}{\pos{Noun}} {\definition{instructor, leader}}
\end{entrylist}

\section*{4.6.1.2 Government official}
\begin{entrylist}
\entry{gabana}\headword{gabana}{\pos{Noun}} {\definition{governor}}
\entry{klak}\headword{klak}{\pos{Noun}} {\definition{court clerk}}
\entry{mätar onyang}\headword{mätar onyang}{\pos{Noun}} {\definition{peace officer}}
\entry{wodd memba}\headword{wodd memba}{\pos{Noun}} {\definition{ward member}}
\end{entrylist}

\section*{4.6.2.1 Foreigner}
\begin{entrylist}
\entry{gabma}\headword{gabma}{\pos{Noun}} {\definition{white person}}
\entry{markae}\headword{markae}{\pos{Noun}} {\definition{white person}}
\end{entrylist}

\section*{4.6.3 Government organization}
\begin{entrylist}
\entry{ministri}\headword{ministri}{\pos{Noun}} {\definition{ministry}}
\entry{waswes}\headword{waswes}{\pos{Noun}} {\definition{political group}}
\end{entrylist}

\section*{4.6.3.1 Governing body}
\begin{entrylist}
\entry{komett}\headword{komett}{\pos{Noun}} {\definition{committee}}
\entry{palament}\headword{palament}{\pos{Noun}} {\definition{parliament}}
\end{entrylist}

\section*{4.6.4 Rule}
\begin{entrylist}
\entry{dändärek}\headword{dändärek}{\pos{Transitive S verb}} {\definition{to control, influence, rule, govern}}
\entry{ulle binang}\headword{ulle binang}{\pos{Noun}} {\definition{master, owner, ruler, important person}}
\entry{zizag}\headword{zizag}{\pos{Noun}} {\definition{owner, master, lord}}
\end{entrylist}

\section*{4.6.5 Subjugate}
\begin{entrylist}
\entry{dändärek}\headword{dändärek}{\pos{Transitive S verb}} {\definition{to control, influence, rule, govern}}
\end{entrylist}

\section*{4.6.6.1 Police}
\begin{entrylist}
\entry{mamos}\headword{mamos}{\pos{Noun}} {\definition{village constable}}
\entry{patrol}\headword{patrol}{\pos{Noun}} {\definition{patrol}}
\entry{polis}\headword{polis}{\pos{Noun}} {\definition{police}}
\end{entrylist}

\section*{4.6.6.3 Represent}
\begin{entrylist}
\entry{eka panypenyang}\headword{eka panypenyang}{\pos{Noun}} {\definition{spokesperson}}
\end{entrylist}

\section*{4.6.7 Region}
\begin{entrylist}
\entry{distrik}\headword{distrik}{\pos{Noun}} {\definition{district}}
\entry{probens}\headword{probens}{\pos{Noun}} {\definition{province}}
\end{entrylist}

\section*{4.6.7.1 Country}
\begin{entrylist}
\entry{Amerika}\headword{Amerika}{\pos{Proper noun}} {\definition{United States of America; America}}
\end{entrylist}

\section*{4.6.7.2 City}
\begin{entrylist}
\entry{kona}\headword{kona}{\pos{Noun}} {\definition{district, section, area (of a settlement)}}
\entry{siti}\headword{siti}{\pos{Noun}} {\definition{city}}
\end{entrylist}

\section*{4.6.7.4 Community}
\begin{entrylist}
\entry{kominiti}\headword{kominiti}{\pos{Noun}} {\definition{community}}
\entry{ma}\headword{ma}{\pos{Noun}} {\definition{community}}
\entry{maduma}\headword{maduma}{\pos{Noun}} {\definition{village}}
\entry{ttängäm}\headword{ttängäm}{\pos{Noun}} {\definition{village}}
\entry{tän}\headword{tän}{\pos{Noun}} {\definition{tribe, nation, people}}
\end{entrylist}

\section*{4.7 Law}
\begin{entrylist}
\entry{gwell}\headword{gwell}{\pos{Noun}} {\definition{rule, law}}
\entry{lo}\headword{lo}{\pos{Noun}} {\definition{law}}
\entry{sabi}\headword{sabi}{\pos{Noun}} {\definition{law, rule}}
\end{entrylist}

\section*{4.7.4 Court of law}
\begin{entrylist}
\entry{kwatt}\headword{kwatt}{\pos{Noun}} {\definition{court}}
\end{entrylist}

\section*{4.7.4.1 Legal personnel}
\begin{entrylist}
\entry{klak}\headword{klak}{\pos{Noun}} {\definition{court clerk}}
\end{entrylist}

\section*{4.7.5.3 Accuse, confront}
\begin{entrylist}
\entry{malmal}\headword{malmal}{\pos{Transitive S verb}} {\definition{to accuse, blame}}
\end{entrylist}

\section*{4.7.6 Judge, render a verdict}
\begin{entrylist}
\entry{pallängkmeny}\headword{pallängkmeny}{\pos{Transitive S verb}} {\definition{to judge}}
\end{entrylist}

\section*{4.7.7 Punish}
\begin{entrylist}
\entry{ddungg}\headword{ddungg}{\pos{Transitive S verb}} {\definition{to decapitate}}
\entry{metmäll}\headword{metmäll}{\pos{Transitive S verb}} {\definition{to beat, flog, hit}}
\entry{panis}\headword{panis}{\pos{Transitive A verb}} {\definition{to punish}}
\end{entrylist}

\section*{4.7.7.3 Imprison}
\begin{entrylist}
\entry{sel}\headword{sel}{\pos{Noun}} {\definition{cell}}
\entry{säremang ma}\headword{säremang ma}{\pos{Noun}} {\definition{prison, jail}}
\entry{zel}\headword{zel}{\pos{Noun}} {\definition{jail}}
\end{entrylist}

\section*{4.7.7.4 Execute}
\begin{entrylist}
\entry{ddänggaddängge}\headword{ddänggaddängge}{\pos{Transitive S verb}} {\definition{to crucify}}
\entry{tärpam}\headword{tärpam}{\pos{Transitive S verb}} {\definition{to crucify}}
\end{entrylist}

\section*{4.7.9 Justice}
\begin{entrylist}
\entry{pallängkmeny}\headword{pallängkmeny}{\pos{Transitive S verb}} {\definition{to judge}}
\end{entrylist}

\section*{4.8 Conflict}
\begin{entrylist}
\entry{kutt gugu}\headword{kutt gugu}{\pos{Noun}} {\definition{type of peace restoration}}
\entry{mäk lla ami doko}\headword{mäk lla ami doko}{\pos{Noun}} {\definition{veteran}}
\entry{mäk lla piar doko}\headword{mäk lla piar doko}{\pos{Noun}} {\definition{veteran}}
\entry{mälla gämäll}\headword{mälla gämäll}{\pos{}} {\definition{adultery (of a man)}}
\entry{mälla wareka}\headword{mälla wareka}{\pos{}} {\definition{jealous of wife}}
\entry{nge ibeny}\headword{nge ibeny}{\pos{Noun}} {\definition{planting a coconut as a gesture of peace}}
\entry{piya gany}\headword{piya gany}{\pos{Noun}} {\definition{planting a piya plant as a gesture of peace}}
\entry{pope mu}\headword{pope mu}{\pos{Noun}} {\definition{uncle payment}}
\entry{äk}\headword{äk}{\pos{Transitive verb}} {\definition{to beat the wife if she doesn't do what the husband says}}
\end{entrylist}

\section*{4.8.2 Fight}
\begin{entrylist}
\entry{eka laem}\headword{eka laem}{\pos{Intransitive S verb}} {\definition{to argue}}
\entry{gäz}\headword{gäz}{\pos{Transitive S verb}} {\definition{to hit, beat}}
\entry{käkllätt}\headword{käkllätt}{\pos{Transitive S verb}} {\definition{to fight, argue}}
\entry{paitt}\headword{paitt}{\pos{Noun}} {\definition{fight}}
\end{entrylist}

\section*{4.8.2.3.1 Ambush}
\begin{entrylist}
\entry{bungg}\headword{bungg}{\pos{Transitive S verb}} {\definition{to ambush}}
\entry{gɨngg}\headword{gɨngg}{\pos{Transitive S verb}} {\definition{to ambush, gang up on, attack in a large group}}
\end{entrylist}

\section*{4.8.2.5 Revenge}
\begin{entrylist}
\entry{mu}\headword{mu}{\pos{Noun}} {\definition{response, reply, answer; repayment, revenge}}
\end{entrylist}

\section*{4.8.2.7 Betray}
\begin{entrylist}
\entry{sänge}\headword{sänge}{\pos{Transitive A verb}} {\definition{to betray}}
\end{entrylist}

\section*{4.8.2.8 Violent}
\begin{entrylist}
\entry{papa}\headword{papa}{\pos{Transitive A verb}} {\definition{to hit, beat}}
\end{entrylist}

\section*{4.8.2.9 Enemy}
\begin{entrylist}
\entry{gidre}\headword{gidre}{\pos{Noun}} {\definition{enemy}}
\end{entrylist}

\section*{4.8.3 War}
\begin{entrylist}
\entry{mäk}\headword{mäk}{\pos{Noun}} {\definition{war}}
\end{entrylist}

\section*{4.8.3.1 Defeat}
\begin{entrylist}
\entry{inmoll}\headword{inmoll}{\pos{Transitive S verb}} {\definition{to step on; vanquish}}
\end{entrylist}

\section*{4.8.3.2 Win}
\begin{entrylist}
\entry{win}\headword{win}{\pos{Noun}} {\definition{win}}
\end{entrylist}

\section*{4.8.3.6.4 Soldier}
\begin{entrylist}
\entry{mäk lla}\headword{mäk lla}{\pos{Noun}} {\definition{soldier}}
\end{entrylist}

\section*{4.8.3.7 Weapon, shoot}
\begin{entrylist}
\entry{baur}\headword{baur}{\pos{Noun}} {\definition{type of spear}}
\entry{bilod}\headword{bilod}{\pos{Noun}} {\definition{type of spear}}
\entry{buidde}\headword{buidde}{\pos{Noun}} {\definition{club (weapon)}}
\entry{buitu}\headword{buitu}{\pos{Noun}} {\definition{stick with a round base}}
\entry{bun}\headword{bun}{\pos{Noun}} {\definition{part of a bow}}
\entry{bägäl}\headword{bägäl}{\pos{Noun}} {\definition{bow}}
\entry{bägälbägäl}\headword{bägälbägäl}{\pos{Noun}} {\definition{small bow}}
\entry{dabit}\headword{dabit}{\pos{Noun}} {\definition{palm spear}}
\entry{ddokddok}\headword{ddokddok}{\pos{Noun}} {\definition{type of spear}}
\entry{ddumbi}\headword{ddumbi}{\pos{Noun}} {\definition{type of spear topped with the claw of a cassowary}}
\entry{diaba}\headword{diaba}{\pos{Noun}} {\definition{type of spear}}
\entry{dompa}\headword{dompa}{\pos{Noun}} {\definition{type of blunt arrow}}
\entry{ebagal}\headword{ebagal}{\pos{Noun}} {\definition{type of spear}}
\entry{gabma bägäl}\headword{gabma bägäl}{\pos{Noun}} {\definition{gun}}
\entry{galib}\headword{galib}{\pos{Noun}} {\definition{type of spear}}
\entry{giri busa}\headword{giri busa}{\pos{Noun}} {\definition{sword}}
\entry{gullme käpang}\headword{gullme käpang}{\pos{Noun}} {\definition{type of spear}}
\entry{gämoe}\headword{gämoe}{\pos{Transitive S verb}} {\definition{to miss}}
\entry{gän}\headword{gän}{\pos{Noun}} {\definition{gun}}
\entry{idoidog}\headword{idoidog}{\pos{Noun}} {\definition{harpoon}}
\entry{iwae}\headword{iwae}{\pos{Noun}} {\definition{type of weapon}}
\entry{kaltakaltamang}\headword{kaltakaltamang}{\pos{Noun}} {\definition{type of spear}}
\entry{kannas}\headword{kannas}{\pos{Noun}} {\definition{type of bow made out of pitpit}}
\entry{karado}\headword{karado}{\pos{Noun}} {\definition{long spear for fishing}}
\entry{klak}\headword{klak}{\pos{Noun}} {\definition{type of harpoon for fishing}}
\entry{koll}\headword{koll}{\pos{Noun}} {\definition{part of a bow}}
\entry{kolldän}\headword{kolldän}{\pos{Transitive S verb}} {\definition{to shoot; stab}}
\entry{kukiny}\headword{kukiny}{\pos{Noun}} {\definition{type of spear made from kukiny grass that is used to hunt birds}}
\entry{kurmirang}\headword{kurmirang}{\pos{Noun}} {\definition{type of spear}}
\entry{käg bänbänang}\headword{käg bänbänang}{\pos{Noun}} {\definition{type of spear}}
\entry{kämser käpang}\headword{kämser käpang}{\pos{Noun}} {\definition{type of arrow}}
\entry{maebo}\headword{maebo}{\pos{Noun}} {\definition{type of spear made from sago}}
\entry{markae bägäl}\headword{markae bägäl}{\pos{Noun}} {\definition{gun}}
\entry{mer}\headword{mer}{\pos{Noun}} {\definition{type of spear}}
\entry{mitmit}\headword{mitmit}{\pos{Noun}} {\definition{blunt axe}}
\entry{nallib}\headword{nallib}{\pos{Noun}} {\definition{type of spear}}
\entry{pakos}\headword{pakos}{\pos{Noun}} {\definition{type of spear}}
\entry{paya}\headword{paya}{\pos{Transitive A verb}} {\definition{to shoot, fire}}
\entry{pengg}\headword{pengg}{\pos{Transitive S verb}} {\definition{to make the killing shot, deliver the final blow}}
\entry{pipi}\headword{pipi}{\pos{Transitive S verb}} {\definition{to shoot, spear}}
\entry{pirik}\headword{pirik}{\pos{Noun}} {\definition{baton, stick}}
\entry{pirngän}\headword{pirngän}{\pos{Transitive S verb}} {\definition{to draw a weapon, take out}}
\entry{pitraempäg}\headword{pitraempäg}{\pos{Transitive verb}} {\definition{to make a failed shot, shoot an arrow that falls}}
\entry{pitt}\headword{pitt}{\pos{Noun}} {\definition{arrowhead hafting string}}
\entry{porak}\headword{porak}{\pos{Noun}} {\definition{type of spear}}
\entry{puku}\headword{puku}{\pos{Noun}} {\definition{type of arrow}}
\entry{rubi}\headword{rubi}{\pos{Noun}} {\definition{type of arrow}}
\entry{sana tätäkang}\headword{sana tätäkang}{\pos{Noun}} {\definition{type of spear}}
\entry{sidompa}\headword{sidompa}{\pos{Noun}} {\definition{type of spear}}
\entry{sära}\headword{sära}{\pos{Noun}} {\definition{end of a bow}}
\entry{tawe ttäp}\headword{tawe ttäp}{\pos{Noun}} {\definition{type of spear}}
\entry{timän}\headword{timän}{\pos{Transitive S verb}} {\definition{to release, fire, shoot (an arrow)}}
\entry{tobäll}\headword{tobäll}{\pos{Noun}} {\definition{long spear shot like an arrow}}
\entry{tobäll käp}\headword{tobäll käp}{\pos{Noun}} {\definition{arrowhead}}
\entry{tobäll käp}\headword{tobäll käp}{\pos{Noun}} {\definition{bullet}}
\entry{tubi}\headword{tubi}{\pos{Noun}} {\definition{type of spear made from sago leaf used to kill birds}}
\entry{tupol}\headword{tupol}{\pos{Noun}} {\definition{type of spear}}
\entry{wadär}\headword{wadär}{\pos{Noun}} {\definition{bowstring}}
\entry{wadär mikel}\headword{wadär mikel}{\pos{Noun}} {\definition{extra bowstring}}
\entry{wan pinga}\headword{wan pinga}{\pos{Noun}} {\definition{metal fish spear}}
\entry{waya}\headword{waya}{\pos{Noun}} {\definition{type of pronged metal spear}}
\entry{wup ttämbällag}\headword{wup ttämbällag}{\pos{Noun}} {\definition{type of spear}}
\entry{yu bägäl}\headword{yu bägäl}{\pos{Noun}} {\definition{gun, firearm}}
\end{entrylist}

\section*{4.8.4 Peace}
\begin{entrylist}
\entry{bädma ibeny}\headword{bädma ibeny}{\pos{Noun}} {\definition{planting a bädma plant as a gesture of peace}}
\entry{lla gugu}\headword{lla gugu}{\pos{Noun}} {\definition{restoring peace after a murder by trading a young girl in the victim's place}}
\entry{mätar onyang}\headword{mätar onyang}{\pos{Noun}} {\definition{peace officer}}
\entry{mätaru}\headword{mätaru}{\pos{Modifier}} {\definition{calm, peaceful, quiet}}
\end{entrylist}

\section*{4.8.4.1 Rebuke}
\begin{entrylist}
\entry{derägmäll}\headword{derägmäll}{\pos{Transitive S verb}} {\definition{to rebuke, scold}}
\end{entrylist}

\section*{4.8.4.9 Reconcile}
\begin{entrylist}
\entry{piya ibeny}\headword{piya ibeny}{\pos{Noun}} {\definition{planting a piya plant as a gesture of peace}}
\end{entrylist}

\section*{4.9 Religion}
\begin{entrylist}
\entry{imomdae ttoenang}\headword{imomdae ttoenang}{\pos{Noun}} {\definition{believer, faithful}}
\entry{mer}\headword{mer}{\pos{Modifier}} {\definition{holy}}
\end{entrylist}

\section*{4.9.1 God}
\begin{entrylist}
\entry{Adi}\headword{Adi}{\pos{Proper noun}} {\definition{God}}
\entry{Godd}\headword{Godd}{\pos{Proper noun}} {\definition{God}}
\entry{anyke}\headword{anyke}{\pos{Noun}} {\definition{spirit}}
\end{entrylist}

\section*{4.9.2 Supernatural being}
\begin{entrylist}
\entry{Adi}\headword{Adi}{\pos{Proper noun}} {\definition{God}}
\entry{enzul}\headword{enzul}{\pos{Noun}} {\definition{angel}}
\end{entrylist}

\section*{4.9.3.1 Sacred writings}
\begin{entrylist}
\entry{baebol}\headword{baebol}{\pos{Noun}} {\definition{Bible}}
\entry{täre buk}\headword{täre buk}{\pos{Noun}} {\definition{Scripture}}
\end{entrylist}

\section*{4.9.4 Miracle, supernatural power}
\begin{entrylist}
\entry{ttowaemang}\headword{ttowaemang}{\pos{Modifier}} {\definition{miraculous}}
\entry{ttowaemang ttoen}\headword{ttowaemang ttoen}{\pos{Noun}} {\definition{miracle}}
\end{entrylist}

\section*{4.9.4.1 Sorcery}
\begin{entrylist}
\entry{mawa}\headword{mawa}{\pos{Noun}} {\definition{magic}}
\entry{midi}\headword{midi}{\pos{Noun}} {\definition{magic type}}
\entry{muro}\headword{muro}{\pos{Noun}} {\definition{magic}}
\entry{omäg}\headword{omäg}{\pos{Noun}} {\definition{magic}}
\entry{omägag}\headword{omägag}{\pos{Noun}} {\definition{magician, sorcerer, fortune-teller}}
\entry{tɨke}\headword{tɨke}{\pos{Noun}} {\definition{magic type}}
\end{entrylist}

\section*{4.9.4.2 Demon possession}
\begin{entrylist}
\entry{saeten}\headword{saeten}{\pos{Proper noun}} {\definition{Satan}}
\end{entrylist}

\section*{4.9.4.3 Bless}
\begin{entrylist}
\entry{mer}\headword{mer}{\pos{Transitive A verb}} {\definition{to bless}}
\end{entrylist}

\section*{4.9.4.4 Curse}
\begin{entrylist}
\entry{inggoemeny}\headword{inggoemeny}{\pos{Intransitive S verb}} {\definition{to blaspheme; insult, mock}}
\entry{märal}\headword{märal}{\pos{Intransitive A verb}} {\definition{to curse, swear}}
\entry{märal}\headword{märal}{\pos{Transitive A verb}} {\definition{to curse}}
\entry{ngänngän}\headword{ngänngän}{\pos{Intransitive S verb}} {\definition{to swear}}
\end{entrylist}

\section*{4.9.5 Practice religion}
\begin{entrylist}
\entry{kawa}\headword{kawa}{\pos{Intransitive A verb}} {\definition{to preach}}
\end{entrylist}

\section*{4.9.5.3 Worship}
\begin{entrylist}
\entry{tubutubu}\headword{tubutubu}{\pos{Modifier}} {\definition{kneeling, on one's knees, on the ground; worshipping}}
\end{entrylist}

\section*{4.9.5.4 Religious ceremony}
\begin{entrylist}
\entry{ali}\headword{ali}{\pos{Noun}} {\definition{conch shell}}
\entry{gwängäm}\headword{gwängäm}{\pos{Noun}} {\definition{sacrifice}}
\entry{kämbäg}\headword{kämbäg}{\pos{Intransitive S verb}} {\definition{to baptize}}
\entry{utt}\headword{utt}{\pos{Noun}} {\definition{conch shell}}
\end{entrylist}

\section*{4.9.5.5 Offering, sacrifice}
\begin{entrylist}
\entry{gal}\headword{gal}{\pos{Noun}} {\definition{food offering}}
\end{entrylist}

\section*{4.9.5.6.1 Taboo}
\begin{entrylist}
\entry{sabi}\headword{sabi}{\pos{Noun}} {\definition{taboo}}
\end{entrylist}

\section*{4.9.6 Heaven, hell}
\begin{entrylist}
\entry{ddapall}\headword{ddapall}{\pos{Noun}} {\definition{heaven}}
\entry{ddapall ma}\headword{ddapall ma}{\pos{Noun}} {\definition{heaven}}
\entry{idd ma}\headword{idd ma}{\pos{Noun}} {\definition{afterlife}}
\entry{pallängkmeny}\headword{pallängkmeny}{\pos{Transitive S verb}} {\definition{to judge}}
\entry{yu ttängäm}\headword{yu ttängäm}{\pos{Noun}} {\definition{hell}}
\end{entrylist}

\section*{4.9.7.1 Religious person}
\begin{entrylist}
\entry{Keriso}\headword{Keriso}{\pos{Proper noun}} {\definition{Christ}}
\entry{eka llɨtɨtang}\headword{eka llɨtɨtang}{\pos{Noun}} {\definition{messenger, prophet}}
\entry{kollmällang}\headword{kollmällang}{\pos{Noun}} {\definition{follower, disciple}}
\entry{kristen}\headword{kristen}{\pos{Modifier}} {\definition{Christian}}
\entry{mänkot}\headword{mänkot}{\pos{Noun}} {\definition{non-believer}}
\entry{pasta}\headword{pasta}{\pos{Noun}} {\definition{pastor}}
\end{entrylist}

\section*{4.9.7.2 Christianity}
\begin{entrylist}
\entry{Iden}\headword{Iden}{\pos{Proper noun}} {\definition{Eden}}
\entry{Keriso}\headword{Keriso}{\pos{Proper noun}} {\definition{Christ}}
\entry{Nazaret}\headword{Nazaret}{\pos{Proper noun}} {\definition{Nazareth}}
\entry{Paelet}\headword{Paelet}{\pos{Proper noun}} {\definition{Pilate}}
\entry{Pita}\headword{Pita}{\pos{Proper noun}} {\definition{male personal name}}
\entry{Yesu}\headword{Yesu}{\pos{Proper noun}} {\definition{Jesus}}
\entry{Zudiya}\headword{Zudiya}{\pos{Proper noun}} {\definition{Judea}}
\entry{Zurusalem}\headword{Zurusalem}{\pos{Proper noun}} {\definition{Jerusalem}}
\entry{ddapall}\headword{ddapall}{\pos{Noun}} {\definition{heaven}}
\entry{ddänggaddängge}\headword{ddänggaddängge}{\pos{Transitive S verb}} {\definition{to crucify}}
\entry{enzul}\headword{enzul}{\pos{Noun}} {\definition{angel}}
\entry{gagäll}\headword{gagäll}{\pos{Noun}} {\definition{sin}}
\entry{krismas}\headword{krismas}{\pos{Noun}} {\definition{Christmas}}
\entry{kristen}\headword{kristen}{\pos{Modifier}} {\definition{Christian}}
\entry{kämbäg}\headword{kämbäg}{\pos{Intransitive S verb}} {\definition{to baptize}}
\entry{pasta}\headword{pasta}{\pos{Noun}} {\definition{pastor}}
\entry{saeten}\headword{saeten}{\pos{Proper noun}} {\definition{Satan}}
\entry{sanawang}\headword{sanawang}{\pos{Noun}} {\definition{parable}}
\entry{sos}\headword{sos}{\pos{Noun}} {\definition{church}}
\entry{tärpamatt}\headword{tärpamatt}{\pos{Noun}} {\definition{cross}}
\entry{yu ttängäm}\headword{yu ttängäm}{\pos{Noun}} {\definition{hell}}
\end{entrylist}

\section*{4.9.7.6 Judaism}
\begin{entrylist}
\entry{Kuddäll Opap}\headword{Kuddäll Opap}{\pos{Proper noun}} {\definition{Passover}}
\entry{Moses}\headword{Moses}{\pos{Proper noun}} {\definition{male personal name}}
\entry{Zudiya}\headword{Zudiya}{\pos{Proper noun}} {\definition{Judea}}
\entry{Zurusalem}\headword{Zurusalem}{\pos{Proper noun}} {\definition{Jerusalem}}
\end{entrylist}

\section*{4.9.8 Religious things}
\begin{entrylist}
\entry{täre}\headword{täre}{\pos{Modifier}} {\definition{holy, sacred}}
\end{entrylist}

\section*{4.9.8.2 Place of worship}
\begin{entrylist}
\entry{ikopse ma}\headword{ikopse ma}{\pos{Noun}} {\definition{church, temple}}
\entry{sos}\headword{sos}{\pos{Noun}} {\definition{church}}
\entry{täre ma}\headword{täre ma}{\pos{Noun}} {\definition{temple}}
\end{entrylist}

\section*{5.1 Household equipment}
\begin{entrylist}
\entry{bägäl odar}\headword{bägäl odar}{\pos{Noun}} {\definition{place for bows and spears}}
\entry{domäll}\headword{domäll}{\pos{Noun}} {\definition{old-style sewn mat made of domäll pandanus}}
\entry{go}\headword{go}{\pos{Noun}} {\definition{drain}}
\entry{kaptte dodro ma}\headword{kaptte dodro ma}{\pos{Noun}} {\definition{washing board}}
\entry{kaptte ittal ma}\headword{kaptte ittal ma}{\pos{Noun}} {\definition{clothes line}}
\entry{ketol}\headword{ketol}{\pos{Noun}} {\definition{kettle}}
\entry{pott}\headword{pott}{\pos{Noun}} {\definition{toilet?}}
\entry{säpalek}\headword{säpalek}{\pos{Noun}} {\definition{type of bag}}
\entry{yu kire}\headword{yu kire}{\pos{Noun}} {\definition{firewood}}
\end{entrylist}

\section*{5.1.1 Furniture}
\begin{entrylist}
\entry{katrekatre}\headword{katrekatre}{\pos{Noun}} {\definition{shelf}}
\end{entrylist}

\section*{5.1.1.1 Table}
\begin{entrylist}
\entry{katrekatre}\headword{katrekatre}{\pos{Noun}} {\definition{table; desk}}
\end{entrylist}

\section*{5.1.1.2 Chair}
\begin{entrylist}
\entry{käg}\headword{käg}{\pos{Property noun}} {\definition{vague container-like thing}}
\entry{tater}\headword{tater}{\pos{Noun}} {\definition{mat}}
\end{entrylist}

\section*{5.1.1.3 Bed}
\begin{entrylist}
\entry{konakone}\headword{konakone}{\pos{Noun}} {\definition{cover, sheet, blanket}}
\end{entrylist}

\section*{5.2 Food}
\begin{entrylist}
\entry{duwem}\headword{duwem}{\pos{Noun}} {\definition{food, meal}}
\entry{llatata}\headword{llatata}{\pos{Noun}} {\definition{Food (such as sago, ripe bananas, and coconut cream, or yams and coconut cream) wrapped in a woven cococnut leaf (with a banana leaf within it) and cooked in a mumu}}
\entry{moko}\headword{moko}{\pos{Noun}} {\definition{taste, flavor}}
\entry{mokoang}\headword{mokoang}{\pos{Modifier}} {\definition{tasty, delicious, flavorful; sweet}}
\entry{otät}\headword{otät}{\pos{Noun}} {\definition{food}}
\entry{ugug}\headword{ugug}{\pos{Transitive S verb}} {\definition{to make mumu}}
\end{entrylist}

\section*{5.2.1 Food preparation}
\begin{entrylist}
\entry{kisin}\headword{kisin}{\pos{Noun}} {\definition{kitchen}}
\entry{kul}\headword{kul}{\pos{Transitive A verb}} {\definition{to smash food}}
\entry{otät yu ma}\headword{otät yu ma}{\pos{Noun}} {\definition{kitchen}}
\end{entrylist}

\section*{5.2.1.1 Cooking methods}
\begin{entrylist}
\entry{binzeg}\headword{binzeg}{\pos{Transitive S verb}} {\definition{to heat, warm}}
\entry{bir}\headword{bir}{\pos{Transitive A verb}} {\definition{to roast}}
\entry{därunggu}\headword{därunggu}{\pos{Noun}} {\definition{bamboo tube}}
\entry{gonagone}\headword{gonagone}{\pos{Transitive S verb}} {\definition{to cook}}
\entry{gonangg}\headword{gonangg}{\pos{Transitive S verb}} {\definition{to cook}}
\entry{kodowa}\headword{kodowa}{\pos{Noun}} {\definition{dish consisting of sago cooked in leaves on the fire}}
\entry{källakälle}\headword{källakälle}{\pos{Transitive S verb}} {\definition{to scrape food off the fire, e.g. banana, taro, yam}}
\entry{mattgal}\headword{mattgal}{\pos{Transitive S verb}} {\definition{to put in fire}}
\entry{mattmett}\headword{mattmett}{\pos{Transitive S verb}} {\definition{to put in oven}}
\entry{nono}\headword{nono}{\pos{Transitive A verb}} {\definition{to crumble, make soft}}
\entry{säspen}\headword{säspen}{\pos{Transitive A verb}} {\definition{to boil}}
\entry{ttägäll}\headword{ttägäll}{\pos{Noun}} {\definition{mumu (oven made in the ground with fire, stones, leaves, and bark; the first mumus were made out of termite mounds)}}
\entry{ttäm}\headword{ttäm}{\pos{Transitive S verb}} {\definition{to burn; heat on a fire}}
\entry{tärpa}\headword{tärpa}{\pos{Transitive A verb}} {\definition{to overcook, burn}}
\entry{ugug}\headword{ugug}{\pos{Transitive S verb}} {\definition{to make mumu}}
\entry{yu}\headword{yu}{\pos{Transitive A verb}} {\definition{to cook over fire}}
\end{entrylist}

\section*{5.2.1.2 Steps in food preparation}
\begin{entrylist}
\entry{bänybäny}\headword{bänybäny}{\pos{Transitive S verb}} {\definition{to cut, slice (flesh)}}
\entry{kälbae}\headword{kälbae}{\pos{Transitive S verb}} {\definition{to singe (use brief heat to remove hair or down)}}
\entry{kängkäm}\headword{kängkäm}{\pos{Transitive S verb}} {\definition{to squeeze, press}}
\entry{laem}\headword{laem}{\pos{Transitive S verb}} {\definition{to roll, wrap}}
\entry{llätät}\headword{llätät}{\pos{Transitive S verb}} {\definition{to get rid of oven stones}}
\entry{nyägae}\headword{nyägae}{\pos{Intransitive S verb}} {\definition{to stir}}
\entry{päddab}\headword{päddab}{\pos{Intransitive S verb}} {\definition{to be cooked, be done}}
\entry{yid}\headword{yid}{\pos{Noun}} {\definition{liquid extracted from a plant}}
\end{entrylist}

\section*{5.2.1.2.1 Remove shell, skin}
\begin{entrylist}
\entry{ddäddäg}\headword{ddäddäg}{\pos{Transitive S verb}} {\definition{to peel, remove}}
\entry{gallgell}\headword{gallgell}{\pos{Intransitive S verb}} {\definition{to remove leaves, take leaves off; remove skin, skin}}
\entry{gllaglle}\headword{gllaglle}{\pos{Transitive S verb}} {\definition{to skin, remove skin}}
\entry{kallkell}\headword{kallkell}{\pos{Transitive S verb}} {\definition{to deleaf a plant to reveal the shoot, or to take the skin off}}
\entry{kokop}\headword{kokop}{\pos{Transitive S verb}} {\definition{to peel, skin, husk}}
\entry{patt}\headword{patt}{\pos{Noun}} {\definition{coconut husking stick}}
\entry{spun}\headword{spun}{\pos{Transitive S verb}} {\definition{to remove an outer layer}}
\end{entrylist}

\section*{5.2.1.3 Cooking utensil}
\begin{entrylist}
\entry{bir}\headword{bir}{\pos{Noun}} {\definition{spit, skewer}}
\entry{därunggu}\headword{därunggu}{\pos{Noun}} {\definition{bamboo tube}}
\entry{komony}\headword{komony}{\pos{Noun}} {\definition{tongs}}
\entry{pane}\headword{pane}{\pos{Noun}} {\definition{pot}}
\entry{säspen}\headword{säspen}{\pos{Noun}} {\definition{pot, saucepan}}
\end{entrylist}

\section*{5.2.1.4 Food storage}
\begin{entrylist}
\entry{bumo}\headword{bumo}{\pos{Noun}} {\definition{tucker bag}}
\entry{zazaba}\headword{zazaba}{\pos{Noun}} {\definition{type of bag}}
\end{entrylist}

\section*{5.2.1.5 Serve food}
\begin{entrylist}
\entry{blläg}\headword{blläg}{\pos{Transitive S verb}} {\definition{to serve}}
\entry{pilatt}\headword{pilatt}{\pos{Noun}} {\definition{plate, dish}}
\end{entrylist}

\section*{5.2.2 Eat}
\begin{entrylist}
\entry{duwem}\headword{duwem}{\pos{Transitive/Intransitive A verb}} {\definition{to eat}}
\entry{duwem}\headword{duwem}{\pos{Noun}} {\definition{food, meal}}
\entry{mändmänd}\headword{mändmänd}{\pos{Transitive S verb}} {\definition{to feed}}
\entry{ngonenngonen duwem}\headword{ngonenngonen duwem}{\pos{Transitive/Intransitive A verb}} {\definition{to scarf down, eat very quickly}}
\entry{ngänyngäny}\headword{ngänyngäny}{\pos{Transitive S verb}} {\definition{to swallow}}
\entry{otät}\headword{otät}{\pos{Transitive S verb}} {\definition{to eat}}
\entry{tlläpmälltlläpmäll}\headword{tlläpmälltlläpmäll}{\pos{Adverb}} {\definition{nibbling}}
\end{entrylist}

\section*{5.2.2.1 Bite, chew}
\begin{entrylist}
\entry{kaekep}\headword{kaekep}{\pos{Transitive S verb}} {\definition{to chew}}
\end{entrylist}

\section*{5.2.2.2 Meal}
\begin{entrylist}
\entry{ag duwem}\headword{ag duwem}{\pos{Noun}} {\definition{breakfast}}
\entry{duwem}\headword{duwem}{\pos{Noun}} {\definition{food, meal}}
\entry{ebdo duwem}\headword{ebdo duwem}{\pos{Noun}} {\definition{lunch}}
\entry{toto duwem}\headword{toto duwem}{\pos{Noun}} {\definition{dinner}}
\entry{yäbäd tuktuk duwem}\headword{yäbäd tuktuk duwem}{\pos{Noun}} {\definition{lunch}}
\end{entrylist}

\section*{5.2.2.3 Feast}
\begin{entrylist}
\entry{duwemduwem}\headword{duwemduwem}{\pos{Noun}} {\definition{feast, fellowship meal}}
\entry{täre}\headword{täre}{\pos{Noun}} {\definition{feast}}
\end{entrylist}

\section*{5.2.2.5 Hungry, thirsty}
\begin{entrylist}
\entry{ddäddäg}\headword{ddäddäg}{\pos{Noun}} {\definition{hunger for meat}}
\entry{mäse}\headword{mäse}{\pos{Modifier}} {\definition{not yet full, unsatisfied}}
\entry{otät}\headword{otät}{\pos{Noun}} {\definition{hunger}}
\entry{otät ngänaeka ttoen}\headword{otät ngänaeka ttoen}{\pos{Noun}} {\definition{starvation}}
\end{entrylist}

\section*{5.2.2.7 Drink}
\begin{entrylist}
\entry{kakoll}\headword{kakoll}{\pos{Noun}} {\definition{cup}}
\entry{nane}\headword{nane}{\pos{Transitive S verb}} {\definition{to drink}}
\entry{trongoe}\headword{trongoe}{\pos{Transitive S verb}} {\definition{to check a bucket or container for water}}
\end{entrylist}

\section*{5.2.2.8 Eating utensil}
\begin{entrylist}
\entry{dägmar}\headword{dägmar}{\pos{Noun}} {\definition{spoon}}
\entry{kakoll}\headword{kakoll}{\pos{Noun}} {\definition{dish}}
\entry{spun}\headword{spun}{\pos{Noun}} {\definition{spoon}}
\end{entrylist}

\section*{5.2.2.9 Fast, not eat}
\begin{entrylist}
\entry{kok tärangg}\headword{kok tärangg}{\pos{Noun}} {\definition{fast}}
\end{entrylist}

\section*{5.2.3 Types of food}
\begin{entrylist}
\entry{llatata}\headword{llatata}{\pos{Noun}} {\definition{Food (such as sago, ripe bananas, and coconut cream, or yams and coconut cream) wrapped in a woven cococnut leaf (with a banana leaf within it) and cooked in a mumu}}
\entry{nge id}\headword{nge id}{\pos{Noun}} {\definition{coconut cream}}
\end{entrylist}

\section*{5.2.3.1 Food from plants}
\begin{entrylist}
\entry{allko wallägnewallägnen}\headword{allko wallägnewallägnen}{\pos{Noun}} {\definition{type of sago}}
\entry{bisel}\headword{bisel}{\pos{Noun}} {\definition{type of sago that grows tall and wide}}
\entry{biye}\headword{biye}{\pos{Noun}} {\definition{taro}}
\entry{bollga}\headword{bollga}{\pos{Noun}} {\definition{type of sago}}
\entry{bänzibänzi}\headword{bänzibänzi}{\pos{Noun}} {\definition{type of sago}}
\entry{bärät}\headword{bärät}{\pos{Noun}} {\definition{type of small yam}}
\entry{dem}\headword{dem}{\pos{Noun}} {\definition{type of sago used for paints}}
\entry{gaora}\headword{gaora}{\pos{Noun}} {\definition{type of sago}}
\entry{gongglem}\headword{gongglem}{\pos{Noun}} {\definition{immature coconut that is partially solid inside}}
\entry{kunur}\headword{kunur}{\pos{Noun}} {\definition{corn, maize}}
\entry{mai}\headword{mai}{\pos{Noun}} {\definition{type of sago}}
\entry{mutae}\headword{mutae}{\pos{Noun}} {\definition{type of yam with a yellow interior, no thorns, and a vine that grows clockwise}}
\entry{mäga}\headword{mäga}{\pos{Noun}} {\definition{type of sago}}
\entry{nge}\headword{nge}{\pos{Noun}} {\definition{coconut}}
\entry{nge dɨdɨr}\headword{nge dɨdɨr}{\pos{Noun}} {\definition{fully dry coconut}}
\entry{oll}\headword{oll}{\pos{Noun}} {\definition{sugarcane}}
\entry{olmopäga}\headword{olmopäga}{\pos{Noun}} {\definition{type of sago}}
\entry{pamker}\headword{pamker}{\pos{Noun}} {\definition{pumpkin}}
\entry{sana}\headword{sana}{\pos{Noun}} {\definition{sago}}
\entry{sanasana}\headword{sanasana}{\pos{Noun}} {\definition{type of edible sago}}
\entry{sili}\headword{sili}{\pos{Noun}} {\definition{chili}}
\entry{sosoga}\headword{sosoga}{\pos{Noun}} {\definition{type of sago}}
\entry{sɨmellkom}\headword{sɨmellkom}{\pos{Noun}} {\definition{type of sago}}
\entry{tamatama}\headword{tamatama}{\pos{Noun}} {\definition{bean}}
\entry{ttɨp}\headword{ttɨp}{\pos{Noun}} {\definition{type of sago}}
\entry{tätkea}\headword{tätkea}{\pos{Noun}} {\definition{type of sago}}
\entry{uttang ttatta}\headword{uttang ttatta}{\pos{Noun}} {\definition{type of sago}}
\entry{wayati}\headword{wayati}{\pos{Noun}} {\definition{watermelon}}
\entry{wit}\headword{wit}{\pos{Noun}} {\definition{wheat}}
\entry{yuddädda}\headword{yuddädda}{\pos{Noun}} {\definition{type of palm with branches used for armbands}}
\entry{yure}\headword{yure}{\pos{Noun}} {\definition{type of sago}}
\end{entrylist}

\section*{5.2.3.1.2 Food from fruit}
\begin{entrylist}
\entry{as}\headword{as}{\pos{Noun}} {\definition{type of introduced banana}}
\entry{asip}\headword{asip}{\pos{Noun}} {\definition{type of introduced banana}}
\entry{bebe}\headword{bebe}{\pos{Noun}} {\definition{type of pandanus with long fruit (~2 feet)}}
\entry{bikme}\headword{bikme}{\pos{Noun}} {\definition{type of palm tree with hanging, poisonous yellow and green fruits that can be eaten after being buried by the creek for up to 2 years and then cooked on the fire}}
\entry{buwo}\headword{buwo}{\pos{Noun}} {\definition{type of native banana}}
\entry{bälläg}\headword{bälläg}{\pos{Noun}} {\definition{type of introduced banana}}
\entry{dauma}\headword{dauma}{\pos{Noun}} {\definition{type of introduced banana}}
\entry{därmir}\headword{därmir}{\pos{Noun}} {\definition{type of introduced banana}}
\entry{därängbun}\headword{därängbun}{\pos{Noun}} {\definition{type of pandanus with a curved fruit shaped like a dog's head}}
\entry{guwaba}\headword{guwaba}{\pos{Noun}} {\definition{guava tree; water steeped with its leaves is used to wash sores}}
\entry{gärep}\headword{gärep}{\pos{Noun}} {\definition{grape}}
\entry{karita}\headword{karita}{\pos{Noun}} {\definition{type of introduced banana}}
\entry{kokol}\headword{kokol}{\pos{Noun}} {\definition{type of introduced banana}}
\entry{kollko}\headword{kollko}{\pos{Noun}} {\definition{breadfruit}}
\entry{kud}\headword{kud}{\pos{Noun}} {\definition{type of pandanus with fat triangular fruit}}
\entry{käp}\headword{käp}{\pos{Noun}} {\definition{fruit}}
\entry{lla up}\headword{lla up}{\pos{Noun}} {\definition{type of introduced banana}}
\entry{mab}\headword{mab}{\pos{Noun}} {\definition{pandanus}}
\entry{madura}\headword{madura}{\pos{Noun}} {\definition{type of introduced banana}}
\entry{maigag}\headword{maigag}{\pos{Noun}} {\definition{type of introduced banana}}
\entry{maiwa}\headword{maiwa}{\pos{Noun}} {\definition{type of pandanus with a long, smooth fruit cooked in mumu}}
\entry{mameat}\headword{mameat}{\pos{Noun}} {\definition{papaya, pawpaw}}
\entry{mamkiel}\headword{mamkiel}{\pos{Noun}} {\definition{type of native banana}}
\entry{mandri}\headword{mandri}{\pos{Noun}} {\definition{cultivated lemon tree}}
\entry{manggo}\headword{manggo}{\pos{Noun}} {\definition{mango tree}}
\entry{mare}\headword{mare}{\pos{Noun}} {\definition{type of pandanus}}
\entry{mekewa}\headword{mekewa}{\pos{Noun}} {\definition{type of introduced banana}}
\entry{miriwa}\headword{miriwa}{\pos{Noun}} {\definition{type of pandanus}}
\entry{mislok}\headword{mislok}{\pos{Noun}} {\definition{type of introduced banana}}
\entry{mollok}\headword{mollok}{\pos{Noun}} {\definition{type of introduced banana}}
\entry{momolltätän}\headword{momolltätän}{\pos{Noun}} {\definition{type of introduced banana}}
\entry{mätemäte}\headword{mätemäte}{\pos{Noun}} {\definition{type of introduced banana}}
\entry{mɨka}\headword{mɨka}{\pos{Noun}} {\definition{type of introduced banana}}
\entry{olib}\headword{olib}{\pos{Noun}} {\definition{olive}}
\entry{padiem}\headword{padiem}{\pos{Noun}} {\definition{type of introduced banana}}
\entry{panya}\headword{panya}{\pos{Noun}} {\definition{pineapple}}
\entry{pepeb wup}\headword{pepeb wup}{\pos{Noun}} {\definition{type of introduced banana}}
\entry{popell}\headword{popell}{\pos{Noun}} {\definition{type of introduced banana}}
\entry{pälläk}\headword{pälläk}{\pos{Noun}} {\definition{type of introduced banana}}
\entry{sakar}\headword{sakar}{\pos{Noun}} {\definition{type of edible pandanus}}
\entry{samoa}\headword{samoa}{\pos{Noun}} {\definition{type of introduced banana}}
\entry{tibra}\headword{tibra}{\pos{Noun}} {\definition{type of native banana}}
\entry{tomato}\headword{tomato}{\pos{Noun}} {\definition{tomato}}
\entry{up}\headword{up}{\pos{Noun}} {\definition{banana}}
\entry{wakata}\headword{wakata}{\pos{Noun}} {\definition{type of introduced banana}}
\entry{wana}\headword{wana}{\pos{Noun}} {\definition{type of introduced banana}}
\entry{wirog}\headword{wirog}{\pos{Noun}} {\definition{type of native banana}}
\entry{wizarab}\headword{wizarab}{\pos{Noun}} {\definition{type of pandanus with red fruit}}
\entry{zib mäka}\headword{zib mäka}{\pos{Noun}} {\definition{type of introduced banana}}
\end{entrylist}

\section*{5.2.3.1.3 Food from vegetables}
\begin{entrylist}
\entry{lläkäm}\headword{lläkäm}{\pos{Noun}} {\definition{mushroom}}
\end{entrylist}

\section*{5.2.3.1.4 Food from leaves}
\begin{entrylist}
\entry{mompel}\headword{mompel}{\pos{Noun}} {\definition{aibika}}
\end{entrylist}

\section*{5.2.3.1.5 Food from roots}
\begin{entrylist}
\entry{aengap}\headword{aengap}{\pos{Noun}} {\definition{type of big yam with a white interior and no thorns}}
\entry{apapi bärät}\headword{apapi bärät}{\pos{Noun}} {\definition{type of yam}}
\entry{atrepo}\headword{atrepo}{\pos{Noun}} {\definition{taro type}}
\entry{bab}\headword{bab}{\pos{Noun}} {\definition{type of small yam with a white interior}}
\entry{babdu}\headword{babdu}{\pos{Noun}} {\definition{type of taro}}
\entry{bagen}\headword{bagen}{\pos{Noun}} {\definition{type of big taro}}
\entry{ballo bällabällott}\headword{ballo bällabällott}{\pos{Noun}} {\definition{type of big taro}}
\entry{banggo}\headword{banggo}{\pos{Noun}} {\definition{type of long yam with a white interior, thorns, and no hair}}
\entry{bazere}\headword{bazere}{\pos{Noun}} {\definition{type of purple yam with hairs and no thorns}}
\entry{begere}\headword{begere}{\pos{Noun}} {\definition{type of long purple yam}}
\entry{bible}\headword{bible}{\pos{Noun}} {\definition{type of big taro}}
\entry{bog}\headword{bog}{\pos{Noun}} {\definition{type of taro}}
\entry{bogobogo}\headword{bogobogo}{\pos{Noun}} {\definition{type of small yam with a pure white interior}}
\entry{bulwem}\headword{bulwem}{\pos{Noun}} {\definition{type of big yam with a white and light purple interior and no hairs}}
\entry{bädde}\headword{bädde}{\pos{Noun}} {\definition{type of big taro}}
\entry{dangkälmang}\headword{dangkälmang}{\pos{Noun}} {\definition{type of medium-sized, long yam with a white interior and thorns}}
\entry{dirindi}\headword{dirindi}{\pos{Noun}} {\definition{type of large yam with a white interior and thorns; with or without hairs}}
\entry{dirom käp}\headword{dirom käp}{\pos{Noun}} {\definition{type of small taro}}
\entry{duwie ku}\headword{duwie ku}{\pos{Noun}} {\definition{type of big purple yam}}
\entry{dädär}\headword{dädär}{\pos{Noun}} {\definition{type of big taro}}
\entry{dändak}\headword{dändak}{\pos{Noun}} {\definition{type of purple yam with purple skin and no thorns or hairs}}
\entry{ewembe}\headword{ewembe}{\pos{Noun}} {\definition{type of very big yam with a white interior, thorns, and hairs}}
\entry{galbe}\headword{galbe}{\pos{Noun}} {\definition{purple/greater yam}}
\entry{galigali}\headword{galigali}{\pos{Noun}} {\definition{type of small taro}}
\entry{girag dirindi}\headword{girag dirindi}{\pos{Noun}} {\definition{type of yam}}
\entry{giritai}\headword{giritai}{\pos{Noun}} {\definition{type of long yam with a white interior and hairs}}
\entry{gugu}\headword{gugu}{\pos{Noun}} {\definition{type of big taro}}
\entry{gwazi}\headword{gwazi}{\pos{Noun}} {\definition{type of big taro}}
\entry{kabär}\headword{kabär}{\pos{Noun}} {\definition{type of big taro}}
\entry{kakud}\headword{kakud}{\pos{Noun}} {\definition{type of thin, curved, and long yam without thorns}}
\entry{kallekalle}\headword{kallekalle}{\pos{Noun}} {\definition{type of yam with a white interior, red skin, and hairs}}
\entry{kargeam}\headword{kargeam}{\pos{Noun}} {\definition{type of big taro}}
\entry{kerema}\headword{kerema}{\pos{Noun}} {\definition{type of taro}}
\entry{ketmar}\headword{ketmar}{\pos{Noun}} {\definition{two yams hanging from a stick in the center of a yam counting pile}}
\entry{komo}\headword{komo}{\pos{Noun}} {\definition{ginger}}
\entry{konskak}\headword{konskak}{\pos{Noun}} {\definition{type of big taro}}
\entry{konzar}\headword{konzar}{\pos{Noun}} {\definition{type of small yam with a white interior, hairs, and small thorns}}
\entry{kunuwälläb}\headword{kunuwälläb}{\pos{Noun}} {\definition{type of big taro}}
\entry{kuram}\headword{kuram}{\pos{Noun}} {\definition{type of long yam with a white interior, hairs, and thorns}}
\entry{kutae}\headword{kutae}{\pos{Noun}} {\definition{type of small yam}}
\entry{kwae}\headword{kwae}{\pos{Noun}} {\definition{type of yam with a white or purple interior}}
\entry{kwas}\headword{kwas}{\pos{Noun}} {\definition{type of taro kongkong}}
\entry{kädebällag mälla}\headword{kädebällag mälla}{\pos{Noun}} {\definition{type of big taro}}
\entry{käman}\headword{käman}{\pos{Noun}} {\definition{traditional type of cassava}}
\entry{kän}\headword{kän}{\pos{Noun}} {\definition{type of big, round yam with a white interior and thorns}}
\entry{käpre}\headword{käpre}{\pos{Noun}} {\definition{type of big yam with a white interior, thorns, and few hairs}}
\entry{manika}\headword{manika}{\pos{Noun}} {\definition{cassava}}
\entry{maribärät}\headword{maribärät}{\pos{Noun}} {\definition{type of long yam with a white interior and hairs}}
\entry{markaebärät}\headword{markaebärät}{\pos{Noun}} {\definition{type of medium-sized yam with a white interior, thorns, and no hairs}}
\entry{metar}\headword{metar}{\pos{Noun}} {\definition{type of long yam with a white interior, white skin, and hairs}}
\entry{misituryam}\headword{misituryam}{\pos{Noun}} {\definition{type of long yam with a white or purple interior}}
\entry{motom}\headword{motom}{\pos{Noun}} {\definition{type of small taro}}
\entry{mugbusu}\headword{mugbusu}{\pos{Noun}} {\definition{type of taro}}
\entry{mällätgugu}\headword{mällätgugu}{\pos{Noun}} {\definition{type of taro}}
\entry{mämbär}\headword{mämbär}{\pos{Noun}} {\definition{type of big, round yam with a white interior, hairs, and no thorns}}
\entry{mätta}\headword{mätta}{\pos{Noun}} {\definition{lesser yam}}
\entry{nae}\headword{nae}{\pos{Noun}} {\definition{sweet potato}}
\entry{ngolo}\headword{ngolo}{\pos{Noun}} {\definition{type of big yam with a purple interior, thorns, and hairs}}
\entry{osne}\headword{osne}{\pos{Noun}} {\definition{type of small taro}}
\entry{panggopanggo}\headword{panggopanggo}{\pos{Noun}} {\definition{type of round yam with a white interior and hairs}}
\entry{pap}\headword{pap}{\pos{Noun}} {\definition{type of round mutae yam}}
\entry{pat}\headword{pat}{\pos{Noun}} {\definition{type of taro that is eaten from the suckers}}
\entry{penganyäm}\headword{penganyäm}{\pos{Noun}} {\definition{type of yam}}
\entry{pollon bällabollott}\headword{pollon bällabollott}{\pos{Noun}} {\definition{type of big taro}}
\entry{pom}\headword{pom}{\pos{Noun}} {\definition{type of long yam with a white interior, thorns, and hairs}}
\entry{ponganem}\headword{ponganem}{\pos{Noun}} {\definition{type of small yam with a white interior, hairs, and thorns}}
\entry{pongoll}\headword{pongoll}{\pos{Noun}} {\definition{type of yam with a white interior and no hairs or thorns}}
\entry{purta}\headword{purta}{\pos{Noun}} {\definition{six groups of six yams}}
\entry{pägamän}\headword{pägamän}{\pos{Noun}} {\definition{type of yam}}
\entry{pänmällang mälla}\headword{pänmällang mälla}{\pos{Noun}} {\definition{type of big taro}}
\entry{retam}\headword{retam}{\pos{Noun}} {\definition{type of big yam}}
\entry{rubi}\headword{rubi}{\pos{Noun}} {\definition{type of long yam with a white interior, white skin, and no thorns}}
\entry{sangapawi}\headword{sangapawi}{\pos{Noun}} {\definition{type of big, round yam with a white interior, white or red skin, hairs, and no thorns}}
\entry{sapebllabllot}\headword{sapebllabllot}{\pos{Noun}} {\definition{type of taro}}
\entry{sare}\headword{sare}{\pos{Noun}} {\definition{type of big taro}}
\entry{saus}\headword{saus}{\pos{Noun}} {\definition{type of yam with a white or purple interior, thorns, and no hairs}}
\entry{sawis}\headword{sawis}{\pos{Noun}} {\definition{type of small yam with a long shape and purple interior}}
\entry{sigip}\headword{sigip}{\pos{Noun}} {\definition{type of big taro}}
\entry{sipik}\headword{sipik}{\pos{Noun}} {\definition{type of large yam with a white and purple interior}}
\entry{suga galbe}\headword{suga galbe}{\pos{Noun}} {\definition{type of large yam with a white interior}}
\entry{sulut}\headword{sulut}{\pos{Noun}} {\definition{type of taro}}
\entry{sägäsägäd manika}\headword{sägäsägäd manika}{\pos{Noun}} {\definition{yellow cassava}}
\entry{sänd}\headword{sänd}{\pos{Noun}} {\definition{type of big yam with a white interior and few hairs}}
\entry{säne}\headword{säne}{\pos{Noun}} {\definition{type of yam}}
\entry{tap}\headword{tap}{\pos{Noun}} {\definition{type of big yam with a white interior, thorns, and hairs}}
\entry{tarmekälla}\headword{tarmekälla}{\pos{Noun}} {\definition{type of taro}}
\entry{ttall mätta}\headword{ttall mätta}{\pos{Noun}} {\definition{type of big yam with a white interior and few hairs}}
\entry{ttattang}\headword{ttattang}{\pos{Noun}} {\definition{type of big, round yam with a white interior}}
\entry{ttɨp}\headword{ttɨp}{\pos{Noun}} {\definition{type of yam with a white interior}}
\entry{tutuaram}\headword{tutuaram}{\pos{Noun}} {\definition{type of taro}}
\entry{täbom}\headword{täbom}{\pos{Noun}} {\definition{type of small yam with a white interior, hairs, and thorns}}
\entry{tämallang mälla}\headword{tämallang mälla}{\pos{Noun}} {\definition{type of big taro}}
\entry{tämani}\headword{tämani}{\pos{Noun}} {\definition{type of large yam with a white or white and red interior}}
\entry{täme käp}\headword{täme käp}{\pos{Noun}} {\definition{type of round yam with a white interior, red skin, and hairs}}
\entry{tärpae}\headword{tärpae}{\pos{Noun}} {\definition{type of yam}}
\entry{ubrattäka}\headword{ubrattäka}{\pos{Noun}} {\definition{type of yam a red interior}}
\entry{wadär käp}\headword{wadär käp}{\pos{Noun}} {\definition{type of big taro}}
\entry{wawonai}\headword{wawonai}{\pos{Noun}} {\definition{type of long yam with a hooked end, white interior, and hairs}}
\entry{wod}\headword{wod}{\pos{Noun}} {\definition{type of long yam with a white interior and few hairs}}
\entry{wärenzbag}\headword{wärenzbag}{\pos{Noun}} {\definition{type of taro}}
\entry{wätaote}\headword{wätaote}{\pos{Noun}} {\definition{type of taro}}
\entry{yaru}\headword{yaru}{\pos{Noun}} {\definition{type of yam with a purple interior, hairs, and thorns}}
\entry{yaryem}\headword{yaryem}{\pos{Noun}} {\definition{type of big yam with a white interior, hairs, and thorns}}
\entry{yaul}\headword{yaul}{\pos{Noun}} {\definition{type of long yam with a white interior and few hairs}}
\entry{yäbäd källa}\headword{yäbäd källa}{\pos{Noun}} {\definition{type of big taro}}
\entry{yämän}\headword{yämän}{\pos{Noun}} {\definition{type of big tuber with a reddish interior (not a yam)}}
\entry{yɨb}\headword{yɨb}{\pos{Noun}} {\definition{type of yam}}
\entry{zarmeny}\headword{zarmeny}{\pos{Noun}} {\definition{type of long yam with a white interior, hairs, and no thorns}}
\end{entrylist}

\section*{5.2.3.2 Food from animals}
\begin{entrylist}
\entry{känär}\headword{känär}{\pos{Noun}} {\definition{type of edible grub found in the bush}}
\entry{kätt}\headword{kätt}{\pos{Noun}} {\definition{bivalve; shell of a mollusc}}
\entry{pidroll}\headword{pidroll}{\pos{Noun}} {\definition{black palm weevil}}
\entry{potne kätt}\headword{potne kätt}{\pos{Noun}} {\definition{bivalve type}}
\entry{winy}\headword{winy}{\pos{Noun}} {\definition{honey}}
\entry{winyteya}\headword{winyteya}{\pos{Noun}} {\definition{honeycomb}}
\end{entrylist}

\section*{5.2.3.2.1 Meat}
\begin{entrylist}
\entry{ddäddäg}\headword{ddäddäg}{\pos{Noun}} {\definition{edible animal, game, meat}}
\entry{midd}\headword{midd}{\pos{Noun}} {\definition{meat}}
\entry{tätän}\headword{tätän}{\pos{Noun}} {\definition{rib, side, flank}}
\end{entrylist}

\section*{5.2.3.3.1 Sugar}
\begin{entrylist}
\entry{loli}\headword{loli}{\pos{Noun}} {\definition{candy}}
\entry{suga}\headword{suga}{\pos{Noun}} {\definition{sugar}}
\end{entrylist}

\section*{5.2.3.3.2 Salt}
\begin{entrylist}
\entry{bile}\headword{bile}{\pos{Noun}} {\definition{salt}}
\entry{solt}\headword{solt}{\pos{Noun}} {\definition{salt}}
\end{entrylist}

\section*{5.2.3.3.4 Leaven}
\begin{entrylist}
\entry{ist}\headword{ist}{\pos{Noun}} {\definition{yeast}}
\end{entrylist}

\section*{5.2.3.4 Prepared food}
\begin{entrylist}
\entry{kabadu}\headword{kabadu}{\pos{Noun}} {\definition{traditional dish consisting of coconut cream, sago, meat, and tulip greens}}
\entry{kodowa}\headword{kodowa}{\pos{Noun}} {\definition{dish consisting of sago cooked in leaves on the fire}}
\entry{llatata}\headword{llatata}{\pos{Noun}} {\definition{Food (such as sago, ripe bananas, and coconut cream, or yams and coconut cream) wrapped in a woven cococnut leaf (with a banana leaf within it) and cooked in a mumu}}
\entry{mubine}\headword{mubine}{\pos{Noun}} {\definition{dish consisting of food (such as banana, yam, or sweet potato) with coconut cream. up mubine, mätta mubine, nai mubine}}
\entry{popell}\headword{popell}{\pos{Noun}} {\definition{dish consisting of ants and various types of bananas (incl. popell)}}
\entry{porma}\headword{porma}{\pos{Noun}} {\definition{traditional dish consisting of meat on top of sago cooked in sago or banana leaves}}
\entry{ttäkäll}\headword{ttäkäll}{\pos{Noun}} {\definition{portion of yams mixed with coconut}}
\end{entrylist}

\section*{5.2.3.6 Beverage}
\begin{entrylist}
\entry{ine kutt}\headword{ine kutt}{\pos{Noun}} {\definition{bucket, water container}}
\entry{nge ine}\headword{nge ine}{\pos{Noun}} {\definition{coconut water}}
\end{entrylist}

\section*{5.2.3.7 Alcoholic beverage}
\begin{entrylist}
\entry{gagäll ine}\headword{gagäll ine}{\pos{Noun}} {\definition{beer}}
\entry{ine}\headword{ine}{\pos{Noun}} {\definition{alcoholic beverage}}
\entry{ine konkonang}\headword{ine konkonang}{\pos{Noun}} {\definition{beer}}
\entry{kae ine}\headword{kae ine}{\pos{Noun}} {\definition{wine}}
\end{entrylist}

\section*{5.2.3.7.2 Drunk}
\begin{entrylist}
\entry{konkon}\headword{konkon}{\pos{Modifier}} {\definition{intoxicated, intoxicating, consciousness-altering, drunk}}
\end{entrylist}

\section*{5.2.4 Tobacco}
\begin{entrylist}
\entry{bore}\headword{bore}{\pos{Noun}} {\definition{traditional bamboo pipe for smoking tobacco}}
\entry{sokpa}\headword{sokpa}{\pos{Noun}} {\definition{tobacco}}
\entry{sokpa kllokllop}\headword{sokpa kllokllop}{\pos{Noun}} {\definition{notebook-sized mat woven of tobacco}}
\entry{sokpa llaweatt}\headword{sokpa llaweatt}{\pos{Noun}} {\definition{tobacco woven like a rope}}
\entry{turku}\headword{turku}{\pos{Noun}} {\definition{thinner piece used with}}
\end{entrylist}

\section*{5.2.5 Narcotic}
\begin{entrylist}
\entry{buata}\headword{buata}{\pos{Noun}} {\definition{betel nut, areca nut (fruit of Areca catechu)}}
\entry{daga}\headword{daga}{\pos{Noun}} {\definition{betel}}
\end{entrylist}

\section*{5.2.6 Stimulant}
\begin{entrylist}
\entry{buata}\headword{buata}{\pos{Noun}} {\definition{betel nut, areca nut (fruit of Areca catechu)}}
\entry{daga}\headword{daga}{\pos{Noun}} {\definition{betel}}
\entry{pänpän}\headword{pänpän}{\pos{Noun}} {\definition{calcium hydroxide, lime}}
\end{entrylist}

\section*{5.3 Clothing}
\begin{entrylist}
\entry{arup}\headword{arup}{\pos{Noun}} {\definition{clothing type}}
\entry{bam bam doros}\headword{bam bam doros}{\pos{Noun}} {\definition{bum bum pants}}
\entry{banggu}\headword{banggu}{\pos{Noun}} {\definition{headdress}}
\entry{doros}\headword{doros}{\pos{Noun}} {\definition{pants}}
\entry{iddpo}\headword{iddpo}{\pos{Noun}} {\definition{clothing, clothes}}
\entry{igi pite}\headword{igi pite}{\pos{Noun}} {\definition{underwear}}
\entry{ikop glas}\headword{ikop glas}{\pos{Noun}} {\definition{eyeglasses, spectacles; goggles}}
\entry{kallkäll sod}\headword{kallkäll sod}{\pos{Noun}} {\definition{long-sleeve shirt; coat}}
\entry{kaptte}\headword{kaptte}{\pos{Noun}} {\definition{clothing, clothes; piece of clothing, garment}}
\entry{kaptte tubutubu}\headword{kaptte tubutubu}{\pos{Noun}} {\definition{short trousers}}
\entry{kaptte tupi}\headword{kaptte tupi}{\pos{Noun}} {\definition{long trousers}}
\entry{kau}\headword{kau}{\pos{Noun}} {\definition{wrestling clothes}}
\entry{kotom}\headword{kotom}{\pos{Noun}} {\definition{head covering, crown}}
\entry{käkoll}\headword{käkoll}{\pos{Noun}} {\definition{baby mat}}
\entry{källän}\headword{källän}{\pos{Noun}} {\definition{belt}}
\entry{nge dara dara pite}\headword{nge dara dara pite}{\pos{Noun}} {\definition{coconut grass skirt}}
\entry{nying kollop}\headword{nying kollop}{\pos{Noun}} {\definition{sandal}}
\entry{nying ttoe}\headword{nying ttoe}{\pos{Noun}} {\definition{shoe}}
\entry{pakätt}\headword{pakätt}{\pos{Noun}} {\definition{widow's robe}}
\entry{ping}\headword{ping}{\pos{Noun}} {\definition{baby pin}}
\entry{pinggudd}\headword{pinggudd}{\pos{Noun}} {\definition{skirt}}
\entry{pite}\headword{pite}{\pos{Noun}} {\definition{grass skirt}}
\entry{pitepite}\headword{pitepite}{\pos{Adverb}} {\definition{under the skirt}}
\entry{saks}\headword{saks}{\pos{Noun}} {\definition{socks}}
\entry{sod}\headword{sod}{\pos{Noun}} {\definition{shirt}}
\entry{sära}\headword{sära}{\pos{Noun}} {\definition{uncut grass skirt}}
\entry{ttang tupiang sod}\headword{ttang tupiang sod}{\pos{Noun}} {\definition{long-sleeved shirt}}
\entry{ttäle pitt}\headword{ttäle pitt}{\pos{Noun}} {\definition{leg band}}
\entry{walap}\headword{walap}{\pos{Noun}} {\definition{cap}}
\end{entrylist}

\section*{5.3.3 Traditional clothing}
\begin{entrylist}
\entry{banggu}\headword{banggu}{\pos{Noun}} {\definition{headdress}}
\entry{därmir}\headword{därmir}{\pos{Noun}} {\definition{type of tree that is used to treat sores}}
\entry{labalaba}\headword{labalaba}{\pos{Noun}} {\definition{lap-lap}}
\entry{pite}\headword{pite}{\pos{Noun}} {\definition{grass skirt}}
\entry{pitepite}\headword{pitepite}{\pos{Adverb}} {\definition{under the skirt}}
\entry{pitt}\headword{pitt}{\pos{Noun}} {\definition{band}}
\entry{saomasaoma}\headword{saomasaoma}{\pos{Noun}} {\definition{dancing band}}
\entry{wanpadam}\headword{wanpadam}{\pos{Noun}} {\definition{lap-lap}}
\end{entrylist}

\section*{5.3.5 Clothes for special people}
\begin{entrylist}
\entry{kolos}\headword{kolos}{\pos{Noun}} {\definition{constable uniform}}
\entry{yunipom}\headword{yunipom}{\pos{Noun}} {\definition{uniform}}
\end{entrylist}

\section*{5.3.7 Wear clothing}
\begin{entrylist}
\entry{dadäräb}\headword{dadäräb}{\pos{Transitive S verb}} {\definition{to dress}}
\entry{dradre}\headword{dradre}{\pos{Transitive S verb}} {\definition{to dress}}
\entry{kanas}\headword{kanas}{\pos{Noun}} {\definition{type of basic arrow}}
\entry{mättmätt}\headword{mättmätt}{\pos{Intransitive S verb}} {\definition{to wear, dress oneself, put on}}
\entry{mättmätt}\headword{mättmätt}{\pos{Intransitive S verb}} {\definition{to dress}}
\end{entrylist}

\section*{5.3.8 Naked}
\begin{entrylist}
\entry{mäsemäse}\headword{mäsemäse}{\pos{Adverb}} {\definition{naked}}
\end{entrylist}

\section*{5.4 Adornment}
\begin{entrylist}
\entry{dadäräb}\headword{dadäräb}{\pos{Transitive S verb}} {\definition{to decorate}}
\entry{kotom}\headword{kotom}{\pos{Noun}} {\definition{head covering, crown}}
\entry{kwib}\headword{kwib}{\pos{Noun}} {\definition{charcoal made from a particular tree called upiye, used for painting during dance and initiation ceremony}}
\entry{pädpäd}\headword{pädpäd}{\pos{Noun}} {\definition{tattoo}}
\end{entrylist}

\section*{5.4.1 Jewelry}
\begin{entrylist}
\entry{kok patar}\headword{kok patar}{\pos{Noun}} {\definition{necklace}}
\entry{pitt}\headword{pitt}{\pos{Noun}} {\definition{band}}
\entry{ttang pitt}\headword{ttang pitt}{\pos{Noun}} {\definition{bracelet}}
\entry{tärke käp}\headword{tärke käp}{\pos{Noun}} {\definition{necklace}}
\end{entrylist}

\section*{5.4.3.4 Hairstyle}
\begin{entrylist}
\entry{amäramär}\headword{amäramär}{\pos{Noun}} {\definition{braid}}
\entry{mäkämäkäp}\headword{mäkämäkäp}{\pos{Noun}} {\definition{dreadlocks}}
\entry{pinzopinzo}\headword{pinzopinzo}{\pos{Noun}} {\definition{curly hair}}
\entry{podd}\headword{podd}{\pos{Noun}} {\definition{bald head, baldness}}
\entry{tɨtɨp}\headword{tɨtɨp}{\pos{Transitive S verb}} {\definition{to braid}}
\end{entrylist}

\section*{5.4.3.6 Shave}
\begin{entrylist}
\entry{ttäkoe}\headword{ttäkoe}{\pos{Transitive S verb}} {\definition{to chop, cut down, mow; shave}}
\end{entrylist}

\section*{5.4.5 Anoint the body}
\begin{entrylist}
\entry{mokon}\headword{mokon}{\pos{Transitive S verb}} {\definition{to anoint}}
\entry{nyäny}\headword{nyäny}{\pos{Transitive S verb}} {\definition{to anoint}}
\end{entrylist}

\section*{5.4.6 Ritual scar}
\begin{entrylist}
\entry{po}\headword{po}{\pos{Transitive S verb}} {\definition{to pierce}}
\end{entrylist}

\section*{5.5 Fire}
\begin{entrylist}
\entry{bäng}\headword{bäng}{\pos{Noun}} {\definition{firestick (to start a fire)}}
\entry{däkna}\headword{däkna}{\pos{Noun}} {\definition{small black termite mound that burns for a long time; after a woman gives birth, it is heated in the fire, wrapped in bark and cloth, placed under a mat, and used to warm the woman's stomach}}
\entry{ikrol}\headword{ikrol}{\pos{Noun}} {\definition{ash}}
\entry{indre}\headword{indre}{\pos{Noun}} {\definition{type of tree that grows in the grassland with edible brown-yellow fruit and good, long-burning firewood}}
\entry{kulläb}\headword{kulläb}{\pos{Noun}} {\definition{large black termite mound}}
\entry{mattgal}\headword{mattgal}{\pos{Transitive S verb}} {\definition{to put in fire}}
\entry{yindrang}\headword{yindrang}{\pos{Modifier}} {\definition{bright, e.g. for torch, fire, etc.}}
\entry{yu}\headword{yu}{\pos{Noun}} {\definition{fire}}
\entry{yu}\headword{yu}{\pos{Noun}} {\definition{firewood}}
\entry{yu dumbrel}\headword{yu dumbrel}{\pos{Noun}} {\definition{flame}}
\entry{yu kire}\headword{yu kire}{\pos{Noun}} {\definition{firewood}}
\entry{yu menanen}\headword{yu menanen}{\pos{Noun}} {\definition{heat of the fire}}
\entry{yu ngongom}\headword{yu ngongom}{\pos{Noun}} {\definition{small fire}}
\end{entrylist}

\section*{5.5.1 Light a fire}
\begin{entrylist}
\entry{lläklläk}\headword{lläklläk}{\pos{Transitive S verb}} {\definition{to spread fire}}
\entry{penongg}\headword{penongg}{\pos{Intransitive S verb}} {\definition{to burn, set on fire, ignite}}
\entry{udaude}\headword{udaude}{\pos{Transitive S verb}} {\definition{to light, start (a fire)}}
\entry{yu bäng}\headword{yu bäng}{\pos{Noun}} {\definition{firestick (to start a fire)}}
\entry{yu torkomoll}\headword{yu torkomoll}{\pos{Noun}} {\definition{charcoal}}
\end{entrylist}

\section*{5.5.3 Extinguish a fire}
\begin{entrylist}
\entry{sae}\headword{sae}{\pos{Transitive S verb}} {\definition{to extinguish, put out}}
\entry{sɨs}\headword{sɨs}{\pos{Transitive S verb}} {\definition{to extinguish, turn off}}
\entry{wamän}\headword{wamän}{\pos{Intransitive S verb}} {\definition{to go out, dissipate, extinguish}}
\end{entrylist}

\section*{5.5.4 Burn}
\begin{entrylist}
\entry{dangg}\headword{dangg}{\pos{Intransitive S verb}} {\definition{to burn}}
\entry{gonagone}\headword{gonagone}{\pos{Transitive S verb}} {\definition{to burn}}
\entry{kukull}\headword{kukull}{\pos{Noun}} {\definition{grassfire}}
\entry{kullkull}\headword{kullkull}{\pos{Noun}} {\definition{grassfire; burnt grass}}
\entry{kullkullatt}\headword{kullkullatt}{\pos{Modifier}} {\definition{burnt (of an area)}}
\entry{kälbae}\headword{kälbae}{\pos{Transitive S verb}} {\definition{to singe (use brief heat to remove hair or down)}}
\entry{penongg}\headword{penongg}{\pos{Intransitive S verb}} {\definition{to burn, set on fire, ignite}}
\entry{pɨnyapɨnye}\headword{pɨnyapɨnye}{\pos{Noun}} {\definition{area with burnt grass}}
\entry{ttäm}\headword{ttäm}{\pos{Transitive S verb}} {\definition{to burn; heat on a fire}}
\entry{ttänttäm}\headword{ttänttäm}{\pos{Intransitive A verb}} {\definition{to burn}}
\entry{wandae}\headword{wandae}{\pos{Transitive S verb}} {\definition{to burn}}
\entry{wanyweny}\headword{wanyweny}{\pos{Transitive S verb}} {\definition{to burn}}
\end{entrylist}

\section*{5.5.5 What fires produce}
\begin{entrylist}
\entry{dale}\headword{dale}{\pos{Noun}} {\definition{ash}}
\entry{ikllo}\headword{ikllo}{\pos{Noun}} {\definition{smoke}}
\entry{ikrol}\headword{ikrol}{\pos{Noun}} {\definition{ash}}
\entry{kullkull}\headword{kullkull}{\pos{Noun}} {\definition{grassfire; burnt grass}}
\entry{kwib}\headword{kwib}{\pos{Noun}} {\definition{charcoal made from a particular tree called upiye, used for painting during dance and initiation ceremony}}
\entry{pɨnyapɨnye}\headword{pɨnyapɨnye}{\pos{Noun}} {\definition{area with burnt grass}}
\entry{yu ttätta}\headword{yu ttätta}{\pos{Noun}} {\definition{burning wood, burnt wood}}
\end{entrylist}

\section*{5.5.6 Fuel}
\begin{entrylist}
\entry{moep}\headword{moep}{\pos{Noun}} {\definition{charcoal dust}}
\end{entrylist}

\section*{5.5.7 Fireplace}
\begin{entrylist}
\entry{bikwem}\headword{bikwem}{\pos{Noun}} {\definition{fireplace}}
\end{entrylist}

\section*{5.6 Cleaning}
\begin{entrylist}
\entry{dodro}\headword{dodro}{\pos{Transitive S verb}} {\definition{to clean}}
\entry{totrop}\headword{totrop}{\pos{Transitive S verb}} {\definition{to clean up}}
\end{entrylist}

\section*{5.6.1 Clean, dirty}
\begin{entrylist}
\entry{bittang}\headword{bittang}{\pos{Transitive A verb}} {\definition{to litter, dirty}}
\entry{iräpang}\headword{iräpang}{\pos{Modifier}} {\definition{dirty}}
\entry{kotang}\headword{kotang}{\pos{Modifier}} {\definition{dirty, unclean}}
\entry{kotkot}\headword{kotkot}{\pos{Modifier}} {\definition{dirty, unclean}}
\entry{kotmeny}\headword{kotmeny}{\pos{Modifier}} {\definition{clean, pure}}
\entry{pollgo suwe}\headword{pollgo suwe}{\pos{Noun}} {\definition{dirt left on skin after coming out of the water}}
\entry{pänpän}\headword{pänpän}{\pos{Noun}} {\definition{dust}}
\entry{yorkollang}\headword{yorkollang}{\pos{Modifier}} {\definition{dirty}}
\end{entrylist}

\section*{5.6.2 Bathe}
\begin{entrylist}
\entry{gollaembäg}\headword{gollaembäg}{\pos{Transitive S verb}} {\definition{applicative form of gollab}}
\entry{sop}\headword{sop}{\pos{Noun}} {\definition{soap}}
\entry{tatu}\headword{tatu}{\pos{Transitive/Intransitive A verb}} {\definition{to bathe, wash oneself; wash (an animate object)}}
\entry{tatuma}\headword{tatuma}{\pos{Noun}} {\definition{washing place, outdoor bathing area}}
\end{entrylist}

\section*{5.6.3 Wash dishes}
\begin{entrylist}
\entry{gänggälläm}\headword{gänggälläm}{\pos{Transitive S verb}} {\definition{to wash (a body part or inanimate object)}}
\end{entrylist}

\section*{5.6.4 Wash clothes}
\begin{entrylist}
\entry{kaptte gällämnan}\headword{kaptte gällämnan}{\pos{Noun}} {\definition{laundry, washing clothes}}
\end{entrylist}

\section*{5.6.5 Sweep, rake}
\begin{entrylist}
\entry{omom}\headword{omom}{\pos{Transitive S verb}} {\definition{to sweep}}
\entry{pedae}\headword{pedae}{\pos{Transitive S verb}} {\definition{to sweep}}
\entry{tan}\headword{tan}{\pos{Noun}} {\definition{broom, can come from different types of palm e.g. k}}
\end{entrylist}

\section*{5.6.6 Wipe, erase}
\begin{entrylist}
\entry{mokmok}\headword{mokmok}{\pos{Verb}} {\definition{to wipe; dry by wiping}}
\end{entrylist}

\section*{5.7 Sleep}
\begin{entrylist}
\entry{anu}\headword{anu}{\pos{Verb}} {\definition{sleep (command given to babies)}}
\entry{bändaeg}\headword{bändaeg}{\pos{Transitive S verb}} {\definition{to pull an all-nighter (stay awake)}}
\entry{inu}\headword{inu}{\pos{Intransitive S verb}} {\definition{to sleep}}
\entry{sipel}\headword{sipel}{\pos{Intransitive A verb}} {\definition{to rest}}
\entry{tanter}\headword{tanter}{\pos{Transitive S verb}} {\definition{to guard in one's sleep, sleep with}}
\end{entrylist}

\section*{5.7.2 Dream}
\begin{entrylist}
\entry{yon}\headword{yon}{\pos{Noun}} {\definition{dream}}
\end{entrylist}

\section*{5.7.3 Wake up}
\begin{entrylist}
\entry{ku}\headword{ku}{\pos{Intransitive S verb}} {\definition{to wake up}}
\entry{ngällbän}\headword{ngällbän}{\pos{Intransitive S verb}} {\definition{to awake, wake up, get up}}
\entry{patt}\headword{patt}{\pos{Intransitive A verb}} {\definition{to wake up late (past 8 am)}}
\entry{pänongg}\headword{pänongg}{\pos{Transitive S verb}} {\definition{to wake up, wake}}
\end{entrylist}

\section*{5.9 Live, stay}
\begin{entrylist}
\entry{giddoll}\headword{giddoll}{\pos{Intransitive S verb}} {\definition{to live, reside}}
\entry{källmakällme}\headword{källmakällme}{\pos{Intransitive S verb}} {\definition{to survive}}
\end{entrylist}

\section*{6.1.1 Worker}
\begin{entrylist}
\entry{melemang}\headword{melemang}{\pos{Noun}} {\definition{servant, laborer}}
\end{entrylist}

\section*{6.1.2 Method}
\begin{entrylist}
\entry{nyongo}\headword{nyongo}{\pos{Noun}} {\definition{way, method}}
\entry{ttoen}\headword{ttoen}{\pos{Noun}} {\definition{way, method}}
\end{entrylist}

\section*{6.1.2.1 Try, attempt}
\begin{entrylist}
\entry{erängg}\headword{erängg}{\pos{Transitive S verb}} {\definition{to test, try; taste}}
\entry{kädbae}\headword{kädbae}{\pos{Transitive S verb}} {\definition{to test, try}}
\entry{soroe}\headword{soroe}{\pos{Transitive S verb}} {\definition{to try, attempt}}
\end{entrylist}

\section*{6.1.2.2 Use}
\begin{entrylist}
\entry{mäkamäke}\headword{mäkamäke}{\pos{Transitive S verb}} {\definition{to use}}
\entry{yus}\headword{yus}{\pos{Transitive A verb}} {\definition{to use}}
\end{entrylist}

\section*{6.1.2.2.5 Take care of something}
\begin{entrylist}
\entry{pänggmeny}\headword{pänggmeny}{\pos{Transitive S verb}} {\definition{to protect, look after, take care of}}
\end{entrylist}

\section*{6.1.2.2.6 Waste}
\begin{entrylist}
\entry{bittang}\headword{bittang}{\pos{Transitive A verb}} {\definition{to litter, dirty}}
\entry{källa}\headword{källa}{\pos{Noun}} {\definition{feces, poop, waste}}
\entry{källakällong}\headword{källakällong}{\pos{Adverb}} {\definition{while pooping; constantly needing to poop}}
\entry{rabis}\headword{rabis}{\pos{Noun}} {\definition{rubbish, trash}}
\entry{tot}\headword{tot}{\pos{Noun}} {\definition{rubbish, trash, junk}}
\end{entrylist}

\section*{6.1.2.3.1 Careful}
\begin{entrylist}
\entry{tonang}\headword{tonang}{\pos{Modifier}} {\definition{careful, cautious}}
\end{entrylist}

\section*{6.1.2.3.2 Work hard}
\begin{entrylist}
\entry{melemang}\headword{melemang}{\pos{Modifier}} {\definition{hardworking}}
\end{entrylist}

\section*{6.1.2.3.3 Busy}
\begin{entrylist}
\entry{kamekong}\headword{kamekong}{\pos{Modifier}} {\definition{busy}}
\end{entrylist}

\section*{6.1.2.3.4 Power, force}
\begin{entrylist}
\entry{ddangoe}\headword{ddangoe}{\pos{Transitive S verb}} {\definition{to force to do}}
\entry{mängall}\headword{mängall}{\pos{Noun}} {\definition{strength, power}}
\end{entrylist}

\section*{6.1.2.3.5 Complete, finish}
\begin{entrylist}
\entry{binbäddbädd}\headword{binbäddbädd}{\pos{Adverb}} {\definition{fully, completely}}
\entry{ttamän}\headword{ttamän}{\pos{Transitive S verb}} {\definition{to finish, end}}
\end{entrylist}

\section*{6.1.2.4.2 Lazy}
\begin{entrylist}
\entry{dämbag}\headword{dämbag}{\pos{Noun}} {\definition{lazy person, weak person}}
\end{entrylist}

\section*{6.1.2.4.3 Give up}
\begin{entrylist}
\entry{tämpeyam}\headword{tämpeyam}{\pos{Transitive S verb}} {\definition{to abandon, give up}}
\end{entrylist}

\section*{6.1.2.5 Plan}
\begin{entrylist}
\entry{pälan}\headword{pälan}{\pos{Noun}} {\definition{plan}}
\entry{täbatäbe}\headword{täbatäbe}{\pos{Intransitive S verb}} {\definition{to plan}}
\end{entrylist}

\section*{6.1.2.6 Prepare}
\begin{entrylist}
\entry{redi}\headword{redi}{\pos{Transitive A verb}} {\definition{to prepare, make ready}}
\entry{särämbae}\headword{särämbae}{\pos{Transitive S verb}} {\definition{to prepare, arrange, get ready}}
\end{entrylist}

\section*{6.1.3 Difficult, impossible}
\begin{entrylist}
\entry{pumi}\headword{pumi}{\pos{Modifier}} {\definition{exhausting, tiring, strenuous}}
\end{entrylist}

\section*{6.1.3.1 Easy, possible}
\begin{entrylist}
\entry{kuddäkuddäll}\headword{kuddäkuddäll}{\pos{Modifier}} {\definition{easy}}
\entry{poapoa}\headword{poapoa}{\pos{Modifier}} {\definition{easy}}
\end{entrylist}

\section*{6.1.3.3 Fail}
\begin{entrylist}
\entry{gämoe}\headword{gämoe}{\pos{Transitive S verb}} {\definition{to miss}}
\end{entrylist}

\section*{6.2 Agriculture}
\begin{entrylist}
\entry{bunmat}\headword{bunmat}{\pos{Noun}} {\definition{center of a garden}}
\entry{bänbän}\headword{bänbän}{\pos{Noun}} {\definition{the fifth stage of coconut growth in which the fruits are beginning to form}}
\entry{gllae}\headword{gllae}{\pos{Transitive S verb}} {\definition{to dig, spade}}
\entry{kip papa}\headword{kip papa}{\pos{Noun}} {\definition{top of fence}}
\entry{kukull}\headword{kukull}{\pos{Noun}} {\definition{grassfire}}
\entry{kullkull}\headword{kullkull}{\pos{Noun}} {\definition{grassfire; burnt grass}}
\entry{källa mit}\headword{källa mit}{\pos{Noun}} {\definition{bottom of fence}}
\entry{mondre}\headword{mondre}{\pos{Noun}} {\definition{gardening, garden work}}
\entry{nyingnying}\headword{nyingnying}{\pos{Noun}} {\definition{foundation of fence}}
\entry{polle bo}\headword{polle bo}{\pos{Noun}} {\definition{base of fence}}
\entry{pu}\headword{pu}{\pos{Noun}} {\definition{swamp garden}}
\entry{ttängämttängäm}\headword{ttängämttängäm}{\pos{Noun}} {\definition{small garden}}
\entry{täme sära}\headword{täme sära}{\pos{Noun}} {\definition{bush rope}}
\end{entrylist}

\section*{6.2.1.2 Growing roots}
\begin{entrylist}
\entry{ttomttom}\headword{ttomttom}{\pos{Noun}} {\definition{yam heap}}
\end{entrylist}

\section*{6.2.1.2.1 Growing potatoes}
\begin{entrylist}
\entry{bib}\headword{bib}{\pos{Intransitive A verb}} {\definition{to break ground, surface}}
\entry{po}\headword{po}{\pos{Noun}} {\definition{mound of dirt around a plant}}
\end{entrylist}

\section*{6.2.1.2.2 Growing cassava}
\begin{entrylist}
\entry{po}\headword{po}{\pos{Noun}} {\definition{mound of dirt around a plant}}
\end{entrylist}

\section*{6.2.1.7 Growing trees}
\begin{entrylist}
\entry{sanawang}\headword{sanawang}{\pos{Noun}} {\definition{area where sago grows}}
\end{entrylist}

\section*{6.2.2 Land preparation}
\begin{entrylist}
\entry{malmal}\headword{malmal}{\pos{Transitive S verb}} {\definition{to mark land (for planting or settlement)}}
\entry{polle}\headword{polle}{\pos{Noun}} {\definition{fence}}
\end{entrylist}

\section*{6.2.2.1 Clear a field}
\begin{entrylist}
\entry{gäglläb}\headword{gäglläb}{\pos{Transitive S verb}} {\definition{to weed for the first time (hard work to remove the large plants)}}
\entry{käklläp}\headword{käklläp}{\pos{Transitive S verb}} {\definition{to weed for the second time (easy work to remove the remaining plants)}}
\entry{käkllätt}\headword{käkllätt}{\pos{Transitive S verb}} {\definition{to weed roughly (leaving bits behind)}}
\entry{llanded}\headword{llanded}{\pos{Transitive S verb}} {\definition{to clear}}
\entry{tätäräp}\headword{tätäräp}{\pos{Noun}} {\definition{man-made clearing}}
\entry{yänddäna}\headword{yänddäna}{\pos{Verb}} {\definition{to clear, e.g. when you clear the grass or bush and the space is open}}
\end{entrylist}

\section*{6.2.3 Plant a field}
\begin{entrylist}
\entry{dana}\headword{dana}{\pos{Transitive A verb}} {\definition{to sprout}}
\entry{ibeny}\headword{ibeny}{\pos{Transitive S verb}} {\definition{to plant}}
\entry{pädrall}\headword{pädrall}{\pos{Intransitive S verb}} {\definition{to spread (out), scatter, sow}}
\entry{pɨnypɨny}\headword{pɨnypɨny}{\pos{Transitive S verb}} {\definition{to plant a lot}}
\end{entrylist}

\section*{6.2.4 Tend a field}
\begin{entrylist}
\entry{po}\headword{po}{\pos{Noun}} {\definition{mound of dirt around a plant}}
\end{entrylist}

\section*{6.2.4.1 Cut grass}
\begin{entrylist}
\entry{dädäräb}\headword{dädäräb}{\pos{Transitive S verb}} {\definition{to cut grass, flowers}}
\entry{kaekep}\headword{kaekep}{\pos{Transitive S verb}} {\definition{to struggle to cut (e.g. with a dull knife)}}
\entry{pape}\headword{pape}{\pos{Transitive S verb}} {\definition{to cut (grass)}}
\entry{ttäkoe}\headword{ttäkoe}{\pos{Transitive S verb}} {\definition{to chop, cut down, mow; shave}}
\end{entrylist}

\section*{6.2.4.2 Uproot plants}
\begin{entrylist}
\entry{idän}\headword{idän}{\pos{Transitive S verb}} {\definition{to pick, harvest; dig up, uproot}}
\end{entrylist}

\section*{6.2.4.4 Trim plants}
\begin{entrylist}
\entry{paopao}\headword{paopao}{\pos{Transitive S verb}} {\definition{to cut around, prune (a plant)}}
\entry{ttaengän}\headword{ttaengän}{\pos{Transitive S verb}} {\definition{to pull (a plant sucker)}}
\end{entrylist}

\section*{6.2.5 Harvest}
\begin{entrylist}
\entry{dabän}\headword{dabän}{\pos{Transitive S verb}} {\definition{to knock a leaf, e.g. a tobacco leaf}}
\entry{dadel}\headword{dadel}{\pos{Noun}} {\definition{harvest}}
\entry{duwel sära}\headword{duwel sära}{\pos{Noun}} {\definition{type of sago bundle wrapped in sago leaves}}
\entry{däbae}\headword{däbae}{\pos{Intransitive S verb}} {\definition{to knock fruit continuously}}
\entry{idän}\headword{idän}{\pos{Transitive S verb}} {\definition{to pick, harvest; dig up, uproot}}
\entry{kire}\headword{kire}{\pos{Modifier}} {\definition{unripe; raw, fresh}}
\entry{lläpän}\headword{lläpän}{\pos{Transitive S verb}} {\definition{to dig, harvest, unearth (a tuber or corm)}}
\entry{o}\headword{o}{\pos{Modifier}} {\definition{ripe}}
\entry{tap}\headword{tap}{\pos{Transitive S verb}} {\definition{to harvest}}
\end{entrylist}

\section*{6.2.5.3 Gather wild plants}
\begin{entrylist}
\entry{ddän}\headword{ddän}{\pos{Transitive S verb}} {\definition{to pick, gather, harvest}}
\entry{dändäratt}\headword{dändäratt}{\pos{Noun}} {\definition{bagful of sago}}
\entry{gullba}\headword{gullba}{\pos{Noun}} {\definition{bundle of sago}}
\entry{idän}\headword{idän}{\pos{Transitive S verb}} {\definition{to pick, harvest; dig up, uproot}}
\end{entrylist}

\section*{6.2.5.4 Plant product}
\begin{entrylist}
\entry{ngällngäll}\headword{ngällngäll}{\pos{Transitive S verb}} {\definition{to produce, yield, bear (fruit)}}
\end{entrylist}

\section*{6.2.6 Process harvest}
\begin{entrylist}
\entry{bollboll}\headword{bollboll}{\pos{Transitive S verb}} {\definition{to open sago}}
\entry{gädagäde}\headword{gädagäde}{\pos{Transitive S verb}} {\definition{to beat sago, pound sago}}
\entry{käg manas}\headword{käg manas}{\pos{Noun}} {\definition{container for squeezing sago}}
\entry{nyukukum}\headword{nyukukum}{\pos{Noun}} {\definition{type of bag, used for squeezing sago, a thing woven from tree bark or reeds to wash sago or to fill with sago.}}
\entry{nyäkukub}\headword{nyäkukub}{\pos{Noun}} {\definition{sago washing basket}}
\entry{tɨt}\headword{tɨt}{\pos{Transitive S verb}} {\definition{to beat sago, pound sago}}
\end{entrylist}

\section*{6.2.6.4 Store the harvest}
\begin{entrylist}
\entry{gag}\headword{gag}{\pos{Transitive S verb}} {\definition{to store}}
\entry{gaguma}\headword{gaguma}{\pos{Noun}} {\definition{yamhouse}}
\end{entrylist}

\section*{6.2.7 Farm worker}
\begin{entrylist}
\entry{pama}\headword{pama}{\pos{Noun}} {\definition{farmer}}
\end{entrylist}

\section*{6.2.8 Agricultural tool}
\begin{entrylist}
\entry{abor}\headword{abor}{\pos{Noun}} {\definition{sago beater}}
\entry{dade}\headword{dade}{\pos{Noun}} {\definition{yam stick}}
\entry{giriwak}\headword{giriwak}{\pos{Noun}} {\definition{type of tool}}
\entry{ibik}\headword{ibik}{\pos{Noun}} {\definition{digging stick}}
\entry{madik}\headword{madik}{\pos{Noun}} {\definition{type of tool}}
\entry{nyukukum}\headword{nyukukum}{\pos{Noun}} {\definition{type of bag, used for squeezing sago, a thing woven from tree bark or reeds to wash sago or to fill with sago.}}
\entry{nyäng}\headword{nyäng}{\pos{Noun}} {\definition{basket; bag}}
\entry{pambu}\headword{pambu}{\pos{Noun}} {\definition{hoe (tool)}}
\entry{patt}\headword{patt}{\pos{Noun}} {\definition{coconut husking stick}}
\entry{sabol}\headword{sabol}{\pos{Noun}} {\definition{shovel, spade}}
\entry{topotopoll}\headword{topotopoll}{\pos{Noun}} {\definition{type of tree that grows in the bush with white flowers; used as a yam stick}}
\entry{yäbik}\headword{yäbik}{\pos{Noun}} {\definition{sharp gardening stick}}
\entry{zazaba}\headword{zazaba}{\pos{Noun}} {\definition{type of bag}}
\entry{zora}\headword{zora}{\pos{Noun}} {\definition{sharp stick for peeling sago before beating it}}
\end{entrylist}

\section*{6.2.9 Farmland}
\begin{entrylist}
\entry{goeg}\headword{goeg}{\pos{Noun}} {\definition{old garden that is ready to be cleared again}}
\entry{sanawang}\headword{sanawang}{\pos{Noun}} {\definition{area where sago grows}}
\entry{ttängäm}\headword{ttängäm}{\pos{Noun}} {\definition{garden}}
\entry{witara}\headword{witara}{\pos{Noun}} {\definition{type of garden}}
\end{entrylist}

\section*{6.3 Animal husbandry}
\begin{entrylist}
\entry{peyom}\headword{peyom}{\pos{Modifier}} {\definition{mating}}
\end{entrylist}

\section*{6.3.1 Domesticated animal}
\begin{entrylist}
\entry{bigma}\headword{bigma}{\pos{Noun}} {\definition{enclosure, pen, sty}}
\entry{mäda}\headword{mäda}{\pos{Noun}} {\definition{owner (of an animal)}}
\end{entrylist}

\section*{6.3.1.2 Sheep}
\begin{entrylist}
\entry{sip}\headword{sip}{\pos{Noun}} {\definition{sheep}}
\end{entrylist}

\section*{6.3.1.4 Pig}
\begin{entrylist}
\entry{mimi}\headword{mimi}{\pos{Noun}} {\definition{pig (hunting word)}}
\entry{sämell}\headword{sämell}{\pos{Noun}} {\definition{pig}}
\entry{sɨmell}\headword{sɨmell}{\pos{Noun}} {\definition{pig}}
\end{entrylist}

\section*{6.3.1.5 Dog}
\begin{entrylist}
\entry{deng}\headword{deng}{\pos{Noun}} {\definition{dog}}
\entry{däräng}\headword{däräng}{\pos{Noun}} {\definition{dog}}
\entry{pollpoll}\headword{pollpoll}{\pos{Transitive S verb}} {\definition{to bark (at)}}
\entry{tänggag}\headword{tänggag}{\pos{Transitive S verb}} {\definition{to make a dog more sensitive to smells by rubbing lemongrass on their nose}}
\end{entrylist}

\section*{6.3.1.6 Cat}
\begin{entrylist}
\entry{pus}\headword{pus}{\pos{Noun}} {\definition{cat}}
\end{entrylist}

\section*{6.3.3 Milk}
\begin{entrylist}
\entry{ngam indäb}\headword{ngam indäb}{\pos{Noun}} {\definition{nipple, teat}}
\end{entrylist}

\section*{6.3.4 Butcher, slaughter}
\begin{entrylist}
\entry{bänybäny}\headword{bänybäny}{\pos{Transitive S verb}} {\definition{to cut, slice (flesh)}}
\entry{koko}\headword{koko}{\pos{Transitive S verb}} {\definition{to cut (flesh or meat), butcher}}
\entry{llɨtɨt}\headword{llɨtɨt}{\pos{Transitive S verb}} {\definition{to butcher, cut}}
\entry{pallam}\headword{pallam}{\pos{Transitive S verb}} {\definition{to cut open}}
\end{entrylist}

\section*{6.4 Hunt and fish}
\begin{entrylist}
\entry{amteamte}\headword{amteamte}{\pos{Noun}} {\definition{bow part}}
\entry{baur}\headword{baur}{\pos{Noun}} {\definition{type of spear}}
\entry{bawur}\headword{bawur}{\pos{Noun}} {\definition{Type of bow made out of a stick.}}
\entry{bilod}\headword{bilod}{\pos{Noun}} {\definition{type of spear}}
\entry{bägäl mäträt}\headword{bägäl mäträt}{\pos{Noun}} {\definition{part of bow}}
\entry{bägäl tangge}\headword{bägäl tangge}{\pos{Noun}} {\definition{extra bowstring}}
\entry{ddumbi}\headword{ddumbi}{\pos{Noun}} {\definition{type of spear topped with the claw of a cassowary}}
\entry{kannas}\headword{kannas}{\pos{Noun}} {\definition{type of bow made out of pitpit}}
\entry{pakos}\headword{pakos}{\pos{Noun}} {\definition{type of spear}}
\entry{puyem}\headword{puyem}{\pos{Noun}} {\definition{hunting event in which one person hits the ground with a short stick}}
\entry{pättkäp}\headword{pättkäp}{\pos{Noun}} {\definition{part of a bow}}
\entry{tae budar}\headword{tae budar}{\pos{Noun}} {\definition{type of edible grub}}
\entry{tawe ttäp}\headword{tawe ttäp}{\pos{Noun}} {\definition{type of spear}}
\entry{tobäll}\headword{tobäll}{\pos{Noun}} {\definition{long spear shot like an arrow}}
\entry{täl wadär}\headword{täl wadär}{\pos{Noun}} {\definition{bamboo string}}
\entry{waya}\headword{waya}{\pos{Noun}} {\definition{type of pronged metal spear}}
\entry{wup ttämbällag}\headword{wup ttämbällag}{\pos{Noun}} {\definition{type of spear}}
\end{entrylist}

\section*{6.4.1 Hunt}
\begin{entrylist}
\entry{awe}\headword{awe}{\pos{Noun}} {\definition{cassowary (used when hunting)}}
\entry{benanbenan}\headword{benanbenan}{\pos{Noun}} {\definition{type of spear}}
\entry{bilod}\headword{bilod}{\pos{Noun}} {\definition{type of spear}}
\entry{bormop}\headword{bormop}{\pos{Noun}} {\definition{type of spear}}
\entry{bägäl}\headword{bägäl}{\pos{Noun}} {\definition{bow}}
\entry{bägäl kuwem}\headword{bägäl kuwem}{\pos{Noun}} {\definition{gunfire}}
\entry{bälwod}\headword{bälwod}{\pos{Noun}} {\definition{Also called pitpit (pidgin word). Tall grass used for pakos (spear). Dry on fire, straighten it, push arrow on, and tie with tulip bark (~2m).}}
\entry{ddäddäg}\headword{ddäddäg}{\pos{Noun}} {\definition{edible animal, game, meat}}
\entry{ddällgoe}\headword{ddällgoe}{\pos{Transitive S verb}} {\definition{to go through thick bush in search of animals}}
\entry{ddänggaddängge}\headword{ddänggaddängge}{\pos{Transitive S verb}} {\definition{to catch, trap}}
\entry{dompadompa}\headword{dompadompa}{\pos{Noun}} {\definition{type of spear}}
\entry{ebagal}\headword{ebagal}{\pos{Noun}} {\definition{type of spear}}
\entry{es}\headword{es}{\pos{Interjection}} {\definition{come (command given to animals)}}
\entry{gabän}\headword{gabän}{\pos{Transitive S verb}} {\definition{to cut in half or to split, e.g. to split a deer thigh with someone.}}
\entry{gamu}\headword{gamu}{\pos{Noun}} {\definition{type of ginger with flat leaves; used as medicine for centipede bites and as bait for catching flying foxes; chew it first and the flying fox will eat it and become lethargic}}
\entry{giri}\headword{giri}{\pos{Noun}} {\definition{knife}}
\entry{gäglib}\headword{gäglib}{\pos{Transitive S verb}} {\definition{to chase, scare away/off (wild animals)}}
\entry{gämoe}\headword{gämoe}{\pos{Transitive S verb}} {\definition{to miss}}
\entry{idoidog}\headword{idoidog}{\pos{Noun}} {\definition{harpoon}}
\entry{kokmer}\headword{kokmer}{\pos{Noun}} {\definition{when a hunter calls out his sister's name after shooting an animal (she will get the back of the animal if it's killed)}}
\entry{kokngal}\headword{kokngal}{\pos{Noun}} {\definition{hunting in the rain}}
\entry{käba}\headword{käba}{\pos{Noun}} {\definition{solo hunt early in the morning}}
\entry{mamoe}\headword{mamoe}{\pos{Transitive S verb}} {\definition{to hunt, go hunting}}
\entry{mamoe}\headword{mamoe}{\pos{Noun}} {\definition{hunt}}
\entry{mimi}\headword{mimi}{\pos{Noun}} {\definition{pig (hunting word)}}
\entry{pakos}\headword{pakos}{\pos{Noun}} {\definition{type of spear}}
\entry{po}\headword{po}{\pos{Intransitive A verb}} {\definition{to block an animal from escaping}}
\entry{pättapätte}\headword{pättapätte}{\pos{Transitive S verb}} {\definition{to hit grass with a stick to scare away animals}}
\entry{rarae}\headword{rarae}{\pos{Noun}} {\definition{hunting technique in which hunters line up to cover ground in the absence of dogs}}
\entry{su}\headword{su}{\pos{Noun}} {\definition{prey}}
\entry{sɨs}\headword{sɨs}{\pos{Interjection}} {\definition{command given to a dog to chase an animal}}
\entry{tobäll abal}\headword{tobäll abal}{\pos{Noun}} {\definition{type of spear}}
\entry{toro}\headword{toro}{\pos{Noun}} {\definition{a symbol of a slain animal, used to display and inform people of the type of animal killed. It was also a hunter's pride to compete or show other hunter' of his skill. Feathers, offcut tail, or animal fur were displayed on a small stick or pitpit or obäll tree stick or stem. In other instances pandanus leaves were symbols for pig, grass for wallaby.}}
\entry{totrop}\headword{totrop}{\pos{Transitive S verb}} {\definition{to scrape, e.g. when you shoot a deer and the arrow just scratches the animal}}
\entry{täräp}\headword{täräp}{\pos{Noun}} {\definition{hunting tally (collection of jaw bones often displayed in the front of people's homes)}}
\entry{täträp}\headword{täträp}{\pos{Noun}} {\definition{season of the end of harvesting and the start of hunting (eighth season; corresponds to late May)}}
\entry{wadär pitt}\headword{wadär pitt}{\pos{Noun}} {\definition{plant from cane, used on both sides of bow}}
\entry{zuwoe}\headword{zuwoe}{\pos{Transitive S verb}} {\definition{to shoot}}
\end{entrylist}

\section*{6.4.1.1 Track an animal}
\begin{entrylist}
\entry{ibiatt}\headword{ibiatt}{\pos{Noun}} {\definition{footprint}}
\entry{llatat}\headword{llatat}{\pos{Transitive S verb}} {\definition{to trace, track, follow the blood (of an animal to kill it)}}
\entry{papiye}\headword{papiye}{\pos{Noun}} {\definition{animal tracks}}
\entry{tänggag}\headword{tänggag}{\pos{Transitive S verb}} {\definition{to make a dog more sensitive to smells by rubbing lemongrass on their nose}}
\end{entrylist}

\section*{6.4.2 Trap}
\begin{entrylist}
\entry{ddänggaddängge}\headword{ddänggaddängge}{\pos{Transitive S verb}} {\definition{to catch, trap}}
\entry{gull}\headword{gull}{\pos{Noun}} {\definition{net}}
\entry{manglle}\headword{manglle}{\pos{Transitive A verb}} {\definition{to lure}}
\entry{po}\headword{po}{\pos{Intransitive A verb}} {\definition{to block an animal from escaping}}
\entry{tätän}\headword{tätän}{\pos{Noun}} {\definition{animal trap}}
\end{entrylist}

\section*{6.4.3 Hunting birds}
\begin{entrylist}
\entry{dompa}\headword{dompa}{\pos{Noun}} {\definition{type of blunt arrow}}
\entry{momana ma}\headword{momana ma}{\pos{Noun}} {\definition{hideout made of leaves for hunting cassowaries}}
\end{entrylist}

\section*{6.4.4 Beekeeping}
\begin{entrylist}
\entry{gongg}\headword{gongg}{\pos{Intransitive S verb}} {\definition{to disturb a bees nest and then to feel the bites}}
\end{entrylist}

\section*{6.4.5 Fishing}
\begin{entrylist}
\entry{ariga}\headword{ariga}{\pos{Noun}} {\definition{hook, in Agob language.}}
\entry{gall}\headword{gall}{\pos{Noun}} {\definition{canoe, boat}}
\entry{ittɨtt}\headword{ittɨtt}{\pos{Transitive S verb}} {\definition{to catch (an aquatic animal)}}
\entry{karado}\headword{karado}{\pos{Noun}} {\definition{long spear for fishing}}
\entry{klak}\headword{klak}{\pos{Noun}} {\definition{type of harpoon for fishing}}
\entry{kokto}\headword{kokto}{\pos{Transitive S verb}} {\definition{to bail (water)}}
\entry{kunu}\headword{kunu}{\pos{Noun}} {\definition{type of short tree that grows in the grassland with poisonous bark for stunning fish}}
\entry{kupull}\headword{kupull}{\pos{Noun}} {\definition{earthworm, worm}}
\entry{pampem}\headword{pampem}{\pos{Transitive S verb}} {\definition{to fish}}
\entry{pomana}\headword{pomana}{\pos{Noun}} {\definition{fishing style in which men climb trees and shoot with ddangol spears}}
\entry{pänbäll}\headword{pänbäll}{\pos{Noun}} {\definition{poisonous vine or root (used in fishing to stun fish)}}
\entry{rarae}\headword{rarae}{\pos{Noun}} {\definition{fishing technique in which ladies line up with nets}}
\entry{sasor}\headword{sasor}{\pos{Noun}} {\definition{trap for fish. Kawam word.}}
\entry{sorsor}\headword{sorsor}{\pos{Verb}} {\definition{to affect at a distance. For example, if poison is poured up stream, the fish down stream will also be affected.}}
\entry{tokong}\headword{tokong}{\pos{Noun}} {\definition{bait}}
\entry{totrop}\headword{totrop}{\pos{Transitive S verb}} {\definition{to remove scales}}
\entry{tudi}\headword{tudi}{\pos{Noun}} {\definition{fishing}}
\end{entrylist}

\section*{6.4.5.1 Fish with net}
\begin{entrylist}
\entry{net}\headword{net}{\pos{Noun}} {\definition{net}}
\end{entrylist}

\section*{6.4.5.2 Fish with hooks}
\begin{entrylist}
\entry{tudi ittaenen}\headword{tudi ittaenen}{\pos{Noun}} {\definition{fishing style in which bait is put on fishing lines that are left in the river and checked later}}
\end{entrylist}

\section*{6.4.5.3 Fishing equipment}
\begin{entrylist}
\entry{dadär}\headword{dadär}{\pos{Noun}} {\definition{type of net suspended in the water by sticks}}
\entry{gull}\headword{gull}{\pos{Noun}} {\definition{net}}
\entry{tada}\headword{tada}{\pos{Noun}} {\definition{fish trap}}
\entry{tudi}\headword{tudi}{\pos{Noun}} {\definition{fishing rod, fishing line}}
\entry{tudi käp}\headword{tudi käp}{\pos{Noun}} {\definition{fishing hook}}
\entry{tudi tär}\headword{tudi tär}{\pos{Noun}} {\definition{fishing line}}
\entry{wan pinga}\headword{wan pinga}{\pos{Noun}} {\definition{metal fish spear}}
\end{entrylist}

\section*{6.5.1 Building}
\begin{entrylist}
\entry{gany}\headword{gany}{\pos{Transitive S verb}} {\definition{to plant, place in the ground, put in}}
\entry{gogo}\headword{gogo}{\pos{Transitive S verb}} {\definition{to build}}
\entry{kollam}\headword{kollam}{\pos{Transitive S verb}} {\definition{to stand up}}
\entry{ngleg}\headword{ngleg}{\pos{Transitive S verb}} {\definition{to clear the ground before building}}
\entry{pos}\headword{pos}{\pos{Noun}} {\definition{post}}
\entry{put}\headword{put}{\pos{Noun}} {\definition{type of brown bark}}
\end{entrylist}

\section*{6.5.1.1 House}
\begin{entrylist}
\entry{bengae}\headword{bengae}{\pos{Transitive S verb}} {\definition{to roof, cover}}
\entry{botta}\headword{botta}{\pos{Noun}} {\definition{lateral beam placed directly on the house post}}
\entry{katre ma}\headword{katre ma}{\pos{Noun}} {\definition{raised house, stilt house}}
\entry{käg botta}\headword{käg botta}{\pos{Noun}} {\definition{horizontal boards on which the floor goes}}
\entry{ma}\headword{ma}{\pos{Noun}} {\definition{house, home}}
\entry{ma kisin}\headword{ma kisin}{\pos{Noun}} {\definition{traditional house built directly on the ground}}
\entry{makäm}\headword{makäm}{\pos{Noun}} {\definition{space under a house}}
\entry{makäm katrekatre}\headword{makäm katrekatre}{\pos{Noun}} {\definition{area to sit or rest underneath the house}}
\entry{toto}\headword{toto}{\pos{Noun}} {\definition{vertical house post}}
\entry{toto botta}\headword{toto botta}{\pos{Noun}} {\definition{horizontal house post}}
\entry{tät}\headword{tät}{\pos{Noun}} {\definition{ladder}}
\end{entrylist}

\section*{6.5.1.3 Land, property}
\begin{entrylist}
\entry{malmal}\headword{malmal}{\pos{Transitive S verb}} {\definition{to mark land (for planting or settlement)}}
\end{entrylist}

\section*{6.5.1.4 Yard}
\begin{entrylist}
\entry{ngätt}\headword{ngätt}{\pos{Noun}} {\definition{yard}}
\entry{yad}\headword{yad}{\pos{Noun}} {\definition{yard}}
\end{entrylist}

\section*{6.5.1.5 Fence, wall}
\begin{entrylist}
\entry{dädäk}\headword{dädäk}{\pos{Noun}} {\definition{wall}}
\entry{kättkätt}\headword{kättkätt}{\pos{Transitive S verb}} {\definition{to fence, wall, build a fence or wall}}
\entry{mäta mamlla}\headword{mäta mamlla}{\pos{Noun}} {\definition{rope made from mäta bark (used to secure fencing)}}
\entry{polle}\headword{polle}{\pos{Noun}} {\definition{fence}}
\end{entrylist}

\section*{6.5.2.1 Wall}
\begin{entrylist}
\entry{dang}\headword{dang}{\pos{Noun}} {\definition{length (of a house)}}
\entry{dum}\headword{dum}{\pos{Noun}} {\definition{width (of a house)}}
\entry{patara}\headword{patara}{\pos{Noun}} {\definition{wall}}
\entry{put}\headword{put}{\pos{Noun}} {\definition{type of brown bark}}
\entry{tarakoll}\headword{tarakoll}{\pos{Noun}} {\definition{protective outer wall built by ancestors}}
\entry{zobo ik}\headword{zobo ik}{\pos{Noun}} {\definition{walling (bark on house between sago walls and roof)}}
\end{entrylist}

\section*{6.5.2.2 Roof}
\begin{entrylist}
\entry{bengae}\headword{bengae}{\pos{Noun}} {\definition{roof, roofing}}
\entry{gugu}\headword{gugu}{\pos{Noun}} {\definition{row of leaves going up the roof}}
\entry{kanas ma}\headword{kanas ma}{\pos{Noun}} {\definition{type of pointed roof not found in Limol}}
\entry{keta}\headword{keta}{\pos{Noun}} {\definition{roof post}}
\entry{kumi}\headword{kumi}{\pos{Noun}} {\definition{central roof beam}}
\entry{kumi kakälläp}\headword{kumi kakälläp}{\pos{Noun}} {\definition{bark on top of roof}}
\entry{kumiye käp}\headword{kumiye käp}{\pos{Noun}} {\definition{bark on roof ridge that prevents rain from coming in}}
\entry{put}\headword{put}{\pos{Noun}} {\definition{type of brown bark}}
\entry{tine}\headword{tine}{\pos{Noun}} {\definition{type of simple roof consisting of a single post covered in sago leaves}}
\entry{ttette}\headword{ttette}{\pos{Noun}} {\definition{rafter}}
\end{entrylist}

\section*{6.5.2.3 Floor}
\begin{entrylist}
\entry{katre}\headword{katre}{\pos{Noun}} {\definition{board; flooring}}
\entry{käg}\headword{käg}{\pos{Noun}} {\definition{floor}}
\entry{käg tater}\headword{käg tater}{\pos{Noun}} {\definition{flooring, floor mat}}
\end{entrylist}

\section*{6.5.2.4 Door}
\begin{entrylist}
\entry{ud}\headword{ud}{\pos{Noun}} {\definition{door; gate}}
\end{entrylist}

\section*{6.5.2.5 Window}
\begin{entrylist}
\entry{wel ud}\headword{wel ud}{\pos{Noun}} {\definition{window}}
\end{entrylist}

\section*{6.5.2.6 Foundation}
\begin{entrylist}
\entry{batt}\headword{batt}{\pos{Noun}} {\definition{central lateral beam of a house}}
\entry{botta}\headword{botta}{\pos{Noun}} {\definition{lateral beam placed directly on the house post}}
\entry{dang toto}\headword{dang toto}{\pos{Noun}} {\definition{corner post}}
\entry{gubare}\headword{gubare}{\pos{Noun}} {\definition{crossbeam}}
\entry{käg botta}\headword{käg botta}{\pos{Noun}} {\definition{horizontal boards on which the floor goes}}
\entry{ngoeang}\headword{ngoeang}{\pos{Noun}} {\definition{traditional Y-shaped house post (used prior to nails)}}
\entry{pos}\headword{pos}{\pos{Noun}} {\definition{post}}
\entry{toto}\headword{toto}{\pos{Noun}} {\definition{vertical house post}}
\entry{toto botta}\headword{toto botta}{\pos{Noun}} {\definition{horizontal house post}}
\end{entrylist}

\section*{6.5.2.7 Room}
\begin{entrylist}
\entry{kisin}\headword{kisin}{\pos{Noun}} {\definition{kitchen}}
\entry{känttatt}\headword{känttatt}{\pos{Noun}} {\definition{bedroom; room, chamber}}
\entry{otät yu ma}\headword{otät yu ma}{\pos{Noun}} {\definition{kitchen}}
\end{entrylist}

\section*{6.5.2.8 Floor, story}
\begin{entrylist}
\entry{mo}\headword{mo}{\pos{Noun}} {\definition{step, stair(s)}}
\entry{mo katrekatre}\headword{mo katrekatre}{\pos{Noun}} {\definition{staircase}}
\end{entrylist}

\section*{6.5.3 Building materials}
\begin{entrylist}
\entry{nil}\headword{nil}{\pos{Noun}} {\definition{nail (made of metal)}}
\entry{pos}\headword{pos}{\pos{Noun}} {\definition{post}}
\end{entrylist}

\section*{6.5.4 Infrastructure}
\begin{entrylist}
\entry{mo}\headword{mo}{\pos{Noun}} {\definition{bridge}}
\entry{parga}\headword{parga}{\pos{Noun}} {\definition{bridge}}
\end{entrylist}

\section*{6.5.4.1 Road}
\begin{entrylist}
\entry{nyongo}\headword{nyongo}{\pos{Noun}} {\definition{road, path, way}}
\entry{tata}\headword{tata}{\pos{Noun}} {\definition{junction, intersection}}
\end{entrylist}

\section*{6.5.4.2 Boundary}
\begin{entrylist}
\entry{ngän}\headword{ngän}{\pos{Noun}} {\definition{boundary}}
\end{entrylist}

\section*{6.6 Occupation}
\begin{entrylist}
\entry{mäk lla}\headword{mäk lla}{\pos{Noun}} {\definition{soldier}}
\entry{tamenyang}\headword{tamenyang}{\pos{Noun}} {\definition{teacher}}
\end{entrylist}

\section*{6.6.1 Working with cloth}
\begin{entrylist}
\entry{kab}\headword{kab}{\pos{Noun}} {\definition{string, rope, fiber}}
\entry{nidol}\headword{nidol}{\pos{Noun}} {\definition{needle}}
\entry{pittpitt}\headword{pittpitt}{\pos{Transitive S verb}} {\definition{to sew, stitch}}
\entry{tär}\headword{tär}{\pos{Noun}} {\definition{string, line}}
\end{entrylist}

\section*{6.6.1.1 Cloth}
\begin{entrylist}
\entry{blengud}\headword{blengud}{\pos{Noun}} {\definition{blanket}}
\entry{kaptte}\headword{kaptte}{\pos{Noun}} {\definition{cloth}}
\end{entrylist}

\section*{6.6.1.4 Weaving cloth}
\begin{entrylist}
\entry{pittpitt}\headword{pittpitt}{\pos{Transitive S verb}} {\definition{to weave}}
\end{entrylist}

\section*{6.6.2.9.1 Explode}
\begin{entrylist}
\entry{ngollot}\headword{ngollot}{\pos{Transitive S verb}} {\definition{to make burst, make explode}}
\end{entrylist}

\section*{6.6.3 Working with wood}
\begin{entrylist}
\entry{popo}\headword{popo}{\pos{Transitive S verb}} {\definition{to carve}}
\entry{sagwan}\headword{sagwan}{\pos{Noun}} {\definition{woodworking tool}}
\entry{turik}\headword{turik}{\pos{Noun}} {\definition{axe}}
\entry{täratäre}\headword{täratäre}{\pos{Transitive S verb}} {\definition{to dig out, hollow out}}
\entry{tɨt}\headword{tɨt}{\pos{Transitive S verb}} {\definition{to hollow out, dig out}}
\entry{yagäl}\headword{yagäl}{\pos{Transitive A verb}} {\definition{to smooth with the yagäl leaf}}
\end{entrylist}

\section*{6.6.3.2 Wood}
\begin{entrylist}
\entry{katre}\headword{katre}{\pos{Noun}} {\definition{board; flooring}}
\entry{llo}\headword{llo}{\pos{Noun}} {\definition{wood}}
\entry{patkoll}\headword{patkoll}{\pos{Noun}} {\definition{bundle}}
\entry{pug}\headword{pug}{\pos{Noun}} {\definition{bundle of bark, two or three inside.}}
\entry{put}\headword{put}{\pos{Noun}} {\definition{type of brown bark}}
\entry{ttäk}\headword{ttäk}{\pos{Noun}} {\definition{soft wood}}
\entry{yu}\headword{yu}{\pos{Noun}} {\definition{firewood}}
\entry{yu ttätta}\headword{yu ttätta}{\pos{Noun}} {\definition{burning wood, burnt wood}}
\end{entrylist}

\section*{6.6.4 Crafts}
\begin{entrylist}
\entry{mamon}\headword{mamon}{\pos{Transitive S verb}} {\definition{to fashion, shape, make}}
\end{entrylist}

\section*{6.6.4.2 Weaving baskets and mats}
\begin{entrylist}
\entry{erongg}\headword{erongg}{\pos{Verb}} {\definition{to start weaving (crossing strings)}}
\entry{gonz}\headword{gonz}{\pos{Noun}} {\definition{reed}}
\entry{gälas}\headword{gälas}{\pos{Noun}} {\definition{weaving pattern with concentric squares}}
\entry{gällall}\headword{gällall}{\pos{Noun}} {\definition{type of pandanus}}
\entry{i}\headword{i}{\pos{Transitive S verb}} {\definition{to weave; interlock}}
\entry{i abal}\headword{i abal}{\pos{Noun}} {\definition{plain weaving pattern}}
\entry{kala}\headword{kala}{\pos{Noun}} {\definition{color, pigment, dye}}
\entry{kättapun}\headword{kättapun}{\pos{Noun}} {\definition{type of reed}}
\entry{kättkätt}\headword{kättkätt}{\pos{Noun}} {\definition{weaving pattern with alternating cross and chevron}}
\entry{llallem}\headword{llallem}{\pos{Transitive S verb}} {\definition{to finish weaving}}
\entry{marat}\headword{marat}{\pos{Noun}} {\definition{weaving pattern with a circular bottom and rectangular sides}}
\entry{nänäm}\headword{nänäm}{\pos{Noun}} {\definition{diagonal checkerboard weaving pattern}}
\entry{opodo}\headword{opodo}{\pos{Noun}} {\definition{weaving pattern with checks}}
\entry{piro}\headword{piro}{\pos{Noun}} {\definition{star weaving pattern}}
\entry{pittpitt}\headword{pittpitt}{\pos{Transitive S verb}} {\definition{to weave}}
\entry{pollgo ttatta kuttang}\headword{pollgo ttatta kuttang}{\pos{Noun}} {\definition{weaving pattern with concentric diamonds}}
\entry{sɨmellmak}\headword{sɨmellmak}{\pos{Noun}} {\definition{weaving pattern with arrows pointing right}}
\entry{ttangkuttangkumang}\headword{ttangkuttangkumang}{\pos{Noun}} {\definition{sideways chevron weaving pattern}}
\entry{wipellgallagallab}\headword{wipellgallagallab}{\pos{Noun}} {\definition{chevron weaving pattern}}
\end{entrylist}

\section*{6.6.5.1 Draw, paint}
\begin{entrylist}
\entry{any}\headword{any}{\pos{Transitive verb}} {\definition{to paint}}
\entry{biro}\headword{biro}{\pos{Noun}} {\definition{pen (writing implement)}}
\entry{nyäny}\headword{nyäny}{\pos{Transitive S verb}} {\definition{to paint}}
\entry{pale}\headword{pale}{\pos{Noun}} {\definition{type of white clay used for painting babies}}
\end{entrylist}

\section*{6.7 Tool}
\begin{entrylist}
\entry{abor}\headword{abor}{\pos{Noun}} {\definition{sago beater}}
\entry{ama}\headword{ama}{\pos{Noun}} {\definition{hammer}}
\entry{bore}\headword{bore}{\pos{Noun}} {\definition{traditional bamboo pipe for smoking tobacco}}
\entry{dangne}\headword{dangne}{\pos{Noun}} {\definition{crawling vine that grows in the grassland with purple flowers; used to make rope}}
\entry{ezi}\headword{ezi}{\pos{Noun}} {\definition{sharp edge}}
\entry{giri}\headword{giri}{\pos{Noun}} {\definition{knife}}
\entry{giriwak}\headword{giriwak}{\pos{Noun}} {\definition{type of tool}}
\entry{komony}\headword{komony}{\pos{Noun}} {\definition{tongs}}
\entry{kätt}\headword{kätt}{\pos{Noun}} {\definition{shell blade}}
\entry{madik}\headword{madik}{\pos{Noun}} {\definition{type of tool}}
\entry{nyängkallddäb}\headword{nyängkallddäb}{\pos{Noun}} {\definition{type of basket}}
\entry{sagwan}\headword{sagwan}{\pos{Noun}} {\definition{woodworking tool}}
\entry{sisel}\headword{sisel}{\pos{Noun}} {\definition{chisel}}
\entry{sur}\headword{sur}{\pos{Noun}} {\definition{pushing tool}}
\entry{tan}\headword{tan}{\pos{Noun}} {\definition{broom, can come from different types of palm e.g. k}}
\entry{tonggo}\headword{tonggo}{\pos{Noun}} {\definition{type of small bamboo that grows in the bush along creeks with a red interior; sharp when split and used as a cutting tool}}
\entry{turik}\headword{turik}{\pos{Noun}} {\definition{axe}}
\entry{täme sära}\headword{täme sära}{\pos{Noun}} {\definition{bush rope}}
\entry{yagäl}\headword{yagäl}{\pos{Transitive A verb}} {\definition{to smooth with the yagäl leaf}}
\entry{yäg}\headword{yäg}{\pos{Noun}} {\definition{bush rope}}
\end{entrylist}

\section*{6.7.1 Cutting tool}
\begin{entrylist}
\entry{kätt}\headword{kätt}{\pos{Noun}} {\definition{shell blade}}
\end{entrylist}

\section*{6.7.1.1 Poking tool}
\begin{entrylist}
\entry{tanyib}\headword{tanyib}{\pos{Noun}} {\definition{radioulna (fused radius and ulna, or arm bone) of a flying fox}}
\entry{waeya}\headword{waeya}{\pos{Noun}} {\definition{wire}}
\end{entrylist}

\section*{6.7.3 Carrying tool}
\begin{entrylist}
\entry{kälämpag}\headword{kälämpag}{\pos{Transitive S verb}} {\definition{to pad (with cloth or leaves to support a heavy load)}}
\entry{nyäng}\headword{nyäng}{\pos{Noun}} {\definition{basket; bag}}
\entry{nyäng tär}\headword{nyäng tär}{\pos{Noun}} {\definition{woven strap or handle of a nyäng}}
\entry{tät}\headword{tät}{\pos{Noun}} {\definition{stretcher}}
\end{entrylist}

\section*{6.7.5 Fastening tool}
\begin{entrylist}
\entry{tär}\headword{tär}{\pos{Noun}} {\definition{string, line}}
\end{entrylist}

\section*{6.7.6 Holding tool}
\begin{entrylist}
\entry{komony}\headword{komony}{\pos{Noun}} {\definition{tongs}}
\end{entrylist}

\section*{6.7.7 Container}
\begin{entrylist}
\entry{ine kutt}\headword{ine kutt}{\pos{Noun}} {\definition{bucket, water container}}
\entry{karpo}\headword{karpo}{\pos{Noun}} {\definition{jar}}
\entry{kube}\headword{kube}{\pos{Noun}} {\definition{bucket}}
\entry{käg}\headword{käg}{\pos{Property noun}} {\definition{vague container-like thing}}
\entry{käg drol}\headword{käg drol}{\pos{Noun}} {\definition{container}}
\entry{käg manas}\headword{käg manas}{\pos{Noun}} {\definition{container for squeezing sago}}
\end{entrylist}

\section*{6.7.7.1 Bag}
\begin{entrylist}
\entry{bumo}\headword{bumo}{\pos{Noun}} {\definition{tucker bag}}
\entry{ddäma}\headword{ddäma}{\pos{Noun}} {\definition{basket}}
\entry{nyäng}\headword{nyäng}{\pos{Noun}} {\definition{basket; bag}}
\entry{nyängkallbidd}\headword{nyängkallbidd}{\pos{Noun}} {\definition{type of bag}}
\entry{säpalek}\headword{säpalek}{\pos{Noun}} {\definition{type of bag}}
\entry{zazaba}\headword{zazaba}{\pos{Noun}} {\definition{type of bag}}
\end{entrylist}

\section*{6.8.1.1 Own, possess}
\begin{entrylist}
\entry{bun}\headword{bun}{\pos{Noun}} {\definition{owner}}
\entry{zizag}\headword{zizag}{\pos{Noun}} {\definition{owner, master, lord}}
\end{entrylist}

\section*{6.8.2.5 Greedy}
\begin{entrylist}
\entry{ttonggmeny}\headword{ttonggmeny}{\pos{Modifier}} {\definition{selfish, greedy}}
\end{entrylist}

\section*{6.8.2.6 Collect}
\begin{entrylist}
\entry{ttotto}\headword{ttotto}{\pos{Transitive S verb}} {\definition{to collect}}
\entry{tukpi}\headword{tukpi}{\pos{Transitive A verb}} {\definition{to heap, pile; gather, collect}}
\entry{umaem}\headword{umaem}{\pos{Intransitive S verb}} {\definition{to gather, collect}}
\end{entrylist}

\section*{6.8.3.1 Give, donate}
\begin{entrylist}
\entry{singoll}\headword{singoll}{\pos{Transitive A verb}} {\definition{to give, provide, share}}
\end{entrylist}

\section*{6.8.3.2 Generous}
\begin{entrylist}
\entry{sinensinen}\headword{sinensinen}{\pos{Modifier}} {\definition{generous, giving}}
\entry{ttonggttongg}\headword{ttonggttongg}{\pos{Modifier}} {\definition{generous, giving}}
\end{entrylist}

\section*{6.8.3.4 Beg}
\begin{entrylist}
\entry{waswes}\headword{waswes}{\pos{Transitive S verb}} {\definition{to ask, beg}}
\end{entrylist}

\section*{6.8.4.1 Buy}
\begin{entrylist}
\entry{llädäd}\headword{llädäd}{\pos{Transitive S verb}} {\definition{to buy}}
\end{entrylist}

\section*{6.8.4.2 Sell}
\begin{entrylist}
\entry{mu}\headword{mu}{\pos{Noun}} {\definition{payment, price, value}}
\entry{sel}\headword{sel}{\pos{Transitive A verb}} {\definition{to sell}}
\end{entrylist}

\section*{6.8.4.3 Price}
\begin{entrylist}
\entry{mu}\headword{mu}{\pos{Noun}} {\definition{payment, price, value}}
\end{entrylist}

\section*{6.8.4.5 Pay}
\begin{entrylist}
\entry{burag}\headword{burag}{\pos{Noun}} {\definition{bride price (given to the bride's family by the groom)}}
\entry{mu}\headword{mu}{\pos{Noun}} {\definition{payment, price, value}}
\end{entrylist}

\section*{6.8.4.8 Store, marketplace}
\begin{entrylist}
\entry{maket}\headword{maket}{\pos{Noun}} {\definition{market}}
\entry{stoa}\headword{stoa}{\pos{Noun}} {\definition{store}}
\end{entrylist}

\section*{6.8.4.9 Exchange, trade}
\begin{entrylist}
\entry{kopae därängmeny}\headword{kopae därängmeny}{\pos{Noun}} {\definition{to go to another village to trade}}
\end{entrylist}

\section*{6.8.5.3 Owe}
\begin{entrylist}
\entry{muminy}\headword{muminy}{\pos{Modifier}} {\definition{in debt}}
\end{entrylist}

\section*{6.8.6 Money}
\begin{entrylist}
\entry{mani käp}\headword{mani käp}{\pos{Noun}} {\definition{money}}
\entry{ttägäll käp}\headword{ttägäll käp}{\pos{Noun}} {\definition{money}}
\end{entrylist}

\section*{6.8.6.1 Monetary units}
\begin{entrylist}
\entry{kina}\headword{kina}{\pos{Noun}} {\definition{kina (PGK, the currency of Papua New Guinea)}}
\entry{mani}\headword{mani}{\pos{Noun}} {\definition{kina (PGK, the currency of Papua New Guinea)}}
\end{entrylist}

\section*{6.8.8 Tax}
\begin{entrylist}
\entry{teks}\headword{teks}{\pos{Noun}} {\definition{tax}}
\end{entrylist}

\section*{6.8.9.1 Steal}
\begin{entrylist}
\entry{gleb}\headword{gleb}{\pos{Transitive S verb}} {\definition{to take, steal}}
\entry{gämäll}\headword{gämäll}{\pos{Transitive A verb}} {\definition{to steal}}
\entry{gämäll tänggag}\headword{gämäll tänggag}{\pos{Transitive S verb}} {\definition{to steal}}
\entry{säkpa mit}\headword{säkpa mit}{\pos{Noun}} {\definition{theft}}
\end{entrylist}

\section*{6.8.9.4 Take by force}
\begin{entrylist}
\entry{gleb}\headword{gleb}{\pos{Transitive S verb}} {\definition{to take, steal}}
\entry{llädäd}\headword{llädäd}{\pos{Transitive S verb}} {\definition{to grab, get, catch}}
\entry{mällam}\headword{mällam}{\pos{Transitive S verb}} {\definition{to hold; get, grab, catch}}
\end{entrylist}

\section*{6.8.9.5 Bribe}
\begin{entrylist}
\entry{iziz}\headword{iziz}{\pos{Noun}} {\definition{bribery}}
\end{entrylist}

\section*{6.9 Business organization}
\begin{entrylist}
\entry{bisnis}\headword{bisnis}{\pos{Noun}} {\definition{business}}
\entry{kampani}\headword{kampani}{\pos{Noun}} {\definition{company}}
\end{entrylist}

\section*{6.9.3 Marketing}
\begin{entrylist}
\entry{duwel sära}\headword{duwel sära}{\pos{Noun}} {\definition{type of sago bundle wrapped in sago leaves}}
\entry{madumame maket}\headword{madumame maket}{\pos{Noun}} {\definition{black market}}
\entry{market}\headword{market}{\pos{Noun}} {\definition{market}}
\entry{wäll}\headword{wäll}{\pos{Noun}} {\definition{sugarcane}}
\end{entrylist}

\section*{7.1.1 Stand}
\begin{entrylist}
\entry{gambäg}\headword{gambäg}{\pos{Intransitive S verb}} {\definition{to stand closely}}
\entry{gambän}\headword{gambän}{\pos{Intransitive S verb}} {\definition{to stand}}
\entry{gugi}\headword{gugi}{\pos{Intransitive S verb}} {\definition{to stand}}
\entry{gägabäll}\headword{gägabäll}{\pos{Intransitive S verb}} {\definition{to stand}}
\end{entrylist}

\section*{7.1.2 Sit}
\begin{entrylist}
\entry{dämen}\headword{dämen}{\pos{Intransitive S verb}} {\definition{to sit}}
\entry{sllollongg}\headword{sllollongg}{\pos{Intransitive S verb}} {\definition{to sit close together}}
\end{entrylist}

\section*{7.1.3 Lie down}
\begin{entrylist}
\entry{kalla}\headword{kalla}{\pos{Intransitive S verb}} {\definition{to lie down}}
\entry{torwam}\headword{torwam}{\pos{Intransitive S verb}} {\definition{to lie down}}
\end{entrylist}

\section*{7.1.4 Kneel}
\begin{entrylist}
\entry{ttangkumttangkum}\headword{ttangkumttangkum}{\pos{Adverb}} {\definition{on all fours}}
\entry{tubutubu}\headword{tubutubu}{\pos{Modifier}} {\definition{kneeling, on one's knees, on the ground; worshipping}}
\end{entrylist}

\section*{7.1.5 Bow}
\begin{entrylist}
\entry{mamon}\headword{mamon}{\pos{Transitive S verb}} {\definition{to string (e.g. a bow, sago beater)}}
\end{entrylist}

\section*{7.1.6 Lean}
\begin{entrylist}
\entry{dangkam}\headword{dangkam}{\pos{Intransitive S verb}} {\definition{to lean}}
\entry{ddogoll}\headword{ddogoll}{\pos{Transitive S verb}} {\definition{to lean}}
\end{entrylist}

\section*{7.1.8 Bend down}
\begin{entrylist}
\entry{pällnampällnam}\headword{pällnampällnam}{\pos{Adverb}} {\definition{squatting, crouching}}
\end{entrylist}

\section*{7.2 Move}
\begin{entrylist}
\entry{inggol}\headword{inggol}{\pos{Intransitive S verb}} {\definition{to move}}
\entry{mondremondre}\headword{mondremondre}{\pos{Transitive/Intransitive A verb}} {\definition{to move}}
\end{entrylist}

\section*{7.2.1.1 Walk}
\begin{entrylist}
\entry{baottbaott}\headword{baottbaott}{\pos{Intransitive A verb}} {\definition{to walk around aimlessly}}
\entry{gäbmäll}\headword{gäbmäll}{\pos{Verb}} {\definition{to skip}}
\entry{ibi}\headword{ibi}{\pos{Intransitive S verb}} {\definition{to walk}}
\entry{kälaepot}\headword{kälaepot}{\pos{Noun}} {\definition{tiptoes}}
\entry{ngon}\headword{ngon}{\pos{Intransitive A verb}} {\definition{to walk in a stylish fashion, to strut}}
\entry{pällttän}\headword{pällttän}{\pos{Intransitive S verb}} {\definition{to set off, start walking}}
\end{entrylist}

\section*{7.2.1.1.1 Run}
\begin{entrylist}
\entry{dindu}\headword{dindu}{\pos{Intransitive S verb}} {\definition{to run, flee, escape}}
\entry{ponor}\headword{ponor}{\pos{Intransitive S verb}} {\definition{to start running}}
\end{entrylist}

\section*{7.2.1.1.3 Jump}
\begin{entrylist}
\entry{gäbaeb}\headword{gäbaeb}{\pos{Intransitive S verb}} {\definition{to jump}}
\entry{gäbän}\headword{gäbän}{\pos{Intransitive S verb}} {\definition{to jump, hop}}
\entry{pänyae}\headword{pänyae}{\pos{Intransitive S verb}} {\definition{to hop}}
\entry{spun}\headword{spun}{\pos{Intransitive S verb}} {\definition{to jump}}
\end{entrylist}

\section*{7.2.1.2 Move quickly}
\begin{entrylist}
\entry{nyägae}\headword{nyägae}{\pos{Intransitive S verb}} {\definition{to flail}}
\end{entrylist}

\section*{7.2.1.2.1 Move slowly}
\begin{entrylist}
\entry{nyäroe}\headword{nyäroe}{\pos{Intransitive S verb}} {\definition{to creep}}
\end{entrylist}

\section*{7.2.1.3 Wander}
\begin{entrylist}
\entry{zanggae}\headword{zanggae}{\pos{Intransitive S verb}} {\definition{to roam, go around}}
\end{entrylist}

\section*{7.2.1.4.1 Clumsy}
\begin{entrylist}
\entry{pendäg}\headword{pendäg}{\pos{Intransitive S verb}} {\definition{to trip}}
\entry{pentngeny}\headword{pentngeny}{\pos{Intransitive S verb}} {\definition{to trip}}
\entry{rorwae}\headword{rorwae}{\pos{Verb}} {\definition{to stagger (e.g. when drunk)}}
\end{entrylist}

\section*{7.2.1.5 Walk with difficulty}
\begin{entrylist}
\entry{ngal}\headword{ngal}{\pos{Intransitive A verb}} {\definition{to waddle, shuffle}}
\entry{nyängoe}\headword{nyängoe}{\pos{Transitive S verb}} {\definition{to walk slowly with someone whose leg is hurt}}
\entry{säkmällsäkmäll}\headword{säkmällsäkmäll}{\pos{Adverb}} {\definition{limping}}
\entry{ttalamttalam}\headword{ttalamttalam}{\pos{Adverb}} {\definition{walking with one's legs spread far apart}}
\entry{ttapeyamttapeyam}\headword{ttapeyamttapeyam}{\pos{Adverb}} {\definition{walking with one's legs spread far apart}}
\entry{ttimattima}\headword{ttimattima}{\pos{Noun}} {\definition{limp}}
\entry{udu}\headword{udu}{\pos{Noun}} {\definition{walking stick, cane, staff}}
\end{entrylist}

\section*{7.2.1.5.1 Slip, slide}
\begin{entrylist}
\entry{dodro}\headword{dodro}{\pos{Intransitive S verb}} {\definition{to slip}}
\end{entrylist}

\section*{7.2.1.6.1 Balance}
\begin{entrylist}
\entry{där}\headword{där}{\pos{Intransitive A verb}} {\definition{to match, balance, agree}}
\end{entrylist}

\section*{7.2.1.7 Move noisily}
\begin{entrylist}
\entry{ddugwemddugwem}\headword{ddugwemddugwem}{\pos{Adverb}} {\definition{stomping}}
\end{entrylist}

\section*{7.2.2.4 Move up}
\begin{entrylist}
\entry{ddänddäl}\headword{ddänddäl}{\pos{Intransitive S verb}} {\definition{to climb}}
\entry{kängkäl}\headword{kängkäl}{\pos{Transitive S verb}} {\definition{to ascend, climb, go up, rise}}
\entry{kängkäl}\headword{kängkäl}{\pos{Transitive S verb}} {\definition{to ascend, climb, go up}}
\entry{ngällbän}\headword{ngällbän}{\pos{Intransitive S verb}} {\definition{to rise, arise, come up}}
\entry{nyärab}\headword{nyärab}{\pos{Intransitive S verb}} {\definition{to go up, ascend}}
\end{entrylist}

\section*{7.2.2.5 Move down}
\begin{entrylist}
\entry{dedre}\headword{dedre}{\pos{Intransitive S verb}} {\definition{to descend, go down}}
\entry{ngetam}\headword{ngetam}{\pos{Intransitive S verb}} {\definition{to come down, descend}}
\end{entrylist}

\section*{7.2.2.5.1 Fall}
\begin{entrylist}
\entry{bäbälläd}\headword{bäbälläd}{\pos{Transitive S verb}} {\definition{to drop without warning}}
\entry{ddälläb}\headword{ddälläb}{\pos{Intransitive S verb}} {\definition{to fall over (of a tree)}}
\entry{pittäpen}\headword{pittäpen}{\pos{Transitive S verb}} {\definition{to drop or bump someone by holding their two legs and hands. To hold an animal (such as a small wallaby) by two legs and swing and whack them against the ground or a tree to kill them}}
\entry{pälengg}\headword{pälengg}{\pos{Intransitive S verb}} {\definition{to drop down, fall}}
\entry{spun}\headword{spun}{\pos{Intransitive S verb}} {\definition{to fall; set}}
\end{entrylist}

\section*{7.2.2.6 Turn}
\begin{entrylist}
\entry{pänae}\headword{pänae}{\pos{Transitive S verb}} {\definition{to turn back, turn around}}
\end{entrylist}

\section*{7.2.2.7 Move in a circle}
\begin{entrylist}
\entry{nganae}\headword{nganae}{\pos{Transitive S verb}} {\definition{to coil, go around}}
\entry{nganae}\headword{nganae}{\pos{Transitive S verb}} {\definition{to spin, rotate}}
\entry{wandawandae}\headword{wandawandae}{\pos{Adverb}} {\definition{rotating, spinning, rolling}}
\end{entrylist}

\section*{7.2.2.8 Move back and forth}
\begin{entrylist}
\entry{popllem}\headword{popllem}{\pos{Intransitive S verb}} {\definition{to flap}}
\entry{wanwen}\headword{wanwen}{\pos{Transitive S verb}} {\definition{to shake, swing}}
\end{entrylist}

\section*{7.2.3.1 Move away}
\begin{entrylist}
\entry{bällge}\headword{bällge}{\pos{Intransitive S verb}} {\definition{to spread, scatter, move away}}
\entry{kunen}\headword{kunen}{\pos{Intransitive S verb}} {\definition{to flee, scatter, run away}}
\entry{pädrall}\headword{pädrall}{\pos{Intransitive S verb}} {\definition{to spread (out), scatter, disperse}}
\entry{ttoengg}\headword{ttoengg}{\pos{Intransitive S verb}} {\definition{to split up, scatter}}
\end{entrylist}

\section*{7.2.3.2 Go}
\begin{entrylist}
\entry{ibi}\headword{ibi}{\pos{Intransitive S verb}} {\definition{to go}}
\entry{pällttän}\headword{pällttän}{\pos{Intransitive S verb}} {\definition{to set off, start walking}}
\end{entrylist}

\section*{7.2.3.3 Leave}
\begin{entrylist}
\entry{inttemängg}\headword{inttemängg}{\pos{Intransitive S verb}} {\definition{to part ways, leave}}
\entry{inttoemängg}\headword{inttoemängg}{\pos{Transitive S verb}} {\definition{to see (someone) off}}
\entry{pällttän}\headword{pällttän}{\pos{Intransitive S verb}} {\definition{to set off, start walking}}
\end{entrylist}

\section*{7.2.3.3.1 Arrive}
\begin{entrylist}
\entry{ddäll}\headword{ddäll}{\pos{Intransitive S verb}} {\definition{to arrive}}
\entry{ngänttäg}\headword{ngänttäg}{\pos{Intransitive S verb}} {\definition{to arrive, return}}
\end{entrylist}

\section*{7.2.3.4 Move in}
\begin{entrylist}
\entry{kakal}\headword{kakal}{\pos{Intransitive S verb}} {\definition{to enter, board, go in}}
\entry{tärak}\headword{tärak}{\pos{Intransitive S verb}} {\definition{to go inside}}
\entry{zan}\headword{zan}{\pos{Intransitive S verb}} {\definition{to enter, go in, go into}}
\end{entrylist}

\section*{7.2.3.4.1 Move out}
\begin{entrylist}
\entry{gazen}\headword{gazen}{\pos{Transitive S verb}} {\definition{to come out, get out, exit, escape}}
\entry{gazen}\headword{gazen}{\pos{Transitive S verb}} {\definition{to take out}}
\entry{irängän}\headword{irängän}{\pos{Intransitive S verb}} {\definition{to come out}}
\entry{kän}\headword{kän}{\pos{Transitive S verb}} {\definition{to withdraw, come out}}
\entry{yattän}\headword{yattän}{\pos{Intransitive S verb}} {\definition{to disembark, get off, get out}}
\end{entrylist}

\section*{7.2.3.5 Move past, over, through}
\begin{entrylist}
\entry{nganzig}\headword{nganzig}{\pos{Transitive S verb}} {\definition{to pass, overtake}}
\entry{opap}\headword{opap}{\pos{Transitive S verb}} {\definition{to cross over, pass over, move across}}
\entry{ullull}\headword{ullull}{\pos{Transitive S verb}} {\definition{to cross over}}
\entry{zäm}\headword{zäm}{\pos{Intransitive S verb}} {\definition{to pass through}}
\entry{zämllall}\headword{zämllall}{\pos{Intransitive A verb}} {\definition{to pass by}}
\end{entrylist}

\section*{7.2.3.6 Return}
\begin{entrylist}
\entry{koen}\headword{koen}{\pos{Intransitive S verb}} {\definition{to turn back}}
\entry{ngänttäg}\headword{ngänttäg}{\pos{Intransitive S verb}} {\definition{to arrive, return}}
\entry{ngäs}\headword{ngäs}{\pos{Intransitive S verb}} {\definition{to return, come back}}
\end{entrylist}

\section*{7.2.4.1.1 Vehicle}
\begin{entrylist}
\entry{dagal}\headword{dagal}{\pos{Transitive S verb}} {\definition{to board, get on}}
\entry{sɨmell}\headword{sɨmell}{\pos{Noun}} {\definition{truck}}
\entry{trak}\headword{trak}{\pos{Noun}} {\definition{truck}}
\end{entrylist}

\section*{7.2.4.2.1 Boat}
\begin{entrylist}
\entry{bott}\headword{bott}{\pos{Noun}} {\definition{boat}}
\entry{dagal}\headword{dagal}{\pos{Transitive S verb}} {\definition{to board, get on}}
\entry{dinggi}\headword{dinggi}{\pos{Noun}} {\definition{dinghy}}
\entry{gall}\headword{gall}{\pos{Noun}} {\definition{canoe, boat}}
\entry{gall gullem}\headword{gall gullem}{\pos{Noun}} {\definition{canoe outrigger}}
\entry{gall ngattongag}\headword{gall ngattongag}{\pos{Noun}} {\definition{navigator (of a canoe)}}
\entry{gall onyang}\headword{gall onyang}{\pos{Noun}} {\definition{operator (of a canoe)}}
\entry{gllae}\headword{gllae}{\pos{Transitive S verb}} {\definition{to paddle; pedal}}
\entry{ine takmäll}\headword{ine takmäll}{\pos{Modifier}} {\definition{seasick}}
\entry{kakal}\headword{kakal}{\pos{Intransitive S verb}} {\definition{to enter, board, go in}}
\entry{kokto}\headword{kokto}{\pos{Transitive S verb}} {\definition{to bail (water)}}
\entry{kuddäb}\headword{kuddäb}{\pos{Noun}} {\definition{raft}}
\entry{mobera}\headword{mobera}{\pos{Noun}} {\definition{outrigger}}
\entry{sur}\headword{sur}{\pos{Noun}} {\definition{pushing tool}}
\entry{tap}\headword{tap}{\pos{Transitive A verb}} {\definition{to dock, land}}
\entry{widere}\headword{widere}{\pos{Noun}} {\definition{paddle, oar}}
\entry{wowo}\headword{wowo}{\pos{Transitive S verb}} {\definition{to clear floating grass by pushing through with a canoe}}
\end{entrylist}

\section*{7.2.4.2.2 Swim}
\begin{entrylist}
\entry{ddänddäm}\headword{ddänddäm}{\pos{Intransitive S verb}} {\definition{to drown}}
\entry{gllangglla}\headword{gllangglla}{\pos{Intransitive S verb}} {\definition{to swim}}
\entry{mänddmändd}\headword{mänddmändd}{\pos{Transitive S verb}} {\definition{to drown, struggle in water}}
\entry{peyam}\headword{peyam}{\pos{Intransitive S verb}} {\definition{to come out from water, surface}}
\end{entrylist}

\section*{7.2.4.2.3 Dive}
\begin{entrylist}
\entry{kämbäg}\headword{kämbäg}{\pos{Intransitive S verb}} {\definition{to dive}}
\end{entrylist}

\section*{7.2.4.3 Fly}
\begin{entrylist}
\entry{gädän}\headword{gädän}{\pos{Intransitive S verb}} {\definition{to land}}
\entry{paplläg}\headword{paplläg}{\pos{Intransitive S verb}} {\definition{to fly}}
\end{entrylist}

\section*{7.2.4.6 Way, route}
\begin{entrylist}
\entry{ngämae}\headword{ngämae}{\pos{Intransitive S verb}} {\definition{to go around, take the long way}}
\entry{nyongo}\headword{nyongo}{\pos{Noun}} {\definition{road, path, way}}
\end{entrylist}

\section*{7.2.4.7 Lose your way}
\begin{entrylist}
\entry{ddällombog}\headword{ddällombog}{\pos{Transitive S verb}} {\definition{to miss}}
\entry{udab}\headword{udab}{\pos{Transitive S verb}} {\definition{to disappear, get lost, go missing}}
\end{entrylist}

\section*{7.2.5 Accompany}
\begin{entrylist}
\entry{gulag}\headword{gulag}{\pos{Transitive A verb}} {\definition{to accompany}}
\end{entrylist}

\section*{7.2.5.2 Follow}
\begin{entrylist}
\entry{beyambäg}\headword{beyambäg}{\pos{Transitive S verb}} {\definition{to chase}}
\entry{ittall}\headword{ittall}{\pos{Verb}} {\definition{to follow closely, stalk (+ peyang)}}
\entry{koenmäll}\headword{koenmäll}{\pos{Transitive S verb}} {\definition{to chase}}
\entry{kollmäll}\headword{kollmäll}{\pos{Transitive S verb}} {\definition{to follow}}
\entry{trongg}\headword{trongg}{\pos{Transitive S verb}} {\definition{to follow}}
\end{entrylist}

\section*{7.2.6.1 Catch, capture}
\begin{entrylist}
\entry{llädäd}\headword{llädäd}{\pos{Transitive S verb}} {\definition{to grab, get, catch}}
\end{entrylist}

\section*{7.2.6.2 Prevent from moving}
\begin{entrylist}
\entry{papek}\headword{papek}{\pos{Transitive S verb}} {\definition{to block; close}}
\entry{tärangg}\headword{tärangg}{\pos{Transitive S verb}} {\definition{to stop, hold back}}
\end{entrylist}

\section*{7.2.6.3 Escape}
\begin{entrylist}
\entry{dindu}\headword{dindu}{\pos{Intransitive S verb}} {\definition{to run, flee, escape}}
\entry{gazen}\headword{gazen}{\pos{Transitive S verb}} {\definition{to come out, get out, exit, escape}}
\entry{kunen}\headword{kunen}{\pos{Intransitive S verb}} {\definition{to flee, scatter, run away}}
\end{entrylist}

\section*{7.2.7.1 Stop moving}
\begin{entrylist}
\entry{llätt}\headword{llätt}{\pos{Intransitive A verb}} {\definition{to stop, end, finish}}
\entry{wi}\headword{wi}{\pos{Intransitive S verb}} {\definition{to settle}}
\end{entrylist}

\section*{7.2.7.2 Stay, remain}
\begin{entrylist}
\entry{giddoll}\headword{giddoll}{\pos{Intransitive S verb}} {\definition{to stay, remain}}
\end{entrylist}

\section*{7.2.7.3 Wait}
\begin{entrylist}
\entry{tomon}\headword{tomon}{\pos{Transitive S verb}} {\definition{to wait}}
\entry{tomon}\headword{tomon}{\pos{Transitive S verb}} {\definition{to wait for, await}}
\end{entrylist}

\section*{7.3 Move something}
\begin{entrylist}
\entry{gllo}\headword{gllo}{\pos{Transitive S verb}} {\definition{to take out, remove}}
\end{entrylist}

\section*{7.3.1 Carry}
\begin{entrylist}
\entry{ellollo}\headword{ellollo}{\pos{Intransitive S verb}} {\definition{to put down load and rest}}
\entry{kangkäg}\headword{kangkäg}{\pos{Transitive S verb}} {\definition{to carry, bear (a load)}}
\entry{kapu}\headword{kapu}{\pos{Transitive A verb}} {\definition{to carry}}
\entry{lugoe}\headword{lugoe}{\pos{Transitive S verb}} {\definition{to drag}}
\entry{menttäg}\headword{menttäg}{\pos{Intransitive S verb}} {\definition{to shoulder, carry one one's shoulders or head}}
\entry{ngänttäg}\headword{ngänttäg}{\pos{Intransitive S verb}} {\definition{to bring, carry}}
\entry{ony}\headword{ony}{\pos{Transitive S verb}} {\definition{to carry; get; bring}}
\entry{täram}\headword{täram}{\pos{Transitive S verb}} {\definition{to lead, take, carry, collect}}
\end{entrylist}

\section*{7.3.1.1 Throw}
\begin{entrylist}
\entry{spun}\headword{spun}{\pos{Transitive S verb}} {\definition{to throw; shoot}}
\end{entrylist}

\section*{7.3.1.2 Catch}
\begin{entrylist}
\entry{gäddgädd}\headword{gäddgädd}{\pos{Transitive S verb}} {\definition{to catch}}
\entry{llädäd}\headword{llädäd}{\pos{Transitive S verb}} {\definition{to grab, get, catch}}
\end{entrylist}

\section*{7.3.1.3 Shake}
\begin{entrylist}
\entry{inngoeinngoe}\headword{inngoeinngoe}{\pos{Modifier}} {\definition{shaky}}
\entry{inungoe}\headword{inungoe}{\pos{Transitive S verb}} {\definition{to shake (when dancing)}}
\entry{pättapätte}\headword{pättapätte}{\pos{Transitive S verb}} {\definition{to shake off}}
\entry{ttaempäg}\headword{ttaempäg}{\pos{Transitive S verb}} {\definition{to shake hands with}}
\entry{wanwen}\headword{wanwen}{\pos{Transitive S verb}} {\definition{to shake, swing}}
\end{entrylist}

\section*{7.3.1.4 Knock over}
\begin{entrylist}
\entry{duab}\headword{duab}{\pos{Transitive S verb}} {\definition{to knock over; blow down}}
\end{entrylist}

\section*{7.3.2.4 Lift}
\begin{entrylist}
\entry{irängän}\headword{irängän}{\pos{Intransitive S verb}} {\definition{to get out, lift out}}
\entry{ngällbän}\headword{ngällbän}{\pos{Intransitive S verb}} {\definition{to lift}}
\entry{zizi}\headword{zizi}{\pos{Transitive S verb}} {\definition{to uncover, lift}}
\end{entrylist}

\section*{7.3.2.4.1 Hang}
\begin{entrylist}
\entry{itrae}\headword{itrae}{\pos{Intransitive S verb}} {\definition{to hang}}
\entry{kollwany}\headword{kollwany}{\pos{Transitive S verb}} {\definition{to hang}}
\entry{pällganen}\headword{pällganen}{\pos{Transitive S verb}} {\definition{to hang}}
\end{entrylist}

\section*{7.3.2.6 Put in}
\begin{entrylist}
\entry{gany}\headword{gany}{\pos{Transitive S verb}} {\definition{to plant, place in the ground, put in}}
\entry{tärak}\headword{tärak}{\pos{Transitive S verb}} {\definition{to put in}}
\entry{zan}\headword{zan}{\pos{Transitive S verb}} {\definition{to put in}}
\entry{zämae}\headword{zämae}{\pos{Transitive S verb}} {\definition{to pour, put, transfer}}
\end{entrylist}

\section*{7.3.2.7 Take something out of something}
\begin{entrylist}
\entry{sänasäne}\headword{sänasäne}{\pos{Transitive S verb}} {\definition{to take out}}
\end{entrylist}

\section*{7.3.2.8 Pull}
\begin{entrylist}
\entry{irängän}\headword{irängän}{\pos{Intransitive S verb}} {\definition{to get out, lift out}}
\entry{ngämar}\headword{ngämar}{\pos{Transitive S verb}} {\definition{to haul}}
\entry{nyongkoe}\headword{nyongkoe}{\pos{Transitive S verb}} {\definition{to pull}}
\entry{rullgoe}\headword{rullgoe}{\pos{Transitive S verb}} {\definition{to drag}}
\end{entrylist}

\section*{7.3.2.9 Push}
\begin{entrylist}
\entry{dämoe}\headword{dämoe}{\pos{Ditransitive S verb}} {\definition{to push}}
\entry{gängglläd}\headword{gängglläd}{\pos{Transitive S verb}} {\definition{to shove, push}}
\entry{imullgoe}\headword{imullgoe}{\pos{Transitive S verb}} {\definition{to drop, push, make fall}}
\entry{pendäg}\headword{pendäg}{\pos{Intransitive S verb}} {\definition{to push to the ground, jostle, trip}}
\entry{täträk}\headword{täträk}{\pos{Transitive S verb}} {\definition{to push in, push through}}
\entry{tɨtɨrɨk}\headword{tɨtɨrɨk}{\pos{Transitive S verb}} {\definition{to push through, make go through; undo by becoming released}}
\end{entrylist}

\section*{7.3.3 Take somewhere}
\begin{entrylist}
\entry{ony}\headword{ony}{\pos{Transitive S verb}} {\definition{to carry; get; bring}}
\entry{täram}\headword{täram}{\pos{Transitive S verb}} {\definition{to lead, take, carry, collect}}
\end{entrylist}

\section*{7.3.3.1 Take something from somewhere}
\begin{entrylist}
\entry{llädäd}\headword{llädäd}{\pos{Transitive S verb}} {\definition{to grab, get, catch}}
\end{entrylist}

\section*{7.3.3.2 Return something}
\begin{entrylist}
\entry{ngäs}\headword{ngäs}{\pos{Transitive S verb}} {\definition{to return, bring back}}
\end{entrylist}

\section*{7.3.3.3 Send}
\begin{entrylist}
\entry{dämoe}\headword{dämoe}{\pos{Ditransitive S verb}} {\definition{to send}}
\end{entrylist}

\section*{7.3.3.4 Chase away}
\begin{entrylist}
\entry{aoao}\headword{aoao}{\pos{Transitive A verb}} {\definition{to chase away (e.g. humans, dogs, chickens, but not wild animals)}}
\entry{beyambäg}\headword{beyambäg}{\pos{Transitive S verb}} {\definition{to chase}}
\entry{imonzimonz}\headword{imonzimonz}{\pos{Noun}} {\definition{tag (game)}}
\entry{koenmäll}\headword{koenmäll}{\pos{Transitive S verb}} {\definition{to chase}}
\entry{mangkimangki}\headword{mangkimangki}{\pos{Noun}} {\definition{type of game involving chasing}}
\entry{modgae}\headword{modgae}{\pos{Verb}} {\definition{to chase away, potentially an Agob word?}}
\end{entrylist}

\section*{7.3.4.1 Touch}
\begin{entrylist}
\entry{dämädämäll}\headword{dämädämäll}{\pos{Modifier}} {\definition{numb, paralyzed}}
\entry{imonz}\headword{imonz}{\pos{Transitive S verb}} {\definition{to touch}}
\entry{ngetae}\headword{ngetae}{\pos{Transitive S verb}} {\definition{to touch}}
\entry{ume ddäddäl}\headword{ume ddäddäl}{\pos{Transitive S verb}} {\definition{to kiss}}
\end{entrylist}

\section*{7.3.4.4 Hold}
\begin{entrylist}
\entry{ddänggab}\headword{ddänggab}{\pos{Transitive S verb}} {\definition{to hold, grab (with one's teeth)}}
\entry{mällam}\headword{mällam}{\pos{Transitive S verb}} {\definition{to hold; get, grab, catch}}
\entry{ngetae}\headword{ngetae}{\pos{Transitive S verb}} {\definition{to grip, hold onto}}
\entry{ngongop}\headword{ngongop}{\pos{Transitive S verb}} {\definition{to hug, embrace}}
\end{entrylist}

\section*{7.3.4.6 Support}
\begin{entrylist}
\entry{ngämingg}\headword{ngämingg}{\pos{Intransitive S verb}} {\definition{to help}}
\entry{udu}\headword{udu}{\pos{Noun}} {\definition{walking stick, cane, staff}}
\end{entrylist}

\section*{7.3.4.7 Extend}
\begin{entrylist}
\entry{gängglläd}\headword{gängglläd}{\pos{Transitive S verb}} {\definition{to extend (e.g. a garden)}}
\entry{ttällam}\headword{ttällam}{\pos{Transitive S verb}} {\definition{to extend, stretch out, reach out, put out}}
\end{entrylist}

\section*{7.3.6 Open}
\begin{entrylist}
\entry{bällgab}\headword{bällgab}{\pos{Transitive S verb}} {\definition{to open (e.g. eyes, flowers)}}
\entry{bänamb}\headword{bänamb}{\pos{Transitive S verb}} {\definition{to open (something folded, e.g. book, mouth)}}
\entry{dalab}\headword{dalab}{\pos{Transitive S verb}} {\definition{to open, pierce, make a hole}}
\entry{gallab}\headword{gallab}{\pos{Transitive S verb}} {\definition{to open (one's mouth)}}
\entry{pallam}\headword{pallam}{\pos{Transitive S verb}} {\definition{to cut open}}
\entry{ttalam}\headword{ttalam}{\pos{Intransitive S verb}} {\definition{to split, crack}}
\entry{ttapeyam}\headword{ttapeyam}{\pos{Transitive S verb}} {\definition{to open (something long, e.g. door, book)}}
\entry{ttattlläb}\headword{ttattlläb}{\pos{Intransitive S verb}} {\definition{to open}}
\entry{tɨram}\headword{tɨram}{\pos{Transitive S verb}} {\definition{to open}}
\end{entrylist}

\section*{7.3.6.1 Shut, close}
\begin{entrylist}
\entry{papek}\headword{papek}{\pos{Transitive S verb}} {\definition{to block; close}}
\entry{sae}\headword{sae}{\pos{Transitive S verb}} {\definition{to close, cover}}
\end{entrylist}

\section*{7.3.6.2 Block, dam up}
\begin{entrylist}
\entry{ngättangätta}\headword{ngättangätta}{\pos{Transitive S verb}} {\definition{to block, obscure}}
\entry{papek}\headword{papek}{\pos{Noun}} {\definition{dam, blockage; wall}}
\entry{ttäbattäbe}\headword{ttäbattäbe}{\pos{Transitive S verb}} {\definition{to block}}
\end{entrylist}

\section*{7.3.7 Cover}
\begin{entrylist}
\entry{bengae}\headword{bengae}{\pos{Transitive S verb}} {\definition{to roof, cover}}
\entry{inam}\headword{inam}{\pos{Transitive S verb}} {\definition{to cover}}
\entry{konakone}\headword{konakone}{\pos{Transitive S verb}} {\definition{to cover}}
\entry{sae}\headword{sae}{\pos{Transitive S verb}} {\definition{to close, cover}}
\entry{wɨndwɨnd}\headword{wɨndwɨnd}{\pos{Transitive S verb}} {\definition{to cover, bury}}
\end{entrylist}

\section*{7.3.7.1 Uncover}
\begin{entrylist}
\entry{zizi}\headword{zizi}{\pos{Transitive S verb}} {\definition{to uncover, lift}}
\end{entrylist}

\section*{7.3.7.2 Wrap}
\begin{entrylist}
\entry{kaen}\headword{kaen}{\pos{Transitive A verb}} {\definition{to wrap up}}
\entry{llatat}\headword{llatat}{\pos{Transitive S verb}} {\definition{to twist, wrap}}
\entry{nganae}\headword{nganae}{\pos{Transitive S verb}} {\definition{to coil, go around}}
\entry{tok}\headword{tok}{\pos{Transitive A verb}} {\definition{to wrap}}
\entry{zigae}\headword{zigae}{\pos{Transitive S verb}} {\definition{to wrap}}
\end{entrylist}

\section*{7.3.7.3 Spread, smear}
\begin{entrylist}
\entry{pädrall}\headword{pädrall}{\pos{Intransitive S verb}} {\definition{to spread (out), scatter, sow}}
\entry{tater}\headword{tater}{\pos{Transitive S verb}} {\definition{to spread out}}
\entry{tergony}\headword{tergony}{\pos{Transitive S verb}} {\definition{to spread, unfold}}
\end{entrylist}

\section*{7.3.8 Transport}
\begin{entrylist}
\entry{gall onyang}\headword{gall onyang}{\pos{Noun}} {\definition{operator (of a canoe)}}
\entry{gllae}\headword{gllae}{\pos{Transitive S verb}} {\definition{to paddle; pedal}}
\entry{nyongo taempägag}\headword{nyongo taempägag}{\pos{Noun}} {\definition{navigator (of a canoe)}}
\entry{plen}\headword{plen}{\pos{Noun}} {\definition{airplane}}
\entry{plenngätt}\headword{plenngätt}{\pos{Noun}} {\definition{airstrip}}
\entry{sämell}\headword{sämell}{\pos{Noun}} {\definition{truck}}
\end{entrylist}

\section*{7.4.1 Give, hand to}
\begin{entrylist}
\entry{pentae}\headword{pentae}{\pos{Intransitive S verb}} {\definition{to transfer, transmit, spread, pass on, pass down}}
\entry{singoll}\headword{singoll}{\pos{Transitive A verb}} {\definition{to give, provide, share}}
\entry{ttongg}\headword{ttongg}{\pos{Ditransitive S verb}} {\definition{to give}}
\entry{ttällam}\headword{ttällam}{\pos{Transitive S verb}} {\definition{to pass, hand}}
\end{entrylist}

\section*{7.4.3 Get}
\begin{entrylist}
\entry{llädäd}\headword{llädäd}{\pos{Transitive S verb}} {\definition{to grab, get, catch}}
\entry{mällam}\headword{mällam}{\pos{Transitive S verb}} {\definition{to hold; get, grab, catch}}
\entry{ngällbän}\headword{ngällbän}{\pos{Intransitive S verb}} {\definition{to get, take}}
\entry{ony}\headword{ony}{\pos{Transitive S verb}} {\definition{to carry; get; bring}}
\entry{yatän}\headword{yatän}{\pos{Transitive S verb}} {\definition{to fetch (water)}}
\end{entrylist}

\section*{7.4.5.1 Leave something}
\begin{entrylist}
\entry{inttemängg}\headword{inttemängg}{\pos{Intransitive S verb}} {\definition{to leave, see off, release, set free}}
\entry{wanseg}\headword{wanseg}{\pos{Transitive S verb}} {\definition{to put, place, set aside, leave}}
\end{entrylist}

\section*{7.4.5.2 Throw away}
\begin{entrylist}
\entry{rabis}\headword{rabis}{\pos{Noun}} {\definition{rubbish, trash}}
\entry{tot}\headword{tot}{\pos{Noun}} {\definition{rubbish, trash, junk}}
\end{entrylist}

\section*{7.5.1 Gather}
\begin{entrylist}
\entry{ddän}\headword{ddän}{\pos{Transitive S verb}} {\definition{to pick, gather, harvest}}
\entry{dumdum}\headword{dumdum}{\pos{Transitive S verb}} {\definition{to surround}}
\entry{gɨg}\headword{gɨg}{\pos{Transitive S verb}} {\definition{to collect ants}}
\entry{klloklloe}\headword{klloklloe}{\pos{Transitive S verb}} {\definition{to gather, join, come together}}
\entry{klloklloe}\headword{klloklloe}{\pos{Transitive S verb}} {\definition{to gather, bring together, mix}}
\entry{tukpi}\headword{tukpi}{\pos{Transitive A verb}} {\definition{to heap, pile; gather, collect}}
\entry{tum}\headword{tum}{\pos{Noun}} {\definition{heap}}
\entry{tumtum}\headword{tumtum}{\pos{Transitive/Intransitive A verb}} {\definition{to gather}}
\entry{umaem}\headword{umaem}{\pos{Intransitive S verb}} {\definition{to gather}}
\entry{zämängg}\headword{zämängg}{\pos{Intransitive S verb}} {\definition{to be in heaps}}
\end{entrylist}

\section*{7.5.1.1 Separate, scatter}
\begin{entrylist}
\entry{bällge}\headword{bällge}{\pos{Intransitive S verb}} {\definition{to spread, scatter, move away}}
\entry{bällängg}\headword{bällängg}{\pos{Intransitive S verb}} {\definition{to split up, separate}}
\entry{pädrall}\headword{pädrall}{\pos{Intransitive S verb}} {\definition{to spread (out), scatter, disperse}}
\entry{ttaempäg}\headword{ttaempäg}{\pos{Transitive S verb}} {\definition{to seperate, divorce}}
\entry{ttoengg}\headword{ttoengg}{\pos{Intransitive S verb}} {\definition{to split up, scatter}}
\end{entrylist}

\section*{7.5.2 Join, attach}
\begin{entrylist}
\entry{ddogoll}\headword{ddogoll}{\pos{Transitive S verb}} {\definition{to join}}
\entry{klloklloe}\headword{klloklloe}{\pos{Transitive S verb}} {\definition{to gather, join, come together}}
\entry{ngetae}\headword{ngetae}{\pos{Transitive S verb}} {\definition{to unite, join}}
\end{entrylist}

\section*{7.5.2.2 Stick together}
\begin{entrylist}
\entry{ddogoll}\headword{ddogoll}{\pos{Transitive S verb}} {\definition{to stick}}
\end{entrylist}

\section*{7.5.2.4 Remove, take apart}
\begin{entrylist}
\entry{ddäddäg}\headword{ddäddäg}{\pos{Transitive S verb}} {\definition{to peel, remove}}
\entry{gäglib}\headword{gäglib}{\pos{Transitive S verb}} {\definition{to pluck}}
\entry{gälbän}\headword{gälbän}{\pos{Transitive S verb}} {\definition{to defeather, depilate, remove hair; remove teeth}}
\entry{kän}\headword{kän}{\pos{Transitive S verb}} {\definition{to remove, take out, take off, undo}}
\end{entrylist}

\section*{7.5.3 Mix}
\begin{entrylist}
\entry{klloklloe}\headword{klloklloe}{\pos{Transitive S verb}} {\definition{to gather, bring together, mix}}
\entry{kul}\headword{kul}{\pos{Transitive A verb}} {\definition{to smash food}}
\entry{miks}\headword{miks}{\pos{Transitive A verb}} {\definition{to mix}}
\entry{nyägae}\headword{nyägae}{\pos{Intransitive S verb}} {\definition{to stir}}
\end{entrylist}

\section*{7.5.4 Tie}
\begin{entrylist}
\entry{ddäddäg}\headword{ddäddäg}{\pos{Transitive S verb}} {\definition{to bind}}
\entry{mäkäp}\headword{mäkäp}{\pos{Noun}} {\definition{knot}}
\entry{mäkäp}\headword{mäkäp}{\pos{Transitive A verb}} {\definition{to knot}}
\entry{mällamälla}\headword{mällamälla}{\pos{Transitive S verb}} {\definition{to tie}}
\entry{pitkae}\headword{pitkae}{\pos{Transitive S verb}} {\definition{to untie}}
\entry{ttotto}\headword{ttotto}{\pos{Transitive S verb}} {\definition{to tie}}
\entry{tɨtɨrɨk}\headword{tɨtɨrɨk}{\pos{Transitive S verb}} {\definition{to untie}}
\end{entrylist}

\section*{7.5.4.1 Rope, string}
\begin{entrylist}
\entry{amamär}\headword{amamär}{\pos{Noun}} {\definition{woven rope}}
\entry{mamlla}\headword{mamlla}{\pos{Noun}} {\definition{rope, string}}
\entry{mantär}\headword{mantär}{\pos{Noun}} {\definition{type of flat woven rope made from tree bark.}}
\entry{pitt}\headword{pitt}{\pos{Noun}} {\definition{arrowhead hafting string}}
\entry{rop}\headword{rop}{\pos{Noun}} {\definition{rope}}
\end{entrylist}

\section*{7.5.4.2 Tangle}
\begin{entrylist}
\entry{opop}\headword{opop}{\pos{Verb}} {\definition{tangle}}
\end{entrylist}

\section*{7.5.5.1 Disorganized}
\begin{entrylist}
\entry{ttagbeag}\headword{ttagbeag}{\pos{Modifier}} {\definition{disorganized, careless}}
\end{entrylist}

\section*{7.5.9 Put}
\begin{entrylist}
\entry{däba}\headword{däba}{\pos{Transitive S verb}} {\definition{to place, put}}
\entry{wanseg}\headword{wanseg}{\pos{Transitive S verb}} {\definition{to put, place, set aside, leave}}
\end{entrylist}

\section*{7.5.9.1 Load, pile}
\begin{entrylist}
\entry{buddobuddog}\headword{buddobuddog}{\pos{Adverb}} {\definition{carrying a load}}
\end{entrylist}

\section*{7.5.9.2 Fill, cover}
\begin{entrylist}
\entry{dändär}\headword{dändär}{\pos{Transitive S verb}} {\definition{to stuff}}
\end{entrylist}

\section*{7.6 Hide}
\begin{entrylist}
\entry{lläblläb}\headword{lläblläb}{\pos{Transitive S verb}} {\definition{to hide}}
\entry{togol}\headword{togol}{\pos{Transitive S verb}} {\definition{to hide; go away to have sex secretly}}
\end{entrylist}

\section*{7.6.1 Search}
\begin{entrylist}
\entry{ngangleb}\headword{ngangleb}{\pos{Transitive S verb}} {\definition{to look for, search for}}
\entry{yagyeg}\headword{yagyeg}{\pos{Transitive S verb}} {\definition{to search, look for}}
\end{entrylist}

\section*{7.6.2 Find}
\begin{entrylist}
\entry{bällabälle}\headword{bällabälle}{\pos{Transitive S verb}} {\definition{to find}}
\end{entrylist}

\section*{7.6.3 Lose, misplace}
\begin{entrylist}
\entry{udab}\headword{udab}{\pos{Transitive S verb}} {\definition{to lose}}
\end{entrylist}

\section*{7.7 Physical impact}
\begin{entrylist}
\entry{pitatep}\headword{pitatep}{\pos{Verb}} {\definition{to lift an animal or person and throw them to the ground, hit them hard against the ground}}
\end{entrylist}

\section*{7.7.1 Hit}
\begin{entrylist}
\entry{gäz}\headword{gäz}{\pos{Transitive S verb}} {\definition{to hit, beat}}
\entry{kuimang}\headword{kuimang}{\pos{Transitive A verb}} {\definition{to knock on}}
\entry{metmäll}\headword{metmäll}{\pos{Transitive S verb}} {\definition{to beat, flog, hit}}
\entry{papa}\headword{papa}{\pos{Transitive A verb}} {\definition{to hit, beat}}
\entry{pirik}\headword{pirik}{\pos{Noun}} {\definition{baton, stick}}
\entry{wabeb}\headword{wabeb}{\pos{Transitive S verb}} {\definition{to beat, smash, pound}}
\end{entrylist}

\section*{7.7.2 Aim at a target}
\begin{entrylist}
\entry{ttalängg}\headword{ttalängg}{\pos{Intransitive S verb}} {\definition{to aim}}
\end{entrylist}

\section*{7.7.4 Press}
\begin{entrylist}
\entry{inam}\headword{inam}{\pos{Transitive S verb}} {\definition{to weigh down, press down}}
\entry{kängkäm}\headword{kängkäm}{\pos{Transitive S verb}} {\definition{to squeeze, press}}
\entry{mälmäl}\headword{mälmäl}{\pos{Transitive S verb}} {\definition{to squeeze}}
\entry{yid}\headword{yid}{\pos{Noun}} {\definition{liquid extracted from a plant}}
\end{entrylist}

\section*{7.7.5 Rub}
\begin{entrylist}
\entry{säsäs}\headword{säsäs}{\pos{Intransitive S verb}} {\definition{to rub}}
\end{entrylist}

\section*{7.8 Divide into pieces}
\begin{entrylist}
\entry{pallängkmeny}\headword{pallängkmeny}{\pos{Transitive S verb}} {\definition{to divide}}
\entry{täräp}\headword{täräp}{\pos{Transitive A verb}} {\definition{to portion, share, split, divide}}
\end{entrylist}

\section*{7.8.1 Break}
\begin{entrylist}
\entry{kalläntäg}\headword{kalläntäg}{\pos{Transitive S verb}} {\definition{to split}}
\entry{kädab}\headword{kädab}{\pos{Transitive S verb}} {\definition{to break, split}}
\entry{käkäp}\headword{käkäp}{\pos{Noun}} {\definition{half}}
\entry{llollom}\headword{llollom}{\pos{Intransitive S verb}} {\definition{to break, be damaged}}
\entry{pape}\headword{pape}{\pos{Transitive S verb}} {\definition{to smash, crush (something soft, but not a flower)}}
\entry{papllek}\headword{papllek}{\pos{Transitive S verb}} {\definition{to chop}}
\entry{papälläk}\headword{papälläk}{\pos{Transitive S verb}} {\definition{to split, chop}}
\entry{pällkam}\headword{pällkam}{\pos{Transitive S verb}} {\definition{to split, break}}
\entry{tarketarke}\headword{tarketarke}{\pos{Modifier}} {\definition{brittle}}
\entry{ttäkam}\headword{ttäkam}{\pos{Transitive S verb}} {\definition{to break, snap}}
\entry{ttängkag}\headword{ttängkag}{\pos{Transitive S verb}} {\definition{to break with force}}
\entry{ttäpen}\headword{ttäpen}{\pos{Transitive S verb}} {\definition{to snap, break}}
\end{entrylist}

\section*{7.8.2 Crack}
\begin{entrylist}
\entry{ttalam}\headword{ttalam}{\pos{Intransitive S verb}} {\definition{to split, crack}}
\end{entrylist}

\section*{7.8.3 Cut}
\begin{entrylist}
\entry{bänybäny}\headword{bänybäny}{\pos{Transitive S verb}} {\definition{to cut, slice (flesh)}}
\entry{ddaddu}\headword{ddaddu}{\pos{Transitive S verb}} {\definition{to remove a shoot}}
\entry{gallwab}\headword{gallwab}{\pos{Transitive S verb}} {\definition{to prune}}
\entry{kaekep}\headword{kaekep}{\pos{Transitive S verb}} {\definition{to struggle to cut (e.g. with a dull knife)}}
\entry{kam}\headword{kam}{\pos{Transitive S verb}} {\definition{to cut (e.g. meat, skin)}}
\entry{koko}\headword{koko}{\pos{Transitive S verb}} {\definition{to cut (flesh or meat), butcher}}
\entry{llɨtɨt}\headword{llɨtɨt}{\pos{Transitive S verb}} {\definition{to butcher, cut}}
\entry{pallam}\headword{pallam}{\pos{Transitive S verb}} {\definition{to cut open}}
\entry{plengg}\headword{plengg}{\pos{Intransitive S verb}} {\definition{to cut down, cut off}}
\entry{potpot}\headword{potpot}{\pos{Transitive S verb}} {\definition{to slice open}}
\entry{ttatta}\headword{ttatta}{\pos{Transitive S verb}} {\definition{to chop a tree}}
\entry{ttäkoe}\headword{ttäkoe}{\pos{Transitive S verb}} {\definition{to chop, cut down, mow; shave}}
\entry{tätäräp}\headword{tätäräp}{\pos{Transitive S verb}} {\definition{to cut}}
\end{entrylist}

\section*{7.8.4 Tear, rip}
\begin{entrylist}
\entry{pinsäg}\headword{pinsäg}{\pos{Intransitive S verb}} {\definition{to tear}}
\entry{pisam}\headword{pisam}{\pos{Intransitive S verb}} {\definition{to tear (of something thin)}}
\entry{pisam}\headword{pisam}{\pos{Intransitive S verb}} {\definition{to tear (something thin)}}
\entry{ttättälläg}\headword{ttättälläg}{\pos{Transitive S verb}} {\definition{to tear, tear up}}
\end{entrylist}

\section*{7.8.5 Make hole, opening}
\begin{entrylist}
\entry{dalab}\headword{dalab}{\pos{Transitive S verb}} {\definition{to open, pierce, make a hole}}
\entry{papoe}\headword{papoe}{\pos{Transitive S verb}} {\definition{to pierce, make a hole}}
\entry{po}\headword{po}{\pos{Transitive S verb}} {\definition{to pierce}}
\entry{pop}\headword{pop}{\pos{Noun}} {\definition{hole}}
\entry{täratäre}\headword{täratäre}{\pos{Transitive S verb}} {\definition{to dig out, hollow out}}
\entry{tɨt}\headword{tɨt}{\pos{Transitive S verb}} {\definition{to hollow out, dig out}}
\end{entrylist}

\section*{7.8.6 Dig}
\begin{entrylist}
\entry{gllae}\headword{gllae}{\pos{Transitive S verb}} {\definition{to dig, spade}}
\entry{ibae}\headword{ibae}{\pos{Transitive S verb}} {\definition{to dig using one's nose}}
\entry{idän}\headword{idän}{\pos{Transitive S verb}} {\definition{to pick, harvest; dig up, uproot}}
\entry{kɨllkɨll}\headword{kɨllkɨll}{\pos{Transitive S verb}} {\definition{to dig}}
\entry{lläpän}\headword{lläpän}{\pos{Transitive S verb}} {\definition{to dig, harvest, unearth (a tuber or corm)}}
\entry{täratäre}\headword{täratäre}{\pos{Transitive S verb}} {\definition{to dig out, hollow out}}
\entry{tɨt}\headword{tɨt}{\pos{Transitive S verb}} {\definition{to hollow out, dig out}}
\end{entrylist}

\section*{7.9.3 Destroy}
\begin{entrylist}
\entry{ddaddällɨg}\headword{ddaddällɨg}{\pos{Transitive S verb}} {\definition{to destroy}}
\entry{llakllek}\headword{llakllek}{\pos{Transitive S verb}} {\definition{to destroy}}
\end{entrylist}

\section*{7.9.4 Repair}
\begin{entrylist}
\entry{ddogoll}\headword{ddogoll}{\pos{Transitive S verb}} {\definition{to put together}}
\entry{mälamäle}\headword{mälamäle}{\pos{Transitive S verb}} {\definition{to patch}}
\entry{pittpitt}\headword{pittpitt}{\pos{Transitive S verb}} {\definition{to sew, stitch}}
\end{entrylist}

\section*{8.1 Quantity}
\begin{entrylist}
\entry{aoli}\headword{aoli}{\pos{Quantifier}} {\definition{multiple, several (i.e. more than one but not necessarily many); few (esp. with the restrictive clitic)}}
\end{entrylist}

\section*{8.1.1 Number}
\begin{entrylist}
\entry{andred}\headword{andred}{\pos{Numeral}} {\definition{hundred}}
\entry{apte gabän}\headword{apte gabän}{\pos{Numeral}} {\definition{fourteen (body counting numeral)}}
\entry{apte kllatolma}\headword{apte kllatolma}{\pos{Numeral}} {\definition{seventeen (body counting numeral)}}
\entry{apte matta}\headword{apte matta}{\pos{Numeral}} {\definition{twelve (body counting numeral)}}
\entry{apte mända}\headword{apte mända}{\pos{Numeral}} {\definition{fifteen (body counting numeral)}}
\entry{apte mätkin}\headword{apte mätkin}{\pos{Numeral}} {\definition{eighteen (body counting numeral)}}
\entry{apte ngam}\headword{apte ngam}{\pos{Numeral}} {\definition{eleven (body counting numeral)}}
\entry{apte ttang kum}\headword{apte ttang kum}{\pos{Numeral}} {\definition{thirteen (body counting numeral)}}
\entry{apte tupi}\headword{apte tupi}{\pos{Numeral}} {\definition{sixteen (body counting numeral)}}
\entry{apte tärangesa}\headword{apte tärangesa}{\pos{Numeral}} {\definition{nineteen (body counting numeral)}}
\entry{damona}\headword{damona}{\pos{Numeral}} {\definition{1296 (yam counting numeral; 6^4)}}
\entry{ddäll}\headword{ddäll}{\pos{Numeral}} {\definition{ten (lit. chest; body counting numeral)}}
\entry{eit}\headword{eit}{\pos{Numeral}} {\definition{eight (English numeral; also general)}}
\entry{gabän}\headword{gabän}{\pos{Numeral}} {\definition{six (lit. wrist; body counting numeral)}}
\entry{komlla}\headword{komlla}{\pos{Numeral}} {\definition{two (yam counting numeral; also general)}}
\entry{komlla komlla}\headword{komlla komlla}{\pos{Numeral}} {\definition{four (yam counting numeral; 2+2)}}
\entry{komlla komlla ttongo dduma}\headword{komlla komlla ttongo dduma}{\pos{Numeral}} {\definition{five (yam counting numeral; 2+2+1)}}
\entry{komlla putt}\headword{komlla putt}{\pos{Numeral}} {\definition{twelve (yam counting numeral; 2×6)}}
\entry{komlla yangete}\headword{komlla yangete}{\pos{Numeral}} {\definition{doubling two, e.g. shoes tied together}}
\entry{kumuddäga}\headword{kumuddäga}{\pos{Numeral}} {\definition{three (yam counting numeral; also general)}}
\entry{källatolma}\headword{källatolma}{\pos{Numeral}} {\definition{three (lit. middle finger; body counting numeral)}}
\entry{manglle kämang}\headword{manglle kämang}{\pos{Modifier}} {\definition{always together}}
\entry{mangllemanglle}\headword{mangllemanglle}{\pos{Modifier}} {\definition{always together}}
\entry{matta}\headword{matta}{\pos{Numeral}} {\definition{eight (lit. shoulder; body counting numeral)}}
\entry{mällamollang}\headword{mällamollang}{\pos{}} {\definition{more than two couples}}
\entry{mända}\headword{mända}{\pos{Numeral}} {\definition{five (lit. thumb; body counting numeral)}}
\entry{mätkin}\headword{mätkin}{\pos{Numeral}} {\definition{two (lit. ring finger; body counting numeral)}}
\entry{naen}\headword{naen}{\pos{Numeral}} {\definition{nine (English numeral; also general)}}
\entry{ngam}\headword{ngam}{\pos{Numeral}} {\definition{nine (lit. breast; body counting numeral)}}
\entry{paeb}\headword{paeb}{\pos{Numeral}} {\definition{five (English numeral)}}
\entry{po}\headword{po}{\pos{Numeral}} {\definition{four (English numeral; also general)}}
\entry{poti}\headword{poti}{\pos{Numeral}} {\definition{forty}}
\entry{purta}\headword{purta}{\pos{Noun}} {\definition{six groups of six yams}}
\entry{putt}\headword{putt}{\pos{Numeral}} {\definition{six (yam counting numeral; 6^1)}}
\entry{pärta}\headword{pärta}{\pos{Numeral}} {\definition{thirty-six (yam counting numeral; 6^2)}}
\entry{seben}\headword{seben}{\pos{Numeral}} {\definition{seven (English numeral; also general)}}
\entry{siks}\headword{siks}{\pos{Numeral}} {\definition{six (English numeral; also general)}}
\entry{siksti}\headword{siksti}{\pos{Numeral}} {\definition{sixty}}
\entry{taosen}\headword{taosen}{\pos{Numeral}} {\definition{thousand}}
\entry{taromba}\headword{taromba}{\pos{Numeral}} {\definition{216 (yam counting numeral; 6^3)}}
\entry{ten}\headword{ten}{\pos{Numeral}} {\definition{ten (English numeral, also general)}}
\entry{teti}\headword{teti}{\pos{Numeral}} {\definition{thirty}}
\entry{ttang kum}\headword{ttang kum}{\pos{Numeral}} {\definition{seven (lit. elbow; body counting numeral)}}
\entry{twelb}\headword{twelb}{\pos{Numeral}} {\definition{twelve (English numeral)}}
\entry{twenti}\headword{twenti}{\pos{Numeral}} {\definition{twenty}}
\entry{tärangesa}\headword{tärangesa}{\pos{Numeral}} {\definition{one (lit. pinky; body counting numeral)}}
\entry{waramakae}\headword{waramakae}{\pos{Numeral}} {\definition{7776 (yam counting numeral; 6^5)}}
\end{entrylist}

\section*{8.1.1.1.1 One}
\begin{entrylist}
\entry{apte}\headword{apte}{\pos{Quantifier}} {\definition{one side, one end, one half, one (of a pair)}}
\end{entrylist}

\section*{8.1.2 Count}
\begin{entrylist}
\entry{ngättangätte}\headword{ngättangätte}{\pos{Transitive S verb}} {\definition{to count}}
\end{entrylist}

\section*{8.1.2.4 Multiply numbers}
\begin{entrylist}
\entry{bombllo}\headword{bombllo}{\pos{Transitive/Intransitive A verb}} {\definition{to increase, proliferate, multiply}}
\entry{bombllo}\headword{bombllo}{\pos{Transitive/Intransitive A verb}} {\definition{to increase, proliferate, multiply}}
\end{entrylist}

\section*{8.1.3.1 Many, much}
\begin{entrylist}
\entry{ai}\headword{ai}{\pos{Quantifier}} {\definition{plenty, abundant}}
\entry{gollaeb}\headword{gollaeb}{\pos{Quantifier}} {\definition{many}}
\entry{kemibi}\headword{kemibi}{\pos{Quantifier}} {\definition{many}}
\entry{tumang}\headword{tumang}{\pos{Quantifier}} {\definition{plenty, many}}
\entry{ulle}\headword{ulle}{\pos{Quantifier}} {\definition{a lot, abundant, plentiful (for uncountable nouns)}}
\entry{yuwog}\headword{yuwog}{\pos{Quantifier}} {\definition{many, a lot of (for countable nouns)}}
\end{entrylist}

\section*{8.1.3.2 Few, little}
\begin{entrylist}
\entry{aolidae}\headword{aolidae}{\pos{Quantifier}} {\definition{few}}
\entry{kälae}\headword{kälae}{\pos{Quantifier}} {\definition{little, few}}
\entry{ttongottongo alle}\headword{ttongottongo alle}{\pos{Quantifier}} {\definition{few}}
\end{entrylist}

\section*{8.1.3.3 Group of things}
\begin{entrylist}
\entry{däg}\headword{däg}{\pos{Noun}} {\definition{hand, group, bunch, set}}
\entry{gul}\headword{gul}{\pos{Noun}} {\definition{crowd, group, mob; school (of fish)}}
\entry{patkoll}\headword{patkoll}{\pos{Noun}} {\definition{bundle}}
\entry{täräk}\headword{täräk}{\pos{Noun}} {\definition{bundle}}
\end{entrylist}

\section*{8.1.4 More}
\begin{entrylist}
\entry{tukituki}\headword{tukituki}{\pos{Adverb}} {\definition{more than, above, over}}
\end{entrylist}

\section*{8.1.4.2 Increase}
\begin{entrylist}
\entry{bombllo}\headword{bombllo}{\pos{Transitive/Intransitive A verb}} {\definition{to increase, proliferate, multiply}}
\entry{bombllo}\headword{bombllo}{\pos{Transitive/Intransitive A verb}} {\definition{to increase, proliferate, multiply}}
\end{entrylist}

\section*{8.1.5 All}
\begin{entrylist}
\entry{bällam}\headword{bällam}{\pos{Quantifier}} {\definition{every}}
\entry{tämamae}\headword{tämamae}{\pos{Quantifier}} {\definition{all, every}}
\end{entrylist}

\section*{8.1.5.1 Some}
\begin{entrylist}
\entry{ddob}\headword{ddob}{\pos{Quantifier}} {\definition{some}}
\end{entrylist}

\section*{8.1.5.6 Almost}
\begin{entrylist}
\entry{aoli}\headword{aoli}{\pos{Adverb}} {\definition{almost, nearly}}
\end{entrylist}

\section*{8.1.5.8 Exact}
\begin{entrylist}
\entry{abal}\headword{abal}{\pos{Adverb}} {\definition{exactly, just}}
\end{entrylist}

\section*{8.1.5.8.1 Approximate}
\begin{entrylist}
\entry{ngata me}\headword{ngata me}{\pos{Adverb}} {\definition{around, about, approximately}}
\end{entrylist}

\section*{8.1.6 Whole, complete}
\begin{entrylist}
\entry{tämamae}\headword{tämamae}{\pos{Modifier}} {\definition{whole, entire}}
\entry{ulle}\headword{ulle}{\pos{Modifier}} {\definition{entire, whole}}
\end{entrylist}

\section*{8.1.6.2 Piece}
\begin{entrylist}
\entry{kodor}\headword{kodor}{\pos{Noun}} {\definition{piece, lump}}
\entry{kädakäde}\headword{kädakäde}{\pos{Noun}} {\definition{pieces}}
\entry{pallkoll}\headword{pallkoll}{\pos{Noun}} {\definition{piece}}
\entry{pisepise}\headword{pisepise}{\pos{Noun}} {\definition{small pieces}}
\entry{pällkapällkae}\headword{pällkapällkae}{\pos{Noun}} {\definition{pieces, shards}}
\entry{tot}\headword{tot}{\pos{Noun}} {\definition{piece}}
\entry{tärpoll}\headword{tärpoll}{\pos{Noun}} {\definition{piece}}
\end{entrylist}

\section*{8.1.7 Enough}
\begin{entrylist}
\entry{mullae}\headword{mullae}{\pos{Modifier}} {\definition{enough}}
\entry{yuwoyuwog}\headword{yuwoyuwog}{\pos{Adverb}} {\definition{needlessly, excessively}}
\end{entrylist}

\section*{8.1.7.3 Need}
\begin{entrylist}
\entry{abo}\headword{abo}{\pos{TAM particle}} {\definition{must, necessative mood}}
\entry{bänga}\headword{bänga}{\pos{TAM particle}} {\definition{should}}
\end{entrylist}

\section*{8.1.7.4 Remain, remainder}
\begin{entrylist}
\entry{kakab}\headword{kakab}{\pos{Noun}} {\definition{leftovers, remainder, remnant}}
\entry{kakab}\headword{kakab}{\pos{Modifier}} {\definition{leftover, remaining}}
\entry{tärpa}\headword{tärpa}{\pos{Noun}} {\definition{leftovers}}
\end{entrylist}

\section*{8.1.8 Full}
\begin{entrylist}
\entry{bodo}\headword{bodo}{\pos{Modifier}} {\definition{full}}
\entry{dändär}\headword{dändär}{\pos{Transitive S verb}} {\definition{to stuff}}
\entry{ttämattäme}\headword{ttämattäme}{\pos{Intransitive S verb}} {\definition{to be crowded}}
\end{entrylist}

\section*{8.2 Big}
\begin{entrylist}
\entry{märäl}\headword{märäl}{\pos{Noun}} {\definition{size}}
\entry{ulle}\headword{ulle}{\pos{Modifier}} {\definition{big, large, great}}
\end{entrylist}

\section*{8.2.1 Small}
\begin{entrylist}
\entry{RED~}\headword{RED~}{\pos{Nominal}} {\definition{attaches to a nominal to derive a new nominal with a smaller referent}}
\entry{kälae}\headword{kälae}{\pos{Modifier}} {\definition{small, little}}
\entry{kälsäre}\headword{kälsäre}{\pos{Modifier}} {\definition{small, little}}
\entry{mermer}\headword{mermer}{\pos{Verb}} {\definition{to shrink, shrivel}}
\entry{minyminy}\headword{minyminy}{\pos{Modifier}} {\definition{small pieces; crumbs}}
\end{entrylist}

\section*{8.2.2 Long}
\begin{entrylist}
\entry{dang}\headword{dang}{\pos{Noun}} {\definition{length (of a house)}}
\entry{tupi}\headword{tupi}{\pos{Modifier}} {\definition{long}}
\entry{tupitupi}\headword{tupitupi}{\pos{Modifier}} {\definition{nonsingular form of tupi}}
\entry{tutpi}\headword{tutpi}{\pos{Modifier}} {\definition{}}
\entry{ulle}\headword{ulle}{\pos{Modifier}} {\definition{long; tall}}
\end{entrylist}

\section*{8.2.2.1 Short, not long}
\begin{entrylist}
\entry{pättäk}\headword{pättäk}{\pos{Modifier}} {\definition{short}}
\entry{tubu}\headword{tubu}{\pos{Modifier}} {\definition{short}}
\end{entrylist}

\section*{8.2.2.2 Tall}
\begin{entrylist}
\entry{tupi}\headword{tupi}{\pos{Modifier}} {\definition{tall}}
\entry{ulle}\headword{ulle}{\pos{Modifier}} {\definition{long; tall}}
\end{entrylist}

\section*{8.2.2.3 Short, not tall}
\begin{entrylist}
\entry{pättäk}\headword{pättäk}{\pos{Modifier}} {\definition{short}}
\entry{tubu}\headword{tubu}{\pos{Modifier}} {\definition{short}}
\end{entrylist}

\section*{8.2.3 Thick}
\begin{entrylist}
\entry{pukong}\headword{pukong}{\pos{Modifier}} {\definition{thick}}
\end{entrylist}

\section*{8.2.3.1 Thin thing}
\begin{entrylist}
\entry{petapeta}\headword{petapeta}{\pos{Modifier}} {\definition{thin (inanimate)}}
\end{entrylist}

\section*{8.2.3.2 Fat person}
\begin{entrylist}
\entry{gullabgullab}\headword{gullabgullab}{\pos{Modifier}} {\definition{fat}}
\end{entrylist}

\section*{8.2.3.3 Thin person}
\begin{entrylist}
\entry{kuttakuttang}\headword{kuttakuttang}{\pos{Modifier}} {\definition{thin, bony}}
\entry{kwaratang}\headword{kwaratang}{\pos{Modifier}} {\definition{thin, skinny, slim (animate)}}
\end{entrylist}

\section*{8.2.4 Wide}
\begin{entrylist}
\entry{dum}\headword{dum}{\pos{Noun}} {\definition{width (of a house)}}
\entry{ubemang}\headword{ubemang}{\pos{Modifier}} {\definition{wide}}
\end{entrylist}

\section*{8.2.4.1 Narrow}
\begin{entrylist}
\entry{ngowangowe}\headword{ngowangowe}{\pos{Modifier}} {\definition{narrow}}
\end{entrylist}

\section*{8.2.6.1 Far}
\begin{entrylist}
\entry{utale}\headword{utale}{\pos{Locational}} {\definition{far}}
\end{entrylist}

\section*{8.2.6.2 Near}
\begin{entrylist}
\entry{dowae}\headword{dowae}{\pos{Locational}} {\definition{vicinity, proximity}}
\end{entrylist}

\section*{8.2.6.4 Low}
\begin{entrylist}
\entry{ekaklle}\headword{ekaklle}{\pos{Locational}} {\definition{low}}
\entry{matu}\headword{matu}{\pos{Locational}} {\definition{lower part}}
\end{entrylist}

\section*{8.2.6.5 Deep, shallow}
\begin{entrylist}
\entry{pasis}\headword{pasis}{\pos{Noun}} {\definition{deep}}
\entry{petapeta}\headword{petapeta}{\pos{Modifier}} {\definition{shallow}}
\entry{teyateyar}\headword{teyateyar}{\pos{Modifier}} {\definition{shallow}}
\entry{ttu}\headword{ttu}{\pos{Noun}} {\definition{deep}}
\entry{täränga}\headword{täränga}{\pos{Modifier}} {\definition{low, scant, shallow}}
\end{entrylist}

\section*{8.2.7.1 Tight}
\begin{entrylist}
\entry{dom}\headword{dom}{\pos{Transitive A verb}} {\definition{to clench (one's fist)}}
\end{entrylist}

\section*{8.2.7.2 Loose}
\begin{entrylist}
\entry{kakakän}\headword{kakakän}{\pos{Modifier}} {\definition{loose, floating, unbound}}
\entry{kakne}\headword{kakne}{\pos{Modifier}} {\definition{floating, loose}}
\end{entrylist}

\section*{8.2.8 Measure}
\begin{entrylist}
\entry{dang}\headword{dang}{\pos{Noun}} {\definition{length (of a house)}}
\entry{dum}\headword{dum}{\pos{Noun}} {\definition{width (of a house)}}
\entry{erär}\headword{erär}{\pos{Transitive S verb}} {\definition{to measure}}
\end{entrylist}

\section*{8.2.9 Weigh}
\begin{entrylist}
\entry{buddo}\headword{buddo}{\pos{Noun}} {\definition{weight}}
\entry{inam}\headword{inam}{\pos{Transitive S verb}} {\definition{to weigh down, press down}}
\end{entrylist}

\section*{8.2.9.1 Heavy}
\begin{entrylist}
\entry{buddog}\headword{buddog}{\pos{Modifier}} {\definition{heavy}}
\end{entrylist}

\section*{8.2.9.2 Light in weight}
\begin{entrylist}
\entry{poapoa}\headword{poapoa}{\pos{Modifier}} {\definition{light (in weight)}}
\end{entrylist}

\section*{8.3.1.2 Line}
\begin{entrylist}
\entry{laen}\headword{laen}{\pos{Noun}} {\definition{line}}
\entry{päpa}\headword{päpa}{\pos{Noun}} {\definition{line}}
\end{entrylist}

\section*{8.3.1.3 Straight}
\begin{entrylist}
\entry{enanae}\headword{enanae}{\pos{Adverb}} {\definition{directly, straight}}
\entry{tonton}\headword{tonton}{\pos{Adverb}} {\definition{directly}}
\entry{ttättle}\headword{ttättle}{\pos{Modifier}} {\definition{straight}}
\end{entrylist}

\section*{8.3.1.3.1 Flat}
\begin{entrylist}
\entry{palltapallta}\headword{palltapallta}{\pos{Modifier}} {\definition{flat}}
\end{entrylist}

\section*{8.3.1.5 Bend}
\begin{entrylist}
\entry{kalokalo}\headword{kalokalo}{\pos{Modifier}} {\definition{pliable, flexible, bendable}}
\entry{mällkae}\headword{mällkae}{\pos{Intransitive S verb}} {\definition{to bend}}
\entry{mällkam}\headword{mällkam}{\pos{Intransitive S verb}} {\definition{to bend}}
\end{entrylist}

\section*{8.3.1.5.1 Roll up}
\begin{entrylist}
\entry{laem}\headword{laem}{\pos{Transitive S verb}} {\definition{to roll, wrap}}
\entry{ngädngäd}\headword{ngädngäd}{\pos{Intransitive S verb}} {\definition{to roll up, curl}}
\entry{zigae}\headword{zigae}{\pos{Transitive S verb}} {\definition{to wrap}}
\end{entrylist}

\section*{8.3.1.5.2 Twist, wring}
\begin{entrylist}
\entry{llatat}\headword{llatat}{\pos{Transitive S verb}} {\definition{to twist, wrap}}
\end{entrylist}

\section*{8.3.1.5.3 Fold}
\begin{entrylist}
\entry{bänamb}\headword{bänamb}{\pos{Transitive S verb}} {\definition{to open (something folded, e.g. book, mouth)}}
\entry{ngädngäd}\headword{ngädngäd}{\pos{Intransitive S verb}} {\definition{to fold}}
\end{entrylist}

\section*{8.3.1.6 Round}
\begin{entrylist}
\entry{bol}\headword{bol}{\pos{Noun}} {\definition{ball}}
\entry{llädae}\headword{llädae}{\pos{Transitive S verb}} {\definition{to roll}}
\entry{ngälngäl}\headword{ngälngäl}{\pos{Modifier}} {\definition{round}}
\end{entrylist}

\section*{8.3.1.6.3 Hollow}
\begin{entrylist}
\entry{kopek}\headword{kopek}{\pos{Noun}} {\definition{pit, hole}}
\entry{kup}\headword{kup}{\pos{Noun}} {\definition{hole, pit}}
\end{entrylist}

\section*{8.3.1.8 Pattern, design}
\begin{entrylist}
\entry{penmällpenmäll}\headword{penmällpenmäll}{\pos{Modifier}} {\definition{spotted}}
\end{entrylist}

\section*{8.3.1.9 Stretch}
\begin{entrylist}
\entry{ttällam}\headword{ttällam}{\pos{Transitive S verb}} {\definition{to extend, stretch out, reach out, put out}}
\end{entrylist}

\section*{8.3.2.1 Smooth}
\begin{entrylist}
\entry{aräram}\headword{aräram}{\pos{Transitive A verb}} {\definition{to sand or smooth a canoe (last step before burning)}}
\entry{ngongo}\headword{ngongo}{\pos{Transitive S verb}} {\definition{to smooth, sand (a surface)}}
\entry{pobllem}\headword{pobllem}{\pos{Modifier}} {\definition{smooth}}
\entry{tärpi}\headword{tärpi}{\pos{Modifier}} {\definition{slippery, smooth}}
\end{entrylist}

\section*{8.3.2.3 Sharp}
\begin{entrylist}
\entry{ezi}\headword{ezi}{\pos{Noun}} {\definition{sharp edge}}
\entry{gädagäde}\headword{gädagäde}{\pos{Transitive S verb}} {\definition{to sharpen}}
\entry{lläng}\headword{lläng}{\pos{Modifier}} {\definition{sharp}}
\entry{met}\headword{met}{\pos{Transitive A verb}} {\definition{to sharpen}}
\entry{popo}\headword{popo}{\pos{Transitive S verb}} {\definition{to sharpen}}
\end{entrylist}

\section*{8.3.2.4 Blunt}
\begin{entrylist}
\entry{ddokddok}\headword{ddokddok}{\pos{Modifier}} {\definition{blunt}}
\entry{llängmeny}\headword{llängmeny}{\pos{Modifier}} {\definition{dull, blunt}}
\entry{mitmit}\headword{mitmit}{\pos{Noun}} {\definition{blunt axe}}
\entry{ttogottogo}\headword{ttogottogo}{\pos{Intransitive A verb}} {\definition{to become blunt}}
\end{entrylist}

\section*{8.3.3 Light}
\begin{entrylist}
\entry{indrang}\headword{indrang}{\pos{Noun}} {\definition{light}}
\entry{to}\headword{to}{\pos{Noun}} {\definition{light}}
\entry{tos}\headword{tos}{\pos{Noun}} {\definition{(Commonwealth) torch, (US) flashlight}}
\end{entrylist}

\section*{8.3.3.1 Shine}
\begin{entrylist}
\entry{benmäll}\headword{benmäll}{\pos{Intransitive S verb}} {\definition{to shine, flash}}
\entry{bädab}\headword{bädab}{\pos{Intransitive S verb}} {\definition{to shine brightly (of the moon)}}
\entry{daramdaram}\headword{daramdaram}{\pos{Adverb}} {\definition{shining brightly}}
\entry{gllae}\headword{gllae}{\pos{Intransitive S verb}} {\definition{to shine}}
\end{entrylist}

\section*{8.3.3.1.2 Bright}
\begin{entrylist}
\entry{indrang}\headword{indrang}{\pos{Modifier}} {\definition{luminous, bright}}
\entry{poapoa}\headword{poapoa}{\pos{Modifier}} {\definition{light, bright}}
\entry{pällämpälläm}\headword{pällämpälläm}{\pos{Color term}} {\definition{white, bright}}
\end{entrylist}

\section*{8.3.3.2 Dark}
\begin{entrylist}
\entry{bɨt}\headword{bɨt}{\pos{Modifier}} {\definition{dark}}
\entry{säremang}\headword{säremang}{\pos{Modifier}} {\definition{dark, dim}}
\end{entrylist}

\section*{8.3.3.2.1 Shadow}
\begin{entrylist}
\entry{gäba}\headword{gäba}{\pos{Noun}} {\definition{shade}}
\entry{gäbagäba}\headword{gäbagäba}{\pos{Modifier}} {\definition{shady}}
\end{entrylist}

\section*{8.3.3.3 Color}
\begin{entrylist}
\entry{bɨtbɨt}\headword{bɨtbɨt}{\pos{Color term}} {\definition{black}}
\entry{bɨtbɨtbɨtbɨt}\headword{bɨtbɨtbɨtbɨt}{\pos{Color term}} {\definition{purple (color of sawis, purple yam)}}
\entry{dɨdɨr}\headword{dɨdɨr}{\pos{Modifier}} {\definition{brown (dried)}}
\entry{iklloikllowang}\headword{iklloikllowang}{\pos{Color term}} {\definition{grey; the color of smoke}}
\entry{kala}\headword{kala}{\pos{Noun}} {\definition{color, pigment, dye}}
\entry{kirekire}\headword{kirekire}{\pos{Color term}} {\definition{green}}
\entry{kukollkukoll}\headword{kukollkukoll}{\pos{Color term}} {\definition{green}}
\entry{kätt ine}\headword{kätt ine}{\pos{Color term}} {\definition{blue (lit. 'shell water')}}
\entry{kätäräl}\headword{kätäräl}{\pos{Noun}} {\definition{color}}
\entry{mam}\headword{mam}{\pos{Color term}} {\definition{red; pink}}
\entry{mam titi popo}\headword{mam titi popo}{\pos{Color term}} {\definition{dark red}}
\entry{mamam}\headword{mamam}{\pos{Color term}} {\definition{red}}
\entry{mamamam}\headword{mamamam}{\pos{Color term}} {\definition{red}}
\entry{sägäsägäd}\headword{sägäsägäd}{\pos{Color term}} {\definition{yellow}}
\entry{yindrang}\headword{yindrang}{\pos{Color term}} {\definition{clear, see-through. Like clear blue water you can see the fish through or the ziplock bag or the plastic water bottle (even though they are also colored red or yellow).}}
\end{entrylist}

\section*{8.3.3.3.1 White}
\begin{entrylist}
\entry{pällämpälläm}\headword{pällämpälläm}{\pos{Color term}} {\definition{white, bright}}
\end{entrylist}

\section*{8.3.3.3.6 Change color}
\begin{entrylist}
\entry{apgllu}\headword{apgllu}{\pos{Noun}} {\definition{small plant with a root that makes a maroon pigment for dyeing grass skirts when mixed with ash (e.g. Acacia ash or coconut ash). When not mixed with ash, it makes a yellow pigment.}}
\end{entrylist}

\section*{8.3.3.3.7 Multicolored}
\begin{entrylist}
\entry{käträkäträl}\headword{käträkäträl}{\pos{Modifier}} {\definition{multicolored}}
\end{entrylist}

\section*{8.3.4 Hot}
\begin{entrylist}
\entry{binzeg}\headword{binzeg}{\pos{Transitive S verb}} {\definition{to heat, warm}}
\entry{gonagone}\headword{gonagone}{\pos{Transitive S verb}} {\definition{to heat}}
\entry{lɨklɨk}\headword{lɨklɨk}{\pos{Intransitive S verb}} {\definition{to melt}}
\entry{mena}\headword{mena}{\pos{Transitive S verb}} {\definition{to scorch}}
\entry{ttänttäm}\headword{ttänttäm}{\pos{Noun}} {\definition{heat}}
\entry{tätäräp}\headword{tätäräp}{\pos{Noun}} {\definition{heat}}
\end{entrylist}

\section*{8.3.4.1 Cold}
\begin{entrylist}
\entry{kallkäll}\headword{kallkäll}{\pos{Noun}} {\definition{cold}}
\entry{kutt tataem}\headword{kutt tataem}{\pos{Intransitive A verb}} {\definition{to shiver}}
\entry{kädkäd}\headword{kädkäd}{\pos{Modifier}} {\definition{cold}}
\end{entrylist}

\section*{8.3.5.2.1 Same}
\begin{entrylist}
\entry{sem}\headword{sem}{\pos{Modifier}} {\definition{same}}
\end{entrylist}

\section*{8.3.5.2.2 Like, similar}
\begin{entrylist}
\entry{=ngänäm}\headword{=ngänäm}{\pos{Nominal enclitic}} {\definition{similative case clitic; like}}
\end{entrylist}

\section*{8.3.5.2.3 Different}
\begin{entrylist}
\entry{sapang}\headword{sapang}{\pos{Modifier}} {\definition{separate, apart, different; own, personal}}
\end{entrylist}

\section*{8.3.5.2.4 Other}
\begin{entrylist}
\entry{ddob}\headword{ddob}{\pos{Determiner}} {\definition{other}}
\end{entrylist}

\section*{8.3.5.2.5 Various}
\begin{entrylist}
\entry{emaemae}\headword{emaemae}{\pos{Modifier}} {\definition{various, many different kinds}}
\entry{sapasapang}\headword{sapasapang}{\pos{Modifier}} {\definition{various, many different}}
\end{entrylist}

\section*{8.3.5.3.1 Usual}
\begin{entrylist}
\entry{mizi}\headword{mizi}{\pos{Adverb}} {\definition{usually}}
\end{entrylist}

\section*{8.3.5.3.3 Unique}
\begin{entrylist}
\entry{ttongo}\headword{ttongo}{\pos{Modifier}} {\definition{unique}}
\end{entrylist}

\section*{8.3.6.2 Hard, firm}
\begin{entrylist}
\entry{dädär}\headword{dädär}{\pos{Modifier}} {\definition{hard}}
\end{entrylist}

\section*{8.3.6.5 Soft, flimsy}
\begin{entrylist}
\entry{kuddäkuddäll}\headword{kuddäkuddäll}{\pos{Modifier}} {\definition{soft}}
\end{entrylist}

\section*{8.3.7 Good}
\begin{entrylist}
\entry{ai}\headword{ai}{\pos{Modifier}} {\definition{good}}
\entry{ulle}\headword{ulle}{\pos{Modifier}} {\definition{important, great}}
\end{entrylist}

\section*{8.3.7.1 Bad}
\begin{entrylist}
\entry{kotkot}\headword{kotkot}{\pos{Modifier}} {\definition{dirty, unclean}}
\entry{zozo}\headword{zozo}{\pos{Intransitive S verb}} {\definition{to rot, go bad}}
\end{entrylist}

\section*{8.3.7.2.1 Worse}
\begin{entrylist}
\entry{llɨtt}\headword{llɨtt}{\pos{Intransitive S verb}} {\definition{to get worse, worsen}}
\end{entrylist}

\section*{8.3.7.5 Important}
\begin{entrylist}
\entry{ulle}\headword{ulle}{\pos{Modifier}} {\definition{important, great}}
\entry{ulle binang}\headword{ulle binang}{\pos{Noun}} {\definition{master, owner, ruler, important person}}
\end{entrylist}

\section*{8.3.7.7 Right, proper}
\begin{entrylist}
\entry{imomdae}\headword{imomdae}{\pos{Modifier}} {\definition{correct, right}}
\end{entrylist}

\section*{8.3.7.8 Decay}
\begin{entrylist}
\entry{zozo}\headword{zozo}{\pos{Intransitive S verb}} {\definition{to rot, go bad}}
\end{entrylist}

\section*{8.3.7.8.2 Blemish}
\begin{entrylist}
\entry{päd}\headword{päd}{\pos{Noun}} {\definition{scar}}
\entry{tokop}\headword{tokop}{\pos{Noun}} {\definition{lump on skin}}
\entry{tomäll}\headword{tomäll}{\pos{Noun}} {\definition{wart; fungal skin infection}}
\end{entrylist}

\section*{8.3.7.9 Value}
\begin{entrylist}
\entry{mu}\headword{mu}{\pos{Noun}} {\definition{payment, price, value}}
\end{entrylist}

\section*{8.3.8 Decorated}
\begin{entrylist}
\entry{dadäräb}\headword{dadäräb}{\pos{Transitive S verb}} {\definition{to decorate}}
\entry{pän}\headword{pän}{\pos{Intransitive S verb}} {\definition{to ripen, e.g. of a papaya. Literally: to decorate a nose.}}
\end{entrylist}

\section*{8.4 Time}
\begin{entrylist}
\entry{ollong}\headword{ollong}{\pos{Noun}} {\definition{time, occasion}}
\entry{taem}\headword{taem}{\pos{Noun}} {\definition{time}}
\entry{täräp}\headword{täräp}{\pos{Noun}} {\definition{time}}
\end{entrylist}

\section*{8.4.1.1 Calendar}
\begin{entrylist}
\entry{ttängattänge}\headword{ttängattänge}{\pos{Noun}} {\definition{date}}
\end{entrylist}

\section*{8.4.1.2 Day}
\begin{entrylist}
\entry{ebdo}\headword{ebdo}{\pos{Noun}} {\definition{day}}
\end{entrylist}

\section*{8.4.1.2.1 Night}
\begin{entrylist}
\entry{iddob}\headword{iddob}{\pos{Noun}} {\definition{night}}
\entry{inu}\headword{inu}{\pos{Noun}} {\definition{night}}
\entry{kokta}\headword{kokta}{\pos{Noun}} {\definition{moonlight}}
\end{entrylist}

\section*{8.4.1.2.2 Yesterday, today, tomorrow}
\begin{entrylist}
\entry{eddom}\headword{eddom}{\pos{Noun}} {\definition{the day before yesterday}}
\entry{känazbag}\headword{känazbag}{\pos{Adverb}} {\definition{tomorrow}}
\entry{sisri ebdo}\headword{sisri ebdo}{\pos{Noun}} {\definition{today}}
\entry{ttongo iddob}\headword{ttongo iddob}{\pos{Noun}} {\definition{the day after tomorrow}}
\entry{tätäm}\headword{tätäm}{\pos{Adverb}} {\definition{yesterday}}
\end{entrylist}

\section*{8.4.1.2.3 Time of the day}
\begin{entrylist}
\entry{ag}\headword{ag}{\pos{Noun}} {\definition{morning (approx. 5 AM–11 AM)}}
\entry{awi}\headword{awi}{\pos{Noun}} {\definition{evening (approx. 5 PM till dark)}}
\entry{ballme}\headword{ballme}{\pos{Noun}} {\definition{dawn, daybreak}}
\entry{bädab}\headword{bädab}{\pos{Intransitive S verb}} {\definition{to dawn, break}}
\entry{bädab}\headword{bädab}{\pos{Noun}} {\definition{dawn}}
\entry{ebdo}\headword{ebdo}{\pos{Noun}} {\definition{noon (approx. 11 AM –1 PM)}}
\entry{gudae}\headword{gudae}{\pos{Modifier}} {\definition{early morning}}
\entry{toto}\headword{toto}{\pos{Noun}} {\definition{afternoon; early evening (approx. 1 PM–5 PM)}}
\entry{yäbäd tuktuk}\headword{yäbäd tuktuk}{\pos{Noun}} {\definition{solar noon (when the sun reaches its zenith)}}
\end{entrylist}

\section*{8.4.1.3 Week}
\begin{entrylist}
\entry{sande}\headword{sande}{\pos{Noun}} {\definition{week}}
\entry{wik}\headword{wik}{\pos{Noun}} {\definition{week}}
\end{entrylist}

\section*{8.4.1.3.1 Days of the week}
\begin{entrylist}
\entry{mandde}\headword{mandde}{\pos{Noun}} {\definition{Monday}}
\entry{praedde}\headword{praedde}{\pos{Noun}} {\definition{Friday}}
\entry{sande}\headword{sande}{\pos{Noun}} {\definition{Sunday}}
\entry{satade}\headword{satade}{\pos{Noun}} {\definition{Saturday}}
\entry{tesde}\headword{tesde}{\pos{Noun}} {\definition{Thursday}}
\entry{tusde}\headword{tusde}{\pos{Noun}} {\definition{Tuesday}}
\entry{winisde}\headword{winisde}{\pos{Noun}} {\definition{Wednesday}}
\end{entrylist}

\section*{8.4.1.4 Month}
\begin{entrylist}
\entry{kok}\headword{kok}{\pos{Noun}} {\definition{month}}
\end{entrylist}

\section*{8.4.1.4.1 Months of the year}
\begin{entrylist}
\entry{kämag}\headword{kämag}{\pos{Noun}} {\definition{season characterized by windy storms from the west (first season; corresponds to January)}}
\end{entrylist}

\section*{8.4.1.5 Season}
\begin{entrylist}
\entry{babaem}\headword{babaem}{\pos{Noun}} {\definition{season characterized by wind and going hunting (fourth season; corresponds to late February)}}
\entry{bawa}\headword{bawa}{\pos{Noun}} {\definition{season characterized by hunting and fishing in heavy rain (ninth season; corresponds to June)}}
\entry{bawa minyminy}\headword{bawa minyminy}{\pos{Noun}} {\definition{season characterized by hunting and fishing in light rain (tenth season; corresponds to July)}}
\entry{beatururang}\headword{beatururang}{\pos{Noun}} {\definition{season characterized by thunderstorms and flooding (second season; corresponds to early February)}}
\entry{dadel}\headword{dadel}{\pos{Noun}} {\definition{season of harvesting young gardens (seventh season; corresponds to early May)}}
\entry{däm ibenen}\headword{däm ibenen}{\pos{Noun}} {\definition{season when crops are planted (fifteenth season; corresponds to late November)}}
\entry{ketrop}\headword{ketrop}{\pos{Noun}} {\definition{season when rain clouds form but are blown away by the wind}}
\entry{kunun}\headword{kunun}{\pos{Noun}} {\definition{season when crops are ready to be harvested (sixth season; corresponds to April)}}
\entry{kämag}\headword{kämag}{\pos{Noun}} {\definition{season characterized by windy storms from the west (first season; corresponds to January)}}
\entry{sis}\headword{sis}{\pos{Noun}} {\definition{season when new gardens are cleared and fenced (fourteenth season; corresponds to early November)}}
\entry{sisor pazi}\headword{sisor pazi}{\pos{Noun}} {\definition{season of New Years (sixteenth season; corresponds to December)}}
\entry{tarme ballmenyang}\headword{tarme ballmenyang}{\pos{Noun}} {\definition{season when crops are bearing fruit (fifth season; corresponds to March)}}
\entry{täträp}\headword{täträp}{\pos{Noun}} {\definition{season of the end of harvesting and the start of hunting (eighth season; corresponds to late May)}}
\entry{umllang bällanen ma skul}\headword{umllang bällanen ma skul}{\pos{Noun}} {\definition{vocational school}}
\entry{ur yogoll}\headword{ur yogoll}{\pos{Noun}} {\definition{flood time, rainy season}}
\entry{yäbäd}\headword{yäbäd}{\pos{Noun}} {\definition{season when the dry season starts and people go camping in the bush (eleventh season; corresponds to August)}}
\entry{yäbäd bäng}\headword{yäbäd bäng}{\pos{Noun}} {\definition{dry, hot season characterized by burning grass (twelfth season; corresponds to September)}}
\entry{yäbäd ttänttämang}\headword{yäbäd ttänttämang}{\pos{Noun}} {\definition{hot season when new gardens are burnt (thirteenth month; corresponds to October)}}
\end{entrylist}

\section*{8.4.1.6 Year}
\begin{entrylist}
\entry{pazi}\headword{pazi}{\pos{Noun}} {\definition{year}}
\end{entrylist}

\section*{8.4.2.3 Forever}
\begin{entrylist}
\entry{enanae}\headword{enanae}{\pos{Adverb}} {\definition{forever, for good}}
\entry{enanaeenanae}\headword{enanaeenanae}{\pos{Modifier}} {\definition{eternal}}
\end{entrylist}

\section*{8.4.2.4 Temporary}
\begin{entrylist}
\entry{yuwet}\headword{yuwet}{\pos{Noun}} {\definition{short period of time}}
\entry{yuwetyuwet}\headword{yuwetyuwet}{\pos{Adverb}} {\definition{temporarily, briefly}}
\end{entrylist}

\section*{8.4.4 Telling time}
\begin{entrylist}
\entry{o klak}\headword{o klak}{\pos{Adverb}} {\definition{o'clock}}
\end{entrylist}

\section*{8.4.5.1.2 First}
\begin{entrylist}
\entry{ngattong}\headword{ngattong}{\pos{Ordinal numeral}} {\definition{first}}
\entry{pes}\headword{pes}{\pos{Ordinal numeral}} {\definition{first}}
\end{entrylist}

\section*{8.4.5.2 Before}
\begin{entrylist}
\entry{ngattong}\headword{ngattong}{\pos{Adverb}} {\definition{first, at first, initially, previously, before}}
\end{entrylist}

\section*{8.4.5.2.1 After}
\begin{entrylist}
\entry{abo}\headword{abo}{\pos{Adverb}} {\definition{then, afterwards}}
\entry{ddägattalle}\headword{ddägattalle}{\pos{Adverb}} {\definition{later, after}}
\entry{dibaballe}\headword{dibaballe}{\pos{Adverbial demonstrative}} {\definition{ablative form of diba; then, thereupon}}
\entry{imne}\headword{imne}{\pos{Adverb}} {\definition{afterwards, after, later}}
\entry{imneimne}\headword{imneimne}{\pos{Adverb}} {\definition{behind, late; later, after}}
\end{entrylist}

\section*{8.4.5.2.2 At the same time}
\begin{entrylist}
\entry{damärärmae}\headword{damärärmae}{\pos{Adverb}} {\definition{simultaneously}}
\end{entrylist}

\section*{8.4.5.2.3 During}
\begin{entrylist}
\entry{makäp}\headword{makäp}{\pos{Noun}} {\definition{duration}}
\end{entrylist}

\section*{8.4.5.3.1 Early}
\begin{entrylist}
\entry{awi}\headword{awi}{\pos{Modifier}} {\definition{early}}
\entry{gudae}\headword{gudae}{\pos{Modifier}} {\definition{early morning}}
\entry{ngattong}\headword{ngattong}{\pos{Noun}} {\definition{beginning; past}}
\end{entrylist}

\section*{8.4.5.3.3 Late}
\begin{entrylist}
\entry{abade}\headword{abade}{\pos{Modifier}} {\definition{future, later, upcoming, impending}}
\entry{imneimne}\headword{imneimne}{\pos{Adverb}} {\definition{behind, late; later, after}}
\entry{källkae}\headword{källkae}{\pos{Adverb}} {\definition{later, in the future}}
\entry{nyamäll}\headword{nyamäll}{\pos{Intransitive S verb}} {\definition{to be late}}
\end{entrylist}

\section*{8.4.6.1 Start something}
\begin{entrylist}
\entry{kam}\headword{kam}{\pos{Transitive S verb}} {\definition{to start}}
\end{entrylist}

\section*{8.4.6.1.1 Beginning}
\begin{entrylist}
\entry{kam}\headword{kam}{\pos{Transitive S verb}} {\definition{to start, begin}}
\entry{mit}\headword{mit}{\pos{Noun}} {\definition{origin, source}}
\entry{ngattong}\headword{ngattong}{\pos{Noun}} {\definition{beginning; past}}
\end{entrylist}

\section*{8.4.6.1.2 Stop something}
\begin{entrylist}
\entry{käbab}\headword{käbab}{\pos{Transitive S verb}} {\definition{to stop}}
\entry{tärangg}\headword{tärangg}{\pos{Transitive S verb}} {\definition{to stop, hold back}}
\end{entrylist}

\section*{8.4.6.1.3 End}
\begin{entrylist}
\entry{dongkal}\headword{dongkal}{\pos{Intransitive S verb}} {\definition{to stop, end}}
\entry{llätt}\headword{llätt}{\pos{Noun}} {\definition{end}}
\entry{timän}\headword{timän}{\pos{Noun}} {\definition{end}}
\entry{tubu}\headword{tubu}{\pos{Noun}} {\definition{end; stump}}
\entry{tära}\headword{tära}{\pos{Transitive A verb}} {\definition{to finish}}
\end{entrylist}

\section*{8.4.6.2 Past}
\begin{entrylist}
\entry{gudae}\headword{gudae}{\pos{Noun}} {\definition{(the) past, before}}
\entry{gudne}\headword{gudne}{\pos{Adverb}} {\definition{long ago}}
\entry{ngattong}\headword{ngattong}{\pos{Noun}} {\definition{beginning; past}}
\end{entrylist}

\section*{8.4.6.3.1 Now}
\begin{entrylist}
\entry{sisri}\headword{sisri}{\pos{Adverb}} {\definition{now}}
\end{entrylist}

\section*{8.4.6.4 Future}
\begin{entrylist}
\entry{abade}\headword{abade}{\pos{Modifier}} {\definition{future, later, upcoming, impending}}
\entry{källkae}\headword{källkae}{\pos{Noun}} {\definition{future}}
\end{entrylist}

\section*{8.4.6.4.4 Immediately}
\begin{entrylist}
\entry{ddänddängeny}\headword{ddänddängeny}{\pos{Adverb}} {\definition{immediately}}
\end{entrylist}

\section*{8.4.6.5 Age}
\begin{entrylist}
\entry{ause}\headword{ause}{\pos{Modifier}} {\definition{old (of a woman)}}
\entry{märäl}\headword{märäl}{\pos{Noun}} {\definition{age-mate, someone of the same age}}
\end{entrylist}

\section*{8.4.6.5.1 Young}
\begin{entrylist}
\entry{män duwar}\headword{män duwar}{\pos{Noun}} {\definition{young girl}}
\entry{pollo}\headword{pollo}{\pos{Modifier}} {\definition{young}}
\entry{sisor}\headword{sisor}{\pos{Modifier}} {\definition{young}}
\end{entrylist}

\section*{8.4.6.5.2 Old, not young}
\begin{entrylist}
\entry{gudne}\headword{gudne}{\pos{Modifier}} {\definition{old}}
\entry{masar}\headword{masar}{\pos{Modifier}} {\definition{ancestral, old (of one's grandparents' time)}}
\entry{mosenmosen}\headword{mosenmosen}{\pos{Noun}} {\definition{elders, seniors}}
\entry{pätär}\headword{pätär}{\pos{Noun}} {\definition{white hair}}
\end{entrylist}

\section*{8.4.6.5.3 New}
\begin{entrylist}
\entry{sisor}\headword{sisor}{\pos{Modifier}} {\definition{new}}
\end{entrylist}

\section*{8.4.6.5.4 Old, not new}
\begin{entrylist}
\entry{gudne}\headword{gudne}{\pos{Modifier}} {\definition{old}}
\entry{masar}\headword{masar}{\pos{Modifier}} {\definition{ancestral, old (of one's grandparents' time)}}
\end{entrylist}

\section*{8.4.6.6.1 Again}
\begin{entrylist}
\entry{kame}\headword{kame}{\pos{Adverb}} {\definition{again}}
\end{entrylist}

\section*{8.4.6.6.4 All the time}
\begin{entrylist}
\entry{wasangwasang}\headword{wasangwasang}{\pos{Adverb}} {\definition{always}}
\end{entrylist}

\section*{8.4.6.6.5 Every time}
\begin{entrylist}
\entry{mullaemullae}\headword{mullaemullae}{\pos{Quantifier}} {\definition{every}}
\end{entrylist}

\section*{8.4.8.1 Quick}
\begin{entrylist}
\entry{mamall}\headword{mamall}{\pos{Adverb}} {\definition{quickly}}
\entry{mängal}\headword{mängal}{\pos{Modifier}} {\definition{quick}}
\entry{ullowae}\headword{ullowae}{\pos{Adverb}} {\definition{fast, quickly}}
\end{entrylist}

\section*{8.4.8.2 Slow}
\begin{entrylist}
\entry{kälepalle}\headword{kälepalle}{\pos{Adverb}} {\definition{slowly}}
\end{entrylist}

\section*{8.5 Location}
\begin{entrylist}
\entry{ma}\headword{ma}{\pos{Noun}} {\definition{place, location}}
\entry{ngata}\headword{ngata}{\pos{Noun}} {\definition{spot}}
\entry{ngättäma}\headword{ngättäma}{\pos{Noun}} {\definition{place, spot}}
\end{entrylist}

\section*{8.5.1.1.1 Behind}
\begin{entrylist}
\entry{golloll}\headword{golloll}{\pos{Locational}} {\definition{back, behind}}
\entry{imne}\headword{imne}{\pos{Locational}} {\definition{rear, behind, back}}
\end{entrylist}

\section*{8.5.1.2.1 Around}
\begin{entrylist}
\entry{dumdum}\headword{dumdum}{\pos{Transitive S verb}} {\definition{to surround}}
\end{entrylist}

\section*{8.5.1.3.2 Under, below}
\begin{entrylist}
\entry{igiigi}\headword{igiigi}{\pos{Locational}} {\definition{under, beneath}}
\entry{täträk}\headword{täträk}{\pos{Transitive S verb}} {\definition{to go underneath}}
\end{entrylist}

\section*{8.5.1.4 Inside}
\begin{entrylist}
\entry{guwo}\headword{guwo}{\pos{Locational}} {\definition{inside}}
\entry{ik}\headword{ik}{\pos{Locational}} {\definition{inside}}
\entry{koll}\headword{koll}{\pos{Locational}} {\definition{inner part}}
\entry{makäp}\headword{makäp}{\pos{Locational}} {\definition{inside, in, within, among}}
\end{entrylist}

\section*{8.5.1.4.1 Out, outside}
\begin{entrylist}
\entry{ddäg}\headword{ddäg}{\pos{Locational}} {\definition{outside}}
\entry{upe}\headword{upe}{\pos{Locational}} {\definition{outside, out}}
\end{entrylist}

\section*{8.5.1.5 Touching, contact}
\begin{entrylist}
\entry{modaeb}\headword{modaeb}{\pos{Transitive S verb}} {\definition{to feel}}
\end{entrylist}

\section*{8.5.1.6 Across}
\begin{entrylist}
\entry{opap}\headword{opap}{\pos{Transitive S verb}} {\definition{to cross over, pass over, move across}}
\entry{tärpam}\headword{tärpam}{\pos{Transitive S verb}} {\definition{to put across}}
\entry{ullull}\headword{ullull}{\pos{Transitive S verb}} {\definition{to cross over}}
\end{entrylist}

\section*{8.5.2 Direction}
\begin{entrylist}
\entry{kämagma}\headword{kämagma}{\pos{Noun}} {\definition{west}}
\entry{pallall}\headword{pallall}{\pos{Noun}} {\definition{direction; area}}
\end{entrylist}

\section*{8.5.2.2 Backward}
\begin{entrylist}
\entry{kumattkumatt}\headword{kumattkumatt}{\pos{Adverb}} {\definition{back}}
\entry{ngäsengäse}\headword{ngäsengäse}{\pos{Adverb}} {\definition{back, the other way}}
\end{entrylist}

\section*{8.5.2.3 Right, left}
\begin{entrylist}
\entry{sawe}\headword{sawe}{\pos{Modifier}} {\definition{left}}
\entry{ttätt}\headword{ttätt}{\pos{Modifier}} {\definition{right}}
\end{entrylist}

\section*{8.5.2.4 Up}
\begin{entrylist}
\entry{tukituki}\headword{tukituki}{\pos{Adverb}} {\definition{uphill}}
\end{entrylist}

\section*{8.5.2.6 Away from}
\begin{entrylist}
\entry{duli}\headword{duli}{\pos{Adverb}} {\definition{away (from a place towards another direction)}}
\end{entrylist}

\section*{8.5.2.8 North, south, east, west}
\begin{entrylist}
\entry{naigae}\headword{naigae}{\pos{Modifier}} {\definition{south}}
\end{entrylist}

\section*{8.5.3.1 Absent}
\begin{entrylist}
\entry{kameny}\headword{kameny}{\pos{Noun}} {\definition{absence}}
\end{entrylist}

\section*{8.5.4.2 Occupy an area}
\begin{entrylist}
\entry{ngättangätta}\headword{ngättangätta}{\pos{Transitive S verb}} {\definition{to occupy, take over}}
\end{entrylist}

\section*{8.6.1 Front}
\begin{entrylist}
\entry{ngattong}\headword{ngattong}{\pos{Locational}} {\definition{front}}
\entry{ngattongattong}\headword{ngattongattong}{\pos{Adverb}} {\definition{ahead, in front}}
\end{entrylist}

\section*{8.6.1.1 Back}
\begin{entrylist}
\entry{golloll}\headword{golloll}{\pos{Locational}} {\definition{back, behind}}
\entry{imne}\headword{imne}{\pos{Locational}} {\definition{rear, behind, back}}
\entry{kumattkumatt}\headword{kumattkumatt}{\pos{Adverb}} {\definition{back}}
\end{entrylist}

\section*{8.6.2 Top}
\begin{entrylist}
\entry{toko}\headword{toko}{\pos{Locational}} {\definition{top}}
\end{entrylist}

\section*{8.6.2.1 Bottom}
\begin{entrylist}
\entry{kum}\headword{kum}{\pos{Locational}} {\definition{butt, base, bottom, lower or back part of something}}
\end{entrylist}

\section*{8.6.3 Side}
\begin{entrylist}
\entry{bowe}\headword{bowe}{\pos{Noun}} {\definition{side}}
\entry{dang}\headword{dang}{\pos{Noun}} {\definition{length (of a house)}}
\entry{dum}\headword{dum}{\pos{Noun}} {\definition{width (of a house)}}
\entry{eroe}\headword{eroe}{\pos{Locational}} {\definition{side}}
\entry{ngalbongalboe}\headword{ngalbongalboe}{\pos{Adverb}} {\definition{on the side}}
\entry{pallall}\headword{pallall}{\pos{Locational}} {\definition{side}}
\end{entrylist}

\section*{8.6.4.1 Outer part}
\begin{entrylist}
\entry{gollob}\headword{gollob}{\pos{Noun}} {\definition{outer layer, hull, shell (e.g. of a turtle, egg)}}
\entry{upe}\headword{upe}{\pos{Locational}} {\definition{outside, out}}
\end{entrylist}

\section*{8.6.5 Middle}
\begin{entrylist}
\entry{amne}\headword{amne}{\pos{Locational}} {\definition{center, middle}}
\entry{ku}\headword{ku}{\pos{Locational}} {\definition{center, core, middle}}
\end{entrylist}

\section*{8.6.6 Edge}
\begin{entrylist}
\entry{päk}\headword{päk}{\pos{Locational}} {\definition{edge}}
\end{entrylist}

\section*{8.6.7 End, point}
\begin{entrylist}
\entry{pot}\headword{pot}{\pos{Noun}} {\definition{tip, end, point; base (of yam)}}
\entry{pud}\headword{pud}{\pos{Noun}} {\definition{end (of a long object)}}
\end{entrylist}

\section*{9.1.1.1 Exist}
\begin{entrylist}
\entry{daden}\headword{daden}{\pos{Copulative verb}} {\definition{to exist, have, there is (present singular form)}}
\end{entrylist}

\section*{9.1.1.2 Become, change state}
\begin{entrylist}
\entry{pänae}\headword{pänae}{\pos{Transitive S verb}} {\definition{to turn, become, transform}}
\end{entrylist}

\section*{9.1.1.3 Have, of}
\begin{entrylist}
\entry{daden}\headword{daden}{\pos{Copulative verb}} {\definition{to exist, have, there is (present singular form)}}
\end{entrylist}

\section*{9.1.2 Do}
\begin{entrylist}
\entry{ngasnges}\headword{ngasnges}{\pos{Transitive S verb}} {\definition{to do}}
\end{entrylist}

\section*{9.1.2.1 Happen}
\begin{entrylist}
\entry{ngasnges}\headword{ngasnges}{\pos{Transitive S verb}} {\definition{to happen}}
\end{entrylist}

\section*{9.1.2.2 React, respond}
\begin{entrylist}
\entry{mu}\headword{mu}{\pos{Noun}} {\definition{response, reply, answer; repayment, revenge}}
\end{entrylist}

\section*{9.1.2.5 Make}
\begin{entrylist}
\entry{mamon}\headword{mamon}{\pos{Transitive S verb}} {\definition{to fashion, shape, make}}
\entry{ngasnges}\headword{ngasnges}{\pos{Transitive S verb}} {\definition{to make}}
\end{entrylist}

\section*{9.1.3 Thing}
\begin{entrylist}
\entry{ngalen}\headword{ngalen}{\pos{Noun}} {\definition{thing}}
\entry{ttoen}\headword{ttoen}{\pos{Noun}} {\definition{thing}}
\entry{za}\headword{za}{\pos{Noun}} {\definition{thing}}
\end{entrylist}

\section*{9.1.4 General adjectives}
\begin{entrylist}
\entry{ddägnan ma}\headword{ddägnan ma}{\pos{Modifier}} {\definition{edible}}
\entry{gagäll}\headword{gagäll}{\pos{Modifier}} {\definition{bad, rotten}}
\entry{kälekäle}\headword{kälekäle}{\pos{Modifier}} {\definition{nonsingular form of kälae}}
\entry{kämgall}\headword{kämgall}{\pos{Noun}} {\definition{underneath}}
\entry{llama}\headword{llama}{\pos{Modifier}} {\definition{other people's, of others}}
\entry{mollong}\headword{mollong}{\pos{Modifier}} {\definition{smelling}}
\entry{mäse}\headword{mäse}{\pos{TAM particle}} {\definition{imminent particle (indicates that something about to take place)}}
\entry{ttam}\headword{ttam}{\pos{Modifier}} {\definition{alive}}
\entry{tubutubu}\headword{tubutubu}{\pos{Modifier}} {\definition{a bit short}}
\entry{tärpitärpi}\headword{tärpitärpi}{\pos{Modifier}} {\definition{nonsingular form of tärpi}}
\end{entrylist}

\section*{9.2.3 Pronouns}
\begin{entrylist}
\entry{beyawaenen}\headword{beyawaenen}{\pos{Copulative verb}} {\definition{copular form of beyawa (present form)}}
\entry{bibi}\headword{bibi}{\pos{Personal pronoun}} {\definition{you all, you (second person nonsingular pronoun, nominative form)}}
\entry{binaene}\headword{binaene}{\pos{Personal pronoun}} {\definition{ablative-possessive form of bibi}}
\entry{ngämi}\headword{ngämi}{\pos{Personal pronoun}} {\definition{we (first person nonsingular exclusive pronoun, nominative form)}}
\entry{ngämo}\headword{ngämo}{\pos{Personal pronoun}} {\definition{possessive form of ngäna}}
\entry{ngäna}\headword{ngäna}{\pos{Personal pronoun}} {\definition{I (first person singular pronoun, nominative form)}}
\entry{ngänawaeben}\headword{ngänawaeben}{\pos{Copulative verb}} {\definition{restrictive copular form of ngäna (present form)}}
\entry{ngänawaenen}\headword{ngänawaenen}{\pos{Copulative verb}} {\definition{copular form of ngäna (present form)}}
\end{entrylist}

\section*{9.2.3.4 Question words}
\begin{entrylist}
\entry{amom}\headword{amom}{\pos{Interrogative pronoun}} {\definition{accusative form of aya}}
\entry{angde}\headword{angde}{\pos{Adverb}} {\definition{when, while, as}}
\entry{aya}\headword{aya}{\pos{Interrogative pronoun}} {\definition{who (singular interrogative pronoun, nominative form)}}
\entry{dade}\headword{dade}{\pos{Particle}} {\definition{ever}}
\entry{ematta}\headword{ematta}{\pos{Adverb}} {\definition{ablative form of e; why (interrogative)}}
\entry{enda}\headword{enda}{\pos{Interrogative pronoun}} {\definition{what}}
\entry{endaeyag}\headword{endaeyag}{\pos{Copulative verb}} {\definition{past plural form of endan}}
\entry{endaya}\headword{endaya}{\pos{Copulative verb}} {\definition{past singular form of endan}}
\entry{erame}\headword{erame}{\pos{Interrogative pronoun}} {\definition{where, in which (locative form of era)}}
\entry{erem}\headword{erem}{\pos{Interrogative pronoun}} {\definition{accusative form of era}}
\end{entrylist}

\section*{9.3 Very}
\begin{entrylist}
\entry{abal}\headword{abal}{\pos{Adverb}} {\definition{very}}
\entry{ai}\headword{ai}{\pos{Adverb}} {\definition{very}}
\entry{ddobae}\headword{ddobae}{\pos{Adverb}} {\definition{very}}
\entry{ddone ada}\headword{ddone ada}{\pos{Adverb}} {\definition{very, a lot (antiphrasis)}}
\end{entrylist}

\section*{9.3.2 Completely}
\begin{entrylist}
\entry{binbäddbädd}\headword{binbäddbädd}{\pos{Adverb}} {\definition{fully, completely}}
\entry{dägadäga}\headword{dägadäga}{\pos{Adverb}} {\definition{completely}}
\entry{yuwoyuwog}\headword{yuwoyuwog}{\pos{Adverb}} {\definition{not completely, not properly}}
\end{entrylist}

\section*{9.4.2.2 Can't}
\begin{entrylist}
\entry{lluwam}\headword{lluwam}{\pos{Transitive S verb}} {\definition{to incapacitate, make unable}}
\end{entrylist}

\section*{9.4.3 Moods}
\begin{entrylist}
\entry{ikop kutt nängazmeny}\headword{ikop kutt nängazmeny}{\pos{Phrase}} {\definition{stare at someone intently when angry with them.}}
\entry{mikuttmeny}\headword{mikuttmeny}{\pos{Modifier}} {\definition{calm}}
\entry{mälläng wällong aga}\headword{mälläng wällong aga}{\pos{Noun}} {\definition{"nose swelled up", very serious, quiet, ignoring people}}
\end{entrylist}

\section*{9.4.4.4 Possible}
\begin{entrylist}
\entry{mullae}\headword{mullae}{\pos{Modifier}} {\definition{able, can, be allowed}}
\end{entrylist}

\section*{9.4.4.6 Unsure}
\begin{entrylist}
\entry{llama}\headword{llama}{\pos{Modifier}} {\definition{hesitant, reluctant}}
\end{entrylist}

\section*{9.4.4.6.2 Maybe}
\begin{entrylist}
\entry{dade}\headword{dade}{\pos{Adverb}} {\definition{maybe}}
\entry{ngasekäma}\headword{ngasekäma}{\pos{Adverb}} {\definition{maybe}}
\end{entrylist}

\section*{9.4.6 Yes}
\begin{entrylist}
\entry{ao}\headword{ao}{\pos{Interjection}} {\definition{yes}}
\entry{daudau}\headword{daudau}{\pos{Intransitive A verb}} {\definition{to nod}}
\entry{se}\headword{se}{\pos{Noun}} {\definition{yes}}
\end{entrylist}

\section*{9.4.6.1 No, not}
\begin{entrylist}
\entry{ddone}\headword{ddone}{\pos{Interjection}} {\definition{no}}
\entry{ddone}\headword{ddone}{\pos{Negation particle}} {\definition{not}}
\entry{ka}\headword{ka}{\pos{Interjection}} {\definition{no}}
\entry{malla}\headword{malla}{\pos{Negation particle}} {\definition{not}}
\entry{wanwen}\headword{wanwen}{\pos{Transitive S verb}} {\definition{to shake, swing}}
\end{entrylist}

\section*{9.5.1.2 Instrument}
\begin{entrylist}
\entry{däba}\headword{däba}{\pos{Noun}} {\definition{type of tree that grows in the grassland with leaves used to wrap sago and durable wood used for kundu drums, house posts, and formerly, bridges}}
\end{entrylist}

\section*{9.5.1.4 Way, manner}
\begin{entrylist}
\entry{ngalen}\headword{ngalen}{\pos{Noun}} {\definition{way, habit, manner, custom}}
\entry{nyongo}\headword{nyongo}{\pos{Noun}} {\definition{way, method}}
\entry{ttoen}\headword{ttoen}{\pos{Noun}} {\definition{way, method}}
\end{entrylist}

\section*{9.5.2.1 Together}
\begin{entrylist}
\entry{llame}\headword{llame}{\pos{Adverb}} {\definition{together}}
\end{entrylist}

\section*{9.6.2.5.1 Reason}
\begin{entrylist}
\entry{mit}\headword{mit}{\pos{Noun}} {\definition{reason, sake}}
\end{entrylist}

\section*{9.7 Name}
\begin{entrylist}
\entry{binang}\headword{binang}{\pos{Noun}} {\definition{namesake}}
\entry{ttam}\headword{ttam}{\pos{Transitive S verb}} {\definition{to call, name}}
\end{entrylist}

\section*{9.7.1 Name of a person}
\begin{entrylist}
\entry{Abagigima}\headword{Abagigima}{\pos{Proper noun}} {\definition{personal name}}
\entry{Abeam}\headword{Abeam}{\pos{Proper noun}} {\definition{male personal name}}
\entry{Abigail}\headword{Abigail}{\pos{Proper noun}} {\definition{female personal name}}
\entry{Adasha}\headword{Adasha}{\pos{Proper noun}} {\definition{male personal name}}
\entry{Adu}\headword{Adu}{\pos{Proper noun}} {\definition{female personal name}}
\entry{Aituru}\headword{Aituru}{\pos{Proper noun}} {\definition{female personal name}}
\entry{Al}\headword{Al}{\pos{Proper noun}} {\definition{female personal name}}
\entry{Alex}\headword{Alex}{\pos{Proper noun}} {\definition{male personal name}}
\entry{Alofa}\headword{Alofa}{\pos{Proper noun}} {\definition{female personal name}}
\entry{Alphones}\headword{Alphones}{\pos{Proper noun}} {\definition{male personal name}}
\entry{Amadu}\headword{Amadu}{\pos{Proper noun}} {\definition{male personal name}}
\entry{Amanda}\headword{Amanda}{\pos{Proper noun}} {\definition{female personal name}}
\entry{Ana}\headword{Ana}{\pos{Proper noun}} {\definition{female personal name}}
\entry{Andrew}\headword{Andrew}{\pos{Proper noun}} {\definition{male personal name}}
\entry{Anna}\headword{Anna}{\pos{Proper noun}} {\definition{female personal name}}
\entry{Ansel}\headword{Ansel}{\pos{Proper noun}} {\definition{male personal name}}
\entry{Anton}\headword{Anton}{\pos{Proper noun}} {\definition{male personal name}}
\entry{Arua}\headword{Arua}{\pos{Proper noun}} {\definition{male personal name}}
\entry{Bablela}\headword{Bablela}{\pos{Proper noun}} {\definition{male personal name}}
\entry{Badu}\headword{Badu}{\pos{Proper noun}} {\definition{male personal name}}
\entry{Baewa}\headword{Baewa}{\pos{Proper noun}} {\definition{male personal name}}
\entry{Barekam}\headword{Barekam}{\pos{Proper noun}} {\definition{male personal name}}
\entry{Bati}\headword{Bati}{\pos{Proper noun}} {\definition{female personal name}}
\entry{Ben}\headword{Ben}{\pos{Proper noun}} {\definition{male personal name}}
\entry{Benson}\headword{Benson}{\pos{Proper noun}} {\definition{male personal name}}
\entry{Benta}\headword{Benta}{\pos{Proper noun}} {\definition{female personal name}}
\entry{Bessie}\headword{Bessie}{\pos{Proper noun}} {\definition{female personal name}}
\entry{Bewag}\headword{Bewag}{\pos{Proper noun}} {\definition{male personal name}}
\entry{Bibiae}\headword{Bibiae}{\pos{Proper noun}} {\definition{female personal name}}
\entry{Bidog}\headword{Bidog}{\pos{Proper noun}} {\definition{male personal name}}
\entry{Bigjay}\headword{Bigjay}{\pos{Proper noun}} {\definition{male personal name}}
\entry{Biku}\headword{Biku}{\pos{Proper noun}} {\definition{male personal name}}
\entry{Bobzag}\headword{Bobzag}{\pos{Proper noun}} {\definition{personal name}}
\entry{Bodog}\headword{Bodog}{\pos{Proper noun}} {\definition{male personal name}}
\entry{Bomso}\headword{Bomso}{\pos{Proper noun}} {\definition{male personal name}}
\entry{Bonibi}\headword{Bonibi}{\pos{Proper noun}} {\definition{female personal name}}
\entry{Breton}\headword{Breton}{\pos{Proper noun}} {\definition{male personal name}}
\entry{Bundae}\headword{Bundae}{\pos{Proper noun}} {\definition{male personal name}}
\entry{Caso}\headword{Caso}{\pos{Proper noun}} {\definition{male personal name}}
\entry{Cathy}\headword{Cathy}{\pos{Proper noun}} {\definition{female personal name}}
\entry{Christina}\headword{Christina}{\pos{Proper noun}} {\definition{female personal name}}
\entry{Dabi}\headword{Dabi}{\pos{Proper noun}} {\definition{female personal name}}
\entry{Daniel}\headword{Daniel}{\pos{Proper noun}} {\definition{male personal name}}
\entry{Danipa}\headword{Danipa}{\pos{Proper noun}} {\definition{male personal name}}
\entry{Dara}\headword{Dara}{\pos{Proper noun}} {\definition{male personal name}}
\entry{Dareda}\headword{Dareda}{\pos{Proper noun}} {\definition{male personal name}}
\entry{Darren}\headword{Darren}{\pos{Proper noun}} {\definition{male personal name}}
\entry{Ddelema}\headword{Ddelema}{\pos{Proper noun}} {\definition{male personal name}}
\entry{Deboa}\headword{Deboa}{\pos{Proper noun}} {\definition{male personal name}}
\entry{Deibid}\headword{Deibid}{\pos{Proper noun}} {\definition{male personal name}}
\entry{Dibor}\headword{Dibor}{\pos{Proper noun}} {\definition{female personal name}}
\entry{Dieb}\headword{Dieb}{\pos{Proper noun}} {\definition{male personal name}}
\entry{Diendra}\headword{Diendra}{\pos{Proper noun}} {\definition{female personal name}}
\entry{Dikai}\headword{Dikai}{\pos{Proper noun}} {\definition{male personal name}}
\entry{Dimson}\headword{Dimson}{\pos{Proper noun}} {\definition{male personal name}}
\entry{Dipa}\headword{Dipa}{\pos{Proper noun}} {\definition{male personal name}}
\entry{Diwa}\headword{Diwa}{\pos{Proper noun}} {\definition{male personal name}}
\entry{Dobola}\headword{Dobola}{\pos{Proper noun}} {\definition{male personal name}}
\entry{Donae}\headword{Donae}{\pos{Proper noun}} {\definition{female personal name}}
\entry{Dore}\headword{Dore}{\pos{Proper noun}} {\definition{female personal name}}
\entry{Dorin}\headword{Dorin}{\pos{Proper noun}} {\definition{female personal name}}
\entry{Dugal}\headword{Dugal}{\pos{Proper noun}} {\definition{male personal name}}
\entry{Duiya}\headword{Duiya}{\pos{Proper noun}} {\definition{male personal name}}
\entry{Edna}\headword{Edna}{\pos{Proper noun}} {\definition{female personal name}}
\entry{Edward}\headword{Edward}{\pos{Proper noun}} {\definition{male personal name}}
\entry{Elisa}\headword{Elisa}{\pos{Proper noun}} {\definition{female personal name}}
\entry{Elsie}\headword{Elsie}{\pos{Proper noun}} {\definition{female personal name}}
\entry{Erga}\headword{Erga}{\pos{Proper noun}} {\definition{female personal name}}
\entry{Eric}\headword{Eric}{\pos{Proper noun}} {\definition{male personal name}}
\entry{Essie}\headword{Essie}{\pos{Proper noun}} {\definition{female personal name}}
\entry{Evelyn}\headword{Evelyn}{\pos{Proper noun}} {\definition{female personal name}}
\entry{Ezra}\headword{Ezra}{\pos{Proper noun}} {\definition{male personal name}}
\entry{Felix}\headword{Felix}{\pos{Proper noun}} {\definition{male personal name}}
\entry{Flora}\headword{Flora}{\pos{Proper noun}} {\definition{female personal name}}
\entry{Francis}\headword{Francis}{\pos{Proper noun}} {\definition{male personal name}}
\entry{Frank}\headword{Frank}{\pos{Proper noun}} {\definition{male personal name}}
\entry{Gaem}\headword{Gaem}{\pos{Proper noun}} {\definition{female personal name}}
\entry{Galo}\headword{Galo}{\pos{Proper noun}} {\definition{male personal name}}
\entry{Galwe}\headword{Galwe}{\pos{Proper noun}} {\definition{male personal name}}
\entry{Garayi}\headword{Garayi}{\pos{Proper noun}} {\definition{male personal name}}
\entry{Garaz}\headword{Garaz}{\pos{Proper noun}} {\definition{male personal name}}
\entry{Gene}\headword{Gene}{\pos{Proper noun}} {\definition{male personal name}}
\entry{Georgina}\headword{Georgina}{\pos{Proper noun}} {\definition{female personal name}}
\entry{Geser}\headword{Geser}{\pos{Proper noun}} {\definition{male personal name}}
\entry{Gibson}\headword{Gibson}{\pos{Proper noun}} {\definition{male personal name}}
\entry{Gidu}\headword{Gidu}{\pos{Proper noun}} {\definition{male personal name}}
\entry{Gina}\headword{Gina}{\pos{Proper noun}} {\definition{female personal name}}
\entry{Ginia}\headword{Ginia}{\pos{Proper noun}} {\definition{male personal name}}
\entry{Giniya}\headword{Giniya}{\pos{Proper noun}} {\definition{male personal name}}
\entry{Giwo}\headword{Giwo}{\pos{Proper noun}} {\definition{male personal name}}
\entry{Goge}\headword{Goge}{\pos{Proper noun}} {\definition{male personal name}}
\entry{Grace}\headword{Grace}{\pos{Proper noun}} {\definition{female personal name}}
\entry{Guar}\headword{Guar}{\pos{Proper noun}} {\definition{male personal name}}
\entry{Hannah}\headword{Hannah}{\pos{Proper noun}} {\definition{female personal name}}
\entry{Hiden}\headword{Hiden}{\pos{Proper noun}} {\definition{female personal name}}
\entry{Ibetty}\headword{Ibetty}{\pos{Proper noun}} {\definition{female personal name}}
\entry{Idan}\headword{Idan}{\pos{Proper noun}} {\definition{male personal name}}
\entry{Ina}\headword{Ina}{\pos{Proper noun}} {\definition{female personal name}}
\entry{Inapa}\headword{Inapa}{\pos{Proper noun}} {\definition{male personal name}}
\entry{Inawa}\headword{Inawa}{\pos{Proper noun}} {\definition{male personal name}}
\entry{Ivan}\headword{Ivan}{\pos{Proper noun}} {\definition{male personal name}}
\entry{Jackae}\headword{Jackae}{\pos{Proper noun}} {\definition{male personal name}}
\entry{Jamilah}\headword{Jamilah}{\pos{Proper noun}} {\definition{female personal name}}
\entry{Jane}\headword{Jane}{\pos{Proper noun}} {\definition{female personal name}}
\entry{Jeff}\headword{Jeff}{\pos{Proper noun}} {\definition{male personal name}}
\entry{Jeped}\headword{Jeped}{\pos{Proper noun}} {\definition{male personal name}}
\entry{Jerry}\headword{Jerry}{\pos{Proper noun}} {\definition{male personal name}}
\entry{Joanna}\headword{Joanna}{\pos{Proper noun}} {\definition{female personal name}}
\entry{Joe}\headword{Joe}{\pos{Proper noun}} {\definition{male personal name}}
\entry{Joe-noh}\headword{Joe-noh}{\pos{Proper noun}} {\definition{male personal name}}
\entry{Joebert}\headword{Joebert}{\pos{Proper noun}} {\definition{male personal name}}
\entry{John}\headword{John}{\pos{Proper noun}} {\definition{male personal name}}
\entry{Johnny}\headword{Johnny}{\pos{Proper noun}} {\definition{male personal name}}
\entry{Jonathan}\headword{Jonathan}{\pos{Proper noun}} {\definition{male personal name}}
\entry{Jordan}\headword{Jordan}{\pos{Proper noun}} {\definition{male personal name}}
\entry{Joseph}\headword{Joseph}{\pos{Proper noun}} {\definition{male personal name}}
\entry{Joshua}\headword{Joshua}{\pos{Proper noun}} {\definition{male personal name}}
\entry{Joy-Lin}\headword{Joy-Lin}{\pos{Proper noun}} {\definition{female personal name}}
\entry{Joys}\headword{Joys}{\pos{Proper noun}} {\definition{female personal name}}
\entry{Jugu}\headword{Jugu}{\pos{Proper noun}} {\definition{male personal name}}
\entry{Julia}\headword{Julia}{\pos{Proper noun}} {\definition{female personal name}}
\entry{Julienne}\headword{Julienne}{\pos{Proper noun}} {\definition{female personal name}}
\entry{Junior}\headword{Junior}{\pos{Proper noun}} {\definition{male personal name}}
\entry{Kagär}\headword{Kagär}{\pos{Proper noun}} {\definition{female personal name}}
\entry{Kakayam}\headword{Kakayam}{\pos{Proper noun}} {\definition{female personal name}}
\entry{Kakos}\headword{Kakos}{\pos{Proper noun}} {\definition{female personal name}}
\entry{Kalamato}\headword{Kalamato}{\pos{Proper noun}} {\definition{female personal name}}
\entry{Kaldon}\headword{Kaldon}{\pos{Proper noun}} {\definition{male personal name}}
\entry{Kange}\headword{Kange}{\pos{Proper noun}} {\definition{male personal name}}
\entry{Karamapopo}\headword{Karamapopo}{\pos{Proper noun}} {\definition{female personal name}}
\entry{Karao}\headword{Karao}{\pos{Proper noun}} {\definition{male personal name}}
\entry{Karau}\headword{Karau}{\pos{Proper noun}} {\definition{male personal name}}
\entry{Karea}\headword{Karea}{\pos{Proper noun}} {\definition{male personal name}}
\entry{Karen}\headword{Karen}{\pos{Proper noun}} {\definition{female personal name}}
\entry{Karis}\headword{Karis}{\pos{Proper noun}} {\definition{female personal name}}
\entry{Kasakmai}\headword{Kasakmai}{\pos{Proper noun}} {\definition{name of a person}}
\entry{Kaso}\headword{Kaso}{\pos{Proper noun}} {\definition{male personal name}}
\entry{Katama}\headword{Katama}{\pos{Proper noun}} {\definition{male personal name}}
\entry{Kate}\headword{Kate}{\pos{Proper noun}} {\definition{female personal name}}
\entry{Katherine}\headword{Katherine}{\pos{Proper noun}} {\definition{female personal name}}
\entry{Kauga}\headword{Kauga}{\pos{Proper noun}} {\definition{male personal name}}
\entry{Kaya}\headword{Kaya}{\pos{Proper noun}} {\definition{male personal name}}
\entry{Kaysy}\headword{Kaysy}{\pos{Proper noun}} {\definition{female personal name}}
\entry{Keith}\headword{Keith}{\pos{Proper noun}} {\definition{male personal name}}
\entry{Keke}\headword{Keke}{\pos{Proper noun}} {\definition{female personal name}}
\entry{Keks}\headword{Keks}{\pos{Proper noun}} {\definition{personal name}}
\entry{Kenny}\headword{Kenny}{\pos{Proper noun}} {\definition{male personal name}}
\entry{Kesama}\headword{Kesama}{\pos{Proper noun}} {\definition{male personal name}}
\entry{Kevelyn}\headword{Kevelyn}{\pos{Proper noun}} {\definition{female personal name}}
\entry{Kewameyato}\headword{Kewameyato}{\pos{Proper noun}} {\definition{female personal name}}
\entry{Kiata}\headword{Kiata}{\pos{Proper noun}} {\definition{male personal name}}
\entry{Kidarga}\headword{Kidarga}{\pos{Proper noun}} {\definition{male personal name}}
\entry{Kila}\headword{Kila}{\pos{Proper noun}} {\definition{personal name}}
\entry{Kipling}\headword{Kipling}{\pos{Proper noun}} {\definition{male personal name}}
\entry{Kobam}\headword{Kobam}{\pos{Proper noun}} {\definition{male personal name}}
\entry{Koe}\headword{Koe}{\pos{Proper noun}} {\definition{male personal name}}
\entry{Kolmet}\headword{Kolmet}{\pos{Proper noun}} {\definition{female personal name}}
\entry{Koloam}\headword{Koloam}{\pos{Proper noun}} {\definition{male personal name}}
\entry{Kols}\headword{Kols}{\pos{Proper noun}} {\definition{male personal name}}
\entry{Kombosie}\headword{Kombosie}{\pos{Proper noun}} {\definition{male personal name}}
\entry{Kral}\headword{Kral}{\pos{Proper noun}} {\definition{female personal name}}
\entry{Kudurwe}\headword{Kudurwe}{\pos{Proper noun}} {\definition{female personal name}}
\entry{Kukua}\headword{Kukua}{\pos{Proper noun}} {\definition{male personal name}}
\entry{Kurupel}\headword{Kurupel}{\pos{Proper noun}} {\definition{male personal name}}
\entry{Kwakmae}\headword{Kwakmae}{\pos{Proper noun}} {\definition{female personal name}}
\entry{Kwalde}\headword{Kwalde}{\pos{Proper noun}} {\definition{male personal name}}
\entry{Kwale}\headword{Kwale}{\pos{Proper noun}} {\definition{female personal name}}
\entry{Kwara}\headword{Kwara}{\pos{Proper noun}} {\definition{female personal name}}
\entry{Kwe}\headword{Kwe}{\pos{Proper noun}} {\definition{male personal name}}
\entry{Lama}\headword{Lama}{\pos{Proper noun}} {\definition{female personal name}}
\entry{Lamlam}\headword{Lamlam}{\pos{Proper noun}} {\definition{name of a female ancestor (sister of Moli)}}
\entry{Letai}\headword{Letai}{\pos{Proper noun}} {\definition{female personal name}}
\entry{Lilian}\headword{Lilian}{\pos{Proper noun}} {\definition{female personal name}}
\entry{Lily}\headword{Lily}{\pos{Proper noun}} {\definition{female personal name}}
\entry{Linda}\headword{Linda}{\pos{Proper noun}} {\definition{female personal name}}
\entry{Linette}\headword{Linette}{\pos{Proper noun}} {\definition{female personal name}}
\entry{Lomae}\headword{Lomae}{\pos{Proper noun}} {\definition{female personal name}}
\entry{Loni}\headword{Loni}{\pos{Proper noun}} {\definition{female personal name}}
\entry{Lovelyn}\headword{Lovelyn}{\pos{Proper noun}} {\definition{female personal name}}
\entry{Ludwina}\headword{Ludwina}{\pos{Proper noun}} {\definition{female personal name}}
\entry{Lydia}\headword{Lydia}{\pos{Proper noun}} {\definition{female personal name}}
\entry{Lyneth}\headword{Lyneth}{\pos{Proper noun}} {\definition{female personal name}}
\entry{Madima}\headword{Madima}{\pos{Proper noun}} {\definition{female personal name}}
\entry{Mado}\headword{Mado}{\pos{Proper noun}} {\definition{male personal name}}
\entry{Madura}\headword{Madura}{\pos{Proper noun}} {\definition{male personal name}}
\entry{Maggie}\headword{Maggie}{\pos{Proper noun}} {\definition{female personal name}}
\entry{Maki}\headword{Maki}{\pos{Proper noun}} {\definition{female personal name}}
\entry{Manaleato}\headword{Manaleato}{\pos{Proper noun}} {\definition{female personal name}}
\entry{Maneya}\headword{Maneya}{\pos{Proper noun}} {\definition{name of a person}}
\entry{Manggeya}\headword{Manggeya}{\pos{Proper noun}} {\definition{female personal name}}
\entry{Mangkol}\headword{Mangkol}{\pos{Proper noun}} {\definition{female personal name}}
\entry{Marega}\headword{Marega}{\pos{Proper noun}} {\definition{male personal name}}
\entry{Maria}\headword{Maria}{\pos{Proper noun}} {\definition{female personal name}}
\entry{Marion}\headword{Marion}{\pos{Proper noun}} {\definition{female personal name}}
\entry{Martha}\headword{Martha}{\pos{Proper noun}} {\definition{female personal name}}
\entry{Martin}\headword{Martin}{\pos{Proper noun}} {\definition{male personal name}}
\entry{Maryanne}\headword{Maryanne}{\pos{Proper noun}} {\definition{female personal name}}
\entry{Mathias}\headword{Mathias}{\pos{Proper noun}} {\definition{male personal name}}
\entry{Matthew}\headword{Matthew}{\pos{Proper noun}} {\definition{male personal name}}
\entry{Mavis}\headword{Mavis}{\pos{Proper noun}} {\definition{female personal name}}
\entry{Max}\headword{Max}{\pos{Proper noun}} {\definition{male personal name}}
\entry{Mecklyn}\headword{Mecklyn}{\pos{Proper noun}} {\definition{female personal name}}
\entry{Megam}\headword{Megam}{\pos{Proper noun}} {\definition{male personal name}}
\entry{Megi}\headword{Megi}{\pos{Proper noun}} {\definition{female personal name}}
\entry{Melvin}\headword{Melvin}{\pos{Proper noun}} {\definition{male personal name}}
\entry{Merol}\headword{Merol}{\pos{Proper noun}} {\definition{female personal name}}
\entry{Mesa}\headword{Mesa}{\pos{Proper noun}} {\definition{male personal name}}
\entry{Michael}\headword{Michael}{\pos{Proper noun}} {\definition{male personal name}}
\entry{Michaelyn}\headword{Michaelyn}{\pos{Proper noun}} {\definition{female personal name}}
\entry{Michelle}\headword{Michelle}{\pos{Proper noun}} {\definition{female personal name}}
\entry{Minong}\headword{Minong}{\pos{Proper noun}} {\definition{male personal name}}
\entry{Misseilene}\headword{Misseilene}{\pos{Proper noun}} {\definition{female personal name}}
\entry{Moli}\headword{Moli}{\pos{Proper noun}} {\definition{name of a male ancestor (brother of Lamlam)}}
\entry{Mome}\headword{Mome}{\pos{Proper noun}} {\definition{female personal name}}
\entry{Moses}\headword{Moses}{\pos{Proper noun}} {\definition{male personal name}}
\entry{Munu}\headword{Munu}{\pos{Proper noun}} {\definition{male personal name}}
\entry{Musato}\headword{Musato}{\pos{Proper noun}} {\definition{female personal name}}
\entry{Muyabag}\headword{Muyabag}{\pos{Proper noun}} {\definition{male personal name}}
\entry{Nagab}\headword{Nagab}{\pos{Proper noun}} {\definition{male personal name}}
\entry{Nageg}\headword{Nageg}{\pos{Proper noun}} {\definition{male personal name}}
\entry{Naklae}\headword{Naklae}{\pos{Proper noun}} {\definition{male personal name}}
\entry{Nakuri}\headword{Nakuri}{\pos{Proper noun}} {\definition{male personal name}}
\entry{Nama}\headword{Nama}{\pos{Proper noun}} {\definition{male personal name}}
\entry{Namaya}\headword{Namaya}{\pos{Proper noun}} {\definition{female personal name}}
\entry{Narma}\headword{Narma}{\pos{Proper noun}} {\definition{male personal name}}
\entry{Nasma}\headword{Nasma}{\pos{Proper noun}} {\definition{male personal name}}
\entry{Nedlyn}\headword{Nedlyn}{\pos{Proper noun}} {\definition{female personal name}}
\entry{Nensi}\headword{Nensi}{\pos{Proper noun}} {\definition{female personal name}}
\entry{Ngerbab}\headword{Ngerbab}{\pos{Proper noun}} {\definition{male personal name}}
\entry{Nicky}\headword{Nicky}{\pos{Proper noun}} {\definition{male personal name}}
\entry{Niki}\headword{Niki}{\pos{Proper noun}} {\definition{male personal name}}
\entry{Nikol}\headword{Nikol}{\pos{Proper noun}} {\definition{female personal name}}
\entry{Niniab}\headword{Niniab}{\pos{Proper noun}} {\definition{male personal name}}
\entry{Noar}\headword{Noar}{\pos{Proper noun}} {\definition{female personal name}}
\entry{Nogat}\headword{Nogat}{\pos{Proper noun}} {\definition{male personal name}}
\entry{Nolin}\headword{Nolin}{\pos{Proper noun}} {\definition{female personal name}}
\entry{Norma}\headword{Norma}{\pos{Proper noun}} {\definition{female personal name}}
\entry{Nuopin}\headword{Nuopin}{\pos{Proper noun}} {\definition{female personal name}}
\entry{Nägäm}\headword{Nägäm}{\pos{Proper noun}} {\definition{male personal name}}
\entry{Nänga}\headword{Nänga}{\pos{Proper noun}} {\definition{female personal name}}
\entry{Obewa}\headword{Obewa}{\pos{Proper noun}} {\definition{male personal name}}
\entry{Ogoa}\headword{Ogoa}{\pos{Proper noun}} {\definition{male personal name}}
\entry{Olalea}\headword{Olalea}{\pos{Proper noun}} {\definition{female personal name}}
\entry{Ongg}\headword{Ongg}{\pos{Proper noun}} {\definition{male personal name}}
\entry{Ouli}\headword{Ouli}{\pos{Proper noun}} {\definition{female personal name}}
\entry{Paelet}\headword{Paelet}{\pos{Proper noun}} {\definition{Pilate}}
\entry{Paine}\headword{Paine}{\pos{Proper noun}} {\definition{male personal name}}
\entry{Papon}\headword{Papon}{\pos{Proper noun}} {\definition{male personal name}}
\entry{Patha}\headword{Patha}{\pos{Proper noun}} {\definition{male personal name}}
\entry{Paul}\headword{Paul}{\pos{Proper noun}} {\definition{male personal name}}
\entry{Pauma}\headword{Pauma}{\pos{Proper noun}} {\definition{female personal name}}
\entry{Pentai}\headword{Pentai}{\pos{Proper noun}} {\definition{female personal name}}
\entry{Petepo}\headword{Petepo}{\pos{Proper noun}} {\definition{female personal name}}
\entry{Pewe}\headword{Pewe}{\pos{Proper noun}} {\definition{male personal name}}
\entry{Piasorosoro}\headword{Piasorosoro}{\pos{Proper noun}} {\definition{male personal name}}
\entry{Pingam}\headword{Pingam}{\pos{Proper noun}} {\definition{female personal name}}
\entry{Pita}\headword{Pita}{\pos{Proper noun}} {\definition{male personal name}}
\entry{Pol}\headword{Pol}{\pos{Proper noun}} {\definition{male personal name}}
\entry{Polin}\headword{Polin}{\pos{Proper noun}} {\definition{female personal name}}
\entry{Priscilla}\headword{Priscilla}{\pos{Proper noun}} {\definition{female personal name}}
\entry{Priski}\headword{Priski}{\pos{Proper noun}} {\definition{female personal name}}
\entry{Puinde}\headword{Puinde}{\pos{Proper noun}} {\definition{male personal name}}
\entry{Queenie}\headword{Queenie}{\pos{Proper noun}} {\definition{female personal name}}
\entry{Quinteth}\headword{Quinteth}{\pos{Proper noun}} {\definition{female personal name}}
\entry{Quinton}\headword{Quinton}{\pos{Proper noun}} {\definition{male personal name}}
\entry{Ranky}\headword{Ranky}{\pos{Proper noun}} {\definition{male personal name}}
\entry{Rasol}\headword{Rasol}{\pos{Proper noun}} {\definition{male personal name}}
\entry{Raynold}\headword{Raynold}{\pos{Proper noun}} {\definition{male personal name}}
\entry{Redley}\headword{Redley}{\pos{Proper noun}} {\definition{male personal name}}
\entry{Reend}\headword{Reend}{\pos{Proper noun}} {\definition{male personal name}}
\entry{Regina}\headword{Regina}{\pos{Proper noun}} {\definition{female personal name}}
\entry{Rex}\headword{Rex}{\pos{Proper noun}} {\definition{male personal name}}
\entry{Rhoda}\headword{Rhoda}{\pos{Proper noun}} {\definition{female personal name}}
\entry{Richard}\headword{Richard}{\pos{Proper noun}} {\definition{male personal name}}
\entry{Rind}\headword{Rind}{\pos{Proper noun}} {\definition{male personal name}}
\entry{Roaele}\headword{Roaele}{\pos{Proper noun}} {\definition{male personal name}}
\entry{Roak}\headword{Roak}{\pos{Proper noun}} {\definition{male personal name}}
\entry{Robai}\headword{Robai}{\pos{Proper noun}} {\definition{female personal name}}
\entry{Roda}\headword{Roda}{\pos{Proper noun}} {\definition{female personal name}}
\entry{Rose}\headword{Rose}{\pos{Proper noun}} {\definition{female personal name}}
\entry{Rosela}\headword{Rosela}{\pos{Proper noun}} {\definition{female personal name}}
\entry{Rowak}\headword{Rowak}{\pos{Proper noun}} {\definition{male personal name}}
\entry{Saduwa}\headword{Saduwa}{\pos{Proper noun}} {\definition{name of a person}}
\entry{Saly}\headword{Saly}{\pos{Proper noun}} {\definition{male personal name}}
\entry{Sam}\headword{Sam}{\pos{Proper noun}} {\definition{male personal name}}
\entry{Samat}\headword{Samat}{\pos{Proper noun}} {\definition{female personal name}}
\entry{Samson}\headword{Samson}{\pos{Proper noun}} {\definition{male personal name}}
\entry{Samuel}\headword{Samuel}{\pos{Proper noun}} {\definition{male personal name}}
\entry{Sandra}\headword{Sandra}{\pos{Proper noun}} {\definition{female personal name}}
\entry{Sanford}\headword{Sanford}{\pos{Proper noun}} {\definition{male personal name}}
\entry{Sapusa}\headword{Sapusa}{\pos{Proper noun}} {\definition{female personal name}}
\entry{Sara}\headword{Sara}{\pos{Proper noun}} {\definition{female personal name}}
\entry{Sarah}\headword{Sarah}{\pos{Proper noun}} {\definition{female personal name}}
\entry{Sarbi}\headword{Sarbi}{\pos{Proper noun}} {\definition{female personal name}}
\entry{Sasi}\headword{Sasi}{\pos{Proper noun}} {\definition{female personal name}}
\entry{Sawa}\headword{Sawa}{\pos{Proper noun}} {\definition{male personal name}}
\entry{Sawapo}\headword{Sawapo}{\pos{Proper noun}} {\definition{male personal name}}
\entry{Sharon}\headword{Sharon}{\pos{Proper noun}} {\definition{female personal name}}
\entry{Sintia}\headword{Sintia}{\pos{Proper noun}} {\definition{female personal name}}
\entry{Sisuar}\headword{Sisuar}{\pos{Proper noun}} {\definition{female personal name}}
\entry{Skola}\headword{Skola}{\pos{Proper noun}} {\definition{female personal name}}
\entry{Soba}\headword{Soba}{\pos{Proper noun}} {\definition{male personal name}}
\entry{Sobam}\headword{Sobam}{\pos{Proper noun}} {\definition{male personal name}}
\entry{Soka}\headword{Soka}{\pos{Proper noun}} {\definition{male personal name}}
\entry{Soma}\headword{Soma}{\pos{Proper noun}} {\definition{male personal name}}
\entry{Sowati}\headword{Sowati}{\pos{Proper noun}} {\definition{male personal name}}
\entry{Stanis}\headword{Stanis}{\pos{Proper noun}} {\definition{male personal name}}
\entry{Stashalyn}\headword{Stashalyn}{\pos{Proper noun}} {\definition{female personal name}}
\entry{Steven}\headword{Steven}{\pos{Proper noun}} {\definition{male personal name}}
\entry{Suliki}\headword{Suliki}{\pos{Proper noun}} {\definition{male personal name}}
\entry{Susan}\headword{Susan}{\pos{Proper noun}} {\definition{female personal name}}
\entry{Suwede}\headword{Suwede}{\pos{Proper noun}} {\definition{male personal name}}
\entry{Sylvien}\headword{Sylvien}{\pos{Proper noun}} {\definition{female personal name}}
\entry{Sägrep}\headword{Sägrep}{\pos{Proper noun}} {\definition{male personal name}}
\entry{Tag}\headword{Tag}{\pos{Proper noun}} {\definition{personal name}}
\entry{Tergo}\headword{Tergo}{\pos{Proper noun}} {\definition{male personal name}}
\entry{Terrance}\headword{Terrance}{\pos{Proper noun}} {\definition{male personal name}}
\entry{Tewa}\headword{Tewa}{\pos{Proper noun}} {\definition{male personal name}}
\entry{Thomas}\headword{Thomas}{\pos{Proper noun}} {\definition{male personal name}}
\entry{Tina}\headword{Tina}{\pos{Proper noun}} {\definition{female personal name}}
\entry{Tomato}\headword{Tomato}{\pos{Proper noun}} {\definition{female personal name}}
\entry{Tonny}\headword{Tonny}{\pos{Proper noun}} {\definition{male personal name. Bearers: (1) Tonny Warama, son of Warama Kurupel and Wagiba Geser and member of Ende Language Committee.}}
\entry{Ttae}\headword{Ttae}{\pos{Proper noun}} {\definition{male personal name}}
\entry{Tube}\headword{Tube}{\pos{Proper noun}} {\definition{male personal name}}
\entry{Tätän}\headword{Tätän}{\pos{Proper noun}} {\definition{unisex personal name}}
\entry{Ubrag}\headword{Ubrag}{\pos{Proper noun}} {\definition{male personal name}}
\entry{Uzaba}\headword{Uzaba}{\pos{Proper noun}} {\definition{male personal name}}
\entry{Uziag}\headword{Uziag}{\pos{Proper noun}} {\definition{male personal name}}
\entry{Vanessa}\headword{Vanessa}{\pos{Proper noun}} {\definition{female personal name}}
\entry{Victoria}\headword{Victoria}{\pos{Proper noun}} {\definition{female personal name}}
\entry{Vincent}\headword{Vincent}{\pos{Proper noun}} {\definition{male personal name}}
\entry{Wagiba}\headword{Wagiba}{\pos{Proper noun}} {\definition{female personal name}}
\entry{Wai}\headword{Wai}{\pos{Proper noun}} {\definition{female personal name}}
\entry{Wainum}\headword{Wainum}{\pos{Proper noun}} {\definition{female personal name}}
\entry{Wala}\headword{Wala}{\pos{Proper noun}} {\definition{personal name}}
\entry{Wane}\headword{Wane}{\pos{Proper noun}} {\definition{male personal name}}
\entry{Warama}\headword{Warama}{\pos{Proper noun}} {\definition{male personal name}}
\entry{Warani}\headword{Warani}{\pos{Proper noun}} {\definition{male personal name}}
\entry{Wareka}\headword{Wareka}{\pos{Proper noun}} {\definition{female personal name}}
\entry{Waweba}\headword{Waweba}{\pos{Proper noun}} {\definition{male personal name}}
\entry{Wed}\headword{Wed}{\pos{Proper noun}} {\definition{female personal name}}
\entry{Wendy}\headword{Wendy}{\pos{Proper noun}} {\definition{female personal name}}
\entry{Wesli}\headword{Wesli}{\pos{Proper noun}} {\definition{male personal name}}
\entry{Wiben}\headword{Wiben}{\pos{Proper noun}} {\definition{male personal name}}
\entry{Willie}\headword{Willie}{\pos{Proper noun}} {\definition{male personal name}}
\entry{Wilma}\headword{Wilma}{\pos{Proper noun}} {\definition{female personal name}}
\entry{Winny}\headword{Winny}{\pos{Proper noun}} {\definition{female personal name}}
\entry{Winson}\headword{Winson}{\pos{Proper noun}} {\definition{male personal name}}
\entry{Wizing}\headword{Wizing}{\pos{Proper noun}} {\definition{male personal name}}
\entry{Wun}\headword{Wun}{\pos{Proper noun}} {\definition{male personal name}}
\entry{Yarbab}\headword{Yarbab}{\pos{Proper noun}} {\definition{male personal name}}
\entry{Yawani}\headword{Yawani}{\pos{Proper noun}} {\definition{male personal name}}
\entry{Yawin}\headword{Yawin}{\pos{Proper noun}} {\definition{female personal name}}
\entry{Yina}\headword{Yina}{\pos{Proper noun}} {\definition{female personal name}}
\entry{Yuga}\headword{Yuga}{\pos{Proper noun}} {\definition{male personal name}}
\entry{Yugui}\headword{Yugui}{\pos{Proper noun}} {\definition{female personal name}}
\entry{Zedem}\headword{Zedem}{\pos{Proper noun}} {\definition{male personal name}}
\entry{Zeims}\headword{Zeims}{\pos{Proper noun}} {\definition{male personal name}}
\entry{Zelma}\headword{Zelma}{\pos{Proper noun}} {\definition{female personal name}}
\entry{Zina}\headword{Zina}{\pos{Proper noun}} {\definition{female personal name}}
\entry{Zon}\headword{Zon}{\pos{Proper noun}} {\definition{male personal name (John)}}
\entry{Zonas}\headword{Zonas}{\pos{Proper noun}} {\definition{male personal name}}
\entry{Zosep}\headword{Zosep}{\pos{Proper noun}} {\definition{male personal name}}
\entry{Zudas}\headword{Zudas}{\pos{Proper noun}} {\definition{male personal name}}
\entry{bin}\headword{bin}{\pos{Noun}} {\definition{name}}
\entry{erär}\headword{erär}{\pos{Transitive S verb}} {\definition{to name; pass down a name}}
\entry{nadum}\headword{nadum}{\pos{Noun}} {\definition{namesake}}
\end{entrylist}

\section*{9.7.1.3 Clan names}
\begin{entrylist}
\entry{bunkuttang}\headword{bunkuttang}{\pos{Noun}} {\definition{catfish}}
\entry{mabun}\headword{mabun}{\pos{Noun}} {\definition{clan totem (the sacred symbol of a clan group; usually a plant, animal, or body part)}}
\entry{tawar}\headword{tawar}{\pos{Noun}} {\definition{totem symbol (thing that represents a clan)}}
\end{entrylist}

\section*{9.7.1.4 Tribal names}
\begin{entrylist}
\entry{sip}\headword{sip}{\pos{Noun}} {\definition{chief}}
\entry{traeb}\headword{traeb}{\pos{Noun}} {\definition{tribe}}
\end{entrylist}

\section*{9.7.1.5 Names of languages}
\begin{entrylist}
\entry{Agob}\headword{Agob}{\pos{Proper noun}} {\definition{language name}}
\entry{Agäb}\headword{Agäb}{\pos{Proper noun}} {\definition{Agob language (Pahoturi River language)}}
\entry{Bine}\headword{Bine}{\pos{Proper noun}} {\definition{Bine language (spoken to the east)}}
\entry{Bitur}\headword{Bitur}{\pos{Proper noun}} {\definition{Bitur language (spoken to the north)}}
\entry{Ddeleag}\headword{Ddeleag}{\pos{Proper noun}} {\definition{Ende dialect spoken in Ddele}}
\entry{Em}\headword{Em}{\pos{Noun}} {\definition{Em language (Pahoturi River language spoken in Kurunti, Kibuli, Beyambod)}}
\entry{Ende}\headword{Ende}{\pos{Proper noun}} {\definition{Ende language (Pahoturi River language spoken in Limol, Malam, and Kinkin)}}
\entry{Gidra}\headword{Gidra}{\pos{Proper noun}} {\definition{(possibly derogatory) Wipi language}}
\entry{Gogodala}\headword{Gogodala}{\pos{Proper noun}} {\definition{Gogodala language}}
\entry{Ibru}\headword{Ibru}{\pos{Noun}} {\definition{Hebrew}}
\entry{Idi}\headword{Idi}{\pos{Proper noun}} {\definition{Idi language (Pahoturi River language spoken in Dimsisi, Sibidiri, Dimiri, Iblamand, and Biram)}}
\entry{Ingglis}\headword{Ingglis}{\pos{Proper noun}} {\definition{English language}}
\entry{Inpir}\headword{Inpir}{\pos{Proper noun}} {\definition{Makayam/Tirio language}}
\entry{Kawam}\headword{Kawam}{\pos{Proper noun}} {\definition{Kawam language (Pahoturi River language spoken in Wim)}}
\entry{Kiwai}\headword{Kiwai}{\pos{Proper noun}} {\definition{Kiwai language (offical language of the region, native language of Daru; children's school songs are sometimes in this language)}}
\entry{Käballag}\headword{Käballag}{\pos{Proper noun}} {\definition{Ende dialect spoken in Käball}}
\entry{Limollang}\headword{Limollang}{\pos{Proper noun}} {\definition{Ende dialect spoken in Limol}}
\entry{Motu}\headword{Motu}{\pos{Proper noun}} {\definition{Hiri Motu language}}
\entry{Mäkayam}\headword{Mäkayam}{\pos{Proper noun}} {\definition{Makayam/Tirio language}}
\entry{Pizin}\headword{Pizin}{\pos{Proper noun}} {\definition{Tok Pisin}}
\entry{Tame}\headword{Tame}{\pos{Proper noun}} {\definition{Taeme language (Pahoturi River language spoken in Kinkin alongside Ende)}}
\entry{Tizag}\headword{Tizag}{\pos{Proper noun}} {\definition{Ende dialect}}
\entry{Tok pisin}\headword{Tok pisin}{\pos{Proper noun}} {\definition{Tok Pisin}}
\entry{Wipi}\headword{Wipi}{\pos{Proper noun}} {\definition{Wipi language}}
\entry{Yao}\headword{Yao}{\pos{Proper noun}} {\definition{Taeme language}}
\entry{markae eka}\headword{markae eka}{\pos{Proper noun}} {\definition{English}}
\entry{pällämpälläm eka}\headword{pällämpälläm eka}{\pos{Proper noun}} {\definition{English}}
\end{entrylist}

\section*{9.7.1.6 Nickname}
\begin{entrylist}
\entry{nikbin}\headword{nikbin}{\pos{Noun}} {\definition{nickname}}
\end{entrylist}

\section*{9.7.2 Name of a place}
\begin{entrylist}
\entry{Abam}\headword{Abam}{\pos{Proper noun}} {\definition{Abam (Wipi-speaking village in Oriomo-Bituri LLG; GPS: -8.926607 143.190246)}}
\entry{Aberegerem}\headword{Aberegerem}{\pos{Proper noun}} {\definition{Aberagerema (in Kiwai Rural LLG)}}
\entry{Abom}\headword{Abom}{\pos{Proper noun}} {\definition{Abom (toponym)}}
\entry{Agan}\headword{Agan}{\pos{Proper noun}} {\definition{Agan (toponym)}}
\entry{Ainor}\headword{Ainor}{\pos{Proper noun}} {\definition{Ainor (toponym)}}
\entry{Ali}\headword{Ali}{\pos{Proper noun}} {\definition{place name}}
\entry{Allambun}\headword{Allambun}{\pos{Proper noun}} {\definition{Allambun (camping place)}}
\entry{Alläpma}\headword{Alläpma}{\pos{Proper noun}} {\definition{Alläpma (toponym)}}
\entry{Amne kona}\headword{Amne kona}{\pos{Proper noun}} {\definition{Central corner (in Limol)}}
\entry{Amrika}\headword{Amrika}{\pos{Proper noun}} {\definition{America}}
\entry{Arupi}\headword{Arupi}{\pos{Proper noun}} {\definition{Arupi (toponym)}}
\entry{Arägapetkae}\headword{Arägapetkae}{\pos{Proper noun}} {\definition{Arägapetkae (toponym)}}
\entry{Awaba}\headword{Awaba}{\pos{Proper noun}} {\definition{Awaba (toponym)}}
\entry{Awi}\headword{Awi}{\pos{Proper noun}} {\definition{Awi (toponym)}}
\entry{Baiduwa}\headword{Baiduwa}{\pos{Proper noun}} {\definition{Baiduwa (toponym)}}
\entry{Baim}\headword{Baim}{\pos{Proper noun}} {\definition{Baim (toponym)}}
\entry{Balimo}\headword{Balimo}{\pos{Proper noun}} {\definition{Balimo (toponym)}}
\entry{Baogab}\headword{Baogab}{\pos{Proper noun}} {\definition{Baogab (an island in Karama swamp used for camping; has coconuts and bananas)}}
\entry{Basido}\headword{Basido}{\pos{Proper noun}} {\definition{Basido (toponym)}}
\entry{Basido kona}\headword{Basido kona}{\pos{Proper noun}} {\definition{Pastor's corner (in Limol)}}
\entry{Ber}\headword{Ber}{\pos{Proper noun}} {\definition{Ber (toponym)}}
\entry{Beradi}\headword{Beradi}{\pos{Proper noun}} {\definition{Beradi (toponym)}}
\entry{Bimadbn}\headword{Bimadbn}{\pos{Proper noun}} {\definition{Bimadbn (toponym)}}
\entry{Binyomoll Källäm}\headword{Binyomoll Källäm}{\pos{Proper noun}} {\definition{Binyomoll Pond (on the road to Kinkin)}}
\entry{Bisuaka}\headword{Bisuaka}{\pos{Proper noun}} {\definition{Bisuaka (Bituri-speaking village in Oriomo-Bituri Rural LLG; has a primary school but no aid post)}}
\entry{Biyewolatt}\headword{Biyewolatt}{\pos{Proper noun}} {\definition{Biyewolatt (toponym)}}
\entry{Bok}\headword{Bok}{\pos{Proper noun}} {\definition{Buk (Taeme-speaking village in Morehead Rural LLG; from Limol, one must pass through Kinkin and Kondobol)}}
\entry{Bolloll}\headword{Bolloll}{\pos{Proper noun}} {\definition{Bolloll (toponym, on the southward road to Malam)}}
\entry{Bonybony}\headword{Bonybony}{\pos{Proper noun}} {\definition{Bonybony (camping, garden, and sago place (AX94))}}
\entry{Boze}\headword{Boze}{\pos{Proper noun}} {\definition{Boze (Agob- and Bine-speaking village in Oriomo-Bituri Rural LLG; from Limol, one must pass through Malam, Kurunti, and Kibuli)}}
\entry{Bozorob}\headword{Bozorob}{\pos{Proper noun}} {\definition{Bozorob (camping place on the road to Kurunti; from Limol, one must pass through Malam)}}
\entry{Buiddobuiddog}\headword{Buiddobuiddog}{\pos{Proper noun}} {\definition{Buiddobuiddog (sago place and creek; also a road (Top L AZ96))}}
\entry{Buyubun}\headword{Buyubun}{\pos{Proper noun}} {\definition{Buyubun (toponym)}}
\entry{Buzi}\headword{Buzi}{\pos{Proper noun}} {\definition{Buzi (in Kiwai Rural LLG)}}
\entry{Bämäg}\headword{Bämäg}{\pos{Proper noun}} {\definition{Bämäg (toponym)}}
\entry{Dabe}\headword{Dabe}{\pos{Proper noun}} {\definition{Dabe (toponym)}}
\entry{Daiba}\headword{Daiba}{\pos{Proper noun}} {\definition{garden and sago place of Jerry Dareda (along the road to Bisuaka)}}
\entry{Derideri}\headword{Derideri}{\pos{Proper noun}} {\definition{Derideri (Nambo-speaking village in Morehead Rural LLG; east of Morehead)}}
\entry{Dewara}\headword{Dewara}{\pos{Proper noun}} {\definition{Dewara (Were/Kiunum-speaking village in Gogodala Rural LLG, along the Fly River; from Limol, one must pass through Upiara and Kondobol)}}
\entry{Digabo Källäm}\headword{Digabo Källäm}{\pos{Proper noun}} {\definition{Digabo Pond (canoe and fishing place in Taolang; road to Taolang is in AX95)}}
\entry{Dikullowang}\headword{Dikullowang}{\pos{Proper noun}} {\definition{Dikullowang (small island and hunting place in Taolang)}}
\entry{Dimiri}\headword{Dimiri}{\pos{Proper noun}} {\definition{Dimiri/Demeri (Idi-speaking village in Morehead Rural LLG; from Limol, one must pass through Kuiwang)}}
\entry{Dimisisi}\headword{Dimisisi}{\pos{Proper noun}} {\definition{Dimisisi (Idi-speaking village in Morehead Rural LLG; from Limol, one must pass through Kinkin and Bok)}}
\entry{Doumori}\headword{Doumori}{\pos{Proper noun}} {\definition{Doumori (in Kiwai Rural LLG)}}
\entry{Dowabunang}\headword{Dowabunang}{\pos{Proper noun}} {\definition{camping, sago, hunting place, and garden of Kaoga Dobola (on the road to Kinkin AZ94)}}
\entry{Duaba}\headword{Duaba}{\pos{Proper noun}} {\definition{Duaba (Gogodala-speaking village in Gogodala Rural LLG)}}
\entry{Dukumiang}\headword{Dukumiang}{\pos{Proper noun}} {\definition{Dukumiang (fishing, sago, hunting place, and garden; camping place of Bewag Bewag; R side of AY96, take southward road; northward road goes to Kapal)}}
\entry{Dum Tutu}\headword{Dum Tutu}{\pos{Proper noun}} {\definition{Dum Mountain}}
\entry{Dumoll}\headword{Dumoll}{\pos{Proper noun}} {\definition{Dumoll (Taolang side garden and camping place of Gidu Jerry, Kols Baewa, and Wareya Giniya)}}
\entry{Duwaba}\headword{Duwaba}{\pos{Proper noun}} {\definition{Duaba (in Gogodala Rural LLG)}}
\entry{Edeb}\headword{Edeb}{\pos{Proper noun}} {\definition{garden, camping, and hunting place of Tewa (on the other side of Karama swamp)}}
\entry{Egapo}\headword{Egapo}{\pos{Proper noun}} {\definition{camping place and large community garden (in Limol; road to garden is visible on L of AZ96)}}
\entry{Enza}\headword{Enza}{\pos{Proper noun}} {\definition{Enza (Bitur-speaking village in Oriomo-Bituri Rural LLG; from Limol, one must pass through Bisuaka)}}
\entry{Eramang}\headword{Eramang}{\pos{Proper noun}} {\definition{Eramang (the main canoe place in Limol; for camping, fishing, hunting)}}
\entry{Gaima}\headword{Gaima}{\pos{Proper noun}} {\definition{Gaima (toponym)}}
\entry{Galibma}\headword{Galibma}{\pos{Proper noun}} {\definition{sacred place of Biku (Madura) Kangge (on the road to Bisuaka, after Limol ma kuddäll; bush area, creek)}}
\entry{Goeg wälläng}\headword{Goeg wälläng}{\pos{Proper noun}} {\definition{garden place in the bush on the west side of the road to Taolang}}
\entry{Gullbe Bikme Auma}\headword{Gullbe Bikme Auma}{\pos{Proper noun}} {\definition{sacred place of Dareda (near Karama swamp)}}
\entry{Gullbe bo llädayatt}\headword{Gullbe bo llädayatt}{\pos{Proper noun}} {\definition{sacred place of Dareda (large hill)}}
\entry{Gullbe bo makollamatt}\headword{Gullbe bo makollamatt}{\pos{Proper noun}} {\definition{Dareda's sacred place (hill on the road to Kinkin)}}
\entry{Gullem suwe}\headword{Gullem suwe}{\pos{Proper noun}} {\definition{gardening place; Galo's sacred place (AZ97)}}
\entry{Gälabi}\headword{Gälabi}{\pos{Proper noun}} {\definition{Gälabi (Wipi-speaking village in Oriomo-Bituri Rural LLG; from Limol, one must pass through Wipim)}}
\entry{Gäläb}\headword{Gäläb}{\pos{Proper noun}} {\definition{Gälab (sago place and Geoff Rowak's camping place in Limol)}}
\entry{Ibikang}\headword{Ibikang}{\pos{Proper noun}} {\definition{Ibikang (Kawam-speaking settlement of Wim; not far from Limol (AY96))}}
\entry{Iden}\headword{Iden}{\pos{Proper noun}} {\definition{Eden}}
\entry{Inpiakma}\headword{Inpiakma}{\pos{Proper noun}} {\definition{Inpiakma (toponym)}}
\entry{Ipott}\headword{Ipott}{\pos{Proper noun}} {\definition{Ipott (on the road to Bisuaka near Limol ma kuddäll)}}
\entry{Iräm}\headword{Iräm}{\pos{Proper noun}} {\definition{creek and washing place (in Limol, AZ95)}}
\entry{Iräm tatuma}\headword{Iräm tatuma}{\pos{Proper noun}} {\definition{Iräm washing place}}
\entry{Isago}\headword{Isago}{\pos{Proper noun}} {\definition{Isago (big island, perhaps in Fly River)}}
\entry{Kadawa}\headword{Kadawa}{\pos{Proper noun}} {\definition{Kadawa (in Kiwai Rural LLG)}}
\entry{Kakeya}\headword{Kakeya}{\pos{Proper noun}} {\definition{Kakeya (bush, camping and sago place of Baba Zi from Upiara; on the road to Bisuaka)}}
\entry{Kapal}\headword{Kapal}{\pos{Proper noun}} {\definition{Kapal (Wipi- and Kawam-speaking village in Oriomo-Bituri Rural LLG; has airstrip, aid post, primary school; from Limol, one must pass through Bisuaka)}}
\entry{Kapangang bun}\headword{Kapangang bun}{\pos{Proper noun}} {\definition{Kapangang bun (sago place on the way to Egapo)}}
\entry{Kaparnaom}\headword{Kaparnaom}{\pos{Proper noun}} {\definition{Capernaum}}
\entry{Karama}\headword{Karama}{\pos{Proper noun}} {\definition{Karama (swamp and canoe place in Limol)}}
\entry{Kasakmai}\headword{Kasakmai}{\pos{Proper noun}} {\definition{Kasakmai (toponym)}}
\entry{Kasimap}\headword{Kasimap}{\pos{Proper noun}} {\definition{Kasimap (Abom-speaking village in Gogodala Rural LLG, near Zanor; has an elementary school)}}
\entry{Kawiapo}\headword{Kawiapo}{\pos{Proper noun}} {\definition{Kaviapu (village in Gogodala Rural LLG, near Tapila)}}
\entry{Kename}\headword{Kename}{\pos{Proper noun}} {\definition{Kename (village in Gogodala Rural LLG; on an island in the Fly River)}}
\entry{Kergowa}\headword{Kergowa}{\pos{Proper noun}} {\definition{Kergowa (in Gogodala Rural LLG; near Balimo)}}
\entry{Keti}\headword{Keti}{\pos{Proper noun}} {\definition{Kurupel Täräp (Limol village), which was moved from Old Limol to Old Man Kurupel's camping place approximately four generations before 2015}}
\entry{Kibobma}\headword{Kibobma}{\pos{Proper noun}} {\definition{Kibobma (previous settlement of Limol village; on the road to Kinkin, near the creeks)}}
\entry{Kibuli}\headword{Kibuli}{\pos{Proper noun}} {\definition{Kibuli (Em-speaking village in Oriomo-Bituri Rural LLG; near Kurunti)}}
\entry{Kikori}\headword{Kikori}{\pos{Proper noun}} {\definition{Kikori (town in Kikori District, located on the Kikori Delta)}}
\entry{Kini}\headword{Kini}{\pos{Proper noun}} {\definition{Kini (in Gogodala Rural LLG; near Balimo and Awaba)}}
\entry{Kinkin}\headword{Kinkin}{\pos{Proper noun}} {\definition{Kinkin (Ende- and Taeme-speaking village in Oriomo-Bituri Rural LLG, near Limol)}}
\entry{Kobemitang}\headword{Kobemitang}{\pos{Proper noun}} {\definition{Kobemitang (toponym)}}
\entry{Koboddag}\headword{Koboddag}{\pos{Proper noun}} {\definition{Koboddag (toponym)}}
\entry{Koebänang}\headword{Koebänang}{\pos{Proper noun}} {\definition{Koebänang (sago and hunting place; old settlement near Buddobuddog)}}
\entry{Koenbäll kutt}\headword{Koenbäll kutt}{\pos{Proper noun}} {\definition{Koenbäll kutt (sago and washing place of Paine and Warama Kurupel)}}
\entry{Kondobol}\headword{Kondobol}{\pos{Proper noun}} {\definition{Kondobol (Taeme-speaking village in Morehead Rural LLG; from Limol, one must pass through Kinkin)}}
\entry{Kondobu}\headword{Kondobu}{\pos{Proper noun}} {\definition{Konedobu (in Gogodala Rural LLG)}}
\entry{Kuiwang}\headword{Kuiwang}{\pos{Proper noun}} {\definition{Kuiwang (Taeme-speaking village in Morehead Rural LLG; from Limol, one must pass through Malam)}}
\entry{Kullme}\headword{Kullme}{\pos{Proper noun}} {\definition{Kullme (garden place near Egapo; filled with abandoned rubber trees)}}
\entry{Kullopang}\headword{Kullopang}{\pos{Proper noun}} {\definition{Kullopang (sago and garden place of Kaoga Dobola; on the shortcut road to Kinkin)}}
\entry{Kunyemäll}\headword{Kunyemäll}{\pos{Proper noun}} {\definition{Kunyemäll (on the road to Malam near Zarma; filled with black palms that were cut for the school)}}
\entry{Kur}\headword{Kur}{\pos{Proper noun}} {\definition{Kur (Wipi-speaking village in Oriomo-Bituri Rural LLG; on the road to Oriomo)}}
\entry{Kurinti}\headword{Kurinti}{\pos{Proper noun}} {\definition{toponym}}
\entry{Kutpi Käp}\headword{Kutpi Käp}{\pos{Proper noun}} {\definition{Kutpi Käp (toponym)}}
\entry{Kuyu}\headword{Kuyu}{\pos{Proper noun}} {\definition{garden place of Matthew Bodog and Kaoga Dobola in Limol}}
\entry{Kwallangkäbäll}\headword{Kwallangkäbäll}{\pos{Proper noun}} {\definition{community garden place in Limol}}
\entry{Källängmäll}\headword{Källängmäll}{\pos{Proper noun}} {\definition{sago place near Old Kibobma}}
\entry{Kättpälläk bällämang}\headword{Kättpälläk bällämang}{\pos{Proper noun}} {\definition{Jerry Dareda's sacred place (near ttälebun, on the road to Kinkin, near Binyomoll)}}
\entry{Käza kup ine ma}\headword{Käza kup ine ma}{\pos{Proper noun}} {\definition{well and sago place of Kwakmae in Limol (behind aid post)}}
\entry{Lewada}\headword{Lewada}{\pos{Proper noun}} {\definition{Lewada (Makayam-speaking village in Gogodala Rural LLG, on the Fly River; GPS: 8.327787, 142.785487)}}
\entry{Mabudawan}\headword{Mabudawan}{\pos{Proper noun}} {\definition{Mabudawan/Mabaduan (Agob-speaking village in Kiwai Rural LLG near Saibai Island)}}
\entry{Mamen}\headword{Mamen}{\pos{Proper noun}} {\definition{Mamen (toponym)}}
\entry{Masingara}\headword{Masingara}{\pos{Proper noun}} {\definition{Masingara (Bine-speaking village in Oriomo-Bituri Rural LLG)}}
\entry{Mata}\headword{Mata}{\pos{Proper noun}} {\definition{Mata (in Morehead Rural LLG)}}
\entry{Meliye}\headword{Meliye}{\pos{Proper noun}} {\definition{Meliye (toponym)}}
\entry{Mingkällbun}\headword{Mingkällbun}{\pos{Proper noun}} {\definition{Mingkällbun (toponym)}}
\entry{Minkom}\headword{Minkom}{\pos{Proper noun}} {\definition{place name}}
\entry{Mompelang}\headword{Mompelang}{\pos{Proper noun}} {\definition{Mompelang (toponym)}}
\entry{Muidebag}\headword{Muidebag}{\pos{Proper noun}} {\definition{Muidebag (toponym)}}
\entry{Mul}\headword{Mul}{\pos{Proper noun}} {\definition{Mul (toponym)}}
\entry{Mur}\headword{Mur}{\pos{Proper noun}} {\definition{Mur (toponym)}}
\entry{Mätär}\headword{Mätär}{\pos{Proper noun}} {\definition{Mätär (toponym)}}
\entry{Nazaret}\headword{Nazaret}{\pos{Proper noun}} {\definition{Nazareth}}
\entry{Ngao}\headword{Ngao}{\pos{Proper noun}} {\definition{Ngao (toponym)}}
\entry{Ngeba}\headword{Ngeba}{\pos{Proper noun}} {\definition{Ngeba (toponym)}}
\entry{Nugini}\headword{Nugini}{\pos{Proper noun}} {\definition{New Guinea}}
\entry{Ogbaperma}\headword{Ogbaperma}{\pos{Proper noun}} {\definition{Ogbaperma (camping place)}}
\entry{Old Maoto}\headword{Old Maoto}{\pos{Proper noun}} {\definition{Old Maoto (toponym)}}
\entry{Ono}\headword{Ono}{\pos{Proper noun}} {\definition{Ono (toponym)}}
\entry{Opo}\headword{Opo}{\pos{Proper noun}} {\definition{Opo (toponym)}}
\entry{Pakllepakllemäll}\headword{Pakllepakllemäll}{\pos{Proper noun}} {\definition{Pakllepakllemäll (camping place)}}
\entry{Parama}\headword{Parama}{\pos{Proper noun}} {\definition{Parama (in Kiwai Rural LLG)}}
\entry{Pedaya}\headword{Pedaya}{\pos{Proper noun}} {\definition{Pedaya (toponym)}}
\entry{Petom}\headword{Petom}{\pos{Proper noun}} {\definition{Petom (toponym)}}
\entry{Pidortama}\headword{Pidortama}{\pos{Proper noun}} {\definition{Pidortama (toponym)}}
\entry{Pisi}\headword{Pisi}{\pos{Proper noun}} {\definition{Pisi (in Gogodala Rural LLG)}}
\entry{Piskae}\headword{Piskae}{\pos{Proper noun}} {\definition{Piskae (toponym)}}
\entry{Podare}\headword{Podare}{\pos{Proper noun}} {\definition{Podare (Wipi-speaking village in Oriomo-Bituri Rural LLG)}}
\entry{Pondollowang}\headword{Pondollowang}{\pos{Proper noun}} {\definition{Pondollowang (camping place)}}
\entry{Pongarke}\headword{Pongarke}{\pos{Proper noun}} {\definition{Pongariki (Nambo-speaking village in Morehead Rural LLG)}}
\entry{Ponongllowang}\headword{Ponongllowang}{\pos{Proper noun}} {\definition{Ponongllowang (toponym)}}
\entry{Pottängäm}\headword{Pottängäm}{\pos{Proper noun}} {\definition{Pottängäm (camp and garden place)}}
\entry{Pällmang}\headword{Pällmang}{\pos{Proper noun}} {\definition{Pällmang (toponym)}}
\entry{Pätta}\headword{Pätta}{\pos{Proper noun}} {\definition{Pätta (toponym)}}
\entry{Raroge}\headword{Raroge}{\pos{Proper noun}} {\definition{Raroge (toponym)}}
\entry{Rual}\headword{Rual}{\pos{Proper noun}} {\definition{Rual (toponym)}}
\entry{Sakoyaratt}\headword{Sakoyaratt}{\pos{Proper noun}} {\definition{Sakoyaratt (toponym)}}
\entry{Samari}\headword{Samari}{\pos{Proper noun}} {\definition{Samari (toponym)}}
\entry{Saweta}\headword{Saweta}{\pos{Proper noun}} {\definition{Saweta (toponym)}}
\entry{Sebe}\headword{Sebe}{\pos{Proper noun}} {\definition{Sebe (Bine-speaking village in Oriomo-Bituri Rural LLG)}}
\entry{Sibideri}\headword{Sibideri}{\pos{Proper noun}} {\definition{Sibidiri (Idi-speaking village in Morehead Rural LLG)}}
\entry{Sibne}\headword{Sibne}{\pos{Proper noun}} {\definition{Sibne (toponym)}}
\entry{Sigabaduru}\headword{Sigabaduru}{\pos{Proper noun}} {\definition{Sigabaduru (in Kiwai Rural LLG)}}
\entry{Sirmitang}\headword{Sirmitang}{\pos{Proper noun}} {\definition{Sirmitang (toponym)}}
\entry{Sogale}\headword{Sogale}{\pos{Proper noun}} {\definition{Sogale (Bine-speaking village in Oriomo-Bituri Rural LLG)}}
\entry{Suame}\headword{Suame}{\pos{Proper noun}} {\definition{Suame (toponym)}}
\entry{Suki}\headword{Suki}{\pos{Proper noun}} {\definition{Suki (in Morehead Rural LLG)}}
\entry{Suwi}\headword{Suwi}{\pos{Proper noun}} {\definition{Sui (in Kiwai Rural LLG)}}
\entry{Taolang}\headword{Taolang}{\pos{Proper noun}} {\definition{Taolang (camping place)}}
\entry{Tapila}\headword{Tapila}{\pos{Proper noun}} {\definition{Tapila (Makayam-speaking village in Gogodala Rural LLG; GPS: -8.414202, 143.016867)}}
\entry{Tapma}\headword{Tapma}{\pos{Proper noun}} {\definition{Tapma (toponym)}}
\entry{Tawabo}\headword{Tawabo}{\pos{Proper noun}} {\definition{Tawabo (toponym)}}
\entry{Tawemitang}\headword{Tawemitang}{\pos{Proper noun}} {\definition{Tawemitang (toponym)}}
\entry{Tayi}\headword{Tayi}{\pos{Proper noun}} {\definition{Tai (in Gogodala Rural LLG)}}
\entry{Tewara}\headword{Tewara}{\pos{Proper noun}} {\definition{Tewara (Bitur-speaking village in Oriomo-Bitur Rural LLG)}}
\entry{Teyapopo}\headword{Teyapopo}{\pos{Proper noun}} {\definition{Teyapopo (toponym)}}
\entry{Tirere}\headword{Tirere}{\pos{Proper noun}} {\definition{Tirere/Tire'ere (Waboda-speaking village in Kiwai Rural LLG)}}
\entry{Titi}\headword{Titi}{\pos{Proper noun}} {\definition{Titi (toponym)}}
\entry{Togowa}\headword{Togowa}{\pos{Proper noun}} {\definition{Togowa (toponym)}}
\entry{Torok mittang}\headword{Torok mittang}{\pos{Proper noun}} {\definition{Torok mittang (toponym)}}
\entry{Ttäbe Ttäbe}\headword{Ttäbe Ttäbe}{\pos{Proper noun}} {\definition{Ttäbe Ttäbe (toponym)}}
\entry{Ttägällag kona}\headword{Ttägällag kona}{\pos{Proper noun}} {\definition{Ttägälläg corner}}
\entry{Ttägällag pollon}\headword{Ttägällag pollon}{\pos{Proper noun}} {\definition{Ttägällag pollon (toponym)}}
\entry{Ttäle Bun}\headword{Ttäle Bun}{\pos{Proper noun}} {\definition{Ttäle Bun (toponym)}}
\entry{Ttäle mitt}\headword{Ttäle mitt}{\pos{Proper noun}} {\definition{place with a well in Limol}}
\entry{Ttälebun}\headword{Ttälebun}{\pos{Proper noun}} {\definition{Ttalebun (toponym)}}
\entry{Tungnu}\headword{Tungnu}{\pos{Proper noun}} {\definition{Tungnu (toponym)}}
\entry{Upiara}\headword{Upiara}{\pos{Proper noun}} {\definition{Upiara (Bitur-speaking village in Oriomo-Bituri Rural LLG; GPS: -8.547170, 142.653008)}}
\entry{Wadär Mitang}\headword{Wadär Mitang}{\pos{Proper noun}} {\definition{Wadär Mitang (toponym)}}
\entry{Wagälla}\headword{Wagälla}{\pos{Proper noun}} {\definition{Wagälla (toponym)}}
\entry{Waka Källäm}\headword{Waka Källäm}{\pos{Proper noun}} {\definition{Waka Pond (in Limol)}}
\entry{Waliyama}\headword{Waliyama}{\pos{Proper noun}} {\definition{Wariama (in Gogodala Rural LLG)}}
\entry{Wamorong}\headword{Wamorong}{\pos{Proper noun}} {\definition{Wamorong (toponym)}}
\entry{Wapotea}\headword{Wapotea}{\pos{Proper noun}} {\definition{Wapotea (toponym)}}
\entry{Wara}\headword{Wara}{\pos{Proper noun}} {\definition{Wara (toponym)}}
\entry{Wariabodolo}\headword{Wariabodolo}{\pos{Proper noun}} {\definition{Variobadoro (in Kiwai Rural LLG)}}
\entry{Warola}\headword{Warola}{\pos{Proper noun}} {\definition{Warola (camping place in Limol)}}
\entry{Wasuwa}\headword{Wasuwa}{\pos{Proper noun}} {\definition{Wasua (in Gogodala Rural LLG)}}
\entry{Wedereamo}\headword{Wedereamo}{\pos{Proper noun}} {\definition{Wederehiamo (on the south side of the mouth of the Fly River)}}
\entry{Widama}\headword{Widama}{\pos{Proper noun}} {\definition{Widama (toponym)}}
\entry{Wim}\headword{Wim}{\pos{Proper noun}} {\definition{Wim (Kawam-speaking village in Oriomo-Bituri Rural LLG; GPS: 8.762144, 142.770164)}}
\entry{Winy}\headword{Winy}{\pos{Proper noun}} {\definition{Winy (toponym)}}
\entry{Wipim}\headword{Wipim}{\pos{Proper noun}} {\definition{Wipim (Wipi- and Kawam-speaking village in Oriomo-Bituri Rural LLG; GPS: -8.786616, 142.872201)}}
\entry{Wonie}\headword{Wonie}{\pos{Proper noun}} {\definition{Wonie (Wipi-speaking village in Oriomo-Bituri Rural LLG; GPS: -8.838939, 142.975007)}}
\entry{Wot}\headword{Wot}{\pos{Proper noun}} {\definition{Wot (toponym)}}
\entry{Wur}\headword{Wur}{\pos{Proper noun}} {\definition{place name}}
\entry{Wurlaimäll}\headword{Wurlaimäll}{\pos{Proper noun}} {\definition{Wurlaimäll (toponym)}}
\entry{Wätt bo ma}\headword{Wätt bo ma}{\pos{Proper noun}} {\definition{Wätt bo ma (toponym)}}
\entry{Yam}\headword{Yam}{\pos{Proper noun}} {\definition{Yam (toponym)}}
\entry{Yamayama}\headword{Yamayama}{\pos{Proper noun}} {\definition{Yamayama (toponym)}}
\entry{Yamega}\headword{Yamega}{\pos{Proper noun}} {\definition{Iamega (Wipi-speaking village in Oriomo-Bituri Rural LLG)}}
\entry{Yarte}\headword{Yarte}{\pos{Proper noun}} {\definition{Yarte (camping place)}}
\entry{Yokas}\headword{Yokas}{\pos{Proper noun}} {\definition{Yokas (camping place in Limol)}}
\entry{Yoteang}\headword{Yoteang}{\pos{Proper noun}} {\definition{Yoteang (toponym)}}
\entry{Yow}\headword{Yow}{\pos{Proper noun}} {\definition{Yau (in Gogodala Rural LLG)}}
\entry{Zanor}\headword{Zanor}{\pos{Proper noun}} {\definition{Zanor (in Gogodala Rural LLG; GPS: -8.459817, 142.688025)}}
\entry{Zarma}\headword{Zarma}{\pos{Proper noun}} {\definition{Zarma (toponym)}}
\entry{Zegma}\headword{Zegma}{\pos{Proper noun}} {\definition{Zegma (camping place)}}
\entry{Zeriko}\headword{Zeriko}{\pos{Proper noun}} {\definition{Jericho}}
\entry{Zudiya}\headword{Zudiya}{\pos{Proper noun}} {\definition{Judea}}
\entry{Zurusalem}\headword{Zurusalem}{\pos{Proper noun}} {\definition{Jerusalem}}
\entry{bobngätt}\headword{bobngätt}{\pos{Noun}} {\definition{place name}}
\end{entrylist}

\section*{9.7.2.1 Names of countries}
\begin{entrylist}
\entry{Australia}\headword{Australia}{\pos{Proper noun}} {\definition{Australia}}
\entry{Indonesia}\headword{Indonesia}{\pos{Proper noun}} {\definition{Indonesia}}
\entry{Koreya}\headword{Koreya}{\pos{Proper noun}} {\definition{Korea}}
\entry{Papua Niugini}\headword{Papua Niugini}{\pos{Proper noun}} {\definition{Papua New Guinea}}
\end{entrylist}

\section*{9.7.2.3 Names of cities}
\begin{entrylist}
\entry{Pot Mosbi}\headword{Pot Mosbi}{\pos{Proper noun}} {\definition{Port Moresby (the capital city of Papua New Guinea)}}
\entry{Rom}\headword{Rom}{\pos{Proper noun}} {\definition{Rome}}
\end{entrylist}

\section*{9.7.2.7 Names of mountains}
\begin{entrylist}
\entry{Dowan}\headword{Dowan}{\pos{Proper noun}} {\definition{name of a mountain}}
\end{entrylist}

\section*{9.7.2.9 Names of rivers}
\begin{entrylist}
\entry{Pawaturi}\headword{Pawaturi}{\pos{Proper noun}} {\definition{Pahoturi River}}
\end{entrylist}

\section*{1 Universe, creation}
\begin{entrylist}
\entry{ekaklle}\headword{ekaklle}{\pos{Noun}} {\definition{Earth}}
\entry{enma}\headword{enma}{\pos{Noun}} {\definition{nature}}
\end{entrylist}

\section*{2 Person}
\begin{entrylist}
\entry{lla}\headword{lla}{\pos{Noun}} {\definition{person, human being}}
\end{entrylist}

\section*{7 Physical actions}
\begin{entrylist}
\entry{anggwuanemeny}\headword{anggwuanemeny}{\pos{}} {\definition{cook}}
\entry{domoe}\headword{domoe}{\pos{Transitive verb}} {\definition{to push}}
\entry{däbäll}\headword{däbäll}{\pos{Transitive S verb}} {\definition{to touch}}
\entry{ergod}\headword{ergod}{\pos{Intransitive S verb}} {\definition{to crawl}}
\entry{gäddgädd}\headword{gäddgädd}{\pos{Transitive S verb}} {\definition{to fight}}
\entry{ingellab}\headword{ingellab}{\pos{Verb}} {\definition{to rise up, said of an animal}}
\entry{ittall}\headword{ittall}{\pos{Verb}} {\definition{to hang}}
\entry{matta}\headword{matta}{\pos{Transitive A verb}} {\definition{to shoulder, carry on one's shoulder}}
\entry{ngänaeka}\headword{ngänaeka}{\pos{Transitive A verb}} {\definition{to cry (about, for)}}
\entry{okok}\headword{okok}{\pos{Transitive S verb}} {\definition{to rub}}
\entry{okookol}\headword{okookol}{\pos{Noun}} {\definition{welcome}}
\entry{sagol nyanyu}\headword{sagol nyanyu}{\pos{Verb}} {\definition{to learn the steps}}
\entry{waowaem}\headword{waowaem}{\pos{Intransitive A verb}} {\definition{to flow}}
\entry{wälle}\headword{wälle}{\pos{Verb}} {\definition{to call}}
\entry{yayo}\headword{yayo}{\pos{Verb}} {\definition{to slither}}
\end{entrylist}

