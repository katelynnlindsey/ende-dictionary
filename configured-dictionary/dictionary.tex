\documentclass{book}
\usepackage[utf8]{inputenc}
\usepackage[T1]{fontenc}
\usepackage{charis}
\usepackage{dictionary}
\title{Agob-English Dictionary}
\author{}
\renewcommand{\cftchapfont}{\bfseries}
\renewcommand{\cftdot}{}
\begin{document}
\maketitle
\tableofcontents
\newpage
\lettersection{A}
\entry{a}{\headword{a}\pos{coord.}\definition{and}\example{Eso a mer toto.}{Thank you and good evening.}\allomorph{wa}}
\entry{=a}{\headword{=a}\pos{disc. ptcl.}\definition{vocative particle}\allomorph{ya}}
\entry{aba}{\headword{aba}\pos{interj.}\definition{go (command given to animals)}\etymology{clipping ofabäll}}
\entry{abade}{\headword{abade}\pos{mod.}\definition{future, later, upcoming, impending}\example{Yesu obo abade kuddäll e a kame ttam e gomballägän.}{Jesus predicted his impending death and resurrection.}}
\entry{Abagigima}{\headword{Abagigima}\pos{pn.}\definition{personal name}}
\entry{abal}{\headword{abal}\pos{adv.}\sensenumber{1}\definition{very}\example{mer abal nag}{very good friends}\sensenumber{2}\definition{exactly, just}\example{sisri abal}{right now}\allomorph{wabal}}
\entry{aball}{\headword{aball}\variant{dial. var. of}{abal}}
\entry{Abam}{\headword{Abam}\pos{pn.}\sensenumber{2}\definition{Abam (Wipi-speaking village in Oriomo-Bituri LLG; GPS: -8.926607 143.190246)}}
\entry{abe}{\headword{abe}\pos{coord.}\sensenumber{2}\definition{but}\etymology{a + be}}
\entry{Abeam}{\headword{Abeam}\pos{pn.}\sensenumber{2}\definition{male personal name}}
\entry{Abere}{\headword{Abere}\pos{pn.}\sensenumber{2}\definition{female personal name}}
\entry{Aberegerem}{\headword{Aberegerem}\pos{pn.}\sensenumber{2}\definition{Aberagerema (in Kiwai Rural LLG)}}
\entry{Abigail}{\headword{Abigail}\pos{pn.}\sensenumber{2}\definition{female personal name}}
\entry{abo}{\headword{abo}\pos{adv.}\sensenumber{1}\definition{then, afterwards}\example{Ngäna abo gongos ma we.}{I then returned home.}\sensenumber{2}\definition{must, necessative mood}\example{Abo bongo ttongo ma sisor de nogo.}{You must build a new house.}}
\entry{Abom}{\headword{Abom}\pos{pn.}\sensenumber{2}\definition{Abom (toponym)}}
\entry{abor}{\headword{abor}\pos{n.}\sensenumber{2}\definition{sago beaterSana tɨnen ma da. (For beating sago.)}\example{Ngäna sana pällkapällke abor alle nuebnegan.}{I beat the little sago pieces with the sago beater.}}
\entry{abwa}{\headword{abwa}\variant{var. of}{aba}}
\entry{ada}{\headword{ada}\pos{adv.}\sensenumber{1}\definition{like this, thus, so}\example{Ada daeya.}{So it was.}\example{Obo mälläng ik a ada ulleulle dageya.}{His nostrils were this big.}\sensenumber{2}\definition{quotative particle (for quoted/direct speech)}\example{Ngänawa "Ngämlle gulag e" ada däga.}{I said, "Come with me."}\example{Bogo ada, "Ngämo pätt a ttattllong agan."}{He said, "My body has already gotten sore."}\sensenumber{3}\definition{that}\example{Ngämo umllang dan ada bongo Grace bo nag dan.}{I know that you are Grace's friend.}\sensenumber{3}\definition{\textbackslashtextasciitilde ada}\sensenumber{1}\definition{\textbackslashtextasciitilde ada}\sensenumber{2}\definition{\textbackslashtextasciitilde ako}\example{Adako bongo mɨnyi nalle ma we.}{Agree that you will go to the house.}\sensenumber{1}\definition{this is why, therefore}\example{Adame ddobag a gudae ngällbaenen eran yunu atta bäne pate llɨtaem eran.}{This is why some wake up early in the morning from sleep and come to you.}\sensenumber{2}\definition{at this moment}\example{Angde Yesu eka gognän, Zudas adame gongttägän lla gul aba peyang.}{Right when Jesus was speaking, Judas arrived with a throng.}\sensenumber{2}\definition{so that}\example{Ngäna Ende eka de tameny anggan adawede ubi ngämo kokok erag Ende eka de umllang bognegnän ade.}{I teach Ende so that my grandchildren will also learn to speak Ende.}\sensenumber{2}\definition{like this}\example{Ende eka de ddob ngämo patme ttättle amallo ada, "Adingoll näpany."}{They correct some of my Ende words, saying, "Say it like this."}\sensenumber{2}\definition{the same way}\example{Ngäna däbe täl de deyanykoe. Adawae gog a dädme mäse ada deyanykoe.}{I pulled that bamboo. I kept doing the same, trying to pull so.}\sensenumber{2}\definition{then, so}\example{Ddia da mängalae abal gurlewän iba ttängäm me. Adawalle de iba ttängäm a ddia peyang gogän.}{The deer reproduced very quickly in our (incl.) village. So our village came to be populated with deer.}\sensenumber{1}\definition{because (introduces a cause or reason)}\example{Diba toto we ngäma pemli aba peyang ngämi mer duwem de dägaeya, adawatta ddia da däbe ddobae gi daeya.}{That evening, we (excl.) ate well in our family because that deer was very fatty.}\example{Mamal gänyeriballe yaupe adawatta da mo da bäne peyang bottkamän.}{Jump quickly to this side: otherwise, the bridge might break with you.}\sensenumber{2}\definition{why, therefore (marks a result or consequence)}\example{Bodog ada ka Satade dan. Bogo adawatta bel de ekameny nägagan.}{Bodog thought it was Saturday. That's why he didn't ring the bell.}\etymology{ada + =ade}\subentry{\headword{adade}\pos{adv.}\definition{\textbackslashtextasciitilde ada}}\subentry{\headword{adako}\pos{adv.}\definition{\textbackslashtextasciitilde ada}}\subentry{\headword{adame}\pos{adv.}\definition{this is why, therefore}}\subentry{\headword{adawede}\pos{subr.}\definition{so that}}\subentry{\headword{adingoll}\pos{adv.}\definition{like this}}\subentry{\headword{adawae}\pos{adv.}\definition{the same way}}\subentry{\headword{adawalle}\pos{adv.}\definition{then, so}}\subentry{\headword{adawatta}\pos{subr.}\definition{because (introduces a cause or reason)}}}
\entry{Adam}{\headword{Adam}\pos{pn.}\sensenumber{2}\definition{PN}}
\entry{Adasha}{\headword{Adasha}\pos{pn.}\sensenumber{2}\definition{male personal name}}
\entry{adatt}{\headword{adatt}\variant{fast speech var. of}{adawatta}}
\entry{ade}{\headword{ade}\pos{adv.}\sensenumber{2}\definition{also}\example{Ngäna Ingglis ekalle eka allan, a iba eka walle ade täräp me.}{I speak in English and also in our (excl.) language sometimes.}\allomorph{wade}}
\entry{Adi}{\headword{Adi}\pos{pn.}\sensenumber{2}\definition{God}\example{Bongo era Adi bo llɨg dan.}{You are the son of God.}}
\entry{adräl}{\headword{adräl}\pos{n.}\sensenumber{2}\definition{type of tree}}
\entry{Adu}{\headword{Adu}\pos{pn.}\sensenumber{2}\definition{female personal name}}
\entry{aduwi}{\headword{aduwi}\pos{n.}\sensenumber{2}\definition{type of tree}}
\entry{ae}{\headword{ae}\pos{interj.}\sensenumber{2}\definition{ah}}
\entry{=ae1}{\headword{=ae1}\pos{cl.}\sensenumber{2}\definition{restrictive clitic; only}\example{Ngäna Upiara me ae skul gogne.}{I was schooling in Upiara only.}\allomorph{wae}}
\entry{=ae2}{\headword{=ae2}\pos{cl.}\sensenumber{2}\definition{adverbializing clitic}\example{Ngäna dinduangmeae gogne llättmenyae.}{I was running nonstop.}\allomorph{yae}}
\entry{=ae3}{\headword{=ae3}\pos{n. cl.}\sensenumber{2}\definition{distal vocative clitic}\example{Llamda ae!}{Hey old man!}}
\entry{aeb1}{\headword{aeb1}\pos{n.}\sensenumber{2}\definition{black-billed/yellow-legged brushturkeyWälläng me giddollag pa dan, tot umaem alle ma ulle mɨnyi bogowän. (It's a bird that lives in the bush; by gathering trash it will build a big nest.)}}
\entry{aebaeb}{\headword{aebaeb}\pos{n.}\sensenumber{2}\definition{type of tree}}
\entry{=aebe}{\headword{=aebe}\pos{n. cl.}\sensenumber{2}\definition{restrictive clitic; only}\example{Kollba da tämamae dadrowän a be kottllam aebe ttam dagirnän.}{The fish all died and only the turtle lived on.}\allomorph{waebe}\allomorph{aeb}}
\entry{aengap}{\headword{aengap}\pos{n.}\sensenumber{2}\definition{type of big yam with a white interior and no thorns}}
\entry{aetruaetru}{\headword{aetruaetru}\pos{n.}\sensenumber{2}\definition{type of tree with edible blue fruits and white flowers, found in the bush and along big creeks}}
\entry{aeya}{\headword{aeya}\variant{sp. var. of}{aya}}
\entry{ag}{\headword{ag}\pos{n.}\sensenumber{2}\definition{morning (approx. 5 AM–11 AM)}\example{Ag me inu da era ddobae llokttang dan.}{Sleeping in the morning is really difficult.}\sensenumber{2}\definition{breakfast}\subentry{\headword{ag duwem}\pos{n.}\definition{breakfast}}}
\entry{Agan}{\headword{Agan}\pos{pn.}\sensenumber{2}\definition{Agan (toponym)}}
\entry{Agäb}{\headword{Agäb}\pos{pn.}\sensenumber{2}\definition{Agob language (Pahoturi River language)}}
\entry{agde}{\headword{agde}\variant{fr. var. of}{Agob loanword}}
\entry{agdea}{\headword{agdea}\variant{fr. var. of}{Agob loanword}}
\entry{Agob}{\headword{Agob}\variant{var. of}{Agäb}}
\entry{Agob loanword}{\headword{Agob loanword}\sensenumber{2}\definition{loanword}}
\entry{ai1}{\headword{ai1}\pos{mod.}\sensenumber{1}\definition{good}\example{ai abal ttängäm}{a very good place}\sensenumber{2}\definition{plenty, abundant}\example{ai kollba, ai wayati}{plenty of fish, lots of watermelons}\sensenumber{3}\definition{modal adjective (expresses permission, possibility, obligation)}\example{Ngäna ai dan gänyme bina peyang bodämen?}{May I sit here with you (pl.)?}\example{Wel a era za gagällang dan a ai dan lla de ade bäbäddän.}{The wind is a bad thing that can also kill people.}\example{Bongo ai da ine de näna.}{You should drink water.}\sensenumber{4}\definition{very}\example{ai mängallang mälla}{a very strong woman}\sensenumber{4}\definition{alright, okay}\example{Piasorosoro ada eka gogon, "Ibi goeg tatäräp e." Mälla da ada gogon, "Aida, ibi beyareya."}{Piasorosoro said, "Let's go to the garden to cut down trees." His wife said, "Okay, let's go."}\sensenumber{4}\definition{properly, well}\example{Eka de ddob malla aiai panynen amallo.}{Some don't speak the language properly.}\etymology{ai + dan}\subentry{\headword{aidan}\pos{interj.}\definition{alright, okay}}\subentry{\headword{aiai}\pos{adv.}\definition{properly, well}}}
\entry{ai2}{\headword{ai2}\pos{interj.}\sensenumber{1}\definition{ah}\sensenumber{2}\definition{hey}}
\entry{ai skul}{\headword{ai skul}\pos{n.}\sensenumber{2}\definition{secondary school, high school}\example{Ngäna ai skul i ddone dall.}{I didn't go to secondary school.}\etymology{from Englishhigh school}}
\entry{aida}{\headword{aida}\variant{fr. var. of}{aidan}}
\entry{aii}{\headword{aii}\variant{sp. var. of}{ai1}}
\entry{Ailin}{\headword{Ailin}\pos{pn.}\sensenumber{2}\definition{female personal name}}
\entry{Ainduru}{\headword{Ainduru}\variant{sp. var. of}{Andrew}}
\entry{ainin}{\headword{ainin}\variant{sp. var. of}{ainen}}
\entry{Ainor}{\headword{Ainor}\pos{pn.}\sensenumber{1}\definition{Ainor (toponym)}\sensenumber{2}\definition{dog name}}
\entry{aitar}{\headword{aitar}\pos{n.}\sensenumber{2}\definition{type of palm tree that grows along creeks with pools; shoots and soft trunk are edible; similar to dumar}}
\entry{Aituru}{\headword{Aituru}\pos{pn.}\sensenumber{2}\definition{female personal name}}
\entry{aiyai}{\headword{aiyai}\variant{sp. var. of}{aiai}}
\entry{Aketa}{\headword{Aketa}\pos{pn.}\sensenumber{2}\definition{Aketa (in Gogodala Rural LLG)}}
\entry{ako}{\headword{ako}\pos{adv.}\sensenumber{1}\definition{also, too}\example{Mägda män de kaptte namättan. Mägda ako kaptte amättan.}{Mother dressed her daughter. Mother also got dressed.}\sensenumber{2}\definition{again}\example{Täre imnealle, ttongo pazi ddägatt auma ma ddaddällɨg e lla da täre de ako mɨnyi dägagallo.}{One year after the feast and destroying the grave, people will have a feast again.}\sensenumber{3}\definition{then}\example{Ge lla da bandra da nällɨtan, ako pomer alle nällɨtan.}{This man sang a song and then whistled it.}\sensenumber{4}\definition{other}\example{ako ttongo llo}{another tree}\example{Ddobagabira gae kaekep a ako llowamang dan.}{Some people chew gae but others don't like to.}}
\entry{Al}{\headword{Al}\pos{pn.}\sensenumber{4}\definition{female personal name}}
\entry{aläm}{\headword{aläm}\pos{A vi.}\sensenumber{1}\definition{to be wary}\example{Abo bongo Daru me ai aläm agne.}{You must be wary in Daru.}\sensenumber{2}\definition{to warn, caution}\example{Yesu ubim ada aläm dägnegän sanawang eka walle.}{Jesus warned them with a parable.}}
\entry{Aleks}{\headword{Aleks}\variant{sp. var. of}{Alex}}
\entry{Alex}{\headword{Alex}\pos{pn.}\sensenumber{2}\definition{male personal name}}
\entry{Ali}{\headword{Ali}\pos{pn.}\sensenumber{2}\definition{place name}}
\entry{ali}{\headword{ali}\pos{n.}\sensenumber{2}\definition{conch shell}}
\entry{Alopa}{\headword{Alopa}\pos{pn.}\sensenumber{2}\definition{female personal name}}
\entry{Alofa}{\headword{Alofa}\variant{sp. var. of}{Alopa}}
\entry{Alphones}{\headword{Alphones}\pos{pn.}\sensenumber{2}\definition{male personal name}}
\entry{Alpons}{\headword{Alpons}\variant{sp. var. of}{Alphones}}
\entry{alla}{\headword{alla}\pos{adv.}\sensenumber{2}\definition{how, what}\example{Alla ulle ine dan?}{How much water is there?}\example{Ge lla da obo lläg komlla de alla tameny yaran.}{This man is teaching his two kids how.}\example{Ibi alla täräp me ibi ge nge de ibeny e dan?}{What time are we (incl.) planting this coconut?}\sensenumber{2}\definition{how}\example{Ngäna bam alla ingollang bangmingg?}{How can I help you?}\example{Ge alla ingollang ngäma lla da gurlewän.}{This is how our (excl.) population grew.}\subentry{\headword{alla ingollang}\pos{adv.}\definition{how}}}
\entry{Allambun}{\headword{Allambun}\pos{pn.}\sensenumber{2}\definition{Allambun (camping place)}}
\entry{allap}{\headword{allap}\variant{dial. var. of}{alläp}}
\entry{alläp}{\headword{alläp}\pos{n.}\sensenumber{1}\definition{kundu drumDäba wa biwäz llo watt täratäre att dan a ttang pallta walle papa ma dan. (Made from the däba and biwäz tree and to hit with your palm.)}\sensenumber{2}\definition{guitar}}
\entry{Alläpma}{\headword{Alläpma}\pos{pn.}\sensenumber{2}\definition{Alläpma (toponym)}}
\entry{=alle1}{\headword{=alle1}\pos{n. cl.}\sensenumber{1}\definition{instrumental case clitic; with (using)}\example{Biye de bongo ai dan ibik alle beni.}{You can plant taro with the ibik stick.}\sensenumber{2}\definition{comitative case clitic; with (together)}\example{Nensi bi nane alle känaebag mɨnyi tudi ma beyareyo.}{Nancy will go fishing tomorrow with her aunt.}\allomorph{walle}\allomorph{alle}}
\entry{=alle2}{\headword{=alle2}\pos{n. cl.}\sensenumber{2}\definition{ablative case clitic; from}\example{Ag alle do yäbäd klaklema.}{From morning to sunset.}\allomorph{walle}\allomorph{alle}\allomorph{olle}\allomorph{balle}}
\entry{allingoll}{\headword{allingoll}\variant{fast speech var. of}{alla ingollang}}
\entry{allingollang}{\headword{allingollang}\variant{fast speech var. of}{alla ingollang}}
\entry{allka}{\headword{allka}\pos{A vi. \textbackslash& vt.}\sensenumber{2}\definition{to shout (at)}\example{Ge lla da allka allan, obom allka eran.}{This man is shouting; he's shouting at her.}}
\entry{allko}{\headword{allko}\pos{n.}\sensenumber{2}\definition{fly (insect)}\example{Allko da yuwog abal dag.}{There are many flies.}}
\entry{allko kukut}{\headword{allko kukut}\pos{n.}\sensenumber{2}\definition{blue fly}}
\entry{allko wallägnewallägnen}{\headword{allko wallägnewallägnen}\pos{n.}\sensenumber{2}\definition{type of sagoLlayaba ttokoe ma sana dan ge, ubiae mɨnyi bäkämeyo. (This is sago for men [i.e. not women] to chop and squeeze.)}}
\entry{am}{\headword{am}\pos{n.}\sensenumber{2}\definition{internode (section of bamboo or sugarcane, separated by nodes)}\example{Angde ttongo am pättkäp de nätäräpän.}{After one internode, cut the node.}}
\entry{ama2}{\headword{ama2}\pos{n.}\sensenumber{2}\definition{hammer}\etymology{from Englishhammer}}
\entry{Amadu}{\headword{Amadu}\pos{pn.}\sensenumber{2}\definition{male personal name}}
\entry{amamär}{\headword{amamär}\pos{n.}\sensenumber{2}\definition{woven ropeYu da a za patkoll ma dan. (It's for bundling wood and things.)}}
\entry{amamärang}{\headword{amamärang}\pos{A vi.}\sensenumber{2}\definition{to echo}\example{Ngämo eka da amamärang agan.}{My voice echoed.}}
\entry{Amanda}{\headword{Amanda}\pos{pn.}\sensenumber{2}\definition{female personal name}}
\entry{amäramär}{\headword{amäramär}\pos{n.}\sensenumber{2}\definition{braid}}
\entry{amba}{\headword{amba}\pos{n.}\sensenumber{2}\definition{type of tree found in the grassland}}
\entry{ambag}{\headword{ambag}\pos{A vi.}\sensenumber{2}\definition{to cause trouble}\example{Bogo mɨnyi ambag bogon.}{He will cause trouble.}}
\entry{Amerika}{\headword{Amerika}\pos{pn.}\sensenumber{2}\definition{United States, America}}
\entry{ami}{\headword{ami}\pos{int. pron.}\sensenumber{1}\definition{who (nonsingular interrogative pronoun, nominative form)}\example{Ubi ami dag?}{Who are they?}\sensenumber{2}\definition{whoever, some people (nonsingular existential pronoun, nominative form)}\example{Da ami botngoenegnän, de ngäna mɨnyi any batngoenen.}{If some people are playing, I will play.}\sensenumber{3}\definition{who (nonsingular relative pronoun, nominative form)}\example{Komlla nag a dadegwaeya ami gudae deyagirneyo walle ddage menae me.}{There were two friends who once lived on the side of a creek.}\sensenumber{1}\definition{possessive form of ami}\sensenumber{2}\definition{possessive form of ami}\sensenumber{3}\definition{possessive form of ami}\example{Mer dan ada yuwog abal lla da wa mälla da gänyag, ama moko dan mällayabira ngämiminy i.}{It is good that there are so many men and women here who want (lit. whose desire it is) to help women.}\sensenumber{1}\definition{comitative of ami}\sensenumber{2}\definition{comitative form of ami}\sensenumber{3}\definition{comitative of ami}\example{Ge lla da ngäna ama bakmall skul att dan.}{These people are who I went to school with.}\sensenumber{1}\definition{comitative form of ami}\example{Bibi amakmall panypeny eralla?}{With whom do you all speak?}\sensenumber{2}\definition{comitative form of ami}\example{Ngämi amakämall skul att dag, gänyme ddone dag.}{Whomever we (excl.) went to school with, they're not here.}\sensenumber{3}\definition{comitative form of ami}\sensenumber{1}\definition{accusative form of ami}\sensenumber{2}\definition{accusative form of ami}\sensenumber{3}\definition{accusative form of ami}\example{kumuddäga pemli de ngäna amim nättaemnegan}{the three families whom I mentioned}\sensenumber{1}\definition{dative form of ami}\sensenumber{2}\definition{dative form of ami}\sensenumber{3}\definition{dative form of ami}\subentry{\headword{ama1}\pos{int. pron.}\definition{possessive form of ami}}\subentry{\headword{ama bakmall}\pos{int. pron.}\definition{comitative of ami}}\subentry{\headword{ama kämall}\pos{int. pron.}\definition{comitative form of ami}}\subentry{\headword{amim}\pos{int. pron.}\definition{accusative form of ami}}\subentry{\headword{amira}\pos{int. pron.}\definition{dative form of ami}}}
\entry{amig}{\headword{amig}\pos{cop.}\sensenumber{3}\definition{contraction of ami dag}}
\entry{amiyeamiye}{\headword{amiyeamiye}\pos{adv.}\sensenumber{3}\definition{upwind, against the wind}}
\entry{amne}{\headword{amne}\pos{loc.}\sensenumber{3}\definition{center, middle}\example{Ge ttoen a era bem amne me de gongesän.}{This story happened in the middle of the sea.}\example{Bogo ttängäm amne we gozenän.}{He went into the village center.}\example{Bogo sisri amne pazi me dan.}{She's middle-aged now.}}
\entry{Amne kona}{\headword{Amne kona}\pos{pn.}\sensenumber{3}\definition{Central corner (in Limol)}}
\entry{Amrika}{\headword{Amrika}\variant{fast speech var. of}{Amerika}}
\entry{amtet}{\headword{amtet}\pos{n.}\sensenumber{1}\definition{breath}\example{Amtet de nängänymällneg.}{[You] take (lit. swallow) a breath.}\example{Obo amtet a gottämänän.}{His breath slowed.}\sensenumber{2}\definition{to breathe}\example{Bogo amtet gognän.}{He was breathing.}\sensenumber{2}\definition{nonstop, without breathing}\etymology{amtet + =meny}\subentry{\headword{amtetmeny}\pos{adv.}\definition{nonstop, without breathing}}}
\entry{Ana}{\headword{Ana}\pos{pn.}\sensenumber{2}\definition{female personal name}}
\entry{andred1}{\headword{andred1}\pos{num.}\sensenumber{2}\definition{hundred}\etymology{from Englishhundred}}
\entry{andredd}{\headword{andredd}\variant{sp. var. of}{andred1}}
\entry{Andrew}{\headword{Andrew}\pos{pn.}\sensenumber{2}\definition{male personal name}}
\entry{=ane}{\headword{=ane}\variant{dial. var. of}{=alle1}}
\entry{ankom}{\headword{ankom}\pos{n.}\sensenumber{2}\definition{antDälag dan. Obo ziz a kängkäm ma dan nane we kumye itrell täräp me. Ako mälläng obäd täräp me ziz a mälläng dungg ma dan llo ttam peyang länenatt me. Ziz de ada ma dan, be mägda da, bällkäp a wa miny a ddägnan ma dag. (They are bitter. Insects are washed and drunk as medicine when you have a cough. Also, when you have a runny nose, you can smell them with crushed leaves. It's done this way with insects, but the mother and eggs are edible.)}}
\entry{Anna}{\headword{Anna}\pos{pn.}\sensenumber{2}\definition{female personal name}}
\entry{Ansel}{\headword{Ansel}\pos{pn.}\sensenumber{2}\definition{male personal name}}
\entry{ansi}{\headword{ansi}\pos{A vi.}\sensenumber{1}\definition{to sneeze}\example{Obo mälläng a nongonongorang gogon, käsre ansi gogon.}{His nose was itchy; then he sneezed.}\sensenumber{2}\definition{achoo}}
\entry{Anton}{\headword{Anton}\pos{pn.}\sensenumber{2}\definition{male personal name}}
\entry{anu}{\headword{anu}\pos{v.}\sensenumber{2}\definition{sleep (command given to babies)}}
\entry{anzag}{\headword{anzag}}
\entry{=ang}{\headword{=ang}\pos{cl.}\sensenumber{2}\definition{multipurpose attributive suffix that functions as an imperfective or resultative participle, agentive nominalizer, and generic adjective-forming suffix}\example{Ngäna dämenang (< dämen) dan.}{I am sitting.}\example{Ud a papekang (< papek) dan.}{The door is closed.}\example{Ge lla da mamoeang (< mamoe) dan.}{This man is a hunter.}\example{ttänttäm, ttänttämang}{heat, hot}\allomorph{wang}\allomorph{yang}\allomorph{ung}\allomorph{ong}\allomorph{ag}\allomorph{yag}\allomorph{wag}\allomorph{og}}
\entry{angäm}{\headword{angäm}\pos{adv.}\sensenumber{2}\definition{quickly}\example{Angäm eka muu nageyo!}{[You all] answer me quickly!}\example{Sisri sisor lla da angämae de pätärang anggan.}{Now, young people are quickly getting white hairs.}\allomorph{angm}}
\entry{angde}{\headword{angde}\pos{adv.}\sensenumber{2}\definition{when, while, as}\example{Angde ma we gongosalle, llɨg kälekäle da mɨnyi gumaemalle oblle ikop e.}{When she would return home, the children would gather to see her.}\example{Bam itrell a angde dangkamän?}{When did you get sick (lit. sickness start on you)?}\example{Dade angde wiaǃ}{Come wheneverǃ}}
\entry{anggog}{\headword{anggog}\pos{n.}\sensenumber{2}\definition{thigh}}
\entry{angkäpäll}{\headword{angkäpäll}\pos{n.}\sensenumber{2}\definition{type of big tree found in the bush with white flowers and fruit, which cassowary eat when they fall}}
\entry{any1}{\headword{any1}\pos{pron.}\sensenumber{1}\definition{something}\example{Obo any de särämbae erallo.}{They are preparing something.}\sensenumber{2}\definition{like this, thus, so}\example{Ause da gänyme ine ik i any gogon.}{The old woman went into this water like this.}}
\entry{anyke}{\headword{anyke}\pos{n.}\sensenumber{2}\definition{spirit}\example{Mer Anyke}{the Holy Spirit}\example{gagäll anyke}{demon, evil spirit}\example{Anykemeny agan.}{I got scared (lit. spiritless).}\sensenumber{2}\definition{to be in shock}\example{Angde Sowati obo bin di dänttamän, Bodog arle gogon a bogo anyke gomplezän.}{When Sowati called out his name, Bodog screamed and his spirit left his body.}\sensenumber{2}\definition{picture, image, reflection}\example{Ngämi up anykeanyke ikop nägagalla walle ik mi.}{We (excl.) saw the reflection of bananas in the water.}\subentry{\headword{anyke plenz}\pos{S vi.}\definition{to be in shock}}\subentry{\headword{anykeanyke}\pos{n.}\definition{picture, image, reflection}}}
\entry{anyki}{\headword{anyki}\variant{dial. var. of}{anyke}}
\entry{ao}{\headword{ao}\pos{interj.}\sensenumber{2}\definition{yes}\example{― Bongo dalle skul i? ― Ao, ngäna skul ibiatt dan.}{― Did you go to school? ― Yes, I went to school.}}
\entry{aoao}{\headword{aoao}\pos{A vt.}\sensenumber{2}\definition{to chase away (e.g. humans, dogs, chickens, but not wild animals)}\example{Llɨg de aoao nägaebeyo.}{[You all] chase away the children.}}
\entry{aoli}{\headword{aoli}\pos{int. pron.}\sensenumber{1}\definition{how many, how much}\example{Aoli dag up sana da?}{How much banana cake is there?}\example{Bäne pazi da aoli dan?}{How old are you (lit. how many are your years)?}\sensenumber{2}\definition{multiple, several (i.e. more than one but not necessarily many); few (esp. with the restrictive clitic)}\example{aoli eka kuttae}{only a few words}\example{Ngämo baba ttongdae ebdo me ai dan auli ddäddäg de bägäddnegän angde obo pollo me dagernän.}{My father could kill multiple animals in a single day when he was young.}\sensenumber{3}\definition{almost, nearly}\example{Lama wa Mana ubi kandärmang me gowensegäneyo adawatta Emi aoli kuddäll de ine ik mi ikop dägagän.}{Lama and Mana were both feeling sorry because they saw that Emi almost died in the water.}\sensenumber{3}\definition{few}\example{Adi ubira ttam e zäme ebdo de aolidae dägnegän.}{God has made their days to live few.}\etymology{aoli + =dae₁}\subentry{\headword{aolidae}\pos{quant.}\definition{few}}}
\entry{ap}{\headword{ap}\pos{n.}\sensenumber{3}\definition{grassland, savannah}\example{Llimoll a ap me da.}{Limol is in the savannah.}}
\entry{Apang}{\headword{Apang}\pos{pn.}\sensenumber{3}\definition{language name}}
\entry{apapi}{\headword{apapi}\pos{n.}\sensenumber{3}\definition{butterflyPa ngänäm ddamba peyang a llo popo nanenang. (With wings like a bird and drinking flowers.)}\example{Apapi da ddob erag käträkäträl dag.}{Some butterflies are multi-colored.}}
\entry{apapi bärät}{\headword{apapi bärät}\pos{n.}\sensenumber{3}\definition{type of yam}}
\entry{apapun}{\headword{apapun}\pos{n.}\sensenumber{3}\definition{type of short grass that disperses seeds by attaching to animal fur}}
\entry{Apdo}{\headword{Apdo}\pos{pn.}\sensenumber{3}\definition{female personal name}}
\entry{apgllu}{\headword{apgllu}\pos{n.}\sensenumber{3}\definition{small plant with a root that makes a maroon pigment for dyeing grass skirts when mixed with ash (e.g. Acacia ash or coconut ash). When not mixed with ash, it makes a yellow pigment.}}
\entry{Apodo}{\headword{Apodo}\pos{pn.}\sensenumber{3}\definition{female personal name}}
\entry{apte}{\headword{apte}\pos{quant.}\sensenumber{3}\definition{one side, one end, one half, one (of a pair)}\example{Gombällämenymom, apte we deyareyo a apte ade deyareyo.}{They split up: two went to one side and two went to the other side.}\example{Apte pud di bod me nowanseg eka we.}{Put one end (of the flute) in your mouth to play.}\example{Ngämo yae obo tubu da apte gagäll dan.}{One of my mother's knees is bad.}}
\entry{apte gabän}{\headword{apte gabän}\pos{num.}\sensenumber{3}\definition{fourteen (body counting numeral)}\etymology{lit. 'other wrist'}}
\entry{apte kllatolma}{\headword{apte kllatolma}\pos{num.}\sensenumber{3}\definition{seventeen (body counting numeral)}\etymology{lit. 'other middle finger'}}
\entry{apte matta}{\headword{apte matta}\pos{num.}\sensenumber{3}\definition{twelve (body counting numeral)}\etymology{lit. 'other shoulder'}}
\entry{apte mända}{\headword{apte mända}\pos{num.}\sensenumber{3}\definition{fifteen (body counting numeral)}\etymology{lit. 'other thumb'}}
\entry{apte mätkin}{\headword{apte mätkin}\pos{num.}\sensenumber{3}\definition{eighteen (body counting numeral)}\etymology{lit. 'other ring finger'}}
\entry{apte ngam}{\headword{apte ngam}\pos{num.}\sensenumber{3}\definition{eleven (body counting numeral)}\etymology{lit. 'other breast'}}
\entry{apte tärangesa}{\headword{apte tärangesa}\pos{num.}\sensenumber{3}\definition{nineteen (body counting numeral)}\etymology{lit. 'other pinky'}}
\entry{apte tupi}{\headword{apte tupi}\pos{num.}\sensenumber{3}\definition{sixteen (body counting numeral)}\etymology{lit. 'other index finger'}}
\entry{apte ttang kum}{\headword{apte ttang kum}\pos{num.}\sensenumber{3}\definition{thirteen (body counting numeral)}\etymology{lit. 'other elbow'}}
\entry{arabuni}{\headword{arabuni}\pos{n.}\sensenumber{3}\definition{red backed buttonquail}}
\entry{Arägapetkae}{\headword{Arägapetkae}\pos{pn.}\sensenumber{3}\definition{Arägapetkae (toponym)}}
\entry{arälle}{\headword{arälle}\variant{var. of}{arle}}
\entry{aräram}{\headword{aräram}\pos{A vt.}\sensenumber{3}\definition{to sand or smooth a canoe (last step before burning)}}
\entry{aräre}{\headword{aräre}\variant{var. of}{arle}}
\entry{arle}{\headword{arle}\pos{n.}\sensenumber{3}\definition{scream}\example{Däräng a ddone ada arlle gogän.}{The dog screamed really badly.}\example{Ttongo lla da lelang atta arle bogon.}{A person can scream from fear.}}
\entry{arlle}{\headword{arlle}\variant{dial. var. of}{arle}}
\entry{Arua}{\headword{Arua}\pos{pn.}\sensenumber{3}\definition{male personal name}}
\entry{arup}{\headword{arup}\pos{n.}\sensenumber{3}\definition{clothing type}}
\entry{Arupi}{\headword{Arupi}\pos{pn.}\sensenumber{3}\definition{Arupi (toponym)}}
\entry{Aruwa}{\headword{Aruwa}\pos{pn.}\sensenumber{3}\definition{male personal name}}
\entry{as}{\headword{as}\pos{n.}\sensenumber{3}\definition{type of introduced bananaUlle pättang dan, obo däg a yuwog dag. Obo käp a o me otät ma dan ako kire me yuma dan. (It has a big trunk; the bunches are plentiful. When ripe, its fruit is eaten, and when unripe, it's cooked.)}}
\entry{asa}{\headword{asa}\variant{fast speech var. of}{ngasekäma}}
\entry{asa bume}{\headword{asa bume}\pos{n.}\sensenumber{3}\definition{type of bird}}
\entry{asiasi}{\headword{asiasi}\pos{n.}\sensenumber{3}\definition{type of large tree that grows in the bush and along big creeks, with white flowers and big red fruits eaten by cassowary and children; the trunk is used to make canoes}}
\entry{Asika}{\headword{Asika}\pos{pn.}\sensenumber{3}\definition{female personal name}}
\entry{asip}{\headword{asip}\pos{n.}\sensenumber{3}\definition{type of introduced bananaUlle tanong dan. Obo käp a o me otät ma dan ako kire me yu ma dan. (It's a bit big. When ripe, its fruit is eaten, and when unripe, it's cooked.)}}
\entry{aska}{\headword{aska}\variant{fast speech var. of}{ngasekäma}}
\entry{atata kottllam}{\headword{atata kottllam}\pos{n.}\sensenumber{3}\definition{type of turtle}}
\entry{atrepo}{\headword{atrepo}\pos{n.}\sensenumber{3}\definition{taro type}}
\entry{=att}{\headword{=att}\pos{n. cl.}\sensenumber{1}\definition{ablative case clitic; from (used for location, time, source, or cause)}\example{Oba moko da up di bigezenallo walle ik att de.}{They want to take the bananas out from the water.}\example{Ge mätta da tätäm att dan.}{This yam is from yesterday.}\example{Ge ngäma masar att eka da.}{This is the language of our (excl.) ancestors.}\example{Nyongo da yäbäd ttänttäm att ibimeny bogon.}{The road can be unwalkable from the heat of the sun.}\sensenumber{2}\definition{past participle clitic}\example{Ud a papekatt dan.}{The door is closed.}\example{Ende eka walle dadäräbatt dan.}{It is written in Ende.}\example{Tabatt ttängäm a ngämenmeny dan.}{The promised land is not yet reached.}\example{Eka tameny att me, ubi guinttemänggeyo.}{Having discussed, the two of them parted ways.}\allomorph{watt}\allomorph{ott}\allomorph{matt}\allomorph{batt}}
\entry{atta}{\headword{atta}\variant{fast speech var. of}{adawatta}}
\entry{au}{\headword{au}\pos{n.}\sensenumber{1}\definition{burial}\example{Au ddägattalle lla da tämamae ma we dingismällalle.}{After the burial, all the people return home.}\sensenumber{2}\definition{to bury}\example{Kuddäll pätt de au dägagallo.}{They will bury the body.}\sensenumber{2}\definition{grave}\example{Mällause bo auma da dan de.}{The old woman's grave is there.}\etymology{au + ma}\subentry{\headword{auma}\pos{n.}\definition{grave}}}
\entry{aulämän}{\headword{aulämän}\pos{n.}\sensenumber{2}\definition{conception}}
\entry{auli}{\headword{auli}\variant{sp. var. of}{aoli}}
\entry{auri}{\headword{auri}\pos{n.}\sensenumber{2}\definition{metal}}
\entry{ause}{\headword{ause}\pos{n.}\sensenumber{1}\definition{old woman}\example{Ause da känyerkänyer däganän.}{The old woman soothed him.}\sensenumber{2}\definition{old (of a woman)}\example{Kak säre zäme ause abal allan.}{Grandmother is sadly very old already.}\sensenumber{1}\definition{nonsingular form of ause}\example{Ngäna ddob täräp me Ende eka de ddone mermer panypeny eran, auseause da ttättle amallo.}{Sometimes I don't speak Ende properly, so old women correct me.}\sensenumber{2}\definition{nonsingular form of ause}\example{Ngämi zime auseause gogmam.}{We women (excl.) have already become old.}\subentry{\headword{auseause}\pos{n.}\definition{nonsingular form of ause}}}
\entry{Australia}{\headword{Australia}\pos{pn.}\sensenumber{2}\definition{Australia}\etymology{from EnglishAustralia}}
\entry{auuma}{\headword{auuma}\variant{sp. var. of}{auma}}
\entry{Awaba}{\headword{Awaba}\pos{pn.}\sensenumber{2}\definition{Awaba (toponym)}}
\entry{Awayang}{\headword{Awayang}\pos{pn.}\sensenumber{2}\definition{male personal name}}
\entry{awäll}{\headword{awäll}\pos{n.}\sensenumber{2}\definition{type of tree that grows in the bush (where yam gardens are made) with white and blue flowers and dark, edible fruit}}
\entry{awe1}{\headword{awe1}\pos{n.}\sensenumber{2}\definition{savannah}}
\entry{awe2}{\headword{awe2}\pos{n.}\sensenumber{2}\definition{cassowary (used when hunting)}\etymology{probably from Idiawea}}
\entry{Awi}{\headword{Awi}\pos{pn.}\sensenumber{2}\definition{Awi (toponym)}}
\entry{awi}{\headword{awi}\pos{n.}\sensenumber{1}\definition{evening (approx. 5 PM till dark)}\example{awi alle do iddob amnong}{from evening to midnight}\sensenumber{2}\definition{early}\example{Ngäna awi meae ttängäm e dalle.}{I went to the garden early (e.g. 8 AM instead of 9 AM).}}
\entry{awoll}{\headword{awoll}\variant{var. of}{awäll}}
\entry{aya}{\headword{aya}\pos{int. pron.}\sensenumber{1}\definition{who (singular interrogative pronoun, nominative form)}\example{Kollba de aya nokowan?}{Who cut the fish?}\sensenumber{2}\definition{whoever, someone (singular existential pronoun, nominative form)}\example{Mulldae dan wadär de bängällbänän, wadär alle bäbäddän, a ge da aeya ambag bogon.}{If there's someone causing trouble, it is possible to get a cane and beat them with it.}\sensenumber{3}\definition{who (singular relative pronoun, nominative form)}\example{Da ngänäm aya bangkollmällän bogo mɨnyi ddone säremang dae ballän.}{He who follows me will not go through darkness.}\sensenumber{3}\definition{copular form of aya (present singular form)}\example{Bäne bin a ainen?}{What's your name?}\example{Ge Sam ainen.}{This is Sam.}\sensenumber{3}\definition{past singular form of aenen}\sensenumber{3}\definition{instrumental-comitative form of aya}\example{Bibi amalle erang gusiya?}{Whom did you do an exchange with?}\sensenumber{1}\definition{ablative-possessive form of aya}\example{Bongo amene män de dɨllɨd?}{Whose daughter did you marry?}\sensenumber{2}\definition{ablative-possessive form of aya}\sensenumber{3}\definition{ablative-possessive form of aya}\sensenumber{1}\definition{dative form of aya}\sensenumber{2}\definition{dative form of aya}\example{Abo mudan abal amlle llätät a!}{You must not tell anyone!}\sensenumber{3}\definition{dative form of aya}\sensenumber{1}\definition{possessive form of aya}\example{Amo sisor gall da däbe?}{Whose new canoe is that?}\sensenumber{2}\definition{possessive form of aya}\example{Amo moko da ulle abal lla we, ede ai dan bogo kälsre abal bogon.}{Whoever wants (lit. whoever's desire it is) to be a very important person, he should be very trivial.}\sensenumber{3}\definition{possessive form of aya}\example{ttongo mälla da amo män a gagäll anyke peyang daeya}{a woman whose daughter was possessed (lit. with bad spirit)}\sensenumber{1}\definition{accusative form of aya}\example{Bongo amom ikop nagagalle?}{Whom did you see?}\sensenumber{2}\definition{accusative form of aya}\example{Yu da amom dättämän, bogo ma we dallän.}{Whomever the fire burnt went home.}\sensenumber{3}\definition{accusative form of aya}\example{Lima bo umllang dan bongo mɨnyi amom känaebag ikop nägag.}{Lima knows whom you will see tomorrow.}\allomorph{aen}\allomorph{ain}\allomorph{waen}\allomorph{wain}\allomorph{waeya}\allomorph{am}\subentry{\headword{aenen}\pos{cop.}\definition{copular form of aya (present singular form)}}\subentry{\headword{aeneya}\pos{cop.}\definition{past singular form of aenen}}\subentry{\headword{amalle}\pos{int. pron.}\definition{instrumental-comitative form of aya}}\subentry{\headword{amene}\pos{int. pron.}\definition{ablative-possessive form of aya}}\subentry{\headword{amlle}\pos{int. pron.}\definition{dative form of aya}}\subentry{\headword{amo}\pos{int. pron.}\definition{possessive form of aya}}\subentry{\headword{amom}\pos{int. pron.}\definition{accusative form of aya}}}
\entry{Azaya}{\headword{Azaya}\pos{pn.}\sensenumber{3}\definition{male personal name}}
\lettersection{Ä ä}
\entry{ändred}{\headword{ändred}\variant{sp. var. of}{andred1}}
\lettersection{B}
\entry{ba}{\headword{ba}\variant{fast speech var. of}{aba}}
\entry{bab}{\headword{bab}\pos{n.}\sensenumber{3}\definition{type of small yam with a white interior}}
\entry{baba}{\headword{baba}\pos{kin.}\sensenumber{3}\definition{fatherZaze me bunlla ngattongang dan. (He's the head of the family.)}\example{Ngämo baba lla ulle deya.}{My father was a big man.}}
\entry{babaem}{\headword{babaem}\pos{n.}\sensenumber{3}\definition{season characterized by wind and going hunting (fourth season; corresponds to late February)}\example{Ddäddäg gäz a babaem täräp me eran ttongo mer abal ttoen dan.}{Killing animals in babaem is a very good thing.}}
\entry{Babaze}{\headword{Babaze}\pos{pn.}\sensenumber{3}\definition{male personal name}}
\entry{babdu}{\headword{babdu}\pos{n.}\sensenumber{3}\definition{type of taro}}
\entry{Bablela}{\headword{Bablela}\pos{pn.}\sensenumber{3}\definition{male personal name}}
\entry{Babra}{\headword{Babra}\pos{pn.}\sensenumber{3}\definition{female personal name}}
\entry{Babu}{\headword{Babu}\pos{pn.}\sensenumber{3}\definition{male personal name}}
\entry{badar}{\headword{badar}\pos{n.}\sensenumber{3}\definition{type of tree that grows in the grassland with white flowersTep a obo pällämpälläm dan a sisor kängkäm peyang otät ma dan källa ine täräp me. Igi ttoe de ade kokllo ma dan. (Its sap is white and is squeezed and eaten to treat diarrhea. The bark is scratched off.)}}
\entry{Badu}{\headword{Badu}\pos{pn.}\sensenumber{3}\definition{male personal name}}
\entry{baddbedd}{\headword{baddbedd}\pos{S vi.}\sensenumber{3}\definition{to dry up}\example{Ine da däbeddnegän, ada källäm a, ada walle mäg a, orbam a wa karama da ade petapeta abal gogon.}{Water dried up, and the ponds, the big creeks, pools, and swamps became very shallow.}\allomorph{bedd}\allomorph{baddnen}\allomorph{badd}\allomorph{bez}}
\entry{baebol}{\headword{baebol}\pos{n.}\sensenumber{3}\definition{Bible}\etymology{from EnglishBible}}
\entry{baet}{\headword{baet}\pos{n.}\sensenumber{3}\definition{cuscus}\example{Abo baet ttoe de näglle nyäng ngasnges e.}{Skin the cuscus to make a bag.}}
\entry{Baewa}{\headword{Baewa}\pos{pn.}\sensenumber{3}\definition{male personal name}}
\entry{bagama}{\headword{bagama}\pos{n.}\sensenumber{3}\definition{collar}}
\entry{bagen}{\headword{bagen}\pos{n.}\sensenumber{3}\definition{type of big taro}}
\entry{Baiduwa}{\headword{Baiduwa}\pos{pn.}\sensenumber{3}\definition{Baiduwa (toponym)}}
\entry{Baim}{\headword{Baim}\pos{pn.}\sensenumber{3}\definition{Baim (toponym)}}
\entry{=bakmall}{\headword{=bakmall}\pos{n. cl.}\sensenumber{3}\definition{comitative case clitic; with (only used after oba, ama, and nouns followed by =aba)}\example{Ngämi oba bakmall eka tameny amalla.}{We (excl.) discussed with them.}\example{Ge lla da ngäna ama bakmall skul att dan.}{These people are who I went to school with.}\example{Inggoemeny e bog tongoenen e bägne, bogmam nagnag aba bakmall.}{We will tell jokes with our friends.}\etymology{contr. of =aba + =kämall}}
\entry{Balimo}{\headword{Balimo}\pos{pn.}\sensenumber{3}\definition{Balimo (toponym)}}
\entry{ballängg}{\headword{ballängg}\variant{var. of}{ballɨngg}}
\entry{ballbell}{\headword{ballbell}\pos{A vi.}\sensenumber{3}\definition{to cook improperly}\example{Sana da yu me ballbell allan.}{Sago is cooking improperly on the fire.}}
\entry{ballɨngg}{\headword{ballɨngg}\pos{S vt.}\sensenumber{1}\definition{to welcome, greet}\example{Ngänaeka alle yamballägallo.}{In tears, they greeted him.}\example{Sisor pazi di bamballmenyneyo.}{They will be celebrating the new year.}\sensenumber{2}\definition{to predict}\example{Bogo abade ngasnen e ttoen de damballmenynegän.}{He predicted things that will happen in the future.}\allomorph{mballɨg/mballäg}\allomorph{mballmeny}\allomorph{ballɨminy}}
\entry{ballma}{\headword{ballma}\pos{n.}\sensenumber{2}\definition{type of biting bee found in trees}}
\entry{ballme}{\headword{ballme}\pos{n.}\sensenumber{2}\definition{dawn, daybreak}}
\entry{ballo bällabällott}{\headword{ballo bällabällott}\pos{n.}\sensenumber{2}\definition{type of big taro}}
\entry{bamearoro}{\headword{bamearoro}\pos{n.}\sensenumber{2}\definition{type of mushroomDdäddäg ma dan, ap me päddabag dan. (It's edible and grows in the grassland.)}}
\entry{Bana}{\headword{Bana}\pos{pn.}\sensenumber{2}\definition{female personal name}}
\entry{band}{\headword{band}\pos{n.}\sensenumber{2}\definition{type of tree}}
\entry{bandra}{\headword{bandra}\pos{n.}\sensenumber{2}\definition{songKili me o ingong me lla yaba bod gallagallab llätnen ma. (When celebrating or dancing, what people open their mouths to sing.)}\example{Wagiba bandra de nɨllɨtnan sisri ag me.}{Wagiba was singing this morning.}}
\entry{banggo}{\headword{banggo}\pos{n.}\sensenumber{2}\definition{type of long yam with a white interior, thorns, and no hair}}
\entry{banggu}{\headword{banggu}\pos{n.}\sensenumber{2}\definition{headdressBun mi mättmätt ma ingong täräp me dirom kom alle iatt. (Worn on the head while dancing, woven from cassowary feathers.)}\example{Däbe bäne banggu di aeya diwän?}{Who wove that headdress for you?}}
\entry{baob}{\headword{baob}\pos{n.}\sensenumber{2}\definition{water lilyKarama me päddabag llo popo, käp a otnan ma dan. (A flower that grows in Karama swamp; its fruit is edible.)}}
\entry{Baogab}{\headword{Baogab}\pos{pn.}\sensenumber{2}\definition{Baogab (an island in Karama swamp used for camping; has coconuts and bananas)}}
\entry{baottbaott}{\headword{baottbaott}\pos{A vi.}\sensenumber{2}\definition{to walk around aimlessly}}
\entry{Barekam}{\headword{Barekam}\pos{pn.}\sensenumber{2}\definition{male personal name}}
\entry{bargae}{\headword{bargae}\pos{n.}\sensenumber{2}\definition{type of fish}}
\entry{Basido}{\headword{Basido}\pos{pn.}\sensenumber{2}\definition{Basido (toponym)}}
\entry{Basido kona}{\headword{Basido kona}\pos{pn.}\sensenumber{2}\definition{Pastor's corner (in Limol)Limol ttängäm me ddob ttängäm bin.}}
\entry{Bati}{\headword{Bati}\pos{pn.}\sensenumber{2}\definition{female personal name}}
\entry{batri}{\headword{batri}\pos{n.}\sensenumber{2}\definition{battery}\sensenumber{2}\definition{battery}\etymology{from Englishbattery}\subentry{\headword{batri käp}\pos{n.}\definition{battery}}}
\entry{batt1}{\headword{batt1}\pos{mod.}\sensenumber{2}\definition{adult, mature}\example{lla batt, mälla batt}{adult man, adult woman}\example{Bogo batt däm de ibeb eran.}{She plants a mature plant.}}
\entry{batt2}{\headword{batt2}\pos{n.}\sensenumber{2}\definition{central lateral beam of a house}}
\entry{baur}{\headword{baur}\pos{n.}\sensenumber{2}\definition{type of spear}}
\entry{bawa}{\headword{bawa}\pos{n.}\sensenumber{1}\definition{season characterized by hunting and fishing in heavy rain (ninth season; corresponds to June)}\example{Bawa täräp me nyongo da era ddobae tärpi dan.}{During the shower times, the roads can be very slippery.}\sensenumber{2}\definition{rain shower}\sensenumber{2}\definition{season characterized by hunting and fishing in light rain (tenth season; corresponds to July)}\sensenumber{2}\definition{waveBem me ine ngänglläb. (Water jumps in the sea.)}\sensenumber{2}\definition{drizzleYogäll minyminy. (Little rain.)}\subentry{\headword{bawa minyminy}\pos{n.}\definition{season characterized by hunting and fishing in light rain (tenth season; corresponds to July)}}\subentry{\headword{bawa pokallmäll}\pos{n.}\definition{waveBem me ine ngänglläb. (Water jumps in the sea.)}}\subentry{\headword{bawa sarasaram}\pos{n.}\definition{drizzleYogäll minyminy. (Little rain.)}}}
\entry{bawur}{\headword{bawur}\variant{sp. var. of}{baur}}
\entry{bazere}{\headword{bazere}\pos{n.}\sensenumber{2}\definition{type of purple yam with hairs and no thorns}}
\entry{bäbälläd}{\headword{bäbälläd}\pos{S vt.}\sensenumber{2}\definition{to drop without warning}\allomorph{blläd}\allomorph{bällädnen}}
\entry{bäblem}{\headword{bäblem}\pos{v.}\sensenumber{2}\definition{shiver}}
\entry{bäbnge}{\headword{bäbnge}\pos{n.}\sensenumber{2}\definition{type of palm with coconuts with a light yellow and green exocarp}}
\entry{bäbrem}{\headword{bäbrem}\pos{ideo.}\sensenumber{2}\definition{sound made by cassowaries}\example{Dirom a bäbrem agan.}{The cassowary rumbled.}}
\entry{bäd}{\headword{bäd}\pos{n.}\sensenumber{2}\definition{type of large tree that grows in the grassland with wood used for firewood and bark used for building walls}}
\entry{bädab}{\headword{bädab}\pos{S vi.}\sensenumber{1}\definition{to shine brightly (of the moon)}\example{Kokta da näbdaban.}{The moon is shining.}\sensenumber{2}\definition{to dawn, break}\example{Ttongo ag a angde däbdabän ge ttoen de ngäna ddobagabira dällätne, ddone ada gotngoenegänän.}{The next morning when it dawn broke, I told this story to others and they laughed a lot.}\sensenumber{3}\definition{dawn}\example{Kikiem a bädab e ekawang pa dan.}{The kikiem is a bird that sings at dawn.}\sensenumber{3}\definition{morning star}\allomorph{bdab}\allomorph{bäd}\allomorph{bd}\subentry{\headword{bädab piro}\pos{n.}\definition{morning star}}}
\entry{Bädämalloang}{\headword{Bädämalloang}\pos{pn.}\sensenumber{3}\definition{Bädämalloang (sago place along the road from Limol to the canoe place where a large tree has fallen halfway over the path (AZ95))}}
\entry{bädma}{\headword{bädma}\pos{n.}\sensenumber{3}\definition{type of medicinal plantTatu ma dan itrell agone obo ine walle, ddob kaepnen ma dag. (Wash with its liquid, and the stem can be chewed.)}\sensenumber{3}\definition{planting a bädma plant as a gesture of peace}\subentry{\headword{bädma ibeny}\pos{n.}\definition{planting a bädma plant as a gesture of peace}}}
\entry{bädmaol}{\headword{bädmaol}\pos{n.}\sensenumber{3}\definition{small sago flower}}
\entry{bädde1}{\headword{bädde1}\pos{n.}\sensenumber{3}\definition{type of large tree found by rivers}}
\entry{bädde2}{\headword{bädde2}\pos{n.}\sensenumber{3}\definition{type of big taro}}
\entry{bägallem}{\headword{bägallem}\pos{n.}\sensenumber{3}\definition{type of tree that grows in the grassland with wood used for firewood and yellow fruit}}
\entry{bägäbägäl}{\headword{bägäbägäl}\pos{n.}\sensenumber{3}\definition{Achilles/calcaneal tendon (on the back of ankle)}\etymology{redup. ofbägäl}}
\entry{bägäl}{\headword{bägäl}\pos{n.}\sensenumber{3}\definition{bowTäl alle ngasngesatt ttoen ddäddäg gäddnan e. Toboll spun ma da. (Thing made of bamboo to kill animals. Used to throw spears.)}\example{Bägäl de popo eran.}{He shapes the bow.}\example{Masar täräp me lla da bägäl alle gogäddnän.}{In the old times, people fought with bows.}\sensenumber{3}\definition{gunfire}\sensenumber{3}\definition{part of bow}\sensenumber{3}\definition{place for bows and spears}\sensenumber{3}\definition{extra bowstring}\sensenumber{3}\definition{small bow}\subentry{\headword{bägäl kuwem}\pos{n.}\definition{gunfire}}\subentry{\headword{bägäl mäträt}\pos{n.}\definition{part of bow}}\subentry{\headword{bägäl odar}\pos{n.}\definition{place for bows and spears}}\subentry{\headword{bägäl tangge}\pos{n.}\definition{extra bowstring}}\subentry{\headword{bägälbägäl}\pos{n.}\definition{small bow}}}
\entry{bägäm}{\headword{bägäm}\pos{n.}\sensenumber{3}\definition{type of tree found by creeks with bark used for weaving bags or strong rope}}
\entry{bägem}{\headword{bägem}\pos{n.}\sensenumber{3}\definition{type of tree with pink flowers and fist-sized fruit with a pit}}
\entry{Bäglle}{\headword{Bäglle}\pos{pn.}\sensenumber{3}\definition{male personal name}}
\entry{bäkän}{\headword{bäkän}\pos{n.}\sensenumber{3}\definition{type of cultivated plant with leaves eaten with sago}}
\entry{bälämbäl}{\headword{bälämbäl}\pos{S vt.}\sensenumber{1}\definition{to miss, long for}\example{Ngäna ttongo lla de bälämbäl eran.}{I miss someone.}\sensenumber{2}\definition{to remember, think of}\example{Angde lla da borale bo eka de bäddareyo, ubi mɨnyi ngattongatt masarmasar de bämbälaebeyo.}{When people hear the song of the flute, they will remember their ancestors from before.}\example{Ngäma ddob gagäll gudae att gagäll de dämbälaebnalla.}{Some of us (excl.) were remembering bad things from the past.}\allomorph{mbäl}\allomorph{bälnan}\allomorph{bäl}}
\entry{bälwäd}{\headword{bälwäd}\variant{var. of}{bolod}}
\entry{bälwod}{\headword{bälwod}\variant{var. of}{bolod}}
\entry{bäll}{\headword{bäll}\pos{n.}\sensenumber{2}\definition{thigh}\sensenumber{2}\definition{femur}\subentry{\headword{bäll kutt}\pos{n.}\definition{femur}}}
\entry{bällabälle}{\headword{bällabälle}\pos{S vt.}\sensenumber{2}\definition{to find}\example{Ngäna kumuddäga ttägäll käp tärpitärpi de nabllenegan.}{I found three smooth stones.}\allomorph{blle}\allomorph{bälle}\allomorph{bällanan}\allomorph{bllanen}\allomorph{bälla}}
\entry{bällam}{\headword{bällam}\pos{quant.}\sensenumber{2}\definition{every}\example{täräp bällam}{always}\example{Ebdo bällam, yogoll a manen eran.}{Every day, it rains.}\example{Bongo tämamae ttoen bällam e zizag dan.}{You are the owner of everything.}}
\entry{bällamb}{\headword{bällamb}\pos{v.}\sensenumber{2}\definition{ambush}}
\entry{bälläg1}{\headword{bälläg1}\pos{n.}\sensenumber{2}\definition{type of introduced bananaTubutubu pänyanzag dan, pätt a kälsere dan. Obo däg a yuwog dag. O me käp a obo mer moko dag, kire da yuma dag. (It grows long and the trunk is small. Its bunches are plentiful. When ripe, they are very tasty; when unripe, they are cooked.)}}
\entry{bälläg2}{\headword{bälläg2}\pos{n.}\sensenumber{2}\definition{Areca palmObo ttam a sana yuma dan. (Its leaves are used to wrap sago.)}}
\entry{bällämbäll}{\headword{bällämbäll}\pos{n.}\sensenumber{2}\definition{thought}\sensenumber{2}\definition{thoughtless, apathetic, uncaring}\example{Däbe lla da eran lla bällämbällämmeny dan.}{That man doesn't care about others.}\etymology{redup. of bälläm + =meny}\subentry{\headword{bällämbällämmeny}\pos{mod.}\definition{thoughtless, apathetic, uncaring}}}
\entry{bällängg}{\headword{bällängg}\pos{S vi.}\sensenumber{2}\definition{to split up, separate}\example{Ubi gombällägmenyamän.}{They were splitting up.}\allomorph{mbälläg}\allomorph{mbällämeny}}
\entry{bällgab}{\headword{bällgab}\pos{S vt.}\sensenumber{2}\definition{to open (e.g. eyes, flowers)}\example{Däräng täräpang dagirnän ikop saeyang käsre ikop däbällgabän.}{Dog stayed there a long time with eyes closed, then opened his eyes.}\allomorph{bällgaeb}\allomorph{bällg}}
\entry{bällge}{\headword{bällge}\pos{S vi.}\sensenumber{2}\definition{to spread, scatter, move away}\example{Ngämo mosemosen a ddob mälla peyang gognegän, ede ubi gobällgewän.}{My older brothers got married, so they moved away (from home).}}
\entry{bällkäp}{\headword{bällkäp}\pos{n.}\sensenumber{2}\definition{"ant egg (large)" - ant pupae. Edible.}}
\entry{bällma}{\headword{bällma}\pos{n.}\sensenumber{2}\definition{spit, saliva}\sensenumber{2}\definition{spit, saliva}\sensenumber{2}\definition{to spit at}\example{Ubi obom bällma dawäntmenyneyo.}{They spat at him.}\subentry{\headword{bällma käkan}\pos{n.}\definition{spit, saliva}}\subentry{\headword{bällma ontog}\pos{S vt.}\definition{to spit at}}}
\entry{bälltoe}{\headword{bälltoe}\pos{n.}\sensenumber{2}\definition{type of tree}}
\entry{Bämäg}{\headword{Bämäg}\pos{pn.}\sensenumber{2}\definition{Bämäg (toponym)}}
\entry{bämäng}{\headword{bämäng}\pos{n.}\sensenumber{2}\definition{type of tree}}
\entry{bänamb}{\headword{bänamb}\pos{S vt.}\sensenumber{2}\definition{to open (something folded, e.g. book, mouth)}\example{Buk di bämbnaebmam.}{We will open the book.}\example{Obo kili peyang ttang de imbinaebnegan.}{He happily arrived with his arms open.}\allomorph{bänaemb}\allomorph{mbänab}\allomorph{mbänaeb}\allomorph{mbnaeb}\allomorph{mbänamb}\allomorph{mbn}\allomorph{mbin}}
\entry{bänäm}{\headword{bänäm}\pos{n.}\sensenumber{1}\definition{type of very small insect that lives on bandicoots}\example{Maigag a bänäm peyang dag.}{Bandicoots have tiny insects.}\sensenumber{2}\definition{thorns on sago leaves}}
\entry{bänäne}{\headword{bänäne}\variant{sp. var. of}{bänene}}
\entry{bändaeg}{\headword{bändaeg}\pos{S vt.}\sensenumber{2}\definition{to pull an all-nighter (stay awake)}\example{Ibi iddob mɨnyi beyambadägeya.}{We (incl.) will pull an all-nighter.}\allomorph{mbädaeg}\allomorph{mbädaemeny}\allomorph{bändaemeny}}
\entry{bändam}{\headword{bändam}\pos{n.}\sensenumber{2}\definition{type of tree}}
\entry{bänz1}{\headword{bänz1}\pos{n.}\sensenumber{2}\definition{mosquito}}
\entry{bänz2}{\headword{bänz2}\pos{n.}\sensenumber{2}\definition{type of biting bee found in trees}}
\entry{bänzibänzi}{\headword{bänzibänzi}\pos{n.}\sensenumber{2}\definition{type of sagoTupi päddabag, pätt ullong sana dan täkällang. (It grows long, with a big, thorny trunk.)}}
\entry{bäng}{\headword{bäng}\pos{n.}\sensenumber{2}\definition{firestick (to start a fire)}}
\entry{bänga}{\headword{bänga}\pos{subr.}\sensenumber{1}\definition{though}\example{Bänga ngäna llamda dan, be ngäna mängallang dan.}{Though I am an old man, I am strong.}\sensenumber{2}\definition{should}\example{Bänga buwensegenyallo.}{We should leave it alone.}}
\entry{bänybäny}{\headword{bänybäny}\pos{S vt.}\sensenumber{2}\definition{to cut, slice (flesh)}\example{Ngämi ge ddia de däbänya.}{We (excl.) cut this deer.}\allomorph{bäny}\allomorph{bänynan}\allomorph{mbäny}}
\entry{bäräbäräl}{\headword{bäräbäräl}\pos{n.}\sensenumber{2}\definition{type of birdYäbäd me ine ma ibiag pa dan. (A bird that walks to the well when it's sunny.)}}
\entry{bärät}{\headword{bärät}\pos{n.}\sensenumber{2}\definition{type of small yam}}
\entry{bärke}{\headword{bärke}\pos{n.}\sensenumber{2}\definition{Papuan eclectusLlo ik me giddollag pa dan. Ttongo mamam dan, ttongo kukollkukoll dan. (A bird that lives in trees. One is red and the other is green.)}}
\entry{bärkebärke}{\headword{bärkebärke}\pos{n.}\sensenumber{2}\definition{type of algae that can be red or green like a parrot}\etymology{redup. of bärke}}
\entry{bät1}{\headword{bät1}\pos{n.}\sensenumber{2}\definition{type of tree}}
\entry{bät2}{\headword{bät2}\variant{var. of}{bɨt}}
\entry{bätäny}{\headword{bätäny}\pos{n.}\sensenumber{2}\definition{type of tree}}
\entry{bätbät}{\headword{bätbät}\variant{sp. var. of}{bɨtbɨt}}
\entry{bätte}{\headword{bätte}\pos{n.}\sensenumber{2}\definition{type of snakeBätbät gullem ulle da. Ddägnan ma da. Malla kuddäll e lla ddäddägang da. (It's a black snake. It's edible. It doesn't bite people to death.)}}
\entry{bättekuibiag}{\headword{bättekuibiag}\variant{var. of}{bätte}}
\entry{be}{\headword{be}\pos{coord.}\sensenumber{2}\definition{but}\example{Pamker a käpang gogon, be tomato da ddone käpang gogon.}{The pumpkin plant is bearing fruit, but the tomato plant is not bearing fruit.}}
\entry{beatururang}{\headword{beatururang}\pos{n.}\sensenumber{2}\definition{season characterized by thunderstorms and flooding (second season; corresponds to early February)}}
\entry{beawa}{\headword{beawa}\variant{sp. var. of}{beyawa}}
\entry{bebe1}{\headword{bebe1}\pos{n.}\sensenumber{2}\definition{type of pandanus with long fruit (\textbackslashtextasciitilde2 feet) Kutt a obo klekle dag, wätät ma mab dan. (Its seeds are small; an edible pandanus.)}}
\entry{bebe2}{\headword{bebe2}\pos{S vi.}\sensenumber{2}\definition{to leak, excrete}\example{Källa bebe we nalle.}{[You] go and pass excrement.}\example{De däräng a källa benanbenan dan.}{This dog is always pooping.}\example{Sana sisor a nyukukum me bebe allan.}{The new sago is leaking in the sago bag.}\allomorph{ben}\allomorph{benan}\allomorph{be}}
\entry{Bebelin}{\headword{Bebelin}\pos{pn.}\sensenumber{2}\definition{female personal name}}
\entry{bebi}{\headword{bebi}\pos{n.}\sensenumber{2}\definition{baby}\example{Bebi da ngänaeka allan.}{The baby is crying.}\etymology{from Englishbaby}}
\entry{begere}{\headword{begere}\pos{n.}\sensenumber{2}\definition{type of long purple yam}}
\entry{beibi}{\headword{beibi}\variant{sp. var. of}{bebi}}
\entry{bel}{\headword{bel}\pos{n.}\sensenumber{2}\definition{bell}\etymology{from Englishbell}}
\entry{bem}{\headword{bem}\pos{n.}\sensenumber{2}\definition{sea, ocean}\example{Kila a Wala gall alle tudi ittaenen e deyareyo bem e.}{Kila and Wala went to the sea with the canoe to fish.}}
\entry{Ben}{\headword{Ben}\pos{pn.}\sensenumber{2}\definition{male personal name}}
\entry{Benaeya}{\headword{Benaeya}\pos{pn.}\sensenumber{2}\definition{male personal name}}
\entry{benanbenan}{\headword{benanbenan}\pos{n.}\sensenumber{2}\definition{type of spear}}
\entry{bendoe}{\headword{bendoe}\pos{S vt.}\sensenumber{2}\definition{to confuse, mix up}\example{Ngäna ubim komlla angde ikop deyagän, ngäna yambedowaeyan.}{When I saw them two, I confused them for each other.}\allomorph{mbedowae}\allomorph{bendoenen}}
\entry{benmäll}{\headword{benmäll}\pos{S vi.}\sensenumber{1}\definition{to shine, flash}\example{Yesu bo pätt a wa iddpo da gombenmällnegnän.}{Jesus' body and clothes shined.}\sensenumber{2}\definition{shine, flash}\example{Gälas ulle alle mullae dan utale we benmäll a ballän.}{With a big mirror, it is possible for a flash of light to travel far away.}\allomorph{mbenmäll}}
\entry{Bensi}{\headword{Bensi}\pos{pn.}\sensenumber{2}\definition{female personal name}}
\entry{Benson}{\headword{Benson}\pos{pn.}\sensenumber{2}\definition{male personal name}}
\entry{Benta}{\headword{Benta}\pos{pn.}\sensenumber{2}\definition{female personal name}}
\entry{bengae}{\headword{bengae}\pos{S vt.}\sensenumber{1}\definition{to roof, cover}\example{Ubi ttägäll de nyeny alle daibengaemallo.}{They covered the mumus with nyeny wood.}\sensenumber{2}\definition{roof, roofing}\allomorph{bängae}\allomorph{ibinge}\allomorph{bengaenen}}
\entry{Ber}{\headword{Ber}\pos{pn.}\sensenumber{2}\definition{Ber (toponym)}}
\entry{Beradi}{\headword{Beradi}\pos{pn.}\sensenumber{2}\definition{Beradi (toponym)Sana ma, inuang ma, ttängäm ma, Bisuaka nyongo me dan. (On the road to Bisuaka; for sago, camping, and gardening.)}}
\entry{beräberäl}{\headword{beräberäl}\pos{A vt.}\sensenumber{2}\definition{to swing}\example{Ingnenang aba pite da mermerae abal beräberäl agnegnan.}{The grass skirts of the dancers swung very nicely.}}
\entry{Bes}{\headword{Bes}\pos{pn.}\sensenumber{2}\definition{female personal name}}
\entry{Bessie}{\headword{Bessie}\pos{pn.}\sensenumber{2}\definition{female personal name}}
\entry{Betliem}{\headword{Betliem}\pos{pn.}\sensenumber{2}\definition{Bethlehem}}
\entry{bette}{\headword{bette}\pos{n.}\sensenumber{2}\definition{crimson finchLlo tätäk alle ma gogowag pa kälsre dan. (A small bird that builds its nest from tree hollows.)}}
\entry{Bewag}{\headword{Bewag}\pos{pn.}\sensenumber{2}\definition{male personal name}}
\entry{beya}{\headword{beya}\variant{dial. var. of}{biye}}
\entry{beyambäg}{\headword{beyambäg}\pos{S vt.}\sensenumber{2}\definition{to chase}\example{Ngäna sɨmell beyambägag dan.}{I am a pig-chaser.}\example{Yäbäd imneimne gumbiebägän ddone dingmenän.}{Sun pursued from behind, but did not reach him.}\allomorph{mbieb}\allomorph{mbiebmeny}\allomorph{mbiebäg}\allomorph{beyambmeny}}
\entry{beyat}{\headword{beyat}\pos{n.}\sensenumber{2}\definition{type of tree that grows in the bush with wood used for house sticks}}
\entry{beyawa}{\headword{beyawa}\pos{pers. pron.}\sensenumber{2}\definition{he, she (emphatic third person singular pronoun)}\example{Beyawa dattkoeyän turik alle.}{He was the one who chopped it with an ax.}\sensenumber{2}\definition{resctrictive form of beyawa}\example{Beyawaebe sana dägdene.}{Only he beat the sago.}\sensenumber{2}\definition{copular form of beyawaebe}\example{Adi da ttongdae dan a ttongo ako ddone dan, be beyawaeben.}{God is the one and there is no other; only He is.}\sensenumber{2}\definition{copular form of beyawa (present form)}\example{Sisri ngäna erame giddollnen allan? Gänyme, ge ttängäm a beyawaenen.}{Where am I living now? Here, this village is where.}\sensenumber{2}\definition{past form of beyawaenen}\example{Ngämo mäg bo bin a beyawaeneya Sara.}{Sara, it was my mother's name.}\sensenumber{2}\definition{present copular form of beyawa}\example{Ngämo bin a beyawan Merol.}{Merol, it is my name.}\subentry{\headword{beyawaebe}\pos{pers. pron.}\definition{resctrictive form of beyawa}}\subentry{\headword{beyawaeben}\pos{cop.}\definition{copular form of beyawaebe}}\subentry{\headword{beyawaenen}\pos{cop.}\definition{copular form of beyawa (present form)}}\subentry{\headword{beyawaeneya}\pos{cop.}\definition{past form of beyawaenen}}\subentry{\headword{beyawan}\pos{cop.}\definition{present copular form of beyawa}}}
\entry{beyawainin}{\headword{beyawainin}\variant{sp. var. of}{beyawaenen}}
\entry{bib1}{\headword{bib1}\pos{A vi.}\sensenumber{2}\definition{to break ground, surface}\example{Mätta da zime bib gognegän.}{The yams have already surfaced.}}
\entry{bib2}{\headword{bib2}\pos{n.}\sensenumber{2}\definition{spring water}}
\entry{bibi}{\headword{bibi}\pos{pers. pron.}\sensenumber{2}\definition{you all, you (second person nonsingular pronoun, nominative form)}\example{Bibi utamom ngattong.}{You all go first.}\sensenumber{2}\definition{accusative form of bibi}\example{Ngäna bibim truminy anggan.}{I am calling you all.}\sensenumber{2}\definition{dative form of bibi}\example{Känazbag, ngäna mɨnyi bibra sisor ttoenttoen de bɨllɨt.}{Tomorrow, I will tell you all a new story.}\sensenumber{2}\definition{possessive form of bibi}\example{Ge bina ma daeya?}{Is this your (pl.) house?}\sensenumber{2}\definition{ablative-possessive form of bibi}\example{Lla da binaene eka de mɨnyi bondärmällnegnän.}{People will be listening to your (pl.) words.}\subentry{\headword{bibim}\pos{pers. pron.}\definition{accusative form of bibi}}\subentry{\headword{bibra}\pos{pers. pron.}\definition{dative form of bibi}}\subentry{\headword{bina}\pos{pers. pron.}\definition{possessive form of bibi}}\subentry{\headword{binaene}\pos{pers. pron.}\definition{ablative-possessive form of bibi}}}
\entry{Bibiae}{\headword{Bibiae}\pos{pn.}\sensenumber{2}\definition{female personal name}}
\entry{Bibiai}{\headword{Bibiai}\variant{sp. var. of}{Bibiae}}
\entry{Bible}{\headword{Bible}\variant{sp. var. of}{baebol}}
\entry{bible}{\headword{bible}\pos{n.}\sensenumber{2}\definition{type of big taro}}
\entry{bibol}{\headword{bibol}\pos{n.}\sensenumber{2}\definition{type of birdBädab dowae e ekawang pa kälsre dan. (A bird that sings around dawn.)}}
\entry{biboz}{\headword{biboz}\pos{n.}\sensenumber{2}\definition{fairywren (emperor, white-shouldered)}}
\entry{Bidog}{\headword{Bidog}\pos{pn.}\sensenumber{2}\definition{male personal name}}
\entry{Big}{\headword{Big}\pos{pn.}\sensenumber{2}\definition{Big (toponym)}}
\entry{big}{\headword{big}\pos{n.}\sensenumber{2}\definition{type of very large tree that grows in the bush with wood used for firewood, especially when camping.}\sensenumber{2}\definition{type of grubWällang llo me, ddäddäg ma dan. (In bush trees; it's edible.)}\subentry{\headword{big budar}\pos{n.}\definition{type of grubWällang llo me, ddäddäg ma dan. (In bush trees; it's edible.)}}}
\entry{Bigag}{\headword{Bigag}\pos{pn.}\sensenumber{2}\definition{male personal name}}
\entry{Bigia}{\headword{Bigia}\variant{sp. var. of}{Bigiya}}
\entry{Bigiya}{\headword{Bigiya}\pos{pn.}\sensenumber{2}\definition{female personal name}}
\entry{bigma}{\headword{bigma}\pos{n.}\sensenumber{2}\definition{enclosure, pen, sty}\example{Bong angde bigma we dallän, sɨmell a dindugän.}{When Bong went to the pen, the pig had escaped.}}
\entry{Bigjay}{\headword{Bigjay}\pos{pn.}\sensenumber{2}\definition{male personal name}}
\entry{bikme}{\headword{bikme}\pos{n.}\sensenumber{2}\definition{type of palm tree with hanging, poisonous yellow and green fruits that can be eaten after being buried by the creek for up to 2 years and then cooked on the fire}\sensenumber{2}\definition{type of bikme palm with very chewy fruit}\sensenumber{2}\definition{type of bikme palm with yellow fruit that is very hard}\sensenumber{2}\definition{type of bikme palm with fruit that is soft like sago}\sensenumber{2}\definition{type of bikme palm with fruit that is medium-firm}\subentry{\headword{bikme laelem}\pos{n.}\definition{type of bikme palm with very chewy fruit}}\subentry{\headword{bikme pipllo}\pos{n.}\definition{type of bikme palm with yellow fruit that is very hard}}\subentry{\headword{bikme sanasana}\pos{n.}\definition{type of bikme palm with fruit that is soft like sago}}\subentry{\headword{bikme yobeg}\pos{n.}\definition{type of bikme palm with fruit that is medium-firm}}}
\entry{bikme tutu}{\headword{bikme tutu}\pos{n.}\sensenumber{2}\definition{type of string game}}
\entry{Biks}{\headword{Biks}\pos{pn.}\sensenumber{2}\definition{personal name}}
\entry{Biku}{\headword{Biku}\pos{pn.}\sensenumber{2}\definition{male personal name}}
\entry{bikwem}{\headword{bikwem}\pos{n.}\sensenumber{2}\definition{fireplaceYu odanen ma ngasnenatt ma ik mi o upe me. (A place for lighting a fire, inside or outside the house.)}\example{Ibi ibra mɨnyi ttongo sisor bikwem de ako bangeseya.}{We will also make a new fireplace for us.}}
\entry{bile}{\headword{bile}\pos{n.}\sensenumber{2}\definition{salt}}
\entry{bilod}{\headword{bilod}\pos{n.}\sensenumber{2}\definition{type of spear}}
\entry{Bimadbn}{\headword{Bimadbn}\pos{pn.}\sensenumber{2}\definition{Bimadbn (toponym)}}
\entry{bin}{\headword{bin}\pos{n.}\sensenumber{2}\definition{name}\example{Ngämo mälla bo bin a Kristina.}{My wife's name is Kristina.}\sensenumber{2}\definition{namesake}\sensenumber{2}\definition{serious}\example{ulle binang itrell peyang lla}{seriously ill person}\etymology{bin+ =ang}\subentry{\headword{binang1}\pos{n.}\definition{namesake}}\subentry{\headword{binang2}\pos{mod.}\definition{serious}}}
\entry{binbäddbädd}{\headword{binbäddbädd}\pos{adv.}\sensenumber{2}\definition{fully, completely}}
\entry{Bine}{\headword{Bine}\pos{pn.}\sensenumber{2}\definition{Bine language (spoken to the east)}}
\entry{binzeg}{\headword{binzeg}\pos{S vt.}\sensenumber{2}\definition{to heat, warm}\allomorph{binzenen}\allomorph{mbizeg}\allomorph{mbize}}
\entry{Binyomoll Källäm}{\headword{Binyomoll Källäm}\pos{pn.}\sensenumber{2}\definition{Binyomoll Pond (on the road to Kinkin)}}
\entry{Bipi}{\headword{Bipi}\pos{pn.}\sensenumber{2}\definition{Bipi (toponym)}}
\entry{bir}{\headword{bir}\pos{n.}\sensenumber{1}\definition{spit, skewer}\sensenumber{2}\definition{to roast}\example{Ngatengate da biratt a ddobae moko dan.}{Roasted possum is very tasty.}}
\entry{Birigi}{\headword{Birigi}\pos{pn.}\sensenumber{2}\definition{male personal name}}
\entry{biro}{\headword{biro}\pos{n.}\sensenumber{2}\definition{pen (writing implement)}}
\entry{bisbis}{\headword{bisbis}\pos{n.}\sensenumber{2}\definition{type of stingless bee}}
\entry{bisel}{\headword{bisel}\pos{n.}\sensenumber{2}\definition{type of sago that grows tall and wideUlle kuttang sana dan. (It's a sago with a lot of pith.)}}
\entry{Bisenmo}{\headword{Bisenmo}\pos{pn.}\sensenumber{2}\definition{Bisenmo (toponym)}}
\entry{bisnis}{\headword{bisnis}\pos{n.}\sensenumber{2}\definition{business}\etymology{from Englishbusiness}}
\entry{Bisuaka}{\headword{Bisuaka}\pos{pn.}\sensenumber{2}\definition{Bisuaka (Bituri-speaking village in Oriomo-Bituri Rural LLG; has a primary school but no aid post)}}
\entry{Biswaka}{\headword{Biswaka}\variant{sp. var. of}{Bisuaka}}
\entry{bitän}{\headword{bitän}\pos{n.}\sensenumber{2}\definition{type of animal}}
\entry{Bitur}{\headword{Bitur}\pos{pn.}\sensenumber{2}\definition{Bitur language (spoken to the north)}}
\entry{bittang}{\headword{bittang}\pos{A vt.}\sensenumber{2}\definition{to litter, dirty}}
\entry{bittott}{\headword{bittott}\pos{n.}\sensenumber{2}\definition{grey squirrel}}
\entry{biwiz}{\headword{biwiz}\pos{n.}\sensenumber{2}\definition{type of large tree that grows in the bush with purple flowers and wood used for kundu drums and canoes}}
\entry{biye}{\headword{biye}\pos{n.}\sensenumber{2}\definition{taro}\example{Biye de bongo ai dan ibik alle nibe.}{You can plant taro with the ibik.}}
\entry{Biyewolatt}{\headword{Biyewolatt}\pos{pn.}\sensenumber{2}\definition{Biyewolatt (toponym)}}
\entry{bɨd}{\headword{bɨd}\pos{n.}\sensenumber{2}\definition{gum tree}}
\entry{bɨk}{\headword{bɨk}\pos{n.}\sensenumber{2}\definition{poisoned creek}}
\entry{bɨt}{\headword{bɨt}\pos{mod.}\sensenumber{2}\definition{dark}\sensenumber{2}\definition{black}\example{Obo ttoe a bɨtbɨt dan.}{His skin is black.}\sensenumber{2}\definition{purple (color of sawis, purple yam)}\subentry{\headword{bɨtbɨt}\pos{col.}\definition{black}}\subentry{\headword{bɨtbɨtbɨtbɨt}\pos{col.}\definition{purple (color of sawis, purple yam)}}}
\entry{blab}{\headword{blab}\pos{S vi.}\sensenumber{2}\definition{to mature, reach puberty}\example{Ge nge da ablaban.}{This coconut has matured.}\example{Män a zäme ablaban.}{The girl has already reached puberty.}\allomorph{blaeb}}
\entry{blengud}{\headword{blengud}\pos{n.}\sensenumber{2}\definition{blanket}\etymology{from Englishblanket}}
\entry{bllablla}{\headword{bllablla}\pos{n.}\sensenumber{2}\definition{type of cordyline with big leaves, red fruit, and leaves that are used to fan fire and tied around the waist and chest for dancing}}
\entry{bllablle}{\headword{bllablle}\variant{fast speech var. of}{bällabälle}}
\entry{bllam}{\headword{bllam}\variant{fast speech var. of}{bällam}}
\entry{blläg}{\headword{blläg}\pos{S vt.}\sensenumber{2}\definition{to serve}\example{Pilat de bäbllägaebeyo.}{They will serve plates.}}
\entry{bllolla}{\headword{bllolla}\pos{n.}\sensenumber{2}\definition{type of tree}}
\entry{bob1}{\headword{bob1}\pos{n.}\sensenumber{2}\definition{floodYogoll mamott me ine da bingkälän a wawaem. (After it rains, water rises and flows.)}\example{Tätäm ttäle bun parga de bob a era dällekän.}{Yesterday, the flood destroyed the Ttäle Bun bridge.}\sensenumber{2}\definition{flooded}\example{Tatuma da bobag daeya.}{The bathing area was flooded.}\etymology{bob + =ang}\subentry{\headword{bobag}\pos{n.}\definition{flooded}}}
\entry{bob2}{\headword{bob2}\pos{n.}\sensenumber{2}\definition{type of tree}\sensenumber{2}\definition{type of mushroomWälläng me päddabag, bob llo dädär me, ddädäg ma dan. (Grows in the bush, on the bob tree; it's edible.)}\subentry{\headword{bob lläkäm}\pos{n.}\definition{type of mushroomWälläng me päddabag, bob llo dädär me, ddädäg ma dan. (Grows in the bush, on the bob tree; it's edible.)}}}
\entry{Bobe}{\headword{Bobe}\pos{pn.}\sensenumber{2}\definition{Bobe (toponym)}}
\entry{bobllem}{\headword{bobllem}\pos{S vi.}\sensenumber{2}\definition{to flap, shake (of skin)}\example{Mälla bo pätt a bobllem allan.}{The skin of the woman's body shakes.}}
\entry{Bobngätt}{\headword{Bobngätt}\pos{pn.}\sensenumber{2}\definition{Bobngätt (toponym)}}
\entry{bobngätt}{\headword{bobngätt}\pos{n.}\sensenumber{2}\definition{place name}}
\entry{Bobzag}{\headword{Bobzag}\pos{pn.}\sensenumber{1}\definition{personal name}\sensenumber{2}\definition{dog name}}
\entry{Bobze}{\headword{Bobze}\pos{pn.}\sensenumber{2}\definition{Bobze (toponym)}\allomorph{Bobz}}
\entry{bod}{\headword{bod}\pos{n.}\sensenumber{1}\definition{mouth}\example{Ngäna mɨnyi mälläng ik dae beyazen a bod ik dae bigezän.}{I will put it in through your nostrils and take it out through your mouth.}\example{Mällaeyaba bod me dan.}{It's on the women's lips (i.e. the subject of gossip).}\sensenumber{2}\definition{beak}\example{Bod a malla ddäddäg ma dan.}{The beak is not edible.}\sensenumber{3}\definition{lip of bag (braid that goes along the rim of the bag to finish it)}\sensenumber{3}\definition{lip}\etymology{lit. 'mouth skin'}\subentry{\headword{bod ttoe}\pos{n.}\definition{lip}}}
\entry{bodo}{\headword{bodo}\pos{mod.}\sensenumber{3}\definition{full}\example{Kap a bodo dan.}{The cup is full.}\example{Ngäna ine de zämaeangae nägawan do angde kap a bodo agan.}{I poured water until the cup was full.}\sensenumber{3}\definition{nonsingular form of bodo}\subentry{\headword{bodobodo}\pos{mod.}\definition{nonsingular form of bodo}}}
\entry{bodobodom}{\headword{bodobodom}\pos{n.}\sensenumber{3}\definition{type of biting ant}}
\entry{Bodog}{\headword{Bodog}\pos{pn.}\sensenumber{3}\definition{male personal name}}
\entry{boddo}{\headword{boddo}\pos{n.}\sensenumber{3}\definition{type of large tree with a big trunk, fruit that deer eat, and aerial prop roots}}
\entry{boe}{\headword{boe}\pos{n.}\sensenumber{3}\definition{type of cultivated tree with edible indigo fruit and white flowers that attract birds and butterfliesTtoe de kängkäm ma dan nane we kumye itrel peyang me. (The bark is squeezed and the water is drunk when sick with cough.)}}
\entry{bog}{\headword{bog}\pos{n.}\sensenumber{3}\definition{type of taro}}
\entry{boga}{\headword{boga}\pos{n.}\sensenumber{3}\definition{type of birdAp me mise ingoll papllägag pa dan. (A flying bird in the savanna, it's similar to the common cicadabird.)}}
\entry{boge}{\headword{boge}\pos{n.}\sensenumber{3}\definition{mudfishBɨtbɨt kollba. (A black fish.)}\example{Boge kollba da mer mokowang ddäddäg dan.}{Mudfish is a very delicious fish.}}
\entry{bogel}{\headword{bogel}\pos{n.}\sensenumber{3}\definition{seaweedWalle ik me towall. (Grass in the water.)}}
\entry{bogo}{\headword{bogo}\pos{pers. pron.}\sensenumber{3}\definition{he, she (third person singular animate pronoun, nominative form)}\example{Bogo daden ma me?}{Is he home?}\sensenumber{3}\definition{additive form of bogo}\example{Bode ngämlle pate Ende eka walle eka gognän.}{He also spoke to me in Ende.}\sensenumber{3}\definition{instrumental-comitative form of bogo}\example{Kukumi oballe beyareyo do Daru wi.}{He will go with Kukumi to Daru.}\sensenumber{3}\definition{ablative-possessive form of bogo}\example{Ngämo baba da ddobae mamoeang daeya. Ngäna obene ddäddäg de däddägnegne.}{My father was a very good hunter. I ate meat from his kills.}\sensenumber{3}\definition{clitic form of obene}\example{Pip bäne ddäddäg a ttällanenang dan.}{The sting of the red bee is painful.}\sensenumber{3}\definition{dative form of bogo}\example{Ngäna oblle toboll de nanttogan.}{I gave her the spear.}\sensenumber{3}\definition{clitic form of oblle}\example{Obo nane bälle ikop e dallän.}{She went to see her aunt.}\sensenumber{3}\definition{possessive form of bogo}\example{Gänyan obo sana da.}{Here is his sago.}\sensenumber{3}\definition{oneself, himself, herself (reflexive form of bogo)}\example{Yesu obozaga bo kuddäll de gomballägän.}{Jesus predicted his own death.}\sensenumber{3}\definition{clitic form of obo}\example{Ngämo yae bo bin a Nowar.}{My mother's name is Nowar.}\sensenumber{3}\definition{oneself, himself, herself (reflexive form of bogo)}\example{Mägda oboobo tatu agan.}{Mother washed herself.}\sensenumber{3}\definition{accusative form of bogo}\example{Ngäna obom dandär.}{I heard him.}\sensenumber{3}\definition{clitic form of obom}\example{Ngäna Wagiba bom dangmingg.}{I helped Wagiba.}\allomorph{ob}\subentry{\headword{bode}\pos{pers. pron.}\definition{additive form of bogo}}\subentry{\headword{obä}\pos{pers. pron.}\definition{instrumental-comitative form of bogo}}\subentry{\headword{obene}\pos{pers. pron.}\definition{ablative-possessive form of bogo}}\subentry{\headword{=bäne}\pos{pron. cl.}\definition{clitic form of obene}}\subentry{\headword{oblle}\pos{pers. pron.}\definition{dative form of bogo}}\subentry{\headword{=bälle}\pos{pron. cl.}\definition{clitic form of oblle}}\subentry{\headword{obo}\pos{pers. pron.}\definition{possessive form of bogo}}\subentry{\headword{obozaga}\pos{pers. pron.}\definition{oneself, himself, herself (reflexive form of bogo)}}\subentry{\headword{=bo}\pos{pron. cl.}\definition{clitic form of obo}}\subentry{\headword{oboobo}\pos{pers. pron.}\definition{oneself, himself, herself (reflexive form of bogo)}}\subentry{\headword{obom}\pos{pers. pron.}\definition{accusative form of bogo}}\subentry{\headword{=bom}\pos{pron. cl.}\definition{clitic form of obom}}}
\entry{bogobogo}{\headword{bogobogo}\pos{n.}\sensenumber{3}\definition{type of small yam with a pure white interior}}
\entry{Bok}{\headword{Bok}\pos{pn.}\sensenumber{3}\definition{Buk (Taeme-speaking village in Morehead Rural LLG; from Limol, one must pass through Kinkin and Kondobol)}}
\entry{boko}{\headword{boko}\pos{n.}\sensenumber{3}\definition{type of lizardTot ik mi giddollag pipllo ulle. (A big lizard that lives in trash.)}\example{Llɨg kälekäle da boko de näbäddallo sisri.}{Small children kill boko lizards.}}
\entry{bol}{\headword{bol}\pos{n.}\sensenumber{3}\definition{ball}\etymology{from Englishball}}
\entry{bolod}{\headword{bolod}\pos{n.}\sensenumber{3}\definition{sugarcane}}
\entry{bolwod}{\headword{bolwod}\variant{sp. var. of}{bolod}}
\entry{boll}{\headword{boll}\variant{var. of}{bol}}
\entry{bollboll}{\headword{bollboll}\pos{S vt.}\sensenumber{3}\definition{to open sago}\example{Bogo ako ttongo pättkäp däbollän gädagäde we.}{He opened another node of the sago trunk to beat.}\allomorph{boll}\allomorph{bollnen}}
\entry{Bollga}{\headword{Bollga}\pos{pn.}\sensenumber{3}\definition{female personal name}}
\entry{bollga}{\headword{bollga}\pos{n.}\sensenumber{3}\definition{type of sagoUlle päddabag sana dan, kol a obo pällämpälläm dan. (A sago that grows big; its pith is white.)}}
\entry{Bolloll}{\headword{Bolloll}\pos{pn.}\sensenumber{3}\definition{Bolloll (toponym, on the southward road to Malam)Joshua bo inuang ma dan. (Camping place of Joshua Ben Danipa.)}}
\entry{boma}{\headword{boma}\pos{n.}\sensenumber{3}\definition{stump}}
\entry{bomall}{\headword{bomall}\pos{n.}\sensenumber{3}\definition{type of tree that grows in the bush with bark used to make sago baskets}}
\entry{bombllo}{\headword{bombllo}\pos{S vi.}\sensenumber{1}\definition{to increase, proliferate, multiply}\example{Ngäma lla da ada ingollang gombllowän.}{Our (excl.) population grew like this.}\sensenumber{2}\definition{to increase, proliferate, multiply}\example{Mätta da bombllo allan.}{The yams proliferate.}\example{Ngäna komlla lla de bombllo yaran.}{My two daughters each married into a family (lit. I multiplied two people; i.e. each daughter will now have a family of her own).}\allomorph{mbllo}}
\entry{bombom}{\headword{bombom}\pos{n.}\sensenumber{2}\definition{type of tree with big leaves and soft wood that floats and is carved by children}}
\entry{bomo}{\headword{bomo}\pos{n.}\sensenumber{2}\definition{aerial root (e.g. of pandanus)}}
\entry{Bomso}{\headword{Bomso}\pos{pn.}\sensenumber{2}\definition{male personal name}}
\entry{Bonibi}{\headword{Bonibi}\pos{pn.}\sensenumber{2}\definition{female personal name}}
\entry{bonzro}{\headword{bonzro}\pos{n.}\sensenumber{2}\definition{dancing on the side playfully}}
\entry{Bong}{\headword{Bong}\pos{pn.}\sensenumber{2}\definition{male personal name}}
\entry{bongo}{\headword{bongo}\pos{pers. pron.}\sensenumber{2}\definition{you (second person singular pronoun, nominative form)}\example{Bongo erowe ibi alle?}{Where are you going?}\sensenumber{2}\definition{additive form of bongo}\example{Senti, bade bablle ikom.}{Senti, you take some for yourself too.}\sensenumber{2}\definition{additive form of bongo}\example{Bako ikop nas.}{You close your eyes too.}\sensenumber{2}\definition{dative form of bongo}\example{Ngäna sana de gämäll agan bablle.}{I stole the sago for you.}\sensenumber{2}\definition{accusative form of bongo}\example{Ngäna bam trungg allan.}{I am calling you.}\sensenumber{2}\definition{possessive form of bongo}\example{Bäne däräng a daden.}{You have a dog (lit. your dog exists).}\sensenumber{2}\definition{yourself (reflexive form of bongo)}\example{Ddone dan otät yu watt a, abo bänezaga yu näga.}{There isn't any cooked food; cook it yourself.}\sensenumber{2}\definition{ablative-possessive form of bongo}\example{Ngäna sana de gämäll agan bänene.}{I stole the sago from you.}\allomorph{ba}\allomorph{bab}\subentry{\headword{bade}\pos{pers. pron.}\definition{additive form of bongo}}\subentry{\headword{bako}\pos{pers. pron.}\definition{additive form of bongo}}\subentry{\headword{bablle}\pos{pers. pron.}\definition{dative form of bongo}}\subentry{\headword{bam}\pos{pers. pron.}\definition{accusative form of bongo}}\subentry{\headword{bäne}\pos{pers. pron.}\definition{possessive form of bongo}}\subentry{\headword{bänezaga}\pos{pers. pron.}\definition{yourself (reflexive form of bongo)}}\subentry{\headword{bänene}\pos{pers. pron.}\definition{ablative-possessive form of bongo}}}
\entry{Bonybony}{\headword{Bonybony}\pos{pn.}\sensenumber{2}\definition{Bonybony (camping, garden, and sago place (AX94))}}
\entry{bonydre}{\headword{bonydre}\pos{n.}\sensenumber{2}\definition{goshawk (grey-headed, brown); collared sparrowhawk}}
\entry{bor}{\headword{bor}\pos{n.}\sensenumber{2}\definition{scar}}
\entry{borale}{\headword{borale}\pos{n.}\sensenumber{2}\definition{traditional bamboo flute used to scare wallabiesTtongdae pattlle am pop peyang a bod alle pädoe ma dan. (One internode of pattlle bamboo with holes, and blown with your mouth.)}\example{Borale da eran ttongo ekaeka ma za dan.}{A flute is a thing that makes noise.}}
\entry{boralle}{\headword{boralle}\variant{dial. var. of}{borale}}
\entry{bore}{\headword{bore}\pos{n.}\sensenumber{2}\definition{traditional bamboo pipe for smoking tobacco}}
\entry{bormop}{\headword{bormop}\pos{n.}\sensenumber{2}\definition{type of spear}}
\entry{boser}{\headword{boser}\pos{n.}\sensenumber{2}\definition{rock found in creek}}
\entry{bott}{\headword{bott}\pos{n.}\sensenumber{2}\definition{boat}\etymology{from Englishboat}}
\entry{botta}{\headword{botta}\pos{n.}\sensenumber{2}\definition{lateral beam placed directly on the house post}\example{Ma da botta käme dämenang da.}{The house already has the lateral beam.}}
\entry{Boze}{\headword{Boze}\pos{pn.}\sensenumber{2}\definition{Boze (Agob- and Bine-speaking village in Oriomo-Bituri Rural LLG; from Limol, one must pass through Malam, Kurunti, and Kibuli)}}
\entry{Bozorob}{\headword{Bozorob}\pos{pn.}\sensenumber{2}\definition{Bozorob (camping place on the road to Kurunti; from Limol, one must pass through Malam)}}
\entry{Bozrob}{\headword{Bozrob}\variant{fast speech var. of}{Bozorob}}
\entry{Breton}{\headword{Breton}\pos{pn.}\sensenumber{2}\definition{male personal name}}
\entry{buata}{\headword{buata}\pos{n.}\sensenumber{2}\definition{betel nut, areca nut (fruit of Areca catechu)Kaekep ma dan. (It's for chewing.)}\example{buata däg}{betel nut bunch}}
\entry{budar}{\headword{budar}\pos{n.}\sensenumber{2}\definition{grub, larva}}
\entry{budombudom}{\headword{budombudom}\pos{n.}\sensenumber{2}\definition{red ants}}
\entry{buddo}{\headword{buddo}\pos{n.}\sensenumber{1}\definition{problem, issue}\example{Mälla papa da buddo ulle dan Papua Niyu Gini mi.}{Beating one's wife is a big problem in Papua New Guinea.}\sensenumber{2}\definition{weight}\example{Buddo da tumang da.}{It's very heavy (lit. the weight is a lot).}\sensenumber{1}\definition{heavy}\example{Mälla da ine de buddog de wony eran.}{The woman is carrying the heavy water.}\sensenumber{2}\definition{deep (of a voice)}\sensenumber{3}\definition{to trouble, bother}\example{Ewatta bibi obom buddog eralla?}{Why are you all bothering her?}\sensenumber{3}\definition{carrying a load}\example{Dagwaeya deyarneyo buddobuddog.}{They (du.) were going with their loads.}\sensenumber{3}\definition{nonsingular form of buddo}\etymology{buddo + =ang}\subentry{\headword{buddog}\pos{mod.}\definition{heavy}}\subentry{\headword{buddobuddog}\pos{adv.}\definition{carrying a load}}\subentry{\headword{buddobuddo}\pos{mod.}\definition{nonsingular form of buddo}}}
\entry{bugu}{\headword{bugu}\pos{n.}\sensenumber{3}\definition{sheath, base, midrib (of a palm leaf)}\example{Nge da ddagemeny dan, obo bugu daeben.}{The coconut palm doesn't have branches, there are only its leaf sheaths.}\example{Bugu dädär da aspunan.}{Dry palm leaf sheaths fall.}}
\entry{Buib}{\headword{Buib}\pos{pn.}\sensenumber{3}\definition{Buib (toponym)}}
\entry{buidde}{\headword{buidde}\pos{n.}\sensenumber{3}\definition{club (weapon)Lla gäz ma da. (For beating people.)}}
\entry{Buiddobuiddog}{\headword{Buiddobuiddog}\pos{pn.}\sensenumber{3}\definition{Buiddobuiddog (sago place and creek; also a road (Top L AZ96))}}
\entry{buitu}{\headword{buitu}\pos{n.}\sensenumber{3}\definition{stick with a round baseLla wa ddäddäg gäz ma da. (For beating people and animals.)}}
\entry{buk}{\headword{buk}\pos{n.}\sensenumber{3}\definition{book}\example{buk ulle}{big book}\etymology{from Englishbook}}
\entry{bulwem}{\headword{bulwem}\pos{n.}\sensenumber{3}\definition{type of big yam with a white and light purple interior and no hairs}}
\entry{bullalla}{\headword{bullalla}\pos{n.}\sensenumber{3}\definition{type of flower}}
\entry{bullull}{\headword{bullull}\pos{n.}\sensenumber{1}\definition{Papuan frogmouthIddob me ekawang pa dan. (It's a bird that sings at night.)}\sensenumber{2}\definition{rufous owl}}
\entry{bumo}{\headword{bumo}\pos{n.}\sensenumber{2}\definition{tucker bag}\example{Bogo gopällttänän obo wätät bumo peyang.}{She set off with her filled tucker bag.}}
\entry{bumrel}{\headword{bumrel}\pos{n.}\sensenumber{2}\definition{beetle}}
\entry{bun}{\headword{bun}\pos{n.}\sensenumber{1}\definition{head}\example{Pa bo bun a ddäddäg ma dan.}{The bird's head is not edible.}\sensenumber{2}\definition{part of a bow}\sensenumber{3}\definition{mouth (of a river)}\example{Ngäna walle bun i ibi allan.}{I am going towards the river mouth.}\sensenumber{4}\definition{owner}\example{Ginarang däbaolle eka kutt e bun dan.}{Ginarang is the owner of that word.}\sensenumber{4}\definition{dandruff}\sensenumber{4}\definition{head}\sensenumber{4}\definition{hair (on head)}\sensenumber{4}\definition{skull}\sensenumber{4}\definition{instructor, leaderTtoen täbatäbeag maduma me. (Plans things in the village.)}\example{Lla da mer bun lla dag.}{The men are good leaders.}\sensenumber{4}\definition{smart}\etymology{lit. 'bursted head'}\subentry{\headword{bun dänräp}\pos{n.}\definition{dandruff}}\subentry{\headword{bun käp}\pos{n.}\definition{head}}\subentry{\headword{bun kom}\pos{n.}\definition{hair (on head)}}\subentry{\headword{bun kutt}\pos{n.}\definition{skull}}\subentry{\headword{bun lla}\pos{n.}\definition{instructor, leaderTtoen täbatäbeag maduma me. (Plans things in the village.)}}\subentry{\headword{bun pallkamatt}\pos{mod.}\definition{smart}}}
\entry{bunbun}{\headword{bunbun}\pos{n.}\sensenumber{4}\definition{type of plant}}
\entry{Bundae}{\headword{Bundae}\pos{pn.}\sensenumber{4}\definition{male personal name}}
\entry{bunkälle bunkälle}{\headword{bunkälle bunkälle}\pos{n.}\sensenumber{4}\definition{type of game where the players tie hair and hide}}
\entry{bunkom}{\headword{bunkom}\variant{sp. var. of}{bun kom}}
\entry{bunkom tätäp}{\headword{bunkom tätäp}\pos{n.}\sensenumber{4}\definition{hair tied with string}}
\entry{bunkombunkom}{\headword{bunkombunkom}\pos{n.}\sensenumber{4}\definition{type of tall, big tree that grows along creek with wood used for canoes}\etymology{redup. ofbunkom}}
\entry{bunkutt}{\headword{bunkutt}\variant{sp. var. of}{bun kutt}}
\entry{bunkuttang}{\headword{bunkuttang}\pos{n.}\sensenumber{4}\definition{catfishBun ullong kollba. (A fish with a big head.)}\etymology{bunkutt + =ang, lit.'big-headed'}}
\entry{Bunkuttangmälla}{\headword{Bunkuttangmälla}\pos{pn.}\sensenumber{4}\definition{female personal name}}
\entry{bunmat}{\headword{bunmat}\pos{n.}\sensenumber{4}\definition{center of a garden}}
\entry{bungg}{\headword{bungg}\pos{S vt.}\sensenumber{4}\definition{to ambush}}
\entry{bur}{\headword{bur}\pos{n.}\sensenumber{4}\definition{type of birdIddob me ekawang pa dan. (It's a bird that sings at night.)}}
\entry{burag}{\headword{burag}\pos{n.}\sensenumber{4}\definition{bride price (given to the bride's family by the groom)Mälla mu, sɨmell sisiang a, wätät a ttägäl käp a. (Wife payment: a tame pig, food, and money.)}}
\entry{burara}{\headword{burara}\pos{n.}\sensenumber{4}\definition{water lilyKarama popo, walle me päddabag dan. (A flower in Karama swamp; it grows in the water.)}}
\entry{buwo}{\headword{buwo}\pos{n.}\sensenumber{4}\definition{type of native bananaTupi dan, obo käp a ulleulle dag, ge mabun up da. Obo käp a o me otät ma dan ako kire me yu ma dan. Da lla da ge up de botän suwe de mɨnyi sägäsägäd de bottän ako sägäsägäd de källa de bobewän. Ine da obo mamang abal dan ako tatu we mäkamäke ma dan itrellang att. (It's long, its fruits are big; this is a sacred banana. When ripe, its fruit is eaten, and when unripe, it's cooked. If a person eats this banana, their urine and feces will be yellow. Its liquid is very red and it can be used to wash to avoid illness.)}}
\entry{Buyubun}{\headword{Buyubun}\pos{pn.}\sensenumber{4}\definition{Buyubun (toponym)}}
\entry{buz}{\headword{buz}\pos{n.}\sensenumber{4}\definition{type of cultivated tree with fist-sized, edible green fruit and light purple flowers}}
\entry{Buzi}{\headword{Buzi}\pos{pn.}\sensenumber{4}\definition{Buzi (in Kiwai Rural LLG)}}
\lettersection{D}
\entry{da1}{\headword{da1}\pos{subr.}\sensenumber{1}\definition{if, when (introduces a condition)}\example{Bongo ai dan da däbe melem de nanges, sisri nanges.}{If you can do that job, do it today.}\example{Da Joshua sɨmell de bäbäddän, ibi mɨnyi toto duwem e bäddägeya.}{If Joshua shoots a pig, we will eat it for dinner.}\example{Da Llimoll me lla da kuddäll bogon, komlla o kumuddäga lla da mɨnyi utt alle eka de bängaeyo.}{When a person dies in Limol, two or three people will deliver the news with a conch shell.}\sensenumber{2}\definition{might, may, could (marks potential mood)}\example{Kollokolloe eka panynen a mudan. Da iba eka de bawengameya.}{Speaking a mixed language (i.e. code-switching) is no good. We might forget our (incl.) language.}}
\entry{da2}{\headword{da2}\variant{fr. var. of}{dan}}
\entry{da3}{\headword{da3}\pos{dem.}\sensenumber{1}\definition{that (mesial determiner and pronoun)}\example{da ma me}{in that house}\example{Da dibeya.}{That's that.}\sensenumber{2}\definition{he, she, it (third person singular animate/inanimate pronoun, nominative only)}\example{Mozaya yu nägagalla. Da därko dägagan.}{We grilled the mozaya fish. It got hard.}\example{Da näkäp buddo me adämenan.}{He's sat with a heavy mind.}\sensenumber{3}\definition{last}\example{da pazi me}{last year}\sensenumber{3}\definition{ablative form of da; after that}\sensenumber{3}\definition{ablative form of da; therefore}\example{Ddob Idi lla da dadeg, damasäm Ende eka de klloklloe panypeny erallo.}{There are also Idi speakers, so people are speaking mixed Ende.}\sensenumber{3}\definition{allative form of da}\example{Ibi daolle beyareya.}{Let's (du.) go over there.}\sensenumber{3}\definition{allative form of da with perlative clitic}\example{Yu da daollemae obo pite we gogon.}{The fire got closer to her grass skirt.}\etymology{da + =alle₂}\subentry{\headword{daballe}\pos{adv. dem.}\definition{ablative form of da; after that}}\subentry{\headword{damasäm}\pos{adv. dem.}\definition{ablative form of da; therefore}}\subentry{\headword{daolle}\pos{adv. dem.}\definition{allative form of da}}\subentry{\headword{daollemae}\pos{adv.}\definition{allative form of da with perlative clitic}}}
\entry{=da1}{\headword{=da1}\pos{n. cl.}\sensenumber{1}\definition{nominative clitic (marks the subject or agent of the verb)}\example{Ddia da wa kottllam a dindugmällneyo.}{The deer and the turtle were going to have a race.}\sensenumber{2}\definition{accusative clitic on conjoined objects (marks the object of the verb in conjoined noun phrases)}\example{Ngämi ngäma toboll a bägäl a danserbeaebeya.}{We (excl.) prepared our bows and arrows.}\allomorph{a}}
\entry{=da2}{\headword{=da2}\pos{n. cl.}\sensenumber{1}\definition{close possessive clitic}\example{Ine kutt da bo lid a popang dan.}{The water container's lid has a hole.}\sensenumber{2}\definition{close possessive kinship clitic (marks third person possession on the nominal phrase; can only attach to phrases with kinship nominal heads)}\example{Däräng käsre nagda bom yagnen dängkamän.}{Dog then started to look for his friend.}\allomorph{a}}
\entry{Dabe}{\headword{Dabe}\pos{pn.}\sensenumber{2}\definition{Dabe (toponym)}}
\entry{Dabi}{\headword{Dabi}\pos{pn.}\sensenumber{2}\definition{female personal name}}
\entry{dabit}{\headword{dabit}\pos{n.}\sensenumber{2}\definition{palm spear}}
\entry{dada}{\headword{dada}\pos{kin.}\sensenumber{2}\definition{older sibling of the same sex (man's older brother or woman's older sister)Zaze me ngattong zegatt. (The firstborn in a generation.)}}
\entry{dadargu}{\headword{dadargu}\pos{n.}\sensenumber{2}\definition{disturbance}}
\entry{dadär}{\headword{dadär}\pos{n.}\sensenumber{2}\definition{type of net suspended in the water by sticks}}
\entry{dadäräb}{\headword{dadäräb}\pos{S vt.}\sensenumber{1}\definition{to write}\example{Ngämo kame dan dadräb a.}{I can't write.}\example{Pizin eka de ai dan badräb.}{I can write in Tok Pisin.}\sensenumber{2}\definition{to dress}\example{Bogo obom dadäräb eran.}{She is dressing her.}\sensenumber{3}\definition{to decorate}\example{Bogo mɨnyi llo ttam alle godräballe ingong de ngkamalle bandra peyang.}{She would decorate herself with leaves and start to dance to the song.}\allomorph{däräb}\allomorph{dräb}\allomorph{darbä}\allomorph{darbnan}\allomorph{drab}\allomorph{darb}}
\entry{dadäräd}{\headword{dadäräd}\pos{S vi.}\sensenumber{3}\definition{to clear (e.g. of the mind, sky)}\example{Näkäp a odardan.}{The mind is cleared.}\allomorph{dard}}
\entry{dade1}{\headword{dade1}\pos{n.}\sensenumber{3}\definition{yam stickDdage peyang llo kälekäle ganenatt mätta po dowae me mättayabira kälnan e. (A stick with branches planted near the yams so the yams can climb them.)}\example{Zugu obo mätta dade de era topotopoll dade dae dägawän.}{Zugu only planted topotopoll yam sticks for his yams.}}
\entry{dade2}{\headword{dade2}\variant{fr. var. of}{daden}}
\entry{dade3}{\headword{dade3}\pos{adv.}\sensenumber{1}\definition{maybe}\example{Bogo täbädd lla bälle panya wo, wup wo, o dade wayati käp de däddänalle.}{He used to pick a ripe pineapple, ripe banana, or maybe a watermelon for the strange man.}\sensenumber{2}\definition{ever}\example{dade aya}{whoever}\example{dade erame}{wherever}\example{dade eremde}{whichever}\example{dade angde}{whenever}}
\entry{dadegeyo}{\headword{dadegeyo}\variant{fr. var. of}{dadegaeyo}}
\entry{dadel}{\headword{dadel}\pos{n.}\sensenumber{1}\definition{harvest}\example{dadel ebdo}{harvest day}\sensenumber{2}\definition{season of harvesting young gardens (seventh season; corresponds to early May)}}
\entry{daden}{\headword{daden}\pos{cop.}\sensenumber{2}\definition{to exist, have, there is (present singular form)}\example{Bäne ine da daden?}{Do you have water (lit. does your water exist)?}\sensenumber{2}\definition{past singular form of daden}\example{Bogo tätäm dadaeya ma me?}{Was he home yesterday?}\sensenumber{2}\definition{present plural form of daden}\example{Kemu bo llabun a gänyme a Malläm me dadeg.}{Kemu has relatives here and in Malam (lit. Kemu's relatives exist here and in Malam).}\sensenumber{2}\definition{past plural form of daden}\example{Tätäm ubi kumuddäga dadegaeya ma me?}{Were they three home yesterday?}\sensenumber{2}\definition{present dual form of daden}\example{Buk a komllaebe dadegeyo.}{There are only two books.}\sensenumber{2}\definition{past dual form of daden}\example{Tätäm ubi komlla dadegwaeya ma me}{Were they two home yesterday?}\subentry{\headword{dadaeya}\pos{cop.}\definition{past singular form of daden}}\subentry{\headword{dadeg}\pos{cop.}\definition{present plural form of daden}}\subentry{\headword{dadegaeya}\pos{cop.}\definition{past plural form of daden}}\subentry{\headword{dadegaeyo}\pos{cop.}\definition{present dual form of daden}}\subentry{\headword{dadegwaeya}\pos{cop.}\definition{past dual form of daden}}}
\entry{Dadi}{\headword{Dadi}\pos{pn.}\sensenumber{2}\definition{male personal name}}
\entry{dadi1}{\headword{dadi1}\variant{fr. var. of}{daden}}
\entry{dadi2}{\headword{dadi2}\variant{sp. var. of}{dedi}}
\entry{dadiweya}{\headword{dadiweya}\variant{fr. var. of}{dadaeya}}
\entry{dadräb}{\headword{dadräb}\variant{fast speech var. of}{dadäräb}}
\entry{=dae1}{\headword{=dae1}\variant{var. of}{=daebe}}
\entry{=dae2}{\headword{=dae2}\pos{n. cl.}\sensenumber{2}\definition{perlative case clitic; along, through}\example{Ngäna wälläng ik dae ibi allan.}{I am going through the bush.}\example{Ngäna walle dae ibi allan.}{I am going along the river.}}
\entry{=daebe}{\headword{=daebe}\pos{n. cl.}\sensenumber{2}\definition{restrictive clitic; only}\sensenumber{2}\definition{copular form of daebe (present singular form)}\example{Däbe mälla bo melem daeben.}{That is just women's work.}\sensenumber{2}\definition{present plural form of daeben}\example{Dedme ddone dan be kumuddäga nyäng daebeg.}{There's nothing there but three bags.}\sensenumber{2}\definition{past plural form of daeben}\example{Misdae up daebegaeya ibebatta.}{The only things planted were bananas.}\sensenumber{2}\definition{present dual form of daeben}\example{Ende ttängäm a komlla daebegeyo, Malläm a Llimoll.}{There are only two Ende villages: Malam and Limol.}\sensenumber{2}\definition{past singular form of daeben}\example{Diba kui mi ttongdae ddia gullbe daebeya.}{On that island, there was only one stag.}\subentry{\headword{daeben}\pos{cop.}\definition{copular form of daebe (present singular form)}}\subentry{\headword{daebeg}\pos{cop.}\definition{present plural form of daeben}}\subentry{\headword{daebegaeya}\pos{cop.}\definition{past plural form of daeben}}\subentry{\headword{daebegeyo}\pos{cop.}\definition{present dual form of daeben}}\subentry{\headword{daebeya}\pos{cop.}\definition{past singular form of daeben}}}
\entry{daem}{\headword{daem}\pos{A vi.}\sensenumber{2}\definition{to blink}\example{Ngäna ikop daem agan.}{I blinked.}}
\entry{Daena}{\headword{Daena}\pos{pn.}\sensenumber{2}\definition{female personal name}}
\entry{daendae}{\headword{daendae}\pos{n.}\sensenumber{2}\definition{flowering plant that is said to be ancestral to the area, with many types; flowers used as adornment when dancing}}
\entry{Daeyagmälla}{\headword{Daeyagmälla}\pos{pn.}\sensenumber{2}\definition{female personal name}}
\entry{Daeyna}{\headword{Daeyna}\pos{pn.}\sensenumber{2}\definition{female personal name}}
\entry{daga}{\headword{daga}\pos{n.}\sensenumber{2}\definition{betel}\etymology{from Tok Pisindaka}}
\entry{dagal}{\headword{dagal}\pos{S vi.}\sensenumber{1}\definition{to board, get on}\example{Ngämi deyareya gall e godagaleya.}{We went to the boat and got on.}\example{Ngäna gongäs gall e godagal.}{I returned to the boat and got on.}\sensenumber{2}\definition{to board, load, put on}\example{Ngämo män kälsre de gall e dädagal.}{I put my daughter onto the boat.}\allomorph{dag}\allomorph{deg}\allomorph{dadeg}}
\entry{Daiba}{\headword{Daiba}\pos{pn.}\sensenumber{2}\definition{garden and sago place of Jerry Dareda (along the road to Bisuaka)}}
\entry{daindai}{\headword{daindai}\variant{sp. var. of}{daendae}}
\entry{daindaim}{\headword{daindaim}\pos{n.}\sensenumber{2}\definition{drizzle}}
\entry{dalab}{\headword{dalab}\pos{S vt.}\sensenumber{2}\definition{to open, pierce, make a hole}\example{Obo käp de pop nedelaban.}{He made a hole in the fruit.}\allomorph{dalaeb}\allomorph{delab}\allomorph{delaeb}}
\entry{dale}{\headword{dale}\pos{n.}\sensenumber{2}\definition{ash}}
\entry{dam}{\headword{dam}\pos{adv.}\sensenumber{2}\definition{then}\example{Angde ngäna otaran, dam yogoll agan.}{I was sleeping, and then it rained.}}
\entry{damärärmae}{\headword{damärärmae}\pos{adv.}\sensenumber{2}\definition{simultaneously}\example{Ibi mɨnyi damärärmae toboll daspunigalla.}{We will shoot arrows at the same time.}}
\entry{damona}{\headword{damona}\pos{num.}\sensenumber{2}\definition{1296 (yam counting numeral; 6\textbackslashtextasciicircum4)}\etymology{from a Yam language; compare Nen damno}}
\entry{damong1}{\headword{damong1}\pos{mod.}\sensenumber{2}\definition{healthy, well}\example{damong mätta}{a plump yam}\example{Bongo ngänäm damong nagalle, a sisri ngäna damong dan.}{You comforted me and now I'm okay.}}
\entry{dan}{\headword{dan}\pos{cop.}\sensenumber{2}\definition{to be, exist, is, there is (present singular form)}\example{Ge nyäng a kälsre dan.}{This bag is small.}\example{Warama era mer lla dan.}{Warama is a good man.}\sensenumber{2}\definition{past singular form of dan}\example{Ngämo baba ttongdae mälla peyang daeya.}{My father had (lit. was with) one wife.}\sensenumber{2}\definition{present plural form of dan}\example{Ubi dag.}{They are there.}\example{Ikop a ulleulle dag.}{The eyes are big.}\sensenumber{2}\definition{past plural form of dan}\example{Obo mälläng ik a ulleulle abal dagaeya.}{His nostrils were very big.}\sensenumber{2}\definition{present dual form of dan}\example{Ubi komlla era nag dageyo.}{The two of them are friends.}\sensenumber{2}\definition{past dual form of dan}\example{Däräng a wa Baet a mer abal nag dagwaeya.}{Dog and Cuscus were very good friends.}\subentry{\headword{daeya}\pos{cop.}\definition{past singular form of dan}}\subentry{\headword{dag}\pos{cop.}\definition{present plural form of dan}}\subentry{\headword{dagaeya}\pos{cop.}\definition{past plural form of dan}}\subentry{\headword{dageyo}\pos{cop.}\definition{present dual form of dan}}\subentry{\headword{dagwaeya}\pos{cop.}\definition{past dual form of dan}}}
\entry{danän}{\headword{danän}\variant{dial. var. of}{dan}}
\entry{Daniel}{\headword{Daniel}\pos{pn.}\sensenumber{2}\definition{male personal name}}
\entry{Danipa}{\headword{Danipa}\pos{pn.}\sensenumber{2}\definition{male personal name}}
\entry{dang}{\headword{dang}\pos{n.}\sensenumber{2}\definition{length (of a house)}\example{dang papek}{wall spanning the length of a building}\sensenumber{2}\definition{corner postDang bo llokott. (Supports the lengths.)}\subentry{\headword{dang toto}\pos{n.}\definition{corner postDang bo llokott. (Supports the lengths.)}}}
\entry{dangg}{\headword{dangg}\pos{S vi.}\sensenumber{2}\definition{to burn}\example{Nyongo da andegan.}{The road is burning.}\allomorph{ndeg}\allomorph{ndiminy}\allomorph{dameny}}
\entry{dangkam}{\headword{dangkam}\pos{S vi.}\sensenumber{1}\definition{to lean}\example{Lla da ttongo lla da bo patme dangkam allan.}{The man is leaning on the other man.}\sensenumber{2}\definition{to rely on}\example{Ngäna ngämo mosen patme dangkaemmäll allan ngämingg i.}{I rely on my brother for help.}\allomorph{ndekam}\allomorph{ndekaem}\allomorph{dangkaem}}
\entry{dangkälmang}{\headword{dangkälmang}\pos{n.}\sensenumber{2}\definition{type of medium-sized, long yam with a white interior and thorns}}
\entry{dangne}{\headword{dangne}\pos{n.}\sensenumber{2}\definition{crawling vine that grows in the grassland with purple flowers; used to make rope}\sensenumber{2}\definition{type of grubAp me ddäddäg ma budar dan. (It's an edible grub that lives in the grassland.)}\subentry{\headword{dangnebudar}\pos{n.}\definition{type of grubAp me ddäddäg ma budar dan. (It's an edible grub that lives in the grassland.)}}}
\entry{dape}{\headword{dape}\pos{n.}\sensenumber{2}\definition{drum headWadär alle iatt gazibra ttoe llokottang e alläp ttatt me. (Gazibra snakeskin tied tightly to the drum rim with a string.)}}
\entry{Dara}{\headword{Dara}\pos{pn.}\sensenumber{2}\definition{male personal name}}
\entry{dara1}{\headword{dara1}\pos{n.}\sensenumber{2}\definition{type of traditional medicineKängkäm ma dan nane we käll itrel täräp me. (It's to be squeezed and taken when the spleen is unhealthy.)}}
\entry{dara2}{\headword{dara2}\pos{n.}\sensenumber{2}\definition{vine\textbackslash_type}}
\entry{darab}{\headword{darab}\variant{dial. var. of}{dalab}}
\entry{daradara}{\headword{daradara}\pos{n.}\sensenumber{2}\definition{type of vine with yellow fruit that are red when ripe}}
\entry{daramdaram}{\headword{daramdaram}\pos{adv.}\sensenumber{2}\definition{shining brightly}\example{Yäbäd a daramdaramang gogon.}{The sun was shining brightly.}}
\entry{Dareda}{\headword{Dareda}\pos{pn.}\sensenumber{2}\definition{male personal name}}
\entry{darkukiny}{\headword{darkukiny}\pos{n.}\sensenumber{2}\definition{type of grass}}
\entry{darombe}{\headword{darombe}\pos{n.}\sensenumber{2}\definition{mouth harp. quarter-moon-shaped bamboo flute with honey inside; takes a day to makePattlle alle nganges att dan, bod alle eka ma dan llatet att kab peyang. (It's made with pattlle bamboo; it's to be played with your mouth with a twisted string.)}}
\entry{Darren}{\headword{Darren}\pos{pn.}\sensenumber{2}\definition{male personal name}}
\entry{Daru}{\headword{Daru}\pos{pn.}\sensenumber{1}\definition{Daru (capital city of Western Province, located on Daru Island)}\sensenumber{2}\definition{Daru Island}}
\entry{daudau}{\headword{daudau}\pos{A vi.}\sensenumber{2}\definition{to nod}}
\entry{dauma}{\headword{dauma}\pos{n.}\sensenumber{2}\definition{type of introduced bananaTupi pänyanzag dan, ulle dan obo pätt a, ako käp a obo ulleulle dag. Käp a obo kire meae yu ma dag. (It grows long and its trunk is big, and the fruit are big. When unripe, its fruit are cooked.)}}
\entry{David}{\headword{David}\variant{sp. var. of}{Deibid}}
\entry{däba1}{\headword{däba1}\pos{n.}\sensenumber{2}\definition{type of tree that grows in the grassland with leaves used to wrap sago and durable wood used for kundu drums, house posts, and formerly, bridgesKokllo att de kaen ma dan, yu me ttänttämang e a ngoe alle ddäddäg alle ngoe ddäddäg me. (The bark is for wrapping food scraps, heating on the fire, and for chewing when with a toothache.)}}
\entry{däba2}{\headword{däba2}\pos{dem.}\sensenumber{2}\definition{that (mesial determiner)}\example{Däba ngättäma we gotarän.}{He slept in that place.}\sensenumber{2}\definition{allative form of däba}\example{Däbaolle ikop e abällan.}{We went there to see.}\sensenumber{2}\definition{copular form of däba (present singular form)}\example{Sisri ngäna pasta dan. Ngämo melem a sisri däban.}{Now I am a pastor. That is my job.}\sensenumber{2}\definition{past singular form of däban}\example{Däbaeya ngämo ttoenttoen a.}{That was my little story.}\example{Däbe.}{That was it. / The end.}\sensenumber{2}\definition{present plural form of däban}\example{Maten bo llɨg a ngämo kokok a däbag.}{Martin's children, they are my grandchildren.}\sensenumber{2}\definition{past plural form of däban}\example{Däbagaeya ngämo bin a.}{Those were my names.}\sensenumber{2}\definition{past dual form of däban}\example{Kurupel a Täm, ngämo baba bi dibagweya.}{Kurupel and Tam, they were my parents.}\sensenumber{2}\definition{ablative form of däba}\example{Däbamasem a, ako ttongo källäm me gobäll.}{From there, we went to another pond.}\example{Däbamatt a ngämi gall deyagllaenalla.}{After that, we paddled the canoe over.}\sensenumber{2}\definition{accusative form of däba}\example{Tämamae llɨg klekle da däbem llo de kälnan erallo.}{All the children are climbing that tree.}\etymology{däba + =masäm}\subentry{\headword{däbaolle}\pos{adv.}\definition{allative form of däba}}\subentry{\headword{däban}\pos{cop.}\definition{copular form of däba (present singular form)}}\subentry{\headword{däbaeya}\pos{cop.}\definition{past singular form of däban}}\subentry{\headword{däbag}\pos{cop.}\definition{present plural form of däban}}\subentry{\headword{däbagaeya}\pos{cop.}\definition{past plural form of däban}}\subentry{\headword{däbagwaeya}\pos{cop.}\definition{past dual form of däban}}\subentry{\headword{däbamasem}\pos{adv. dem.}\definition{ablative form of däba}}\subentry{\headword{däbem}\pos{nom. dem.}\definition{accusative form of däba}}}
\entry{däba3}{\headword{däba3}\pos{S vt.}\sensenumber{2}\definition{to place, put}\example{Ngämo toboll de dädäbaneg llo mit mi.}{I placed my spears at the base of the tree.}}
\entry{däbae}{\headword{däbae}\pos{S vi.}\sensenumber{1}\definition{to drizzle}\example{Sisri abal yogoll a adäbaeyan.}{It just started to drizzle.}\sensenumber{2}\definition{to knock fruit continuously}\allomorph{däbaenen}\allomorph{dbae}\allomorph{ndbae}}
\entry{däbamattäm}{\headword{däbamattäm}\variant{var. of}{däbamasem}}
\entry{däbäll}{\headword{däbäll}\pos{S vt.}\sensenumber{2}\definition{to touch}\example{Nädbällnegan.}{He touched him.}\example{Nädbällnegan}{He touched three people at different times.}\example{Lla da ubim yadbällan.}{He touched three people at the same time.}\allomorph{dbäll}\allomorph{däbull}}
\entry{däbe1}{\headword{däbe1}\variant{fr. var. of}{däbaeya}}
\entry{däbe2}{\headword{däbe2}\variant{var. of}{däba2}}
\entry{däbi}{\headword{däbi}\pos{n.}\sensenumber{2}\definition{green-backed honeyeater}}
\entry{dädäk}{\headword{dädäk}\pos{n.}\sensenumber{2}\definition{wall}}
\entry{dädär1}{\headword{dädär1}\pos{n.}\sensenumber{1}\definition{stone, rock}\example{Ngängälatt dädär a mänyi tutu atta bolldaeyän kopek e.}{A round rock will roll from the hill to the valley.}\sensenumber{2}\definition{dry}\example{Ttängäm ulle da tämamae dallän dädär abal gogon.}{All the big villages became very dry.}\sensenumber{3}\definition{hard}\example{Bogo dɨdɨr bun peyang dan.}{He has a hard head (i.e. he is stubborn).}\sensenumber{3}\definition{stone}\example{dädär käp turik}{stone ax}\subentry{\headword{dädär käp}\pos{n.}\definition{stone}}}
\entry{dädär2}{\headword{dädär2}\pos{n.}\sensenumber{3}\definition{type of big taro}}
\entry{dädäräb}{\headword{dädäräb}\pos{S vt.}\sensenumber{3}\definition{to cut grass}\allomorph{däräb}\allomorph{däräbnan}}
\entry{dädri}{\headword{dädri}\variant{var. of}{didri}}
\entry{däen}{\headword{däen}\pos{n.}\sensenumber{3}\definition{type of snakeBätbät gullem ulle da. Ddäddäg ma da. (It's a big, black snake. It's edible.)}}
\entry{däg}{\headword{däg}\pos{n.}\sensenumber{3}\definition{hand, group, bunch, set}\example{buata däg}{betelnut bunch}\example{manggo däg}{bunch of mango}\example{nge däg}{coconut bunch}\example{tämamae däg}{an entire set (e.g. of teeth)}}
\entry{dägadäga}{\headword{dägadäga}\pos{adv.}\sensenumber{3}\definition{completely}\example{Ngäna ngämo sana de ot dägadäga dättemän.}{I finished eating all my sago.}}
\entry{dägädägäl}{\headword{dägädägäl}\variant{sp. var. of}{dägäldägäl}}
\entry{dägäldägäl}{\headword{dägäldägäl}\pos{n.}\sensenumber{3}\definition{intestines}\example{Dägäldägäl a ddäddäg ma dan.}{The intenstines are edible.}}
\entry{dägmar}{\headword{dägmar}\pos{n.}\sensenumber{1}\definition{tongue}\example{Dägmar a malla ddäddäg ma dan.}{The tongue is not edible.}\sensenumber{2}\definition{spoon}\example{Dägmar alle wätät de notnan kakoll att.}{He was eating food from a plate with a spoon.}}
\entry{däkäna}{\headword{däkäna}\variant{sp. var. of}{däkna}}
\entry{däkna}{\headword{däkna}\pos{n.}\sensenumber{2}\definition{small black termite mound that burns for a long time; after a woman gives birth, it is heated in the fire, wrapped in bark and cloth, placed under a mat, and used to warm the woman's stomach}}
\entry{däl}{\headword{däl}\variant{sp. var. of}{dɨl}}
\entry{däm}{\headword{däm}\pos{n.}\sensenumber{2}\definition{plant}\example{Bogo gärep kabkab däm de dibewän diba ttängäm me.}{He planted grape plants in that place.}}
\entry{däm ibenen}{\headword{däm ibenen}\pos{n.}\sensenumber{2}\definition{season when crops are planted (fifteenth season; corresponds to late November)}}
\entry{dämar}{\headword{dämar}\pos{n.}\sensenumber{2}\definition{type of palm tree (\textbackslashtextasciitilde2 m) that used to be cooked and eaten; also fed to pigsDu wätät dan ap me. Utt ddaddu att de ankom peyang otnan ma dan umettäp koep me. (A wild palm in the grassland. The shoots are removed and the sour ants inside are eaten.)}}
\entry{dämädämäll}{\headword{dämädämäll}\pos{mod.}\sensenumber{2}\definition{numb, paralyzed}\example{apte pätt dämädämällang lla}{man with one half of his body paralyzed}}
\entry{dämbaemeny}{\headword{dämbaemeny}\pos{S vi.}\sensenumber{2}\definition{to doze off}\example{Dagirnän e: yunu gondäbaemenynän.}{and stayed there for a long, long time until he dozed off.}\allomorph{ndäbae}\allomorph{ndäbaemeny}\allomorph{ndbae}\allomorph{ndbaemeny}}
\entry{dämbag}{\headword{dämbag}\pos{n.}\sensenumber{2}\definition{lazy person, weak person}}
\entry{dämen}{\headword{dämen}\pos{S vi.}\sensenumber{2}\definition{to sitKum alle ekaklle me dämenang giddoll o ttongdae ngättma me giddoll. (Sitting down on the ground, or staying in one place.)}\example{Ngäna dämenang dan.}{I am sitting.}\example{Adämeneyo!}{(You two) sit down!}\example{Ttongo lla da dämenma toko me adämenan.}{A person sat on top of the chair.}\sensenumber{1}\definition{causative-applicative form of dämen}\example{Nändämenanggan.}{He sat him down.}\sensenumber{2}\definition{applicative form of dämen}\example{Ngäna Kate bom täräb ma me dändämenang.}{I sat down with Kate at the funeral.}\allomorph{dmen}\allomorph{dma}\allomorph{däma}\allomorph{dämadäme}\allomorph{dme}\allomorph{ndämen}\allomorph{ndäma}\allomorph{dm}\allomorph{däm}\allomorph{dman}\etymology{dämen + -ngg}\subentry{\headword{dämenangg}\pos{S vt.}\definition{causative-applicative form of dämen}}}
\entry{dämoe}{\headword{dämoe}\pos{S vt.}\sensenumber{1}\definition{to push}\example{Ngäna abo däbe sɨmell de misdae ada dandämoe.}{Then I just pushed the pig like that.}\example{Angde obom ttu wi dandämoeyän, bogo ddone dängllawän, be bogo gonddämän.}{When she pushed him into the deep, he didn't swim: rather, he drowned.}\sensenumber{2}\definition{to send}\example{Mälla de dandmoeaebeya wayati kongkom e.}{We sent the women to fetch the watermelons.}\example{Giniya män kälsre de nandämoeyan Bodog pate.}{Giniya sent the girl to Bodog.}\sensenumber{2}\definition{causative-applicative form of dämoe}\example{Ngäna bam sɨmell ikop e dandämoengg.}{I sent you to see the pig.}\allomorph{ndmoe}\allomorph{ndämoe}\etymology{dämoe + -ngg}\subentry{\headword{dämoengg}\pos{S vt.}\definition{causative-applicative form of dämoe}}}
\entry{dänäk}{\headword{dänäk}\pos{n.}\sensenumber{2}\definition{type of small bush with reddish fruit that is black when ripe}}
\entry{dändak}{\headword{dändak}\pos{n.}\sensenumber{2}\definition{type of purple yam with purple skin and no thorns or hairs}}
\entry{dändär1}{\headword{dändär1}\pos{S vt.}\sensenumber{1}\definition{to hear, listen}\example{Ngäna obom dandär.}{I heard him.}\example{Ngämne eka de andär!}{[You] listen to my words!}\sensenumber{2}\definition{to sense, feel}\example{Ngäna otät molle de dändär eran.}{I smell food (lit. sense the smell of food).}\example{Ngämo matta me tätäp dändär eran.}{I feel pain in my shoulder.}\sensenumber{3}\definition{listen}\allomorph{ndär}\allomorph{därmäll}}
\entry{dändär2}{\headword{dändär2}\pos{S vt.}\sensenumber{3}\definition{to stuff}\example{Mänmän a sana de zazaba we dändäraebeyo.}{The girls stuffed the sago into the bag.}\sensenumber{3}\definition{bagful of sago}\allomorph{ndär}\allomorph{därmäll}\etymology{dändär + =att}\subentry{\headword{dändäratt}\pos{n.}\definition{bagful of sago}}}
\entry{dändäräm}{\headword{dändäräm}\pos{n.}\sensenumber{3}\definition{type of small tree with white and purple flowers and many hard, marble-sized, green fruit that children play with}\sensenumber{3}\definition{type of marble game}\subentry{\headword{dändäräm käp}\pos{n.}\definition{type of marble game}}}
\entry{dändäräp}{\headword{dändäräp}\variant{dial. var. of}{dänräp}}
\entry{dändärek}{\headword{dändärek}\pos{S vt.}\sensenumber{3}\definition{to control, influence, rule, govern}\example{Bogo mɨnyi llɨg de gagäll ttoen ngasnges e bändärekän.}{He will pressure the boy to do bad things.}\allomorph{ndärek}}
\entry{dändrek}{\headword{dändrek}\variant{sp. var. of}{dändärek}}
\entry{dänräp}{\headword{dänräp}\pos{n.}\sensenumber{1}\definition{fish scale}\sensenumber{2}\definition{scab}}
\entry{dängam}{\headword{dängam}\pos{n.}\sensenumber{2}\definition{Blyth's hornbilLlo ik me giddollag pa ulle dan. (It's a large bird that lives inside trees.)}}
\entry{dänyäk}{\headword{dänyäk}\pos{n.}\sensenumber{2}\definition{small plant that grows in the grassland with purple and white flowers and blue fruit that children like to eat}}
\entry{där}{\headword{där}\pos{n.}\sensenumber{1}\definition{pair}\sensenumber{2}\definition{to match, balance, agree}\example{Mätta de där yag.}{Match these yams (i.e. bring the same number).}\example{Yuwog abal lla da Yesu bo pallall kuki ikopatt eka de dällätaemneyo, be oba kuki eka da ddone där gognegnän.}{Many people were making false testimonies about Jesus, but their lies didn't add up.}\sensenumber{2}\definition{in pairs}\example{Ubi mänmän de därdäragae dällädnegeyo.}{They get married with women in pairs.}\subentry{\headword{därdärag}\pos{adv.}\definition{in pairs}}}
\entry{Därall}{\headword{Därall}\pos{pn.}\sensenumber{2}\definition{Därall (toponym)}}
\entry{däräng}{\headword{däräng}\pos{n.}\sensenumber{2}\definition{dog}\example{Däräng bo llan a tubutubu dag.}{The dog's ears are short.}}
\entry{däräng olleolle}{\headword{däräng olleolle}\pos{n.}\sensenumber{2}\definition{type of possum-like animal with spotted skin}}
\entry{därängbun}{\headword{därängbun}\pos{n.}\sensenumber{2}\definition{type of pandanus with a curved fruit shaped like a dog's headWätät ma mab dan, ulle käpang dan. (It's an edible pandanus with large fruit.)}\etymology{däräng + bun, lit. 'dog head'}}
\entry{därängg}{\headword{därängg}\pos{S vt.}\sensenumber{2}\definition{to lead}\example{Mälla da obom ttängäm att de ngattong dändrägän menae e.}{The woman led him outside the village.}\allomorph{ndräg}\allomorph{dändäräg}\allomorph{ndaräg}}
\entry{Därängge}{\headword{Därängge}\pos{pn.}\sensenumber{2}\definition{male personal name}}
\entry{därängge}{\headword{därängge}\pos{n.}\sensenumber{2}\definition{small orchid with blue, yellow, white, purple flowersNyäng, pite inen ma dan. (It's for weaving bags and skirts.)}}
\entry{däränggedärängge}{\headword{däränggedärängge}\pos{n.}\sensenumber{2}\definition{large wild orchid}\etymology{redup. of därängge}}
\entry{därba}{\headword{därba}\pos{n.}\sensenumber{2}\definition{type of snakeTubutubu da. Lla ddäddägang da. (It's long. It bites people.)}}
\entry{därmir1}{\headword{därmir1}\pos{n.}\sensenumber{2}\definition{type of introduced bananaTupi pänyanzag dan, obo däg a yuwog dag, ako käp a obo tutupi kanong dag. O me käp a obo otät ma da ako binzenen ma dag yu mi, a kire da yu ma dag. Ada pätt a obo tärpän ma dan popel e ankom peyang kaekep e. Pag peyang mer mokowang dan. (It grows big; its bunches are plentiful, and its fruit are a bit long. When ripe, its fruit is eaten and heated on the fire, and when unripe, it's cooked. The stem is cut for popel and chewed with ants. It's tasty with salt.)}}
\entry{därmir2}{\headword{därmir2}\pos{n.}\sensenumber{2}\definition{type of tree that is used to treat sores}}
\entry{däroledärole}{\headword{däroledärole}\pos{mod.}\sensenumber{2}\definition{dry}\example{Da bongo sana däroledärole de därunggu alle yu nägag, mɨnyi mer abal bogon.}{If you cook the dry sago with the bamboo tubes, it will be very good.}}
\entry{därollog}{\headword{därollog}\pos{n.}\sensenumber{2}\definition{brolgaKulläb ik me giddollag pa dan. (It's a bird that lives in big termite mounds.)}}
\entry{därunggu}{\headword{därunggu}\pos{n.}\sensenumber{2}\definition{bamboo tubePattlle sisor erem de sana zazer e tärpän erallo a yu mi wandawandae dättämalle. (They cut young bamboo, which is filled with sago and rolled on the fire to cook.)}\example{Da bongo sana däroledärole de därunggu alle yu nägag, mɨnyi mer abal bogon.}{If you cook the dry sago in the new bamboo, it will be very good.}}
\entry{de1}{\headword{de1}\pos{dem.}\sensenumber{1}\definition{that (mesial determiner)}\example{de llɨg kälsre}{that small boy}\sensenumber{2}\definition{there (mesial)}\example{Käsre de amne me dantämonän, llo gäba me.}{Then, he waited there in the middle, in the shade of the tree.}\sensenumber{3}\definition{next}\example{de sände me}{next week}\sensenumber{3}\definition{there}\example{Dedo otät a ddone dan?}{Isn't there any food?}\example{Ngäna dedo melem gogne.}{I was working there.}\sensenumber{3}\definition{there}\sensenumber{3}\definition{then, at that time}\example{Skul täräp a dedam ddone deya.}{There was no schooltime back then.}\example{Angde nyongo amne we gogaebne, yogoll a dedam disamän.}{When we were halfway through the journey, that's when the rain stopped.}\sensenumber{3}\definition{there}\example{Dedme ddone dan be kumuddäga nyäng daebeg.}{There's nothing there except for three bags.}\example{Ge llɨg a dedme gunziwän.}{This boy settled there.}\subentry{\headword{dedo}\pos{dem.}\definition{there}}\subentry{\headword{dädme}\pos{adv. dem.}\definition{there}}\subentry{\headword{dedam}\pos{adv. dem.}\definition{then, at that time}}\subentry{\headword{dedme}\pos{adv. dem.}\definition{there}}}
\entry{de2}{\headword{de2}\variant{fast speech var. of}{daeya}}
\entry{de3}{\headword{de3}\variant{var. of}{da1}}
\entry{=de}{\headword{=de}\pos{n. cl.}\sensenumber{1}\definition{accusative clitic}\example{Ngämi nge de dibenyeya.}{We planted the coconut.}\sensenumber{2}\definition{argument focus marker}\example{Ddia da angde yinu kuddäll gogon, kottllam a de deyanzigän.}{He had been really dead asleep, and turtle had passed him!}\allomorph{di}}
\entry{deba}{\headword{deba}\variant{sp. var. of}{diba}}
\entry{debag}{\headword{debag}\variant{sp. var. of}{dibag}}
\entry{Deboa}{\headword{Deboa}\pos{pn.}\sensenumber{2}\definition{male personal name}}
\entry{dedi}{\headword{dedi}\pos{kin.}\sensenumber{2}\definition{daddy}\etymology{from Englishdaddy}}
\entry{dedre}{\headword{dedre}\pos{S vi.}\sensenumber{2}\definition{to descend, go down}\example{Ubi tutu atta walle we godrenegnän.}{They were descending from the mountain to the river.}\allomorph{dre}\allomorph{dernan}}
\entry{Deibid}{\headword{Deibid}\pos{pn.}\sensenumber{2}\definition{male personal name}}
\entry{dektta}{\headword{dektta}\pos{n.}\sensenumber{2}\definition{doctor}\example{Ubi mɨnyi dektta we moko bognegän.}{They will want a doctor.}\etymology{from Englishdoctor}}
\entry{del}{\headword{del}\pos{n.}\sensenumber{2}\definition{coconut lorikeetLlo ik me giddollag pa dan. (It's a bird that lives in trees.)}}
\entry{Delema}{\headword{Delema}\pos{pn.}\sensenumber{2}\definition{Delema (toponym)}\allomorph{Dele}}
\entry{dem}{\headword{dem}\pos{n.}\sensenumber{2}\definition{type of sago used for paintsTtongo tubutubu ngällngällang sana dan. (It's a sago that bears long fruit.)}}
\entry{deng}{\headword{deng}\variant{baby talk var. of}{däräng}}
\entry{deodeo}{\headword{deodeo}\pos{n.}\sensenumber{2}\definition{termite}}
\entry{derägmäll}{\headword{derägmäll}\pos{S vt.}\sensenumber{2}\definition{to rebuke, scold}\example{Yesu ubim dädergmällnegän ubim adawatta oba tikop a llokollokott dagaeya.}{Jesus rebuked them (i.e. his disciples) for the hardness of their hearts.}\allomorph{dergmäll}}
\entry{Derideri}{\headword{Derideri}\pos{pn.}\sensenumber{2}\definition{Derideri (Nambo-speaking village in Morehead Rural LLG; east of Morehead)}}
\entry{Dettall}{\headword{Dettall}\pos{pn.}\sensenumber{2}\definition{Dettall (toponym)}}
\entry{Dewara}{\headword{Dewara}\pos{pn.}\sensenumber{2}\definition{Dewara (Were/Kiunum-speaking village in Gogodala Rural LLG, along the Fly River; from Limol, one must pass through Upiara and Kondobol)}}
\entry{diaba}{\headword{diaba}\pos{n.}\sensenumber{2}\definition{type of spear}}
\entry{Diandra}{\headword{Diandra}\pos{pn.}\sensenumber{2}\definition{female personal name}}
\entry{diba}{\headword{diba}\pos{dem.}\sensenumber{2}\definition{that (mesial determiner)}\example{Bongo Saken bo amamär de ewede diba llo me nganae eralle?}{Why are you coiling Saken's rope around that tree?}\sensenumber{2}\definition{copular form of diba (present singular form)}\example{Kak Wagiba diban.}{There's Grandmother Wagiba.}\sensenumber{2}\definition{plural present form of diban}\example{Ngämo kokok a dibag.}{Those are my grandchildren.}\sensenumber{2}\definition{past plural form of diban}\example{Llɨg a dibagaeya.}{Those were the boys.}\sensenumber{2}\definition{present dual form of diban}\example{Karea a Duya, ngämo märäll lla da dibageyo.}{Karea and Duya, they are my age-mates.}\sensenumber{2}\definition{past dual form of diban}\example{Ngämo baba bi dibagwaeya.}{Those were my parents.}\sensenumber{2}\definition{ablative form of diba; then, thereupon}\example{Mällause da dagirnän, dibaballe goeg a dättämän.}{The old woman was staying; thereafter, the garden burned.}\sensenumber{2}\definition{ablative form of diba; because}\example{Dibamasäma, dallän do Yop e.}{From there, she went to Yop.}\example{Kollokolloe panypeny erallo, dibamasäma ngämi erag kollokolloe dag gänyme.}{They are speaking mixed (i.e. code-switching), because we (incl.) are mixed here.}\sensenumber{2}\definition{allative form of diba}\example{Ngäna dibaolle melem e gozen.}{I entered that job.}\sensenumber{2}\definition{accusative form of diba}\example{Bogo dibem llo de kälnan eran.}{She climbs that tree.}\etymology{diba + =alle₂}\subentry{\headword{diban2}\pos{cop.}\definition{copular form of diba (present singular form)}}\subentry{\headword{dibag}\pos{cop.}\definition{plural present form of diban}}\subentry{\headword{dibagaeya}\pos{cop.}\definition{past plural form of diban}}\subentry{\headword{dibageyo}\pos{cop.}\definition{present dual form of diban}}\subentry{\headword{dibagwaeya}\pos{cop.}\definition{past dual form of diban}}\subentry{\headword{dibaballe}\pos{adv. dem.}\definition{ablative form of diba; then, thereupon}}\subentry{\headword{dibamasäma}\pos{adv. dem.}\definition{ablative form of diba; because}}\subentry{\headword{dibaolle}\pos{adv. dem.}\definition{allative form of diba}}\subentry{\headword{dibem}\pos{nom. dem.}\definition{accusative form of diba}}}
\entry{dibaeya}{\headword{dibaeya}\pos{cop.}\sensenumber{2}\definition{cop.that.pst.sgS}}
\entry{dibamattäm}{\headword{dibamattäm}\variant{var. of}{dibamasäma}}
\entry{diban1}{\headword{diban1}\pos{quant.}\sensenumber{1}\definition{another, young people now say däba.}\sensenumber{2}\definition{there}\sensenumber{3}\definition{that one}}
\entry{dibaya}{\headword{dibaya}\variant{fr. var. of}{dibaeya}}
\entry{dibeya}{\headword{dibeya}\variant{fr. var. of}{dibaeya}}
\entry{dibie}{\headword{dibie}\pos{n.}\sensenumber{3}\definition{spectacled longbill}}
\entry{Dibllag}{\headword{Dibllag}\pos{pn.}\sensenumber{3}\definition{personal name}}
\entry{Dibllagmälla}{\headword{Dibllagmälla}\pos{pn.}\sensenumber{3}\definition{female personal name}}
\entry{Dibor}{\headword{Dibor}\pos{pn.}\sensenumber{3}\definition{female personal name}}
\entry{diboz}{\headword{diboz}\pos{n.}\sensenumber{3}\definition{type of bird}}
\entry{didiri}{\headword{didiri}\variant{sp. var. of}{didri}}
\entry{didri}{\headword{didri}\pos{adv. dem.}\sensenumber{3}\definition{there}\example{Da ngäna sawe nyongo de bongkollmäll, ngaska didri lla da ddone dan.}{If I follow the left road, maybe there won't be any people there.}\sensenumber{3}\definition{ablative form of didri}\example{Ngämi didribatt dag Boze att dag.}{We (excl.) are from there, from Boze.}\etymology{didri + =att}\subentry{\headword{didribatt}\pos{adv. dem.}\definition{ablative form of didri}}}
\entry{Didroe}{\headword{Didroe}\pos{pn.}\sensenumber{3}\definition{male personal name}}
\entry{Dieb}{\headword{Dieb}\pos{pn.}\sensenumber{3}\definition{male personal name}}
\entry{Diendra}{\headword{Diendra}\pos{pn.}\sensenumber{3}\definition{female personal name}}
\entry{Digabo Källäm}{\headword{Digabo Källäm}\pos{pn.}\sensenumber{3}\definition{Digabo Pond (canoe and fishing place in Taolang; road to Taolang is in AX95)}}
\entry{digodigol}{\headword{digodigol}\pos{n.}\sensenumber{3}\definition{type of tree}\etymology{partial redup. of digol}}
\entry{digol}{\headword{digol}\pos{n.}\sensenumber{3}\definition{type of big, tall tree that grows near gardens in the bush with white flowers, green fruits, and special, valuable red wood that is used for ax handles}}
\entry{Dikae}{\headword{Dikae}\pos{pn.}\sensenumber{3}\definition{male personal name}}
\entry{Dikai}{\headword{Dikai}\variant{sp. var. of}{Dikae}}
\entry{Dikiboe}{\headword{Dikiboe}\pos{pn.}\sensenumber{3}\definition{personal name}}
\entry{Dikullowang}{\headword{Dikullowang}\pos{pn.}\sensenumber{3}\definition{Dikullowang (small island and hunting place in Taolang)}}
\entry{dikun}{\headword{dikun}\pos{n.}\sensenumber{3}\definition{deacon}\etymology{from Englishdeacon}}
\entry{dimes}{\headword{dimes}\pos{n.}\sensenumber{3}\definition{type of cultivated tree with sour, mango-like fruit}}
\entry{Dimgi}{\headword{Dimgi}\pos{pn.}\sensenumber{3}\definition{Dimgi (toponym)}}
\entry{Dimiri}{\headword{Dimiri}\pos{pn.}\sensenumber{3}\definition{Dimiri/Demeri (Idi-speaking village in Morehead Rural LLG; from Limol, one must pass through Kuiwang)}}
\entry{Dimisisi}{\headword{Dimisisi}\pos{pn.}\sensenumber{3}\definition{Dimisisi (Idi-speaking village in Morehead Rural LLG; from Limol, one must pass through Kinkin and Bok)}}
\entry{Dimoe}{\headword{Dimoe}\pos{pn.}\sensenumber{3}\definition{female personal name}}
\entry{Dimoi}{\headword{Dimoi}\variant{var. of}{Dimoe}}
\entry{Dimsis}{\headword{Dimsis}\variant{var. of}{Dimisisi}}
\entry{Dimsisi}{\headword{Dimsisi}\variant{var. of}{Dimisisi}}
\entry{Dimson}{\headword{Dimson}\pos{pn.}\sensenumber{3}\definition{male personal name}}
\entry{dindu}{\headword{dindu}\pos{S vi.}\sensenumber{1}\definition{to run, flee, escapeMängamängal abal ttäle spall. (Moving legs very quickly.)}\example{Ngäna dindu allan.}{I am running.}\example{Ddia da walle we nindugan.}{The deer ran to the water.}\example{Bong angde bigma we dallän, sɨmell a dindugän.}{When Bong went to the pen, the pig had fled.}\example{Obo pate bizmollne.}{We will be running to her.}\sensenumber{2}\definition{race}\example{Oba abo dindu da dädme llätt gogon.}{Then, their race stopped there.}\allomorph{ndug}\allomorph{dindudindu}\allomorph{zmoll}\allomorph{wɨzmoll}\allomorph{indug}\allomorph{izmoll}}
\entry{dini}{\headword{dini}\pos{n.}\sensenumber{2}\definition{type of small tree that grows in the bush with white flowers and red fruit that cassowaries like to eat}}
\entry{dinidini}{\headword{dinidini}\pos{n.}\sensenumber{2}\definition{type of small tree that grows in the bush with red fruit}\etymology{redup. of dini}}
\entry{dinggel}{\headword{dinggel}\pos{n.}\sensenumber{2}\definition{sugar glider}}
\entry{dinggi}{\headword{dinggi}\pos{n.}\sensenumber{2}\definition{dinghy}\etymology{from Englishdinghy}}
\entry{dinggoll}{\headword{dinggoll}\pos{n.}\sensenumber{2}\definition{opossum}}
\entry{Dipa}{\headword{Dipa}\pos{pn.}\sensenumber{2}\definition{male personal name}}
\entry{dirindi1}{\headword{dirindi1}\pos{n.}\sensenumber{2}\definition{type of large yam with a white interior and thorns; with or without hairs}}
\entry{dirindi2}{\headword{dirindi2}\pos{n.}\sensenumber{2}\definition{type of tree}}
\entry{dirom}{\headword{dirom}\pos{n.}\sensenumber{2}\definition{southern cassowaryWälläng me giddollag ddäddäg ulle dan. (It's a large game animal that lives in the bush.)}\example{Dirom a ttongo mer mokowang ddäddäg dan.}{Cassowary is a very delicious animal.}\sensenumber{2}\definition{cassowary egg}\sensenumber{2}\definition{to menstruate for the first time}\etymology{lit. 'kill the cassowary'}\subentry{\headword{dirom käp1}\pos{n.}\definition{cassowary egg}}\subentry{\headword{dirom gäz}\pos{S vt.}\definition{to menstruate for the first time}}}
\entry{dirom käp2}{\headword{dirom käp2}\pos{n.}\sensenumber{2}\definition{type of small taro}\etymology{lit. 'cassowary egg'}}
\entry{dirom mas}{\headword{dirom mas}\pos{n.}\sensenumber{2}\definition{Cassowary fibula, the smaller leg bone that is next to the larger leg bone.}}
\entry{diromdirom}{\headword{diromdirom}\pos{n.}\sensenumber{2}\definition{type of small tree with round, flat, edible fruit that are green and red when ripe}\etymology{redup. of dirom}}
\entry{distrik}{\headword{distrik}\pos{n.}\sensenumber{2}\definition{district}\etymology{from Englishdistrict}}
\entry{district}{\headword{district}\variant{sp. var. of}{distrik}}
\entry{dit}{\headword{dit}\pos{n.}\sensenumber{2}\definition{type of cane used for building houses, bows, and canoes}}
\entry{Diwa}{\headword{Diwa}\pos{pn.}\sensenumber{2}\definition{male personal name}}
\entry{dɨbeya}{\headword{dɨbeya}\variant{fr. var. of}{däbaeya}}
\entry{dɨdɨr}{\headword{dɨdɨr}\variant{var. of}{dädär1}}
\entry{dɨl}{\headword{dɨl}\pos{mod.}\sensenumber{2}\definition{bitter; sour}}
\entry{do1}{\headword{do1}\pos{adv. dem.}\sensenumber{1}\definition{over there (distal)}\example{Ngämo za da dag do Pauma bo patme.}{My things are there at Pauma's.}\sensenumber{2}\definition{to, until}\example{Ngämi däpllätt abo do Taolang gall tapma.}{Then we (excl.) started walking to Taolang canoe place.}\example{Ako polle kättnan gongkaemallo do dättemänaeballo.}{They started to build the fences until they were finished.}\example{iddob amnong alle do bädab}{from midnight to dawn}\sensenumber{2}\definition{there}\example{Ngämi dodo dagirni.}{We (excl.) were living there.}\sensenumber{2}\definition{ablative form of do}\example{Obo masar a dowattäm da Kulläntti.}{Her ancestor is from there, Kurunti.}\etymology{do + =mattäm}\subentry{\headword{dodo2}\pos{adv. dem.}\definition{there}}\subentry{\headword{dowattäm}\pos{mod.}\definition{ablative form of do}}}
\entry{do2}{\headword{do2}\pos{n.}\sensenumber{1}\definition{handle}\example{Dägmar a do peyang dan.}{The spoon has a handle.}\sensenumber{2}\definition{femur}}
\entry{Dobola}{\headword{Dobola}\pos{pn.}\sensenumber{2}\definition{male personal name}}
\entry{dodro1}{\headword{dodro1}\pos{S vi.}\sensenumber{1}\definition{to slip}\example{Nikki bo kaptte da dronendronen dag.}{Nikki's clothes are slipping off.}\sensenumber{2}\definition{to die}\example{Kollba da tämamae dadrowän a be kottllam aebe ttam dagirnän.}{The fish all died but the turtle lived on.}\allomorph{duro}\allomorph{dro}\allomorph{dronen}\allomorph{dru}}
\entry{dodro2}{\headword{dodro2}\pos{S vt.}\sensenumber{2}\definition{to clean}\example{Bogo llan de komlla deyadurowän mermerae.}{She washed her two ears well.}\allomorph{duro}\allomorph{dro}\allomorph{dronen}\allomorph{dru}}
\entry{dogma}{\headword{dogma}\pos{n.}\sensenumber{2}\definition{type of tree that grows in the bush with wood that smells like matches and is good for house posts}}
\entry{Dokoe}{\headword{Dokoe}\pos{pn.}\sensenumber{2}\definition{male personal name}}
\entry{Doli}{\headword{Doli}\pos{pn.}\sensenumber{2}\definition{male personal name}}
\entry{dom}{\headword{dom}\pos{A vt.}\sensenumber{2}\definition{to clench (one's fist)}\example{Dom nägagan ttang de.}{He clenched his fist.}}
\entry{domäll}{\headword{domäll}\pos{n.}\sensenumber{1}\definition{type of pandanusWälläng wallemäg me päddabag dan. (It grows in the bush.)}\sensenumber{2}\definition{old-style sewn mat made of domäll pandanus}}
\entry{dombak}{\headword{dombak}\variant{var. of}{dompak}}
\entry{domoe}{\headword{domoe}\variant{dial. var. of}{dämoe}}
\entry{dompa}{\headword{dompa}\pos{n.}\sensenumber{1}\definition{type of blunt arrowToboll e llo popoatt pa gäddnan e. (Wood sharpened into a spear to kill birds)}\example{Ge baba bäne digol dompa popoatt dan.}{Father made this dompa out of digol tree.}\sensenumber{2}\definition{penis (slang)}}
\entry{dompadompa}{\headword{dompadompa}\pos{n.}\sensenumber{2}\definition{type of spear}\etymology{redup. of dompa}}
\entry{dompak}{\headword{dompak}\pos{n.}\sensenumber{2}\definition{eelKollba dan be gullem ingoll dan. (It's a fish but it's like a snake.)}}
\entry{Donae}{\headword{Donae}\pos{pn.}\sensenumber{2}\definition{female personal name}}
\entry{Donai}{\headword{Donai}\variant{sp. var. of}{Donae}}
\entry{Donsi}{\headword{Donsi}\pos{pn.}\sensenumber{2}\definition{female personal name}}
\entry{dongkal}{\headword{dongkal}\pos{S vi.}\sensenumber{2}\definition{to stop, end}\example{Wällän a dädär me gondokalmällnegän.}{The roots ended on the rock.}\allomorph{ndokal}\allomorph{ndokalmäll}\allomorph{dongkalmäll}}
\entry{dongkäral}{\headword{dongkäral}\pos{n.}\sensenumber{2}\definition{type of lizard}}
\entry{dongki}{\headword{dongki}\pos{n.}\sensenumber{2}\definition{donkey}\etymology{from Englishdonkey}}
\entry{dor}{\headword{dor}\pos{n.}\sensenumber{2}\definition{stalk}\example{Sana dor de adingoll däpittaemeya.}{We weaved the sago stalks like this.}}
\entry{Dore}{\headword{Dore}\pos{pn.}\sensenumber{2}\definition{female personal name}}
\entry{Dorin}{\headword{Dorin}\pos{pn.}\sensenumber{2}\definition{female personal name}}
\entry{dorko}{\headword{dorko}\pos{mod.}\sensenumber{2}\definition{dry}\example{sana dorko}{dry sago}\example{Ekaklle da dorko dan.}{The ground is dry.}}
\entry{dorllog}{\headword{dorllog}\pos{n.}\sensenumber{2}\definition{rufous-bellied kookaburra}}
\entry{doros}{\headword{doros}\pos{n.}\sensenumber{2}\definition{pants}\etymology{from Englishdrawers}}
\entry{Doumori}{\headword{Doumori}\pos{pn.}\sensenumber{2}\definition{Doumori (in Kiwai Rural LLG)}}
\entry{dowa}{\headword{dowa}\pos{n.}\sensenumber{2}\definition{type of tree that grows in the grassland along creeks with wood used for firewoodAmtet itrel me kängkäm ma dan nane we. (When it's hard to breathe, it's to be squeezed and smoked.)}}
\entry{Dowabunang}{\headword{Dowabunang}\pos{pn.}\sensenumber{2}\definition{camping, sago, hunting place, and garden of Kaoga Dobola (on the road to Kinkin AZ94)}}
\entry{dowae}{\headword{dowae}\pos{loc.}\sensenumber{2}\definition{vicinity, proximity}\example{Ngäna ibi allan wup dowae e.}{I am walking towards the banana.}\example{Grace obo zegatt ebdo dowae me dan.}{Grace's birthday is soon.}\sensenumber{2}\definition{close together, next to each other, neighboring, adjacent}\example{ekaklle me dämadäme dowadowae}{sitting on the ground next to each other}\sensenumber{2}\definition{straight}\example{Käsre ada gognegän doweae Bundae bälle yagyagag.}{Then they went like this straight to find Bundae.}\allomorph{dowe}\etymology{from dowae + =ae₂}\subentry{\headword{dowadowae}\pos{adv.}\definition{close together, next to each other, neighboring, adjacent}}\subentry{\headword{doweae}\pos{adv.}\definition{straight}}}
\entry{Dowan}{\headword{Dowan}\pos{pn.}\sensenumber{2}\definition{name of a mountain}}
\entry{dradre1}{\headword{dradre1}\pos{n.}\sensenumber{2}\definition{type of tree that grows in swamp with edible, currant-sized blue fruit}}
\entry{dradre2}{\headword{dradre2}\pos{S vt.}\sensenumber{2}\definition{to dress}\example{Ngäna obom dädre.}{I dressed him.}\allomorph{dre}\allomorph{dra}\allomorph{dära}\allomorph{däre}}
\entry{droledrole}{\headword{droledrole}\variant{fr. var. of}{däroledärole}}
\entry{druwem}{\headword{druwem}\pos{A vt.}\sensenumber{2}\definition{to knock on}\example{Ngäna ud de druwem nägawan.}{I knocked on the door.}}
\entry{du}{\headword{du}\pos{mod.}\sensenumber{2}\definition{wild (of plants)}}
\entry{du kyakya}{\headword{du kyakya}\pos{n.}\sensenumber{2}\definition{hook-billed kingfisher}}
\entry{duab}{\headword{duab}\pos{S vt.}\sensenumber{2}\definition{to knock over; blow down}\example{Däräng a ine de naduaban.}{The dog knocked the water over.}\example{Llo de duduaibnegnän a dattkaemnegän.}{Wind blew down the trees and broke them.}\allomorph{du}\allomorph{duaeb}}
\entry{Duaba}{\headword{Duaba}\pos{pn.}\sensenumber{2}\definition{Duaba (Gogodala-speaking village in Gogodala Rural LLG)}}
\entry{dubllodubllom}{\headword{dubllodubllom}\pos{n.}\sensenumber{2}\definition{type of tree that grows in the bush with yellow fruit that have a hard seed, in which there is an edible nut}}
\entry{Duboläpläp}{\headword{Duboläpläp}\variant{var. of}{Dubolläplläp}}
\entry{Dubolläplläp}{\headword{Dubolläplläp}\pos{pn.}\sensenumber{2}\definition{Dubolläplläp (big hill and camping place on the road to Karama swamp; had houses on top in 2015)}}
\entry{Dubollopllop}{\headword{Dubollopllop}\variant{var. of}{Dubolläplläp}}
\entry{dudli}{\headword{dudli}\variant{fr. var. of}{duduli}}
\entry{Dugal}{\headword{Dugal}\pos{pn.}\sensenumber{2}\definition{male personal name}}
\entry{Dugi}{\headword{Dugi}\pos{pn.}\sensenumber{2}\definition{male personal name}}
\entry{dugo}{\headword{dugo}\pos{n.}\sensenumber{2}\definition{type of birdSanawang me ekawang pa dan. (It's a bird that sings in sago places.)}}
\entry{Duiya}{\headword{Duiya}\pos{pn.}\sensenumber{2}\definition{male personal name}}
\entry{Duks}{\headword{Duks}\pos{pn.}\sensenumber{2}\definition{male personal name}}
\entry{Dukumiang}{\headword{Dukumiang}\pos{pn.}\sensenumber{2}\definition{Dukumiang (fishing, sago, hunting place, and garden; camping place of Bewag Bewag; R side of AY96, take southward road; northward road goes to Kapal)}}
\entry{duli}{\headword{duli}\pos{adv. dem.}\sensenumber{1}\definition{over there (distal)}\example{Obo ma da duli dan kona me dan.}{His house is over there on the corner.}\example{Duli dan ddob ttängäm me dan.}{She is over there, somewhere else.}\sensenumber{2}\definition{away (from a place towards another direction)}\example{Ine da däbe llɨg kälsre de dätramän duli.}{The water carried the little boy away.}\example{Duli abäll bibi.}{You all, go away.}\sensenumber{2}\definition{that way}\example{Llɨg a duduli ibi allan.}{The boy is going that way.}\sensenumber{2}\definition{ablative form of duli}\example{Ubi duliduli gobällän a ako duliballe ngäsmäll eran.}{They had migrated far away and are now returning from there.}\sensenumber{2}\definition{ablative form of duli}\example{Obo masar a dulibatta deyarän gänyaolle.}{His grandfather came here from over there.}\etymology{duli + =alle₂}\subentry{\headword{duduli}\pos{adv.}\definition{that way}}\subentry{\headword{duliballe}\pos{adv.}\definition{ablative form of duli}}\subentry{\headword{dulibattäm}\pos{adv. dem.}\definition{ablative form of duli}}}
\entry{Dum}{\headword{Dum}\pos{pn.}\sensenumber{2}\definition{Dum (big hill on the road to Malam; Bamboo and Zarma creek)}}
\entry{dum1}{\headword{dum1}\pos{n.}\sensenumber{2}\definition{type of big tree that grows in bush with strong wood used for making canoes and paddles, sap used to paint bowstrings or to patch holes, yellow flowers, and green fruit; planted near house for shade}}
\entry{dum2}{\headword{dum2}\pos{n.}\sensenumber{2}\definition{placenta}}
\entry{dum3}{\headword{dum3}\pos{n.}\sensenumber{2}\definition{width (of a house)}\example{dum papek}{wall spanning the width of a building}\sensenumber{2}\definition{to surround}\example{Lla gul ulle da obom dandumeyo.}{The large crowd surrounded him.}\subentry{\headword{dumdum}\pos{S vt.}\definition{to surround}}}
\entry{Dum Tutu}{\headword{Dum Tutu}\pos{pn.}\sensenumber{2}\definition{Dum Mountain}}
\entry{dumbi}{\headword{dumbi}\pos{n.}\sensenumber{2}\definition{type of red treeNyeny ingoll llo dan awe me be obo ttam a kälekäle dag. Ada ingoll ngasnges ma dan ge moll ttam de alla ingollang ngasnen eralla tämamae pätt ttattlle me. (It's like the nyeny tree in the savannah, but its leaves are small. It's used like how leaves of the moll tree are used when any part of the body is in pain.)}}
\entry{Dumoll}{\headword{Dumoll}\pos{pn.}\sensenumber{2}\definition{Dumoll (Taolang side garden and camping place of Gidu Jerry, Kols Baewa, and Wareya Giniya)}}
\entry{dun}{\headword{dun}\pos{ideo.}\sensenumber{2}\definition{sound of a drum}}
\entry{dundu kllamen}{\headword{dundu kllamen}\pos{n.}\sensenumber{2}\definition{type of game involving a race}}
\entry{duny}{\headword{duny}\pos{n.}\sensenumber{2}\definition{beetle}}
\entry{dupi}{\headword{dupi}\pos{n.}\sensenumber{2}\definition{stomach}}
\entry{dur}{\headword{dur}\pos{n.}\sensenumber{2}\definition{type of medium-sized bamboo that grows along creeks; used for dancing; young plants used for cooking sago}}
\entry{durgu}{\headword{durgu}\pos{n.}\sensenumber{2}\definition{cliff}}
\entry{Duwaba}{\headword{Duwaba}\pos{pn.}\sensenumber{2}\definition{Duaba (in Gogodala Rural LLG)}}
\entry{duwel}{\headword{duwel}\pos{n.}\sensenumber{2}\definition{type of tall tree that grows in the bush near yam gardens}}
\entry{duwel sära}{\headword{duwel sära}\pos{n.}\sensenumber{1}\definition{type of sago bundle wrapped in sago leaves}\sensenumber{2}\definition{the third stage of sago growth in which the leaves are shorter and the pith is almost ready to be harvested}\example{Sana da duwelsärawang gogon.}{They all ate.}}
\entry{duwem}{\headword{duwem}\pos{A vi. \textbackslash& vt.}\sensenumber{1}\definition{to eat}\example{Ubi tämamae duwem gognegän.}{They all ate.}\example{Duwem bägaeya.}{We ate it.}\sensenumber{2}\definition{food, meal}\example{Ada gogon, "Ddone, ge ade ngämo da yäbdo duwem e a toto duwem e."}{‎He said, "No, this is for my lunch and for my dinner."}\example{yäbdo duwem}{lunch}\example{toto duwem}{dinner}\sensenumber{2}\definition{feast, fellowship meal}\example{Bangeseya ibi duwemduwem de.}{We (incl.) will make a feast.}\subentry{\headword{duwemduwem}\pos{n.}\definition{feast, fellowship meal}}}
\entry{duwie ku}{\headword{duwie ku}\pos{n.}\sensenumber{2}\definition{type of big purple yam}}
\entry{Duya}{\headword{Duya}\variant{sp. var. of}{Duiya}}
\lettersection{Dd dd}
\entry{ddadd}{\headword{ddadd}\pos{n.}\sensenumber{2}\definition{type of large tree with white flowers and blue fruit; wood used for making canoes}}
\entry{ddaddällɨg}{\headword{ddaddällɨg}\pos{S vt.}\sensenumber{2}\definition{to destroy}\example{Kuddäll lla da bäne ma de däddellɨgallo.}{They destroy the house of the dead man.}\allomorph{ddellɨg}\allomorph{ddallɨgnen}\allomorph{ddellg}}
\entry{ddaddellɨg}{\headword{ddaddellɨg}\variant{var. of}{ddaddällɨg}}
\entry{ddaddlläg}{\headword{ddaddlläg}\variant{sp. var. of}{ddaddällɨg}}
\entry{ddaddu}{\headword{ddaddu}\pos{S vt.}\sensenumber{2}\definition{to remove a shoot}\allomorph{ddu}\allomorph{ddanun}}
\entry{ddaebän}{\headword{ddaebän}\pos{S vt.}\sensenumber{1}\definition{to divorce}\example{Ngäna mɨnyi mälla de bäddebän.}{I will divorce my wife.}\sensenumber{2}\definition{to adopt}\example{Ngäna Jepet bom ngämlle llɨg de däddebän.}{I adopted Jepet as my son.}\allomorph{ddaebnen}\allomorph{ddebän}\allomorph{ddeb}\allomorph{ddaeb}}
\entry{ddage}{\headword{ddage}\pos{n.}\sensenumber{1}\definition{branch}\example{Kollko ddage me godämenän.}{He sat on the breadfruit tree branch.}\sensenumber{2}\definition{stream, tributary}\example{Ngämi däbe ddage de danttogeya.}{We (excl.) followed that stream.}\sensenumber{2}\definition{river mouth}\example{ddage bun e}{downstream}\sensenumber{2}\definition{river source}\example{ddage llätt e}{upstream}\sensenumber{1}\definition{the fifth stage of sago growth in which the inflorescence has emerged}\example{Bäne sana da zäme ddageddage gogon.}{Your sago palm is already forming flowers.}\sensenumber{2}\definition{with many branches}\example{Ge manggo da ddageddage dan.}{That mango tree has too many branches.}\subentry{\headword{ddage bun}\pos{n.}\definition{river mouth}}\subentry{\headword{ddage llätt}\pos{n.}\definition{river source}}\subentry{\headword{ddageddage}\pos{n.}\definition{the fifth stage of sago growth in which the inflorescence has emerged}}}
\entry{ddallwe}{\headword{ddallwe}\pos{n.}\sensenumber{2}\definition{type of tree}}
\entry{ddamba}{\headword{ddamba}\pos{n.}\sensenumber{1}\definition{wing}\sensenumber{2}\definition{pectoral fin}\sensenumber{2}\definition{metathorax}\subentry{\headword{ddamba mit}\pos{n.}\definition{metathorax}}}
\entry{ddangoe}{\headword{ddangoe}\pos{S vt.}\sensenumber{2}\definition{to force to do}\example{Andrew Yuga bom era tongoe e daddengoenän.}{Andrew was forcing Yuga to play a game.}\allomorph{ddengoe}\allomorph{ddangoenen}}
\entry{ddangol}{\headword{ddangol}\pos{n.}\sensenumber{2}\definition{type of spear}}
\entry{ddapall}{\headword{ddapall}\pos{n.}\sensenumber{1}\definition{sky}\example{Ddapall a siremang gogon.}{The sky got dark.}\sensenumber{2}\definition{heaven}\example{Ddapall e ibi allan.}{I am going to heaven.}\sensenumber{2}\definition{cloudPällämpälläm o säresärem. (White or dark.)}\example{Yäbäd de ddapall käkan da dakonewän.}{Clouds covered the sun.}\sensenumber{2}\definition{cloud}\sensenumber{2}\definition{heaven}\example{Ddapall ma da kili ma ttängäm dan.}{Heaven is a place of happiness.}\etymology{lit. 'sky poop'}\subentry{\headword{ddapall käkan}\pos{n.}\definition{cloudPällämpälläm o säresärem. (White or dark.)}}\subentry{\headword{ddapall källa}\pos{n.}\definition{cloud}}\subentry{\headword{ddapall ma}\pos{n.}\definition{heaven}}}
\entry{ddäb}{\headword{ddäb}\pos{n.}\sensenumber{2}\definition{anus}}
\entry{ddäddäbeabag}{\headword{ddäddäbeabag}\pos{mod.}\sensenumber{2}\definition{independent, resourceful}}
\entry{ddäddäg1}{\headword{ddäddäg1}\pos{n.}\sensenumber{1}\definition{edible animal, game, meat}\example{Bogo ddäddäg de dokonggän oblle.}{He cut the meat for her.}\sensenumber{2}\definition{to eat meat; bite}\example{Ddob ddägnan ma gullem a dadeg.}{There are some edible snakes.}\example{Ge ddia de däbänya wa däddägaebeya.}{We cut this deer and ate it.}\example{Llɨg a ddäddäg de näddägan.}{The boy ate the meat.}\example{Käza da ge ttang me daddägän ngänäm.}{The crocodile bit me in this hand.}\example{Da bam kungge da naddägän, bongo mɨnyi ttattlle ampug.}{If a spider bites you, you will become very ill.}\sensenumber{3}\definition{bite (of an animal)}\example{Ddob llayabira komo ddäddäg a malla ttällanenang dan.}{For some people, the centiepede's bite isn't painful.}\sensenumber{4}\definition{hunger for meat}\example{Ngämim ddäddäg abal da deyagnän.}{We (excl.) were very hungry for meat (lit. hunger got us).}\sensenumber{4}\definition{leather, hide, animal skin}\example{ddäddäg ttoe nyäng}{leather bag}\allomorph{ddäg}\allomorph{ddägnan}\subentry{\headword{ddäddäg ttoe}\pos{n.}\definition{leather, hide, animal skin}}}
\entry{ddäddäg2}{\headword{ddäddäg2}\pos{S vt.}\sensenumber{4}\definition{to peel, remove}\example{Ngäna ma papek de näddägaban.}{I removed the walls.}\allomorph{ddäg}\allomorph{ddägnan}}
\entry{ddäddäg3}{\headword{ddäddäg3}\pos{S vt.}\sensenumber{4}\definition{to pain, ache, hurt}\example{Ge ute da naddägnan ngänäm.}{This sore is hurting me.}\allomorph{ddäg}\allomorph{ddägnan}}
\entry{ddäddäg4}{\headword{ddäddäg4}\pos{S vi.}\sensenumber{1}\definition{to look, watch}\sensenumber{2}\definition{to look, watch}\allomorph{ddäg}\allomorph{ddägnan}}
\entry{ddäddäg5}{\headword{ddäddäg5}\pos{S vt.}\sensenumber{2}\definition{to bind}\example{Sana dor däpittaemeya a däddägaemeya.}{We wove and bound the sago stalks.}\allomorph{ddäg}\allomorph{ddägnan}}
\entry{ddäddäl}{\headword{ddäddäl}\pos{S vt.}\sensenumber{2}\definition{to shove}}
\entry{ddäddäll}{\headword{ddäddäll}\pos{n.}\sensenumber{2}\definition{thunder}}
\entry{ddäg}{\headword{ddäg}\pos{n.}\sensenumber{1}\definition{back}\example{Ddia ddäg de nällɨt.}{[You] cut the back of the deer.}\sensenumber{2}\definition{outside}\example{Bongo polle ddäg me nagrine.}{[You] stay outside the fence.}\sensenumber{2}\definition{backbone, spine}\sensenumber{2}\definition{later, after}\example{ttongo pazi ddägatt}{one year later}\example{au ddägattalle}{after the burial}\sensenumber{2}\definition{from behind}\example{Bogo ddägddäg deyarän.}{He came from behind.}\etymology{ddäg + =att + =alle₂}\subentry{\headword{ddäg kutt}\pos{n.}\definition{backbone, spine}}\subentry{\headword{ddägattalle}\pos{adv.}\definition{later, after}}\subentry{\headword{ddägddäg}\pos{adv.}\definition{from behind}}}
\entry{ddägaddäge}{\headword{ddägaddäge}\pos{S vt.}\sensenumber{1}\definition{to write}\example{Amo ingoll anykeanyke da bin peyang goddgenegän gänya mani me?}{Whose likeness and name are inscribed on this coin?}\sensenumber{2}\definition{to branch out}\allomorph{ddge}\allomorph{ddäge}}
\entry{ddägatt}{\headword{ddägatt}\variant{fast speech var. of}{ddägattalle}}
\entry{ddägnan ma}{\headword{ddägnan ma}\pos{mod.}\sensenumber{2}\definition{edible}}
\entry{ddäkop}{\headword{ddäkop}\variant{var. of}{ddokop}}
\entry{ddäll1}{\headword{ddäll1}\pos{n.}\sensenumber{1}\definition{chest}\example{Llɨg kälsre da mälla bo ddäll me dan.}{The child is on the woman's chest.}\sensenumber{2}\definition{part of the sago trunk closest to the leaves before the shoot}\sensenumber{3}\definition{ten (lit. chest; body counting numeral)}\sensenumber{3}\definition{chest hair}\sensenumber{3}\definition{sternum, breastbone}\subentry{\headword{ddäll kom}\pos{n.}\definition{chest hair}}\subentry{\headword{ddäll kutt}\pos{n.}\definition{sternum, breastbone}}}
\entry{ddäll2}{\headword{ddäll2}\pos{S vi.}\sensenumber{3}\definition{to arrive}\example{Ubi ddob ttängäm atta guddällmamän.}{They arrived from another village.}\allomorph{ddull}\allomorph{nddäll}}
\entry{ddälläb}{\headword{ddälläb}\pos{S vi.}\sensenumber{3}\definition{to fall over (of a tree)}\example{Llo käkäp a goddälläbnalle.}{Half of the tree fell over.}}
\entry{ddällgoe}{\headword{ddällgoe}\pos{S vt.}\sensenumber{1}\definition{to go through thick bush in search of animals}\example{Lla da däräng peyang tawa de daddällgoeneyo.}{The man perused the swamp together with the dog.}\sensenumber{2}\definition{to disturb a beehive}}
\entry{ddällombog}{\headword{ddällombog}\pos{S vt.}\sensenumber{2}\definition{to miss}\example{Bogo nyongo de nanddällbogan.}{He missed the road.}\example{Bogo angde tuk i nallnan mo de nanddällbogan ttongo.}{She missed a step going up the ladder.}\allomorph{ddällombomeny}\allomorph{nddällbog}\allomorph{nddällbomeny}}
\entry{ddällpoyampoyam}{\headword{ddällpoyampoyam}\pos{n.}\sensenumber{2}\definition{type of mushroomAp me päddabag, malla ddäddäg ma dan. (It grows in the savannah; it's not edible.)}}
\entry{ddäma}{\headword{ddäma}\pos{n.}\sensenumber{1}\definition{basketLlɨg mapät komnen ma nyäng llo ttoe alle iatt. (A bag woven from tree bark to carry babies.)}\example{Naomi bo llɨg a ddäma me dan angällbänan.}{Naomi's baby woke up in the baby basket.}\sensenumber{2}\definition{pouch of a marsupial}\sensenumber{3}\definition{uterus}\sensenumber{3}\definition{birth payment made to the maternal uncle}\example{Ngämi ngäma pope aba pate zime ddäma mu de dangeseya.}{We (excl.) already paid our uncles the basket payment.}\subentry{\headword{ddäma mu}\pos{n.}\definition{birth payment made to the maternal uncle}}}
\entry{Ddämir}{\headword{Ddämir}\pos{pn.}\sensenumber{3}\definition{Ddamir (toponym)}}
\entry{ddämoemkäp kuibiag}{\headword{ddämoemkäp kuibiag}\pos{n.}\sensenumber{3}\definition{type of pythonBätbät da, lla kuddäll e ddäddägang da. (It's black and bites people to death.)}}
\entry{ddän}{\headword{ddän}\pos{S vt.}\sensenumber{3}\definition{to pick, gather, harvest}\example{Dädme ddob wayati de däddaebeya.}{We harvested some watermelons there.}\allomorph{dd}}
\entry{ddänddäl}{\headword{ddänddäl}\pos{S vi.}\sensenumber{3}\definition{to climb}\example{Baet a llo ddage me gonddälän.}{The cuscus climbed on the tree branch.}\allomorph{nddäl}\allomorph{ddälnan}\allomorph{ddäl}}
\entry{ddänddäm}{\headword{ddänddäm}\pos{S vi.}\sensenumber{1}\definition{to drown}\example{Obom dandämoyän ttu we, angde obom ttu we dandämoyän bogo ddone dängllawän, be bogo gonddämän.}{She pushed him into the deep, and when she pushed him into the deep, he didn't swim; he drowned.}\example{Kauga angde Matthew bom ikop dägagän ddänddäm me, bogo mängalae gogbänän walle we, Matthew bom dirängänän.}{When Kauga saw Matthew drowning, he quickly jumped into the water and lifted Matthew out.}\sensenumber{2}\definition{to set (of the moon)}\example{Kok a de ddänddäm allan.}{The moon is setting.}\sensenumber{3}\definition{to drown, sink}\example{Kollba da ddone dan, ede enda nett de nänddäman?}{There are no fish, so what sank the net?}\sensenumber{4}\definition{to worry}\example{Bogo ereya gagäll de bällnan eran. Diba ttoen da obo näkäp de ddänddäm eran.}{She's remembering bad news. This thing makes her worry.}\sensenumber{5}\definition{sink}\allomorph{nddäm}\allomorph{ddämnen}}
\entry{ddänddängeny}{\headword{ddänddängeny}\pos{adv.}\sensenumber{5}\definition{immediately}\example{Ngäna ddänddängeny me gogäbän gall atta.}{I jumped out of the canoe immediately.}}
\entry{ddänmäll}{\headword{ddänmäll}\pos{S vt.}\sensenumber{5}\definition{to struggle}\example{Ngäna gall bllablle we däddänmällneg.}{I struggled to find the canoe.}\allomorph{ddänmällneg}\allomorph{ddänmällaem}}
\entry{ddängall}{\headword{ddängall}\pos{n.}\sensenumber{5}\definition{type of stinging bee found in trees}}
\entry{ddänggab}{\headword{ddänggab}\pos{S vt.}\sensenumber{5}\definition{to hold, grab (with one's teeth)}\example{Sɨmell a ngoi alle ere ada dänddgabän.}{The pig held it with its teeth.}\allomorph{nddgab}\allomorph{nddg}}
\entry{ddänggaddängge}{\headword{ddänggaddängge}\pos{S vt.}\sensenumber{1}\definition{to crucify}\example{Yesu bom llo tärpamatt toko me danddgeyo.}{They crucified Jesus on a wooden cross.}\sensenumber{2}\definition{to catch, trap}\example{Martin käza de danddgewän.}{Martin trapped the crocodile.}\allomorph{nddge}\allomorph{ddänganen}\allomorph{nddäge}}
\entry{ddel}{\headword{ddel}\pos{S vt.}\sensenumber{2}\definition{to explain}\example{Obo kollmällang aba pate bogo tämamae eka midd de däddelnegnän.}{He was explaining all the meanings to his disciples.}}
\entry{Ddele}{\headword{Ddele}\pos{pn.}\sensenumber{2}\definition{Ddele (toponym)}\sensenumber{2}\definition{Ende dialect spoken in Ddele}\etymology{Ddele + =ang}\subentry{\headword{Ddeleag}\pos{pn.}\definition{Ende dialect spoken in Ddele}}}
\entry{Ddelema}{\headword{Ddelema}\pos{pn.}\sensenumber{2}\definition{male personal name}}
\entry{ddia}{\headword{ddia}\pos{n.}\sensenumber{2}\definition{deer}\example{Ddia da dindu allan.}{The deer is running.}\etymology{from Englishdeer}}
\entry{ddob}{\headword{ddob}\pos{quant.}\sensenumber{1}\definition{some}\example{Ngämira ddob kollba de deyasiyu.}{They gave us (excl.) some fish.}\sensenumber{2}\definition{other}\example{Ako ddob kemibi ttoen a dadeg ngasnen ma da.}{There are also many other things to do.}\sensenumber{2}\definition{some, others}\example{Ddobag a Ende eka de panypeny erallo, ddobag a kllokloe panypeny erallo.}{Some are speaking Ende; others are speaking a mix.}\etymology{ddob + =ang}\subentry{\headword{ddobag}\pos{pron.}\definition{some, others}}}
\entry{ddobae}{\headword{ddobae}\pos{adv.}\sensenumber{2}\definition{very}\example{ddobae mangallang lla}{very strong person}\example{ddobae melemang lla}{very hardworking person}\sensenumber{2}\definition{very, extremely}\example{Bundae ddobaeddobae mikutt gogon.}{Bundae was very, very angry.}\etymology{ddob + =ae₂, lit. 'particularly'}\subentry{\headword{ddobaeddobae}\pos{adv.}\definition{very, extremely}}}
\entry{ddoddllem}{\headword{ddoddllem}\variant{fast speech var. of}{ddoddollem}}
\entry{ddoddollem}{\headword{ddoddollem}\pos{A vi. \textbackslash& vt.}\sensenumber{2}\definition{to make noise}\example{Bogo auma me do wattällang a endagaeya däbem ddoddollem dägnegän.}{He used the those things left at the grave to make noise.}\allomorph{ddollem}}
\entry{ddoga}{\headword{ddoga}\pos{n.}\sensenumber{2}\definition{type of tree}}
\entry{ddogllop}{\headword{ddogllop}\variant{fast speech var. of}{ddogollop}}
\entry{ddogoll}{\headword{ddogoll}\pos{S vt.}\sensenumber{1}\definition{to put together}\example{Angde tine pittnen a gottamänän, dibaballe ddäganen de dängkam.}{After the sago leaf weaving finished, I started putting together the roof.}\example{Lla da päre abo gobällan kottllam bo gollob pallkepallke de dänglläbeyo a kottllam bo patme däddäganeyo.}{Then the people sadly went to the turtle, collected all the pieces of his shell, and put him back together.}\sensenumber{2}\definition{to stick}\example{Tongle da addgollan.}{The leech is stuck.}\sensenumber{3}\definition{to join}\example{Ngäna goddägol.}{I joined.}\sensenumber{4}\definition{to lean}\example{Llo pätt me ddogollag dan.}{It's leaning on the tree trunk.}\allomorph{ddägan}\allomorph{ddäganen}\allomorph{ddäga}}
\entry{ddogollop}{\headword{ddogollop}\pos{n.}\sensenumber{4}\definition{reptile scale}}
\entry{ddokddok1}{\headword{ddokddok1}\pos{n.}\sensenumber{4}\definition{type of spear}}
\entry{ddokddok2}{\headword{ddokddok2}\pos{mod.}\sensenumber{4}\definition{blunt}\example{Tätäm ibik a ddokddok gogon.}{Yesterday, the digging stick got blunt.}}
\entry{ddokop}{\headword{ddokop}\pos{n.}\sensenumber{4}\definition{kidney}}
\entry{ddol}{\headword{ddol}\pos{n.}\sensenumber{4}\definition{foam, bubbles, gas}\example{Ttongo lla da ddone mullae dan sisor gärep kae ine de gudne nyäng e banzämaeyän, adawatta gärep kae ine da mɨnyi ddol bogon a nyäng de bungolltän.}{A man cannot pour new wine into an old bag, because the wine will become gassy and it will make the bag explode.}\sensenumber{4}\definition{foamy, bubbly, gassy}\etymology{ddol + =ang}\subentry{\headword{ddolag}\pos{mod.}\definition{foamy, bubbly, gassy}}}
\entry{ddollombog}{\headword{ddollombog}\pos{S vt.}\sensenumber{4}\definition{to misspeak, speak with mistakes}\example{Tame eka de da ngäna bäpany, da erame ngäna banddollbog, ede mäzi ngänäm ttättle bageyo.}{When I speak Taeme, whenever I make mistakes, they will correct me.}\allomorph{nddllobog}\allomorph{nddollbog}}
\entry{ddonddo}{\headword{ddonddo}\pos{mod.}\sensenumber{1}\definition{proud}\example{Ngämi bam ge ddobae ddonddo nalla.}{We (excl.) are very proud of you for this.}\sensenumber{2}\definition{to boast, brag}\example{Ubi oba mängall de gonddoneyo.}{They were boasting about their strength.}\allomorph{nddo}\allomorph{nddomam}\allomorph{ddonen}\allomorph{ddo}}
\entry{ddone}{\headword{ddone}\pos{interj.}\sensenumber{1}\definition{no}\example{― Bäne kokok a daden? ― Ddone.}{― Do you have grandchildren? ― No.}\example{Be bogo ddone allan.}{But he said no.}\sensenumber{2}\definition{not}\example{Da bongo ddone kandärmang ekag, bäne ikop a säremang bognegnän.}{If you don't say sorry, you will go blind.}\sensenumber{2}\definition{nothing}\example{Ddone any a lel gogalle.}{She wasn't afraid of anything.}\sensenumber{2}\definition{no one, anyone (nominative form)}\example{Ddone aya mɨnyi ngänäm batramän ttongo ttängäm e.}{No one will take me away to another village.}\sensenumber{2}\definition{accusative form of ddone aya}\example{Ubi ddone amom ikop dägaeyo.}{They didn't see anyone.}\sensenumber{2}\definition{very, a lot (antiphrasis)}\example{Llamda da ddone ada mikutt gogon obo pate.}{The old man got very angry at them.}\subentry{\headword{ddone any}\pos{pron.}\definition{nothing}}\subentry{\headword{ddone aya}\pos{pers. pron.}\definition{no one, anyone (nominative form)}}\subentry{\headword{ddone amom}\pos{pers. pron.}\definition{accusative form of ddone aya}}\subentry{\headword{ddone ada}\pos{adv.}\definition{very, a lot (antiphrasis)}}}
\entry{ddongddong}{\headword{ddongddong}\pos{n.}\sensenumber{2}\definition{thick cluster of short grass}}
\entry{ddugwemddugwem}{\headword{ddugwemddugwem}\pos{adv.}\sensenumber{2}\definition{stomping}}
\entry{ddumbi}{\headword{ddumbi}\pos{n.}\sensenumber{2}\definition{type of spear topped with the claw of a cassowary}}
\entry{ddungg}{\headword{ddungg}\pos{S vt.}\sensenumber{2}\definition{to decapitate}\example{Zon bom inkätt danddugän.}{He decapitated John.}\allomorph{nddug}\allomorph{ndduminy}\allomorph{dduminy}}
\lettersection{E}
\entry{e1}{\headword{e1}\pos{interj.}\sensenumber{1}\definition{whoa}\sensenumber{2}\definition{let's go}}
\entry{e2}{\headword{e2}\pos{rel. pron.}\sensenumber{1}\definition{which, what, that}\example{Iba kame da e kollba we gopänaeyän.}{We (incl.) don't know which fish she turned into.}\sensenumber{2}\definition{which, what}\example{E pazi me ka?}{In which year?}\example{Bongo e bin dan ma me?}{What is his position in the community?}\sensenumber{2}\definition{what}\example{Ge enda?}{What is this?}\sensenumber{2}\definition{copular form of enda (present singular form)}\example{Bäne mokowang melem a endan?}{What is your favorite work?}\sensenumber{2}\definition{past singular form of endan}\example{Diba eka bo bin a endaeya?}{What was the name of that language?}\sensenumber{2}\definition{past plural form of endan}\sensenumber{2}\definition{present plural form of endan}\example{Bäne eka muu da endag ubira?}{What answers do you have for them?}\sensenumber{2}\definition{past plural form of endan}\example{Bogo tämamae wäte endagaeya mermerae dänggllämnegän.}{Whatever wounds there were, she cleaned them all properly.}\sensenumber{2}\definition{past singular form of endan}\sensenumber{2}\definition{accusative form of enda}\example{Sisri iddob e, bongo ende ngasnges e dan?}{What are you doing tonight?}\sensenumber{1}\definition{which, that, who}\example{Bam era mälla da papa nagan ngämo nag dan.}{The woman that hit you is my friend.}\example{Däräng a bam era da naddägan, ngämo dan.}{The dog that bit you is mine.}\example{Era lla da bongnonggän ddone mɨnyi kuddäll bogän.}{The man who believes will not die.}\sensenumber{2}\definition{which, what}\example{Ibi ngasekäma era nyongo ibi wi dag?}{Which road might we (incl.) take?}\example{Bäne era eka dag umllang a?}{What languages do you speak?}\sensenumber{3}\definition{focus marker}\example{Bongo era Adi bo llɨg dan.}{YOU are the son of God.}\sensenumber{3}\definition{copular form of era (present singular form)}\example{Däbe abo igi mi eran, däbe bamllamän.}{Then she will hold that one that's on the bottom.}\example{Obo sana da eran?}{Where is his sago? / Which is his sago?}\example{Bäne eka da eran?}{What is your language?}\sensenumber{3}\definition{past singular form of eran}\example{Kilikiliangae dinduag a ada gogeyo do oba mokowang abal tatuma da eraeya.}{They ran happily like this to where their favorite washing place was.}\sensenumber{3}\definition{past plural form of eran}\sensenumber{3}\definition{present plural form of eran}\example{Gagäll rabis wattällnen ma da erag duli utale mae.}{The places that are for discarding rubbish are far away.}\example{Ngattong ddone ada deya, ddäddäg a erag?}{Before there were lots of them; where are the animals?}\sensenumber{3}\definition{past plural form of eran}\example{Ddob kollba da ngäma eragaeya, ddob llayabira dasiaemeya.}{Some of the fish that were ours (excl.), we gave them away to other people.}\sensenumber{3}\definition{past dual form of eran}\example{Oba tämamae ttoen bällam ngasnen a eragwaeya, llame dagwaeya.}{Everything that the two of them did, they did together.}\sensenumber{3}\definition{instrumental form of era}\example{nyäng iatt mälla da eraballe za kemibi bokomän}{bag with which women carry many things}\sensenumber{1}\definition{locative form of era; where, in which}\example{Llimoll a ttängäm dan ngäna erame gozeg.}{Limol is the place where I was born.}\example{Ngämo kame dan däbe pazi da ngäna erame gozeg.}{I do not know in which year I was born.}\sensenumber{2}\definition{where, in which (locative form of era)}\example{Erame bibi sana de nattkoealla?}{Where did you (pl.) cut the sago?}\sensenumber{1}\definition{accusative form of era}\example{Angde ubi gongoseyo, ubi ikop dägageyo mamam mätta da Meri erem de dibeyän audaban.}{When they returned, they saw that the red yam that Mary had planted disappeared.}\example{Ngäna gem eka de ttättle anggan, ge ddob lla da erem gomoenen amallo.}{I correct these words that some people mispronounce.}\sensenumber{2}\definition{accusative form of era}\example{Erem mätta de bongo moko eralle?}{Which yam do you like?}\example{dade eremde}{whichever}\sensenumber{1}\definition{where}\example{Dirom de ngäna däbe darälgoe do llo da ulle da ero daeya,}{I dragged that cassowary to where that big tree was.}\sensenumber{2}\definition{where}\example{Ero dan ma da?}{Where is the house?}\sensenumber{1}\definition{ablative form of ero}\example{Ddone aya umllang gogon ada erowattäm da ge za da gogezän.}{No one knew from where these things came out.}\sensenumber{2}\definition{ablative form of ero}\example{Ge lla da erowattäm da ngänttägmäll allan?}{From where is this man arriving?}\sensenumber{1}\definition{allative form of ero}\example{Ge mälla da iba kame da erowede ibi allan.}{We (incl.) don't know where this woman is going.}\sensenumber{2}\definition{allative form of ero}\example{Bongo erowe ibi alle?}{Where are you going?}\sensenumber{1}\definition{why}\example{Census, ddob ako ttoen a dadegaeya, ewede deyaralle patrol a.}{There was the census, and also other things: that's why the patrol came.}\sensenumber{2}\definition{why}\example{Ewede bongo bäne känyer yununin alle?}{Why are you sleeping by yourself?}\sensenumber{2}\definition{accusative form of e}\example{Bongo em skul de dättemän?}{Which level of school did you finish?}\sensenumber{1}\definition{ablative form of e; why (relative)}\example{Ngämo kame dan ematta bongo mikutt alle.}{I don't know why you're mad.}\example{Bogo gongosalle ag me ma we a llabunda bim, umllang dägnegalle ewatta kuddäll gogon lla da.}{He would return to the house of the relatives in the morning and inform them why the person died.}\sensenumber{2}\definition{ablative form of e; why (interrogative)}\example{Bongo ematta däräng de nazuwalle?}{Why did you shoot the dog?}\example{Bibi komlla ewatta käptang abal agalla?}{Why are you two so wet?}\etymology{era + =alle₁}\subentry{\headword{enda}\pos{int. pron.}\definition{what}}\subentry{\headword{endan}\pos{cop.}\definition{copular form of enda (present singular form)}}\subentry{\headword{endaeya}\pos{cop.}\definition{past singular form of endan}}\subentry{\headword{endaeyag}\pos{cop.}\definition{past plural form of endan}}\subentry{\headword{endag}\pos{cop.}\definition{present plural form of endan}}\subentry{\headword{endagaeya}\pos{cop.}\definition{past plural form of endan}}\subentry{\headword{endaya}\pos{cop.}\definition{past singular form of endan}}\subentry{\headword{ende}\pos{int. pron.}\definition{accusative form of enda}}\subentry{\headword{era}\pos{rel. pron.}\definition{which, that, who}}\subentry{\headword{eran}\pos{cop.}\definition{copular form of era (present singular form)}}\subentry{\headword{eraeya}\pos{cop.}\definition{past singular form of eran}}\subentry{\headword{eraeyag}\pos{cop.}\definition{past plural form of eran}}\subentry{\headword{erag}\pos{cop.}\definition{present plural form of eran}}\subentry{\headword{eragaeya}\pos{cop.}\definition{past plural form of eran}}\subentry{\headword{eragwaeya}\pos{cop.}\definition{past dual form of eran}}\subentry{\headword{eraballe}\pos{rel. pron.}\definition{instrumental form of era}}\subentry{\headword{erame}\pos{rel. pron.}\definition{locative form of era; where, in which}}\subentry{\headword{erem}\pos{rel. pron.}\definition{accusative form of era}}\subentry{\headword{ero}\pos{rel. pron.}\definition{where}}\subentry{\headword{erowattäm}\pos{rel. pron.}\definition{ablative form of ero}}\subentry{\headword{erowe}\pos{rel. pron.}\definition{allative form of ero}}\subentry{\headword{ewede}\pos{rel. pron.}\definition{why}}\subentry{\headword{em}\pos{int. pron.}\definition{accusative form of e}}\subentry{\headword{ematta}\pos{adv.}\definition{ablative form of e; why (relative)}}}
\entry{=e1}{\headword{=e1}\pos{n. cl.}\sensenumber{1}\definition{allative case clitic; to, into, towards}\example{Mälla da maket e nallan.}{The woman goes to the market.}\example{Ngämo näkäp e gongttagän ada ngaska zime iddob amänong agan.}{It came to my mind that maybe it was already midnight.}\example{Ngäna mɨnyi bäne eka de bäpänaeneg Ingglis e.}{I'm going to translate your words into English.}\sensenumber{2}\definition{for, to (purposive use of the allative case; also follows nonfinite verbs to express desire, intention, or the start of an action)}\example{Ge mätta da känaebag e dan.}{This yam is for tomorrow.}\example{Obo moko da gäbän e dan.}{He wants to jump (lit. his desire is to jump).}\example{Alla täräp me ibi ge nge de ibeny e dan?}{At what time are we going to plant this coconut?}\example{Gäddgädd e nängkaman.}{He started to fight.}\allomorph{wee}\allomorph{we}\allomorph{e}\allomorph{wi}\allomorph{i}\allomorph{wi}}
\entry{=e2}{\headword{=e2}\pos{disc. ptcl.}\sensenumber{2}\definition{vocative}\allomorph{we}}
\entry{ebagal}{\headword{ebagal}\pos{n.}\sensenumber{2}\definition{type of spear}}
\entry{ebdo}{\headword{ebdo}\pos{n.}\sensenumber{1}\definition{day}\example{ebdo ulle}{all day}\sensenumber{2}\definition{noon (approx. 11 AM –1 PM)}\sensenumber{2}\definition{lunch}\subentry{\headword{ebdo duwem}\pos{n.}\definition{lunch}}}
\entry{Eda}{\headword{Eda}\variant{sp. var. of}{Edna}}
\entry{ede}{\headword{ede}\pos{subr.}\sensenumber{1}\definition{so, therefore}\example{Be ngämo ttɨle da attkaman, ede ngämlle ddone mullae dan, ngäna tongoe e beyar.}{But my leg broke, so I can't go play.}\sensenumber{2}\definition{then}\example{Da obo ngoe a mamam gognegän, ede Yokon a bäne mätta de notan.}{If her teeth are red, then Yokon has eaten your yam.}}
\entry{Edeb}{\headword{Edeb}\pos{pn.}\sensenumber{1}\definition{garden, camping, and hunting place of Tewa (on the other side of Karama swamp)}\sensenumber{2}\definition{personal name}}
\entry{Edna}{\headword{Edna}\pos{pn.}\sensenumber{2}\definition{female personal name}}
\entry{Edward}{\headword{Edward}\pos{pn.}\sensenumber{2}\definition{male personal name}}
\entry{eddom}{\headword{eddom}\pos{n.}\sensenumber{2}\definition{the day before yesterday}}
\entry{Egapo}{\headword{Egapo}\pos{pn.}\sensenumber{2}\definition{camping place and large community garden (in Limol; road to garden is visible on L of AZ96)}}
\entry{ei}{\headword{ei}\pos{interj.}\sensenumber{2}\definition{hey}}
\entry{eighty}{\headword{eighty}\variant{sp. var. of}{eiti}}
\entry{eit}{\headword{eit}\pos{num.}\sensenumber{2}\definition{eight (English numeral; also general)}\etymology{from Englisheight}}
\entry{eiti}{\headword{eiti}\pos{num.}\sensenumber{2}\definition{eighty}\etymology{from Englisheighty}}
\entry{eitin}{\headword{eitin}\pos{num.}\sensenumber{2}\definition{eighteen (English numeral)}\etymology{from Englisheighteen}}
\entry{eiz}{\headword{eiz}\pos{n.}\sensenumber{2}\definition{HIV; AIDS}\etymology{from EnglishAIDS}}
\entry{eka}{\headword{eka}\pos{n.}\sensenumber{1}\definition{language}\example{Motu eka, pällämpälläm eka}{Hiri Motu, English}\example{Ngämo Ende eka da umllang dan, be bäne ddone mäzi umllang dan.}{My Ende is fluent, but yours isn't.}\sensenumber{2}\definition{word, message, news}\example{Ngäna mɨnyi bäne eka de bäpänaeneg Ingglis e.}{I'm going to translate your words into English.}\example{Komlla o kumuddäga lla da mɨnyi utt alle eka de bängaeyo.}{Two or three men will send word with the conch shell.}\sensenumber{3}\definition{story}\example{Ge eka da gongesän ngattong.}{This story happened first.}\example{Ngämo eka kälsre da däbe.}{That was my little story.}\sensenumber{4}\definition{sound, song, call}\example{Gontomoneyo wutt eka we.}{They were waiting for the sound of the conch shell.}\example{Alläp de eka eran.}{He plays the drum.}\sensenumber{5}\definition{to say, speak, talk, tellEka panypeny o ttongo lla walle eka sämpallmeny tameny. (Speaking or telling someone something.)}\example{Ngäna mɨnyi ada eka bog.}{I will say so.}\example{Män a llɨg de ada eka dägagän, "Ma we nalle."}{The girl told the boy, "Go home."}\sensenumber{5}\definition{word (single unit)}\example{Gudae masamasar aba eka kutt panypeny da kälakälae sapasapang gognegän.}{The words of our ancestors were slightly different.}\sensenumber{5}\definition{messenger, prophet}\example{Bogo Adi bäne ttongo llɨtɨtang dan.}{He is one of God's messengers. / He is a prophet.}\sensenumber{5}\definition{meaning}\example{Oba pate bogo tämamae eka midd de däddelnegnän}{He was explaining all the meanings to them.}\sensenumber{5}\definition{answer, reply}\example{Ngäna ubim eka mu anggan Mägayam eka walle.}{I answer them in Makayam.}\example{Kottllam a ada eka mu gogän, "Kandärmang."}{Turtle replied, "Sorry."}\sensenumber{1}\definition{to discuss, converse}\example{Ngämo nagnag aba kämall eka gontemenyne.}{I was conversing with my friends.}\sensenumber{2}\definition{to tell; converse with, speak to}\example{Ubi ngämim gänya eka walle mae eka deyantemenynegnän.}{They were only speaking to us in this language.}\sensenumber{2}\definition{disagreement, debate, argument}\example{Angde Yesu a sedusi lla da eka täp me gognegnän, ttongo zuu llayaba sabi tamenyang a dallän oba pate.}{When Jesus and the Sadducee were debating, a Pharisee (lit. teacher of the Jewish law) came towards them.}\sensenumber{2}\definition{throat}\sensenumber{2}\definition{to argue}\example{Eka golaemeyo.}{They argued.}\sensenumber{2}\definition{to speak}\example{Ddone ngänäm mulldae allan eka panypeny e.}{I am unable to speak.}\example{Ge Llimollang a alla ingollang eka däpanyneyo.}{This is how Limol people spoke it.}\sensenumber{1}\definition{speaker}\example{Ge Ende eka panypenyang lla dan.}{This is an Ende speaker.}\sensenumber{2}\definition{spokespersonPalament me eka panypenyang bun lla. (The speaking head of parliament.)}\example{Ngäna eka panypenyang dan kominiti mi.}{I am a spokesperson in the community.}\sensenumber{2}\definition{to make noise}\example{Borale da eran ttongo ekaeka ma za dan.}{The flute is a thing that makes noise.}\allomorph{yeka}\allomorph{eka}\etymology{llɨtɨt + =ang}\subentry{\headword{eka kutt}\pos{n.}\definition{word (single unit)}}\subentry{\headword{eka llɨtɨtang}\pos{n.}\definition{messenger, prophet}}\subentry{\headword{eka midd}\pos{n.}\definition{meaning}}\subentry{\headword{eka mu}\pos{n.}\definition{answer, reply}}\subentry{\headword{eka tameny}\pos{S vi.}\definition{to discuss, converse}}\subentry{\headword{eka täp}\pos{n.}\definition{disagreement, debate, argument}}\subentry{\headword{ekamatär}\pos{n.}\definition{throat}}\subentry{\headword{eka laem}\pos{S vi.}\definition{to argue}}\subentry{\headword{eka panypeny}\pos{S vt.}\definition{to speak}}\subentry{\headword{eka panypenyang}\pos{n.}\definition{speaker}}\subentry{\headword{ekaeka}\pos{A vi.}\definition{to make noise}}}
\entry{eka malam}{\headword{eka malam}\pos{v.}\sensenumber{2}\definition{obedientDa ttongo lla da bam e ngasges e ada nagän, bongo mɨnyi nanges.}}
\entry{ekaklle}{\headword{ekaklle}\pos{n.}\sensenumber{1}\definition{land}\example{Ddia da ge ulle abal ddäddäg dan iba ekaklle me.}{Deer is a major animal in our (excl.) land.}\sensenumber{2}\definition{ground}\example{Ekaklle da llokttang dan.}{The ground is hard.}\sensenumber{3}\definition{Earth}\example{Kok ekaklle de nganaenen eran.}{The moon rotates around Earth.}\sensenumber{4}\definition{low}\example{De nyonga da ekaklle abal me paplläg eran.}{That cockatoo is flying very low.}\sensenumber{4}\definition{world}\example{Mälla papa da buddo ulle dan Papua Niugini mi a be kemibi ngättma me ge ekaklle ulle me ako.}{Beating one's wife is a big issue in Papua New Guinea but also in many places in the world.}\subentry{\headword{ekaklle ulle}\pos{n.}\definition{world}}}
\entry{eke}{\headword{eke}\variant{var. of}{ke}}
\entry{eleben}{\headword{eleben}\pos{num.}\sensenumber{4}\definition{eleven (English numeral)}\etymology{from Englisheleven}}
\entry{elementeri skul}{\headword{elementeri skul}\variant{sp. var. of}{elementri skul}}
\entry{elementri skul}{\headword{elementri skul}\pos{n.}\sensenumber{4}\definition{elementary school}\etymology{from Englishelementary school}}
\entry{Elisa}{\headword{Elisa}\pos{pn.}\sensenumber{4}\definition{female personal name}}
\entry{Elizabeth}{\headword{Elizabeth}\pos{pn.}\sensenumber{4}\definition{female personal name}}
\entry{Elsie}{\headword{Elsie}\pos{pn.}\sensenumber{4}\definition{female personal name}}
\entry{Em}{\headword{Em}\pos{n.}\sensenumber{4}\definition{Em language (Pahoturi River language spoken in Kurunti, Kibuli, Beyambod)}\etymology{from Emem de 'what (acc.)'}}
\entry{emaemae}{\headword{emaemae}\pos{mod.}\sensenumber{4}\definition{various, many different kinds}\example{Dageya, emaemae pa de dägäddaebeyo.}{They went and killed many different types of birds.}}
\entry{Emi}{\headword{Emi}\pos{pn.}\sensenumber{4}\definition{female personal name}}
\entry{enaenae}{\headword{enaenae}\variant{var. of}{enanae}}
\entry{enanae}{\headword{enanae}\pos{adv.}\sensenumber{1}\definition{directly, straight}\example{Llɨg kälsre de ine da enanae dae dätramän do ddob llayaba pate dängttägän.}{The water carried the small boy straight through and he arrived near some people.}\sensenumber{2}\definition{forever, for good}\example{Obo ttängäm de enanae dowansegän.}{He left his village forever.}\sensenumber{3}\definition{final; finally, at last}\example{God bo enanane ttoen pallängkmeny}{God's final judgment}\example{Enanae abal ako ngäna gongkäbämenyne.}{Then, at long last, I was diving.}\sensenumber{3}\definition{eternal}\example{Era lla da bongnonggän ddone mɨnyi kuddäll bogän, be enanaeenanae ttam de bälldän.}{He who believes will not die, but will receive eternal life.}\subentry{\headword{enanaeenanae}\pos{mod.}\definition{eternal}}}
\entry{Ende}{\headword{Ende}\pos{pn.}\sensenumber{3}\definition{Ende language (Pahoturi River language spoken in Limol, Malam, and Kinkin)}\etymology{from Ende ende 'what (acc.)'}}
\entry{endeya}{\headword{endeya}\variant{fr. var. of}{endaeya}}
\entry{Endo}{\headword{Endo}\pos{pn.}\sensenumber{3}\definition{female personal name}}
\entry{enddäna}{\headword{enddäna}\pos{n.}\sensenumber{3}\definition{clearing}\example{Tatuma da enddäna me dan.}{The washing place is in the clearing.}\sensenumber{3}\definition{in the open, openly, freely}\example{Bogo llameny känyärtto ttängäm me dagirnän, ddone enddnaenddna taon e dallnän.}{He was staying in quiet places without people; he wasn't going openly into town.}\subentry{\headword{enddänaenddäna}\pos{adv.}\definition{in the open, openly, freely}}}
\entry{enddna}{\headword{enddna}\variant{fast speech var. of}{enddäna}}
\entry{Enza}{\headword{Enza}\pos{pn.}\sensenumber{3}\definition{Enza (Bitur-speaking village in Oriomo-Bituri Rural LLG; from Limol, one must pass through Bisuaka)}}
\entry{enzul}{\headword{enzul}\pos{n.}\sensenumber{3}\definition{angel}\etymology{from Englishangel}}
\entry{Englis}{\headword{Englis}\variant{sp. var. of}{Ingglis}}
\entry{English}{\headword{English}\variant{sp. var. of}{Ingglis}}
\entry{Erabal}{\headword{Erabal}\pos{pn.}\sensenumber{3}\definition{female personal name}}
\entry{eragän}{\headword{eragän}\variant{dial. var. of}{erag}}
\entry{erageya}{\headword{erageya}\variant{fr. var. of}{eragaeya}}
\entry{eragwe}{\headword{eragwe}\variant{fr. var. of}{eragwaeya}}
\entry{eragweya}{\headword{eragweya}\variant{fr. var. of}{eragwaeya}}
\entry{Eramang}{\headword{Eramang}\pos{pn.}\sensenumber{3}\definition{Eramang (the main canoe place in Limol; for camping, fishing, hunting)}}
\entry{eranän}{\headword{eranän}\variant{dial. var. of}{eran}}
\entry{erang}{\headword{erang}\pos{kin.}\sensenumber{3}\definition{exchange sibling; exchange brother (man's wife's brother who marries the man's sister; reciprocal); exchange sister (woman's husband's sister who marries the woman's brother; reciprocal)Mälla de asiallo erang alle. Lla da mälla de erang alle lladäd era. (Marrying the woman by exchange. The man marries the woman by exchange.)}\example{Ngämlle erang a ddone dan.}{I don't have an exchange sibling.}\sensenumber{3}\definition{exchange cousin (one's parent's exchange sibling's child)Erang alle llɨg ngasngesatt a ubi erngazeg dageyo. (The children from exchange marrriages are exchange cousins.)}\sensenumber{3}\definition{exchange uncle (one's parent's exchange brother)Llɨg a mägda bo mang de erngazenda eka eran. (A son calls his mother's exchange brother erngazenda.)}\sensenumber{3}\definition{exchange aunt (one's parent's exchange sister)Llɨg a mädada bo män de erngazmäg eka eran. (A son calls his father's exchange sister erngazmäg.)}\etymology{probably from erang + zeg}\subentry{\headword{erngazeg}\pos{kin.}\definition{exchange cousin (one's parent's exchange sibling's child)Erang alle llɨg ngasngesatt a ubi erngazeg dageyo. (The children from exchange marrriages are exchange cousins.)}}\subentry{\headword{erngazenda}\pos{kin.}\definition{exchange uncle (one's parent's exchange brother)Llɨg a mägda bo mang de erngazenda eka eran. (A son calls his mother's exchange brother erngazenda.)}}\subentry{\headword{erngazmäg}\pos{kin.}\definition{exchange aunt (one's parent's exchange sister)Llɨg a mädada bo män de erngazmäg eka eran. (A son calls his father's exchange sister erngazmäg.)}}}
\entry{erany}{\headword{erany}\pos{n.}\sensenumber{1}\definition{scream}\example{Ngäna erany de dandär.}{I heard a scream.}\sensenumber{2}\definition{to scream, shout, yell (at)}\example{Bogo erany allan.}{He is screaming.}\example{Lla da män kälsre de erany nägawan.}{The man shouted at the little girl.}\allomorph{era}}
\entry{eraya}{\headword{eraya}\variant{fr. var. of}{eraeya}}
\entry{eräl}{\headword{eräl}\variant{dial. var. of}{erär}}
\entry{erängg}{\headword{erängg}\pos{S vt.}\sensenumber{2}\definition{to test, try; taste}\example{Tos de dägazen nyäng ik att de a batri käp de dazer. Angde daerängg mermer gogllayän.}{I took out the torch from inside my bag and put in the batteries. When I tested it, it properly shone.}\example{Be ami ddäddägeyo, mokwang abal daeränggeyo.}{But those who tried eating it found that it tasted delicious.}\allomorph{ermeny}\allomorph{erämeny}\allomorph{eräng}}
\entry{erär}{\headword{erär}\pos{S vt.}\sensenumber{1}\definition{to name; pass down a name}\example{Ngäma pemli makäp me gänyan ge ernan erallo.}{Here in our (excl.) family, we are passing on this name.}\example{Ngämo kame dan aeya ngämo bin di ge daerän.}{I don't know who gave me this name.}\sensenumber{2}\definition{to measure}\example{Ge za da llo ernan ma dan.}{This thing is for measuring trees.}\allomorph{er}}
\entry{ere}{\headword{ere}\variant{fr. var. of}{eraeya}}
\entry{ereya}{\headword{ereya}\variant{fr. var. of}{eraeya}}
\entry{Erga}{\headword{Erga}\pos{pn.}\sensenumber{2}\definition{female personal name}}
\entry{ergod}{\headword{ergod}\pos{S vi.}\sensenumber{2}\definition{to crawl}\example{Llɨg kälsre da ergodnen allan.}{The child is crawling.}\sensenumber{2}\definition{toddler}\allomorph{argod}\allomorph{ergodnen}\allomorph{argodaeb}\allomorph{argodmam}\etymology{ergod + =ang}\subentry{\headword{ergodag}\pos{n.}\definition{toddler}}}
\entry{eria}{\headword{eria}\pos{n.}\sensenumber{2}\definition{area}\example{Kawam eria me dan oba ttängäm a duli.}{Their land is there, in the Kawam area.}\etymology{from Englisharea}}
\entry{Eric}{\headword{Eric}\pos{pn.}\sensenumber{2}\definition{male personal name}}
\entry{erkoll}{\headword{erkoll}\pos{n.}\sensenumber{2}\definition{dirty water}\example{Bongo ngänäm dangelbne, ddone ikop dagne, erkoll a dangttawalle.}{You were looking for me and didn't see me; the dirty water hid me.}}
\entry{Erme}{\headword{Erme}\pos{pn.}\sensenumber{2}\definition{male personal name}}
\entry{Erodias}{\headword{Erodias}\pos{pn.}\sensenumber{2}\definition{PN}}
\entry{eroe}{\headword{eroe}\pos{loc.}\sensenumber{2}\definition{side}}
\entry{erongg}{\headword{erongg}\pos{v.}\sensenumber{2}\definition{to start weaving (crossing strings)}\example{Essie nga wiya-o, bongo ngämo naerongg tater de.}{Essie, you come and start weaving my mat.}\allomorph{eromeny}}
\entry{es}{\headword{es}\pos{interj.}\sensenumber{2}\definition{come (command given to animals)}}
\entry{esam}{\headword{esam}\pos{n.}\sensenumber{2}\definition{lemongrassGe towall ingoll olang ttongdae mit me päddpäddag dan. Ttam de näträpneg a ine ttäntämang e naspull pane we. Nakone kaptte alle diba peyang ttänttäm itrellang me. Ako tatuwag diba ine walle mae. Ako däräng tänggag ma dan. (This grass grows from a single main stem at the base. Cut the leaves and put them in a pot of hot water. Cover it with cloth and use when sick with fever. Also wash with it. Also given to dogs.)}}
\entry{Ese}{\headword{Ese}\pos{pn.}\sensenumber{2}\definition{male personal name}}
\entry{eso}{\headword{eso}\pos{interj.}\sensenumber{2}\definition{thank you}\example{Eso eka nägagan.}{He thanked him.}\example{Eso ulle.}{Thank you very much.}\example{Eso ulle bänene kandärmang e.}{Thank you for the gift.}\example{Eso ulle bänene ngämingg e.}{Thank you for the help.}}
\entry{Essie}{\headword{Essie}\pos{pn.}\sensenumber{2}\definition{female personal name}}
\entry{Esta}{\headword{Esta}\pos{pn.}\sensenumber{2}\definition{female personal name}}
\entry{ewembe}{\headword{ewembe}\pos{n.}\sensenumber{2}\definition{type of very big yam with a white interior, thorns, and hairs}}
\entry{ezi}{\headword{ezi}\pos{n.}\sensenumber{2}\definition{sharp edgeMa llo gädanen ma dan. (It's for sharpening house sticks.)}\etymology{from Englishedge}}
\entry{Ezra}{\headword{Ezra}\pos{pn.}\sensenumber{2}\definition{male personal name}}
\entry{Evelyn}{\headword{Evelyn}\pos{pn.}\sensenumber{2}\definition{female personal name}}
\entry{ẽː}{\headword{ẽː}}
\lettersection{G}
\entry{gaall}{\headword{gaall}\variant{sp. var. of}{gall}}
\entry{Gabadi}{\headword{Gabadi}\pos{pn.}\sensenumber{2}\definition{female personal name}}
\entry{gabana}{\headword{gabana}\pos{n.}\sensenumber{2}\definition{governorProbins me llayabaene ngällbänatt lla da aya mer ttoen de täbanen anggan. (The person elected by the people in the province who plans good things.)}\etymology{from Englishgovernor}}
\entry{gabän1}{\headword{gabän1}\pos{n.}\sensenumber{1}\definition{wrist}\sensenumber{2}\definition{six (lit. wrist; body counting numeral)}}
\entry{gablle}{\headword{gablle}\pos{n.}\sensenumber{2}\definition{hip; waist}\example{Obo gablle mättnan ma källän a ddäddäg ttoe dädär alle ngasngesatt daeya.}{The belt he wore around his waist was made from animal hide.}}
\entry{gabma}{\headword{gabma}\pos{n.}\sensenumber{1}\definition{white person}\example{ttongo gabma bo bin}{a white man's name}\sensenumber{2}\definition{white, Western; foreign}\sensenumber{2}\definition{gun}\etymology{probably a clipping of gabmantt 'government'}\subentry{\headword{gabma bägäl}\pos{n.}\definition{gun}}}
\entry{gabman}{\headword{gabman}\variant{var. of}{gabmantt}}
\entry{gabmantt}{\headword{gabmantt}\pos{n.}\sensenumber{2}\definition{government}\etymology{from Englishgovernment}}
\entry{gae nge}{\headword{gae nge}\pos{n.}\sensenumber{2}\definition{type of palm with coconuts with a red or green exocarp; the husk is chewed and the young coconut water is drunk}}
\entry{Gaem}{\headword{Gaem}\pos{pn.}\sensenumber{2}\definition{female personal name}}
\entry{gagäll}{\headword{gagäll}\pos{mod.}\sensenumber{1}\definition{bad, rotten}\example{Ge kollba da gagäll dan.}{This fish is bad.}\example{Da ngäna bol de ttang alle bäbädd, ngämo tupi di mɨnyi gagäll bäga.}{If I hit the ball with my hand, I will hurt my index finger.}\example{Ngämo giddoll a ngattong ai daeyag a be ngäna sisri gagäll ttam e azenan.}{My life was good at first, but now, I have entered a bad life.}\example{Panya dae ddone gagäll gogon.}{Only the pineapple did not go bad.}\sensenumber{2}\definition{dead}\example{Ngämo kameny me de gagäll gogon.}{He died during my absence.}\sensenumber{3}\definition{sin}\example{Ibim deyanglläbnegän gagäll kup att de.}{He brought us out of the depths of sin.}\sensenumber{3}\definition{badly, poorly}\example{Ubi Ende eka de gagägagäll däpanyallo.}{They spoke the Ende language poorly.}\subentry{\headword{gagägagäll}\pos{adv.}\definition{badly, poorly}}}
\entry{gagäll ine}{\headword{gagäll ine}\pos{n.}\sensenumber{3}\definition{beer}\etymology{lit. 'bad water'}}
\entry{gageb}{\headword{gageb}\pos{n.}\sensenumber{3}\definition{hind leg}}
\entry{gaguma}{\headword{gaguma}\pos{n.}\sensenumber{3}\definition{yamhouse}\example{Mäkat llɨgllɨgag a dan bäne gaguma me.}{There is a rat with babies in your yamhouse.}}
\entry{Gaima}{\headword{Gaima}\pos{pn.}\sensenumber{3}\definition{Gaima (toponym)}}
\entry{gaimbi}{\headword{gaimbi}\pos{n.}\sensenumber{3}\definition{type of fruit tree}}
\entry{gal}{\headword{gal}\pos{n.}\sensenumber{3}\definition{food offeringMänamänang o päzäpäzäg abira sämell sisiang o wätät de nasineg. (Giving sisters or siblings-in-law a pig or you give them food.)}}
\entry{galbe}{\headword{galbe}\pos{n.}\sensenumber{3}\definition{purple/greater yam}}
\entry{galbi}{\headword{galbi}\variant{sp. var. of}{galbe}}
\entry{galib}{\headword{galib}\pos{n.}\sensenumber{3}\definition{type of spear}}
\entry{Galibma}{\headword{Galibma}\pos{pn.}\sensenumber{3}\definition{sacred place of Biku (Madura) Kangge (on the road to Bisuaka, after Limol ma kuddäll; bush area, creek)}}
\entry{galigali}{\headword{galigali}\pos{n.}\sensenumber{3}\definition{type of small taro}}
\entry{Galo}{\headword{Galo}\pos{pn.}\sensenumber{3}\definition{male personal name}}
\entry{Galwe}{\headword{Galwe}\pos{pn.}\sensenumber{3}\definition{male personal name}}
\entry{gall}{\headword{gall}\pos{n.}\sensenumber{3}\definition{canoe, boatLlo tɨtatt walle me gllaenen ma. Llo ulle popoatt ine toko me ibnin e. (A hollowed out tree for floating on the water. A big carved tree used to go on the water.)}\example{Amo gall sisor da däbe?}{Whose new canoe is this?}\sensenumber{3}\definition{canoe outriggerGall bo tungg. (A canoe's support.)}\sensenumber{3}\definition{navigator (of a canoe)}\sensenumber{3}\definition{operator (of a canoe)}\etymology{lit. 'canoe snake'}\subentry{\headword{gall gullem}\pos{n.}\definition{canoe outriggerGall bo tungg. (A canoe's support.)}}\subentry{\headword{gall ngattongag}\pos{n.}\definition{navigator (of a canoe)}}\subentry{\headword{gall onyang}\pos{n.}\definition{operator (of a canoe)}}}
\entry{gallab}{\headword{gallab}\pos{S vt.}\sensenumber{3}\definition{to open (one's mouth)}\example{Eka tameny a bongo abo mudan, kälae ttatt gallab a mudan.}{You are not allowed to speak or open your jaw even a little.}\example{Abo ttatt de gallab a mudan.}{Don't you say a word.}\allomorph{gallaeb}\allomorph{gellab}\allomorph{gellaeb}\allomorph{gallaebmällam}\allomorph{gall}}
\entry{gallgall}{\headword{gallgall}\pos{n.}\sensenumber{3}\definition{bank, coast, shore}\example{Ubi komlla apte walle gallgall e dängkällneneyo.}{The two of them climbed onto the other shore.}\example{Sämell a dizmollän gälgälag Geleli walle we guspullän a tämamae gomänddnegän.}{The pigs ran to the banks of the Sea of Galilee and fell in, and all of them drowned.}}
\entry{Gamaewe}{\headword{Gamaewe}\pos{pn.}\sensenumber{3}\definition{Gamaewe (toponym)}}
\entry{gambän}{\headword{gambän}\pos{S vi.}\sensenumber{3}\definition{to stand}\sensenumber{3}\definition{causative-applicative form of gambän}\example{Ngäna bam danggebänängg.}{I made you stand.}\allomorph{nggebän}\etymology{gambän + -ngg}\subentry{\headword{gämbanengg}\pos{S vt.}\definition{causative-applicative form of gambän}}}
\entry{gambenängg}{\headword{gambenängg}\variant{var. of}{gämbanengg}}
\entry{gamo1}{\headword{gamo1}\pos{n.}\sensenumber{3}\definition{type of large sea turtle}}
\entry{gamo2}{\headword{gamo2}\pos{n.}\sensenumber{3}\definition{plant\textbackslash_type}}
\entry{gamu}{\headword{gamu}\pos{n.}\sensenumber{3}\definition{type of ginger with flat leaves; used as medicine for centipede bites and as bait for catching flying foxes; chew it first and the flying fox will eat it and become lethargic}}
\entry{gan}{\headword{gan}\variant{dial. var. of}{dadegän}}
\entry{gangan}{\headword{gangan}\pos{n.}\sensenumber{3}\definition{type of tree}}
\entry{gany}{\headword{gany}\pos{S vt.}\sensenumber{1}\definition{to plant, place in the ground, put in}\example{Kup de däkällneg a pos de däganeg.}{I dug the holes and put the posts in.}\example{Angde pos ganen a gottamänän ngäna sipel gog.}{When the post planting finished, I took a rest.}\sensenumber{2}\definition{to focus, tune}\example{Llan de ngämi däganyeya eka we.}{We tuned our ears for the sound.}\allomorph{ganen}\allomorph{geny}\allomorph{ga}\allomorph{gage}}
\entry{Gao}{\headword{Gao}\pos{pn.}\sensenumber{2}\definition{male personal name}}
\entry{gaopi}{\headword{gaopi}\pos{n.}\sensenumber{2}\definition{Australian pelican}}
\entry{gaora}{\headword{gaora}\pos{n.}\sensenumber{2}\definition{type of sagoUlle kuttang sana dan. (It's a sago with a lot of pith.)}}
\entry{gara}{\headword{gara}\pos{n.}\sensenumber{2}\definition{aquatic leechWalle tongle lla pätt me ddogollag dan, mam naneang dan. (It's an aquatic leech that sticks to a person's body; it drinks blood.)}}
\entry{Garai}{\headword{Garai}\variant{sp. var. of}{Garayi}}
\entry{Garayi}{\headword{Garayi}\pos{pn.}\sensenumber{2}\definition{male personal name}}
\entry{Garaz}{\headword{Garaz}\pos{pn.}\sensenumber{2}\definition{male personal name}}
\entry{gastol}{\headword{gastol}\pos{n.}\sensenumber{2}\definition{type of fish}}
\entry{gaugau}{\headword{gaugau}\pos{n.}\sensenumber{2}\definition{type of big tree that grows in the bush along creeks with wood used for canoes}}
\entry{gazen}{\headword{gazen}\pos{S vi.}\sensenumber{1}\definition{to come out, get out, exit, escape}\example{Nag wiya agezän.}{Friend, come out.}\example{Bogo ma ik atta gogezänän wälläng e dindugän.}{He got out from inside and ran into the bush.}\example{Kandärmang, yu da daollemae llamda da bälle gazenma da ddone mullae gogon.}{Sadly, the fire got closer to the old man and he didn't have a way he could escape.}\sensenumber{2}\definition{to rise, come out; shine}\example{Yäbäd a mermer abal gogezänän.}{The sun came out, shining nicely.}\example{Kok a gazen allan.}{The moon is rising.}\sensenumber{3}\definition{to take out}\example{Käsre tos de dägazen nyäng ik att de a batri käp de dazer.}{Then I took the torch out from the bag and put in the batteries.}\example{Aeya obom bigezänän?}{Who will come take him out?}\allomorph{gez}\allomorph{gezän}\allomorph{gazgez}\allomorph{gaz}\allomorph{gezen}}
\entry{gazibra}{\headword{gazibra}\pos{n.}\sensenumber{3}\definition{type of edible water snake with skin like sandpaperIne ik mi giddollag gullem a ako obo ttoe a ddobae mu dan. (A snake living in the water, and its skin is very valuable.)}\example{Alläp de era gazibra ttoe alle ttoe amallo.}{Drums are skinned with gazibra snake skin.}}
\entry{gajen}{\headword{gajen}\variant{sp. var. of}{gazen}}
\entry{gäba}{\headword{gäba}\pos{n.}\sensenumber{3}\definition{shade}\example{Kottllam a bogo ada gongnomenynän adaka ddia da zäme ngattong agan, be bogo de amne me gotarnän llo gäba me.}{Turtle thought that Deer was already first, but he was in the middle sleeping in the shade of a tree.}\example{Ubi ttoenttoen gogneyo llo gäba me.}{They were telling stories in the shade.}\sensenumber{3}\definition{shady}\example{Llo gäbagäba de ikop dägagän.}{He saw a shady tree.}\subentry{\headword{gäbagäba}\pos{mod.}\definition{shady}}}
\entry{gäbaeb}{\headword{gäbaeb}\pos{S vi.}\sensenumber{3}\definition{to jump}\example{Män a kädkädag walle we mɨnyi bogbaebän.}{The girl will jump into the cold water.}\allomorph{gbaeb}\allomorph{gäb}}
\entry{Gäbag}{\headword{Gäbag}\pos{pn.}\sensenumber{3}\definition{male personal name}}
\entry{gäbän}{\headword{gäbän}\pos{S vi.}\sensenumber{3}\definition{to jump, hop}\example{Gäbänmällang dan.}{He is hopping.}\example{Män a gall att agäbänan.}{The girl jumped out of the canoe.}\allomorph{gäb}\allomorph{gbä}\allomorph{gäbän}\allomorph{gbän}\allomorph{gb}\allomorph{gäbä}}
\entry{gäbgäb}{\headword{gäbgäb}\pos{n.}\sensenumber{3}\definition{type of tree}}
\entry{gäboll}{\headword{gäboll}\pos{n.}\sensenumber{3}\definition{magpie-lark}}
\entry{gädagäde}{\headword{gädagäde}\pos{S vt.}\sensenumber{1}\definition{to beat sago, pound sago}\example{Beyawaebe dägdene.}{Alone, he beat sago.}\sensenumber{2}\definition{to sharpen}\example{Ngäna tri pos dae dägdeneg.}{I only sharpened three posts.}\sensenumber{3}\definition{to throw a tantrum}\allomorph{gäde}\allomorph{gädanen}\allomorph{gde}}
\entry{gäddgädd}{\headword{gäddgädd}\pos{S vt.}\sensenumber{1}\definition{to fightLla da mängall bognän ttongo lla pate o ttongo za watta. (A person will use power against someone or because of something.)}\example{Lla da gäddgädd e bobllabän.}{The men go to fight.}\example{Bongo sämell aba peyang gäddgädd alle.}{You are fighting with the pig.}\example{Ubi gogäddän.}{They fought.}\sensenumber{2}\definition{to catch}\example{Mälla da sisri ag me kok de nägäddaeballo kollba tokong e.}{This morning, the women caught grasshopers for bait.}\sensenumber{3}\definition{to cut (leaves)}\example{Ngäma Dowabunang me gällall de ke ami dägäddaebeyo?}{Who cut our (excl.) pandanus leaves in Dowabunang?}\allomorph{gädd}\allomorph{gäddnan}}
\entry{gägabäll}{\headword{gägabäll}\pos{S vi.}\sensenumber{3}\definition{to stand}\example{Tutu we dirngäneyo a ubi dägabllabän.}{She pulled her to land, and they stood.}\example{Dibaya kälepalle danyroene, angde dowae mäse diga, ddia da dam dindugän a dägabällän.}{I crept slowly towards it, and when I got close, the deer ran away and stood.}\example{Oba ulle binang a dägabälleyo ada ddia bo sawe ttang alle a kottllam bo ttätt ttang alle.}{Their big bosses stood like this: the deer's on the left hand side and the turtle's on the right hand side.}\allomorph{gabäll}\allomorph{gabllab}\allomorph{gägabllaeb}\allomorph{gabällab}\allomorph{gabmäll}\allomorph{gägabmäll}\allomorph{gägabll}}
\entry{gägäb ine}{\headword{gägäb ine}\pos{n.}\sensenumber{3}\definition{dew}\example{Apapi da llo ttam me gägäb ine nanong dan.}{The butterfly drinks the dew on leaves.}}
\entry{gäglib}{\headword{gäglib}\pos{S vt.}\sensenumber{1}\definition{to chase, scare away/off (wild animals)}\example{Däräng a ddia de diglibiyu Madura pate.}{The dogs chased the deer, steering it towards Madura.}\example{Ngäna käza de nägliban.}{I scared away the crocodile.}\sensenumber{2}\definition{to pluck}\example{Pa kom de dägläbaebeyo, tot do dangesaemeyo.}{They plucked the bird's feathers and made a mess there.}\allomorph{gläb}\allomorph{glib}}
\entry{gäglläb}{\headword{gäglläb}\pos{S vt.}\sensenumber{2}\definition{to weed for the first time (hard work to remove the large plants)}}
\entry{gäl}{\headword{gäl}\pos{n.}\sensenumber{2}\definition{type of tree that grows in the bush with white flowers, brown seeds, and fruit with a yellow pericarp and green exocarp}}
\entry{Gälabi}{\headword{Gälabi}\pos{pn.}\sensenumber{2}\definition{Gälabi (Wipi-speaking village in Oriomo-Bituri Rural LLG; from Limol, one must pass through Wipim)}}
\entry{gälas}{\headword{gälas}\pos{n.}\sensenumber{2}\definition{weaving pattern with concentric squares}}
\entry{Gäläb}{\headword{Gäläb}\pos{pn.}\sensenumber{2}\definition{Gälab (sago place and Geoff Rowak's camping place in Limol)}}
\entry{gälbän}{\headword{gälbän}\pos{S vt.}\sensenumber{1}\definition{to knock (a fruit)}\example{Polle ma ik i angde gozenän, up wo gälbänanatt de ikop dägagän.}{When he went inside the fence, he saw the fallen ripe banana.}\sensenumber{2}\definition{to defeather, depilate, remove hair; remove teeth}\allomorph{gälb}\allomorph{gälbänan}}
\entry{gäleb}{\headword{gäleb}\pos{n.}\sensenumber{2}\definition{type of tree}}
\entry{gälgäl}{\headword{gälgäl}\variant{var. of}{gallgall}}
\entry{gällall}{\headword{gällall}\pos{n.}\sensenumber{2}\definition{type of pandanusWälläng o walle mäg me päddabag dan. Obo ttam a tater inen e gäddnan ma dag. (They grow in the bush or near creeks. Its leaves are cut for making mats.)}\example{Ngäma Dowabunang me gällall de ke ami dägäddaebeyo?}{Who cut our pandanus leaves at Dowabunang?}}
\entry{gällanggälla}{\headword{gällanggälla}\variant{sp. var. of}{gllangglla}}
\entry{gällatater}{\headword{gällatater}\pos{n.}\sensenumber{2}\definition{type of tree}}
\entry{gällbän}{\headword{gällbän}\pos{S vi.}\sensenumber{2}\definition{to break}\example{Ngämo llimba da agällbänan.}{My fingernail broke.}}
\entry{gälle}{\headword{gälle}\pos{n.}\sensenumber{2}\definition{type of big tree that grows in the bush near big creeks with wood used for canoes, white and blue flowers, and edible red fruit}}
\entry{gällgäll}{\headword{gällgäll}\variant{var. of}{gallgall}}
\entry{gämäll}{\headword{gämäll}\pos{A vt.}\sensenumber{2}\definition{to steal}\example{gämällang lla}{thief}\example{mälla gämäll}{adultery (lit. 'wife-stealing')}\example{Ngäna sana de gämäll agan bänene.}{I stole the sago from you.}\example{Ngäna sana de gämäll agan bablle.}{I stole the sago for you.}\sensenumber{2}\definition{to steal}\example{Da bongo ddone otät de nonttog, bam mɨnyi gämäll nantägegän.}{If you do not give him food, he will steal.}\subentry{\headword{gämäll tänggag}\pos{S vt.}\definition{to steal}}}
\entry{gämoe}{\headword{gämoe}\pos{S vt.}\sensenumber{1}\definition{to miss}\example{Ddia de ddone dagmoe bazu.}{Shooting the deer, I didn't miss.}\example{Ttongo eka nagmoeyan.}{I forgot to say one thing.}\sensenumber{2}\definition{to miss, feel longing for}\example{Ngäna bam gämoe nallan.}{I am missing you.}\allomorph{gmoe}\allomorph{gämoenen}}
\entry{gän}{\headword{gän}\pos{n.}\sensenumber{2}\definition{gun}\etymology{from Englishgun}}
\entry{gänggälläm}{\headword{gänggälläm}\pos{S vt.}\sensenumber{2}\definition{to wash (a body part or inanimate object)}\example{Ddob mälla da pane moepang de ikrol alle gällämnan amallo.}{Some women wash dirty pots with ash.}\example{Bogo pilatt de dängllämnegan.}{He washed the dishes.}\example{Da bongo källama atta angos, ttang anggälläm.}{If you come back from the toilet, wash your hands.}\allomorph{ngglläm}\allomorph{gällämnan}\allomorph{nggälläm}\allomorph{gälläm}\allomorph{nglläm}}
\entry{gänggllam}{\headword{gänggllam}\pos{A vi.}\sensenumber{2}\definition{to snore}\example{Ge lla da dämen ma toko me inunin allan a mälläng gänggllam allan.}{This man is sleeping on a chair and snoring.}}
\entry{gängglläd}{\headword{gängglläd}\pos{S vt.}\sensenumber{1}\definition{to shove, push}\example{Ngäna sɨmell de duduli dänggllädalle.}{I shoved the pig that way.}\sensenumber{2}\definition{to extend (e.g. a garden)}\allomorph{ngglläd}}
\entry{gängglläm}{\headword{gängglläm}\variant{fast speech var. of}{gänggälläm}}
\entry{gänya}{\headword{gänya}\pos{adv. dem.}\sensenumber{1}\definition{here (proximal)}\example{Bogo gänya angttägan.}{He is coming here.}\example{Ngäna sisri gänya bäne peyang eka tameny allan.}{Now I am here talking with you.}\sensenumber{2}\definition{this (proximal determiner)}\example{Bogo gänya nyongo me ibiag da.}{He often walks on this road.}\example{Da ngäna ddone yu bipellknegalle bablle, bongo ddone gänya mer yu mi kallkäll bongllaegalle.}{If I hadn't chopped this wood for you, you wouldn't have this nice fire to warm up by.}\sensenumber{2}\definition{here}\example{Kemu bo llabun a gänyme a Malläm me dadeg.}{Kemu has relatives here and in Malam.}\example{Tämamae lla da Ende eka de gänyme Llimoll me panypeny erallo.}{All people here in Limol speak Ende.}\sensenumber{2}\definition{restrictive copular form of gänya (present singular form)}\example{Ngämo eka da gänyaeben Ende eka da ge ngäna ere eka allan.}{The only language that I speak is this Ende language.}\sensenumber{2}\definition{copular form of gänya (present singular form)}\example{Gänyan obo sana da.}{Here is his sago.}\sensenumber{2}\definition{past singular form of gänyan}\example{Ede ngämo ttoenttoen llätt a gänyaeya.}{So here was the end of my story.}\sensenumber{2}\definition{present plural form of gänyan}\example{Oba za da gänyag.}{Here are their things.}\sensenumber{2}\definition{past plural form of gänyan}\example{Oba bin a gänyagaeya ada, Samson, Wawase, a Pedro.}{Here were their names: Samson, Wawase, and Pedro.}\sensenumber{2}\definition{present dual form of gänyan}\example{Ge lla komlla da gänyageyo, ibnin allo.}{These two people here, they are going.}\sensenumber{2}\definition{past dual form of gänyan}\example{Tätäm ubi komlla gänyagwaeya gonzämeyo.}{Yesterday, the two of them were here passing through.}\sensenumber{2}\definition{this way}\example{Ngämo mäda gänyeri guddällän, gänyeri mälla de dällädän.}{My father came over here and got a local wife.}\sensenumber{2}\definition{gänyeri with perlative clitic}\example{Bongo wiya gänyerimae.}{You come over here.}\sensenumber{2}\definition{allative form of gänyeri}\example{Wendy, mamal gänyeriballe yaupe adawatta da mo da bäne peyang bottkamän.}{Wendy, jump quickly to this side in case the bridge breaks with you.}\sensenumber{2}\definition{ablative form of gänya}\example{Gänyaballe dallän do we.}{From here, he went there.}\example{Iba ge kollba da mullae dag, ibi gänyaballe ma we bängäsmäll.}{We (incl.) have enough fish; from here we shall return home.}\sensenumber{2}\definition{ablative form of gänya}\example{Mälla gänyamasäm dan.}{The woman is from here.}\example{Gänyamasäm a ddone mermer Ende eka de panypeny erallo.}{Here's why they aren't speaking Ende well.}\sensenumber{2}\definition{allative form of gänya}\example{Bäne za de gänyaolle ikom.}{Bring your things over here.}\example{Eka sasapang a gänyaolle llɨtaem eran mälla walle.}{Various languages are arriving here along with women.}\sensenumber{2}\definition{allative form of gänya with perlative clitic}\example{Bongo gänyaollemae wiya.}{You come over here.}\allomorph{gänye}\allomorph{gäny}\allomorph{gänyä}\etymology{from gänya + =me}\subentry{\headword{gänyme}\pos{adv. dem.}\definition{here}}\subentry{\headword{gänyaeben}\pos{cop.}\definition{restrictive copular form of gänya (present singular form)}}\subentry{\headword{gänyan}\pos{cop.}\definition{copular form of gänya (present singular form)}}\subentry{\headword{gänyaeya}\pos{cop.}\definition{past singular form of gänyan}}\subentry{\headword{gänyag}\pos{cop.}\definition{present plural form of gänyan}}\subentry{\headword{gänyagaeya}\pos{cop.}\definition{past plural form of gänyan}}\subentry{\headword{gänyageyo}\pos{cop.}\definition{present dual form of gänyan}}\subentry{\headword{gänyagwaeya}\pos{cop.}\definition{past dual form of gänyan}}\subentry{\headword{gänyeri}\pos{adv. dem.}\definition{this way}}\subentry{\headword{gänyerimae}\pos{adv. dem.}\definition{gänyeri with perlative clitic}}\subentry{\headword{gänyeriballe}\pos{adv. dem.}\definition{allative form of gänyeri}}\subentry{\headword{gänyaballe}\pos{adv. dem.}\definition{ablative form of gänya}}\subentry{\headword{gänyamasäm}\pos{adv. dem.}\definition{ablative form of gänya}}\subentry{\headword{gänyaolle}\pos{adv. dem.}\definition{allative form of gänya}}\subentry{\headword{gänyaollemae}\pos{adv. dem.}\definition{allative form of gänya with perlative clitic}}}
\entry{gänyagaya}{\headword{gänyagaya}\variant{fr. var. of}{gänyagaeya}}
\entry{gänyageya}{\headword{gänyageya}\variant{fr. var. of}{gänyagaeya}}
\entry{gänyanän}{\headword{gänyanän}\variant{dial. var. of}{gänyan}}
\entry{gänyo}{\headword{gänyo}\variant{var. of}{gänya}}
\entry{gärep}{\headword{gärep}\pos{n.}\sensenumber{2}\definition{grape}\etymology{from Englishgrape}}
\entry{gäz}{\headword{gäz}\pos{S vt.}\sensenumber{1}\definition{to kill}\example{Emaemae pa de dägäddaebeyo.}{They killed many different types of birds.}\example{Ge lla da sisri gäz e dan.}{This man is to be killed now.}\sensenumber{2}\definition{to hit, beat}\example{Ge lla da mälla da bom näbäddan.}{This man beat his wife.}\example{Bibi komllaebe gäddnan a mudageyo.}{You two, don't fight.}\allomorph{gädd}\allomorph{bädd}\allomorph{gäddnan}}
\entry{ge1}{\headword{ge1}\pos{dem.}\sensenumber{2}\definition{this (proximal determiner and pronoun)}\example{Ge wätät a ddobae moko dan.}{This food is delicious.}\example{Ge obo mälla dan.}{This is his wife.}\sensenumber{2}\definition{accusative form of ge}\example{Gem de nɨpnae.}{You will turn this around.}\example{Ngäna gem eka de ttättle anggan.}{I correct these words.}\subentry{\headword{gem}\pos{dem.}\definition{accusative form of ge}}}
\entry{geagell}{\headword{geagell}\pos{n.}\sensenumber{2}\definition{Lewin's rail}}
\entry{geawe}{\headword{geawe}\pos{n.}\sensenumber{2}\definition{type of big tree that grows along creeks with bright purple flowers and wood used for canoes}}
\entry{gee}{\headword{gee}\pos{adv.}\sensenumber{2}\definition{intensifier}\example{Ubi dade melem gogneyo gee.}{They were putting up yam sticks for a long time.}}
\entry{Geleli}{\headword{Geleli}\pos{pn.}\sensenumber{2}\definition{Galilee}}
\entry{Gene}{\headword{Gene}\pos{pn.}\sensenumber{2}\definition{male personal name}}
\entry{Georgina}{\headword{Georgina}\pos{pn.}\sensenumber{2}\definition{female personal name}}
\entry{Geoff}{\headword{Geoff}\pos{pn.}\sensenumber{2}\definition{male personal name}}
\entry{ger}{\headword{ger}\pos{n.}\sensenumber{2}\definition{type of big tree that grows in the bush}}
\entry{Geser}{\headword{Geser}\pos{pn.}\sensenumber{2}\definition{male personal name}}
\entry{gi}{\headword{gi}\pos{n.}\sensenumber{2}\definition{grease, fat}\example{Ddia da däbe ddobae gi daeya.}{That deer was fatty.}}
\entry{Gibson}{\headword{Gibson}\pos{pn.}\sensenumber{2}\definition{male personal name}}
\entry{Gidra}{\headword{Gidra}\pos{pn.}\sensenumber{2}\definition{(possibly derogatory) Wipi language}}
\entry{gidre}{\headword{gidre}\pos{n.}\sensenumber{2}\definition{enemy}}
\entry{Gidu}{\headword{Gidu}\pos{pn.}\sensenumber{2}\definition{male personal name}}
\entry{giddoll}{\headword{giddoll}\pos{n.}\sensenumber{1}\definition{life}\example{Oba giddoll makäp me ubi ttoen de mermerangae dangesnegneyo, wätät yagnen a,}{In their life, they did all things properly.}\example{Bäne giddoll a mɨnyi ada bogon ddobae llayaba za gämällangag.}{Your life will be like this, stealing things from people.}\sensenumber{2}\definition{to live, reside}\example{Ttongo ause bo bin a Madima, Kinkin ttängäm me giddollag daeya.}{An old woman, her name is Madima, was living in Kinkin village.}\example{Män a Daru me dagirnän.}{The girl lived in Daru.}\sensenumber{3}\definition{to stay, remain}\example{Bongo polle ddäg me nagirne, ngäna do amne we balle.}{You stay outside the fence; I will go in the middle.}\example{Ende eka da ddone budabän be bagirnän enanae enanae.}{The Ende language will not be lost, but will remain forever.}\sensenumber{3}\definition{menstruating, on one's period}\sensenumber{3}\definition{in menopause}\allomorph{gir}\allomorph{zir}\allomorph{ger}\allomorph{zer}\allomorph{giddollnen}\allomorph{zär}\etymology{giddoll + =ang, lit. 'living'}\subentry{\headword{giddollag}\pos{mod.}\definition{menstruating, on one's period}}\subentry{\headword{giddollmeny}\pos{mod.}\definition{in menopause}}}
\entry{giegier}{\headword{giegier}\pos{n.}\sensenumber{3}\definition{white-browed crake}}
\entry{Gilbet}{\headword{Gilbet}\pos{pn.}\sensenumber{3}\definition{male personal name}}
\entry{gilib}{\headword{gilib}\pos{n.}\sensenumber{3}\definition{type of birdYogoll me tongoeang pa kälsre dan. (It's a small bird that sings during rainy season.)}}
\entry{Gimaga}{\headword{Gimaga}\pos{pn.}\sensenumber{3}\definition{personal name}}
\entry{Gina}{\headword{Gina}\variant{sp. var. of}{Zina}}
\entry{Ginarang}{\headword{Ginarang}\pos{pn.}\sensenumber{3}\definition{male personal name (the original Ende man [see SE\textbackslash_PN022])}}
\entry{Gini}{\headword{Gini}\pos{pn.}\sensenumber{3}\definition{female personal name}}
\entry{Ginia}{\headword{Ginia}\pos{pn.}\sensenumber{3}\definition{male personal name}}
\entry{Giniya}{\headword{Giniya}\pos{pn.}\sensenumber{3}\definition{male personal name}}
\entry{girag}{\headword{girag}\pos{n.}\sensenumber{3}\definition{long-nosed echymipera}}
\entry{girag dirindi}{\headword{girag dirindi}\pos{n.}\sensenumber{3}\definition{type of yam}}
\entry{giragirag}{\headword{giragirag}\pos{n.}\sensenumber{3}\definition{type of tree}}
\entry{giri}{\headword{giri}\pos{n.}\sensenumber{3}\definition{knifeDdob melem kälekäle ngasnen ma da. (A thing used to do small tasks.)}\example{Ge lla da nyäng de giri alle papoe eran.}{This man is piercing the bag with a knife.}\example{Ngäna giri de dipirngän.}{I took out the knife.}\sensenumber{3}\definition{sword}\subentry{\headword{giri busa}\pos{n.}\definition{sword}}}
\entry{giritai}{\headword{giritai}\pos{n.}\sensenumber{3}\definition{type of long yam with a white interior and hairs}}
\entry{giriwak}{\headword{giriwak}\pos{n.}\sensenumber{3}\definition{type of toolMondre ma da. (It's for gardening.)}}
\entry{giro}{\headword{giro}\pos{n.}\sensenumber{3}\definition{chicken pox}}
\entry{gita}{\headword{gita}\pos{n.}\sensenumber{3}\definition{guitar}\etymology{from Englishguitar}}
\entry{giwi}{\headword{giwi}\pos{n.}\sensenumber{3}\definition{fruit dove (coroneted, orange-bellied, pink-spotted, orange-fronted)}}
\entry{Giwo}{\headword{Giwo}\pos{pn.}\sensenumber{3}\definition{male personal name}}
\entry{gɨg}{\headword{gɨg}\pos{S vt.}\sensenumber{3}\definition{to collect ants}}
\entry{gɨngg}{\headword{gɨngg}\pos{S vt.}\sensenumber{3}\definition{to ambush, gang up on, attack in a large group}\example{Nogat bom ubony a ere daeränggeyo, gudae enanae dɨnggɨgeyo.}{When bees started biting Nogat, they really attacked him.}\allomorph{nggɨg}}
\entry{Gladis}{\headword{Gladis}\pos{pn.}\sensenumber{3}\definition{female personal name}}
\entry{Glan}{\headword{Glan}\pos{pn.}\sensenumber{3}\definition{female personal name}}
\entry{glas}{\headword{glas}\pos{n.}\sensenumber{3}\definition{glass}\etymology{from Englishglass}}
\entry{gleb}{\headword{gleb}\pos{S vt.}\sensenumber{3}\definition{to take, steal}\example{Aya ngämo up näglebän?}{Who took my banana?}}
\entry{Gledis}{\headword{Gledis}\variant{sp. var. of}{Gladis}}
\entry{Glendis}{\headword{Glendis}\pos{pn.}\sensenumber{3}\definition{female personal name}}
\entry{Gloria}{\headword{Gloria}\pos{pn.}\sensenumber{3}\definition{female personal name}}
\entry{gllae1}{\headword{gllae1}\pos{S vi.}\sensenumber{1}\definition{to float, drift; paddle}\example{Net a gllaenen allan.}{The net is floating.}\example{Ngämi (excl.) angde gogllaenalla, däbe za da dibeya ka ili därän.}{When we paddled, that thing almost disappeared.}\sensenumber{2}\definition{to paddle; pedal}\example{Ngäna mɨnyi gall de bogllae.}{I will paddle the canoe.}\example{Ngämi kili gogman, kili peyang gall de dagllaeya do Upiara dälltam.}{We (excl.) were so happy, with joy we paddled the canoe until we arrived at Upiara.}\example{Ngäna gall de mängamängall dagllae ddia koenmäll e, gall alle gumbiebmeny.}{I paddled the canoe quickly to chase the deer with the boat.}\sensenumber{3}\definition{to dig, spade}\example{Gällae de dängkamalle, dagllaealle dagllaealle.}{They started digging; they dug and dug.}\allomorph{glla}\allomorph{glle}\allomorph{gällae}}
\entry{gllae2}{\headword{gllae2}\pos{S vi.}\sensenumber{3}\definition{to shine}\example{Yäbäd a mermerae gogllaewän.}{The sun shined nicely.}\example{Ngäna angde gongllae ddäg e, ge indrang a ngämo dowae me gogllayän.}{When I looked to the back, this light was shining close to me.}\allomorph{glla}\allomorph{glle}}
\entry{gllaglle}{\headword{gllaglle}\pos{S vt.}\sensenumber{3}\definition{to skin, remove skin}\example{Ngäna ttall ttoe de nägllewan.}{I removed the wallaby skin.}\allomorph{glle}\allomorph{glla}}
\entry{gllangglla}{\headword{gllangglla}\pos{S vi.}\sensenumber{3}\definition{to swim}\example{Gällanggälla eran.}{He is swimming.}\example{Tätam ngämi bun i dängallaebeya, ako mutt i dingllaebeya.}{Yesterday, we (excl.) swam down the river and then back up the river.}\allomorph{nglla}\allomorph{ngglla}\allomorph{nggll}}
\entry{glle}{\headword{glle}\pos{n.}\sensenumber{3}\definition{type of tree with small edible fruit and wood used to make canoes}}
\entry{gllo}{\headword{gllo}\pos{S vt.}\sensenumber{3}\definition{to take out, remove}\example{Däräng källa de ma ngätt att de baglloaebnalla.}{We will take dog poop out from the yards.}}
\entry{gllogllo}{\headword{gllogllo}\pos{n.}\sensenumber{3}\definition{marbled frogmouthIddob me ekawang pa dan. (It's a bird that sings at night.)}}
\entry{Gllu}{\headword{Gllu}\pos{pn.}\sensenumber{3}\definition{Gllu (toponym)}}
\entry{gllu}{\headword{gllu}\pos{ideo.}\sensenumber{3}\definition{splash}}
\entry{go1}{\headword{go1}\pos{n.}\sensenumber{3}\definition{drain}\example{Mälla da go bangesnegnän ine da eramae bizmollnän.}{The women will make drains so that the water will run.}}
\entry{Goballwang}{\headword{Goballwang}\pos{pn.}\sensenumber{3}\definition{Goballwang (camping, hunting, and garden place in Karama swamp between Upiara and Limol)}}
\entry{god}{\headword{god}\pos{n.}\sensenumber{3}\definition{type of cultivated fruit tree similar to dimes tree}}
\entry{Godd}{\headword{Godd}\pos{pn.}\sensenumber{3}\definition{God}\example{Godd ibim deyangesnegän.}{God created us (incl.).}\etymology{from EnglishGod}}
\entry{goeg}{\headword{goeg}\pos{n.}\sensenumber{3}\definition{old garden that is ready to be cleared again}\example{goeg kuddäll}{old garden}\example{Lla da goeg melem anggan.}{People do garden work.}}
\entry{Goeg wälläng}{\headword{Goeg wälläng}\pos{pn.}\sensenumber{3}\definition{garden place in the bush on the west side of the road to Taolang}}
\entry{gogäle}{\headword{gogäle}\pos{n.}\sensenumber{3}\definition{noise}\example{Lla llɨg a dandärmom mänmän abaene gogäle de.}{The boys heard the noise from the girls.}}
\entry{Goge}{\headword{Goge}\pos{pn.}\sensenumber{3}\definition{male personal name}}
\entry{Gogina}{\headword{Gogina}\variant{sp. var. of}{Georgina}}
\entry{Gogo}{\headword{Gogo}\variant{fast speech var. of}{Gogodala}}
\entry{gogo1}{\headword{gogo1}\pos{S vt.}\sensenumber{3}\definition{to build}\example{Ma de gogo eran.}{He is building the house.}\example{Nogowallo.}{They built it.}\example{Ttongo täräp me ngäna ngämo ma gogo we gotäbane.}{One time, I planned to build my house.}\example{Abo bongo ttongo ma sisor de nogo.}{You must build a new house.}\allomorph{go}\allomorph{gonen}\allomorph{nggo}}
\entry{gogo2}{\headword{gogo2}\pos{n.}\sensenumber{3}\definition{varieties of palms with coconuts with a dark green exocarp}}
\entry{gogo kottllam}{\headword{gogo kottllam}\pos{n.}\sensenumber{3}\definition{type of turtle with yellow scales on neck}}
\entry{Gogodala}{\headword{Gogodala}\pos{pn.}\sensenumber{3}\definition{Gogodala language}}
\entry{Gogodara}{\headword{Gogodara}\variant{sp. var. of}{Gogodala}}
\entry{gogodd}{\headword{gogodd}\pos{n.}\sensenumber{1}\definition{spleen}\example{Pa bo gogodd a malla ddäddäg ma dan.}{The bird's spleen is not edible.}\sensenumber{2}\definition{bladder}\sensenumber{2}\definition{water breaking}\subentry{\headword{gogodd pänddäg}\definition{water breaking}}}
\entry{gol}{\headword{gol}\pos{n.}\sensenumber{2}\definition{goal}\etymology{from Englishgoal}}
\entry{golgol}{\headword{golgol}\pos{n.}\sensenumber{2}\definition{type of tree that grows in the bush with a straight trunk and wood used for house sticks}}
\entry{gollab}{\headword{gollab}\pos{S vt.}\sensenumber{1}\definition{to pour}\example{Ibi kumuddäga ine kutt de dogollaebaebeya.}{We (incl.) poured out three water buckets.}\example{Nogollaeb.}{[You] pour it.}\sensenumber{2}\definition{flow}\sensenumber{2}\definition{applicative form of gollab}\example{Mänmän mängalae ine walle danggollaebägeyo.}{The girls quickly poured water on someone.}\allomorph{gllaeb}\allomorph{gollaeb}\allomorph{gll}\allomorph{goll}\etymology{gollaeb- [gollab] + -ngg}\subentry{\headword{gollaembäg}\pos{S vt.}\definition{applicative form of gollab}}}
\entry{gollaeb}{\headword{gollaeb}\pos{quant.}\sensenumber{2}\definition{many}\example{Gaem ai tater gollaeb de inen anggan.}{Gaem is weaving many good mats.}}
\entry{gollob}{\headword{gollob}\pos{n.}\sensenumber{2}\definition{outer layer, hull, shell (e.g. of a turtle, egg)}\example{wit gollob}{wheat hull}\example{Lla da päre abo gobällan kottllam bo gollob pallkepallke de dänglläbeyo.}{Then the people sadly went to the turtle and collected all the pieces of his shell.}}
\entry{golloll}{\headword{golloll}\pos{loc.}\sensenumber{2}\definition{back, behind}\example{Llamda da gudae dallän polle we gozenän a up golloll me gotogolän.}{The old man went through the fence early in the morning and hid behind the banana tree.}}
\entry{gollolla}{\headword{gollolla}\pos{n.}\sensenumber{2}\definition{type of palmPaklle käp zaenen ma dan, llo toko me. (Paklle snake eggs are laid on the treetop.)}}
\entry{gomoe}{\headword{gomoe}\pos{S vt.}\sensenumber{2}\definition{to make a mistake}\example{Ngäna gem eka de ttättle anggan, ge ddob lla da erem gomoenen amallo.}{I correct these words, which other people mispronounce.}\allomorph{gomoenen}}
\entry{gonagone}{\headword{gonagone}\pos{S vt.}\sensenumber{1}\definition{to cook}\example{Bongo nalle, otät e nognen.}{You, go and cook the food.}\sensenumber{2}\definition{to burn}\example{Llo mit de dugonaeballo.}{They burned the tree bases.}\sensenumber{3}\definition{to heat}\example{Ngäna ine ttämttämang yu dägag a ngämo ttäle de dogonaeya.}{I boiled water and heated my leg.}\allomorph{gone}\allomorph{gän}\allomorph{gna}\allomorph{gona}\allomorph{gne}}
\entry{gonz}{\headword{gonz}\pos{n.}\sensenumber{3}\definition{reedTtope ingoll kälekäle dag a nyäng inen ma dag. (They're like small ropes used to weave bags.)}\example{Buk nyäng a gonz alle inenatt a mer abal dan.}{Gonz reed is very good to weave book bags with.}}
\entry{gonzagonzar}{\headword{gonzagonzar}\pos{n.}\sensenumber{3}\definition{type of tree}\sensenumber{3}\definition{type of grubWälläng me ddäddäg ma budar dan. (It's an edible grub found in the bush.)}\subentry{\headword{gonzagonzar budar}\pos{n.}\definition{type of grubWälläng me ddäddäg ma budar dan. (It's an edible grub found in the bush.)}}}
\entry{Gonzer}{\headword{Gonzer}\pos{pn.}\sensenumber{3}\definition{female personal name}}
\entry{gongg}{\headword{gongg}\pos{S vi.}\sensenumber{3}\definition{to disturb a bees nest and then to feel the bites}\allomorph{nggog}\allomorph{nggomeny}\allomorph{gonggmeny}\allomorph{nggmeny}\allomorph{ngg}}
\entry{gongglem}{\headword{gongglem}\pos{n.}\sensenumber{1}\definition{immature coconut that is partially solid inside}\example{Gongglem daeben yäbäd ttänttämang me nane ma da.}{Immature coconuts are the best when drunk in the hot sun.}\sensenumber{2}\definition{the eighth stage of coconut growth during which the endosperm of the fruit is solidifying}}
\entry{gonggo}{\headword{gonggo}\pos{n.}\sensenumber{2}\definition{piping bellbird/crested pitohuiWälläng me amtetmeny ekawang pa dan. (A bird that sings in the bush nonstop.)}}
\entry{gonya}{\headword{gonya}\variant{dial. var. of}{gänya}}
\entry{gonyeya}{\headword{gonyeya}\variant{fr. var. of}{gänyaeya}}
\entry{gonyo}{\headword{gonyo}\variant{dial. var. of}{gänya}}
\entry{gora}{\headword{gora}\pos{n.}\sensenumber{2}\definition{rattleTtang alle tatraem ma ingong täräp me. (To play with your hand when dancing.)}}
\entry{goral}{\headword{goral}\pos{n.}\sensenumber{2}\definition{type of tree}}
\entry{goro}{\headword{goro}\pos{mod.}\sensenumber{2}\definition{lush, overgrown, wild}\sensenumber{2}\definition{jungle}\subentry{\headword{goro wälläng}\pos{n.}\definition{jungle}}}
\entry{grawa}{\headword{grawa}\pos{n.}\sensenumber{1}\definition{Australasian darterWalle me kämbägag pa dan. (It's a bird that dives in the water.)}\sensenumber{2}\definition{little corella}}
\entry{Grace}{\headword{Grace}\pos{pn.}\sensenumber{2}\definition{female personal name}}
\entry{greid}{\headword{greid}\pos{n.}\sensenumber{2}\definition{grade}\example{greid siks me}{in grade six}\etymology{from Englishgrade}}
\entry{Greis}{\headword{Greis}\variant{sp. var. of}{Grace}}
\entry{Guar}{\headword{Guar}\pos{pn.}\sensenumber{2}\definition{male personal name}}
\entry{Gubam}{\headword{Gubam}\pos{pn.}\sensenumber{2}\definition{Gubam (in Morehead Rural LLG)}}
\entry{gubare}{\headword{gubare}\pos{n.}\sensenumber{2}\definition{crossbeam}}
\entry{guboll}{\headword{guboll}\pos{n.}\sensenumber{2}\definition{New Guinean magpieBädab dowae me ekawang pa dan. (It's a bird that sings around dawn.)}}
\entry{gudae1}{\headword{gudae1}\pos{mod.}\sensenumber{1}\definition{early morning}\example{Gudae ag särem abal me ngäma ngämi otat de mängalae yu dägaebeya.}{In the darkness of early morning, we (excl.) quickly cooked our food.}\example{Bogo gudae abal angällbänan.}{He woke up very early in the morning.}\sensenumber{2}\definition{(the) past, before}\example{Ngäma ddob gudae att gagäll de dämbälaebnalla.}{Some of us (excl.) were remembering bad things from the past.}\sensenumber{3}\definition{earlier, previously, in the past, before}\example{gudae gudae täräp me}{long, long ago}\example{Komlla nag a dadegwaeya ami gudae deyagirneyo walle ddage menae me.}{There were two friends who used to live on the side of the stream.}\example{Dedme mɨnyi bibi obom ikop nägaeyo ada ingollang bogo alla umllang deyagnegän bibim gudae.}{You (pl.) will see him there, just like how he told you before.}}
\entry{gudae2}{\headword{gudae2}\pos{quant.}\sensenumber{3}\definition{plenty}\example{Wayati da duli ge ttängäm gudae ada enddäna gogon wayati dae gogon.}{There were so many watermelon there, this place was like a clearing of only watermelon.}}
\entry{gudne}{\headword{gudne}\pos{mod.}\sensenumber{1}\definition{old}\example{gudne lla}{old man}\example{Ge ma da gudne dan.}{This house is old.}\sensenumber{2}\definition{long ago}\example{Däräng a gudne ulle gänyaolle ngäsangngäsang allan.}{The dog grew up here a long time ago and is always returning.}\allomorph{gudäne}}
\entry{guem}{\headword{guem}\pos{n.}\sensenumber{2}\definition{channel; deep part of riverWalle ddage me kopek. (A valley inside a river.)}}
\entry{gugall}{\headword{gugall}\pos{n.}\sensenumber{2}\definition{type of plant with red, yellow, and white flowers and fruit with small, round seeds; children use the fruit in bamboo blow guns}}
\entry{gugi}{\headword{gugi}\pos{S vi.}\sensenumber{2}\definition{to stand}\example{Angde Kwalde amne we gogän, bogo obo däräng de dangermällnegän a dugiwän.}{When Kwalde got back to the center of the bush, he met his dogs and stood.}\example{Dedme bugimällämän lla da.}{People will be standing there.}\allomorph{gi}\allomorph{gugimällaem}\allomorph{gimälläm}}
\entry{Gugu}{\headword{Gugu}\pos{pn.}\sensenumber{2}\definition{Gugu (toponym)}}
\entry{gugu1}{\headword{gugu1}\pos{n.}\sensenumber{2}\definition{type of big taro}}
\entry{gugu2}{\headword{gugu2}\pos{n.}\sensenumber{2}\definition{row of leaves going up the roof}}
\entry{Gugu Gel}{\headword{Gugu Gel}\pos{pn.}\sensenumber{2}\definition{female personal name}}
\entry{Guim}{\headword{Guim}\pos{pn.}\sensenumber{2}\definition{personal name}}
\entry{gul}{\headword{gul}\pos{n.}\sensenumber{2}\definition{crowd, group, mob; school (of fish)}\example{Bogo lla gul ulle de ikop dägagän.}{He saw the big mob of people.}\example{Ngämo yae a ngämo baba gogezäneyo Ende gul att.}{My mother and my father came from the Ende crowd.}\sensenumber{1}\definition{to accompany}\example{Ngäna Senti bom umllang däga ngämlle gulag e.}{I told Senti to accompany me.}\sensenumber{2}\definition{crowd}\etymology{gul + =ang}\subentry{\headword{gulag}\pos{A vt.}\definition{to accompany}}}
\entry{guli}{\headword{guli}\pos{n.}\sensenumber{2}\definition{type of tree}}
\entry{gulin}{\headword{gulin}\pos{n.}\sensenumber{2}\definition{crab}}
\entry{gull}{\headword{gull}\pos{n.}\sensenumber{2}\definition{netWalle darnen ma dan. Kollba irnen e. (It's to put in the water. For catching fish.)}\example{Mälla da kemibi kollba de nägäddallo gull alle.}{The women caught many fish with the net.}\example{Angde ngäna gull de ik i däga, käza da gull ik i gozenän.}{When I lifted the net up, the crocodile went inside it.}}
\entry{gullabgullab}{\headword{gullabgullab}\pos{mod.}\sensenumber{2}\definition{fat}\example{Ngäna sɨmell de yagnen anggan, käm gullabgullab a.}{I'm looking for a pig, one with a fat belly.}}
\entry{gullba}{\headword{gullba}\pos{n.}\sensenumber{2}\definition{bundle of sago}}
\entry{gullbe}{\headword{gullbe}\pos{kin.}\sensenumber{1}\definition{husband}\example{gullbe da walle}{couple}\example{Bäne gullbe da daden?}{Do you have a husband (lit. does your husband exist)?}\example{Gullbeda däban.}{That's her husband.}\sensenumber{2}\definition{male animal}\example{ddia gullbe, sɨmell gullbe}{buck, boar}\sensenumber{3}\definition{huge, big}\example{Wel gullbe da gongttägän a mägda bo pite de däzinän.}{A big wind blew and lifted his mother's skirt.}\example{Ngämi yu dägaebnalla angde yogoll gullbe da dipliwän abo.}{We (excl.) were cooking them when a rainstorm started.}\sensenumber{3}\definition{married (of a female)}\sensenumber{3}\definition{married (of a female)}\example{Bäne yae auli gullbe peyang daeya?}{How many husbands does your mother have?}\etymology{gullbe + =ang}\subentry{\headword{gullbog}\pos{mod.}\definition{married (of a female)}}\subentry{\headword{gullbe peyang}\pos{mod.}\definition{married (of a female)}}}
\entry{Gullbe Bikme Auma}{\headword{Gullbe Bikme Auma}\pos{pn.}\sensenumber{3}\definition{sacred place of Dareda (near Karama swamp)}}
\entry{Gullbe bo llädayatt}{\headword{Gullbe bo llädayatt}\pos{pn.}\sensenumber{3}\definition{sacred place of Dareda (large hill)}}
\entry{Gullbe bo makollamatt}{\headword{Gullbe bo makollamatt}\pos{pn.}\sensenumber{3}\definition{Dareda's sacred place (hill on the road to Kinkin)}}
\entry{gullbi}{\headword{gullbi}\variant{sp. var. of}{gullbe}}
\entry{gullem}{\headword{gullem}\pos{n.}\sensenumber{3}\definition{snake}\example{Ge ekaklle ulle me, gullem sapasapang a dadeg.}{In this world, there are many kinds of snakes.}\example{Gullem llɨg a däbe agezänan tatuma me.}{A baby snake came out in that washing place.}}
\entry{Gullem suwe}{\headword{Gullem suwe}\pos{pn.}\sensenumber{3}\definition{gardening place; Galo's sacred place (AZ97)}}
\entry{gullem suwetar}{\headword{gullem suwetar}\pos{n.}\sensenumber{3}\definition{type of tree that is used as medicine for snake bites and to repel snakes}}
\entry{gullme}{\headword{gullme}\pos{n.}\sensenumber{3}\definition{type of flowering tree that grows on the riverside in swamps with wood used for firewood and long, hanging red flowers that smell nice}}
\entry{gullme käpang}{\headword{gullme käpang}\pos{n.}\sensenumber{3}\definition{type of spear}}
\entry{gungg}{\headword{gungg}\pos{S vt.}\sensenumber{3}\definition{to marry a widow}\example{Angde Kwalde kuddäll agan, Kidarga ako mosen da bäne mik di nanggugan.}{When Kwalde died, Kidarga married his older brother's (i.e. Kwalde's) widow.}\allomorph{nggug}\allomorph{ngguminy}\allomorph{guminy}}
\entry{Gurel}{\headword{Gurel}\pos{pn.}\sensenumber{3}\definition{male personal name}}
\entry{guwaba}{\headword{guwaba}\pos{n.}\sensenumber{3}\definition{guava tree; water steeped with its leaves is used to wash soresTtam käkäm de kaepnen ma dan källa ine täräp me. (The young leaves are for chewing when one has diarrhea.)}}
\entry{guwo}{\headword{guwo}\pos{n.}\sensenumber{1}\definition{heart}\example{Yesu lla yaba guwo watt gagäll anyke yuwog de däkällttanegnän.}{Jesus cast many evil spirits out from the hearts of men.}\example{Sali bo guwo watt dirom llɨg de ekaekong digezänän.}{Out of Sali's chest came a chirping cassowary chick.}\sensenumber{2}\definition{inside}\example{Bogo dadiweya gall guwo me dämenang dagirnän.}{He was sitting in the canoe.}}
\entry{guzi}{\headword{guzi}\pos{n.}\sensenumber{2}\definition{yabby (type of crayfish)Sambuag ingollae dan. (It's like a shrimp.)}\example{Bogo guzi de daittän.}{She caught a crayfish.}}
\entry{guziguzi}{\headword{guziguzi}\pos{n.}\sensenumber{2}\definition{type of tree}}
\entry{gwaga}{\headword{gwaga}\pos{n.}\sensenumber{1}\definition{type of big tree that grows in the bush with big, edible fruit that are yellow outside, red inside, and stain lips brown; when ripe, it will open and drop the heart-shaped seed}\sensenumber{2}\definition{the sixth stage of sago growth in which the fruits have matured and the pith is drier}\sensenumber{2}\definition{fruit of sago palm}\example{Baba bo sana da gwaga käpang gogon kikib me.}{Dad's sago palm is already producing fruit.}\example{Llɨg kälekäle da sana gwaga käp alle gotongoenegnän.}{The children were playing with the sago fruit.}\subentry{\headword{gwaga käp}\pos{n.}\definition{fruit of sago palm}}}
\entry{gwara}{\headword{gwara}\pos{n.}\sensenumber{2}\definition{lightning}}
\entry{gwazi}{\headword{gwazi}\pos{n.}\sensenumber{2}\definition{type of big taro}}
\entry{gwälläd}{\headword{gwälläd}\pos{n.}\sensenumber{2}\definition{type of tree}}
\entry{gwängäm}{\headword{gwängäm}\pos{***}\sensenumber{2}\definition{sacrifice}}
\entry{gwell}{\headword{gwell}\pos{n.}\sensenumber{1}\definition{rule, law}\example{Moses adawatta ge gwell de dadräbnegän bibra, bina tikop a llokollokott dag Adi bäne Mer Eka malaem e.}{Moses wrote these laws for you all, because your hearts are stubborn to obey God's gospel.}\sensenumber{2}\definition{secret}\example{Gwell eka de ngäna llɨtɨt eran.}{I'm telling a secret story.}}
\entry{Gwen}{\headword{Gwen}\pos{pn.}\sensenumber{2}\definition{female personal name}}
\lettersection{H}
\entry{Hannah}{\headword{Hannah}\pos{pn.}\sensenumber{2}\definition{female personal name}}
\entry{hedmasta}{\headword{hedmasta}\pos{n.}\sensenumber{2}\definition{headmaster}\etymology{from Englishheadmaster}}
\entry{Helen}{\headword{Helen}\pos{pn.}\sensenumber{2}\definition{female personal name}}
\entry{Hiden}{\headword{Hiden}\pos{pn.}\sensenumber{2}\definition{female personal name}}
\entry{hundred}{\headword{hundred}\variant{sp. var. of}{andred1}}
\lettersection{I}
\entry{i}{\headword{i}\pos{S vt.}\sensenumber{2}\definition{to weave; interlock}\example{Bogo nyäng de i eran.}{She is weaving the basket.}\example{Tater de aya diwän?}{Who wove the mat?}\example{Ngämo yae era lla ngänäm polle dinignän.}{My mother, she used to build sturdy fences like a man.}\sensenumber{2}\definition{plain weaving pattern}\subentry{\headword{i abal}\pos{n.}\definition{plain weaving pattern}}}
\entry{Iamega}{\headword{Iamega}\variant{sp. var. of}{Yamega}}
\entry{Ib}{\headword{Ib}\pos{pn.}\sensenumber{2}\definition{female personal name}}
\entry{ibe}{\headword{ibe}\pos{n.}\sensenumber{2}\definition{mist; fog}}
\entry{ibeny}{\headword{ibeny}\pos{S vt.}\sensenumber{2}\definition{to plant}\example{Bogo ibeny eran.}{He is planting.}\example{Biye de bongo ai dan ibik alle nibe.}{You can plant taro with the ibik stick.}\example{Ttängäm e gongttägeyo, misdae up daebegaeya ibebatta.}{They arrived at the garden; only bananas were planted there.}\allomorph{ibeb}\allomorph{ibe}}
\entry{Ibeti}{\headword{Ibeti}\variant{sp. var. of}{Ibetty}}
\entry{Ibetty}{\headword{Ibetty}\pos{pn.}\sensenumber{2}\definition{female personal name}}
\entry{ibi1}{\headword{ibi1}\pos{pers. pron.}\sensenumber{2}\definition{we (first person nonsingular inclusive pronoun, nominative form)}\example{Koe, ai dan ibi gänyme beyagernalla.}{Koe, it's good the two of us will be living here.}\sensenumber{2}\definition{genitive form of ibi}\example{Iba nag ttoen a alla ingoll utale ballän?}{How far will our friendship go?}\sensenumber{2}\definition{accusative form of ibi}\example{Ai dan, käza da ddone ibim beyaddägän.}{It's alright, the crocodile won't bite us.}\sensenumber{2}\definition{dative form of ibi}\example{Bongo nalle sana ma, ngäna tudi ma balle ibra kollba ma.}{You go for sago, and I will go fishing to get fish for us.}\subentry{\headword{iba}\pos{pers. pron.}\definition{genitive form of ibi}}\subentry{\headword{ibim}\pos{pers. pron.}\definition{accusative form of ibi}}\subentry{\headword{ibra}\pos{pers. pron.}\definition{dative form of ibi}}}
\entry{ibi2}{\headword{ibi2}\pos{S vi.}\sensenumber{1}\definition{to go}\example{Ngämi ttongo ebdo me Karama we gobllab inu wi.}{One day, we (excl.) went to Karama to sleep.}\example{Ngäna maket e ibi allan.}{I am going to the market.}\example{Mälla da gobällän ttängäm e melem e.}{The women went to the garden to work.}\example{Duli abäll bibi.}{You all go away.}\sensenumber{2}\definition{to walk}\example{Ge lla da udu peyang ibiag dan.}{This man walks with a stick.}\example{Män kälsäre da dägabällän a ibnin de gongkamän.}{The little girl stood up and began to walk.}\sensenumber{3}\definition{let's go}\sensenumber{3}\definition{footprint}\example{Däräng aba mäda Kwalde bäne ibiatt de dongkollmälleyo.}{The dogs followed their owner Kwalde's footprints.}\allomorph{yibi}\allomorph{bällab}\allomorph{bälläb}\allomorph{ib}\allomorph{bäll}\allomorph{bllaeb}\allomorph{bllab}\allomorph{bll}\allomorph{bollab}\allomorph{ll}\allomorph{r}\allomorph{lle}\etymology{ibi + =att}\subentry{\headword{ibiatt}\pos{n.}\definition{footprint}}}
\entry{ibik}{\headword{ibik}\pos{n.}\sensenumber{3}\definition{digging stickDäm idnen ma da. Llo popoatt otät ibenen e. (It's for uprooting plants. Sharpened wood for planting food.)}\example{Tätäm ibik a ddokddok gogon.}{Yesterday, the digging stick became blunt.}\example{Biye de bongo ai dan ibik alle nibe.}{You can plant taro with the ibik stick.}}
\entry{Ibikang}{\headword{Ibikang}\pos{pn.}\sensenumber{3}\definition{Ibikang (Kawam-speaking settlement of Wim; not far from Limol (AY96))}}
\entry{ibira}{\headword{ibira}\variant{var. of}{ibra}}
\entry{Ibru}{\headword{Ibru}\pos{n.}\sensenumber{3}\definition{Hebrew}\etymology{from EnglishHebrew}}
\entry{id}{\headword{id}\variant{sp. var. of}{yid}}
\entry{idaida}{\headword{idaida}\pos{n.}\sensenumber{3}\definition{type of game played outside in open space}}
\entry{Idan}{\headword{Idan}\pos{pn.}\sensenumber{3}\definition{male personal name}}
\entry{idän}{\headword{idän}\pos{S vt.}\sensenumber{3}\definition{to pick, harvest; dig up, uproot}\example{Ubi mɨnyi mätta de bidaebneyo.}{They will dig up the yams and bring them over.}\example{Ada wayati da moko alle dädeyo.}{[You] pick watermelons as you please.}\allomorph{id}\allomorph{d}\allomorph{idd}}
\entry{Iden}{\headword{Iden}\pos{pn.}\sensenumber{3}\definition{Eden}}
\entry{Idi}{\headword{Idi}\pos{pn.}\sensenumber{3}\definition{Idi language (Pahoturi River language spoken in Dimsisi, Sibidiri, Dimiri, Iblamand, and Biram)}}
\entry{idoidog}{\headword{idoidog}\pos{n.}\sensenumber{3}\definition{harpoon}}
\entry{Idugoe}{\headword{Idugoe}\pos{pn.}\sensenumber{3}\definition{male personal name}}
\entry{idd}{\headword{idd}\pos{n.}\sensenumber{3}\definition{ghost}\example{Ubi ada ka bogo idd daeya.}{They thought she was a ghost.}\sensenumber{3}\definition{afterlife}\subentry{\headword{idd ma}\pos{n.}\definition{afterlife}}}
\entry{iddi}{\headword{iddi}\pos{n.}\sensenumber{3}\definition{soup}}
\entry{iddnge}{\headword{iddnge}\pos{n.}\sensenumber{3}\definition{type of coconut palm}}
\entry{iddob}{\headword{iddob}\pos{n.}\sensenumber{3}\definition{night}\example{Angde iddob gogon, ngäna inu wi dalle.}{When night came, I went to sleep.}\example{Bäne moko daden ibi kollba ma sisiri beyareya, ako iddob abal e bonttägmällnalla.}{You want for us to go fishing now, so we will arrive late at night.}\example{Däbe ttall a iddob me guddällän.}{That wallaby arrived at night.}\sensenumber{3}\definition{midnight}\example{Ttongdae mɨnyi awi alle do iddob amnong a ttongo ako iddob amnong alle do bädab bagirnän.}{One will stay from evening to midnight, and another will stay from midnight until dawn.}\subentry{\headword{iddob amnong}\pos{n.}\definition{midnight}}}
\entry{iddoiddob}{\headword{iddoiddob}\pos{n.}\sensenumber{3}\definition{type of tree}}
\entry{iddpo}{\headword{iddpo}\pos{n.}\sensenumber{3}\definition{clothing, clothes}\example{Ttongo mälla da iddpo de nällpäganegnan.}{One woman was hanging the clothes.}\example{Obo iddpo da pällämpälläm a gombenmällnegnän.}{His clothes were white and shining.}\etymology{from Kawamidd + ttupo, lit. 'devil skin'}}
\entry{igi}{\headword{igi}\pos{loc.}\sensenumber{1}\definition{bottom}\example{Tuk me gogäbaebän dowede ada gogon, ine ik gongkäbägän do igi abal.}{On top he jumped like this into the water, down to the very bottom.}\example{Igi me era däbe namllam.}{[You] grab the one on the bottom.}\sensenumber{2}\definition{underneath}\example{Ge dädär de igi me era ekaklle da dakonenegän.}{These rocks underneath, the soil covered them.}\sensenumber{2}\definition{underwearLla o mällayaba igi me mättnan ma kaptte. (Men's or women's clothing worn underneath.)}\example{Bongo bäne igi pite de zäme amättalle?}{Did you already put on your underpants?}\sensenumber{2}\definition{under, beneath}\example{Ge lla da aeya ngämo imnealle beyarän, bogo mängallang abal dan, ngäna obo igiigi me dan.}{The man that will come after me, he is very powerful; I am beneath him.}\subentry{\headword{igi pite}\pos{n.}\definition{underwearLla o mällayaba igi me mättnan ma kaptte. (Men's or women's clothing worn underneath.)}}\subentry{\headword{igiigi}\pos{loc.}\definition{under, beneath}}}
\entry{iid}{\headword{iid}\variant{sp. var. of}{yid}}
\entry{ik}{\headword{ik}\pos{loc.}\sensenumber{2}\definition{inside}\example{Bogo dindugän do kukiny mama de ikop dägagän dibaeya däbe ik e gotärakän do gotägolän.}{He ran to a pile of tall grass that he saw, went inside, and hid there.}\example{Lama wa Mana ubi kandärmang me goensegäneyo adawatta Emi aoli kuddäll de ine ik me ikop dägagän.}{Lama and Mana were both feeling sorry because they saw that Emi almost died in the water.}\example{Bogo poper gogon, ma ik atta gogezänän wälläng e dindugän.}{He was shocked, got out from inside, and ran into the bush.}}
\entry{ikikib}{\headword{ikikib}\pos{n.}\sensenumber{2}\definition{dizziness}\example{Ngänäm ikikib da nallan.}{My head is spinning (lit. dizziness has gotten me).}}
\entry{ikllo}{\headword{ikllo}\pos{n.}\sensenumber{2}\definition{smoke}\example{Yu ikllo da ttongo eka ngangema ttoen dan.}{Smoke is one way of sending a message.}\sensenumber{2}\definition{grey; the color of smoke}\subentry{\headword{iklloikllowang}\pos{col.}\definition{grey; the color of smoke}}}
\entry{ikoll}{\headword{ikoll}\pos{n.}\sensenumber{2}\definition{incident, problem, trouble}\example{Ubi abo diba ikoll me ma we dängäsmällän.}{They then returned home from the incident.}\example{Bongo adawede eka dändärmeny agnalle, ikoll a nagan.}{You weren't listening, so it came back to bite you (lit. 'the trouble got you').}}
\entry{ikop}{\headword{ikop}\pos{n.}\sensenumber{1}\definition{eyeTtongo toko lla pätt me ttoen ikop e. (A thing on the top part of a person's body for seeing.)}\example{Llamäg o mällause täräp me iba ikop a mɨnyi säremang bognegän.}{During old age, our (incl.) eyes will go dim.}\sensenumber{2}\definition{to see}\example{Lla da lla de ikop eran.}{A person sees a person.}\example{Ngäna ddia de ikop nägagan a nazuan.}{I saw a deer and shot it.}\sensenumber{3}\definition{to look}\example{Ikopag imne we.}{(You) look back.}\example{Omägag a obo nyäng e ikop gogon.}{The fortune-teller looked into her bag.}\example{Lla da däbe ikop dägneyo käza de.}{People were looking at that crocodile.}\example{Ge lla de ikop nägayaebeyo!}{(You all) look at these people!}\sensenumber{3}\definition{eyeglasses, spectacles; goggles}\example{kämbmeny ma ikop glas}{diving goggles}\sensenumber{3}\definition{eyeball}\sensenumber{3}\definition{eyelash}\sensenumber{3}\definition{pupil}\example{Steven a Kange, däräng de dällädeyo, pakos de dängkäneyo, ikop ku att de, däräng a ttam gogän.}{Steven and Kange grabbed the dog, pulled out the arrow from his pupil, and the dog lived.}\sensenumber{3}\definition{fainting, losing consciousness}\example{Ubi mɨnyi ikop särem atta buspullnän nyongo me, adawatta ma we ibima da utale abal dan.}{They will faint and fall on the road because going home is a very long journey.}\sensenumber{3}\definition{blind}\example{Däbe ikop songgorag lla da ada eka dägagän, "Ngämo moko da mermer ikopang e dan."}{That blind man said, "I wish to see properly."}\sensenumber{3}\definition{to watch, look after, patrol}\example{sip ikop täbaebag}{shepherd}\sensenumber{3}\definition{eyelid}\sensenumber{3}\definition{unseen, invisible}\example{Eiz itrell a ikopmeny dan, lla bo mam me dan.}{HIV is invisible; it's in a person's blood.}\example{Ause da kollba we gopnaeyän a ine ik mi ikopmeny gogän.}{The woman turned into a fish and disappeared into the water.}\sensenumber{1}\definition{to look, watch}\example{Llɨg a wa män a ubi ikop goddägneyo.}{The boy and the girl looked at each other.}\sensenumber{2}\definition{to look, watch}\example{Masar era ngämim ikop deyaddägnän.}{Grandfather, he was watching us.}\sensenumber{2}\definition{wink; eyebrow raise}\sensenumber{2}\definition{to peek}\example{Llamda da up golloll me gotogolän a ikop su gognän känyertto ubira ikop e ami obo up de gämäll alle däglebneyo.}{The old man hid behind the banana tree and peeked quietly to see those who were stealing his banana.}\sensenumber{2}\definition{to watch}\sensenumber{2}\definition{to watch}\example{Ubi ikopikop gogaebne.}{They were watching.}\example{Ge lla da obom ikoikop eran.}{This person is watching him.}\allomorph{ikoikop}\allomorph{ikop}\etymology{lit. 'dark eyes'}\subentry{\headword{ikop glas}\pos{n.}\definition{eyeglasses, spectacles; goggles}}\subentry{\headword{ikop käp}\pos{n.}\definition{eyeball}}\subentry{\headword{ikop kom}\pos{n.}\definition{eyelash}}\subentry{\headword{ikop ku}\pos{n.}\definition{pupil}}\subentry{\headword{ikop särem}\pos{n.}\definition{fainting, losing consciousness}}\subentry{\headword{ikop songgorag}\pos{mod.}\definition{blind}}\subentry{\headword{ikop täbab}\pos{S vt.}\definition{to watch, look after, patrol}}\subentry{\headword{ikop ttoe}\pos{n.}\definition{eyelid}}\subentry{\headword{ikopmeny}\pos{mod.}\definition{unseen, invisible}}\subentry{\headword{ikop ddäddäg}\pos{S vi.}\definition{to look, watch}}\subentry{\headword{ikop sära}\pos{n.}\definition{wink; eyebrow raise}}\subentry{\headword{ikop su}\pos{A vi.}\definition{to peek}}\subentry{\headword{ikoikop}\pos{v.}\definition{to watch}}\subentry{\headword{ikopikop}\pos{A vi. \textbackslash& vt.}\definition{to watch}}}
\entry{ikopse}{\headword{ikopse}\pos{n.}\sensenumber{2}\definition{prayer}\example{Känaebag, ngäna mɨnyi ikopse we beyar, ikopse ttamänatt me mɨnyi Upiara we bongos.}{Tomorrow, I am going to go pray, and after I finish praying, I will return to Upiara.}\sensenumber{2}\definition{church, temple}\example{Bongo mɨnyi mikutt mi iba ikopse ma de nälläk?}{Will you burn our (incl.) church in anger?}\sensenumber{2}\definition{no definition provided}\example{Oba masamasar abaene ttoen de zuu aba ikopse ma ma me däntamenyaemneyo.}{They learned the ways of their ancestors in the Jewish temple.}\subentry{\headword{ikopse ma}\pos{n.}\definition{church, temple}}}
\entry{ikopsi}{\headword{ikopsi}\variant{sp. var. of}{ikopse}}
\entry{ikrol}{\headword{ikrol}\pos{n.}\sensenumber{2}\definition{ash}\example{Ikrol da duwamänän.}{The ash dissipated.}}
\entry{Ilaeza}{\headword{Ilaeza}\pos{pn.}\sensenumber{2}\definition{Elijah}}
\entry{ileben}{\headword{ileben}\variant{sp. var. of}{eleben}}
\entry{ili}{\headword{ili}\pos{rel. pron.}\sensenumber{1}\definition{where}\example{Ngämi ili ibi eran mondre ma dan.}{It's where we (excl.) go gardening.}\example{Da duli melem e ballän, ili Mospi o ili ballän, ede Ingglis eka de mɨnyi bäpanyän.}{If she goes away for work, in Port Moresby or wherever she goes, then she will speak English.}\sensenumber{2}\definition{where}\example{Bongo ili kanyekanye gogäne?}{Where have you travelled to?}\sensenumber{2}\definition{ablative form of ili}\example{Da ilibattäm lla da bogon, da ada Upiara watt, ngäna obom era ede mɨnyi pällämpälläm eka walle eka bäntameny.}{If the person is from elsewhere (i.e. non-Ende villages), then I would speak with him using English.}\etymology{ili + =mattäm}\subentry{\headword{ilibattäm}\pos{pron.}\definition{ablative form of ili}}}
\entry{Imanuel}{\headword{Imanuel}\pos{pn.}\sensenumber{2}\definition{Imanuel}}
\entry{imne}{\headword{imne}\pos{loc.}\sensenumber{1}\definition{rear, behind, back}\example{Gall me bogo ngattong e, ngäna imne we, käza da amne me daeya.}{She was in front of the canoe, I was in the rear, and the crocodile was in the middle.}\example{Ddia da säre imne gogon, kottllam a ngattong gongttägän ttamän ma ngättma we.}{Sadly, Deer was behind; Turtle had arrived to the finish line first.}\example{Ddob lla da ami imne dongkollmällaemneyo lel gognegnän}{The others who were following in the back were scared.}\example{Ngäna bäne imne ballne.}{I will walk behind you.}\sensenumber{2}\definition{afterwards, after, later}\example{Ngäma mäda gagäll gogon, ngäma mäg ako imne gagäll gogon.}{Our (excl.) father passed away; later, our mother passed away too.}\example{Imne duwem ag, ngattong ttang de anggälläm.}{Eat after you wash your hands first.}\sensenumber{2}\definition{after}\example{Imnong e ubi dädmawän sipel e.}{After a while, they sat down to rest.}\sensenumber{1}\definition{from behind}\example{Yäbäd imneimne gongbeabogän ddone dingmenän.}{Sun chased him from behind, but did not reach him.}\sensenumber{2}\definition{behind, late; later, after}\example{Ngämo nag zäme angttägan be ngäna mɨnyi imneimne bongttäg.}{My friend already arrived, but I will arrive later.}\sensenumber{2}\definition{after, later}\example{däbe imnealle}{after that}\example{kumuddäga pazi imnealle}{three years later}\etymology{imne + =ang}\subentry{\headword{imnong}\pos{mod.}\definition{after}}\subentry{\headword{imneimne}\pos{adv.}\definition{from behind}}\subentry{\headword{imnealle}\pos{post.}\definition{after, later}}}
\entry{imomdae}{\headword{imomdae}\pos{n.}\sensenumber{1}\definition{truth}\example{Ngäna bibim imomdae de umllang anggan.}{I am telling you (pl.) truths.}\example{Bongo imomdae de ekalle.}{You are speaking the truth.}\sensenumber{2}\definition{true, real, actual}\example{Ge gänyan Ende imomdae eka dan.}{This here is the true Ende language.}\example{Ubi anykeanyke dae ikop dägageyo, ubi ddone imomdae up pätt de ikop dägageyo.}{They saw only the reflection; they did not see the actual banana plant.}\example{Sos me ada eka anggan, "Iba pätt a God bo imomdae giddoll ma dan."}{They say in church, "Our (incl.) bodies is where God truly resides."}\sensenumber{3}\definition{correct, right}\example{Bongo imomdae abal de ekalle ada, Adi da ttongdae dan.}{You are very right in saying that God is one.}\sensenumber{4}\definition{to believe}\example{Bogo ddone komlla näkäpang bogon obo tikop me, be imomdae bägagän.}{He will not be doubtful (lit. two-minded) in his heart, but will believe it.}\sensenumber{4}\definition{honest}\sensenumber{4}\definition{honest}\example{Ngäma umllang dan bongo imomdae ttättlle panypenyang dan.}{We (excl.) know that you speak honestly.}\sensenumber{4}\definition{faith, religion}\example{Imomdae ttoen peyang ikopse dae ada ingollang gagäll anyke de bäkällttanegän.}{Only with faith and prayer can the evil spirits be cast out.}\sensenumber{4}\definition{believer, faithful}\example{Tämamae ttoen a kuddukuddull dag oblle aeya imomdae ttoenang bognän.}{Everything is easy for he who may be a believer.}\etymology{ngasnges + =ang}\subentry{\headword{imomdae ngasngesang}\pos{mod.}\definition{honest}}\subentry{\headword{imomdae ttättlle}\pos{mod.}\definition{honest}}\subentry{\headword{imomdae ttoen}\pos{n.}\definition{faith, religion}}\subentry{\headword{imomdae ttoenang}\pos{n.}\definition{believer, faithful}}}
\entry{imonzimonz}{\headword{imonzimonz}\pos{n.}\sensenumber{4}\definition{tag (game)}\example{Mosenda kaptte de dängllämnegnän a mänyanda bi komllaebe tupi imonzimonz gotngoeneyo walle ik mi.}{The oldest washed the clothes and the younger two played tag in the water.}}
\entry{imullgoe}{\headword{imullgoe}\pos{S vt.}\sensenumber{4}\definition{to drop, push, make fall}\example{Bogo obom daimullgoeyän.}{She pushed him over.}}
\entry{Ina}{\headword{Ina}\pos{pn.}\sensenumber{4}\definition{female personal name}}
\entry{inam}{\headword{inam}\pos{S vt.}\sensenumber{1}\definition{to cover}\example{Ekaklle kok de mɨnyi enanae binamän.}{Earth will completely cover the moon (i.e. during a lunar eclipse).}\sensenumber{2}\definition{to weigh down, press down}\example{Ngäna sɨmell de dinam ekaklle we.}{I weighed the pig down to the ground.}}
\entry{Inapa}{\headword{Inapa}\pos{pn.}\sensenumber{2}\definition{male personal name}}
\entry{Inawa}{\headword{Inawa}\pos{pn.}\sensenumber{2}\definition{male personal name}}
\entry{inbo}{\headword{inbo}\pos{kin.}\sensenumber{1}\definition{brother-in-law (woman's husband's brother)}\sensenumber{2}\definition{sister-in-law (man's elder brother's wife or woman's husband's younger sister; reciprocal)}}
\entry{inbunatt}{\headword{inbunatt}\pos{n.}\sensenumber{2}\definition{mallet fish (big, lives in creek, lean)}}
\entry{Indonesia}{\headword{Indonesia}\pos{pn.}\sensenumber{2}\definition{Indonesia}}
\entry{indrang}{\headword{indrang}\pos{n.}\sensenumber{1}\definition{light}\example{Ngäna ngämo tos indrang de dɨs.}{I put out my flashlight.}\example{Indrang a ngämo dowae me gogllaeyän.}{The light shined close to me.}\sensenumber{2}\definition{luminous, bright}\example{ttongo mer indrang ag}{one very bright morning}\example{Yäbäd a mɨnyi säremang bogon ebdo me a kok a ade mɨnyi ddone indrang bognän iddob me.}{The sun will be dark during the day and also, the moon will not be bright at night.}\sensenumber{3}\definition{to show, reveal, illuminate, bring to light}\example{Ngäna Adi lla bo mer de indrang eran.}{I am showing God's goodness.}\example{Aeya ge eka de indrang eran?}{Who is announcing this news?}\sensenumber{3}\definition{clearly}\example{Mak ibim indraindrang umllang anggan ada Yesu era Adi bo llɨg aenen.}{Mark told us (incl.) clearly that Jesus is the Son of God.}\subentry{\headword{indraindrang}\pos{adv.}\definition{clearly}}}
\entry{indre}{\headword{indre}\pos{n.}\sensenumber{3}\definition{type of tree that grows in the grassland with edible brown-yellow fruit and good, long-burning firewood}\sensenumber{3}\definition{type of snakeTubutubu da. Kuddäll e lla ddäddägang da. (It's long. It bites people to death.)}\subentry{\headword{indre gullem}\pos{n.}\definition{type of snakeTubutubu da. Kuddäll e lla ddäddägang da. (It's long. It bites people to death.)}}}
\entry{indugoeg}{\headword{indugoeg}\pos{S vt.}\sensenumber{3}\definition{to command}\example{Bongo gagäll anyke, ngäna bam indugoeg nallan, agezän obo guwo watta.}{You evil spirit, I command you to come out of his heart.}\example{Mägda män de naindugoegan kaptte mättmätt e.}{The girl's mother commanded her to put on clothes.}\allomorph{indugoe}}
\entry{ine}{\headword{ine}\pos{n.}\sensenumber{1}\definition{water; liquidNane ma dan. Wätät yuma dan a ddob gällämnan ttoen ngasnen ma dag. (It's to drink. It's for cooking and also for washing things.)}\example{nge ine}{coconut water}\example{Ge ine da nanott dan.}{This water has been drunk.}\example{Bogo mängalae gogbaebän ine ik e oblle ngämingg e.}{She quickly jumped into the water to help him.}\sensenumber{2}\definition{alcoholic beverage}\example{Ubira mer näkäpngon a ddone dan adawatta ubi ine de nanen amallo.}{They don't have good judgment because they are drinking alcohol.}\sensenumber{2}\definition{spring (water source)Yäbäd bäng me malla baddbeddag dan. (It doesn't dry up in the dry season.)}\sensenumber{2}\definition{mud, clayDdage kälekäle me ine baddnenatt wärpir. (In small rivers, slimy mud from water drying up.)}\sensenumber{2}\definition{well (water source)Ine nanen ma kopek. (A hole for drinking water.)}\sensenumber{2}\definition{bucket, water containerIne nanenma pane kutt. (A hard vessel for drinking water)}\example{Ngäna bänene ine kutt de yuwetyuwet yuwenyan.}{I borrowed your bucket.}\sensenumber{2}\definition{well (water source)}\example{Bongo ine ma nalle oba bongo ine neyatan.}{You, go to the well and fetch water.}\sensenumber{2}\definition{seasick}\allomorph{yine}\etymology{lit. 'water poop'}\subentry{\headword{ine bib}\pos{n.}\definition{spring (water source)Yäbäd bäng me malla baddbeddag dan. (It doesn't dry up in the dry season.)}}\subentry{\headword{ine källa}\pos{n.}\definition{mud, clayDdage kälekäle me ine baddnenatt wärpir. (In small rivers, slimy mud from water drying up.)}}\subentry{\headword{ine kup}\pos{n.}\definition{well (water source)Ine nanen ma kopek. (A hole for drinking water.)}}\subentry{\headword{ine kutt}\pos{n.}\definition{bucket, water containerIne nanenma pane kutt. (A hard vessel for drinking water)}}\subentry{\headword{ine ma}\pos{n.}\definition{well (water source)}}\subentry{\headword{ine takmäll}\pos{mod.}\definition{seasick}}}
\entry{ine konkonang}{\headword{ine konkonang}\pos{n.}\sensenumber{2}\definition{beer}\etymology{lit. 'crazy water'}}
\entry{ine mallmell}{\headword{ine mallmell}\pos{n.}\sensenumber{2}\definition{type of snakeGullem tubutubu da ma ik me giddollag da. Kuddäll e lla ddäddägang da. (It's a long snake that lives in houses. It bites people to death.)}}
\entry{inkäm}{\headword{inkäm}\pos{n.}\sensenumber{1}\definition{voice}\example{Ngämi bam ge ddobae ddonddo nalla adawatta bäne ge mer abal inkäm dan.}{We are very proud of you because of your very nice voice.}\sensenumber{2}\definition{neck}\example{Obo inkäm a tubutubu dan.}{His neck is short.}\sensenumber{2}\definition{voice}\sensenumber{2}\definition{throat}\example{Pa bo inkäm tär a ddäddäg ma dan.}{A bird's throat is edible.}\subentry{\headword{inkäm tän}\pos{n.}\definition{voice}}\subentry{\headword{inkäm tär}\pos{n.}\definition{throat}}}
\entry{inkätt}{\headword{inkätt}\pos{n.}\sensenumber{1}\definition{neck; throat}\example{Bogo Zon ine kämbägag bom inkätt danddugän.}{He sliced through the throat of John the Baptist.}\sensenumber{2}\definition{hollow of drumAmne me alläp pekang. (The hollow center of the drum.)}}
\entry{inmol}{\headword{inmol}\pos{n.}\sensenumber{2}\definition{type of small tree that grows in the bush along creeks; used to weave grass skirts after being pounded}}
\entry{inmoll}{\headword{inmoll}\pos{S vt.}\sensenumber{1}\definition{to cover}\example{Ddapall a säremang gogon a deyainmollän.}{The sky got dark and overcast.}\sensenumber{2}\definition{to step on; vanquish}\example{Ngäna bäne mäkang lla de mängallmeny bägneg a bäne ttäle koll igiigi me bowattäll ubira inmoll e.}{I will make your enemies powerless and put them beneath your feet, vanquished.}}
\entry{inngoeinngoe}{\headword{inngoeinngoe}\pos{mod.}\sensenumber{2}\definition{shaky}\example{Ngämi tätäm era inngoeinngoe mo dae daupeya.}{Yesterday, we (excl.) crossed a shaky bridge.}}
\entry{inpiak}{\headword{inpiak}\pos{n.}\sensenumber{2}\definition{whistling kiteKuddäll ddädägag pa dan. (It's a bird of prey.)}}
\entry{Inpiakma}{\headword{Inpiakma}\pos{pn.}\sensenumber{2}\definition{Inpiakma (toponym)}}
\entry{Inpir}{\headword{Inpir}\pos{pn.}\sensenumber{2}\definition{Makayam/Tirio language}}
\entry{intot}{\headword{intot}\pos{n.}\sensenumber{2}\definition{brow bone}\sensenumber{2}\definition{eyebrow}\subentry{\headword{intot kom}\pos{n.}\definition{eyebrow}}}
\entry{inttemängg}{\headword{inttemängg}\pos{S vi.}\sensenumber{1}\definition{to part ways, leave}\example{Eka tamenyatt me, ubi guinttemänggeyo a oba ma we deyareyo.}{After discussing, they parted ways and went to their own homes.}\sensenumber{2}\definition{to leave, see off, release, set free}\example{Kottllam bom dätrameyo bem ulle we dainttemonggeyo.}{They carried the turtle to the ocean and set him free.}\allomorph{inttemä}\allomorph{inttoem}\allomorph{inttem}\allomorph{inttemo}}
\entry{inu}{\headword{inu}\pos{S vi.}\sensenumber{1}\definition{to sleep}\example{Ngäna inunin allan.}{I am sleeping.}\example{Gänyme inuag!}{[You] sleep here!}\example{Bogo gotarnän pollon me.}{He was sleeping in a bush.}\sensenumber{2}\definition{sleep}\example{Bogo inu atta angällbänan.}{He awoke from sleep.}\sensenumber{3}\definition{asleep, sleeping}\example{Bundae inu dan.}{Bundae is asleep.}\sensenumber{4}\definition{night}\example{Dädme ngämi kumuddäga inu gogmam.}{We (excl.) stayed there three nights.}\sensenumber{4}\definition{fast asleep, dead asleep}\example{Llamäg a inu kuddäll daeya.}{The old man was fast asleep.}\sensenumber{4}\definition{to guard in one's sleep, sleep with}\example{Bogo gotarnän ma da däbe yunu danteralle.}{She slept over, guarding that house.}\sensenumber{4}\definition{sleepy}\example{Ngäna bam inuinu nallan.}{I'm making you sleepy.}\allomorph{otar}\allomorph{wotar}\allomorph{yinu}\subentry{\headword{inu kuddäll}\pos{mod.}\definition{fast asleep, dead asleep}}\subentry{\headword{inu tanter}\pos{S vt.}\definition{to guard in one's sleep, sleep with}}\subentry{\headword{inuinu1}\pos{adv.}\definition{sleepy}}}
\entry{inuinu2}{\headword{inuinu2}\pos{n.}\sensenumber{4}\definition{white-faced robin}}
\entry{inungoe}{\headword{inungoe}\pos{S vt.}\sensenumber{4}\definition{to shake (when dancing)}\example{Bongo nainungoenalleǃ}{You were shaking!}}
\entry{Ingglis}{\headword{Ingglis}\pos{pn.}\sensenumber{4}\definition{English language}}
\entry{ingglis}{\headword{ingglis}\variant{sp. var. of}{Ingglis}}
\entry{inggoe}{\headword{inggoe}\variant{dial. var. of}{inggol}}
\entry{inggoemeny}{\headword{inggoemeny}\pos{S vi.}\sensenumber{1}\definition{to joke}\example{Ende walle inggoemeny amalla.}{We are making jokes in Ende.}\example{Ngäna Ingglis eka walle de eka bogne, buinggoemenyne.}{I will speak in English, making jokes.}\sensenumber{2}\definition{to blaspheme; insult, mock}\example{Bogo Adi bom inggoemeny eran.}{He is blaspheming God.}\example{Sabi tamenyang a inggoemeny eka de däpanyaemneyo.}{The teachers of the law were saying insults.}\allomorph{inggoe}}
\entry{inggol}{\headword{inggol}\pos{S vi.}\sensenumber{2}\definition{to move}\example{Mälla da angde ikop dägagän, llɨg bo dowae e guinggolän.}{When the woman saw it, she moved closer to the boy.}\allomorph{inggoemeny}\allomorph{inggul}\allomorph{inggui}\allomorph{inggolnen}\allomorph{inggoenen}\allomorph{inggolmäll}\allomorph{inggoemäll}}
\entry{ingguimeny}{\headword{ingguimeny}\variant{sp. var. of}{inggoemeny}}
\entry{ingoll}{\headword{ingoll}\pos{n.}\sensenumber{1}\definition{face}\example{Angde bongo bäne mälla bo ingoll papa eralle, bongo God bo ingoll papa eralle.}{When you hit your wife in the face, you are hitting God in the face.}\example{Ubi obo ingoll de iddpo alle dämllaeyo.}{They tied his face with cloth (i.e. they blindfolded him).}\sensenumber{2}\definition{front}\example{Ttongo lla da agbänmällnan a ttongo lla da nindugan obo ingoll me.}{One man was jumping and another man ran in front of him.}\example{Ulle binang itrel peyang lla bo ingoll alle eka tameny a mudan.}{Don't converse in front of a very sick man.}\allomorph{yingoll}}
\entry{=ingoll}{\headword{=ingoll}\pos{n. cl.}\sensenumber{2}\definition{similative clitic; like}\example{Bänz a pam nidol ingoll dagaeya.}{The mosquitoes were like needles.}\example{Täme da käza ingoll ddäddäg be ap me giddollag dan.}{The goanna is crocodile-like animal but it lives in the grassland.}\allomorph{yingoll}}
\entry{ingong}{\headword{ingong}\pos{n.}\sensenumber{1}\definition{dance}\example{ingong täräp}{dance time}\sensenumber{2}\definition{sing-sing}\sensenumber{3}\definition{to dance}\example{Obo moko da ingong e dan.}{He wants to dance.}\example{Ngäna gänyaolle bengne.}{I will dance this way.}\example{Ngäna bam ingongang nallan.}{I am making you dance.}\example{Lla da ada dänyängnän menae me adawatta bogo ngattong abal gongttägän ttamän ma ngättma we.}{People were dancing like this on the sides because he got to the finish line first.}\sensenumber{3}\definition{dancing around}\example{Bogo ingnenignen giddollag dan.}{He lives dancing about.}\example{Ngäna ingnenignenang ibi allan gänyaolle.}{I am coming this way dancing.}\allomorph{ng}\allomorph{ing}\subentry{\headword{ingneningnen}\pos{adv.}\definition{dancing around}}}
\entry{ioläm}{\headword{ioläm}\pos{n.}\sensenumber{3}\definition{type of bird}}
\entry{ip}{\headword{ip}\pos{n.}\sensenumber{3}\definition{type of tree that grows in the grassland with bark that is chewed and sap used as an adhesive or poured on a spear to strengthen it}}
\entry{Ipott}{\headword{Ipott}\pos{pn.}\sensenumber{3}\definition{Ipott (on the road to Bisuaka near Limol ma kuddäll)}}
\entry{Iräm}{\headword{Iräm}\pos{pn.}\sensenumber{3}\definition{creek and washing place (in Limol, AZ95)}\sensenumber{3}\definition{Iräm creek}\sensenumber{3}\definition{Iräm washing place}\subentry{\headword{Iräm ine}\pos{pn.}\definition{Iräm creek}}\subentry{\headword{Iräm tatuma}\pos{pn.}\definition{Iräm washing place}}}
\entry{irängän}{\headword{irängän}\pos{S vi.}\sensenumber{1}\definition{to come out}\example{Lla da ine ik atta guirängänän.}{The man come out from the water.}\sensenumber{2}\definition{to get out, lift out}\example{Angde ngäna net de tuk i dirängän, käza da net ik i gozenän.}{When I lifted the net into the air, a crocodile went inside it.}\example{Mana ddone mullae gogon obo känyer, ede adawatta bogo Lama bälle wälle gogon, Emi bälle yirngän e.}{Mana wasn't able to herself, so she called for Lama to lift Emi out.}\allomorph{irngän}\allomorph{irnen}\allomorph{ir}\allomorph{ire}\allomorph{iräng}\allomorph{yirngän}\allomorph{iräre}}
\entry{iräpang}{\headword{iräpang}\pos{mod.}\sensenumber{2}\definition{dirty}\example{Bogo ine me gongkäbägän, ede obo inggol iräpang agan.}{He dived into the water, so his face is dirty.}}
\entry{irngän}{\headword{irngän}\variant{fast speech var. of}{irängän}}
\entry{irwe}{\headword{irwe}\pos{n.}\sensenumber{2}\definition{type of cultivated tree with white flowers and juicy, red and white fruit with two seeds; used to treat cough}\sensenumber{2}\definition{two female friends who share a twin fruit from the irwe tree}\subentry{\headword{irwema}\pos{n.}\definition{two female friends who share a twin fruit from the irwe tree}}}
\entry{Isago}{\headword{Isago}\pos{pn.}\sensenumber{2}\definition{Isago (big island, perhaps in Fly River)}}
\entry{Isaiah}{\headword{Isaiah}\variant{sp. var. of}{Azaya}}
\entry{ist}{\headword{ist}\pos{n.}\sensenumber{2}\definition{yeast}\etymology{from Englishyeast}}
\entry{ita}{\headword{ita}\pos{n.}\sensenumber{2}\definition{type of sedge}}
\entry{itaita}{\headword{itaita}\pos{n.}\sensenumber{2}\definition{type of big tree that grows in the bush with red fruit and a straight trunk used for timber}}
\entry{itbonmäll}{\headword{itbonmäll}\pos{n.}\sensenumber{2}\definition{dirt}\example{Ngämlle ge kaptte da ibonmällang agnegan, ede ngänäm tatuma natram.}{These clothes got dirty on me, so [you] show me to the washing place.}}
\entry{itrell}{\headword{itrell}\pos{n.}\sensenumber{1}\definition{disease, illness, sickness}\example{Kandärmang säre bablle pätt a ddone abal agan itrell atta.}{Sorry that the illness has made your body so unwell.}\sensenumber{2}\definition{to get hurt}\example{Da ubi gullem de bänglläbaebneyo, ubi mɨnyi ddone buitrellnegnän.}{If they pick up the snakes, they won't get hurt.}\sensenumber{2}\definition{ill, sick}\example{Yesu yuwog abal itrellang lla de ai dägnegnän.}{Jesus healed many sick men.}\etymology{itrell + =ang}\subentry{\headword{itrellang}\pos{mod.}\definition{ill, sick}}}
\entry{ittal}{\headword{ittal}\variant{var. of}{ittall}}
\entry{ittall}{\headword{ittall}\pos{v.}\sensenumber{1}\definition{to hang}\example{Kaptte da ittananang da pätanan e.}{The clothes are hanging to dry.}\example{Mälla da net de mɨnyi bonyän baittallän.}{The woman will take the net and hang it.}\example{Molemoleg a llo ttam me itraenen anggan.}{Caterpillars hang from tree leaves.}\sensenumber{2}\definition{to follow closely, stalk (+ peyang)}\example{Ngäna Wagibo bo peyang buittall.}{I will stalk Wagibo.}\allomorph{ittae}\allomorph{itrae}}
\entry{ittitt}{\headword{ittitt}\variant{sp. var. of}{ittɨtt}}
\entry{ittɨtt}{\headword{ittɨtt}\pos{S vt.}\sensenumber{2}\definition{to catch (an aquatic animal)}\example{Angde toto gognän, Kila a Wala gall alle tudi ittaenen e deyareyo bem e.}{When evening came, Kila and Wala went to the sea to fish with the canoe.}\example{Bogo komlla guzi de deyaittän.}{She caught two crawfish.}\example{Ttongdae kollba da guittän.}{One fish was caught.}\allomorph{itt}\allomorph{ittae}\allomorph{tto}}
\entry{ittma}{\headword{ittma}\pos{n.}\sensenumber{2}\definition{husband's house}\example{ittma we zan}{ceremony in which a woman enters her husband's household}}
\entry{ittpo}{\headword{ittpo}\variant{sp. var. of}{iddpo}}
\entry{iwae}{\headword{iwae}\pos{n.}\sensenumber{2}\definition{type of weaponSɨmell gäz ma da. (It's for killing pigs.)}}
\entry{iya}{\headword{iya}\pos{n.}\sensenumber{2}\definition{Australian masked owlIddob me ekawang pa dan. (It's a bird that sings at night.)}}
\entry{iyeiyem}{\headword{iyeiyem}\pos{n.}\sensenumber{2}\definition{common emerald doveWällang me ekawang pa dan. (It's a bird that sings in the bush.)}}
\entry{Izag}{\headword{Izag}\pos{n.}\sensenumber{2}\definition{name of clan}}
\entry{izig}{\headword{izig}\pos{kin.}\sensenumber{1}\definition{co-sister-in-law (woman's husband's brother's wife)}\sensenumber{2}\definition{co-wife (another woman married to the same husband)Komlla mälla llädadag. (Marrying two women.)}\sensenumber{2}\definition{polygynous marriage}\etymology{izig + =ang}\subentry{\headword{izigag}\pos{n.}\definition{polygynous marriage}}}
\entry{iziz}{\headword{iziz}\pos{n.}\sensenumber{2}\definition{briberyTtongo lla da obo mokoang ttoen ngasnges e lla de mu bägagän. (A man will pay someone in order to get what he wants.)}}
\entry{Ivan}{\headword{Ivan}\pos{pn.}\sensenumber{2}\definition{male personal name}}
\lettersection{K}
\entry{ka1}{\headword{ka1}\pos{TAM ptcl.}\sensenumber{1}\definition{counterfactual particle (often preceded by ada)}\example{Ngämaene gall a ada ka bopänayän, adawatta däbe za da ngämim deyangkollmällnän.}{We thought our (excl.) boat was going to capsize, because that thing was following us.}\example{Ddia da ttam agan adawatta ngäna bam ada ka bongo walle we aspunalle.}{The deer is alive because I thought that you fell into the water.}\example{Mer llo popo alle bongo darbänen alle, ada ka män duwar da bun mi popo de nowattällan.}{You decorate with nice tree flowers like a young girl puts flowers in her hair.}\example{Da klloklloe eka bäpanyeyo, ka Ende eka mɨnyi budabän.}{If they speak a mixed language, it seems the Ende language will be lost.}\sensenumber{2}\definition{question particle}\example{E pazi me ka?}{In what year?}\sensenumber{3}\definition{emphatic particle}\example{Däbe tätämatt winy a ka tomowangtomowang moko allan.}{That honey from yesterday, it's tasting a bit bitter.}\allomorph{aka}}
\entry{ka2}{\headword{ka2}\pos{interj.}\sensenumber{3}\definition{no}\example{― Skul e angosan. ― Ka, ma me daden.}{― He returned to school. ― No, he's at home.}\allomorph{aka}}
\entry{kab}{\headword{kab}\pos{n.}\sensenumber{3}\definition{string, rope, fiber}\example{kab llatet att tär}{string made from twisted fiber}\example{Ddone aeya mullae gogalle oblle kab alle mällamälla we.}{No one was able to tie him up with a rope.}}
\entry{kabadu}{\headword{kabadu}\pos{n.}\sensenumber{3}\definition{traditional dish consisting of coconut cream, sago, meat, and tulip greens}}
\entry{kabag}{\headword{kabag}\pos{n.}\sensenumber{3}\definition{type of thick grass that grows in swamps}}
\entry{kabär}{\headword{kabär}\pos{n.}\sensenumber{3}\definition{type of big taro}}
\entry{kabkab}{\headword{kabkab}\pos{n.}\sensenumber{3}\definition{type of vine-like plantWälläng me yuwog dagän, yu patkoll ma dan. (They're plentiful in the bush; it's for bundling wood)}\example{gärep kabkab}{grapevine}\etymology{redup. of kab}}
\entry{Kadawa}{\headword{Kadawa}\pos{pn.}\sensenumber{3}\definition{Kadawa (in Kiwai Rural LLG)}}
\entry{kae}{\headword{kae}\pos{n.}\sensenumber{3}\definition{a type of plant traditionally chewed and used to welcome people}}
\entry{kae ine}{\headword{kae ine}\pos{n.}\sensenumber{3}\definition{wine}}
\entry{kaeg}{\headword{kaeg}\pos{n.}\sensenumber{3}\definition{close male friend of the same age who went through initiation at the same time}}
\entry{kaekae}{\headword{kaekae}\pos{n.}\sensenumber{3}\definition{type of small plant with blue and purple flowers and black fruit that cassowaries eat}}
\entry{kaekep}{\headword{kaekep}\pos{S vt.}\sensenumber{1}\definition{to chew}\example{Gae da nge ingollae dan, be obo täma da kaekep ma dan.}{The gae palm is like coconut, but its husk is for chewing.}\sensenumber{2}\definition{to struggle to cut (e.g. with a dull knife)}\allomorph{kep}\allomorph{kaepnen}}
\entry{kaembre}{\headword{kaembre}\pos{n.}\sensenumber{2}\definition{type of tree}}
\entry{kaemne}{\headword{kaemne}\pos{n.}\sensenumber{2}\definition{beeAllko märäll ziz be obaene ddäddäg a era ddobae tätäp dan. Ap me a ako wälläng me giddollag dan. (A fly-sized insect, but its sting is very painful. It lives in the grassland and bush.)}\example{Kaemne de ttang alle mällam a mudan, adawatta da bam ttang me naddägän.}{Don't hold bees with your hand, because they might sting you in the hand.}}
\entry{kaen}{\headword{kaen}\pos{A vt.}\sensenumber{2}\definition{to wrap up}\example{Ngämi käza de net alle kaen eralla, mängallmeny agan zime käza da.}{We (excl.) wrap up the crocodile with the net; the crocodile is already weak.}}
\entry{kaepse}{\headword{kaepse}\pos{n.}\sensenumber{2}\definition{type of tree that grows in the bush with white flowers and big, yellow fruit that cassowaries and deer eat}}
\entry{Kagär}{\headword{Kagär}\pos{pn.}\sensenumber{2}\definition{female personal name}}
\entry{Kago}{\headword{Kago}\pos{pn.}\sensenumber{2}\definition{male personal name}}
\entry{kak}{\headword{kak}\pos{kin.}\sensenumber{1}\definition{grandmother (one's parent's mother; reciprocal)Baba bo o yae bo mäg. (Dad's or mom's mother.)}\example{Kak säre zäme ause abal allan.}{Grandmother is already growing very old.}\sensenumber{2}\definition{grandchild (woman's child's child; reciprocal)}\sensenumber{3}\definition{mother-in-law (woman's husband's mother; reciprocal)}\sensenumber{4}\definition{daughter-in-law (woman's son's wife; reciprocal)}\sensenumber{4}\definition{nonsingular form of kak}\subentry{\headword{kakak}\pos{n.}\definition{nonsingular form of kak}}}
\entry{kakab}{\headword{kakab}\pos{n.}\sensenumber{1}\definition{leftovers, remainder, remnant}\example{Ddob kakab de dowattälleya ttongo ebdo we.}{They left some leftovers for another day.}\sensenumber{2}\definition{leftover, remaining}\example{up wo kakab}{remaining ripe bananas}}
\entry{kakakän}{\headword{kakakän}\pos{n.}\sensenumber{1}\definition{(outer) space}\example{Kok a wa piro da ddapall me kakakän giddollnen eran.}{The moon and stars are in space.}\sensenumber{2}\definition{loose, floating, unbound}\example{nil kakakän}{loose nail}\example{Kollba da kakakän ibnin allan.}{The fish is floating on the water.}\example{Ttongdae kui kälsre dae obo känyer kakänkakän gognän.}{A small island was all alone.}\sensenumber{3}\definition{float}\sensenumber{3}\definition{floating, loose}\example{kakne mättnan ma iddpo}{cloak (lit. 'garment for wearing loose')}\subentry{\headword{kakne}\pos{mod.}\definition{floating, loose}}}
\entry{kakal}{\headword{kakal}\pos{S vi.}\sensenumber{3}\definition{to enter, board, go in}\example{Bogo gall e gokakalän.}{She got in the canoe.}\example{Ngäna sɨmell lelang atta llo we okakalan.}{I was afraid of the pig, so I went in the tree.}\allomorph{kak}\allomorph{kakek}}
\entry{Kakaya}{\headword{Kakaya}\pos{pn.}\sensenumber{3}\definition{Kakaya (toponym)}}
\entry{Kakayam}{\headword{Kakayam}\pos{pn.}\sensenumber{3}\definition{female personal name}}
\entry{kakayam}{\headword{kakayam}\pos{n.}\sensenumber{3}\definition{greater bird-of-paradiseWällang me ekawang pa pitong dan. (It's a big-tailed bird that sings in the bush.)}}
\entry{kakänkakän}{\headword{kakänkakän}\variant{var. of}{kakakän}}
\entry{kakep}{\headword{kakep}\pos{n.}\sensenumber{3}\definition{pain}\example{Bam tutu kälngkäl me ttäle kakep a nallan?}{Are your legs hurting (lit. is the leg pain getting you) from climbing the mountain?}}
\entry{Kakeya}{\headword{Kakeya}\pos{pn.}\sensenumber{3}\definition{Kakeya (bush, camping and sago place of Baba Zi from Upiara; on the road to Bisuaka)}}
\entry{kakoll}{\headword{kakoll}\pos{n.}\sensenumber{1}\definition{dish}\example{Ttongo lla da ttongo za de nanyagaenan kakoll me.}{A man was stirring something in a dish.}\sensenumber{2}\definition{cup}\example{ine nane ma kakoll}{cup for drinking water}\sensenumber{2}\definition{endocarp of coconut}\subentry{\headword{kakoll kutt}\pos{n.}\definition{endocarp of coconut}}}
\entry{Kakos}{\headword{Kakos}\pos{pn.}\sensenumber{2}\definition{female personal name}}
\entry{kakud}{\headword{kakud}\pos{n.}\sensenumber{2}\definition{type of thin, curved, and long yam without thorns}}
\entry{kala}{\headword{kala}\pos{n.}\sensenumber{2}\definition{color, pigment, dye}\example{Ngäna ag me kala de yu nägnegan.}{I boiled the pigments in the morning.}\etymology{from Englishcolor}}
\entry{Kalamato}{\headword{Kalamato}\pos{pn.}\sensenumber{2}\definition{female personal name}}
\entry{Kaldon}{\headword{Kaldon}\pos{pn.}\sensenumber{2}\definition{male personal name}}
\entry{kalemtoe}{\headword{kalemtoe}\pos{n.}\sensenumber{2}\definition{type of tree that grows in the bush with small, long, thin edible nuts that must be broken with stone; nuts are edible when steamed and are also eaten by cassowaries and doves; flesh is used for cream; similar to toe tree}}
\entry{kalkmo}{\headword{kalkmo}\pos{n.}\sensenumber{2}\definition{joints}}
\entry{kalmoe}{\headword{kalmoe}\pos{mod.}\sensenumber{2}\definition{pliable, bendable}\sensenumber{2}\definition{welcoming}\example{Ngämo yae lla kalmokalmoe deya.}{My mother was a welcoming person.}\subentry{\headword{kalmokalmoe}\pos{mod.}\definition{welcoming}}}
\entry{kalokalo}{\headword{kalokalo}\pos{mod.}\sensenumber{2}\definition{pliable, flexible, bendable}}
\entry{kaltakaltamang}{\headword{kaltakaltamang}\pos{n.}\sensenumber{2}\definition{type of spear}}
\entry{kalla}{\headword{kalla}\pos{S vi.}\sensenumber{2}\definition{to lie down}\example{Ngäna obom dangkallängg.}{I made him lie down.}\allomorph{kall}\allomorph{ngkall}}
\entry{kallag}{\headword{kallag}\variant{var. of}{kalläg}}
\entry{kallakalla}{\headword{kallakalla}\pos{n.}\sensenumber{2}\definition{arrow\textbackslash_type}}
\entry{kallamatt}{\headword{kallamatt}\pos{n.}\sensenumber{2}\definition{Oriental dollarbirdYogoll mameatt me kiliang pa dan. (It's a bird that's happy after the rain.)}}
\entry{kalläg}{\headword{kalläg}\pos{n.}\sensenumber{2}\definition{type of edible, fatty fish with big scales; found in the swamp, similar to barramundi}}
\entry{kalläntäg}{\headword{kalläntäg}\pos{S vt.}\sensenumber{2}\definition{to split}\example{Bogo mätta ulleulle de deyangkälltägalle.}{She split the big yams.}\allomorph{ngkälltäg}}
\entry{kallekalle}{\headword{kallekalle}\pos{n.}\sensenumber{2}\definition{type of yam with a white interior, red skin, and hairs}}
\entry{kallɨngg}{\headword{kallɨngg}\pos{S vt.}\sensenumber{2}\definition{to kill}\example{Sɨmell de mɨnyi bangkallɨgiyu.}{They will kill the pig.}\allomorph{ngkallɨg}}
\entry{kallkäll}{\headword{kallkäll}\pos{n.}\sensenumber{2}\definition{cold}\example{Kallkäll ngänäm dagän.}{I got cold (lit. the cold got me).}\example{Gongkäbägän, abo kallkäll atta guiringänän.}{He dived, then got out because of the cold.}\sensenumber{2}\definition{long-sleeve shirt; coat}\sensenumber{2}\definition{to warm up, get warm, warm oneself}\example{Bogo yu me kallkäll gonggllaemenynän.}{He was warming up by the fire.}\example{Da ngäna ddone yu bipellknegalle bablle, bongo ddone gänya mer yu mi kallkäll bonggllaegalle.}{If I hadn't chopped wood for you, you couldn't get warm by this nice fire.}\subentry{\headword{kallkäll sod}\pos{n.}\definition{long-sleeve shirt; coat}}\subentry{\headword{kallkäll gllaengg}\pos{S vi.}\definition{to warm up, get warm, warm oneself}}}
\entry{kallkell}{\headword{kallkell}\pos{S vt.}\sensenumber{2}\definition{to deleaf a plant to reveal the shoot, or to take the skin off}}
\entry{kallmo}{\headword{kallmo}\pos{n.}\sensenumber{2}\definition{butcherbird (black-backed, hooded)Wällang ma ttängäm me giddollag pa dan. (It's a bird that lives in villages in the bush.)}}
\entry{kalltam}{\headword{kalltam}\pos{S vt.}\sensenumber{2}\definition{to split}\example{Keta da sanador kalltaematt dan.}{A roof post is a split sago stalk.}\allomorph{kalltaem}\allomorph{kallt}}
\entry{kam1}{\headword{kam1}\pos{S vi.}\sensenumber{1}\definition{to start, begin}\example{kam ma ngättma}{place for starting (i.e. starting line)}\example{Yogoll a gongkamän.}{The rain started.}\example{Obo ma we ibi we kam allan.}{He is starting to go home.}\sensenumber{2}\definition{to start}\example{Edeb a ine kämbmeny de dängkamän.}{Edeb started diving in the water.}\allomorph{ngk}\allomorph{k}\allomorph{ngkam}\allomorph{ngkaem}}
\entry{kam2}{\headword{kam2}\pos{S vt.}\sensenumber{2}\definition{to cut (e.g. meat, skin)}\example{Ngäna ngämo tupi di nakaman.}{I cut my index finger.}\allomorph{k}}
\entry{kamäkamät}{\headword{kamäkamät}\pos{n.}\sensenumber{2}\definition{type of big tree that grows in the bush with yellow flowers and big yellow fruit that are collected}}
\entry{kambäg}{\headword{kambäg}\pos{S vi.}\sensenumber{2}\definition{to escape, flee}\example{Ngäna gänyeri angkebägan.}{I fled this way.}\allomorph{kambämäll}\allomorph{ngkebäg}}
\entry{kame1}{\headword{kame1}\pos{n.}\sensenumber{1}\definition{ignorance, non-knowing; incomprehension, non-understanding}\example{Ende eka kame lla}{person who doesn't speak Ende}\example{Obo kame dan amom bällädän.}{She doesn't know (lit. her ignorance is) whom she will marry.}\example{Ngäma kame da ada käza da daden walle ik mi.}{We (excl.) don't realize (lit. it's our ignorance) that there's a crocodile in the water.}\example{Ngämo kame dan ematta bongo mikutt alle.}{I don't see (lit. my incomprehension is) why you're mad.}\sensenumber{2}\definition{unknown, unfamiliar}\example{kame lla}{stranger}\example{Da kame ttängämang lla de bognegän, ede ngäna mɨnyi Ingglis eka walle bäntameny.}{If it's a man from an unfamiliar village, then I will converse using English.}\sensenumber{2}\definition{unknowingly, mistakenly, wrongly}\example{Ngämi kamekame ada ka up a walle ik mi dan.}{We (excl.) mistakenly thought the banana was inside the water.}\subentry{\headword{kamekame}\pos{adv.}\definition{unknowingly, mistakenly, wrongly}}}
\entry{kame2}{\headword{kame2}\pos{adv.}\sensenumber{2}\definition{again}\example{Kame ge ttoen bongo ddone mɨnyi nanges.}{You must not do this thing again.}\example{Kumuddäga we ebdo me bogo mɨnyi kame ttam bogon kuddäll atta.}{In three days, he will rise from the dead and live again.}}
\entry{kame tameny ma skul}{\headword{kame tameny ma skul}\sensenumber{2}\definition{open distance school}}
\entry{kamebi}{\headword{kamebi}\variant{sp. var. of}{kemibi}}
\entry{kamekong}{\headword{kamekong}\pos{mod.}\sensenumber{2}\definition{busy}\example{Mosen da angde kamekong gogon kaptte gllämnan me, mänyan da bi komllaebe ada tongoenen me kamekong gogeyo.}{While the eldest was busy washing clothes, the two younger ones were busy playing.}}
\entry{kameny}{\headword{kameny}\pos{n.}\sensenumber{2}\definition{absence}\example{Ngämo kameny me de ngämo masar a gagäll gogon.}{My grandfather passed away in my absence (i.e. before I was born).}}
\entry{kamibi}{\headword{kamibi}\variant{sp. var. of}{kemibi}}
\entry{kamo}{\headword{kamo}\pos{n.}\sensenumber{2}\definition{reciprocal term for the young man and the older man that takes him through initiationIttma bin llayaba. (A ceremonial term for people.)}}
\entry{kampani}{\headword{kampani}\pos{n.}\sensenumber{2}\definition{company}\etymology{from Englishcompany}}
\entry{kamuka}{\headword{kamuka}\pos{n.}\sensenumber{2}\definition{type of cultivated thorny citrus tree with small white flowers and softball-size fruit with thick green skin}}
\entry{kanas}{\headword{kanas}\pos{n.}\sensenumber{2}\definition{type of basic arrowToboll e pattlle pallkoll popoatt. (A piece of bamboo sharpened for an arrow.)}\example{Masamasar täräp me ngattong kanas toboll de dokomneyo.}{In the old times, people first carried kanas spears.}}
\entry{kanas ma}{\headword{kanas ma}\pos{n.}\sensenumber{2}\definition{type of pointed roof not found in Limol}}
\entry{kandärmang1}{\headword{kandärmang1}\pos{mod.}\sensenumber{1}\definition{sorry, apologetic, regretful}\example{Kandärmang nägagallo.}{They were feeling sorry for them.}\example{Kandärmang allan ngäna bäne pate, ddone mulldae dan.}{I feel sorry towards you, but I'm unable.}\example{Bogo kandärmang me dan.}{She's feeling regretful.}\sensenumber{2}\definition{sorry}\sensenumber{3}\definition{pity, sympathy}\example{Kandärmang a era ulle ttoen dan.}{Sympathy is an important habit.}\sensenumber{4}\definition{sorry, pitiful, unfortunate, sad}\example{Ge lla ddobae kandärmang dan, bogo ekameny dan, llan ttomoll dan.}{This man is very unfortunate; he is mute and deaf.}\example{Kandärmang, llamda da bälle gazen ma da ddone mullae gogon.}{Sadly, the old man didn't have a way to get out.}\example{Emi ngänaeka gogon kandärmang ingoll gogon.}{Emi cried pitifully.}\sensenumber{4}\definition{apology}\example{Ngäna bänene kandärmang eka de malam eran.}{I accept your apology.}\subentry{\headword{kandärmang eka}\pos{n.}\definition{apology}}}
\entry{kandärmang2}{\headword{kandärmang2}\pos{n.}\sensenumber{4}\definition{present, gift}\example{Ngämo kandärmang a gänyan bablle.}{My present for you is here.}\example{Eso ulle bänene kandärmang de.}{Thanks a lot for your gift.}}
\entry{Kanengga}{\headword{Kanengga}\pos{pn.}\sensenumber{4}\definition{male personal name}}
\entry{Kaninga}{\headword{Kaninga}\sensenumber{4}\definition{male personal name}}
\entry{kanken}{\headword{kanken}\pos{n.}\sensenumber{4}\definition{type of mushroomBawa me päddabag pollopollonang me sana peyang ddägnan ma dan. (It grows during the rainy season in the bushes; it's to eat with sago.)}}
\entry{kannas}{\headword{kannas}\pos{n.}\sensenumber{4}\definition{type of bow made out of pitpit}}
\entry{kanoe}{\headword{kanoe}\pos{n.}\sensenumber{4}\definition{type of tree with fruits that cassowary, pigs, and deer eat and bark that is put in the water to kill fish}}
\entry{kanong}{\headword{kanong}\variant{dial. var. of}{tanong}}
\entry{kansel}{\headword{kansel}\pos{n.}\sensenumber{4}\definition{counsel}\etymology{from Englishcounsel}}
\entry{kantärpie}{\headword{kantärpie}\pos{n.}\sensenumber{4}\definition{log stuck in the waterWalle ik me llo me ddäganenang dag.}}
\entry{kang}{\headword{kang}\pos{n.}\sensenumber{4}\definition{sucker (additional unwanted shoot that grows by the base of a tree)}\example{Kak sana kang de ttaengän eran.}{Grandmother pulls a sago sucker.}}
\entry{Kange}{\headword{Kange}\pos{pn.}\sensenumber{4}\definition{male personal name}}
\entry{Kangge}{\headword{Kangge}\pos{pn.}\sensenumber{4}\definition{male personal name}}
\entry{kangkäg}{\headword{kangkäg}\pos{S vt.}\sensenumber{4}\definition{to carry, bear (a load)}}
\entry{kanyam}{\headword{kanyam}\pos{S vi.}\sensenumber{4}\definition{bend}}
\entry{kanyekanye}{\headword{kanyekanye}\pos{A vi.}\sensenumber{1}\definition{to move around, travel}\example{Lla da yuwog abal dagaeya a dazirnän ada sip a alla ubira pänggmenyang llameny kanyekanye bognegnän.}{There were so many people gathered, there like how sheep will move around without a shepherd.}\example{Ngämo kanyekanye att a ngättma da gänyag ada, Balimo, Daru, Mospi.}{Here are the places I've travelled to: Balimo, Daru, and Port Moresby.}\sensenumber{2}\definition{to go hunting}\example{Ngäna iddob kanyekanye we gotäba.}{I planned to go on a hunt at night.}\allomorph{kanye}}
\entry{Kaoga}{\headword{Kaoga}\pos{pn.}\sensenumber{2}\definition{male personal name}}
\entry{kaonggall}{\headword{kaonggall}\pos{n.}\sensenumber{2}\definition{yellow-faced myna}}
\entry{kap}{\headword{kap}\pos{n.}\sensenumber{2}\definition{cup}\etymology{from Englishcup}}
\entry{kapa}{\headword{kapa}\pos{n.}\sensenumber{2}\definition{knife\textbackslash_type}}
\entry{Kapal}{\headword{Kapal}\pos{pn.}\sensenumber{2}\definition{Kapal (Wipi- and Kawam-speaking village in Oriomo-Bituri Rural LLG; has airstrip, aid post, primary school; from Limol, one must pass through Bisuaka)}}
\entry{kapalla}{\headword{kapalla}\pos{n.}\sensenumber{2}\definition{floating grass Walle me kakakän päddäbang za dan, ine me päddnenang ttam ulloullong za a obo ttam a sana yu ma dan. (It's a thing that grows on the surface of the water with big leaves, and its leaves are used when cooking sago.)}\example{Ngämlle kapalla ttam alle sana yuatt a ddobae moko dan.}{I like to eat sago wrapped in kapalla leaves.}}
\entry{kapang}{\headword{kapang}\pos{n.}\sensenumber{2}\definition{AcaciaLlo tupi dan. Kakne ttoe a ddogllopang dan. Dädär me kollop a mer bängang dan otät yu we. Ma gogo we mer llo dan pallkänen e. Yukroll a obene panya me a ankom nyäny ma dan däl llätt e otät e. Obo pätt me budar a mer moko dag. Igi me ttoe a wabeb ma dan, ddoll a iid peyang ute nyäny ma dan, ute kätkät e. Ako id a obene nane ma dan itrellang me. Ako tongle säpallnen ma dan ttäle watt nyongo me wälläng dae ibnen me. Ttoe a tupiae kädkädag dan za mällanen e. (It's a tall tree. The outer bark has scales. When the bark is dry, it's good as firewood for cooking. It's a good tree to split for building houses. Its ash is to stop the sour taste of ants or pineapple. The grubs in its trunk are tasty. The inner bark is beaten; the foam and extract is for putting on wounds, to  the wound. Its extract is also for drinking when sick with cough. It's also for removing leeches from legs when traveling on the road through the bush. The bark comes off in long pieces and is used to tie things.)}\sensenumber{2}\definition{type of grubBudar ulle dan kapang llo ik me, ddäddägma dan. (It's a big grub in the Acacia tree; it's edible.)}\subentry{\headword{kapang budar}\pos{n.}\definition{type of grubBudar ulle dan kapang llo ik me, ddäddägma dan. (It's a big grub in the Acacia tree; it's edible.)}}}
\entry{kapang bile}{\headword{kapang bile}\pos{n.}\sensenumber{2}\definition{type of medicine}}
\entry{Kapangang bun}{\headword{Kapangang bun}\pos{pn.}\sensenumber{2}\definition{Kapangang bun (sago place on the way to Egapo)}}
\entry{kapangmändär}{\headword{kapangmändär}\pos{n.}\sensenumber{2}\definition{type of mushroomAp me päddabag, ako ddäddäg ma dan. (It grows in the grassland; it's also edible.)}}
\entry{Kaparnaom}{\headword{Kaparnaom}\pos{pn.}\sensenumber{2}\definition{Capernaum}}
\entry{kapän}{\headword{kapän}\pos{n.}\sensenumber{2}\definition{wrist}}
\entry{kapera}{\headword{kapera}\pos{n.}\sensenumber{2}\definition{male friend or partner who comes from out of townIttma bin llayaba. (Ceremonial term for someone.)}}
\entry{kapkap}{\headword{kapkap}\pos{n.}\sensenumber{2}\definition{mudskipperBem me pipllo ingoll za, gäbmällgäbmäll ibiag dan. (It's a lizardlike thing in the sea; it skips around.)}}
\entry{kapräl}{\headword{kapräl}\pos{n.}\sensenumber{2}\definition{type of tree}}
\entry{kaptte}{\headword{kaptte}\pos{n.}\sensenumber{1}\definition{cloth}\example{Mälla da kaptte gullun alle obo bun di dɨmllawän.}{The woman tied her head with an old cloth.}\sensenumber{2}\definition{clothing, clothes; piece of clothing, garment}\example{Kaptte de kotang de ngäsangngäsang de mättnan a mudan.}{Don't wear dirty clothes over and over.}\example{Ngämo kaptte da era komlla dageyo.}{I have two pieces of clothing.}\sensenumber{2}\definition{washing board}\sensenumber{2}\definition{clothes line}\sensenumber{2}\definition{short trousers}\sensenumber{2}\definition{long trousers}\sensenumber{2}\definition{laundry, washing clothes}\example{Lama kamekong gogon kaptte gällämnan me.}{Lama was busy doing the laundry.}\etymology{from Agobkaboll ttoe, lit. 'devil skin'}\subentry{\headword{kaptte dodro ma}\pos{n.}\definition{washing board}}\subentry{\headword{kaptte ittal ma}\pos{n.}\definition{clothes line}}\subentry{\headword{kaptte tubutubu}\pos{n.}\definition{short trousers}}\subentry{\headword{kaptte tupi}\pos{n.}\definition{long trousers}}\subentry{\headword{kaptte gällämnan}\pos{n.}\definition{laundry, washing clothes}}}
\entry{kapu}{\headword{kapu}\pos{A vt.}\sensenumber{2}\definition{to carry}\example{Ngäna llɨg di kapu iran.}{I am carrying the baby.}}
\entry{karado}{\headword{karado}\pos{n.}\sensenumber{2}\definition{long spear for fishing}}
\entry{Karama}{\headword{Karama}\pos{pn.}\sensenumber{2}\definition{Karama (swamp and canoe place in Limol)}\example{Ngämi ttongo ebdo me Karama we gobllab inu we, Eramang gall tapma we ddob llayaba kämäll.}{One day, we went to Karama to sleep, to the canoe place with some people.}\example{Karama kona me giddollag dan.}{He lives in Karama district.}\example{Bobag daya dedam karama walle da adawatta yogoll ulle da dämanän.}{Karama river was flooded because there had been big rains.}}
\entry{karama}{\headword{karama}\pos{n.}\sensenumber{2}\definition{swamp}\example{Karama da ade petapeta abal gogon.}{The swamp also became very shallow.}}
\entry{Karama Popo}{\headword{Karama Popo}\variant{sp. var. of}{Karamapopo}}
\entry{Karamapopo}{\headword{Karamapopo}\pos{pn.}\sensenumber{2}\definition{female personal name}}
\entry{Karao}{\headword{Karao}\pos{pn.}\sensenumber{2}\definition{male personal name}}
\entry{Karau}{\headword{Karau}\variant{sp. var. of}{Karao}}
\entry{Karea}{\headword{Karea}\pos{pn.}\sensenumber{2}\definition{male personal name}}
\entry{Karen}{\headword{Karen}\pos{pn.}\sensenumber{2}\definition{female personal name}}
\entry{Kares}{\headword{Kares}\pos{pn.}\sensenumber{2}\definition{female personal name}}
\entry{kargeam}{\headword{kargeam}\pos{n.}\sensenumber{2}\definition{type of big taro}}
\entry{Karis}{\headword{Karis}\pos{pn.}\sensenumber{2}\definition{female personal name}}
\entry{karita}{\headword{karita}\pos{n.}\sensenumber{2}\definition{type of introduced bananaTupi pänyanzag dan, däg a obo yuwog dag. Käp a obo o me otänan ma dag, yu mi binzenen ma dag, a kire da yu ma dag otänan e. (It grows long; its bunches are plentiful. When ripe, its fruit are eaten and heated on the fire, and when unripe, they are cooked to be eaten.)}}
\entry{karpo}{\headword{karpo}\pos{n.}\sensenumber{2}\definition{jar}}
\entry{Kas}{\headword{Kas}\pos{pn.}\sensenumber{2}\definition{male personal name}}
\entry{Kasakmai}{\headword{Kasakmai}\pos{pn.}\sensenumber{1}\definition{Kasakmai (toponym)}\sensenumber{2}\definition{name of a person}}
\entry{Kasimap}{\headword{Kasimap}\pos{pn.}\sensenumber{2}\definition{Kasimap (Abom-speaking village in Gogodala Rural LLG, near Zanor; has an elementary school)}}
\entry{Kasir}{\headword{Kasir}\pos{pn.}\sensenumber{2}\definition{male personal name}}
\entry{Kaso}{\headword{Kaso}\pos{pn.}\sensenumber{2}\definition{male personal name}}
\entry{kastom}{\headword{kastom}\pos{n.}\sensenumber{2}\definition{custom}\etymology{from Englishcustom}}
\entry{Katama}{\headword{Katama}\pos{pn.}\sensenumber{2}\definition{male personal name}}
\entry{Kate}{\headword{Kate}\pos{pn.}\sensenumber{2}\definition{female personal name}}
\entry{Katherine}{\headword{Katherine}\pos{pn.}\sensenumber{2}\definition{female personal name}}
\entry{katkatre}{\headword{katkatre}\variant{fast speech var. of}{katrekatre}}
\entry{katre}{\headword{katre}\pos{n.}\sensenumber{1}\definition{board; flooring}\example{Llɨg a katre me dämanenang nazernan.}{The children were sitting on the floor.}\sensenumber{2}\definition{raised, elevated}\example{Yesu ubim dokomän obo peyang ada Pita, Zeims, a Zon tuk abal tutu katre we.}{Jesus took Peter, James, and John with him up a very tall mountain.}\sensenumber{2}\definition{raised house, stilt house}\example{Bogo ikop täbaeb e katre ma de dogowän adawatta lla da gämäll bognegnän.}{He built a raised house for keeping watch, because people would steal.}\sensenumber{1}\definition{table; desk}\example{Mälla da kakoll de katkatre toko me naspunan.}{The woman threw the plate onto the table.}\sensenumber{2}\definition{shelfMa ik mi o upe me katre kälekäle ddob za wattällnen e. (A small board in or outside of the house to put some things on.)}\example{Abo ddäddäg de nokomeyo a katrekatre toko me nowattälleyo.}{You (pl.) must take the meat and put it on the shelf.}\subentry{\headword{katre ma}\pos{n.}\definition{raised house, stilt house}}\subentry{\headword{katrekatre}\pos{n.}\definition{table; desk}}}
\entry{katt1}{\headword{katt1}\pos{n.}\sensenumber{2}\definition{Meyer's friarbird}}
\entry{katt2}{\headword{katt2}\pos{n.}\sensenumber{2}\definition{type of medium-sized rodent that lives in the bush}}
\entry{kau}{\headword{kau}\pos{n.}\sensenumber{2}\definition{wrestling clothes}}
\entry{Kauga}{\headword{Kauga}\pos{pn.}\sensenumber{2}\definition{male personal name}}
\entry{Kawa}{\headword{Kawa}\pos{pn.}\sensenumber{2}\definition{male personal name}}
\entry{kawa}{\headword{kawa}\pos{n.}\sensenumber{1}\definition{announcement, notice; plea}\sensenumber{2}\definition{to preach}\example{kawawang lla}{preacher}\example{Komllakomlla melem agnan ada, ag me medicine melem, toto we God bäne eka kawa gogalle.}{They did both jobs: medicine in the morning, and preaching God's word in the evening.}}
\entry{Kawam}{\headword{Kawam}\pos{pn.}\sensenumber{2}\definition{Kawam language (Pahoturi River language spoken in Wim)}}
\entry{Kawiapo}{\headword{Kawiapo}\pos{pn.}\sensenumber{2}\definition{Kaviapu (village in Gogodala Rural LLG, near Tapila)}}
\entry{Kawito}{\headword{Kawito}\pos{pn.}\sensenumber{2}\definition{Kawito (station in Gogodala Rural LLG)}}
\entry{Kaya}{\headword{Kaya}\pos{pn.}\sensenumber{2}\definition{male personal name}}
\entry{Kaysy}{\headword{Kaysy}\pos{pn.}\sensenumber{2}\definition{female personal name}}
\entry{käba}{\headword{käba}\pos{n.}\sensenumber{2}\definition{solo hunt early in the morning}\example{Bogo käba walle ddäddäg gäzag deya.}{He killed animals on hunts.}\example{Ngäna käba ma ibi allan.}{I'm going hunting.}}
\entry{käbab}{\headword{käbab}\pos{S vi.}\sensenumber{1}\definition{to stop}\example{Ngämo moko da käbab e dan.}{I want to stop.}\example{Okbab.}{[You] stop.}\sensenumber{2}\definition{to stop}\example{Näkbabeyo!}{[You all] stop it!}\allomorph{kbab}\allomorph{kb}}
\entry{Käball}{\headword{Käball}\pos{pn.}\sensenumber{2}\definition{Kaball (toponym)}\sensenumber{1}\definition{Ende dialect spoken in Käball}\sensenumber{2}\definition{Ende clan with the dog totem}\subentry{\headword{Käballag}\pos{pn.}\definition{Ende dialect spoken in Käball}}}
\entry{käban}{\headword{käban}\pos{n.}\sensenumber{2}\definition{louse}\example{Ngämi nazernan gänyme käban namoeaemnalla.}{We were here, hunting lice.}\sensenumber{2}\definition{nit, louse egg}\subentry{\headword{käban käp}\pos{n.}\definition{nit, louse egg}}}
\entry{käbädral}{\headword{käbädral}\pos{n.}\sensenumber{2}\definition{type of tree that grows in the bush with sturdy wood}}
\entry{Käbäll}{\headword{Käbäll}\variant{var. of}{Käball}}
\entry{käbäll}{\headword{käbäll}\pos{n.}\sensenumber{2}\definition{type of tree that grows in the bush with soft wood used for canoe paddles}}
\entry{kädaeb}{\headword{kädaeb}\pos{S vt.}\sensenumber{1}\definition{to break, split}\example{Ngäna nge de dakdab.}{I split the coconut open.}\example{Mälla da lla de bun dakdaebaebneyo.}{The women broke the men's skulls.}\sensenumber{2}\definition{to share, split, portion}\example{Wayati de nakdab.}{[You] portion the watermelon.}\example{Ttängäm dakdaeballo ada ttongdae bo gänyan, ttongdae bo gänyan, do tämamae lla da mullae täräp gogmallo.}{They split the garden like here's one for him, one for him, until everyone got their fair share.}\sensenumber{2}\definition{pieces}\example{sana kädakäde}{pieces of sago}\allomorph{kd}\allomorph{käd}\subentry{\headword{kädakäde}\pos{n.}\definition{pieces}}}
\entry{kädbae}{\headword{kädbae}\pos{S vt.}\sensenumber{2}\definition{to test, try}\example{Ddia midd de kälakälae dakädbaeneyo.}{Little by little, they tried the deer meat.}\allomorph{kädb}}
\entry{kädebällag mälla}{\headword{kädebällag mälla}\pos{n.}\sensenumber{2}\definition{type of big taro}}
\entry{kädgal}{\headword{kädgal}\pos{n.}\sensenumber{2}\definition{type of tree that grows in the bushObo tep a ute mälanen ma dan. (Its sap is for patching wounds.)}}
\entry{kädkäd1}{\headword{kädkäd1}\pos{mod.}\sensenumber{2}\definition{cold}\example{Män a kädkädag walle we agäbänan.}{The girl jumped in the cold water.}\example{Ngämo ttäle da kädkädag agallo.}{My legs got cold.}}
\entry{kädkäd2}{\headword{kädkäd2}\pos{S vt.}\sensenumber{2}\definition{to remove bark, debark}\example{Ngäna put de näkädan.}{I removed the bark.}\allomorph{käd}}
\entry{kädkäd3}{\headword{kädkäd3}\pos{n.}\sensenumber{2}\definition{type of initiation that one must do before you get something from someone}\example{Bobzag a kädkäd de oba zime ngasnges allo.}{Bobzag clan already did their kädkäd.}\allomorph{käd}}
\entry{käg1}{\headword{käg1}\pos{n.}\sensenumber{1}\definition{type of palm with wood used for flooring}\example{Llɨg a walle menae me mälla da käg toko me daeya.}{A boy was beside the water; a woman was on top of the käg palm.}\sensenumber{2}\definition{floor}\example{Llɨg a käg me nädämawan.}{The children sat on the floor.}\sensenumber{2}\definition{horizontal boards on which the floor goes}\sensenumber{2}\definition{flooring, floor mat}\example{Käg tater de dikomeyo.}{They brought a floor mat.}\subentry{\headword{käg botta}\pos{n.}\definition{horizontal boards on which the floor goes}}\subentry{\headword{käg tater}\pos{n.}\definition{flooring, floor mat}}}
\entry{käg2}{\headword{käg2}\pos{property n.}\sensenumber{2}\definition{vague container-like thing}\example{sana käg}{sago container}\example{Däbe llɨg kälsre de däbe käg ik i däzanän.}{She put that small boy in the container.}\example{Bogo käg me adämenan ekaklle me dɨba katkatre toko dowae me.}{He sat in the seat on the floor, near the tabletop.}\sensenumber{2}\definition{container}\sensenumber{2}\definition{container for squeezing sago}\subentry{\headword{käg drol}\pos{n.}\definition{container}}\subentry{\headword{käg manas}\pos{n.}\definition{container for squeezing sago}}}
\entry{käg bänbänang}{\headword{käg bänbänang}\pos{n.}\sensenumber{2}\definition{type of spear}}
\entry{käk}{\headword{käk}\pos{n.}\sensenumber{2}\definition{bubble}\example{Obo enda käk agnan walle mäg me.}{Something is making bubbles in the creek.}}
\entry{käkan}{\headword{käkan}\pos{n.}\sensenumber{1}\definition{tideIne kälangkäl a ine baddbedd. (Water rising and going down.)}\sensenumber{2}\definition{vague amorphous and/or liquid thing}\example{bällma käkan, ddapall käkan}{saliva, cloud}}
\entry{käkäm}{\headword{käkäm}\pos{n.}\sensenumber{2}\definition{young leaf}}
\entry{käkäp}{\headword{käkäp}\pos{n.}\sensenumber{2}\definition{half}\example{Mäse ada llo me gomllamalle llo käkäp ada goddälläbnalle.}{I tried to hold onto the tree, but it broke in half.}}
\entry{käkäpyo}{\headword{käkäpyo}\pos{n.}\sensenumber{2}\definition{type of tree that grows in the grassland and savannah with flowers that wallaby and deer eat}}
\entry{käklläp}{\headword{käklläp}\pos{S vt.}\sensenumber{1}\definition{to weed for the second time (easy work to remove the remaining plants)}\example{Bogo ttängäm de bäklläpän.}{He will weed the garden.}\example{Ngämo mokowang melem a klläpnan.}{My favorite work is easy weeding.}\sensenumber{2}\definition{to sting, be painful}\sensenumber{3}\definition{to be amused}\allomorph{källp}\allomorph{klläp}}
\entry{käkllätt}{\headword{käkllätt}\pos{S vt.}\sensenumber{1}\definition{to weed roughly (leaving bits behind)}\example{Ngäna tätäm towall de däkllätt.}{Yesterday, I roughly weeded the grass.}\sensenumber{2}\definition{to fight, argue}\example{Tätäm däkllättnegeya ngämo mäg alle.}{Yesterday, my mother and I had an argument.}}
\entry{käkoll}{\headword{käkoll}\pos{n.}\sensenumber{2}\definition{baby mat}}
\entry{käkpäl1}{\headword{käkpäl1}\pos{n.}\sensenumber{2}\definition{type of tree that grows in the bush; burned to fertilize the ground}}
\entry{käkpäl2}{\headword{käkpäl2}\pos{n.}\sensenumber{2}\definition{sago cooked directly over the fire}}
\entry{kälae}{\headword{kälae}\pos{mod.}\sensenumber{1}\definition{small, little}\example{ttoenttoen kälae}{short little story}\example{Däbe källäm a ade yäbäd ulle atta dallän do kälae abal gowensegän ine da.}{That pond became very small from the strong sun.}\sensenumber{2}\definition{little, few}\example{kälae eka kutt}{a few words}\example{Ine da kälae dan.}{The amount of water is little.}\sensenumber{1}\definition{a little}\example{Ngäna kälakälae gontämon do yäbäd a angde goklawän dam ngäna ibi de gongkam.}{I waited a little until the sun set; then I started walking.}\example{Ende eka da kälakälae duli mäzi budabän.}{The Ende language might get lost in the near future.}\example{Eka kutt a ddob kälakälae sapasapang a dadegaeya.}{There were other slight differences in their words.}\sensenumber{2}\definition{into pieces}\example{Gollab a ekaklle we enanae gollomän kälakälae.}{The turtle shell got smashed into pieces on the ground.}\sensenumber{2}\definition{nonsingular form of kälae}\example{ngämi angde käkle dagaeya}{when we (excl.) were little}\example{Ge nyäng a klekle dag.}{These bags are small.}\example{Oba llɨg kälekäle tämamae ma me dowattälleyo.}{They left all their small children at home.}\allomorph{käle}\allomorph{kle}\subentry{\headword{kälakälae}\pos{adv.}\definition{a little}}\subentry{\headword{kälekäle}\pos{mod.}\definition{nonsingular form of kälae}}}
\entry{kälaepot}{\headword{kälaepot}\pos{n.}\sensenumber{2}\definition{tiptoes}\example{Kälaepot alle ibi allan.}{She's tiptoeing.}}
\entry{kälakäle}{\headword{kälakäle}\pos{S vi.}\sensenumber{2}\definition{to set (of the sun)}\example{Yäbäd a goklanän.}{The sun was setting.}\allomorph{kla}\allomorph{klakle}\allomorph{kle}}
\entry{kälas}{\headword{kälas}\variant{sp. var. of}{klas}}
\entry{kälängkäl}{\headword{kälängkäl}\variant{var. of}{kängkäl}}
\entry{kälbae}{\headword{kälbae}\pos{S vt.}\sensenumber{2}\definition{to singe (use brief heat to remove hair or down)}\example{Mälla da ttägäll baugän bottamänän, mätta de bakälbeaebneyo.}{The woman will make mumu, finish, and then singe the yams.}\allomorph{kälbe}}
\entry{kälepalle}{\headword{kälepalle}\pos{adv.}\sensenumber{2}\definition{slowly}\example{Ngäna kälepalle danyroene, angde dowae mäse däga, ddia da  dindugän.}{I was creeping slowly, and when I almost had it, the deer fled.}}
\entry{kälkäl}{\headword{kälkäl}\pos{S vt.}\sensenumber{2}\definition{to lie, fabricate}\example{Ngäna obo pallall kälkäl eran.}{I'm making up stories about her.}\allomorph{käl}}
\entry{kälngkäl}{\headword{kälngkäl}\variant{var. of}{kängkäl}}
\entry{Kälnyam}{\headword{Kälnyam}\pos{pn.}\sensenumber{2}\definition{male personal name}}
\entry{kälpalle}{\headword{kälpalle}\variant{fast speech var. of}{kälepalle}}
\entry{kälsäre}{\headword{kälsäre}\pos{mod.}\sensenumber{2}\definition{small, little}\example{Ge mätta da käläsre dan be nge da ulle dan.}{This yam is smaller than the coconut.}}
\entry{kälsre}{\headword{kälsre}\variant{fast speech var. of}{kälsäre}}
\entry{käll}{\headword{käll}\pos{n.}\sensenumber{2}\definition{spleen}}
\entry{källa}{\headword{källa}\pos{n.}\sensenumber{1}\definition{feces, poop, waste}\example{Däräng källa de ngätt att de baglloeabnalla.}{We will take dog poop out from the yards.}\example{Däräng a yuwog ngätt me källa bebeyag dan.}{The dog takes craps in many yards.}\sensenumber{2}\definition{intestines, bowels, guts, innards (of an animal)}\example{Sɨmell de enanae dukolldänegän idoidog alle, källa de tämamae däträpnegän.}{He shot the pig dead with the spear; its guts were all cut open.}\sensenumber{3}\definition{vague amorphous and/or soft thing, lump (e.g. cloud, fish tail, lump of earwax)}\sensenumber{3}\definition{toilet (lit. hole for pooping)}\sensenumber{1}\definition{while pooping; constantly needing to poop}\sensenumber{2}\definition{traveling urgently, arduously, or nonstop as if to a toilet}\example{Bogo källkällong dindu allanǃ}{He's running like he needs to poop!}\example{Ubi tätäm ddia de kapu digeyo källakällong ma we.}{Yesterday, they carried the deer all the way home.}\etymology{källa + ma}\subentry{\headword{källama kup}\pos{n.}\definition{toilet (lit. hole for pooping)}}\subentry{\headword{källakällong}\pos{adv.}\definition{while pooping; constantly needing to poop}}}
\entry{källa mit}{\headword{källa mit}\pos{n.}\sensenumber{2}\definition{bottom of fence}\etymology{lit. 'poop base'}}
\entry{källakällae}{\headword{källakällae}\pos{mod.}\sensenumber{2}\definition{hospitable}}
\entry{källakällawang}{\headword{källakällawang}\variant{var. of}{källakällong}}
\entry{källakälle}{\headword{källakälle}\pos{S vt.}\sensenumber{1}\definition{to poison the river}\example{walle kllaklle}{to poison the river}\sensenumber{2}\definition{to scrape food off the fire, e.g. banana, taro, yam}}
\entry{källatolma}{\headword{källatolma}\pos{n.}\sensenumber{1}\definition{middle finger}\sensenumber{2}\definition{three (lit. middle finger; body counting numeral)}}
\entry{källayoyo}{\headword{källayoyo}\pos{n.}\sensenumber{2}\definition{type of tree that grows in the bush with leaves used as toilet paper}}
\entry{Källäk}{\headword{Källäk}\pos{pn.}\sensenumber{2}\definition{Kallak (toponym)}}
\entry{källäll}{\headword{källäll}\variant{var. of}{kɨllɨll}}
\entry{källäm}{\headword{källäm}\pos{n.}\sensenumber{2}\definition{pond; lagoonTutu me guwem dan, oblle bun a wa sära da ddone dan.}\example{Kottllam a wa kollba da ttongo källäm me dazernän.}{Turtles and fish were living in a pond.}}
\entry{källän}{\headword{källän}\pos{n.}\sensenumber{1}\definition{belt}\example{Zon bo gablle mättnan ma källän a ddäddäg ttoe dädär alle ngasngesatt daeya.}{John's belt for wearing around his waist was made from dry animal hide.}\sensenumber{2}\definition{waist}}
\entry{Källängmäll}{\headword{Källängmäll}\pos{pn.}\sensenumber{2}\definition{sago place near Old Kibobma}}
\entry{Källid}{\headword{Källid}\pos{pn.}\sensenumber{2}\definition{female personal name}}
\entry{källkae}{\headword{källkae}\pos{n.}\sensenumber{1}\definition{future}\example{Ende tän a ako ngänam eran ada llɨg klekle a komuniti da bo källkae dag.}{The Ende tribe also understands that children are the community's future.}\sensenumber{2}\definition{later, in the future}\example{Aska källkae ngämo llɨg a mɨnyi pällämpälläm eka de bepänyneyo.}{I think my children will speak in English in the future.}}
\entry{källkäll}{\headword{källkäll}\variant{dial. var. of}{kɨllkɨll}}
\entry{källmakällme}{\headword{källmakällme}\pos{S vi.}\sensenumber{2}\definition{to survive}\example{Da ttongo kantri me lla da bompallängkmenynegän obaoba komlla kullum e, obaoba däbe kantri da mɨnyi ddone bakällmewän.}{If the people in a country divide themselves into two groups, that country will not survive.}\allomorph{källme}}
\entry{Källnyam}{\headword{Källnyam}\pos{pn.}\sensenumber{2}\definition{male personal name}}
\entry{källpalle}{\headword{källpalle}\variant{sp. var. of}{kälpalle}}
\entry{Källtae}{\headword{Källtae}\pos{pn.}\sensenumber{2}\definition{female personal name}}
\entry{Källtai}{\headword{Källtai}\variant{sp. var. of}{Källtae}}
\entry{källttakälltte}{\headword{källttakälltte}\pos{S vt.}\sensenumber{2}\definition{to cast away, expel}\example{Yesu gagäll anyke de däkällttawän lla bo guwo watt de.}{Jesus cast the bad spirit out from the man's heart.}\allomorph{källtta}}
\entry{Käm}{\headword{Käm}\pos{pn.}\sensenumber{2}\definition{female personal name}}
\entry{käm1}{\headword{käm1}\pos{n.}\sensenumber{1}\definition{stomach}\example{Ngämo käm a ulle agan.}{I'm full (lit. my stomach got big).}\example{Sɨmell bo käm de dupotän.}{He sliced the pig's stomach open.}\example{Kämang mälla da gänya itrell peyang a mänyi oba käm me llɨg abira basiaemneyo.}{Pregnant women with this disease will give it to their children in their womb.}\sensenumber{2}\definition{womb}\example{Kämang mälla da gänya itrell peyang a mänyi oba käm me llɨg abira basiaemneyo.}{Pregnant women with this disease will give it to their children in their womb.}\sensenumber{2}\definition{pregnant}\example{Da män a kämang allan, ede llɨg a zeg allan.}{If a woman gets preganant, a child will be born.}\etymology{käm + =ang}\subentry{\headword{kämang}\pos{mod.}\definition{pregnant}}}
\entry{käm2}{\headword{käm2}\pos{loc.}\sensenumber{2}\definition{underneath, under, beneath}\example{Mälla da ma käm me dan.}{The woman is beneath the house.}}
\entry{käm3}{\headword{käm3}\pos{n.}\sensenumber{2}\definition{love, enjoyment}\example{eka kämang}{talkative}\example{Ddobae togotogol tongoe kämang dagwaeya.}{They (du.) really loved to play hide-and-seek.}}
\entry{käm4}{\headword{käm4}\pos{S vt.}\sensenumber{2}\definition{to heal}\example{Itrellang lla de omäg alle dakämeyo.}{They healed the sick man with magic.}\allomorph{käme}}
\entry{käma}{\headword{käma}\variant{fast speech var. of}{ngasekäma}}
\entry{kämag}{\headword{kämag}\pos{mod.}\sensenumber{1}\definition{west, western}\example{kämag ingong}{a dance from the west}\sensenumber{2}\definition{round dance with singers and kundu drum in the middle}\sensenumber{3}\definition{west wind; windy storm from the west}\example{Tɨtɨm kämag peyang daeya.}{Yesterday, there was a windy storm.}\sensenumber{4}\definition{season characterized by windy storms from the west (first season; corresponds to January)}\example{Kämag amne me, bängnameyo abo ada mätta da zäme redi allan.}{In the middle of kämag, they will understand that the yams are ready.}\example{Kämag a mätta polle me mondre täräp dan.}{This big wind is a bad thing and it can kill people.}\sensenumber{4}\definition{Big winds in January, a big wind from the west that blows trees downWel ulle sisor pazi me. Kämag ma alle llo dunenang wel ulle.}\example{Kämag wel a era za gagällang dan a ai dan lla de ade bäbäddän.}{This big wind is a bad thing and it can kill people.}\sensenumber{4}\definition{west}\subentry{\headword{kämag wel}\pos{n.}\definition{Big winds in January, a big wind from the west that blows trees downWel ulle sisor pazi me. Kämag ma alle llo dunenang wel ulle.}}\subentry{\headword{kämagma}\pos{n.}\definition{west}}}
\entry{kämag ma}{\headword{kämag ma}\variant{sp. var. of}{kämagma}}
\entry{=kämall}{\headword{=kämall}\pos{n. cl.}\sensenumber{4}\definition{comitative case clitic; with (only used after oba, ama, and nouns followed by =aba)}\example{Deyarän däba ine nanen e oba kämall.}{He went there to drink water with them.}\example{Ge lla da ngäna ama kämall skul att dan.}{These people are who I went to school with.}\example{Ngämi gobllab Eramang gall tapma we ddob llayaba kämall.}{We (excl.) went to Eramang canoe dock with some other people.}}
\entry{käman}{\headword{käman}\pos{n.}\sensenumber{4}\definition{traditional type of cassava}}
\entry{kämany}{\headword{kämany}\pos{n.}\sensenumber{4}\definition{type of friendshipIttma bin llayaba. (Ceremonial term for someone.)}}
\entry{=kämäll}{\headword{=kämäll}\variant{dial. var. of}{=kämall}}
\entry{kämätt}{\headword{kämätt}\pos{n.}\sensenumber{4}\definition{testicle}}
\entry{kämbäg}{\headword{kämbäg}\pos{S vi.}\sensenumber{1}\definition{to dive}\example{Ngämi ine kämbmeny de dängkameya.}{We (excl.) started diving.}\example{Wiya angkäbäg!}{[You] come dive!}\sensenumber{2}\definition{to baptize}\example{Zon ine kämbägag}{John the Baptist}\example{Pasta da obom ine dängkäbägän dedme.}{The pastor baptized him there.}\allomorph{ngkäbäg}\allomorph{ngkäbmeny}\allomorph{kämbämeny}\allomorph{kämbmeny}\allomorph{ngkäbämeny}\allomorph{kämb}\allomorph{ngkäb}}
\entry{kämgag}{\headword{kämgag}\pos{n.}\sensenumber{2}\definition{type of lizard}}
\entry{kämlla}{\headword{kämlla}\pos{n.}\sensenumber{2}\definition{short-beaked echidna}\example{Kämlla da täkäll peyang dan.}{The echidna has spines.}}
\entry{kämo}{\headword{kämo}\variant{var. of}{komo1}}
\entry{kämoe}{\headword{kämoe}\pos{n.}\sensenumber{2}\definition{famine}}
\entry{kämser käpang}{\headword{kämser käpang}\pos{n.}\sensenumber{2}\definition{type of arrow}}
\entry{kämsir}{\headword{kämsir}\pos{n.}\sensenumber{2}\definition{type of tree the grows in the bush with fruit that is black outside and green and red inside and is eaten by cassowaries; similar to sir tree fruit}\sensenumber{2}\definition{type of grubAko wälläng llo me dan, ddäddäg ma dan.}\subentry{\headword{kämsir budar}\pos{n.}\definition{type of grubAko wälläng llo me dan, ddäddäg ma dan.}}}
\entry{kämtupi}{\headword{kämtupi}\pos{n.}\sensenumber{2}\definition{type of ginger}}
\entry{kän1}{\headword{kän1}\pos{n.}\sensenumber{2}\definition{type of big, round yam with a white interior and thorns}}
\entry{kän2}{\headword{kän2}\pos{S vi.}\sensenumber{1}\definition{to withdraw, come out}\example{Ngäna skul atta gongkän greid paeb me.}{I withdrew from school in grade five.}\sensenumber{2}\definition{to remove, take out, take off, undo}\example{Bäne sod de nängkän.}{Take off your shirt.}\example{Ngäna net de apte tär nängkänan.}{I undid the string on one side of the net.}\example{Gänyme nge kopnen ma patt de aeya dängkänän?}{Who took out the coconut dehusker (from the ground) here?}\allomorph{känan}\allomorph{ngkän}\allomorph{ngk}}
\entry{känaebag}{\headword{känaebag}\variant{dial. var. of}{känazbag}}
\entry{känakone}{\headword{känakone}\variant{var. of}{konakone}}
\entry{känazbag}{\headword{känazbag}\pos{adv.}\sensenumber{2}\definition{tomorrow}\example{Abo känazbag ikop bogmam.}{We will see each other tomorrow.}}
\entry{känär}{\headword{känär}\pos{n.}\sensenumber{1}\definition{type of edible grub found in the bush}}
\entry{käntrokäntrom nane}{\headword{käntrokäntrom nane}\pos{ideo.}\sensenumber{1}\definition{gulp}\example{Ngäna ine de käntrokäntrom nänawan.}{I gulped the water.}}
\entry{känttatt}{\headword{känttatt}\pos{n.}\sensenumber{1}\definition{bedroom; room, chamber}\example{Bogo mɨnyi tuk me ulle abal känttatt de yantepägän.}{He will show you (pl.) a very big room above.}}
\entry{känz}{\headword{känz}\pos{S vi.}\sensenumber{1}\definition{to go aside, go off course}\example{Bogo llo gäbagäba de ikop dägagän a dädme towall gongkäzän a ttäle gottkewän.}{He saw a shady tree and went off into the grass there and folded his legs.}\allomorph{ngkäz}\allomorph{ngkäzmäll}\allomorph{känzmäll}}
\entry{kängkäl}{\headword{kängkäl}\pos{S vi.}\sensenumber{1}\definition{to ascend, climb, go up, rise}\example{Yäbäd a dingkälän tuk e.}{The sun rose up into the sky.}\sensenumber{2}\definition{to ascend, climb, go up}\example{Lla da llo de kälngkäl eran.}{The man is climbing the tree.}\example{Bogo manggo de dängkälän.}{She climbed the mango tree.}\allomorph{ngkäl}\allomorph{kälnan}\allomorph{käl}\allomorph{ngkäll}}
\entry{kängkäm}{\headword{kängkäm}\pos{S vt.}\sensenumber{2}\definition{to squeeze, press}\example{Ge kup a gärep käp yid kämnan e dan.}{This hole is for pressing grape juice (i.e. a winepress).}\example{Ubi sana kämnanang dag.}{They are squeezing sago.}\allomorph{käme}\allomorph{käm}}
\entry{känyär}{\headword{känyär}\pos{mod.}\sensenumber{1}\definition{quiet}\example{Dagirne dirom a känyär dae gongttägän.}{I stayed there and a cassowary came up quietly.}\example{Känyär giddollag be ddobae melemang deya.}{He was humble (lit. one who lives quietly) but hardworking.}\sensenumber{2}\definition{secret, secretly}\example{Bongo känyär nonttog, Godd aebe umllang bogon.}{If you give it secretly, only God will know.}\example{Obo pate oba känyär me gobällän a obom dangnoeyo.}{They went to him in secret and asked him.}\sensenumber{3}\definition{alone, by oneself (follows a genitive noun)}\example{Ngämi ngäma känyär ulleulle gogmam.}{We (excl.) grew up by ourselves.}\example{Bogo ma me daeya obo känyär.}{He was home alone.}\example{Bina känyär wiyamom ge ngättma we.}{Come by yourselves to this place.}\sensenumber{3}\definition{secret}\sensenumber{3}\definition{secret}\sensenumber{3}\definition{silent, quiet}\example{Imanuel gall guwo me känyärtto gogän.}{Imanuel was silent in the heart of the canoe.}\example{Yuwog abal lla da obom umllang dägaeyo känyärtto we.}{Many people told him to be quiet.}\sensenumber{3}\definition{to soothe}\example{Ause da llɨg kälsre de mäse känyerkänyer dägnän, be ddone mullae gogon.}{The old woman was trying to soothe the boy, but she couldn't.}\sensenumber{3}\definition{secretly}\example{Tämamae oba tatuma ibi ttoen de täbetäbe de dätäbeyo känyärkänyär.}{They all secretly planned a way to go to their washing place.}\subentry{\headword{känyär eka}\pos{n.}\definition{secret}}\subentry{\headword{känyär ttoen}\pos{n.}\definition{secret}}\subentry{\headword{känyärtto}\pos{mod.}\definition{silent, quiet}}\subentry{\headword{känyärkänyär2}\pos{A vt.}\definition{to soothe}}\subentry{\headword{känyärkänyär1}\pos{adv.}\definition{secretly}}}
\entry{känyer}{\headword{känyer}\variant{sp. var. of}{känyär}}
\entry{känyertto}{\headword{känyertto}\variant{var. of}{känyärtto}}
\entry{Känykäny}{\headword{Känykäny}\variant{dial. var. of}{Kinykiny}}
\entry{käp1}{\headword{käp1}\pos{n.}\sensenumber{1}\definition{fruit}\example{Llɨg a llo käp de notan.}{The boy ate the fruit.}\sensenumber{2}\definition{egg}\example{Ge pa da käp tumang zazeag dan.}{This bird lays many eggs.}\sensenumber{3}\definition{vague relatively small and round thing}\example{wayati käp, dädär käp, ttägäll käp, batri käp, tudi käp}{watermelon, rock, money, battery, fish hook}\example{Malla ngäna tumang kollba de naittnegan, tri käpdae.}{I didn't catch many fish, only three.}\sensenumber{4}\definition{nit}\sensenumber{4}\definition{bearing fruit}\example{Tomato da ddone käpang gogon.}{The tomato plant had no fruit.}\etymology{käp + =ang}\subentry{\headword{käpang}\pos{mod.}\definition{bearing fruit}}}
\entry{käp2}{\headword{käp2}\variant{sp. var. of}{kap}}
\entry{käpät}{\headword{käpät}\pos{n.}\sensenumber{4}\definition{moisture}\sensenumber{4}\definition{wet}\example{Ngämo tater a käptang agan.}{My mat get wet.}\sensenumber{4}\definition{wet, soaked}\example{Bogo walle we guspunän, käsre ma we käptakäptae gongosän.}{He fell in the water, so he came back home soaking wet.}\allomorph{käpt}\etymology{käpät + =ang}\subentry{\headword{käpätang}\pos{mod.}\definition{wet}}\subentry{\headword{käpäkäpät}\pos{adv.}\definition{wet, soaked}}}
\entry{käpkumett}{\headword{käpkumett}\pos{n.}\sensenumber{4}\definition{type of tree}}
\entry{käpom}{\headword{käpom}\pos{n.}\sensenumber{4}\definition{type of big tree that grows in the bush with white flowers and edible white fruit; used as medicine for cough}}
\entry{käpre}{\headword{käpre}\pos{n.}\sensenumber{4}\definition{type of big yam with a white interior, thorns, and few hairs}}
\entry{kär pipiem}{\headword{kär pipiem}\pos{n.}\sensenumber{4}\definition{purple-tailed imperial pigeonDdobae pipllugag pa dan. (It's a bird that's a strong flier.)}}
\entry{käsre}{\headword{käsre}\pos{adv.}\sensenumber{4}\definition{then}\example{Polle da popang gogän, käsre ge ddäddäg a gazgez de gongkaemnegän.}{The fence got a hole, and then this animal started to escape.}}
\entry{kästom}{\headword{kästom}\variant{var. of}{kastom}}
\entry{kätam}{\headword{kätam}\pos{S vi.}\sensenumber{1}\definition{to splash (of an aquatic animal)}\example{Tärko da mängalae källa gokätaemän.}{The small fish quickly splashed its tail.}\sensenumber{2}\definition{to flip, do a cartwheel}\allomorph{kätaem}}
\entry{kätäräl}{\headword{kätäräl}\pos{n.}\sensenumber{2}\definition{color}\example{Obo ikop kätäräl bätbät att a opnaeyan pällampällam e.}{His eyes changed from black to white.}\sensenumber{2}\definition{multicolored}\example{Apapi da ddob erag käträkäträl dag.}{There are some butterflies which are multicolored.}\subentry{\headword{käträkäträl}\pos{mod.}\definition{multicolored}}}
\entry{käträl}{\headword{käträl}\variant{fast speech var. of}{kätäräl}}
\entry{kätt}{\headword{kätt}\pos{n.}\sensenumber{1}\definition{bivalve; shell of a molluscWalle ik me täpe me giddollag ddäddäg. Upe kutt a nge kokllo ma dan. (An edible animal that lives in the mud in the water. The outer shell is for scratching coconuts.)}\sensenumber{2}\definition{shell bladeTropnen ma dan. Otɨt yuatt kllanen ma dan. (It's for cutting. It's for scraping cooked food.)}\sensenumber{3}\definition{(slang) vagina, vulva}\sensenumber{4}\definition{vague hard and thin thing (e.g. a blade)}\example{tubukätt}{kneecap}\sensenumber{4}\definition{blue (lit. 'shell water')}\sensenumber{4}\definition{type of game played with shells}\subentry{\headword{kätt ine}\pos{col.}\definition{blue (lit. 'shell water')}}\subentry{\headword{kätt tongoe}\pos{n.}\definition{type of game played with shells}}}
\entry{kättapun}{\headword{kättapun}\pos{n.}\sensenumber{4}\definition{type of reedTater inen ma za källäm me päddabag dan. (It grows in ponds and is used for weaving mats.)}}
\entry{kättekätte}{\headword{kättekätte}\pos{n.}\sensenumber{4}\definition{red-cheeked parrotLlo ik me giddollag dan. (It lives in trees.)}}
\entry{kättkätt1}{\headword{kättkätt1}\pos{n.}\sensenumber{1}\definition{weaving pattern with alternating cross and chevron}\sensenumber{2}\definition{to start a new weaving pattern}\example{Ibetty tätäm wipellgallagallab alle nyäng de däkättän.}{Yesterday, Ibetty switched from the wipellgallagallab design to a new one.}\allomorph{kätt}}
\entry{kättkätt2}{\headword{kättkätt2}\pos{S vt.}\sensenumber{2}\definition{to fence, wall, build a fence or wall}\example{Ngäna polle de käg alle kättkätt eran.}{I'm making a fence with the palm.}\example{Bogo dädär alle polle de däkättän.}{He built a wall out of stone.}\allomorph{kätt}}
\entry{kättlla}{\headword{kättlla}\pos{n.}\sensenumber{2}\definition{type of tree that grows in the bush with white flowers}}
\entry{Kättpälläk bällämang}{\headword{Kättpälläk bällämang}\pos{pn.}\sensenumber{2}\definition{Jerry Dareda's sacred place (near ttälebun, on the road to Kinkin, near Binyomoll)}}
\entry{käza}{\headword{käza}\pos{n.}\sensenumber{1}\definition{Hall's New Guinea crocodile}\example{Ngäna käza de nägliban.}{I scared away the crocodile.}\sensenumber{2}\definition{iguana}}
\entry{käza allko}{\headword{käza allko}\pos{n.}\sensenumber{2}\definition{type of fly that antagonizes other flies}}
\entry{käza bädma}{\headword{käza bädma}\pos{n.}\sensenumber{2}\definition{type of plant that only the crocodile clan wears when going hunting for crocodiles; also a traditional medicine}}
\entry{käza burala}{\headword{käza burala}\pos{n.}\sensenumber{2}\definition{water lily}}
\entry{Käza kup ine ma}{\headword{Käza kup ine ma}\pos{pn.}\sensenumber{2}\definition{well and sago place of Kwakmae in Limol (behind aid post)}}
\entry{käza wirwir1}{\headword{käza wirwir1}\pos{n.}\sensenumber{2}\definition{frilled monarch}}
\entry{käza wirwir2}{\headword{käza wirwir2}\pos{A vi.}\sensenumber{2}\definition{to call a crocodile out from the water}\example{Niki sisri ag me käza wirwir agnan walle menae me.}{This morning, Niki was calling the crocodile on the side of the water.}}
\entry{käzabun}{\headword{käzabun}\pos{n.}\sensenumber{2}\definition{large sago flower}\etymology{lit. 'crocodile head'}}
\entry{käzapig}{\headword{käzapig}\pos{n.}\sensenumber{2}\definition{type of tree with big, sour, black fruit and white and blue flowers}}
\entry{ke}{\headword{ke}\pos{int. ptcl.}\sensenumber{1}\definition{question particle}\example{Ewatta ke bibi ddobaeddobae lel amalla?}{Why are you all so afraid?}\example{Aenen ke ngämo mäg a?}{Who is my mother?}\sensenumber{2}\definition{counterfactual}}
\entry{keam}{\headword{keam}\pos{ideo.}\sensenumber{2}\definition{sound made by deer}}
\entry{Keisi}{\headword{Keisi}\pos{pn.}\sensenumber{2}\definition{female personal name}}
\entry{Keit}{\headword{Keit}\variant{sp. var. of}{Kate}}
\entry{Keith}{\headword{Keith}\pos{pn.}\sensenumber{2}\definition{male personal name}}
\entry{kek}{\headword{kek}\pos{n.}\sensenumber{2}\definition{orange-footed scrubfowlBädab dowae e ekawang pa dan. (It's a bird that sings near dawn.)}}
\entry{Keke}{\headword{Keke}\pos{pn.}\sensenumber{2}\definition{female personal name}}
\entry{Keks}{\headword{Keks}\pos{pn.}\sensenumber{2}\definition{personal name}}
\entry{kemibi}{\headword{kemibi}\pos{quant.}\sensenumber{2}\definition{many}\example{Ngäma ma me nongg a kemibi abal dag.}{In my house, there are many geckos.}}
\entry{kemol}{\headword{kemol}\pos{n.}\sensenumber{2}\definition{camel}\etymology{from Englishcamel}}
\entry{kemp}{\headword{kemp}\pos{n.}\sensenumber{2}\definition{camp}\etymology{from Englishcamp}}
\entry{Kemu}{\headword{Kemu}\pos{pn.}\sensenumber{2}\definition{male personal name}}
\entry{Kename}{\headword{Kename}\pos{pn.}\sensenumber{2}\definition{Kename (village in Gogodala Rural LLG; on an island in the Fly River)}}
\entry{Keni}{\headword{Keni}\variant{sp. var. of}{Kenny}}
\entry{Kenny}{\headword{Kenny}\pos{pn.}\sensenumber{2}\definition{male personal name}}
\entry{kep}{\headword{kep}\pos{n.}\sensenumber{2}\definition{hip}\sensenumber{2}\definition{hip bone}\subentry{\headword{kep kutt}\pos{n.}\definition{hip bone}}}
\entry{keräma pudd}{\headword{keräma pudd}\pos{n.}\sensenumber{2}\definition{type of small yam with a white interior}}
\entry{kerema}{\headword{kerema}\pos{n.}\sensenumber{2}\definition{type of taro}}
\entry{Keren}{\headword{Keren}\pos{pn.}\sensenumber{2}\definition{female personal name}}
\entry{Kergowa}{\headword{Kergowa}\pos{pn.}\sensenumber{2}\definition{Kergowa (in Gogodala Rural LLG; near Balimo)}}
\entry{Keriso}{\headword{Keriso}\pos{pn.}\sensenumber{2}\definition{Christ}}
\entry{Kesa}{\headword{Kesa}\pos{pn.}\sensenumber{2}\definition{male personal name}}
\entry{Kesama}{\headword{Kesama}\pos{pn.}\sensenumber{2}\definition{male personal name}}
\entry{keta}{\headword{keta}\pos{n.}\sensenumber{2}\definition{roof postSanador kalltaematt täne pittnen e. (Split sago stalks for weaving roofing onto.)}\example{Täne pittnen e keta da dagaya ottamänan.}{The stalk weaving has finished for the roof.}}
\entry{Keti}{\headword{Keti}\pos{pn.}\sensenumber{2}\definition{Kurupel Täräp (Limol village), which was moved from Old Limol to Old Man Kurupel's camping place approximately four generations before 2015}}
\entry{ketmar}{\headword{ketmar}\pos{n.}\sensenumber{1}\definition{type of tree that grows in old gardens; after being skinned and soaked, it is weaved into skirts}\sensenumber{2}\definition{two yams hanging from a stick in the center of a yam counting pile}}
\entry{ketol}{\headword{ketol}\pos{n.}\sensenumber{2}\definition{kettle}\etymology{from Englishkettle}}
\entry{Ketrin}{\headword{Ketrin}\pos{pn.}\sensenumber{2}\definition{female personal name}}
\entry{ketrop}{\headword{ketrop}\pos{n.}\sensenumber{1}\definition{season when rain clouds form but are blown away by the wind}\sensenumber{2}\definition{to stop the rain}\example{Bogo mɨnyi yogoll de ketrop bägagän.}{She will stop the rain.}}
\entry{Kewameyato}{\headword{Kewameyato}\pos{pn.}\sensenumber{2}\definition{female personal name}}
\entry{keyadaola}{\headword{keyadaola}\pos{n.}\sensenumber{2}\definition{type of bird}}
\entry{Kevelyn}{\headword{Kevelyn}\pos{pn.}\sensenumber{2}\definition{female personal name}}
\entry{Kiata}{\headword{Kiata}\pos{pn.}\sensenumber{2}\definition{male personal name}}
\entry{Kibobma}{\headword{Kibobma}\pos{pn.}\sensenumber{2}\definition{Kibobma (previous settlement of Limol village; on the road to Kinkin, near the creeks)}}
\entry{Kibul}{\headword{Kibul}\variant{var. of}{Kibuli}}
\entry{Kibuli}{\headword{Kibuli}\pos{pn.}\sensenumber{2}\definition{Kibuli (Em-speaking village in Oriomo-Bituri Rural LLG; near Kurunti)}}
\entry{Kidarga}{\headword{Kidarga}\pos{pn.}\sensenumber{2}\definition{male personal name}}
\entry{kidwe}{\headword{kidwe}\pos{n.}\sensenumber{2}\definition{millipede}}
\entry{kikiem}{\headword{kikiem}\pos{n.}\sensenumber{2}\definition{type of birdGudae ag sirem me ekawang pa dan. (It's a bird that sings at dawn.)}}
\entry{kiklem}{\headword{kiklem}\pos{n.}\sensenumber{2}\definition{type of small edible rodent}}
\entry{Kikori}{\headword{Kikori}\pos{pn.}\sensenumber{1}\definition{Kikori (town in Kikori District, located on the Kikori Delta)}\sensenumber{2}\definition{Kikori (river that flows into the Gulf of Papua)}}
\entry{Kila}{\headword{Kila}\pos{pn.}\sensenumber{2}\definition{personal name}}
\entry{kili}{\headword{kili}\pos{n.}\sensenumber{1}\definition{happiness, joy}\example{Kili da daden.}{There is joy.}\sensenumber{2}\definition{happy}\example{Ngämi tämamae ddone ada kili gogmam.}{We (excl.) were all so happy.}\example{Ngäna bäne pate ddone kili allan.}{I am not happy with you.}\sensenumber{3}\definition{to praise}\example{Bogo kili nägagan.}{He praised him.}\sensenumber{3}\definition{happy}\example{Obo ingoll a ddone kiliang agan.}{His face was not happy.}\sensenumber{1}\definition{unhappy, sad, annoyed}\example{Män kälsre da kilimeny agan.}{The little girl was sad.}\example{Bogo kilimeny gongäsän ma we.}{He went home unhappy.}\sensenumber{2}\definition{to insult}\example{Bogo kilimeny nägagan.}{He insulted him.}\sensenumber{1}\definition{to greet}\example{Bogo mɨnyi ballän kilikili bägawän a ttang bänttepägän.}{He will go greet him and shake his hand.}\sensenumber{2}\definition{to rejoice, celebrate}\example{Däbe amom de kilikili nägnallo didri?}{Whom are they celebrating?}\example{Lla da bobällnän kilikili bägayaebneyo.}{People will go and celebrate.}\sensenumber{2}\definition{happily, joyfully}\example{Meri ma we kilikiliangae dallän.}{Mary went home happily.}\allomorph{kili}\etymology{kili + =ang}\subentry{\headword{kiliang}\pos{mod.}\definition{happy}}\subentry{\headword{kilimeny}\pos{mod.}\definition{unhappy, sad, annoyed}}\subentry{\headword{kilikili}\pos{A vt.}\definition{to greet}}\subentry{\headword{kilikiliang}\pos{adv.}\definition{happily, joyfully}}}
\entry{kina}{\headword{kina}\pos{n.}\sensenumber{2}\definition{kina (PGK, the currency of Papua New Guinea)}}
\entry{kinekineang}{\headword{kinekineang}\pos{mod.}\sensenumber{2}\definition{smart}}
\entry{Kini}{\headword{Kini}\pos{pn.}\sensenumber{2}\definition{Kini (in Gogodala Rural LLG; near Balimo and Awaba)}}
\entry{Kinkin}{\headword{Kinkin}\pos{pn.}\sensenumber{2}\definition{Kinkin (Ende- and Taeme-speaking village in Oriomo-Bituri Rural LLG, near Limol)}}
\entry{kinpop}{\headword{kinpop}\pos{n.}\sensenumber{2}\definition{type of tree that grows in the bush with white flowers and wood used for house sticks and rafters}}
\entry{Kingsli}{\headword{Kingsli}\pos{pn.}\sensenumber{2}\definition{male personal name}}
\entry{Kinykiny}{\headword{Kinykiny}\variant{var. of}{Kinkin}}
\entry{Kiongga}{\headword{Kiongga}\pos{pn.}\sensenumber{2}\definition{Kiongga (toponym)}}
\entry{kip}{\headword{kip}\pos{n.}\sensenumber{2}\definition{top of a plant}\example{llo kip}{treetop}}
\entry{kip papa}{\headword{kip papa}\pos{n.}\sensenumber{2}\definition{top of fence}}
\entry{Kiplin}{\headword{Kiplin}\pos{pn.}\sensenumber{2}\definition{male personal name}}
\entry{Kipling}{\headword{Kipling}\pos{pn.}\sensenumber{2}\definition{male personal name}}
\entry{kire}{\headword{kire}\pos{mod.}\sensenumber{2}\definition{unripe; raw, fresh}\sensenumber{2}\definition{green}\subentry{\headword{kirekire}\pos{col.}\definition{green}}}
\entry{kisin}{\headword{kisin}\pos{n.}\sensenumber{2}\definition{kitchen}\etymology{from Englishkitchen}}
\entry{Kit}{\headword{Kit}\variant{sp. var. of}{Keith}}
\entry{kitapatt}{\headword{kitapatt}\pos{n.}\sensenumber{2}\definition{type of bandicoot-like animal}}
\entry{kitar}{\headword{kitar}\pos{n.}\sensenumber{2}\definition{floating grass}}
\entry{kito}{\headword{kito}\pos{n.}\sensenumber{2}\definition{type of black palm; in this immature stage, the shoots eaten as medicine and used to make baskets Used to build house and mat in the bushDu wätät dan wälläng me. (A wild food source in the bush.)}}
\entry{Kiwae}{\headword{Kiwae}\variant{sp. var. of}{Kiwai}}
\entry{Kiwai}{\headword{Kiwai}\pos{pn.}\sensenumber{2}\definition{Kiwai language (offical language of the region, native language of Daru; children's school songs are sometimes in this language)}}
\entry{kiyaddadda}{\headword{kiyaddadda}\pos{n.}\sensenumber{2}\definition{paradise kingfisher (common, buff-breasted, little)Wällang me kulläb ik me giddollag dan. (It's a bird that lives in big termite mounds in the bush.)}}
\entry{kɨllakɨlle}{\headword{kɨllakɨlle}\pos{S vt.}\sensenumber{2}\definition{to scrape}\allomorph{klla}\allomorph{klle}}
\entry{kɨllɨll}{\headword{kɨllɨll}\pos{n.}\sensenumber{2}\definition{type of snakeKukollkukoll a tubutubu dan. Kuddäll e lla ddäddägang dan. (It's green and short. It bites people to death.)}\example{Ngäna kɨllɨll di näddägan.}{I ate the killill snake.}}
\entry{kɨllɨm}{\headword{kɨllɨm}\variant{var. of}{källäm}}
\entry{kɨllkɨll}{\headword{kɨllkɨll}\pos{S vt.}\sensenumber{2}\definition{to dig}\example{Auma däkällallo.}{They dig a grave.}\example{Gottamänän dibaballe kup di däkällneg a pos de däganeg.}{I finished, and then I dug the holes and put the posts in.}\allomorph{käll}\allomorph{kɨll}}
\entry{klak1}{\headword{klak1}\pos{n.}\sensenumber{2}\definition{court clerkLla da aya darbnen att peba de pänggmeny anggan. (The person who looks after documents.)}\etymology{from Englishclerk}}
\entry{klak2}{\headword{klak2}\pos{n.}\sensenumber{2}\definition{type of harpoon for fishing}}
\entry{klas}{\headword{klas}\pos{n.}\sensenumber{2}\definition{class}\etymology{from Englishclass}}
\entry{klasrum}{\headword{klasrum}\pos{n.}\sensenumber{2}\definition{classroom}}
\entry{klloklloe}{\headword{klloklloe}\pos{S vi.}\sensenumber{1}\definition{to gather, join, come together}\example{Ubi goklloeyän Malläm ttängäm e.}{They gathered in Malam village.}\sensenumber{2}\definition{to gather, bring together, mix}\example{Chairman a obo melem a ada dan ge, lla de baklloenegnän eka tameny e.}{This is the chairman's job: he will gather the people for discussions.}\example{Ngämi eka de klloklloe panynen eralla.}{We are speaking mixing languages.}\allomorph{kolloe}\allomorph{klloe}\allomorph{kllo}}
\entry{kllomokllomoll}{\headword{kllomokllomoll}\pos{adv.}\sensenumber{2}\definition{downwind, leeward}\example{Däräng dedme dinduag a ada gogon, wel kllomokllomoll dallän.}{Dog went running like this; he went downwind.}}
\entry{kllum}{\headword{kllum}\variant{fast speech var. of}{kullum}}
\entry{ko1}{\headword{ko1}\variant{sp. var. of}{koo}}
\entry{ko2}{\headword{ko2}\variant{fast speech var. of}{ako}}
\entry{Kobam}{\headword{Kobam}\pos{pn.}\sensenumber{2}\definition{male personal name}}
\entry{kobädd}{\headword{kobädd}\pos{n.}\sensenumber{2}\definition{type of tree}}
\entry{Kobddag}{\headword{Kobddag}\pos{pn.}\sensenumber{2}\definition{Kobddag (toponym)}}
\entry{Kobe}{\headword{Kobe}\pos{pn.}\sensenumber{2}\definition{female personal name}}
\entry{kobe}{\headword{kobe}\pos{n.}\sensenumber{2}\definition{type of tree that grows along creeks (\textbackslashtextasciitilde 3 m) with white and red or blue and purple flowers and edible red fruit with black or white stripes and 4-6 seeds inside}}
\entry{kobeam}{\headword{kobeam}\pos{kin.}\sensenumber{2}\definition{co-brother-in-law (man's wife's sister's husband) Ttongdae mäda watt mälla tramang. (Taking two women from the same father.)}}
\entry{Kobemitang}{\headword{Kobemitang}\pos{pn.}\sensenumber{2}\definition{Kobemitang (toponym)}}
\entry{kobeyam}{\headword{kobeyam}\variant{sp. var. of}{kobeam}}
\entry{Koboddag}{\headword{Koboddag}\pos{pn.}\sensenumber{2}\definition{Koboddag (toponym)}}
\entry{kodor}{\headword{kodor}\pos{n.}\sensenumber{2}\definition{piece, lump}\example{ekaklle kodor}{plot of land}\example{Alla ubira sana kodor de basiaemeya?}{Can you give them a piece of sago?}}
\entry{kodowa}{\headword{kodowa}\pos{n.}\sensenumber{2}\definition{dish consisting of sago cooked in leaves on the fire}}
\entry{Koe}{\headword{Koe}\pos{pn.}\sensenumber{2}\definition{male personal name}}
\entry{Koebänang}{\headword{Koebänang}\pos{pn.}\sensenumber{2}\definition{Koebänang (sago and hunting place; old settlement near Buddobuddog)}}
\entry{Koebnang}{\headword{Koebnang}\variant{fast speech var. of}{Koebänang}}
\entry{koeme}{\headword{koeme}\pos{n.}\sensenumber{2}\definition{type of tree that grows along creeks with edible, round red fruit}}
\entry{koemekoeme}{\headword{koemekoeme}\pos{n.}\sensenumber{2}\definition{type of tree that grows along creeks with inedible, small red fruit}\etymology{redup. of koeme}}
\entry{koen}{\headword{koen}\pos{S vi.}\sensenumber{2}\definition{to turn back}\example{Ngäna llowam atta koen allan.}{I'm turning back due to fatigue.}\allomorph{koemeny}\allomorph{ngkoen}}
\entry{koenbäll}{\headword{koenbäll}\pos{n.}\sensenumber{2}\definition{type of tree that grows in the bush (especially in old gardens) with hanging green fruit and liquid used to treat sores}}
\entry{Koenbäll kutt}{\headword{Koenbäll kutt}\pos{pn.}\sensenumber{2}\definition{Koenbäll kutt (sago and washing place of Paine and Warama Kurupel)}}
\entry{koenmäll}{\headword{koenmäll}\pos{S vt.}\sensenumber{2}\definition{to chase}\example{Bogo obom nängkoenmällan.}{He chased him.}\example{Ttongo täräp me llɨg kälekäle da gotäbanegän pa koenmäll e.}{One time, little boys planned to chase birds.}\example{Dadabi, mäsemäse llɨg a dongkoenmällaemneyo.}{The boys chased them naked.}\example{Lla da poura de nängkoenmällnegan.}{The man chased four chickens.}\allomorph{ngkoin}\allomorph{koin}\allomorph{koen}\allomorph{ngkoen}\allomorph{ngkoenmäll}\allomorph{ngkoinmäll}}
\entry{koepang}{\headword{koepang}\pos{mod.}\sensenumber{2}\definition{sour}\example{Ddob sana gudne da koepangkoepang mokowang dag.}{Some old pieces of sago taste sour.}}
\entry{koinmäll}{\headword{koinmäll}\variant{sp. var. of}{koenmäll}}
\entry{kok1}{\headword{kok1}\pos{kin.}\sensenumber{1}\definition{grandchild (one's child's child)Mända bäne llɨg de medäda kok eka eran. (The father calls his daughter's children kok.)}\sensenumber{2}\definition{daughter-in-law (one's son's wife)}\sensenumber{2}\definition{nonsingular form of kok}\allomorph{ko}\subentry{\headword{kokok}\pos{kin.}\definition{nonsingular form of kok}}}
\entry{kok2}{\headword{kok2}\pos{n.}\sensenumber{1}\definition{moonDdapall me to ulle dan, iddob me indrang allan. (It's a big light in the sky; it shines at night.)}\example{Kok ekaklle de nganaenen eran.}{The moon spins around Earth.}\sensenumber{2}\definition{month}\example{Ada gongnomenyne, ada ka ttongdae kok me daeya sukul a.}{I used to think that school was one month long.}\sensenumber{2}\definition{half moon}\sensenumber{2}\definition{moonlight}\sensenumber{2}\definition{crescent moon}\sensenumber{2}\definition{new moon}\sensenumber{2}\definition{first quarter moon}\sensenumber{2}\definition{full moon}\sensenumber{2}\definition{first light (the end of a new moon)}\sensenumber{2}\definition{orgy}\sensenumber{2}\definition{moonlight}\example{Kokta da näbdaban.}{Moonlight is dawning.}\example{Kokta da indrang abal bogon.}{Moon shines so bright.}\sensenumber{2}\definition{to go a month without menstruating}\sensenumber{2}\definition{fast}\example{Ubi mɨnyi kok tärangg me bazärnän ge wik me.}{They will be fasting this week.}\etymology{lit. 'sliver banana moon'}\subentry{\headword{kok apte}\pos{n.}\definition{half moon}}\subentry{\headword{kok indrang}\pos{n.}\definition{moonlight}}\subentry{\headword{kok kätt pälläk}\pos{n.}\definition{crescent moon}}\subentry{\headword{kok kuddäll}\pos{n.}\definition{new moon}}\subentry{\headword{kok pälläk}\pos{n.}\definition{first quarter moon}}\subentry{\headword{kok pllayang}\pos{n.}\definition{full moon}}\subentry{\headword{kok sisor}\pos{n.}\definition{first light (the end of a new moon)}}\subentry{\headword{kok to}\pos{n.}\definition{orgy}}\subentry{\headword{kokta}\pos{n.}\definition{moonlight}}\subentry{\headword{kok a nganzig}\pos{p.}\definition{to go a month without menstruating}}\subentry{\headword{kok tärangg}\pos{n.}\definition{fast}}}
\entry{kok3}{\headword{kok3}\pos{n.}\sensenumber{2}\definition{grasshopper}\example{Mälla da sisri ag me kok de nägäddaeballo kollba tokong e.}{This morning, the women caught grasshoppers for fishing bait.}}
\entry{kok patar}{\headword{kok patar}\pos{n.}\sensenumber{2}\definition{necklace}}
\entry{kokall}{\headword{kokall}\pos{n.}\sensenumber{2}\definition{type of palm with fruit that are smaller than coconutsWälläng me nge ingoll käpang za dan. (A thing in the bush with coconut-like fruits.)}\sensenumber{2}\definition{two friends who share a twin fruit from the kokall tree (palm tree in the bush)Mällayaba ttaem. (Amongst women.)}\subentry{\headword{kokallma}\pos{n.}\definition{two friends who share a twin fruit from the kokall tree (palm tree in the bush)Mällayaba ttaem. (Amongst women.)}}}
\entry{kokallkokall}{\headword{kokallkokall}\pos{n.}\sensenumber{2}\definition{type of tree}\etymology{redup. of kokall}}
\entry{kokäl}{\headword{kokäl}\pos{n.}\sensenumber{2}\definition{small mudcrab}}
\entry{Koke}{\headword{Koke}\pos{pn.}\sensenumber{2}\definition{Koke (toponym)}}
\entry{kokeya moleg}{\headword{kokeya moleg}\pos{n.}\sensenumber{2}\definition{lost sacred stone with markings on it; represents a girl that belongs to the Dumollang clan}}
\entry{kokllo}{\headword{kokllo}\pos{S vt.}\sensenumber{2}\definition{to scratch, scrape}\example{Ngäna nge de kokllo eran.}{I'm scraping a coconut.}\example{Lla ulle da bandrabandrog bun akllonan.}{The big man was scratching his head while dancing.}\allomorph{kllo}\allomorph{koll}}
\entry{Kokma}{\headword{Kokma}\pos{pn.}\sensenumber{2}\definition{Kokma (toponym)}}
\entry{kokmer}{\headword{kokmer}\pos{n.}\sensenumber{2}\definition{when a hunter calls out his sister's name after shooting an animal (she will get the back of the animal if it's killed)}}
\entry{kokne}{\headword{kokne}\pos{n.}\sensenumber{2}\definition{type of tree that grows in the grassland with blue flowers and edible blue fruitKängkäm ma dan nane we itrellang me. (To crush and drink the water when sick.)}}
\entry{kokngal}{\headword{kokngal}\pos{n.}\sensenumber{1}\definition{hunting in the rain}\example{Kokngal e ibi täräp a eran yogoll täräp me dan.}{The time to go kokngal hunting is when it's rainy.}\sensenumber{2}\definition{season characterized by heavy rain, hunting, and fishing with lines and nets (third season; corresponds to mid-February)}}
\entry{koko1}{\headword{koko1}\pos{S vt.}\sensenumber{2}\definition{to cut (flesh or meat), butcher}\example{Bogo ddäddäg de dokonggän oblle.}{He cut the meat for her.}\example{Ddia koko we ngäna llɨg di dokom.}{I took the boys to butcher the deer.}\allomorph{ko}\allomorph{ngko}}
\entry{koko2}{\headword{koko2}\pos{n.}\sensenumber{2}\definition{shoot (of a plant)}\example{Mätta koko da dade de nganae eran.}{The yam shoot is wrapping around the yam stick.}\sensenumber{1}\definition{sprouted}\example{Nge kokowang a ibeny ma dan a ddob otnan ma dag.}{Some sprouted cocounts are for planting and others are for eating.}\sensenumber{2}\definition{the first stage of coconut growth in which the shoot has just emerged (before planting)}\etymology{koko + =ang}\subentry{\headword{kokowang}\pos{mod.}\definition{sprouted}}}
\entry{kokoang}{\headword{kokoang}\variant{sp. var. of}{kokowang}}
\entry{kokol}{\headword{kokol}\pos{n.}\sensenumber{2}\definition{type of introduced bananaTupi pättang dan. Käp a, pätt a, bugu da, wa ttam a, tämamae kätäräll a obo sägäsägäd dag. O me käp a yu ma da, ada kire da ade. Be obo moko da kälakälae tomowang dan. (It has a tall trunk. The fruit, stem, and leaves are all a yellow color. When ripe, and also when unripe, its fruit is cooked. But its taste is a bit sour.)}}
\entry{kokop}{\headword{kokop}\pos{S vt.}\sensenumber{2}\definition{to peel, skin, husk}\example{Ge lla da nae de kokop eran.}{This man is peeling a sweet potato.}\example{Ttongo lla ulle da up o de näkopnan.}{A big man peeled a ripe banana.}\allomorph{kop}}
\entry{kokopasi}{\headword{kokopasi}\pos{n.}\sensenumber{2}\definition{shrikethrush (Arafura, rufous)Obo ma da nyeny pimbyom alle gogoatt dan. (Its nest is made from hollowed out nyeny bark.)}}
\entry{kokpe}{\headword{kokpe}\pos{n.}\sensenumber{2}\definition{type of tree with yellow flowers and leaves that are used to wrap food for cooking on the fire}}
\entry{koktakokta}{\headword{koktakokta}\pos{n.}\sensenumber{2}\definition{when dead trees or tree branches become dry and shine in the night}\etymology{redup. of kokta}}
\entry{kokto}{\headword{kokto}\pos{S vt.}\sensenumber{2}\definition{to bail (water)}\example{Ngäna gall e godagal, ine dokto.}{I got in the canoe and bailed the water out.}\allomorph{kto}}
\entry{kol}{\headword{kol}\pos{n.}\sensenumber{2}\definition{sago pith}}
\entry{kolem}{\headword{kolem}\pos{n.}\sensenumber{2}\definition{type of palm with coconuts that come in red or green varietiesMamam kollang dan, mäg a obene kaekepang dan. Täkällang dan. Wälläng menae o pällonpällon me päddabag dan, utt a wätat ma dan. (It has a red core; its base is hard to cut. It's thorny. It grows in the bush or among bushes; the shoots are eaten.)}}
\entry{koles}{\headword{koles}\variant{sp. var. of}{kolos}}
\entry{Kolmet}{\headword{Kolmet}\pos{pn.}\sensenumber{2}\definition{female personal name}}
\entry{Koloam}{\headword{Koloam}\pos{pn.}\sensenumber{2}\definition{male personal name}}
\entry{kolos}{\headword{kolos}\pos{n.}\sensenumber{2}\definition{constable uniform}\etymology{from Englishcolors}}
\entry{Kols}{\headword{Kols}\pos{pn.}\sensenumber{2}\definition{male personal name}}
\entry{kolwa}{\headword{kolwa}\pos{n.}\sensenumber{2}\definition{type of tree that grows in grassland with a trunk that has a diameter of \textbackslashtextasciitilde9 cm, but very strong and used for spears and weapons}}
\entry{koll}{\headword{koll}\pos{loc.}\sensenumber{1}\definition{inner part}\example{ttang, ttang koll}{hand, palm}\sensenumber{2}\definition{part of a bow}}
\entry{kollam}{\headword{kollam}\pos{S vt.}\sensenumber{2}\definition{to stand up}\example{Ngäna ma de dakollam.}{I stood the house up.}}
\entry{kollba}{\headword{kollba}\pos{n.}\sensenumber{2}\definition{fishWalle ik me giddollnenang ddäddäg dag. (They are animals that live in the water.)}\example{Bäne kollba mu da auli dan?}{How much for the fish?}\example{Ngämo kollba mu da ttongo ttang lläpät dan.}{My fish is five kina.}\example{Kollba da tämamae dadrowän a be kottllam aebe ttam dagirnän.}{The fish all died but the turtle lived on.}\example{Bogo komlla kollba ulle de deyaittän.}{He caught two big fish.}\sensenumber{2}\definition{fish fin}\subentry{\headword{kollba täkäll}\pos{n.}\definition{fish fin}}}
\entry{kolldab}{\headword{kolldab}\pos{v.}\sensenumber{2}\definition{stab}\allomorph{kolld}}
\entry{kolldän}{\headword{kolldän}\pos{S vt.}\sensenumber{2}\definition{to shoot; stab}\example{Sɨmell de enanae dukollädnegän pakos alle.}{He shot the pig straight through with the arrow.}\allomorph{kolld}\allomorph{kolläd}}
\entry{kollko}{\headword{kollko}\pos{n.}\sensenumber{2}\definition{breadfruitObo käp a wätät dan. Tep a ute mälanen ma dan. (Its fruit is food. The sap is put on sores.)}}
\entry{kollkokollko}{\headword{kollkokollko}\pos{n.}\sensenumber{2}\definition{type of tree that grows in bush with edible red fruit with an edible, big round nut in the seed}\etymology{redup. of kollko}}
\entry{kollkollatt}{\headword{kollkollatt}\variant{fr. var. of}{kullkullatt}}
\entry{kollmäll}{\headword{kollmäll}\pos{S vt.}\sensenumber{2}\definition{to follow}\example{kollmällang e ebdo}{the following day}\example{Llɨg a mänmän de dongkollmällaemeyo.}{The boys followed the girls.}\example{Abo bongo mamamamall ngämim yangkollmällneg.}{You must follow us quickly.}\example{Lama ngämo kollmällang dan.}{Lama is the second child after me (lit. Lama's following me).}\sensenumber{2}\definition{follower, disciple}\example{Yesu duwem gognän obo kolmällang aba peyang.}{Jesus ate with his disciples.}\allomorph{ngkollmäll}\allomorph{koll}\etymology{kollmäll + =ang}\subentry{\headword{kollmällang}\pos{n.}\definition{follower, disciple}}}
\entry{kollmoll}{\headword{kollmoll}\variant{var. of}{kollmäll}}
\entry{kollmos ttam}{\headword{kollmos ttam}\pos{n.}\sensenumber{2}\definition{soul}\example{Ngämo kollmos ttam de nänddämän kuddäll ngättma we.}{It sunk my spirit to a place of death.}}
\entry{kolloekolloe}{\headword{kolloekolloe}\variant{sp. var. of}{klloklloe}}
\entry{kollokolloe}{\headword{kollokolloe}\variant{sp. var. of}{klloklloe}}
\entry{kollong}{\headword{kollong}\pos{n.}\sensenumber{2}\definition{type of tree that grows in the bush; used for grass skirts}}
\entry{Kollwam}{\headword{Kollwam}\pos{pn.}\sensenumber{2}\definition{male personal name}}
\entry{kollwany}{\headword{kollwany}\pos{S vt.}\sensenumber{2}\definition{to hang}\example{Pakätt de llo ngoeang me dukullwenyallo.}{They hung the robe on the tree branch.}\allomorph{kullweny}\allomorph{kullwe}\allomorph{kollwanen}}
\entry{kom}{\headword{kom}\pos{n.}\sensenumber{1}\definition{hair; hair-like thing}\example{bun kom, ttatt kom, mätta kom}{hair on head, beard, yam root hair}\sensenumber{2}\definition{feather}\example{Bogo ansi gogon a kom a pädrällag gognegän.}{He sneezed and the feathers flew.}}
\entry{komälle}{\headword{komälle}\pos{n.}\sensenumber{2}\definition{type of game played with string}}
\entry{Kombosie}{\headword{Kombosie}\pos{pn.}\sensenumber{2}\definition{male personal name}}
\entry{komene}{\headword{komene}\pos{n.}\sensenumber{2}\definition{postpartum period during which a woman warms herself by the fire}}
\entry{kominiti}{\headword{kominiti}\pos{n.}\sensenumber{2}\definition{community}\etymology{from Englishcommunity}}
\entry{komiti}{\headword{komiti}\pos{n.}\sensenumber{2}\definition{committee}\etymology{from Englishcommittee}}
\entry{komlla}{\headword{komlla}\pos{num.}\sensenumber{2}\definition{two (yam counting numeral; also general)}\sensenumber{2}\definition{second}\example{Komlla we nge da ngäna notan.}{This is the second coconut I ate today.}\sensenumber{2}\definition{exactly two; both}\example{Mänyanda bi komllaebe ade tatu de gongkameyo.}{The younger two also started to wash.}\sensenumber{2}\definition{both}\example{Bogo komllakomlla panypenyang deya ada Tame eka da wa Ende eka da.}{She spoke both Taeme and Ende languages.}\etymology{komlla + =aebe}\subentry{\headword{komlla we}\pos{ord. num.}\definition{second}}\subentry{\headword{komllaebe}\pos{num.}\definition{exactly two; both}}\subentry{\headword{komllakomlla}\pos{quant.}\definition{both}}}
\entry{komlla komlla}{\headword{komlla komlla}\pos{num.}\sensenumber{2}\definition{four (yam counting numeral; 2+2)}\sensenumber{2}\definition{exactly four (2+2)}\etymology{redup. of komlla}\subentry{\headword{komllaebekomllaebe}\pos{num.}\definition{exactly four (2+2)}}}
\entry{komlla komlla ttongo dduma}{\headword{komlla komlla ttongo dduma}\pos{num.}\sensenumber{2}\definition{five (yam counting numeral; 2+2+1)}}
\entry{komlla putt}{\headword{komlla putt}\pos{num.}\sensenumber{2}\definition{twelve (yam counting numeral; 2×6)}}
\entry{komllaebme}{\headword{komllaebme}\variant{fr. var. of}{komllaebe}}
\entry{komllazegatt}{\headword{komllazegatt}\pos{n.}\sensenumber{2}\definition{twin}}
\entry{komlle}{\headword{komlle}\pos{n.}\sensenumber{2}\definition{type of game played with string}}
\entry{komo1}{\headword{komo1}\pos{n.}\sensenumber{1}\definition{centipede}\example{Maryanne bom komo da näddägan.}{Maryanne was bit by a centipede.}\sensenumber{2}\definition{scorpion}\sensenumber{2}\definition{type of black scorpion}\sensenumber{2}\definition{type of scorpion}\subentry{\headword{komo mogmog}\pos{n.}\definition{type of black scorpion}}\subentry{\headword{komo takle}\pos{n.}\definition{type of scorpion}}}
\entry{komo2}{\headword{komo2}\pos{n.}\sensenumber{2}\definition{gingerMälläng ik e kängkäm ma dan ansi täräp me. Ako otnan ma dan kumye täräp me. (It's to put in your nose when you have a runny nose. Also, it's to eat when coughing.)}}
\entry{komony}{\headword{komony}\pos{n.}\sensenumber{2}\definition{tongs}}
\entry{komotupi molle molle}{\headword{komotupi molle molle}\pos{n.}\sensenumber{2}\definition{type of dancing game that involves ginger}}
\entry{komuniti}{\headword{komuniti}\variant{sp. var. of}{kominiti}}
\entry{kona}{\headword{kona}\pos{n.}\sensenumber{1}\definition{corner}\example{Obo ma da duli kona me dan.}{His house is over there, on the corner.}\sensenumber{2}\definition{district, section, area (of a settlement)}\example{Ngäna sisri Llimoll me giddollnen allan Basido kona me.}{I am now living in Limol, in the Basido district.}\etymology{from Englishcorner}}
\entry{konakone}{\headword{konakone}\pos{S vt.}\sensenumber{1}\definition{to cover}\example{Yäbäd de ddapall käkan da dakonewän.}{Clouds covered the sun.}\sensenumber{2}\definition{cover, sheet, blanket}\example{Däbe män a konakone alle gokonewän.}{That girl covered up with a blanket.}\allomorph{kone}\allomorph{konanen}\allomorph{käne}\allomorph{kona}}
\entry{Kondobäll}{\headword{Kondobäll}\variant{sp. var. of}{Kondoboll}}
\entry{Kondobol}{\headword{Kondobol}\pos{pn.}\sensenumber{2}\definition{Kondobol (Taeme-speaking village in Morehead Rural LLG; from Limol, one must pass through Kinkin)}}
\entry{Kondoboll}{\headword{Kondoboll}\variant{var. of}{Kondobol}}
\entry{Kondobu}{\headword{Kondobu}\pos{pn.}\sensenumber{2}\definition{Konedobu (in Gogodala Rural LLG)}}
\entry{konkon}{\headword{konkon}\pos{mod.}\sensenumber{1}\definition{crazy, mad, insane, mentally ill}\example{Ge ause da konkonang daeya.}{This old woman was crazy.}\example{Ngäna konkon allan.}{I'm getting mad.}\example{Ngämo yae konkonang gogon.}{My mother became mentally ill.}\sensenumber{2}\definition{stupid, ignorant, foolish}\example{Ngäna konkonang dan, ngämo kame dan bina eka da.}{I am ignorant; I don't know your (pl.) language.}\sensenumber{3}\definition{intoxicated, intoxicating, consciousness-altering, drunk}\example{ine konkon me}{drunk}\example{Ine da ngänäm gagäll konkonang dagän.}{The drink made me really drunk.}}
\entry{konskak}{\headword{konskak}\pos{n.}\sensenumber{3}\definition{type of big taro}}
\entry{konzar}{\headword{konzar}\pos{n.}\sensenumber{3}\definition{type of small yam with a white interior, hairs, and small thorns}}
\entry{konykony}{\headword{konykony}\pos{n.}\sensenumber{3}\definition{type of stinging insect that lives in the ground}}
\entry{konymad}{\headword{konymad}\pos{n.}\sensenumber{3}\definition{man who steals a woman's things during her initiation ceremony}}
\entry{koo}{\headword{koo}\pos{TAM ptcl.}\sensenumber{3}\definition{until}}
\entry{kopae därängmeny}{\headword{kopae därängmeny}\pos{n.}\sensenumber{3}\definition{to go to another village to trade}\example{Llimollang a kopae därängmeny e Wim e abällan.}{Limol people went to Wim to trade.}}
\entry{kopek}{\headword{kopek}\pos{n.}\sensenumber{1}\definition{valley}\example{Ngängälatt dädär a mänyi tutu atta bolldaeyän kopek e.}{A round rock will roll from the hill to the valley.}\sensenumber{2}\definition{pit, hole}\example{källama kopek}{toilet (lit. 'hole for pooping')}}
\entry{kopllalle}{\headword{kopllalle}\pos{n.}\sensenumber{2}\definition{oriole (brown, olive-backed, green)Kämag me ekawang pa dan. (It's a bird that sings during storms.)}}
\entry{koplle}{\headword{koplle}\pos{n.}\sensenumber{2}\definition{type of big tree that grows in the bush with fruit that cassowaries eat}\sensenumber{2}\definition{type of game involving throwing koplle fruit}\subentry{\headword{koplle täkmäl täkmäl}\pos{n.}\definition{type of game involving throwing koplle fruit}}}
\entry{Koreya}{\headword{Koreya}\pos{pn.}\sensenumber{2}\definition{Korea}}
\entry{kormas}{\headword{kormas}\pos{n.}\sensenumber{2}\definition{type of birdTäl ik me giddollag dan. (It lives in bamboo.)}}
\entry{korolläm}{\headword{korolläm}\pos{n.}\sensenumber{2}\definition{type of vine that grows beside creeks. Leaves are used to wrap things.Sana tokma dan ako ddäddäg kodowa ma dan. (It's to wrap sago and for meat.)}}
\entry{kos}{\headword{kos}\pos{n.}\sensenumber{2}\definition{course}\etymology{from Englishcourse}}
\entry{kosrom}{\headword{kosrom}\pos{n.}\sensenumber{2}\definition{type of large mushroom that grows in the winterWälläng me päddabag, ddäddäg ma dan. (It grows in the bush; it's edible.)}}
\entry{kot1}{\headword{kot1}\pos{n.}\sensenumber{2}\definition{dirt}\sensenumber{2}\definition{dirty, unclean}\example{ttang kotang peyang duwem}{eating with dirty hands}\example{Ngäna sisri kaptte kotang de gällämnan anggan.}{I'm cleaning the dirty clothes now.}\sensenumber{2}\definition{clean, pure}\example{Ngämo umllang dan bongo aenen, bongo Adi bo Kotmeny dan!}{I know who you are; you are God's pure one!}\sensenumber{2}\definition{dirty, unclean}\example{gagäll kotkot anyke}{bad, unclean spirit}\etymology{kot + =ang}\subentry{\headword{kotang}\pos{mod.}\definition{dirty, unclean}}\subentry{\headword{kotmeny}\pos{mod.}\definition{clean, pure}}\subentry{\headword{kotkot}\pos{mod.}\definition{dirty, unclean}}}
\entry{kot2}{\headword{kot2}\variant{var. of}{kwatt}}
\entry{kote}{\headword{kote}\pos{n.}\sensenumber{2}\definition{napeBun a wa matta da oba amne me toko dan. (It's in between the head and shoulders and on top.)}\sensenumber{2}\definition{cervical vertebrae}\sensenumber{2}\definition{to hang one's head}\example{Lla da kote gottkamän.}{The man hung his head.}\etymology{lit. 'break the nape'}\subentry{\headword{kote kutt}\pos{n.}\definition{cervical vertebrae}}\subentry{\headword{kote ttäkam}\pos{S vi.}\definition{to hang one's head}}}
\entry{kotol}{\headword{kotol}\pos{n.}\sensenumber{1}\definition{traditional practice of sterilizing women after having 6–7 kids; the placenta is buried and a coconut is planted on top}\sensenumber{2}\definition{traditional practice of stopping smoking}}
\entry{kotom}{\headword{kotom}\pos{n.}\sensenumber{2}\definition{head covering, crown}}
\entry{kott}{\headword{kott}\pos{n.}\sensenumber{2}\definition{testicles}}
\entry{kottllam}{\headword{kottllam}\pos{n.}\sensenumber{2}\definition{turtle, tortoise}\example{Kottllam a ttongo källäm me dazernän.}{The turtles were living in a pond.}}
\entry{kowa}{\headword{kowa}\pos{n.}\sensenumber{2}\definition{upper back}}
\entry{kowatt}{\headword{kowatt}\variant{sp. var. of}{kwatt}}
\entry{Kral}{\headword{Kral}\pos{pn.}\sensenumber{2}\definition{female personal name}}
\entry{krismas}{\headword{krismas}\pos{n.}\sensenumber{2}\definition{Christmas}\etymology{from EnglishChristmas}}
\entry{kristen}{\headword{kristen}\pos{mod.}\sensenumber{2}\definition{Christian}}
\entry{kristin}{\headword{kristin}\variant{sp. var. of}{kristen}}
\entry{Kristina}{\headword{Kristina}\pos{pn.}\sensenumber{2}\definition{female personal name}}
\entry{KT}{\headword{KT}\variant{sp. var. of}{Keti}}
\entry{ku1}{\headword{ku1}\pos{loc.}\sensenumber{2}\definition{center, core, middle}\example{nyongo ku me}{in the middle of the road}\sensenumber{2}\definition{the ninth stage of coconut growth in which the endosperm of the fruit is fully formed and coconut water remains}\example{Da bongo nge de nängkäl, abo kuang dae nädd.}{Grandfather made me a water bucket.}\etymology{ku + =ang}\subentry{\headword{kuang}\pos{n.}\definition{the ninth stage of coconut growth in which the endosperm of the fruit is fully formed and coconut water remains}}}
\entry{kubäll}{\headword{kubäll}\variant{sp. var. of}{kubull}}
\entry{kube}{\headword{kube}\pos{n.}\sensenumber{2}\definition{bucket}\example{Masar ine kube de ngämo nangesan.}{Grandfather made me a water bucket.}}
\entry{kubllu}{\headword{kubllu}\pos{n.}\sensenumber{2}\definition{type of tree that grows in the bush and savannah with bark used for string; similar to kapang}}
\entry{kubull}{\headword{kubull}\pos{n.}\sensenumber{2}\definition{bush wallaby, dusky pademelon}}
\entry{kud}{\headword{kud}\pos{n.}\sensenumber{2}\definition{type of pandanus with fat triangular fruitWälläng wallemäg me päddabag dan. Ttongo ulle bomaboma mab dan, wätät ma dan. (It grows in the bush and by creeks. It has a wide trunk and is edible.)}}
\entry{kud dämadämar}{\headword{kud dämadämar}\pos{n.}\sensenumber{2}\definition{dragonfly}}
\entry{kudädäri}{\headword{kudädäri}\pos{n.}\sensenumber{2}\definition{Zoe's imperial pigeonObo ma da llo tomtom alle o llo toko me gogo eran. (It builds its nest from piles of sticks or on top of trees.)}}
\entry{kudu}{\headword{kudu}\pos{n.}\sensenumber{2}\definition{corner}}
\entry{Kudurwe}{\headword{Kudurwe}\pos{pn.}\sensenumber{2}\definition{female personal name}}
\entry{kuddäb}{\headword{kuddäb}\pos{n.}\sensenumber{2}\definition{raft}}
\entry{kuddäkuddäll}{\headword{kuddäkuddäll}\pos{mod.}\sensenumber{1}\definition{soft}\example{Mätta da kuddällkuddäll bognegän.}{The yams will become soft.}\sensenumber{2}\definition{easy}\example{Orpmang eka da kuddäkuddäll a era ngäna ai dan bondär.}{Wipi language is easy; I can understand it.}\example{Kuddällkuddäll dan baob wätät e.}{It's easy to eat waterlily.}\etymology{redup. of kuddäll}}
\entry{kuddäll}{\headword{kuddäll}\pos{n.}\sensenumber{1}\definition{death}\example{Gagäll dan lla da bem e buspunän kuddäll e.}{It is bad for a man to throw himself into the ocean to his death.}\sensenumber{2}\definition{to die}\example{Gullem a kuddäll agan.}{The snake died.}\sensenumber{3}\definition{dead}\example{sɨmell kuddäll}{dead pig}\example{kuddäll pätt}{dead body}\sensenumber{4}\definition{abandoned}\example{ma kuddäll, ttängäm kuddäll}{abandoned house, abandoned garden}\sensenumber{4}\definition{Passover}\sensenumber{4}\definition{life-or-death, to death, as if one may die}\example{Kuddäll ttam ngäna dindug.}{I ran for my life.}\example{Ubi kuddäll ttam gotongoenegnän.}{They were laughing to death.}\etymology{lit. 'death passing over'}\subentry{\headword{Kuddäll Opap}\pos{pn.}\definition{Passover}}\subentry{\headword{kuddäll ttam}\pos{adv.}\definition{life-or-death, to death, as if one may die}}}
\entry{kuddällkuddäll}{\headword{kuddällkuddäll}\variant{var. of}{kuddäkuddäll}}
\entry{kuddɨll}{\headword{kuddɨll}\variant{sp. var. of}{kuddäll}}
\entry{kuddukuddäll}{\headword{kuddukuddäll}\variant{var. of}{kuddäkuddäll}}
\entry{kuddukuddull}{\headword{kuddukuddull}\variant{sp. var. of}{kuddäkuddäll}}
\entry{Kui}{\headword{Kui}\pos{pn.}\sensenumber{4}\definition{Kui (toponym)}}
\entry{kui}{\headword{kui}\pos{n.}\sensenumber{4}\definition{island}\example{Ubi deagerneyo ttongo kui me.}{They (two) lived on an island.}}
\entry{kuib}{\headword{kuib}\pos{n.}\sensenumber{4}\definition{type of drumUpye llo att yu att kuib alläp pätt nyäny ma. (Made from a painted, fired upye tree.)}}
\entry{kuibiag}{\headword{kuibiag}\pos{n.}\sensenumber{4}\definition{Papuan black snakeAp me giddollag gullem a ako ddobae lla kuddäll e ddäddägang dan. Bätbät a mamam sära peyang dan. Ttägäl ddumuddumu me giddollag dan. (It lives in the grassland and it bites people to death. Its body is black and red. It lives in termite mounds.)}\example{Llɨg kälekäle, towall e ibnin a mudag, kuibiag a amom bäddägän.}{Children, do not walk in the grass; Papuan black snacks will bite anyone.}}
\entry{kuimang}{\headword{kuimang}\pos{A vt.}\sensenumber{4}\definition{to knock on}\example{Obo ma ud de kuimang dägagän.}{She knocked on his door.}}
\entry{Kuiwang}{\headword{Kuiwang}\pos{pn.}\sensenumber{4}\definition{Kuiwang (Taeme-speaking village in Morehead Rural LLG; from Limol, one must pass through Malam)}}
\entry{kuki}{\headword{kuki}\pos{mod.}\sensenumber{1}\definition{false, deceptive}\example{kuki ttoen}{trick}\example{kukiang keriso}{false messiah}\sensenumber{2}\definition{to deceive, trick, lie to}\example{Bogo gongnamän ada mälla da obom era kuki dägagän.}{He realized that the woman deceived him.}\example{Ddob kukiang lla da mɨnyi bälltaemnän.}{Other deceiving men will be coming.}\example{Ai dan God ngänäm märal bagän da kuki allan.}{I swear to God I'm not lying (lit. God can curse me if I'm lying).}\sensenumber{2}\definition{lie}\example{Ubi sapasapang kuki eka de däpanyaemeyo Yesu bo pallall.}{They told various lies about Jesus.}\subentry{\headword{kuki eka}\pos{n.}\definition{lie}}}
\entry{kukiny}{\headword{kukiny}\pos{n.}\sensenumber{1}\definition{type of long-leaf grass}\sensenumber{2}\definition{type of spear made from kukiny grass that is used to hunt birds}}
\entry{kukoll}{\headword{kukoll}\pos{mod.}\sensenumber{2}\definition{healthy, fertile, vibrant}\example{Kutt a kukollang ekaklle me guspullän a mermerae mängallang kukollang däpäddän.}{The seed fell to the fertile ground and sprouted vibrantly.}\sensenumber{2}\definition{green}\subentry{\headword{kukollkukoll}\pos{col.}\definition{green}}}
\entry{kukpi}{\headword{kukpi}\pos{n.}\sensenumber{2}\definition{type of tree that grows in the bush with big fruit that children play with}}
\entry{Kukpikukpi}{\headword{Kukpikukpi}\pos{pn.}\sensenumber{2}\definition{Kukpikukpi (toponym)}}
\entry{Kuks}{\headword{Kuks}\pos{pn.}\sensenumber{2}\definition{male personal name}}
\entry{kuku}{\headword{kuku}\pos{n.}\sensenumber{2}\definition{type of grass}}
\entry{Kukua}{\headword{Kukua}\pos{pn.}\sensenumber{2}\definition{male personal name}}
\entry{kukull}{\headword{kukull}\variant{fast speech var. of}{kullkull}}
\entry{Kukumi}{\headword{Kukumi}\pos{pn.}\sensenumber{2}\definition{male personal name}}
\entry{kul}{\headword{kul}\pos{A vt.}\sensenumber{2}\definition{to smash food}\example{Pamker a mätta de kul bägayaebneyo.}{They will be smashing the yams and pumpkins together.}}
\entry{kulläb}{\headword{kulläb}\pos{n.}\sensenumber{2}\definition{large black termite mound}}
\entry{Kulläntti}{\headword{Kulläntti}\variant{var. of}{Kurunti}}
\entry{Kullintti}{\headword{Kullintti}\variant{var. of}{Kurunti}}
\entry{kullkull}{\headword{kullkull}\pos{n.}\sensenumber{2}\definition{grassfire; burnt grass}\example{Kullkull a dallän llamda pate.}{The grassfire moved towards the old man.}\example{Ngäna deyarne kullkull att dae.}{I was returning to the burnt grass.}\sensenumber{2}\definition{burnt (of an area)}\example{Ttongo däräng a kullkullat enddäna me pollpoll dängkamän.}{A dog started the bark in the burnt-up clearing.}\etymology{kullkull + =att}\subentry{\headword{kullkullatt}\pos{mod.}\definition{burnt (of an area)}}}
\entry{Kullme}{\headword{Kullme}\pos{pn.}\sensenumber{2}\definition{Kullme (garden place near Egapo; filled with abandoned rubber trees)}}
\entry{Kullopang}{\headword{Kullopang}\pos{pn.}\sensenumber{2}\definition{Kullopang (sago and garden place of Kaoga Dobola; on the shortcut road to Kinkin)}}
\entry{kullum}{\headword{kullum}\pos{n.}\sensenumber{2}\definition{group}\example{Da ttongo kantri me lla da bompallängkmenynegän obaoba komlla kullum e a bogäddeyo obaoba, däbe kantri da mɨnyi ddone bakällmewän.}{If in a country, the people divide themselves into two groups and fight each other, that country will not survive.}}
\entry{Kullwam}{\headword{Kullwam}\pos{pn.}\sensenumber{2}\definition{male personal name}}
\entry{kum}{\headword{kum}\pos{n.}\sensenumber{1}\definition{buttocks, butt}\sensenumber{2}\definition{abdomen of an insect}\sensenumber{3}\definition{butt, base, bottom, lower or back part of something}\example{Mätta bo kum a angttägan matu ik dae.}{The base of the yam comes up through the ground.}\sensenumber{3}\definition{stinger (of an insect)}\sensenumber{3}\definition{back}\example{Bogo kumattkumatt gongosän.}{He turned and went back.}\etymology{redup. of kum + =att}\subentry{\headword{kumpit}\pos{n.}\definition{stinger (of an insect)}}\subentry{\headword{kumattkumatt}\pos{adv.}\definition{back}}}
\entry{kumddäga}{\headword{kumddäga}\variant{fast speech var. of}{kumuddäga}}
\entry{kumi}{\headword{kumi}\pos{n.}\sensenumber{3}\definition{central roof beam}\sensenumber{3}\definition{bark on top of roof}\subentry{\headword{kumi kakälläp}\pos{n.}\definition{bark on top of roof}}}
\entry{kumie}{\headword{kumie}\variant{sp. var. of}{kumiye}}
\entry{kumiye}{\headword{kumiye}\pos{n.}\sensenumber{3}\definition{cough}\example{Llamda da kumiye agan.}{The old man coughed.}\sensenumber{1}\definition{phlegm}\sensenumber{2}\definition{part of a fishUme kollkäll. (The gills of the mouth.)}\sensenumber{3}\definition{bark on roof ridge that prevents rain from coming in}\subentry{\headword{kumiye käp}\pos{n.}\definition{phlegm}}}
\entry{kumuddäga}{\headword{kumuddäga}\pos{num.}\sensenumber{3}\definition{three (yam counting numeral; also general)}\example{Ag mullaemullae kumuddäga täräp pauro eka watt me.}{Every morning, I wake up after the third rooster crow.}\sensenumber{3}\definition{third}\sensenumber{3}\definition{exactly three}\etymology{kumuddäga + =aebe}\subentry{\headword{kumuddäga we}\pos{ord. num.}\definition{third}}\subentry{\headword{kumuddägaebe}\pos{num.}\definition{exactly three}}}
\entry{kumuddägakumuddäga}{\headword{kumuddägakumuddäga}\pos{num.}\sensenumber{3}\definition{six (3+3)}}
\entry{Kumull}{\headword{Kumull}\pos{pn.}\sensenumber{3}\definition{Kumull (toponym)}}
\entry{kumye}{\headword{kumye}\variant{sp. var. of}{kumiye}}
\entry{Kuna}{\headword{Kuna}\pos{pn.}\sensenumber{3}\definition{male personal name}}
\entry{kunen}{\headword{kunen}\pos{S vi.}\sensenumber{3}\definition{to flee, scatter, run away}\example{Nikol angde walle menae e nallan, ddone ada kollba nukuaemallo.}{When Nikol went to the side of the river, many fish scattered away.}\allomorph{ku}}
\entry{kunob}{\headword{kunob}\pos{n.}\sensenumber{3}\definition{type of tree that grows around creeks with light wood used for house sticks}}
\entry{kunu}{\headword{kunu}\pos{n.}\sensenumber{3}\definition{type of short tree that grows in the grassland with poisonous bark for stunning fish}}
\entry{kunun}{\headword{kunun}\pos{n.}\sensenumber{3}\definition{season when crops are ready to be harvested (sixth season; corresponds to April)}}
\entry{kunur}{\headword{kunur}\pos{n.}\sensenumber{3}\definition{corn, maize}\example{Ttongo lla da kunur däm kutt de ekaklle me bibewän.}{A man will plant the corn kernels in the ground.}\sensenumber{3}\definition{corn kernel}\subentry{\headword{kunur käp}\pos{n.}\definition{corn kernel}}}
\entry{kunuwälläb}{\headword{kunuwälläb}\pos{n.}\sensenumber{3}\definition{type of big taro}}
\entry{kungge1}{\headword{kungge1}\pos{n.}\sensenumber{3}\definition{spiderKemibi ttäleang a obaene ddäddäg a ddobae ttällanenang dan. (It has many legs and its bite is very painful.)}\example{Da bam kungge da naddägän, bongo mɨnyi ttattlle ampug.}{If a spider bites you, you will be very ill.}}
\entry{kungge2}{\headword{kungge2}\pos{n.}\sensenumber{3}\definition{type of tree}}
\entry{kunglle}{\headword{kunglle}\pos{ideo.}\sensenumber{3}\definition{sound made by flying foxes}}
\entry{Kunyemäll}{\headword{Kunyemäll}\pos{pn.}\sensenumber{3}\definition{Kunyemäll (on the road to Malam near Zarma; filled with black palms that were cut for the school)}}
\entry{kup}{\headword{kup}\pos{n.}\sensenumber{1}\definition{hole, pit}\example{Ngäna ngämo män kälsäre de umllang dägag, ada ibi sisri mɨnyi kup di bäklleya.}{I told my little girl that we are going to dig a hole now.}\sensenumber{2}\definition{well}\example{Ine da bodo dan kup mi.}{The well is full of water.}\sensenumber{3}\definition{valley}\example{Obo ma duli dan kup mi.}{His house is there in the valley.}\sensenumber{3}\definition{small hole, pothole}\subentry{\headword{kupkup}\pos{n.}\definition{small hole, pothole}}}
\entry{kupi käp}{\headword{kupi käp}\pos{n.}\sensenumber{3}\definition{type of fruit (\textbackslashtextasciitilde4 cm in diameter) that children shoot at birds}}
\entry{kupoll}{\headword{kupoll}\variant{var. of}{kupull}}
\entry{kupull}{\headword{kupull}\pos{n.}\sensenumber{3}\definition{earthworm, worm}}
\entry{Kur}{\headword{Kur}\pos{pn.}\sensenumber{3}\definition{Kur (Wipi-speaking village in Oriomo-Bituri Rural LLG; on the road to Oriomo)}}
\entry{kuram}{\headword{kuram}\pos{n.}\sensenumber{3}\definition{type of long yam with a white interior, hairs, and thorns}}
\entry{kurikuri}{\headword{kurikuri}\pos{n.}\sensenumber{3}\definition{game involving spinning tree fruit on one's hand}}
\entry{Kurinti}{\headword{Kurinti}\variant{var. of}{Kurunti}}
\entry{Kurintti}{\headword{Kurintti}\variant{var. of}{Kurunti}}
\entry{kurkur}{\headword{kurkur}\pos{n.}\sensenumber{3}\definition{type of birdLlo toko me pentaeang pa bätbät dan. (It's a black bird that gathers on treetops.)}}
\entry{kurmirang}{\headword{kurmirang}\pos{n.}\sensenumber{3}\definition{type of spear}}
\entry{Kurunti}{\headword{Kurunti}\pos{pn.}\sensenumber{3}\definition{Kurunti (Em-speaking village in Oriomo-Bituri Rural LLG; from Limol, one must pass through Malam)}}
\entry{Kurupel}{\headword{Kurupel}\pos{pn.}\sensenumber{3}\definition{male personal name}}
\entry{kus}{\headword{kus}\pos{n.}\sensenumber{3}\definition{type of birdKollba gäzag pa dan, kämbägag alle ine ik me ibiag dan. (It's a bird that kills fish; it goes in the water by diving.)}}
\entry{kusi}{\headword{kusi}\pos{post.}\sensenumber{3}\definition{through}}
\entry{kut}{\headword{kut}\pos{n.}\sensenumber{3}\definition{raincloud}\example{Kut a ddone bogon.}{They aren't rainclouds.}}
\entry{kutae}{\headword{kutae}\pos{n.}\sensenumber{3}\definition{type of small yam}}
\entry{kutkut}{\headword{kutkut}\pos{n.}\sensenumber{3}\definition{name of clan}}
\entry{Kutpi Käp}{\headword{Kutpi Käp}\pos{pn.}\sensenumber{3}\definition{Kutpi Käp (toponym)}}
\entry{kutt}{\headword{kutt}\pos{n.}\sensenumber{1}\definition{bone}\example{ddäg kutt}{backbone}\example{Lepresi da kutt ngämnanang itrell dan.}{Leprosy is a disease that can affect the bones.}\sensenumber{2}\definition{seed, core}\example{Yokon kutt di dibewän.}{Yokon plants the seeds.}\sensenumber{3}\definition{vague hard thing}\example{ine kutt, pane kutt}{water bucket, pan}\sensenumber{3}\definition{thin, bony}\sensenumber{3}\definition{to shiver}\example{Ikopse agaebne adawede ge ttoen a ddone kutt tataemang kallkällang abal täräp me bongesän.}{[You all] pray that this thing doesn't happen at a shiveringly cold time.}\etymology{redup. of kutt + =ang}\subentry{\headword{kuttakuttang}\pos{mod.}\definition{thin, bony}}\subentry{\headword{kutt tataem}\pos{A vi.}\definition{to shiver}}}
\entry{kutt gugu}{\headword{kutt gugu}\pos{n.}\sensenumber{3}\definition{type of peace restorationDa bongo ngämo lla de kuddäll e näbadd, bongo ngämlle mɨnyi kutt gugu män de nanttog mu we. (If you kill one of my people, you will give me a girl as payment.)}}
\entry{kutt llo}{\headword{kutt llo}\pos{n.}\sensenumber{3}\definition{type of tree that grows in the bush with bark that is scraped and rubbed on soresTubu däl me nyäny ma dan a nane ma dan lla bo igi me tokop täräp me. (To rub on sore knees and to drink when the underside is bumpy.)}}
\entry{kutt snameny}{\headword{kutt snameny}\pos{S vt.}\sensenumber{3}\definition{a funeral tradition, where you take the body to an isolated place and one person will stay with the body and a spirit will come to tell them how they died.}\allomorph{nsnemeny}\allomorph{sänameny}\allomorph{nsänemeny}}
\entry{kuu}{\headword{kuu}\variant{var. of}{ku1}}
\entry{Kuyu}{\headword{Kuyu}\pos{pn.}\sensenumber{3}\definition{garden place of Matthew Bodog and Kaoga Dobola in Limol}}
\entry{kuyu}{\headword{kuyu}\pos{n.}\sensenumber{3}\definition{type of tree that grows in the bush or along creeks with soft wood that rots easily}}
\entry{kwa}{\headword{kwa}\pos{ideo.}\sensenumber{3}\definition{sound made by a dog in pain}}
\entry{kwae}{\headword{kwae}\pos{n.}\sensenumber{3}\definition{type of yam with a white or purple interior}}
\entry{kwaena kutt}{\headword{kwaena kutt}\pos{n.}\sensenumber{3}\definition{hip bone}}
\entry{kwakall}{\headword{kwakall}\pos{n.}\sensenumber{3}\definition{type of tree}}
\entry{kwakasru}{\headword{kwakasru}\pos{n.}\sensenumber{3}\definition{type of snakeDdägnang ma gullem dan. Bätbät dan. (It's an edible snake. It's black.)}}
\entry{Kwakmae}{\headword{Kwakmae}\pos{pn.}\sensenumber{3}\definition{female personal name}}
\entry{Kwalde}{\headword{Kwalde}\pos{pn.}\sensenumber{3}\definition{male personal name}}
\entry{Kwale}{\headword{Kwale}\pos{pn.}\sensenumber{3}\definition{female personal name}}
\entry{kwallang}{\headword{kwallang}\pos{n.}\sensenumber{3}\definition{type of bush used for posts}}
\entry{Kwallangkäbäll}{\headword{Kwallangkäbäll}\pos{pn.}\sensenumber{3}\definition{community garden place in Limol}}
\entry{kwantta}{\headword{kwantta}\pos{n.}\sensenumber{3}\definition{type of tree that grows in the grassland (\textbackslashtextasciitilde9 m) with green and yellow leaves used as bow pigment and hard fruit used to play a hockey-like game; also used for posts; liquid is extracted from the bark and given to dogs when they show signs of sickness}}
\entry{Kwangka}{\headword{Kwangka}\sensenumber{3}\definition{Kwangka (toponym)}}
\entry{kwangka}{\headword{kwangka}\pos{n.}\sensenumber{3}\definition{Torresian crowTot ttonenang pa bätbät dan. (It's a black bird that gathers trash.)}}
\entry{kwangka lläkäm}{\headword{kwangka lläkäm}\pos{n.}\sensenumber{3}\definition{type of inedible mushroom}}
\entry{Kwangkangatt}{\headword{Kwangkangatt}\pos{pn.}\sensenumber{3}\definition{sacred place of Dobola (on the road to Malam)}}
\entry{Kwara}{\headword{Kwara}\pos{pn.}\sensenumber{3}\definition{female personal name}}
\entry{kwarakwara}{\headword{kwarakwara}\pos{n.}\sensenumber{3}\definition{eastern hooded pittaKämag me ekawang pa dan wälläng me. (It's a bird in the bush that sings during storms.)}}
\entry{kwaratang}{\headword{kwaratang}\pos{mod.}\sensenumber{3}\definition{thin, skinny, slim (animate)}\example{Ge däräng a kwaratang abal dan.}{This dog is very thin.}}
\entry{kwas}{\headword{kwas}\pos{n.}\sensenumber{3}\definition{type of taro kongkong}}
\entry{kwata}{\headword{kwata}\pos{n.}\sensenumber{3}\definition{type of tree that grows in the bush; used for house sticks}}
\entry{kwatt}{\headword{kwatt}\pos{n.}\sensenumber{3}\definition{court}\example{Angde ubi bibim kwatt e yatraemneyo, mudag näkäp ttomoenen a.}{When they take you all to court, do not worry.}\example{Yesu bom dämllaeyo a diba imnealle Paelet pate dätrameyo kwatt e.}{They tied up Jesus and after that, they took him to Pilate in court.}\etymology{from Englishcourt}}
\entry{kwäp}{\headword{kwäp}\variant{var. of}{kup}}
\entry{kwätäs}{\headword{kwätäs}\pos{n.}\sensenumber{3}\definition{type of tree}}
\entry{Kwe}{\headword{Kwe}\pos{pn.}\sensenumber{3}\definition{male personal name}}
\entry{kwib}{\headword{kwib}\pos{n.}\sensenumber{3}\definition{charcoal made from a particular tree called upiye, used for painting during dance and initiation ceremony}}
\entry{Kwinton}{\headword{Kwinton}\variant{sp. var. of}{Quinton}}
\lettersection{L}
\entry{labalaba}{\headword{labalaba}\pos{n.}\sensenumber{3}\definition{lap-lap}\example{Yäbäd angde ttänttämang abal gogon, däbe lla da obo labalaba de dängkänän ttänttäm atta.}{When the sun got very hot, that man took off his lap-lap because of the heat.}\etymology{ultimately from a Polynesian language; compare Samoan lavalava}}
\entry{labelabet}{\headword{labelabet}\pos{n.}\sensenumber{3}\definition{type of game involving planting a stick in the middle of a circle}}
\entry{laem}{\headword{laem}\pos{S vt.}\sensenumber{1}\definition{to roll, wrap}\example{laemnenatt otät}{rolled food, wrap}\example{Bogo ttäkäll de dälemnegän.}{She rolled up balls of yam and coconut.}\sensenumber{2}\definition{to argue}\allomorph{lem}}
\entry{laen}{\headword{laen}\pos{n.}\sensenumber{2}\definition{line}\etymology{from Englishline}}
\entry{Lama}{\headword{Lama}\pos{pn.}\sensenumber{2}\definition{female personal name}}
\entry{Lamlam}{\headword{Lamlam}\pos{pn.}\sensenumber{2}\definition{name of a female ancestor (sister of Moli)}}
\entry{Lauren}{\headword{Lauren}\pos{pn.}\sensenumber{2}\definition{female personal name}}
\entry{läläm}{\headword{läläm}\pos{n.}\sensenumber{2}\definition{muddy spot}}
\entry{läpon}{\headword{läpon}\pos{S vi.}\sensenumber{2}\definition{to be amazed, be in awe}\example{Ubira tämamae lla yaba läponang gogon.}{All of them were amazed.}\example{Ubi daläponeyo.}{They were amazed.}}
\entry{Lei}{\headword{Lei}\pos{pn.}\sensenumber{2}\definition{Lei (toponym)}}
\entry{Lek Märi}{\headword{Lek Märi}\pos{pn.}\sensenumber{2}\definition{Lake Murray (in Lake Murray Rural LLG)}\etymology{from EnglishLake Murray}}
\entry{lel}{\headword{lel}\pos{n.}\sensenumber{1}\definition{fear}\example{Ubi dindugeyo lel me.}{They ran in fear.}\sensenumber{2}\definition{to fear, be afraid of}\example{Lla da sɨmell de lel dägagän.}{The man feared the pig.}\example{Ge kungge da bäne lel ma za dan.}{This spider is what you fear.}\sensenumber{3}\definition{shame}\sensenumber{4}\definition{to be ashamed (of)}\example{Bogo lel allan ngämo mit me.}{She is ashamed because of me.}\example{Baba da mɨnyi obom lel bägagän.}{Father will be ashamed of him.}\sensenumber{1}\definition{afraid, scared}\example{Bogo deya lelang llo toko e gogäbänän.}{He was so scared he jumped into the treetop.}\sensenumber{2}\definition{shameful}\example{Däbe mɨnyi ddobae lelang abal ttoen da bogalle obo kuki tab e.}{If he broke the promise, that would be a very shameful thing.}\sensenumber{2}\definition{brave, fearless, bold}\etymology{lel + =ang}\subentry{\headword{lelang}\pos{mod.}\definition{afraid, scared}}\subentry{\headword{lelmeny}\pos{mod.}\definition{brave, fearless, bold}}}
\entry{lepade}{\headword{lepade}\pos{n.}\sensenumber{2}\definition{type of cultivated tree with purple and white flowers and edible black fruit that kids like to eat}}
\entry{lepresi}{\headword{lepresi}\pos{n.}\sensenumber{2}\definition{leprosy}\etymology{from Englishleprosy}}
\entry{lesna}{\headword{lesna}\pos{n.}\sensenumber{2}\definition{type of tree}}
\entry{Letai}{\headword{Letai}\pos{pn.}\sensenumber{2}\definition{female personal name}}
\entry{Lewada}{\headword{Lewada}\pos{pn.}\sensenumber{2}\definition{Lewada (Makayam-speaking village in Gogodala Rural LLG, on the Fly River; GPS: 8.327787, 142.785487)}}
\entry{lid}{\headword{lid}\pos{n.}\sensenumber{2}\definition{lid}\etymology{from Englishlid}}
\entry{lida}{\headword{lida}\pos{n.}\sensenumber{2}\definition{leader}\etymology{from Englishleader}}
\entry{Lidiya}{\headword{Lidiya}\pos{pn.}\sensenumber{2}\definition{female personal name}}
\entry{liglig}{\headword{liglig}\pos{S vi.}\sensenumber{2}\definition{to have sex, copulate}\example{Bogo liglig eran.}{He is having sex.}\example{Ubi do me alignallo.}{They two had sex there.}\example{Ubi iddob me ddone bulignegnän.}{They will not be having sex at night.}\sensenumber{2}\definition{having a lot of sex}\sensenumber{2}\definition{sexual relations}\example{Bäne gullbe o mälla walle mae liglig ttoen de nangesnegneyo.}{Only have sexual relations with your husband or wife.}\allomorph{lig}\subentry{\headword{liglig suang}\pos{mod.}\definition{having a lot of sex}}\subentry{\headword{liglig ttoen}\pos{n.}\definition{sexual relations}}}
\entry{Lili}{\headword{Lili}\pos{pn.}\sensenumber{2}\definition{female personal name}}
\entry{Lilian}{\headword{Lilian}\pos{pn.}\sensenumber{2}\definition{female personal name}}
\entry{Lily}{\headword{Lily}\variant{sp. var. of}{Lili}}
\entry{Limoll}{\headword{Limoll}\pos{pn.}\sensenumber{2}\definition{Limol (Ende-speaking village in Morehead Rural LLG; GPS: -8.641783, 142.682533)}\sensenumber{1}\definition{Limol villager, person from Limol}\sensenumber{2}\definition{Ende dialect spoken in Limol}\etymology{Limoll + =ang}\subentry{\headword{Limollang}\pos{n.}\definition{Limol villager, person from Limol}}}
\entry{Linda}{\headword{Linda}\pos{pn.}\sensenumber{2}\definition{female personal name}}
\entry{Linette}{\headword{Linette}\pos{pn.}\sensenumber{2}\definition{female personal name}}
\entry{linge}{\headword{linge}\pos{n.}\sensenumber{2}\definition{type of palm tree with blue flowers Nyäng inen ma, pite inen ma za dan. (It's used to weave bags and grass skirts.)}}
\entry{Liseng}{\headword{Liseng}\pos{pn.}\sensenumber{2}\definition{PN}}
\entry{lizom}{\headword{lizom}\pos{n.}\sensenumber{2}\definition{type of mushroomWälläng me päddabag dan, ddäddäg ma dan. (It grows in the bush; it's edible.)}}
\entry{lɨklɨk}{\headword{lɨklɨk}\pos{S vi.}\sensenumber{2}\definition{to melt}\example{Ngämo ine kutt a daeya yu da lɨklɨk eran.}{Fire is melting my water container.}}
\entry{lo}{\headword{lo}\pos{n.}\sensenumber{2}\definition{law}\etymology{from Englishlaw}}
\entry{Lois}{\headword{Lois}\pos{pn.}\sensenumber{2}\definition{female personal name}}
\entry{lok}{\headword{lok}\pos{A vt.}\sensenumber{2}\definition{lock}\etymology{from Englishlock}}
\entry{loli}{\headword{loli}\pos{n.}\sensenumber{2}\definition{candy}\etymology{from Englishlolly}}
\entry{Lomae}{\headword{Lomae}\pos{pn.}\sensenumber{2}\definition{female personal name}}
\entry{lomenyang}{\headword{lomenyang}\pos{n.}\sensenumber{2}\definition{someone who complains, complainer}}
\entry{Loni}{\headword{Loni}\pos{pn.}\sensenumber{2}\definition{female personal name}}
\entry{lonsis}{\headword{lonsis}\pos{n.}\sensenumber{2}\definition{lemon plant}}
\entry{longgo}{\headword{longgo}\pos{n.}\sensenumber{2}\definition{noise}\example{Llɨg obaene longgo de dändär, gongällbän a.}{I heard the kids making noise and woke up.}}
\entry{Lovelyn}{\headword{Lovelyn}\pos{pn.}\sensenumber{2}\definition{female personal name}}
\entry{Ludwina}{\headword{Ludwina}\pos{pn.}\sensenumber{2}\definition{female personal name}}
\entry{Luke}{\headword{Luke}\pos{pn.}\sensenumber{2}\definition{male personal name}}
\entry{Lulu}{\headword{Lulu}\pos{pn.}\sensenumber{2}\definition{female personal name}}
\entry{lulu}{\headword{lulu}\pos{S vt.}\sensenumber{2}\definition{to gossip about, tell rumors about}\example{Luluang lla da ai dan gagäll ud de bätramän.}{One who gossips can open the door to bad things.}\example{Ngäna dandärmällnegne mälla de ami bam daluneyo.}{I heard the women who were gossiping about you.}\allomorph{lu}}
\entry{lus}{\headword{lus}\pos{A vi.}\sensenumber{2}\definition{lose}\etymology{from Englishlose}}
\entry{Lydia}{\headword{Lydia}\pos{pn.}\sensenumber{2}\definition{female personal name}}
\entry{Lynet}{\headword{Lynet}\variant{sp. var. of}{Linette}}
\entry{Lyneth}{\headword{Lyneth}\pos{pn.}\sensenumber{2}\definition{female personal name}}
\lettersection{Ll ll}
\entry{lla}{\headword{lla}\pos{n.}\sensenumber{1}\definition{man, male}\example{Lla pättäpättäk a Meri bom dangnoeyän llädäd e.}{The short man asked Mary to marry him.}\sensenumber{2}\definition{husband}\sensenumber{3}\definition{person, human being}\example{Obo pallall lla da ddone ada kilikili gognegnän a tuk i gogäbnän.}{The people next to him were so happy and were jumping in the air.}\sensenumber{3}\definition{generation}\example{Ewatta ke gänya täräp me lla bombllo da wasnen anggan ttowamang ttoen tonton ikop e ddapall att?}{Why does the current generation beg to directly see miraculous things from heaven?}\sensenumber{3}\definition{crowd}\example{Ddobag a bi dingismällalle täräb ma we, lla gulag a erame dag.}{The others return to the funeral home, where the crowd is.}\sensenumber{3}\definition{audience}\sensenumber{3}\definition{boy}\example{Zon lla llɨg dan.}{John is a boy.}\sensenumber{3}\definition{relative, kinsman, clansmanttongdae tän me lla bombllo}\example{Kemu bo llabun a gänyme a Malläm me dadeg.}{Kemu has relatives here and in Malam.}\sensenumber{3}\definition{other people's, of others}\example{Ende eka da ngäma eka dan, Ingglis a llama eka dan.}{Ende is our (excl.) language; English is the language of other people.}\example{Llama ttoen da ge, be ngämi zäme ngämira dägageya ge ttoen de.}{This is another people's custom, but we are already doing this custom.}\sensenumber{3}\definition{old manobo mängall a daden melem ngasnen e.}\example{llamäg a auseause}{old men and women}\sensenumber{3}\definition{old man}\example{Llamda da kumiye agan.}{The old man coughed.}\etymology{lla + bun}\subentry{\headword{lla bombllo}\pos{n.}\definition{generation}}\subentry{\headword{lla gul}\pos{n.}\definition{crowd}}\subentry{\headword{lla ikoikopang}\pos{n.}\definition{audience}}\subentry{\headword{lla llɨg}\pos{n.}\definition{boy}}\subentry{\headword{llabun}\pos{n.}\definition{relative, kinsman, clansmanttongdae tän me lla bombllo}}\subentry{\headword{llama1}\pos{mod.}\definition{other people's, of others}}\subentry{\headword{llamäg}\pos{n.}\definition{old manobo mängall a daden melem ngasnen e.}}\subentry{\headword{llamda}\pos{n.}\definition{old man}}}
\entry{lla diben}{\headword{lla diben}\pos{n.}\sensenumber{3}\definition{type of snakeDdäddäg ma gullem ulle dan. Bätbät dan. Ttoe a obo era iklloikllowang dan. Wälläng me giddolag da. (It's a big, edible snake. It's black. Its skin is smoke-colored.)}}
\entry{lla gugu}{\headword{lla gugu}\pos{n.}\sensenumber{3}\definition{restoring peace after a murder by trading a young girl in the victim's place}}
\entry{lla up}{\headword{lla up}\pos{n.}\sensenumber{3}\definition{type of introduced bananaTupi pänyanzag dan. Obo pätt a ulle dan a däg a obo yuwog dag käp peyang. O me käp a binzenen ma dag ako kire me yu ma dag. Ako pätt a obo popel e kaepnen ma da ankom peyang a pag a diba toko me mer mokowang e. (It grows long. Its trunk is big and its bunches have plentiful fruit. When ripe, its fruit is heated, and when unripe, it's cooked. Also, its stem is for chewing popel, with ants and salt on top to make it delicious.)}}
\entry{lla winy}{\headword{lla winy}\pos{n.}\sensenumber{3}\definition{type of biting bee that lives in trees}}
\entry{llakällakätt}{\headword{llakällakätt}\pos{n.}\sensenumber{3}\definition{type of tree that grows in the bush; used for house sticks}}
\entry{llakllek}{\headword{llakllek}\pos{S vt.}\sensenumber{3}\definition{to destroy}\example{Tätäm ttäle bun parga de bob a era dällekän.}{Yesterday the flood destroyed the Ttäle Bun bridge.}\allomorph{llek}\allomorph{llak}}
\entry{llama2}{\headword{llama2}\pos{mod.}\sensenumber{1}\definition{unsatisfied}\example{Kwalde säre llama näkäp me gongosmällnän.}{Kwalde sadly returned unsatisfied.}\sensenumber{2}\definition{hesitant, reluctant}\example{Däräng a llama näkäp me gongezänmällnän mama watta.}{The dog was coming out of the grass pile hesitantly.}}
\entry{llame}{\headword{llame}\pos{adv.}\sensenumber{2}\definition{together}\example{Ngämi llame deyareya mamoema.}{We (excl.) went hunting together.}\example{Ngämi llameae gobällne.}{We (excl.) were going together.}\example{Oba za sinen a llamealle mae.}{They shared their things together.}\etymology{lla + me}}
\entry{llan}{\headword{llan}\pos{n.}\sensenumber{2}\definition{ear}\example{Obo llan a ulle abal dan.}{His ear is very big.}\sensenumber{2}\definition{eardrum}\sensenumber{2}\definition{earwax}\sensenumber{2}\definition{gill}\sensenumber{2}\definition{ear hair}\sensenumber{2}\definition{operculum (gill cover)}\sensenumber{2}\definition{to listen}\example{Bongo mɨnyi llandräg ag?}{Are you going to listen?}\example{Abo llandräg ag Malläm att eka we.}{Expect some news from Malam.}\example{Daeya do llandär gognän.}{She was there listening.}\sensenumber{2}\definition{to turn one's ear, listen intently}\example{Ubi tätäm kaemne de llan donggoeyo.}{Yesterday, they listened intently to the bee.}\etymology{from llan + dändär}\subentry{\headword{llan ik}\pos{n.}\definition{eardrum}}\subentry{\headword{llan källa}\pos{n.}\definition{earwax}}\subentry{\headword{llan kollkoll}\pos{n.}\definition{gill}}\subentry{\headword{llan kom}\pos{n.}\definition{ear hair}}\subentry{\headword{llan mit}\pos{n.}\definition{operculum (gill cover)}}\subentry{\headword{llandräg}\pos{A vi.}\definition{to listen}}\subentry{\headword{llan gonggo}\pos{S vt.}\definition{to turn one's ear, listen intently}}}
\entry{llanded}{\headword{llanded}\pos{S vt.}\sensenumber{1}\definition{to clear}\example{Oblle nyongo de dallendneyo.}{They were clearing the road for her.}\sensenumber{2}\definition{to clarify, make clear, explain}\example{Bogo eka midd de dallendän.}{He clarified the meaning.}\allomorph{llandnen}\allomorph{llend}\allomorph{lland}}
\entry{llapu}{\headword{llapu}\pos{n.}\sensenumber{2}\definition{type of big tree that grows in the bush with red fruit and white flowers}}
\entry{llapuyurwe}{\headword{llapuyurwe}\pos{n.}\sensenumber{2}\definition{type of tree with seedless, edible red fruit}}
\entry{llatat1}{\headword{llatat1}\pos{S vt.}\sensenumber{2}\definition{to trace, track, follow the blood (of an animal to kill it)}\example{Gongosän ma we, däräng a wa kumuddäga lla da dokomän, ddia llatat e.}{He returned home, and took his dog and three men to track the deer.}\allomorph{llat}}
\entry{llatat2}{\headword{llatat2}\pos{S vt.}\sensenumber{2}\definition{to twist, wrap}\example{Gull a kab llatnenatt alle iatt kollba gäddnan ma dan.}{A net is a thing for catching fish, woven from twisted string.}\allomorph{llat}}
\entry{llatata}{\headword{llatata}\pos{n.}\sensenumber{2}\definition{Food (such as sago, ripe bananas, and coconut cream, or yams and coconut cream) wrapped in a woven cococnut leaf (with a banana leaf within it) and cooked in a mumu}}
\entry{llatet}{\headword{llatet}\variant{var. of}{llatat2}}
\entry{lläb}{\headword{lläb}\pos{S vi.}\sensenumber{2}\definition{to go under}\example{Kungge da llo ttam dädär ik i golläbän.}{The spider went under the dry tree leaves.}}
\entry{lläbe}{\headword{lläbe}\pos{n.}\sensenumber{2}\definition{nail, claw (of a finger or toe)}}
\entry{llädae}{\headword{llädae}\pos{S vt.}\sensenumber{2}\definition{to roll}\example{Llɨg a bal de llädaenen eran.}{The boy is rolling the ball.}\example{Ubi dädär ulle de dalldaeyo auma ud de däpakeyo.}{They rolled a large rock and blocked the grave entrance.}\allomorph{lldae}}
\entry{llädäd}{\headword{llädäd}\pos{S vt.}\sensenumber{1}\definition{to grab, get, catch}\example{Ngäna walle we gäbän allan ddia llädäd e.}{I am jumping in the water to grab the deer.}\example{Ubi guspullän, be ubi obaoba gollädnegän.}{They fell, but they caught themselves.}\example{Iba pätt a ttongdae dan, ako kame ddone bällädeya.}{We (incl.) only have one body; we don't get another.}\sensenumber{2}\definition{to buy}\example{Da ttongo lla da bäne baba bälle ten taosen mani bonttogän, ende bogo bällädän?}{If someone gave your dad ten thousand kina, what would he buy?}\example{Tu andred alle dällädeyo ttoe de.}{He bought the skin for two hundred.}\sensenumber{3}\definition{to marry}\example{E mälla de ke bongo dälläd?}{Which woman did you marry?}\example{Da bongo lla tupi di nälläd, bongo mɨnyi llɨg kamebiag ag.}{If you marry the tall man, you will have many children.}\sensenumber{4}\definition{to come, arrive (figuratively)}\example{Angde obo täräp a ngäs e dällädalle, bogo mɨnyi do alle olle gogalle oba pate.}{When his time to return came, he would call out to them.}\allomorph{lläd}\allomorph{lld}\allomorph{llädnan}\allomorph{llädänan}\allomorph{llɨd}\allomorph{llädäd}\allomorph{lländ}}
\entry{lläg}{\headword{lläg}\variant{sp. var. of}{llɨg}}
\entry{lläkäm}{\headword{lläkäm}\pos{n.}\sensenumber{4}\definition{mushroom}\example{Lläkäm a mamall zozoang dan.}{A mushroom goes rotten quickly.}}
\entry{lläklläk}{\headword{lläklläk}\pos{S vt.}\sensenumber{4}\definition{to spread fire}\example{Sali yu di mermerae lläklläk eran.}{Sali is spreading the fire nicely.}\allomorph{lläk}}
\entry{llällam}{\headword{llällam}\pos{ideo.}\sensenumber{4}\definition{rustling (of plants); roaring (of water)}\example{Gullem a sisri ag me towall llällallällamang yaran.}{This morning the snake came rustling the grass.}\example{Däbaballe ine llällam de dandärän.}{From there, he heard the water roaring.}\sensenumber{4}\definition{noisy}\etymology{llällam + =ang}\subentry{\headword{llällamang}\pos{mod.}\definition{noisy}}}
\entry{llälläp}{\headword{llälläp}\pos{n.}\sensenumber{4}\definition{type of small snake that catches frogs}}
\entry{lläntäg}{\headword{lläntäg}\pos{S vt.}\sensenumber{4}\definition{to tell}\example{Golläntmenyän tämamae ngalen de.}{He told him everything.}\allomorph{llänt}\allomorph{lläntminy}\allomorph{lläntmeny}\allomorph{lläntmi}}
\entry{lläng}{\headword{lläng}\pos{mod.}\sensenumber{4}\definition{sharp}\example{Ttam a ddobae lläng dag.}{The leaves are very sharp.}\sensenumber{4}\definition{dull, blunt}\example{Gudne giri da llängmeny dan.}{The old knife is dull.}\etymology{lläng + =meny}\subentry{\headword{llängmeny}\pos{mod.}\definition{dull, blunt}}}
\entry{lläpän}{\headword{lläpän}\pos{S vt.}\sensenumber{4}\definition{to dig, harvest, unearth (a tuber or corm)}\example{Mälla da dagaeya ttängäm e abällan mätta lläplläp e.}{The women went to the garden to dig the yams.}\example{Bogo biye kemibi dälläpnegalle.}{He used to harvest many taro roots.}\allomorph{lläp}}
\entry{lläpät}{\headword{lläpät}\pos{n.}\sensenumber{4}\definition{digit}\example{ttang lläpät, nying lläpät}{finger, toe}}
\entry{llät}{\headword{llät}\variant{fast speech var. of}{llɨtɨt1}}
\entry{llätät1}{\headword{llätät1}\variant{dial. var. of}{llɨtɨt1}}
\entry{llätät2}{\headword{llätät2}\pos{S vt.}\sensenumber{4}\definition{to get rid of oven stones}\example{Llɨg kälakäle, ttägäll käp de nälläteyoǃ}{Small children, get rid of these mumu stones!}}
\entry{llätmäll}{\headword{llätmäll}\pos{S vt.}\sensenumber{4}\definition{to choose, select}\example{Bogo tuwelb lla de dällätmällnegän.}{He chose twelve men.}}
\entry{llätt}{\headword{llätt}\pos{n.}\sensenumber{1}\definition{end}\example{ekaklle llätt e}{to the ends of the earth}\example{pazi llätt me}{at the end of the year}\example{Ge ttoen a llätt e gongttägän.}{This story is coming to an end.}\sensenumber{2}\definition{to stop, end, finish}\example{Oba abo dindu da dädme llätt gogon.}{Their race stopped there.}\example{Oba nag ttoen a llätt gogon.}{Their friendship ended.}\example{Llɨtt agmom!}{(You all) stop!}\sensenumber{2}\definition{nonstop}\example{Ngäna dinduangmeae gogne llättmenyae.}{I was running nonstop.}\etymology{llätt + =meny}\subentry{\headword{llättmeny}\pos{adv.}\definition{nonstop}}}
\entry{llimba}{\headword{llimba}\pos{n.}\sensenumber{2}\definition{fingernail}\example{Ngämo llimba da agällbänan.}{My fingernail broke.}}
\entry{Llimoll}{\headword{Llimoll}\variant{dial. var. of}{Limoll}}
\entry{llɨg}{\headword{llɨg}\pos{n.}\sensenumber{1}\definition{boy}\example{llɨg mänmän}{boys and girls}\sensenumber{2}\definition{son}\example{Godd bo Llɨg indrang da.}{The Son of God is light.}\sensenumber{3}\definition{child}\example{Bäne llɨg bo bin a ainin?}{What is your child's name?}\sensenumber{1}\definition{womb, uterus}\sensenumber{2}\definition{placenta}\sensenumber{2}\definition{infant}\example{Ddäma da llɨg mapät komnen ma dan.}{A basket is for carrying infants.}\sensenumber{2}\definition{childless}\example{Ngäna llɨgmeny dan.}{I don't have children. / I'm childless.}\sensenumber{2}\definition{nonsingular form of llɨg}\example{Mäkat llɨgllɨgag a dan bäne gaguma me.}{There is a rat with babies in your yamhouse.}\etymology{llɨg + =meny}\subentry{\headword{llɨg dum}\pos{n.}\definition{womb, uterus}}\subentry{\headword{llɨg mapät}\pos{n.}\definition{infant}}\subentry{\headword{llɨgmeny}\pos{mod.}\definition{childless}}\subentry{\headword{llɨgllɨg}\pos{n.}\definition{nonsingular form of llɨg}}}
\entry{llɨkɨm}{\headword{llɨkɨm}\variant{sp. var. of}{lläkäm}}
\entry{llɨng}{\headword{llɨng}\variant{var. of}{lläng}}
\entry{llɨpɨt}{\headword{llɨpɨt}\variant{dial. var. of}{lläpät}}
\entry{llɨtɨt1}{\headword{llɨtɨt1}\pos{S vt.}\sensenumber{1}\definition{to tell, report, say}\example{Bogo eka de llɨtɨtang meae nägnan.}{He told the same story repeatedly.}\example{Män a llɨg bälle tongoeang ttoen di dɨllɨtän.}{The girl told the boy a funny story.}\example{Ubi ddone amlle ende golläntmenynegän adawatta ddobaeddobae lel gognegnän.}{They didn't say anything to anyone because they were very afraid.}\sensenumber{2}\definition{to sing}\example{Wagiba bandra de nɨllɨtnan sisri ag me.}{Wagiba was singing this morning.}\allomorph{llɨt}\allomorph{llät}\allomorph{llänt}\allomorph{llɨnt}}
\entry{llɨtɨt2}{\headword{llɨtɨt2}\pos{S vt.}\sensenumber{2}\definition{to butcher, cut}\example{Ddia ddäg de nällɨt.}{[You] cut the deer back.}\allomorph{llɨt}\allomorph{llät}\allomorph{llänt}}
\entry{llɨtt1}{\headword{llɨtt1}\pos{S vi.}\sensenumber{2}\definition{to get worse, worsen}\example{Lla da angde itrel a dägagalle, itrel atta bollɨttnän bollɨttnän angde ako kuddäll a damllamalle, kuddäll gogalle.}{When a man gets sick, and he gets worse and worse, then death catches him and he dies.}}
\entry{llɨtt2}{\headword{llɨtt2}\variant{sp. var. of}{llätt}}
\entry{llo}{\headword{llo}\pos{n.}\sensenumber{1}\definition{tree}\example{Ngäna llo de plengg eran.}{I am chopping down the tree.}\example{Tätäm Matthew llo de dängkälän.}{Yesterday, Matthew climbed the tree.}\sensenumber{2}\definition{wood}\example{llo pallkoll tubu}{short piece of wood}\example{Melemang dag ubi, llo ma abällaban.}{They are working; they went for wood.}\sensenumber{3}\definition{stick}\example{Lla da llo alle ekaklle me dadräbän.}{The man wrote on the ground with a stick.}\sensenumber{3}\definition{wooden stick}\example{Ngäna ttongo llo batt e dängällbän.}{I got one stick.}\sensenumber{3}\definition{tree branch}\example{Ngäna llo ddage me ingongang dan.}{I am dancing on the tree branch.}\sensenumber{3}\definition{forked tree branch}\sensenumber{3}\definition{fallen tree}\example{Ngäna llo patt toko me godmen.}{I sat on the fallen tree.}\sensenumber{3}\definition{tree trunk}\sensenumber{3}\definition{tree stump}\sensenumber{3}\definition{tree leaf}\example{Molemoleg a llo ttam otnanang dag.}{Caterpillars eat tree leaves.}\sensenumber{3}\definition{tree bark}\subentry{\headword{llo batt}\pos{n.}\definition{wooden stick}}\subentry{\headword{llo ddage}\pos{n.}\definition{tree branch}}\subentry{\headword{llo ngoeang}\pos{n.}\definition{forked tree branch}}\subentry{\headword{llo patt}\pos{n.}\definition{fallen tree}}\subentry{\headword{llo pätt}\pos{n.}\definition{tree trunk}}\subentry{\headword{llo tubu}\pos{n.}\definition{tree stump}}\subentry{\headword{llo ttam}\pos{n.}\definition{tree leaf}}\subentry{\headword{llo ttoe}\pos{n.}\definition{tree bark}}}
\entry{llokott}{\headword{llokott}\pos{mod.}\sensenumber{1}\definition{hard, firm}\example{Ekaklle da llokttang dan.}{The ground is hard.}\example{Däbe lla da obo labalaba de llokttangae damllamän.}{That man held his lap-lap firmly.}\sensenumber{2}\definition{difficult, hard}\example{Oba Balimo me ttam giddoll a ddobae llokott daeya.}{Their life in Balimo was very difficult.}\sensenumber{3}\definition{strong}\example{Imomdae lla da llokottang a ainin, ibi wi do ma.}{He who is truly strong is going to the house.}\sensenumber{1}\definition{hard}\example{mamlla llokollokott}{hard rope}\sensenumber{2}\definition{difficult}\example{bo ttam llokollokott}{her difficult life}\sensenumber{3}\definition{stubborn}\example{Oba tikop a llokollokott dagaeya.}{Their hearts were stubborn.}\sensenumber{3}\definition{firmly, tightly, strongly}\example{Ngäna sɨmell de llokollokott dinam.}{I pressed the pig down tightly.}\allomorph{lloktt}\subentry{\headword{llokollokott1}\pos{mod.}\definition{hard}}\subentry{\headword{llokollokott2}\pos{adv.}\definition{firmly, tightly, strongly}}}
\entry{llollom}{\headword{llollom}\pos{S vi.}\sensenumber{1}\definition{to break, be damaged}\example{Ekaklle we enanae gollomän kälakälae.}{It hit the ground, smashing into pieces.}\sensenumber{2}\definition{to break}\example{Bogo karpo inkätt de dullomän.}{She broke the jar seal.}\allomorph{llom}}
\entry{llowam}{\headword{llowam}\pos{n.}\sensenumber{1}\definition{fatigue, tiredness}\example{Mark bom llowam da dägagän.}{Mark was tired (lit. fatigue got Mark).}\sensenumber{2}\definition{disdain, hate}\example{Ngänäm llowam da nallan sana wätät a.}{I don't want to eat sago (lit. disdain to eat sago gets me).}\sensenumber{1}\definition{tired}\example{Ngämo pätt a llowamang gogon.}{My body got tired.}\sensenumber{2}\definition{unpreferable, unpleasant, tiresome (to someone in the dative)}\example{Ddob llayabira sosoga sana kängkäm a ddobae llowamang dan.}{Squeezing sosoga sago is really unpleasant to some people.}\example{Ngämlle llowamang dan gänya llɨg bo peyang eka tameny a, be de maigag de aya näbäddan.}{I don't want to talk with this boy (lit. talking with this boy is unpreferable to me), but with the one who killed the bandicoot.}\sensenumber{3}\definition{upset, annoyed}\example{Meri dallän Yokon pate ako llowamang dägagän obom.}{Mary went to Yokon and then got upset at her.}\etymology{llowam + =ang}\subentry{\headword{llowamang}\pos{mod.}\definition{tired}}}
\entry{llowawi}{\headword{llowawi}\pos{n.}\sensenumber{3}\definition{type of tree}}
\entry{llune}{\headword{llune}\pos{mod.}\sensenumber{3}\definition{wild}\example{Llune ddäddäg a dadegaeya.}{There were wild animals.}}
\entry{llupi}{\headword{llupi}\pos{n.}\sensenumber{3}\definition{branch}\example{Mälla da llupi de mɨnyi bäträpnegnän giri alle.}{The woman will cut the branches with a knife.}}
\entry{llupi ttäganen ttägnen}{\headword{llupi ttäganen ttägnen}\pos{n.}\sensenumber{3}\definition{type of game involving hiding a tree branch in the water}}
\entry{lluwam}{\headword{lluwam}\pos{S vt.}\sensenumber{3}\definition{to incapacitate, make unable}\example{Ngänäm wän a nying me nalluwaman Malläm e ibi we.}{This boil on my foot has made me unable to walk to Malam.}}
\lettersection{M}
\entry{ma}{\headword{ma}\pos{n.}\sensenumber{1}\definition{house, home}\example{Ubi dadeg ma me?}{Are they home?}\example{Abo bongo ttongo ma sisor de nogo.}{You must build a new house.}\sensenumber{2}\definition{place, location}\example{kwatt ma, ine ma, Ende ma}{courthouse, well, Ende-speaking place}\sensenumber{3}\definition{community}\example{Bin lla da mer de ngasnen erallo ma makäp me.}{Law enforcers do good in the community.}\sensenumber{1}\definition{clan totem (the sacred symbol of a clan group; usually a plant, animal, or body part)}\example{Obo mabun a dirom.}{Her totem is cassowary.}\sensenumber{2}\definition{sacred}\example{mabun ttoen}{sacred story}\sensenumber{2}\definition{space under a house}\example{Ngämi makäm me gägaebllaeb eran.}{We (excl.) are standing underneath the house.}\etymology{ma + bun}\subentry{\headword{mabun}\pos{n.}\definition{clan totem (the sacred symbol of a clan group; usually a plant, animal, or body part)}}\subentry{\headword{makäm}\pos{n.}\definition{space under a house}}}
\entry{=ma}{\headword{=ma}\pos{cl.}\sensenumber{1}\definition{characteristic case clitic (indicates purpose, source, or reason)}\example{ddäddäg, ddäddäg ma}{eat, edible}\example{ikopse, ikopse ma ma}{prayer, house for prayer}\example{Widere da gall gllae ma ttoen dan.}{An oar is a thing to paddle a canoe.}\sensenumber{2}\definition{nominalizer}\example{kaptte ittall, kaptte ittall ma}{hanging clothes, clothesline}\etymology{from ma}}
\entry{ma kisin}{\headword{ma kisin}\pos{n.}\sensenumber{2}\definition{traditional house built directly on the ground}\etymology{lit. 'kitchen house'}}
\entry{ma ttängäm}{\headword{ma ttängäm}\pos{n.}\sensenumber{2}\definition{type of tree}}
\entry{maa}{\headword{maa}\variant{sp. var. of}{ma}}
\entry{maaduma}{\headword{maaduma}\variant{sp. var. of}{maduma}}
\entry{mab}{\headword{mab}\pos{n.}\sensenumber{2}\definition{pandanusKäp tupiang dan, o me mɨnyi mamam bogän, wätät ma dan. (With long fruit, it will be red when ripe; it's edible.)}\example{Bogo tätäm mab de däddänän.}{Yesterday, he picked pandanus.}}
\entry{Mabudawan}{\headword{Mabudawan}\pos{pn.}\sensenumber{2}\definition{Mabudawan/Mabaduan (Agob-speaking village in Kiwai Rural LLG near Saibai Island)}}
\entry{madä}{\headword{madä}\variant{dial. var. of}{mäda}}
\entry{madik}{\headword{madik}\pos{n.}\sensenumber{2}\definition{type of toolTtängäm melem ma dan. Llo ollondd ddälläbnen ma dan. (It's for garden work. It's for breaking tree roots.)}}
\entry{Madima}{\headword{Madima}\pos{pn.}\sensenumber{2}\definition{female personal name}}
\entry{Madlin}{\headword{Madlin}\pos{pn.}\sensenumber{2}\definition{female personal name}}
\entry{madmed}{\headword{madmed}\pos{n.}\sensenumber{2}\definition{type of big tree that grows in the bush}}
\entry{Mado}{\headword{Mado}\pos{pn.}\sensenumber{2}\definition{male personal name}}
\entry{maduma}{\headword{maduma}\pos{n.}\sensenumber{2}\definition{village}\example{Obo dindugän Llimoll maduma we.}{He ran to Limol village.}}
\entry{Madura}{\headword{Madura}\pos{pn.}\sensenumber{2}\definition{male personal name}}
\entry{madura}{\headword{madura}\pos{n.}\sensenumber{2}\definition{type of introduced bananaUlle dan, obo däg a yuwog dag. Käp a o me otnan ma dag, ine ttänttämang e säpall ma dag. Kire da yu ma dag otnan e. Ako nge mubine da ngasnges ma dan mer mokowang e. (It's big; its bunches are plentiful. When ripe, its fruit is eaten; it's put in boiling water. When unripe, it's cooked to be eaten. Unripe fruit are cooked to eat. It's also for making a tasty mubine.)}}
\entry{=mae1}{\headword{=mae1}\pos{n. cl.}\sensenumber{2}\definition{restrictive clitic; only}\example{Ngäna Ende eka walle mae eka allan.}{I only speak in Ende.}\etymology{from =me + =ae₁}}
\entry{=mae2}{\headword{=mae2}\pos{n. cl.}\sensenumber{2}\definition{perlative case clitic; through}\example{Däbe dirom ulle de dazu kuddäll e mae.}{I shot that big cassowary to death.}\etymology{from =me + =ae₂}}
\entry{maebo}{\headword{maebo}\pos{n.}\sensenumber{2}\definition{type of spear made from sago}}
\entry{maemae}{\headword{maemae}\pos{n.}\sensenumber{2}\definition{type of big tree that grows in places where yam gardens are planted; used for making canoes and paddles}}
\entry{maer}{\headword{maer}\pos{n.}\sensenumber{2}\definition{myrrh}\etymology{from Englishmyrrh}}
\entry{mag}{\headword{mag}\variant{dial. var. of}{mäg 1}}
\entry{Maggie}{\headword{Maggie}\variant{sp. var. of}{Megi}}
\entry{mai}{\headword{mai}\pos{n.}\sensenumber{2}\definition{type of sagoAi mägag täkällang sana dan. (It's a thorny sago that produces well.)}}
\entry{maigag1}{\headword{maigag1}\pos{n.}\sensenumber{2}\definition{northern brown bandicoot}}
\entry{maigag2}{\headword{maigag2}\pos{n.}\sensenumber{2}\definition{type of introduced bananaBo pätt a kälsäre dan, ddone tupi pänyanzag dan. Obo käp a o me kukokukollae dag, ako mer mokowang dag be kire me ddone yu ma dag. (Its trunk is small; it doesn't grow tall. When ripe, the fruit are green and tasty, but when unripe, they aren't cooked.)}}
\entry{maiwa}{\headword{maiwa}\pos{n.}\sensenumber{2}\definition{type of pandanus with a long, smooth fruit cooked in mumu}}
\entry{maiya}{\headword{maiya}\pos{n.}\sensenumber{2}\definition{bulbil (fruit of yam that is planted)}}
\entry{Mak}{\headword{Mak}\pos{pn.}\sensenumber{2}\definition{male personal name}}
\entry{mak}{\headword{mak}\pos{n.}\sensenumber{1}\definition{mark}\sensenumber{2}\definition{to choose}\example{Ubi komllaebmae llo de mak dägaeyo.}{They each chose a tree.}\etymology{from Englishmark}}
\entry{Makaka}{\headword{Makaka}\pos{pn.}\sensenumber{2}\definition{female personal name}}
\entry{makäp}{\headword{makäp}\pos{loc.}\sensenumber{1}\definition{inside, in, within, among}\example{Kumuddäga ebdo makäp me, melem bättemän.}{I will finish the work in three days.}\example{Mɨnyi bina tuwelb makäp att ttongo da sänge bagän ngänäm.}{One among you twelve will betray me.}\example{Angde dälltamän, ngäna wayati de dänye ngämo nagnag aba makäp me.}{When they arrived, I shared the watermelon with my friends.}\sensenumber{2}\definition{duration}\example{Diba mamoe makäp me, ngämi misde ddäddäg abaene towall llälläm emae lländräg gogaebne.}{During the hunt, we were just listening for the rustling of the animals in the grass.}\example{Bogo dallnän yuwog abal dektta yaba pate oblle ngämingg e, be bogo ako orwa gognän oba ngämingg makäp me.}{She would go to many doctors so they could help her, but during their treatment, she would suffer again.}}
\entry{maket}{\headword{maket}\pos{n.}\sensenumber{2}\definition{market}\example{Ngäna maket me melemang dan.}{I work in the market.}\etymology{from Englishmarket}}
\entry{makett}{\headword{makett}\variant{fr. var. of}{maket}}
\entry{Maki}{\headword{Maki}\pos{pn.}\sensenumber{2}\definition{female personal name}}
\entry{malam}{\headword{malam}\pos{S vt.}\sensenumber{1}\definition{to obey, followDa ttongo lla da bam e ngasges e ada nagän, bongo mɨnyi nanges. (If someone tells you to do something, you will do it.)}\example{Lla da obäne eka de dämalaemneyo.}{People used to obey his words.}\sensenumber{2}\definition{to accept}\example{Ngäna bänene kandärmang eka de malam eran.}{I accept your apology.}\allomorph{malaem}\allomorph{mal}}
\entry{malmal}{\headword{malmal}\pos{S vt.}\sensenumber{1}\definition{to mark land (for planting or settlement)}\example{Bobäll ttängäm malmal e.}{Let's go mark a garden spot.}\example{Ge ma de dämaleyo.}{They chose to settle this area.}\sensenumber{2}\definition{to accuse, blame}\example{Ubi yuwog abal ttoen e bam malmal nallo.}{They are accusing you of so many things.}\allomorph{malnen}\allomorph{mal}}
\entry{malonäbe}{\headword{malonäbe}\pos{n.}\sensenumber{2}\definition{type of reed that is too soft for weaving}}
\entry{malla}{\headword{malla}\pos{neg. ptcl.}\sensenumber{2}\definition{not}\example{Ngäna tatuma ibi allan be malla mɨnyi tatuma me de.}{I am going to wash, but not at the washing place.}\example{Bogo malla gänyme zegatt dan.}{He was not born here.}}
\entry{mallam}{\headword{mallam}\variant{var. of}{Malläm}}
\entry{Malläm}{\headword{Malläm}\pos{pn.}\sensenumber{2}\definition{Malam (Ende-speaking village in Morehead Rural LLG; a two-hour walk 9.3km) from Limol; GPS: -8.712716, 142.656097)}}
\entry{mallet}{\headword{mallet}\variant{var. of}{maret}}
\entry{mallmell}{\headword{mallmell}\variant{var. of}{malmal}}
\entry{mam}{\headword{mam}\pos{n.}\sensenumber{1}\definition{blood}\example{Ngäna ddia mam de dalltäne.}{I was following the blood.}\sensenumber{2}\definition{to bleed}\example{Da ngäna volleyball botngoe känaebag, tupi da mɨnyi mam bogon.}{If I play volleyball tomorrow, my pointer finger will bleed.}\sensenumber{3}\definition{red; pink}\example{Emi ngänaeka gogon a obo ikop a mamang gognegän.}{Emi cried and her eyes were red.}\sensenumber{3}\definition{bloody}\example{mamaemamae kanas}{bloody spear}\sensenumber{3}\definition{red}\example{Ngämo mamam mätta de aya notan?}{Who ate my red yam?}\sensenumber{3}\definition{red}\example{Gabma yaba pätt a ai dan mamall yäbäd att a mamamamang bogon.}{White people's bodies can quickly become red from the sun.}\subentry{\headword{mamaemamae}\pos{mod.}\definition{bloody}}\subentry{\headword{mamam}\pos{col.}\definition{red}}\subentry{\headword{mamamam}\pos{col.}\definition{red}}}
\entry{mama1}{\headword{mama1}\pos{n.}\sensenumber{1}\definition{grass pile}\example{Däräng dallän ada kukiny mama de ikop dägagän.}{The dog went and saw a pile of kukiny grass.}\sensenumber{2}\definition{small\textbackslash_house}}
\entry{mama2}{\headword{mama2}\pos{kin.}\sensenumber{2}\definition{mother}}
\entry{mama3}{\headword{mama3}\pos{n.}\sensenumber{2}\definition{food (baby talk word)}}
\entry{mamal}{\headword{mamal}\variant{var. of}{mamall}}
\entry{mamall}{\headword{mamall}\pos{adv.}\sensenumber{2}\definition{quickly}\example{Lläkäm a ttongo mamall zozoang dan.}{A mushroom goes rotten quickly.}\sensenumber{2}\definition{immediately, right away}\example{Itrell me mamalldae meresin e gogmallo.}{In sickness, they go for medicine immediately.}\example{Mamalldae wiyamom!}{(You all) come right away!}\sensenumber{2}\definition{quickly}\example{Abo bongo mamamamall ngämim yangkollmällneg.}{You must follow us quickly.}\etymology{mamall + =dae₂}\subentry{\headword{mamalldae}\pos{adv.}\definition{immediately, right away}}\subentry{\headword{mamamamall}\pos{adv.}\definition{quickly}}}
\entry{mamamemett}{\headword{mamamemett}\pos{mod.}\sensenumber{2}\definition{fast beatMamamamall kälepale bandra lätät a tatatatraem papa ma ngalen}}
\entry{mamälldae}{\headword{mamälldae}\variant{var. of}{mamalldae}}
\entry{mambag}{\headword{mambag}\pos{n.}\sensenumber{2}\definition{type of goannaOnggall ingoll za kälsäre be llanmeny dan a ddäddäg ma dan. (An animal like a frilled lizard, but it has no frills and is edible.}\example{Sali bi mambag de ttägäll ik mi näbäddallo.}{Saly and them killed the mambag in the anthill.}}
\entry{Mame}{\headword{Mame}\pos{pn.}\sensenumber{2}\definition{female personal name}}
\entry{mame}{\headword{mame}\pos{S vi.}\sensenumber{2}\definition{to fall, rain}\example{Yogoll ulle da dämanän.}{It was raining a lot.}\allomorph{mam}\allomorph{ma}\allomorph{me}}
\entry{mameat}{\headword{mameat}\pos{n.}\sensenumber{2}\definition{papaya, pawpawTtänttam itrell me kutt de otnan ma dag a ako ttam utt de sespen ma dag id nanen e. (When sick with malaria, the seeds are eaten, and the young leaves are boiled so the extract can be drunk.)}}
\entry{Mamen}{\headword{Mamen}\pos{pn.}\sensenumber{2}\definition{Mamen (toponym)}}
\entry{mami}{\headword{mami}\pos{kin.}\sensenumber{2}\definition{mommy, mummy}}
\entry{mamkiel}{\headword{mamkiel}\pos{n.}\sensenumber{2}\definition{type of native bananaUlle dan. Käp a o me otät ma da ako kire da yu ma dan. Obo ine da kälae nane ma dan meresin ingoll ddob itrel me. (It's big. When ripe, its fruit is eaten, and when unripe, it's cooked. Its water is drunk a little like medicine for some illnesses.)}}
\entry{mamlla}{\headword{mamlla}\pos{n.}\sensenumber{2}\definition{rope, string}\example{Obo llan de mamlla walle dämällanegän.}{He tied his ears with string.}}
\entry{mamoe}{\headword{mamoe}\pos{S vt.}\sensenumber{1}\definition{to hunt, go hunting}\example{Obo moko da mamoe e dan.}{He wants to go hunting.}\example{Ubi därängmeny damoenegnän.}{They were hunting without dogs.}\example{Namoeaemnalla.}{We were hunting them.}\sensenumber{2}\definition{hunt}\example{Mamoe a angde gongkamalle, ddob lla da mɨnyi po we ngattong e gobällalle.}{When the hunt starts, some men will go first to block the way.}\allomorph{moe}\allomorph{mae}\allomorph{namoe}}
\entry{mamon}{\headword{mamon}\pos{S vt.}\sensenumber{1}\definition{to string (e.g. a bow, sago beater)}\example{Tätäm ngäna bägäl de damon.}{Yesterday, I strung the bow.}\example{Bägäl a zime amonan?}{Is the bow strung?}\example{Ttäle da amonan.}{My leg is strung (i.e. it's tight and I can't flex it).}\sensenumber{2}\definition{to fashion, shape, make}\example{Mälla da mɨnyi abor de bamonän.}{The woman will fashion a sago beater.}\allomorph{mon}}
\entry{mamos1}{\headword{mamos1}\pos{n.}\sensenumber{2}\definition{comb-crested jacanaWalle menae toromenyang pa dan. (It's a bird that is always standing by the water.)}}
\entry{mamos2}{\headword{mamos2}\pos{n.}\sensenumber{2}\definition{village constableLLG kullum me ttoen täbanenang. (The planner in the local level government.)}\example{Mamos a ttängäm ikopang lla deya.}{The constable was a man who looked after the village.}}
\entry{Mana}{\headword{Mana}\pos{pn.}\sensenumber{2}\definition{female personal name}}
\entry{Manaleato}{\headword{Manaleato}\pos{pn.}\sensenumber{2}\definition{female personal name}}
\entry{Manaliato}{\headword{Manaliato}\variant{sp. var. of}{Manaleato}}
\entry{Manang}{\headword{Manang}\pos{pn.}\sensenumber{2}\definition{male personal name}}
\entry{mandri}{\headword{mandri}\pos{n.}\sensenumber{2}\definition{cultivated lemon tree}}
\entry{mandde}{\headword{mandde}\pos{n.}\sensenumber{2}\definition{Monday}\etymology{from EnglishMonday}}
\entry{Maneya}{\headword{Maneya}\pos{pn.}\sensenumber{2}\definition{name of a person}}
\entry{mani}{\headword{mani}\pos{n.}\sensenumber{1}\definition{money}\example{Bongo ge käza de mani muang nägag.}{You should sell this crocodile for money.}\sensenumber{2}\definition{kina (PGK, the currency of Papua New Guinea)}\sensenumber{2}\definition{money}\etymology{from Englishmoney}\subentry{\headword{mani käp}\pos{n.}\definition{money}}}
\entry{manika}{\headword{manika}\pos{n.}\sensenumber{2}\definition{cassava}}
\entry{mankäp}{\headword{mankäp}\pos{n.}\sensenumber{2}\definition{calf (back of leg)}}
\entry{mantär}{\headword{mantär}\pos{n.}\sensenumber{2}\definition{type of flat woven rope made from tree bark.Za patkol ma dan. (It's for tying things.)}}
\entry{Mang}{\headword{Mang}\pos{pn.}\sensenumber{2}\definition{Mang (toponym)}}
\entry{mang}{\headword{mang}\pos{kin.}\sensenumber{2}\definition{brother (of a woman)Män bo gulag zeg lla llɨg. (A boy born in the same group as a girl.)}\example{Niki era Nikol bo mang dan.}{Nicky is Nicole's brother.}\sensenumber{2}\definition{nonsingular form of mang}\subentry{\headword{mangmang}\pos{kin.}\definition{nonsingular form of mang}}}
\entry{Mangel}{\headword{Mangel}\pos{pn.}\sensenumber{2}\definition{Mangel (toponym)}}
\entry{Manggeya}{\headword{Manggeya}\pos{pn.}\sensenumber{2}\definition{female personal name}}
\entry{manggo}{\headword{manggo}\pos{n.}\sensenumber{2}\definition{mango treeGullem ddäddägatt me mängalae ttoe de kokllo ma dan gullem ddäddägatt nyäny e. (After a snake bite, the bark is quickly scratched to apply on the snake bite.)}\example{manggo däg}{bunch of mango}}
\entry{mangkimangki}{\headword{mangkimangki}\pos{n.}\sensenumber{2}\definition{type of game involving chasing}}
\entry{Mangkol}{\headword{Mangkol}\pos{pn.}\sensenumber{2}\definition{female personal name}}
\entry{manglle}{\headword{manglle}\pos{A vt.}\sensenumber{2}\definition{to lure}\example{Ngäna ddia llɨg eka däga adawede oba manglle bäga.}{I made baby deer sounds to lure her.}\sensenumber{2}\definition{friendly, appealing, sociable}\example{Ngäna lla bo peyang mangllong dan.}{I am friendly with people.}\etymology{manglle + =ang}\subentry{\headword{mangllong}\pos{mod.}\definition{friendly, appealing, sociable}}}
\entry{marat}{\headword{marat}\pos{n.}\sensenumber{2}\definition{weaving pattern with a circular bottom and rectangular sides}}
\entry{mare}{\headword{mare}\pos{n.}\sensenumber{2}\definition{type of pandanusDu mab, penz me päddabag dan, wätät ma dan. (Wild pandanus that grows in the river; it's edible.)}}
\entry{Mareas}{\headword{Mareas}\pos{pn.}\sensenumber{2}\definition{personal name}}
\entry{Marega}{\headword{Marega}\pos{pn.}\sensenumber{2}\definition{male personal name}}
\entry{maret}{\headword{maret}\pos{n.}\sensenumber{2}\definition{marriage}\etymology{from Englishmarriage}}
\entry{Mareyas}{\headword{Mareyas}\pos{pn.}\sensenumber{2}\definition{Mareyas (toponym)}}
\entry{Maria}{\headword{Maria}\pos{pn.}\sensenumber{2}\definition{female personal name}}
\entry{Marian}{\headword{Marian}\pos{pn.}\sensenumber{2}\definition{female personal name}}
\entry{Marias}{\headword{Marias}\pos{pn.}\sensenumber{2}\definition{male personal name}}
\entry{maribärät}{\headword{maribärät}\pos{n.}\sensenumber{2}\definition{type of long yam with a white interior and hairs}}
\entry{Marion}{\headword{Marion}\pos{pn.}\sensenumber{2}\definition{female personal name}}
\entry{Mark}{\headword{Mark}\variant{sp. var. of}{Mak}}
\entry{markae}{\headword{markae}\pos{n.}\sensenumber{1}\definition{white person}\sensenumber{2}\definition{white, Western; foreign}\sensenumber{2}\definition{gun}\sensenumber{2}\definition{English}\etymology{from Taeme}\subentry{\headword{markae bägäl}\pos{n.}\definition{gun}}\subentry{\headword{markae eka}\pos{pn.}\definition{English}}}
\entry{markaebärät}{\headword{markaebärät}\pos{n.}\sensenumber{2}\definition{type of medium-sized yam with a white interior, thorns, and no hairs}}
\entry{market}{\headword{market}\variant{sp. var. of}{maket}}
\entry{Martha}{\headword{Martha}\pos{pn.}\sensenumber{2}\definition{female personal name}}
\entry{Martin}{\headword{Martin}\variant{sp. var. of}{Maten}}
\entry{Mary}{\headword{Mary}\variant{sp. var. of}{Meri}}
\entry{Maryanne}{\headword{Maryanne}\variant{sp. var. of}{Merian}}
\entry{Mas}{\headword{Mas}\pos{pn.}\sensenumber{2}\definition{male personal name}}
\entry{mas}{\headword{mas}\pos{n.}\sensenumber{2}\definition{small bone found in cassowaries}}
\entry{masaka}{\headword{masaka}\pos{n.}\sensenumber{2}\definition{single stem plant with inedible fruit}}
\entry{Masam}{\headword{Masam}\pos{pn.}\sensenumber{2}\definition{Bitur language}}
\entry{masar}{\headword{masar}\pos{kin.}\sensenumber{1}\definition{grandfather (one's parent's father; reciprocal); ancestorbaba bo o yae bo mäda}\example{Masar, bablle ngoe a ddone dag.}{Grandfather, you have no teeth.}\sensenumber{2}\definition{grandchild (man's child's child; reciprocal)}\sensenumber{3}\definition{father-in-law (woman's husband's father; reciprocal)}\sensenumber{4}\definition{daughter-in-law (man's son's wife; reciprocal)}\sensenumber{5}\definition{uncle-in-law (woman's husband's mother's younger brother; reciprocal)}\sensenumber{6}\definition{niece-in-law (man's elder sister's son's wife; reciprocal)}\sensenumber{7}\definition{ancestral, old (of one's grandparents' time)}\example{masar täräp me}{in the olden days}\example{masar eka}{ancestral language}\sensenumber{1}\definition{grandparents}\example{Ngämo masamasar a ngämo pate gognegnän Ende eka walle de.}{My grandparents spoke to me in Ende.}\sensenumber{2}\definition{forefathers, ancestors}\example{God bo eka da angde Ende tän e gozenän, lla da Ende tän bo masamasar ngasnen de däpnaeaemeyo.}{When God's word entered the Ende tribe, people turned away from the ways of the Ende tribe's ancestors.}\allomorph{masa}\subentry{\headword{masamasar}\pos{kin.}\definition{grandparents}}}
\entry{=masäm}{\headword{=masäm}\variant{dial. var. of}{=mattäm}}
\entry{=masem}{\headword{=masem}\variant{var. of}{=mattäm}}
\entry{=masen}{\headword{=masen}\variant{var. of}{=mattäm}}
\entry{Masingara}{\headword{Masingara}\pos{pn.}\sensenumber{2}\definition{Masingara (Bine-speaking village in Oriomo-Bituri Rural LLG)}}
\entry{Masta}{\headword{Masta}\pos{pn.}\sensenumber{2}\definition{female personal name}}
\entry{Mata}{\headword{Mata}\pos{pn.}\sensenumber{2}\definition{Mata (in Morehead Rural LLG)}}
\entry{matamata}{\headword{matamata}\pos{n.}\sensenumber{2}\definition{type of tree that grows along swamps with white and blue flowers and edible fruit (black outside, red inside) that ripen during January and February}}
\entry{Mataru}{\headword{Mataru}\pos{pn.}\sensenumber{2}\definition{personal name}}
\entry{Matär}{\headword{Matär}\pos{pn.}\sensenumber{2}\definition{male personal name}}
\entry{Maten}{\headword{Maten}\pos{pn.}\sensenumber{2}\definition{male personal name}}
\entry{Mathias}{\headword{Mathias}\variant{sp. var. of}{Matias}}
\entry{Mathilda}{\headword{Mathilda}\pos{pn.}\sensenumber{2}\definition{female personal name}}
\entry{Matias}{\headword{Matias}\pos{pn.}\sensenumber{2}\definition{male personal name}}
\entry{Maties}{\headword{Maties}\variant{sp. var. of}{Matias}}
\entry{Matilda}{\headword{Matilda}\variant{sp. var. of}{Mathilda}}
\entry{matu}{\headword{matu}\pos{n.}\sensenumber{1}\definition{ground}\example{Baet abo gongetamän matu we.}{Cuscus came down to the ground.}\sensenumber{2}\definition{lower part}\example{Matu me kollko ddage me dedme gotogolän.}{He hid there on the lowest branch.}}
\entry{matta}{\headword{matta}\pos{n.}\sensenumber{1}\definition{shoulder}\sensenumber{2}\definition{to shoulder, carry on one's shoulder}\example{Angde llo täraem a gottamänän, ngäna dibaballe matta däganegne ma we.}{When the tree cutting was finished, I then shouldered them to the house.}\sensenumber{3}\definition{eight (lit. shoulder; body counting numeral)}}
\entry{=mattäm}{\headword{=mattäm}\pos{n. cl.}\sensenumber{3}\definition{ablative case clitic; from}\allomorph{wattäm}\allomorph{battäm}\allomorph{attäm}\allomorph{mase}}
\entry{mattgal}{\headword{mattgal}\pos{S vt.}\sensenumber{3}\definition{to put in fire}\example{Yu de daudewän a otät de yu wi dämattgalän.}{He lit a fire and put the food in the fire.}}
\entry{Matthew}{\headword{Matthew}\variant{sp. var. of}{Metyu}}
\entry{mattmett1}{\headword{mattmett1}\pos{S vt.}\sensenumber{3}\definition{to put in oven}\example{Sɨmell gullbe de ddob sana peyang dazgiaebeya, ada dämetteyo ttägäll ik i.}{We wrapped the boar with some sago and put it in the mumu.}\allomorph{mett}}
\entry{mattmett2}{\headword{mattmett2}\pos{n.}\sensenumber{3}\definition{plain, flat area}}
\entry{mawa}{\headword{mawa}\pos{n.}\sensenumber{3}\definition{magic}\example{mawa lla}{fairy, elf}}
\entry{maza}{\headword{maza}\pos{n.}\sensenumber{3}\definition{reef}}
\entry{Mavis}{\headword{Mavis}\pos{pn.}\sensenumber{3}\definition{female personal name}}
\entry{Max}{\headword{Max}\variant{sp. var. of}{Meks}}
\entry{mäd kubull}{\headword{mäd kubull}\pos{n.}\sensenumber{3}\definition{black bush wallaby}}
\entry{mäda}{\headword{mäda}\pos{kin.}\sensenumber{1}\definition{father}\example{Ge obo mäda wainen.}{This is his father.}\sensenumber{2}\definition{owner (of an animal)}\example{Däräng a oba mäda Kwalde bäne ibiatt de dongkollmälleyo.}{The dogs followed their owner Kwalde's footprints.}\sensenumber{1}\definition{aunt (one's father's elder sister)}\sensenumber{2}\definition{uncle (one's father's elder brother)}\example{Mädaulle daeya pu wi nallan.}{Uncle went to the swamp garden.}\sensenumber{3}\definition{brother-in-law (woman's husband's elder brother)}\sensenumber{4}\definition{sister-in-law (woman's husband's elder sister or woman's younger brother's wife; reciprocal)}\sensenumber{4}\definition{fatherless}\etymology{mäda + ulle, lit. 'older than father'}\subentry{\headword{mädaulle}\pos{kin.}\definition{aunt (one's father's elder sister)}}\subentry{\headword{mädameny}\pos{mod.}\definition{fatherless}}}
\entry{mädaolle}{\headword{mädaolle}\variant{sp. var. of}{mädaulle}}
\entry{mäde}{\headword{mäde}\variant{sp. var. of}{mäda}}
\entry{mädi}{\headword{mädi}\variant{dial. var. of}{mɨnyi}}
\entry{mäg}{\headword{mäg}\pos{kin.}\sensenumber{1}\definition{motherZaze me baba bälle ngäminggag dan. (She helps the father to make a family.)}\example{Ngämo mäg bo bin a Sana.}{My mother's name is Sana.}\sensenumber{2}\definition{source}\example{Ge buk a ibim taempmeny anggan mägda abal eka midd de.}{This book shows us the original meaning.}\sensenumber{2}\definition{aunt (one's father's elder brother's wife)}\sensenumber{2}\definition{motherless}\etymology{lit. 'older than mother'}\subentry{\headword{mäg ulle}\pos{kin.}\definition{aunt (one's father's elder brother's wife)}}\subentry{\headword{mägmeny}\pos{mod.}\definition{motherless}}}
\entry{mäga}{\headword{mäga}\pos{n.}\sensenumber{2}\definition{type of sagoTubutubu ngällngällang dan. (It produces when short.)}}
\entry{Mägaem}{\headword{Mägaem}\variant{var. of}{Mäkayam}}
\entry{Mägayam}{\headword{Mägayam}\variant{var. of}{Mäkayam}}
\entry{mägäll}{\headword{mägäll}\pos{n.}\sensenumber{1}\definition{type of tree with bark used to tie spears, white flowers, edible leaves, and finger-sized, edible red fruit with edible seeds}\sensenumber{2}\definition{strength}\sensenumber{2}\definition{friends who share a twin fruit from the mägäll treeMällayaba ttaem. (Amongst women.)}\subentry{\headword{mägällma}\pos{n.}\definition{friends who share a twin fruit from the mägäll treeMällayaba ttaem. (Amongst women.)}}}
\entry{mägda}{\headword{mägda}\pos{n.}\sensenumber{2}\definition{coconut body (as opposed to the shoot)}\etymology{lit. 'the mother'}}
\entry{mäk}{\headword{mäk}\pos{n.}\sensenumber{2}\definition{war}\example{Ngäma yae bi dazernän, ge mäk a angde gongesän.}{Our (excl.) mother lived there when this war happened.}\sensenumber{2}\definition{soldier}\example{Paelet Yesu bom Rom mäk lla yaba pate däntäpeyamän.}{Pilate handed Jesus over to the Roman soldiers.}\subentry{\headword{mäk lla}\pos{n.}\definition{soldier}}}
\entry{mäka}{\headword{mäka}\pos{n.}\sensenumber{2}\definition{plantar wart (on the soles of feet; caused by a virus)}}
\entry{mäkamäke}{\headword{mäkamäke}\pos{S vt.}\sensenumber{2}\definition{to use}\example{Ibi ge up de mullamulla we mäkanen eralla.}{We use this banana for medicine.}}
\entry{mäkan}{\headword{mäkan}\pos{n.}\sensenumber{2}\definition{desire (of someone)}\example{Ngämo mäkan a pällämpälläm eka umllang e dan.}{I want (lit. my desire is) to learn English.}}
\entry{mäkat}{\headword{mäkat}\pos{n.}\sensenumber{2}\definition{ratMaigag ingoll ddäddäg kälsäre a ddäddäg ma dan. (It's a small bandicoot-like animal and it's edible.)}\example{Mäkat llɨgllɨgag a dan bäne gaguma me.}{There is a rat with babies in your yamhouse.}}
\entry{Mäkayam}{\headword{Mäkayam}\pos{pn.}\sensenumber{2}\definition{Makayam/Tirio language}}
\entry{mäkämäkäp}{\headword{mäkämäkäp}\pos{n.}\sensenumber{2}\definition{dreadlocks}}
\entry{mäkäp}{\headword{mäkäp}\pos{n.}\sensenumber{1}\definition{knot}\sensenumber{2}\definition{to knot}}
\entry{mälamäle}{\headword{mälamäle}\pos{S vt.}\sensenumber{1}\definition{to patch}\sensenumber{2}\definition{to dress (a wound)}\example{Känaebag, ngäna mɨnyi ute de bamle.}{Tomorrow, I will dress the wound.}\allomorph{mle}\allomorph{mälanen}\allomorph{mla}}
\entry{mälmäl}{\headword{mälmäl}\pos{S vt.}\sensenumber{1}\definition{to squeeze}\sensenumber{2}\definition{accuse}}
\entry{mälla}{\headword{mälla}\pos{n.}\sensenumber{1}\definition{woman, female}\sensenumber{2}\definition{wife}\example{Bäne baba auli mälla peyang dan?}{How many wives does your father have?}\sensenumber{2}\definition{old womanUdu peyang ibiag dan, oblle mängall a ddone dan. (She walks with a stick and has no strength.)}\example{Mällause bo auma da dan de.}{The old woman's grave is there.}\sensenumber{2}\definition{married (of a male)}\sensenumber{2}\definition{adultery (of a man)}\etymology{from mälla + ause}\subentry{\headword{mällause}\pos{n.}\definition{old womanUdu peyang ibiag dan, oblle mängall a ddone dan. (She walks with a stick and has no strength.)}}\subentry{\headword{mällawang}\pos{mod.}\definition{married (of a male)}}\subentry{\headword{mälla gämäll}\definition{adultery (of a man)}}}
\entry{mälla ause}{\headword{mälla ause}\variant{fr. var. of}{mällause}}
\entry{mälla yae}{\headword{mälla yae}\pos{kin.}\sensenumber{2}\definition{aunt (one's mother's elder sister)}}
\entry{mällakutang}{\headword{mällakutang}\pos{n.}\sensenumber{2}\definition{type of tree that grows in the bush with white flowers, black bark, and wood used for house sticks}}
\entry{mällam1}{\headword{mällam1}\pos{n.}\sensenumber{2}\definition{type of plant}\allomorph{mllam}\allomorph{mllaem}\allomorph{mll}\allomorph{mäll}}
\entry{mällam2}{\headword{mällam2}\pos{S vt.}\sensenumber{2}\definition{to hold; get, grab, catch}\example{Rop de namllam ttang me.}{[You] hold the rope in your hand.}\example{Ttongo ngäna skul me tresära bin di mällaem eran.}{I hold the position of treasurer at a school.}\example{Bogo llo käp de naspunan nag da pate be nag da ddone namllaman.}{She threw the fruit to her friend but they didn't catch it.}\example{Kallkäll a damllamän obom.}{He got cold (lit. the cold got him).}\allomorph{mllam}\allomorph{mllaem}\allomorph{mll}\allomorph{mäll}}
\entry{mällamälla}{\headword{mällamälla}\pos{S vt.}\sensenumber{2}\definition{to tie}\example{Yu wi daspuniyu, mällamällawatt de, ada oba yu a bättämän.}{They threw him on the fire and tied him so that the fire would burn him.}\example{Ttongo lla da ddone mullae dan mängallang lla bo ma we bozenän obo za gämäll e, be dam de da bogo ngattong däbe lla de bämllawän.}{A person cannot enter the house of a powerful man to steal his things unless they tie that man up first.}\allomorph{mälla}\allomorph{mlla}}
\entry{mälläll}{\headword{mälläll}\variant{var. of}{märäl1}}
\entry{mälläng}{\headword{mälläng}\pos{n.}\sensenumber{2}\definition{nose}\example{Ngämo mälläng a nongonongorang dan.}{My nose is itchy.}\sensenumber{2}\definition{nostril}\example{Obo mälläng ik a ulleulle abal dagaeya.}{His nostrils were very big.}\sensenumber{1}\definition{nose hair}\sensenumber{2}\definition{antenna}\sensenumber{2}\definition{snout}\subentry{\headword{mälläng ik}\pos{n.}\definition{nostril}}\subentry{\headword{mälläng kom}\pos{n.}\definition{nose hair}}\subentry{\headword{mälläng sobasoba}\pos{n.}\definition{snout}}}
\entry{mällänggäbe}{\headword{mällänggäbe}\pos{n.}\sensenumber{2}\definition{type of plant that grows along the riverside}}
\entry{mällät}{\headword{mällät}\pos{n.}\sensenumber{2}\definition{type of tree that grows in the grassland with fruit used to ignite fires and yellow flowers with edible nectar}\sensenumber{2}\definition{type of grubAp llo me ddäddägma budar dan. (It's an edible grub on grassland trees.)}\subentry{\headword{mällätbudar}\pos{n.}\definition{type of grubAp llo me ddäddägma budar dan. (It's an edible grub on grassland trees.)}}}
\entry{mällät käp1}{\headword{mällät käp1}\pos{n.}\sensenumber{2}\definition{second stage of sago growth}\example{Sana da mällät käpang dan.}{The sago is in its second stage of growth.}}
\entry{mällät käp2}{\headword{mällät käp2}\pos{n.}\sensenumber{2}\definition{D'Albertis pythonDdägnan ma gullem da. Towall sängatt me giddollag da. (It's an edible snake. It lives in the grass.)}\example{Ag me ibi näddägallo mällät käp gullem de.}{This morning, we (incl.) ate a mällät käp snake.}}
\entry{mällätgugu}{\headword{mällätgugu}\pos{n.}\sensenumber{2}\definition{type of taro}}
\entry{mällɨng}{\headword{mällɨng}\variant{sp. var. of}{mälläng}}
\entry{mällkae}{\headword{mällkae}\pos{S vi.}\sensenumber{2}\definition{to bend}\example{Kote dämällkaenän obo pate.}{He was bowing his head (lit. bent nape) to her.}}
\entry{mällkakallamatt}{\headword{mällkakallamatt}\pos{n.}\sensenumber{2}\definition{type of venomous snake}}
\entry{mällkam}{\headword{mällkam}\pos{S vi.}\sensenumber{2}\definition{to bend}\example{Kote gomällkamän ngänaeka de gongkamän.}{He hung his head (lit. bent nape) and started to cry.}\sensenumber{2}\definition{bending over}\allomorph{mällk}\subentry{\headword{mällkamällkam}\pos{adv.}\definition{bending over}}}
\entry{mällong}{\headword{mällong}\variant{fast speech var. of}{mällawang}}
\entry{mällpa}{\headword{mällpa}\pos{kin.}\sensenumber{2}\definition{aunt (one's mother's sister)Yae bo peyang gulag zeg. (Born together with mother.)}}
\entry{mällpe}{\headword{mällpe}\pos{n.}\sensenumber{2}\definition{mucus}}
\entry{mämbär}{\headword{mämbär}\pos{n.}\sensenumber{2}\definition{type of big, round yam with a white interior, hairs, and no thorns}}
\entry{mämrem}{\headword{mämrem}\pos{A vi.}\sensenumber{2}\definition{to growl}\example{Bogo mämrem peyang eka gogon.}{He spoke with a growl.}}
\entry{män1}{\headword{män1}\pos{n.}\sensenumber{2}\definition{type of big tree cultivated in the bush with white flowers and blue and purple fruit that cassowary eat; tobacco is planted in the soil near this tree after it is burned}}
\entry{män3}{\headword{män3}\pos{n.}\sensenumber{1}\definition{girl}\example{Ngäna ge män de dangem.}{I adopted this girl.}\sensenumber{2}\definition{sister}\sensenumber{3}\definition{She's my daughter.}\example{Bogo ngämo män kälsre dan.}{The young girl put flowers in her hair.}\sensenumber{3}\definition{young girl}\example{Män duwar da bun mi popo de nowattällan.}{The young girl put flowers in her hair.}\sensenumber{3}\definition{nonsingular form of män duwar}\sensenumber{3}\definition{nonsingular form of män}\example{Mänmän a sana de zazaba we dändäraebeyo.}{Girls stuffed sago into sago bags.}\example{Ngämo mänmän a aba bin a ada, Ina, Lulu, Jamila.}{My sisters' names are Ina, Lulu, and Jamila.}\subentry{\headword{män duwar}\pos{n.}\definition{young girl}}\subentry{\headword{mänmän duwaduwar}\pos{n.}\definition{nonsingular form of män duwar}}\subentry{\headword{mänmän}\pos{n.}\definition{nonsingular form of män}}}
\entry{mänang}{\headword{mänang}\pos{kin.}\sensenumber{1}\definition{father-in-law (man's wife's father; reciprocal)}\sensenumber{2}\definition{mother-in-law (man's wife's mother; reciprocal)}\sensenumber{3}\definition{son-in-law (one's daughter's husband; reciprocal)}\sensenumber{4}\definition{brother-in-law (one's younger sister's husband)}}
\entry{mända}{\headword{mända}\pos{n.}\sensenumber{1}\definition{thumb; big toe}\sensenumber{2}\definition{five (lit. thumb; body counting numeral)}\example{mända ollong}{five times}}
\entry{mändmänd}{\headword{mändmänd}\pos{S vt.}\sensenumber{1}\definition{to raise, rear}\example{Ubi obom damändeyo.}{They raised him.}\example{Indonesia me gänya ddäddäg de damändaebneyo.}{In Indonesia, they tamed this animal.}\sensenumber{2}\definition{to feed}\example{Ngäna obom kumuddäga nge alle bamänd.}{I will feed him three coconuts.}\allomorph{mänd}\allomorph{mändnan}}
\entry{mänddmändd}{\headword{mänddmändd}\pos{S vi.}\sensenumber{1}\definition{to drown, struggle in waterWalle ik me kuddäll e kämbäg. (Going into the water to death.)}\example{Bogo mänddmändd e gongkamän walle we.}{She started to drown in the water.}\example{Bogo aoli gomänddän walle me.}{She almost drowned in the water.}\sensenumber{2}\definition{to drown, force underwater}\example{Bogo ddia de ine dämänddän.}{He drowned the deer in the water.}\sensenumber{3}\definition{sink}\allomorph{mändd}\allomorph{mänddnan}}
\entry{mänkot}{\headword{mänkot}\pos{n.}\sensenumber{1}\definition{uninitiated person}\sensenumber{2}\definition{non-believer}}
\entry{mänmänpitepite}{\headword{mänmänpitepite}\pos{n.}\sensenumber{2}\definition{type of plant that grows by the river; eaten by wallabies}}
\entry{mängal}{\headword{mängal}\pos{mod.}\sensenumber{2}\definition{quick}\example{Ge towall a mängalae penongg allan.}{This grass is burning quickly.}\sensenumber{2}\definition{quick}\example{Ngämi wätät de mängamängalae yu dägaebeya.}{We (excl.) quickly cooked the food.}\subentry{\headword{mängalmängal}\pos{mod.}\definition{quick}}}
\entry{mängall}{\headword{mängall}\pos{n.}\sensenumber{1}\definition{strength, power}\example{Bäne mängall a ulle dan, be ngämo kälsre dan.}{You are stronger than me (lit. your strength is big, but mine is small).}\sensenumber{2}\definition{to struggle}\example{Däbe mälla da ada mängall gognän tudi nyongkoe e.}{That woman was struggling to pull the fishing line.}\sensenumber{3}\definition{poison}\sensenumber{3}\definition{encouragement, cheer}\example{Obo pallall lla da mäse obom mängall eka dägneyo.}{His supporters tried to cheer him on.}\sensenumber{3}\definition{weak}\example{Sisri obo pätt a mängallmeny allan.}{His body is weak now.}\example{Wel ulle da mängallmeny gogon.}{The storm grew weak.}\allomorph{mänga}\etymology{mängall + =meny}\subentry{\headword{mängall eka}\pos{n.}\definition{encouragement, cheer}}\subentry{\headword{mängallmeny}\pos{mod.}\definition{weak}}}
\entry{mängallang}{\headword{mängallang}\pos{mod.}\sensenumber{3}\definition{strong}\example{ddobae mängallang lla}{very strong person}}
\entry{mängamängal}{\headword{mängamängal}\variant{fast speech var. of}{mängalmängal}}
\entry{mängäll}{\headword{mängäll}\variant{var. of}{mängall}}
\entry{mänyan}{\headword{mänyan}\pos{kin.}\sensenumber{1}\definition{younger sibling of the same-sex (man's younger brother or woman's younger sister)Zaze me kollmällang zegatt. (The one born later in a generation.)}\sensenumber{2}\definition{co-sibling-in-law (man's wife's younger sister's husband or woman's husband's younger brother's wife)}\sensenumber{2}\definition{younger siblings}\subentry{\headword{mänyanmänyan}\pos{kin.}\definition{younger siblings}}}
\entry{mänyän}{\headword{mänyän}\pos{n.}\sensenumber{2}\definition{type of fish}}
\entry{mänyi}{\headword{mänyi}\variant{var. of}{mɨnyi}}
\entry{mänymäny}{\headword{mänymäny}\pos{v.}\sensenumber{2}\definition{to vomit}\allomorph{mäny}}
\entry{märal}{\headword{märal}\pos{A vi.}\sensenumber{1}\definition{to curse, swear}\sensenumber{2}\definition{to curse}\example{Bongo ikopse alle lla de ddone märal näga adawede obo kuddäll bogon.}{You cannot curse a person in prayer so that they die.}}
\entry{märäl1}{\headword{märäl1}\pos{n.}\sensenumber{2}\definition{size}\example{allko märäl ziz}{a fly-sized insect}\example{Lla zazer ma mälläll be obo pätt a ulle abal daeya.}{The size was so that a person could enter, but his body was really big.}}
\entry{märäl2}{\headword{märäl2}\pos{n.}\sensenumber{2}\definition{age-mate, someone of the same age}\example{Warani ngämo märäl dan.}{Warani is my age-mate.}\example{Ngämo märäl a aenen ke?}{Who's the same age as me (lit. my age-mate)?}}
\entry{märäll}{\headword{märäll}\variant{var. of}{märäl1}}
\entry{mäsdae}{\headword{mäsdae}\variant{dial. var. of}{misdae}}
\entry{mäsde}{\headword{mäsde}\variant{dial. var. of}{misdae}}
\entry{mäse}{\headword{mäse}\pos{TAM ptcl.}\sensenumber{1}\definition{imminent particle (indicates that something about to take place)}\example{Ngäna ako mäse tutu de dängkälne, gänyaollemae ikop dige sɨmell da angttägan.}{As I was about to climb the mountain, I saw a pig coming towards me.}\sensenumber{2}\definition{conative particle (indicates an attempt; to try to do something)}\example{Ngäna ngängälläb e mäse däga.}{I tried to lift it.}\sensenumber{3}\definition{not yet full, unsatisfied}\example{Ngämlle kame ako otät de nasi, ngämo käm a mäse agan.}{Give me more food; my stomach isn't full yet.}\sensenumber{1}\definition{naked}\example{Dadabi, mäsemäse llɨg a dongkoenmällaemneyo.}{The boys chased them naked.}\sensenumber{2}\definition{empty-handed}\example{Mäsemäse ma gongttägeya.}{We arrived home empty-handed.}\subentry{\headword{mäsemäse}\pos{adv.}\definition{naked}}}
\entry{mäta}{\headword{mäta}\pos{n.}\sensenumber{2}\definition{type of tree with a red stem, young green leaves, and bark used as rope when old; big trees are used to light fire; liquid is extracted and drunk to treat cough}\sensenumber{2}\definition{rope made from mäta bark (used to secure fencing)}\subentry{\headword{mäta mamlla}\pos{n.}\definition{rope made from mäta bark (used to secure fencing)}}}
\entry{mätar}{\headword{mätar}\variant{var. of}{mätaru}}
\entry{mätar onyang2}{\headword{mätar onyang2}\pos{n.}\sensenumber{2}\definition{type of tree}}
\entry{mätaru}{\headword{mätaru}\pos{mod.}\sensenumber{2}\definition{calm, peaceful, quiet}\example{Tämamae ttängäm a mätaru abal gogon.}{The whole place became very calm.}\sensenumber{2}\definition{peace officerLla da aya mätar eka de paenen era mer giddol e. (The person who says things about peace so people live well.)}\etymology{ony + =ang, lit. 'bringer of peace'}\subentry{\headword{mätar onyang1}\pos{n.}\definition{peace officerLla da aya mätar eka de paenen era mer giddol e. (The person who says things about peace so people live well.)}}}
\entry{mätämätär}{\headword{mätämätär}\pos{n.}\sensenumber{2}\definition{type of tree}}
\entry{Mätär}{\headword{Mätär}\pos{pn.}\sensenumber{2}\definition{Mätär (toponym)}}
\entry{mätemäte}{\headword{mätemäte}\pos{n.}\sensenumber{2}\definition{type of introduced bananaTupi tanong dan. Obo käp a o me otät ma da ako binzeg ma dan otät e. Adawae kire da ako yu ma dan. Käp yu att moep a bod uteute nyänan ma dan. (It's a bit long. When ripe, it's eaten and heated on the fire to be eaten. Similarly, it's cooked when unripe. The ash made after cooking it is rubbed on lip sores.)}}
\entry{mätka}{\headword{mätka}\pos{n.}\sensenumber{2}\definition{type of tall grass with a single stem and bunches of up to 12 small, edible fruit that grow near the base}}
\entry{mätkakallamatt}{\headword{mätkakallamatt}\pos{n.}\sensenumber{2}\definition{type of snakeGullem tubutubu dan. Sägädsägäd da a kukolkukoll ttoe ingollang dan. Kuddäll e lla ddäddägang dan. (It's a short snake. It has yellow or green skin. It bites people to death.)}}
\entry{mätkin}{\headword{mätkin}\pos{n.}\sensenumber{1}\definition{ring finger}\sensenumber{2}\definition{two (lit. ring finger; body counting numeral)}}
\entry{mätta}{\headword{mätta}\pos{n.}\sensenumber{2}\definition{lesser yam}\example{Mätta da nyäng de bodo dägagän.}{Yams filled the basket.}}
\entry{mätta pirpir}{\headword{mätta pirpir}\pos{n.}\sensenumber{2}\definition{rainbow bee-eaterKullum me pallganenang pa dag. (They are birds that perch in groups.)}}
\entry{mätta pirpirik}{\headword{mätta pirpirik}\variant{var. of}{mätta pirpir}}
\entry{mättae}{\headword{mättae}\pos{S vt.}\sensenumber{2}\definition{to threaten, raise fists}\example{Ngämo mälla bom mättaenen nängkaman.}{I started to threaten my wife.}\example{Ngäna obom damttaene.}{I was threatening him.}\allomorph{mttae}}
\entry{mättmätt}{\headword{mättmätt}\pos{S vi.}\sensenumber{1}\definition{to wear, dress oneself, put on}\example{Bogo banggu mättmätt allan.}{He is wearing a headdress.}\example{Mägda sisor kaptte amättan.}{Mother put on new clothes.}\sensenumber{2}\definition{to dress}\example{Ngäna obom mättmätt eran.}{I am dressing her.}\allomorph{mätt}\allomorph{mättnan}}
\entry{mäzi}{\headword{mäzi}\variant{dial. var. of}{mɨnyi}}
\entry{=me}{\headword{=me}\pos{n. cl.}\sensenumber{1}\definition{locative case clitic; in, at, on}\example{Ngäna ngämo pällämpälläm palle de ngämo mittapa me nowattällan.}{I put my white clay on my cheeks.}\example{bem ulle me}{in the big sea}\example{däbe täräp me}{at that time}\sensenumber{2}\definition{during, while}\example{Bogo iang me bandra allan.}{He was singing while weaving.}\allomorph{mi}\allomorph{mi}\allomorph{äme}}
\entry{meda}{\headword{meda}\variant{dial. var. of}{mäda}}
\entry{Medang}{\headword{Medang}\pos{pn.}\sensenumber{2}\definition{Medang (toponym)}}
\entry{medä}{\headword{medä}\variant{var. of}{mäda}}
\entry{medol}{\headword{medol}\pos{n.}\sensenumber{2}\definition{medal}\etymology{from Englishmedal}}
\entry{=mee}{\headword{=mee}\variant{sp. var. of}{=me}}
\entry{Megam}{\headword{Megam}\pos{pn.}\sensenumber{2}\definition{male personal name}}
\entry{Megi}{\headword{Megi}\pos{pn.}\sensenumber{2}\definition{female personal name}}
\entry{mekae}{\headword{mekae}\pos{n.}\sensenumber{2}\definition{type of tree with white flowers and edible nuts that children eat}}
\entry{mekewa}{\headword{mekewa}\pos{n.}\sensenumber{2}\definition{type of introduced bananaTupi pänyanzag dan, ttongo mer abal up dan, obo däg a yuwog abal dag. O me obo käp a mer abal mokowang dag. Ako kire da yu ma dag otnan e. (It grows tall; it's a very good banana; its bunches are plentiful. When ripe, its fruit is very tasty. When unripe, it's cooked to be eaten.)}}
\entry{Meklin}{\headword{Meklin}\pos{pn.}\sensenumber{2}\definition{female personal name}}
\entry{Meks}{\headword{Meks}\pos{pn.}\sensenumber{2}\definition{male personal name}}
\entry{melem}{\headword{melem}\pos{n.}\sensenumber{2}\definition{work}\example{Oba melem dättemäneyo a gongoseyo ma we.}{They finished their work and returned home.}\example{Obo peyang mɨnyi melem bog.}{I will work with her.}\sensenumber{1}\definition{hardworking}\example{Bogo ddobae melemang daeya.}{He was very hardworking.}\sensenumber{2}\definition{servant, laborer}\example{Ttängäm mondrog lla da melemang de dällädeyo a gagagäll duwebeyo.}{The garden workers grabbed the servant and beat him badly.}\etymology{melem + =ang}\subentry{\headword{melemang}\pos{mod.}\definition{hardworking}}}
\entry{Meliye}{\headword{Meliye}\pos{pn.}\sensenumber{2}\definition{Meliye (toponym)}}
\entry{Melvin}{\headword{Melvin}\pos{pn.}\sensenumber{2}\definition{male personal name}}
\entry{mem}{\headword{mem}\variant{dial. var. of}{mam}}
\entry{memba}{\headword{memba}\pos{n.}\sensenumber{2}\definition{member}\etymology{from Englishmember}}
\entry{memram}{\headword{memram}\pos{n.}\sensenumber{2}\definition{sweat}\example{Memram a dägagän.}{He got sweaty (lit. sweat got him).}}
\entry{mena}{\headword{mena}\pos{S vt.}\sensenumber{2}\definition{to scorch}\example{Yäbäd ttänttäm a sisor koko de dämenanegän.}{The heat of the sun scorched the young shoots.}}
\entry{menae}{\headword{menae}\pos{loc.}\sensenumber{2}\definition{side, edge; vicinity}\example{walle menae me}{by the water}\example{nyongo menae me}{on the roadside}}
\entry{mend}{\headword{mend}\pos{n.}\sensenumber{2}\definition{type of birdKänyer giddollag pa dan wälläng me. (It's a bird living alone in the bush.)}}
\entry{menizment}{\headword{menizment}\pos{n.}\sensenumber{2}\definition{management}\etymology{from Englishmanagement}}
\entry{menttäg}{\headword{menttäg}\pos{S vi.}\sensenumber{2}\definition{to shoulder, carry one one's shoulders or head}\example{Nyäng goimänttmenyeyo deyarneyo abo ma we.}{They put the baskets on their shoulders and headed home.}\allomorph{imänttäg}\allomorph{imänttmeny}\allomorph{imäntt}}
\entry{=meny}{\headword{=meny}\pos{n. cl.}\sensenumber{2}\definition{privative clitic; without}\example{mängall, mängallmeny; mu, miny}{strength, weak; payment, in debt}\allomorph{miny}}
\entry{mer1}{\headword{mer1}\pos{mod.}\sensenumber{1}\definition{good}\example{Ngäna ttongo mer kollba de näddägan.}{I ate one of the good fish.}\sensenumber{2}\definition{holy}\example{Mer Anyke}{Holy Spirit}\sensenumber{3}\definition{to bless}\example{Yesu llɨg kälekäle de mer dägnegän.}{Jesus blessed the children.}\sensenumber{3}\definition{properly, nicely, well}\example{Ngäna ngämo toboll de mermer met däganeg.}{I sharpened my spears well.}\allomorph{Mer}\allomorph{mer}\subentry{\headword{mermer1}\pos{adv.}\definition{properly, nicely, well}}}
\entry{mer2}{\headword{mer2}\pos{n.}\sensenumber{3}\definition{type of spear}}
\entry{mer ag}{\headword{mer ag}\pos{p.}\sensenumber{3}\definition{good morning}}
\entry{mer awi}{\headword{mer awi}\pos{p.}\sensenumber{3}\definition{good evening}}
\entry{mer iddob}{\headword{mer iddob}\pos{p.}\sensenumber{3}\definition{good night}}
\entry{mer toto}{\headword{mer toto}\pos{p.}\sensenumber{3}\definition{good afternoon}}
\entry{Meragag}{\headword{Meragag}\pos{pn.}\sensenumber{3}\definition{Meragag (toponym)}}
\entry{Meramerall}{\headword{Meramerall}\pos{pn.}\sensenumber{3}\definition{Meramerall (toponym)}}
\entry{meresen}{\headword{meresen}\variant{sp. var. of}{meresin}}
\entry{meresin}{\headword{meresin}\pos{n.}\sensenumber{3}\definition{medicine}\etymology{from Englishmedicine}}
\entry{Meri}{\headword{Meri}\pos{pn.}\sensenumber{3}\definition{female personal name}}
\entry{Merian}{\headword{Merian}\pos{pn.}\sensenumber{3}\definition{female personal name}}
\entry{Merien}{\headword{Merien}\variant{sp. var. of}{Merian}}
\entry{Meroka}{\headword{Meroka}\pos{pn.}\sensenumber{3}\definition{female personal name}}
\entry{Merol}{\headword{Merol}\pos{pn.}\sensenumber{3}\definition{female personal name}}
\entry{Mesa}{\headword{Mesa}\pos{pn.}\sensenumber{3}\definition{male personal name}}
\entry{met}{\headword{met}\pos{A vt.}\sensenumber{3}\definition{to sharpen}\example{Ttongo giri de met dägageyo.}{They sharpened a knife.}}
\entry{metar}{\headword{metar}\pos{n.}\sensenumber{3}\definition{type of long yam with a white interior, white skin, and hairs}}
\entry{Metiyu}{\headword{Metiyu}\variant{sp. var. of}{Metyu}}
\entry{metmäll}{\headword{metmäll}\pos{S vt.}\sensenumber{3}\definition{to beat, flog, hit}}
\entry{Metyu}{\headword{Metyu}\pos{pn.}\sensenumber{3}\definition{male personal name}}
\entry{Mewato}{\headword{Mewato}\pos{pn.}\sensenumber{3}\definition{female personal name}}
\entry{meyang}{\headword{meyang}\pos{kin.}\sensenumber{1}\definition{uncle (one's father's younger brother)baba bo mänyan}\example{Meyang daeya tatu ma nallan.}{Uncle went to wash.}\sensenumber{2}\definition{brother-in-law (woman's husband's younger brother)}}
\entry{Mecklyn}{\headword{Mecklyn}\pos{pn.}\sensenumber{2}\definition{female personal name}}
\entry{midi}{\headword{midi}\pos{n.}\sensenumber{2}\definition{magic\textbackslash_type}}
\entry{midd}{\headword{midd}\pos{n.}\sensenumber{1}\definition{meat}\example{sɨmell midd}{pork}\sensenumber{2}\definition{meaning; core, essence}\example{eka bo midd}{a word's meaning}}
\entry{Migul}{\headword{Migul}\pos{pn.}\sensenumber{2}\definition{male personal name}}
\entry{mik}{\headword{mik}\pos{n.}\sensenumber{2}\definition{widow, widower}\example{Ttongo llɨgmeny mik da oblle damändän.}{A childless widow raised him.}\sensenumber{2}\definition{nonsingular form of mik}\subentry{\headword{mikmik1}\pos{n.}\definition{nonsingular form of mik}}}
\entry{miks}{\headword{miks}\pos{n.}\sensenumber{1}\definition{mix, mixture}\example{Ngäma eka da miks ingoll gogon.}{Our (excl.) language has become like a mix.}\sensenumber{2}\definition{to mix}\etymology{from Englishmix}}
\entry{mikutt}{\headword{mikutt}\pos{mod.}\sensenumber{2}\definition{angry, mad}\example{Obo mikutt a ttäntämang e gopnaeän.}{His anger made him hot.}\example{Da ttongo lla da bäne za de gämäll bogon, bongo mɨnyi mikutt ag.}{If someone steals your thing, you will be angry.}\example{Yae mikutt allan.}{Mother is mad.}\example{Bogo mikutt gogon.}{He got angry.}\sensenumber{1}\definition{angry, short-tempered}\example{Bogo era ddobae mikuttang lla dan.}{He's a very angry person.}\sensenumber{2}\definition{to hate}\example{Mikuttang eran.}{He hates him.}\sensenumber{2}\definition{calm}\example{Bogo era mikuttmeny lla dan.}{He's a calm fellow.}\etymology{mikutt + =ang}\subentry{\headword{mikuttang}\pos{mod.}\definition{angry, short-tempered}}\subentry{\headword{mikuttmeny}\pos{mod.}\definition{calm}}}
\entry{mimi}{\headword{mimi}\pos{n.}\sensenumber{2}\definition{pig (hunting word)}}
\entry{mimidämäll}{\headword{mimidämäll}\pos{n.}\sensenumber{2}\definition{type of snakeBätbät dan. Ddägnan ma dan. (It's black. It's edible.)}}
\entry{ministri}{\headword{ministri}\pos{n.}\sensenumber{2}\definition{ministry}\etymology{from Englishministry}}
\entry{Minkäm}{\headword{Minkäm}\pos{pn.}\sensenumber{2}\definition{Minkam (toponym)}}
\entry{Minkom}{\headword{Minkom}\variant{var. of}{Minkäm}}
\entry{Minkomminkomang}{\headword{Minkomminkomang}\pos{pn.}\sensenumber{2}\definition{Minkomminkomang (toponym)}}
\entry{Minong}{\headword{Minong}\pos{pn.}\sensenumber{2}\definition{male personal name}}
\entry{mintor}{\headword{mintor}\pos{n.}\sensenumber{2}\definition{type of tree with yellow flowers that bloom in June and July and a root is used as a kind of hockey stickObo ttoe a kädkäd ma dan a kaepnen ma dan kumyewang me. (Its bark is taken off and chewed when coughing.)}}
\entry{minggore manggo}{\headword{minggore manggo}\pos{n.}\sensenumber{2}\definition{type of tree}}
\entry{Mingkällbun}{\headword{Mingkällbun}\pos{pn.}\sensenumber{2}\definition{Mingkällbun (toponym)}}
\entry{miny}{\headword{miny}\pos{n.}\sensenumber{2}\definition{ant egg/larvae (small)}}
\entry{minyminy}{\headword{minyminy}\variant{var. of}{miny}}
\entry{mipdab}{\headword{mipdab}\pos{A vt.}\sensenumber{2}\definition{to accommodate; offer food to a visitor}\example{Ngänäm Malläm me mipdab dageyo.}{They accomodated me in Malam.}}
\entry{Miriang}{\headword{Miriang}\pos{pn.}\sensenumber{2}\definition{male personal name}}
\entry{miriwa}{\headword{miriwa}\pos{n.}\sensenumber{2}\definition{type of pandanusTtongo ulle käp tupiang mab da otät ma dan. (A pandanus with long fruit; it's edible.)}}
\entry{miroli}{\headword{miroli}\pos{n.}\sensenumber{2}\definition{black-capped loryLlo ik me giddollag pa dan. (It's a bird that lives in trees.)}}
\entry{misdae}{\headword{misdae}\pos{adv.}\sensenumber{2}\definition{just, simply}\example{Ngäna misdae ada dagerne.}{I was just staying like this.}\example{Ttomoe a mudan, misdae känyer ag.}{Don't you worry, just be quiet.}}
\entry{misde}{\headword{misde}\variant{fast speech var. of}{misdae}}
\entry{mise}{\headword{mise}\pos{n.}\sensenumber{2}\definition{common cicadabirdLlo dop alle ma gogowag pa dan. (It's a bird that builds nests using sticks.)}}
\entry{misituryam}{\headword{misituryam}\pos{n.}\sensenumber{2}\definition{type of long yam with a white or purple interior}}
\entry{mislok}{\headword{mislok}\pos{n.}\sensenumber{2}\definition{type of introduced bananaTtongo mer wup dan, obo käp a o me mer moko dan ako sana peyang yu ma dan. Ako mubine da ngasnges ma dan nge dädär alle. Be kire da ddone yu ma da. (It's a very good banana; when ripe, its fruit is tasty and is cooked with sago. Also, it's made into mubine with dry coconut. But when unripe, it's not cooked.)}}
\entry{mismis}{\headword{mismis}\pos{n.}\sensenumber{2}\definition{type of tree used for firewood}}
\entry{Misseilene}{\headword{Misseilene}\pos{pn.}\sensenumber{2}\definition{female personal name}}
\entry{mista}{\headword{mista}\pos{n.}\sensenumber{2}\definition{mister}\etymology{from Englishmister}}
\entry{mistae}{\headword{mistae}\variant{sp. var. of}{misdae}}
\entry{mit}{\headword{mit}\pos{n.}\sensenumber{1}\definition{base (of a plant)}\example{Ubi llo mit mi dagaeya.}{They were at the base of the tree.}\sensenumber{2}\definition{reason, sake}\example{Mit a gänyan.}{Here is the reason.}\example{Ttongo lla da säremang ma me daeya kuddäll e lla gäz mit me.}{A man was in prison because he beat someone to death.}\example{Iba mit me däpleon.}{He died for our (incl.) sake.}\sensenumber{3}\definition{origin, source}\example{Ge Ende eka da bo mit eka dan.}{This is the origin story of the Ende language.}\sensenumber{4}\definition{guilt}\sensenumber{4}\definition{to blame}\example{Ngänäm mitamitang dagän.}{They were blaming me.}\etymology{redup. of mit + =ang}\subentry{\headword{mitamitang}\pos{A vt.}\definition{to blame}}}
\entry{mitin}{\headword{mitin}\variant{sp. var. of}{miting}}
\entry{miting}{\headword{miting}\pos{n.}\sensenumber{4}\definition{meeting}\etymology{from Englishmeeting}}
\entry{mitmit1}{\headword{mitmit1}\pos{A vt.}\sensenumber{4}\definition{to miss, long for}\example{Lla da obom ddobae ddobae mitmit nägagallo.}{They were thinking of him when he left.}\example{Da medäda erowe gopällttänalle llɨg da mɨnyi mitmit bäganän ngänaeka peyang.}{‎If father set off somewhere, the children will miss him in tears.}}
\entry{mitmit2}{\headword{mitmit2}\pos{n.}\sensenumber{4}\definition{blunt axe}}
\entry{mitoem}{\headword{mitoem}\pos{n.}\sensenumber{4}\definition{type of mushroomWälläng me päddabag llo patt me, ddäddäg ma dan. (It grows on trees in the bush; it's edible.)}}
\entry{mittapa}{\headword{mittapa}\pos{n.}\sensenumber{4}\definition{extremities of face (i.e. temples, cheek, chin)}}
\entry{mittäpa}{\headword{mittäpa}\variant{var. of}{mittapa}}
\entry{mittin}{\headword{mittin}\variant{var. of}{miting}}
\entry{miwiwi}{\headword{miwiwi}\pos{n.}\sensenumber{1}\definition{dabbling duck (pacific black duck, grey teal)Walle me giddollag pa dan. (It's a bird that lives in water.)}\sensenumber{2}\definition{spotted whistling duck}}
\entry{mizi}{\headword{mizi}\pos{adv.}\sensenumber{2}\definition{usually}\example{Mizi mällaeyaba dan ge melem a.}{Usually, this is women's work.}}
\entry{Michael}{\headword{Michael}\pos{pn.}\sensenumber{2}\definition{male personal name}}
\entry{Michaelyn}{\headword{Michaelyn}\pos{pn.}\sensenumber{2}\definition{female personal name}}
\entry{Michelle}{\headword{Michelle}\pos{pn.}\sensenumber{2}\definition{female personal name}}
\entry{mɨka}{\headword{mɨka}\pos{n.}\sensenumber{2}\definition{type of introduced bananaBo käp a o me otät ma dan. Ako o da binzeg ma dan yu mi. Adawae kire da yu ma dan. Bo pätt a ulle tanong dan, ddone tupiae pänyanzag dan. (When ripe, its fruit is eaten. It's also heated on the fire. Similarly, when unripe, it's cooked. Its trunk is a bit big; it doesn't grow that tall.)}}
\entry{mɨllɨng}{\headword{mɨllɨng}\variant{sp. var. of}{mälläng}}
\entry{mɨnyi}{\headword{mɨnyi}\pos{TAM ptcl.}\sensenumber{2}\definition{future tense particle}\example{Mɨnyi mer abal bogon.}{It will be very good.}}
\entry{mo}{\headword{mo}\pos{n.}\sensenumber{1}\definition{step, stair(s)Ma ik e kälängkäl ma. (For ascending into the house.)}\example{Medäda mo toko alle ikop eran llɨgda bom.}{The father is looking at his son from the top of the stairs.}\example{Bogo gompedägän mo me.}{She tripped on the stairs.}\sensenumber{2}\definition{bridgeWalle opap ma. (For crossing the water.)}\example{Ubi tätäm ttongo mo sisor de daulliyu.}{Yesterday, they crossed over a new bridge.}\sensenumber{2}\definition{staircase}\subentry{\headword{mo katrekatre}\pos{n.}\definition{staircase}}}
\entry{mobera}{\headword{mobera}\pos{n.}\sensenumber{2}\definition{outriggerGall bo llokott. (A canoe's support.)}}
\entry{Moed}{\headword{Moed}\pos{pn.}\sensenumber{2}\definition{Moed (toponym)}}
\entry{Moem}{\headword{Moem}\pos{pn.}\sensenumber{2}\definition{Moem (toponym)}}
\entry{moep}{\headword{moep}\pos{n.}\sensenumber{2}\definition{charcoal dust}\sensenumber{1}\definition{matured}\example{Bogo moepang agan.}{He is all grown up.}\sensenumber{2}\definition{charcoal-stained}\etymology{moep + =ang}\subentry{\headword{moepang}\pos{mod.}\definition{matured}}}
\entry{moepo}{\headword{moepo}\pos{n.}\sensenumber{2}\definition{type of tree that grows in the bush with white flowers, inedible red fruit, and poisonous seeds and bark; an indicator that the soil is fertile and good for making a garden}}
\entry{moepotatae}{\headword{moepotatae}\pos{n.}\sensenumber{2}\definition{type of tree}}
\entry{mok}{\headword{mok}\pos{n.}\sensenumber{2}\definition{friarbird (noisy, little, helmeted)Ap me ekawang pa dan. (It's a bird that sings in the grassland.)}}
\entry{moko}{\headword{moko}\pos{n.}\sensenumber{1}\definition{desire, want; love}\example{Ngäna market e ibi allan, bäne moko da ewe dan?}{I am going to the market; do you want anything (lit. does your desire exist for anything)?}\example{Endan bäne moko da ngäna bablle banges}{What do you want me to do for you?}\example{Ngämi moko gogeya.}{We (excl.) two fell in love.}\example{Ngäna up i moko allan.}{I like bananas.}\sensenumber{2}\definition{taste, flavor}\example{Solt bo moko da ddone budabän.}{The salt won't lose its taste.}\example{Sana gudne da koepang moko allan.}{The old sago tastes sour.}\example{Ngatengate da biratt a ddobe moko dan.}{Roasted possum is very tasty.}\sensenumber{1}\definition{desirable, enjoyable; preferred, favorite}\example{Ngäna mokoang eran ngämo melem de.}{I love my job.}\example{Ngäna bam mokoang nallan.}{I love you.}\example{Ngäna ngämo mokowang bandra de llɨtnen eran.}{I am singing my favorite song.}\sensenumber{2}\definition{tasty, delicious, flavorful; sweet}\example{Ge wätät a mokoang dan.}{This food is tasty.}\etymology{moko + =ang}\subentry{\headword{mokoang}\pos{mod.}\definition{desirable, enjoyable; preferred, favorite}}}
\entry{mokoll}{\headword{mokoll}\pos{n.}\sensenumber{2}\definition{type of tree that grows in the bush with thick bark, white flowers, and small green fruit}}
\entry{mokon}{\headword{mokon}\pos{S vt.}\sensenumber{2}\definition{to anoint}\example{Bogo itrellang lla de owel alle domokonmällaemneyo.}{He was anointing the sick men with oil.}}
\entry{moksir}{\headword{moksir}\pos{n.}\sensenumber{2}\definition{type of tree}}
\entry{molemoleg}{\headword{molemoleg}\pos{n.}\sensenumber{2}\definition{type of grubAp me a wälläng me, malla ddäddäg ma dan. (In the grassland and the bush; they are not edible.)}}
\entry{Moli}{\headword{Moli}\pos{pn.}\sensenumber{2}\definition{name of a male ancestor (brother of Lamlam)}}
\entry{moll}{\headword{moll}\pos{n.}\sensenumber{2}\definition{type of tree that grows around Malam with white flowersAwe me llo dan. Ine ttänttämang e ttam de säpall ma dag tatu we itrellang me. Ako oil ngasnen ma dan pätt me nyänan e a molle dumeny e itrellang me. (It's a tree in the savanna. The leaves are put in boiling water to bathe with when sick. It's also used to make an oil to put on the body and inhale when sick.)}}
\entry{molle}{\headword{molle}\pos{n.}\sensenumber{2}\definition{scent, odor, smell}\example{kuddäll molle}{rotten smell}\example{Llo popo molle da mer dan.}{The scent of the flowers is good.}\example{Ngäna otät molle de dändär eran.}{I am smelling food.}\sensenumber{2}\definition{smelling}\sensenumber{2}\definition{scented}\example{mer molleang owel}{fragrant oil}\sensenumber{2}\definition{odorless}\sensenumber{2}\definition{to smell, perceive a smell}\example{Ngäna sɨmell de gänyaollemae molle deyandärneg.}{I smelled pigs coming this way.}\sensenumber{2}\definition{to smell, take a whiff}\sensenumber{2}\definition{to sniff, smell}\example{Däräng a llɨg di mollemolle nägagan.}{The dog sniffed the boy.}\etymology{molle + =ang}\subentry{\headword{mollong}\pos{mod.}\definition{smelling}}\subentry{\headword{molleang}\pos{mod.}\definition{scented}}\subentry{\headword{mollemeny}\pos{mod.}\definition{odorless}}\subentry{\headword{molle dändär}\pos{S vt.}\definition{to smell, perceive a smell}}\subentry{\headword{molle dungg}\pos{S vt.}\definition{to smell, take a whiff}}\subentry{\headword{mollemolle}\pos{A vt.}\definition{to sniff, smell}}}
\entry{mollok}{\headword{mollok}\pos{n.}\sensenumber{2}\definition{type of introduced bananaTupi tanong pänyanzag dan. Obo däg a yuwog dag. Käp a o me obo yu ma dag, ada kire da ade yu ma dag. Ako o da wa kire da ine ttänttämang e säpall ma dag otänan e. (It grows a little tall. Its bunches are plentiful. When ripe, its fruit are cooked; same when unripe. Also, ripe or unripe, they're put in boiling water to be eaten.)}}
\entry{momana ma}{\headword{momana ma}\pos{n.}\sensenumber{2}\definition{hideout made of leaves for hunting cassowaries}\example{Ddob llaeyabaene momana ma alle dirom gädnanatt a kemibi dag.}{Many people kill cassowaries by using a hideout.}}
\entry{Mome}{\headword{Mome}\pos{pn.}\sensenumber{2}\definition{female personal name}}
\entry{momea gäl}{\headword{momea gäl}\pos{n.}\sensenumber{2}\definition{type of tree}}
\entry{Momeya}{\headword{Momeya}\pos{pn.}\sensenumber{2}\definition{Momeya (toponym)}}
\entry{momolltätän}{\headword{momolltätän}\pos{n.}\sensenumber{2}\definition{type of introduced bananaDdone tupi pänyanzag dan, adawae käp a obo däg me ddone yuwog dag, ako pätt a kälsäre dan. Käp a obo tutupi dag, a o me binzenen ma dag, ako kire da yu ma dag. (It doesn't grow tall; similarly, it doesn't have many fruit and its trunk is small. Its fruit are long, and when ripe, they're heated, and when unripe, they're cooked.)}}
\entry{mompara}{\headword{mompara}\pos{n.}\sensenumber{2}\definition{type of tree that grows near swamps and creeks with white flowers and yellow and brown fruit that ripen in April and May and are eaten by deer}}
\entry{mompel}{\headword{mompel}\pos{n.}\sensenumber{2}\definition{aibika}}
\entry{Mompelang}{\headword{Mompelang}\pos{pn.}\sensenumber{2}\definition{Mompelang (toponym)}}
\entry{mondo}{\headword{mondo}\pos{n.}\sensenumber{2}\definition{type of tree that grows in the bush with green or purple fruits that animals eat}}
\entry{mondre}{\headword{mondre}\pos{n.}\sensenumber{2}\definition{gardening, garden work}\example{Bogo mondre bognän.}{She will be gardening.}\sensenumber{2}\definition{gardener, having a green thumb}\example{Ddobae mondrog lla deya.}{He was a great gardener.}\etymology{mondre + =ang}\subentry{\headword{mondreag}\pos{mod.}\definition{gardener, having a green thumb}}}
\entry{mondremondre}{\headword{mondremondre}\pos{A vi. \textbackslash& vt.}\sensenumber{2}\definition{to move}\example{Bäne ttang de ade mɨnyi mondremondre bägagän.}{Your hand will also be moving (lit. movement will also get your hand).}\example{Däbe lla da mondremondre gogon.}{That man moved.}}
\entry{mondrog}{\headword{mondrog}\variant{fast speech var. of}{mondreag}}
\entry{mopmop}{\headword{mopmop}\pos{n.}\sensenumber{2}\definition{type of tree that grows in the grassland (\textbackslashtextasciitilde3 m) with white flowers and red fruit that are eaten to treat cough}}
\entry{mormor}{\headword{mormor}\pos{n.}\sensenumber{2}\definition{type of small herb with white flowers and ovate leaves}}
\entry{mos}{\headword{mos}\pos{n.}\sensenumber{2}\definition{type of goanna}}
\entry{Mosbi}{\headword{Mosbi}\variant{fast speech var. of}{Pot Mosbi}}
\entry{mosen}{\headword{mosen}\pos{kin.}\sensenumber{1}\definition{older sibling of the same-sex (man's older brother or woman's older sister)Zaze me ngattong zegatt. (The firstborn in a generation.)}\example{Paul ngämo mosen dan.}{Paul is my older brother (said by a male speaker).}\sensenumber{2}\definition{eldest, firstborn}\example{Ngäna mosen llɨg dan.}{I'm the firstborn.}\sensenumber{1}\definition{older siblings}\sensenumber{2}\definition{elders, seniors}\example{Ngäma kame dan, be mosenmosen lla da mɨnyi ngämim umllang beyageyo.}{We (excl.) don't know, but our elders will tell us.}\subentry{\headword{mosenmosen}\pos{kin.}\definition{older siblings}}}
\entry{Moses}{\headword{Moses}\pos{pn.}\sensenumber{2}\definition{male personal name}}
\entry{Mospi}{\headword{Mospi}\variant{var. of}{Mosbi}}
\entry{motom}{\headword{motom}\pos{n.}\sensenumber{2}\definition{type of small taro}}
\entry{Motu}{\headword{Motu}\pos{pn.}\sensenumber{2}\definition{Hiri Motu language}}
\entry{Motu loanword}{\headword{Motu loanword}\sensenumber{2}\definition{loanword}\allomorph{lawana}}
\entry{Moyabag}{\headword{Moyabag}\pos{pn.}\sensenumber{2}\definition{male personal name}}
\entry{Moyäm}{\headword{Moyäm}\pos{n.}\sensenumber{2}\definition{Moyäm (toponym)}}
\entry{moza}{\headword{moza}\pos{n.}\sensenumber{2}\definition{type of venomous snakeDompak ingoll dan be kälsre dan, malla ddäddäg ma dan. (It's like an eel but small; it's not edible.)}}
\entry{mozaya}{\headword{mozaya}\pos{n.}\sensenumber{2}\definition{type of large, edible fish found in the swamp}}
\entry{Mozbi}{\headword{Mozbi}\variant{sp. var. of}{Mosbi}}
\entry{mu}{\headword{mu}\pos{n.}\sensenumber{1}\definition{payment, price, value}\example{skul mu}{school fees}\example{Bablle mɨnyi mu banttog.}{I will pay you (lit. give you payment).}\example{300 kina mu me dägageyo däbe käza ttoe de.}{That crocodile skin sold for 300 kina.}\sensenumber{2}\definition{response, reply, answer; repayment, revenge}\example{Obo moko da mu wi dan.}{He wants to respond.}\example{Polis a ngämo gullbe de namllamallo, ngämo mu di nangesallo.}{The police got my husband; they carried out my revenge.}\sensenumber{3}\definition{valuable}\example{Obo ttoe a ddobae mu dan.}{Its skin is very valuable.}\sensenumber{3}\definition{in debt}\example{Ngämo ttongo ttang lläpät a muminy dan.}{I owe five. / I am five in debt.}\etymology{mu + =meny}\subentry{\headword{muminy}\pos{mod.}\definition{in debt}}}
\entry{mubine}{\headword{mubine}\pos{n.}\sensenumber{3}\definition{dish consisting of food (such as banana, yam, or sweet potato) with coconut cream. up mubine, mätta mubine, nai mubine}}
\entry{mudan}{\headword{mudan}\pos{cop.}\sensenumber{3}\definition{prohibitive copula (present singular form)}\example{Ngänäm kollmäl a mudan.}{[You] don't follow me.}\example{Mudan lel a.}{[You] don't be afraid.}\sensenumber{3}\definition{present plural form of mudan}\example{Abo källkae ibi da mudag.}{[You all] never go there again.}\example{Bibi eka da mudag.}{You all, don't talk.}\sensenumber{3}\definition{present dual form of mudan}\example{Bibi komllaebe gäddnan a mudageyo.}{You two, don't fight.}\allomorph{muda}\allomorph{mada}\subentry{\headword{mudag}\pos{cop.}\definition{present plural form of mudan}}\subentry{\headword{mudageyo}\pos{cop.}\definition{present dual form of mudan}}}
\entry{mugbusu}{\headword{mugbusu}\pos{n.}\sensenumber{3}\definition{type of taro}}
\entry{Mugi}{\headword{Mugi}\pos{pn.}\sensenumber{3}\definition{male personal name}}
\entry{Muidebag}{\headword{Muidebag}\pos{pn.}\sensenumber{3}\definition{Muidebag (toponym)}}
\entry{Mul}{\headword{Mul}\pos{pn.}\sensenumber{3}\definition{Mul (toponym)}}
\entry{mulkul}{\headword{mulkul}\pos{n.}\sensenumber{3}\definition{brain}}
\entry{mulmul}{\headword{mulmul}\pos{n.}\sensenumber{3}\definition{rite of passage}}
\entry{Mull}{\headword{Mull}\pos{pn.}\sensenumber{3}\definition{Mull (toponym)}}
\entry{mullae}{\headword{mullae}\pos{mod.}\sensenumber{1}\definition{enough}\example{Iba ge kollba da mullae dag.}{We (incl.) have enough fish.}\sensenumber{2}\definition{able, can, be allowed}\example{Llamda da bälle gazenma da ddone mullae gogon.}{The old man was unable to escape.}\example{Bäne ma me ngäna mullae dan bodmen?}{Can I stay at your house?}\sensenumber{2}\definition{nonsingular form of mullae}\example{Tämamae ttoen a dedme mullamullae dag.}{There's enough of everything there.}\subentry{\headword{mullaemullae1}\pos{mod.}\definition{nonsingular form of mullae}}}
\entry{mullaemullae2}{\headword{mullaemullae2}\pos{quant.}\sensenumber{2}\definition{every}\example{Ngäna pazi mullamullae mänmän de ngällabnanangmeae anggan.}{I pick up the girls every year.}}
\entry{mullamulla}{\headword{mullamulla}\pos{n.}\sensenumber{2}\definition{medicine}\etymology{from Hiri Motumuramura}}
\entry{mullamullae}{\headword{mullamullae}\variant{fast speech var. of}{mullaemullae1}}
\entry{mulldae}{\headword{mulldae}\variant{var. of}{mullae}}
\entry{Munu}{\headword{Munu}\pos{pn.}\sensenumber{2}\definition{male personal name}}
\entry{mupni}{\headword{mupni}\pos{n.}\sensenumber{2}\definition{type of cultivated mango tree with fruit with yellow skin and a white interior that is juiced}}
\entry{Mur}{\headword{Mur}\pos{pn.}\sensenumber{2}\definition{Mur (toponym)}}
\entry{mur}{\headword{mur}\pos{A vt.}\sensenumber{2}\definition{to clean, tidy}}
\entry{muramura}{\headword{muramura}\variant{sp. var. of}{mullamulla}}
\entry{muro}{\headword{muro}\pos{n.}\sensenumber{2}\definition{magic}}
\entry{Musato}{\headword{Musato}\pos{pn.}\sensenumber{2}\definition{female personal name}}
\entry{mutae}{\headword{mutae}\pos{n.}\sensenumber{2}\definition{type of yam with a yellow interior, no thorns, and a vine that grows clockwise}}
\entry{Muti}{\headword{Muti}\variant{dial. var. of}{Motu}}
\entry{Mutu}{\headword{Mutu}\variant{sp. var. of}{Motu}}
\entry{mutt}{\headword{mutt}\pos{n.}\sensenumber{2}\definition{river sourceWalle pot. (Endpoint of a river.)}\example{Ngäna walle mutt i ibi allan.}{I am going upstream.}}
\entry{muttbul}{\headword{muttbul}\pos{mod.}\sensenumber{2}\definition{quiet}}
\entry{muttmutt}{\headword{muttmutt}\pos{n.}\sensenumber{2}\definition{type of grub that lives in the ground and eats crops}}
\entry{muu}{\headword{muu}\variant{sp. var. of}{mu}}
\entry{Muyabag}{\headword{Muyabag}\pos{pn.}\sensenumber{2}\definition{male personal name}}
\lettersection{N}
\entry{nadum}{\headword{nadum}\pos{n.}\sensenumber{2}\definition{namesake}\example{Ngämo nadum a daden.}{I have a namesake.}}
\entry{nae}{\headword{nae}\pos{n.}\sensenumber{2}\definition{sweet potato}}
\entry{naeka}{\headword{naeka}\variant{fast speech var. of}{ngänaeka}}
\entry{Naemäll}{\headword{Naemäll}\pos{pn.}\sensenumber{2}\definition{female personal name}}
\entry{naen}{\headword{naen}\pos{num.}\sensenumber{2}\definition{nine (English numeral; also general)}\etymology{from Englishnine}}
\entry{naenti}{\headword{naenti}\pos{num.}\sensenumber{2}\definition{ninety}\etymology{from Englishninety}}
\entry{naentin}{\headword{naentin}\pos{num.}\sensenumber{2}\definition{nineteen (English numeral)}\etymology{from Englishnineteen}}
\entry{nag}{\headword{nag}\pos{n.}\sensenumber{2}\definition{friend}\sensenumber{2}\definition{friendship}\example{Oba nag ttoen a dädme llätt gogon.}{Their friendship ended there.}\sensenumber{2}\definition{nonsingular form of nag}\subentry{\headword{nag ttoen}\pos{n.}\definition{friendship}}\subentry{\headword{nagnag}\pos{n.}\definition{nonsingular form of nag}}}
\entry{Nagab}{\headword{Nagab}\pos{pn.}\sensenumber{2}\definition{male personal name}}
\entry{Nagat}{\headword{Nagat}\pos{pn.}\sensenumber{2}\definition{male personal name}}
\entry{Nageg}{\headword{Nageg}\pos{pn.}\sensenumber{2}\definition{male personal name}}
\entry{naigae}{\headword{naigae}\pos{mod.}\sensenumber{2}\definition{south}\example{Ngäna naigae pallall e ibi allan.}{I am heading south.}}
\entry{nain}{\headword{nain}\variant{sp. var. of}{naen}}
\entry{nainti}{\headword{nainti}\variant{sp. var. of}{naenti}}
\entry{naintin}{\headword{naintin}\variant{sp. var. of}{naentin}}
\entry{Nakaku}{\headword{Nakaku}\pos{pn.}\sensenumber{2}\definition{Nakaku (toponym)}}
\entry{Naklae}{\headword{Naklae}\pos{pn.}\sensenumber{2}\definition{male personal name}}
\entry{Nakllae}{\headword{Nakllae}\variant{var. of}{Naklae}}
\entry{Nakuri}{\headword{Nakuri}\pos{pn.}\sensenumber{2}\definition{male personal name}}
\entry{Nalon}{\headword{Nalon}\pos{pn.}\sensenumber{2}\definition{male personal name}}
\entry{nallib}{\headword{nallib}\pos{n.}\sensenumber{2}\definition{type of spear}}
\entry{Nama}{\headword{Nama}\pos{pn.}\sensenumber{2}\definition{male personal name}}
\entry{Namaya}{\headword{Namaya}\pos{pn.}\sensenumber{2}\definition{female personal name}}
\entry{nane1}{\headword{nane1}\pos{kin.}\sensenumber{2}\definition{aunt (one's parent's younger sister)Yae bo o baba bo mänyan män. Baba bo peyang gulag zeg, llɨgda ba nane dan. (Mother or father's younger sister. Born together with father; aunt of his children.)}\example{Nensi bi nane alle känaebag mɨnyi tudi ma beyareyo}{Nancy will go fishing tomorrow with her aunt.}}
\entry{nane2}{\headword{nane2}\pos{S vt.}\sensenumber{2}\definition{to drink}\example{Bogo nane eran.}{He is drinking.}\example{Ngäna konkonang ine de bäna.}{I will drink the beer.}\example{Ngämi dänaeyaemnalla.}{We (excl.) were drinking.}\allomorph{na}\allomorph{ne}}
\entry{Nancy}{\headword{Nancy}\variant{sp. var. of}{Nensi}}
\entry{Nanggon}{\headword{Nanggon}\pos{pn.}\sensenumber{2}\definition{male personal name}}
\entry{Naomi}{\headword{Naomi}\pos{pn.}\sensenumber{2}\definition{female personal name}}
\entry{Narma}{\headword{Narma}\pos{pn.}\sensenumber{2}\definition{male personal name}}
\entry{Nasma}{\headword{Nasma}\pos{pn.}\sensenumber{2}\definition{male personal name}}
\entry{Nazaret}{\headword{Nazaret}\pos{pn.}\sensenumber{2}\definition{Nazareth}}
\entry{Nägäm}{\headword{Nägäm}\pos{pn.}\sensenumber{2}\definition{male personal name}}
\entry{näkäp}{\headword{näkäp}\pos{n.}\sensenumber{2}\definition{mind, mindset, consciousness; thoughts}\example{Ngämo näkäp e gongttägän ada ngaska zime iddob amänong agan.}{It came to my mind that maybe it's already past midnight.}\example{Oblle mer näkäp a ddone ngänttägmäll allan.}{He is not having good thoughts.}\sensenumber{2}\definition{thought, knowledge}\example{Obo näkäpngon tämamae ai da erageya audaebnegan.}{He lost all his thoughts that were good.}\example{Obo näkäpngon alle eka panypeny eran.}{She is speaking with wisdom.}\sensenumber{2}\definition{smart}\sensenumber{2}\definition{foolish}\sensenumber{2}\definition{to change one's mind}\example{Ngäna mäse karama we ibi we daeya, be näkäp de näpänaeyan wälläng e.}{I was supposed to go to the canoe place, but I changed my mind to go to the bush.}\sensenumber{2}\definition{to worry}\example{Bogo näkäp ttomoenen me giddollnen allan.}{He is living worrying.}\etymology{näkäp + =ang}\subentry{\headword{näkäpngon}\pos{n.}\definition{thought, knowledge}}\subentry{\headword{näkäpang}\pos{mod.}\definition{smart}}\subentry{\headword{näkäpmeny}\pos{mod.}\definition{foolish}}\subentry{\headword{näkäp de pänae}\pos{p.}\definition{to change one's mind}}\subentry{\headword{näkäp ttomoe}\pos{S vi.}\definition{to worry}}}
\entry{nälan}{\headword{nälan}\pos{adv.}\sensenumber{2}\definition{accidentally}\example{Bogo ankom de näbäddan nälan ttang alle.}{He accidentally killed the ant with his hand.}}
\entry{nänäm}{\headword{nänäm}\pos{n.}\sensenumber{2}\definition{diagonal checkerboard weaving pattern}}
\entry{Nänga}{\headword{Nänga}\pos{pn.}\sensenumber{2}\definition{female personal name}}
\entry{nängga}{\headword{nängga}\pos{n.}\sensenumber{2}\definition{type of pandanus with bunches of fruit with strong shells; they fall one at a timeAp me päddabag dan, obo ku da wätät ma dan. (It grows in the grassland; its nut is edible.)}}
\entry{nätnät}{\headword{nätnät}\pos{n.}\sensenumber{2}\definition{insects}}
\entry{Nedlyn}{\headword{Nedlyn}\pos{pn.}\sensenumber{2}\definition{female personal name}}
\entry{Nensi}{\headword{Nensi}\pos{pn.}\sensenumber{2}\definition{female personal name}}
\entry{nes}{\headword{nes}\pos{n.}\sensenumber{2}\definition{nurse}\etymology{from Englishnurse}}
\entry{net}{\headword{net}\pos{n.}\sensenumber{2}\definition{net}\example{Ngäna ttongo net de dirängän ine watt.}{I lifted a net out from the water.}\etymology{from Englishnet}}
\entry{nett}{\headword{nett}\variant{var. of}{net}}
\entry{nidol}{\headword{nidol}\pos{n.}\sensenumber{2}\definition{needle}\etymology{from Englishneedle}}
\entry{nikbin}{\headword{nikbin}\pos{n.}\sensenumber{2}\definition{nickname}\etymology{partial calque of Englishnickname}}
\entry{Niki}{\headword{Niki}\pos{pn.}\sensenumber{2}\definition{male personal name}}
\entry{Nikol}{\headword{Nikol}\pos{pn.}\sensenumber{2}\definition{female personal name}}
\entry{nil}{\headword{nil}\pos{n.}\sensenumber{2}\definition{nail (made of metal)}\etymology{from Englishnail}}
\entry{ninety}{\headword{ninety}\variant{sp. var. of}{naenti}}
\entry{nineyem}{\headword{nineyem}\pos{n.}\sensenumber{2}\definition{whisper}\example{Angde llamda da togolag dagirnän, llayabaene eka nineyem de dandärän.}{While the old man was hiding, he heard whispers from people.}}
\entry{Niniab}{\headword{Niniab}\pos{pn.}\sensenumber{2}\definition{male personal name}}
\entry{Nicky}{\headword{Nicky}\variant{sp. var. of}{Niki}}
\entry{Nicole}{\headword{Nicole}\variant{sp. var. of}{Nikol}}
\entry{Nixon}{\headword{Nixon}\pos{pn.}\sensenumber{2}\definition{male personal name}}
\entry{Noar}{\headword{Noar}\pos{pn.}\sensenumber{2}\definition{female personal name}}
\entry{Nogat}{\headword{Nogat}\pos{pn.}\sensenumber{2}\definition{male personal name}}
\entry{Nolin}{\headword{Nolin}\pos{pn.}\sensenumber{2}\definition{female personal name}}
\entry{nongg}{\headword{nongg}\pos{n.}\sensenumber{2}\definition{type of geckoBänz a allko ddägnanang pipllo. (A lizard that eats mosquitoes and flies.)}\example{Ngäma ma me nongg a kemibi abal dag.}{In my house there are many geckos.}}
\entry{nongonongor}{\headword{nongonongor}\pos{mod.}\sensenumber{2}\definition{itchy}\example{Mälläng a nongonongorang gogon, käsre ansi gogon.}{His nose got itchy; then he sneezed.}}
\entry{nop mu}{\headword{nop mu}\pos{n.}\sensenumber{2}\definition{inter-tribe paymentBongo ngämlle nanttog, ngäna abo bäne lläg abira källkae. (You will give me something; then, later, I will give it to your children.)}}
\entry{Nope}{\headword{Nope}\pos{pn.}\sensenumber{2}\definition{male personal name}}
\entry{nopmäg}{\headword{nopmäg}\pos{kin.}\sensenumber{2}\definition{one's mother's exchange sister}}
\entry{noponda}{\headword{noponda}\pos{kin.}\sensenumber{2}\definition{one's father's exchange brother (still owes him)Lla tän me män taempägag. (The man in a tribe showing a girl for marriage.)}}
\entry{nora}{\headword{nora}\pos{n.}\sensenumber{2}\definition{type of big introduced tree with red flowers that attract birds}}
\entry{Norma}{\headword{Norma}\pos{pn.}\sensenumber{2}\definition{female personal name}}
\entry{noto}{\headword{noto}\variant{dial. var. of}{Taeme loanword}}
\entry{Nowar}{\headword{Nowar}\variant{sp. var. of}{Noar}}
\entry{Nuam}{\headword{Nuam}\pos{pn.}\sensenumber{2}\definition{female personal name}}
\entry{Nugini}{\headword{Nugini}\pos{pn.}\sensenumber{2}\definition{New Guinea}\etymology{from Tok PisinHiri MotuNiugini}}
\entry{Nuopin}{\headword{Nuopin}\pos{pn.}\sensenumber{2}\definition{female personal name}}
\entry{nurae}{\headword{nurae}\pos{n.}\sensenumber{2}\definition{type of large birdDdäddäg ma pa ulle da. (It's a big, edible bird.)}}
\lettersection{Ng ng}
\entry{nga}{\headword{nga}\pos{TAM ptcl.}\sensenumber{2}\definition{marks immediate or near future}\example{Nga yu ttätta de däbe gänyaolle iweny.}{Now [you] bring that piece of burned wood over here.}\example{Nga bibi ddone ngänaem amalla?}{Are you all not understanding?}}
\entry{ngaeka}{\headword{ngaeka}\variant{fast speech var. of}{ngänaeka}}
\entry{ngak}{\headword{ngak}\variant{var. of}{ngok}}
\entry{ngal}{\headword{ngal}\pos{A vi.}\sensenumber{2}\definition{to waddle, shuffle}}
\entry{ngalbongalboe}{\headword{ngalbongalboe}\pos{adv.}\sensenumber{2}\definition{on the side}\example{Bogo ngalbongalboe bognegnän.}{He will be on the side.}}
\entry{ngalen}{\headword{ngalen}\pos{n.}\sensenumber{1}\definition{way, habit, manner, custom}\example{Gabma yaba ngalen a gongttägän ngäma pate.}{White people's customs came to us (excl.).}\sensenumber{2}\definition{thing}\example{Golläntmenyän tämamae ngalen de.}{He told him everything.}}
\entry{ngallmeny}{\headword{ngallmeny}\pos{S vt.}\sensenumber{2}\definition{to advise}\example{Llɨg ngallmeny a era kälsre meae dan mer a.}{The best time to advise children is in childhood.}}
\entry{ngallngall}{\headword{ngallngall}\pos{n.}\sensenumber{2}\definition{catbird (spotted, ochre-breasted)Wälläng me ekawang pa dan. (It's a bird that sings in the bush.)}}
\entry{ngam}{\headword{ngam}\pos{n.}\sensenumber{1}\definition{breast}\example{llɨg ngam sinenang mälla}{breastfeeding woman}\sensenumber{2}\definition{nine (lit. breast; body counting numeral)}\sensenumber{2}\definition{nipple, teat}\sensenumber{2}\definition{breast}\subentry{\headword{ngam indäb}\pos{n.}\definition{nipple, teat}}\subentry{\headword{ngam koll}\pos{n.}\definition{breast}}}
\entry{ngamtep}{\headword{ngamtep}\pos{n.}\sensenumber{2}\definition{type of mushroomAp me päddabag, ddäddäg ma dan. (It grows in the grassland; it's edible.)}}
\entry{nganae}{\headword{nganae}\pos{S vt.}\sensenumber{1}\definition{to coil, go around}\example{Mätta da dade de nganae eran.}{The yam is coiling around the yam stick.}\sensenumber{2}\definition{to spin, rotate}\example{Ekaklle da nganaenen allan.}{The earth is spinning.}\sensenumber{2}\definition{wrapping around}\subentry{\headword{nganaenganae}\pos{mod.}\definition{wrapping around}}}
\entry{nganzig}{\headword{nganzig}\pos{S vt.}\sensenumber{2}\definition{to pass, overtake}\example{Ddia da angde inu kuddäll gogon, kottllam a deyanzigän.}{While Deer was dead sleep, Tortoise passed right by him.}\allomorph{ngzig}\allomorph{ngziminy}\allomorph{nganziminy}\allomorph{nzig}}
\entry{ngange}{\headword{ngange}\pos{S vt.}\sensenumber{1}\definition{to communicate, deliver a message}\example{ngangema ttoen}{method of communication}\example{Lla da mɨnyi utt alle eka de bängaeyo.}{Men will deliver the message with the conch shell.}\sensenumber{1}\definition{messenger}\example{Ngangeyang abo umllang dägaeballo aeya kuddäll agan.}{Then the messengers will say who died.}\allomorph{nga}\allomorph{nge}\etymology{ngange + =ang}\subentry{\headword{ngangeyang}\pos{n.}\definition{messenger}}}
\entry{ngangem}{\headword{ngangem}\pos{S vt.}\sensenumber{1}\definition{to adopt}\example{Ngäna ge män de obaene de dangem.}{I adopted this girl from them.}\allomorph{ngem}}
\entry{ngangleb}{\headword{ngangleb}\pos{S vt.}\sensenumber{1}\definition{to look for, search for}\example{Lla da mälla de dängalbänän.}{The man searched for the woman.}\allomorph{ngalb}\allomorph{ngelb}\allomorph{ngalbän}}
\entry{Ngao}{\headword{Ngao}\pos{pn.}\sensenumber{1}\definition{Ngao (toponym)}}
\entry{ngarängg}{\headword{ngarängg}\pos{S vt.}\sensenumber{1}\definition{to encounter, meet, run into}\example{Da ngäna lla de bangerängg nyongo me, mɨnyi Ende eka walle eka bäntameny.}{If I encounter someone on the street, I will speak in Ende to them.}\example{Gänya ngata me ddädddäg de ddone dangerängg.}{I didn't encounter any animals at that spot.}\allomorph{ngermäll}\allomorph{ngerängg}}
\entry{ngasa}{\headword{ngasa}\variant{fast speech var. of}{ngasekäma}}
\entry{ngase}{\headword{ngase}\pos{TAM ptcl.}\sensenumber{1}\definition{hortative/optative particle}\example{Ngase ibi bobäll bem de apte menae e baupeya.}{Let us go over to the other side of the sea.}\example{Ngase Ende eka de mɨnyi mokowang bägawän.}{Let them love to speak Ende.}}
\entry{ngaseka}{\headword{ngaseka}\variant{fast speech var. of}{ngasekäma}}
\entry{ngasekäma}{\headword{ngasekäma}\pos{TAM ptcl.}\sensenumber{1}\definition{potential marker; may, could, might}\example{Erame dan ngasekäma?}{Where could they be?}\example{Da ttongo lla da bäne baba bälle 1000 mani bonttogän, ende bäne baba ngasekäma bällädän?}{If someone gave your father 1000 kina, what might he buy?}\sensenumber{2}\definition{maybe}\example{Mɨnyi ngaska llame duwem bognegnän oba peyang.}{Maybe they will eat together with them.}}
\entry{ngaska}{\headword{ngaska}\variant{fast speech var. of}{ngasekäma}}
\entry{ngaskäma}{\headword{ngaskäma}\variant{fast speech var. of}{ngasekäma}}
\entry{ngaskma}{\headword{ngaskma}\variant{fast speech var. of}{ngasekäma}}
\entry{ngasnges}{\headword{ngasnges}\pos{S vt.}\sensenumber{1}\definition{to do}\example{Ende bongo ngasnges eralle ttongo lla bo pate, bongo ngasnges eralle God bo pate.}{What you do to another person, you do to God.}\example{Bongo bäne kumuddäga melem de nangesneg.}{Do your three chores.}\sensenumber{2}\definition{to make}\example{Ibi ibra mɨnyi ttongo sisor bikwem de bangeseya.}{We (excl.) will make a new fireplace for us.}\example{Masar ine kube de ngämo nangesan.}{Grandfather made my water bucket.}\sensenumber{3}\definition{to happen}\example{Ge ttoen a gongesän ngämo patme.}{This thing happened to me.}\allomorph{nges}}
\entry{ngata}{\headword{ngata}\pos{n.}\sensenumber{3}\definition{spot}\example{Ge ttongo ngata me dädme gongkäbmenyne ako.}{We dived again at this other spot.}\sensenumber{3}\definition{around, about, approximately}\example{Ngaska sikstis, däba ngata me ngäna gozeg.}{Maybe sometime around the sixties I was born.}\example{Sɨmell a tu taosen namba ngata me dagaeya.}{There were about two thousand pigs.}\sensenumber{3}\definition{small space}\example{Ngäna lla yaba ngatangata dae ibi allan.}{I am going through a tight space crowded with people.}\subentry{\headword{ngata me}\pos{adv.}\definition{around, about, approximately}}\subentry{\headword{ngatangata}\pos{n.}\definition{small space}}}
\entry{ngatengate}{\headword{ngatengate}\pos{n.}\sensenumber{3}\definition{gliding possum, sugar gliderLlo toko me baet ingoll ddäddäg kälsäre dan, be ddamba peyang dan a papllägag dan. (A small cuscus-like animal in treetops, but it has wings and glides)}\example{Ngatengate da biratt a ddobae moko dan.}{Roasted possum is very tasty.}}
\entry{ngatt}{\headword{ngatt}\pos{n.}\sensenumber{3}\definition{type of tree}}
\entry{ngattong}{\headword{ngattong}\pos{ord. num.}\sensenumber{1}\definition{first}\example{ngattong e ebdo me}{on the first day}\example{Ge obo ngattong täräp dan.}{This is his first time.}\sensenumber{2}\definition{first, at first, initially, previously, before}\example{Ngäna ngattong botogol.}{I will hide first.}\example{Ngattong skulmeny dagaeya.}{Before, there was no school.}\sensenumber{3}\definition{beginning; past}\example{ngattong att otät yu ngalen}{cooking methods from the past}\example{Ngattong att gagäll ttoen de bälnan amallo.}{They're remembering bad things from before.}\sensenumber{4}\definition{front}\example{Gall e ngämi godagaleya, ngäna imne we, ede Imanuel ngattong e godmenän.}{We (excl.) boarded the canoe with me in the back; Imanuel sat in front.}\example{Bogo tubutubu gogon obo ngattong alle.}{He kneeled before him.}\sensenumber{4}\definition{the very beginning}\example{ngattong mullmull e}{at the very beginning}\example{Ngäna mɨnyi bongkam ngattong mullmull abal ttängäm de alla ingollang ngasnges erallo.}{I will start from the very beginning and show how we make a garden.}\sensenumber{4}\definition{very beginning, very first}\example{ngattongattong täräp me}{in the very beginning}\sensenumber{4}\definition{ahead, in front}\example{Ddob imneimne dag a ddob ngattongattong dag.}{Some are behind and some are ahead.}\example{Obo ngattong täräp daeya ngattongattong ibnin e.}{It was his first time going in front.}\subentry{\headword{ngattong mullmull}\pos{n.}\definition{the very beginning}}\subentry{\headword{ngattongattong2}\pos{mod.}\definition{very beginning, very first}}\subentry{\headword{ngattongattong1}\pos{adv.}\definition{ahead, in front}}}
\entry{ngädngäd}{\headword{ngädngäd}\pos{S vi.}\sensenumber{1}\definition{to roll up, curl}\example{Gullem a gongädän.}{The snake coiled up.}\sensenumber{2}\definition{to roll up, curl}\example{Ngämi tater nängädaeballa.}{We (excl.) rolled up the mats.}\sensenumber{3}\definition{to fold}\example{Bogo ttang nängädnegan.}{She folded her arms.}\allomorph{ngäd}\allomorph{ngädnan}}
\entry{ngälen}{\headword{ngälen}\variant{sp. var. of}{ngalen}}
\entry{ngälngäl}{\headword{ngälngäl}\pos{mod.}\sensenumber{3}\definition{round}\example{Ngälngälatt dädär a mänyi tutu atta bolldaeyan kopek e.}{A round rock will roll from the hill to the valley.}}
\entry{ngälladab}{\headword{ngälladab}\pos{v.}\sensenumber{3}\definition{conceive}}
\entry{ngällae}{\headword{ngällae}\pos{S vi.}\sensenumber{3}\definition{to look back}\example{Ngäna ngllae allan.}{I am looking back.}\example{Bogo mäse dinduag me ako gongllaeyän däräng bälle, be ddone obom ikop dägagän.}{While running, she tried to look back for the dog, but she couldn't see him.}\allomorph{ngllae}\allomorph{ngangllae}\allomorph{ngälla}}
\entry{ngällban}{\headword{ngällban}\variant{var. of}{ngällbän}}
\entry{ngällbän}{\headword{ngällbän}\pos{S vi.}\sensenumber{1}\definition{to rise, arise, come up}\example{Ine ulle da gongälllbänmällnän.}{The big wave was rising.}\sensenumber{2}\definition{to awake, wake up, get up}\example{Ddobag a gudae ngällbaenen eran yunu atta.}{Some wake up from sleep early.}\sensenumber{3}\definition{to lift}\example{Kottllam de kapu wi dängällbäneyo.}{They lifted the turtle up to carry.}\sensenumber{4}\definition{to get, take}\example{teks mani ngällbänang lla}{tax collector}\example{Dindugän, do täl pallkoll dingällbänän.}{He ran and got a piece of bamboo.}\sensenumber{5}\definition{to wake up, to wake}\example{Ngäna bam danglläbän.}{I woke you up.}\allomorph{ngällbänen}\allomorph{nglläbän}\allomorph{nglläb}\allomorph{ngänglläb}\allomorph{ngälläb}\allomorph{ngällb}\allomorph{ngllib}\allomorph{ngollbän}\allomorph{ngollb}\allomorph{ngällbae}\allomorph{ngängälläb}}
\entry{ngällngäll}{\headword{ngällngäll}\pos{S vt.}\sensenumber{5}\definition{to produce, yield, bear (fruit)}\example{Wit däm a ulleulle gognegän a käp de yuwog gongällanegän.}{The wheat grew big and produced lots of grain.}\example{Nge da ngällngällang dan.}{The coconut tree is bearing fruit.}\allomorph{ngäll}}
\entry{ngämae}{\headword{ngämae}\pos{S vi.}\sensenumber{5}\definition{to go around, take the long way}\example{Kurupel gungmaeyän ttongo nyongo dae dallän do auma.}{Kurupel went using another path to the grave.}\allomorph{ngmae}}
\entry{ngämangäma}{\headword{ngämangäma}\pos{pers. pron.}\sensenumber{5}\definition{1nsg.excl.refl}}
\entry{ngämar}{\headword{ngämar}\pos{S vt.}\sensenumber{5}\definition{to haul}\example{Mälla da wayati deyangmereyo.}{The women hauled over the watermelons.}\allomorph{ngmer}}
\entry{ngämälle}{\headword{ngämälle}\variant{sp. var. of}{ngämlle}}
\entry{ngämäne}{\headword{ngämäne}\variant{sp. var. of}{ngämene}}
\entry{ngämen}{\headword{ngämen}\pos{S vt.}\sensenumber{5}\definition{to reach, catch up to}\example{Ngäna bam mɨnyi bangmen.}{I will catch up to you.}\allomorph{ngämenmäll}\allomorph{ngmen}}
\entry{ngämengäme}{\headword{ngämengäme}\pos{n.}\sensenumber{5}\definition{type of vine with sweet, edible fruit that are yellow when ripe}}
\entry{ngämi}{\headword{ngämi}\pos{pers. pron.}\sensenumber{5}\definition{we (first person nonsingular exclusive pronoun, nominative form)}\example{Ngämi mɨnyi bam batrameya.}{We will carry you.}\sensenumber{5}\definition{possessive form of ngämi}\example{Bäne ngämingg a ulle abal dan ngäma pate.}{Your help towards us is great.}\sensenumber{5}\definition{ablative-possessive form of ngämi}\example{Ngäna bablle mɨnyi ngämaene ine kämbmenyatt eka de bɨllɨtne.}{I will be telling you our diving story.}\sensenumber{5}\definition{accusative form of ngämi}\example{Abo bongo ngämim yangkollmällneg.}{You must follow us.}\sensenumber{5}\definition{dative form of ngämi}\example{Da ngämi e we bam baweseya, ngäma moko da nangesneg ngämira.}{Whatever we may ask of you, we want you to do it for us.}\subentry{\headword{ngäma}\pos{pers. pron.}\definition{possessive form of ngämi}}\subentry{\headword{ngämaene}\pos{pers. pron.}\definition{ablative-possessive form of ngämi}}\subentry{\headword{ngämim}\pos{pers. pron.}\definition{accusative form of ngämi}}\subentry{\headword{ngämira}\pos{pers. pron.}\definition{dative form of ngämi}}}
\entry{ngämingg}{\headword{ngämingg}\pos{S vt.}\sensenumber{1}\definition{to help}\example{Lla da ttongo lla de nangminggan towall ttokoe e.}{A man helps another man to cut grass.}\example{Ngäna bam mɨnyi bangmingg.}{I will help you.}\example{Ibi mɨnyi iba mälla de bangmiminyaemnalla.}{We (incl.) will be supporting our women.}\sensenumber{2}\definition{to answer}\example{Bogo do gungminggän ada, "Aoǃ"}{There he answered, "Yesǃ"}\sensenumber{2}\definition{helpful, supportive}\example{Bogo ngäminggag yae dan.}{She is a supportive mother.}\allomorph{ngmingg}\allomorph{ngämiminy}\allomorph{ngmiminy}\allomorph{ngämi}\allomorph{ngmi}\etymology{ngämingg + =ang}\subentry{\headword{ngäminggag}\pos{mod.}\definition{helpful, supportive}}}
\entry{ngämlla}{\headword{ngämlla}\variant{dial. var. of}{ngämlle}}
\entry{ngämne}{\headword{ngämne}\variant{fast speech var. of}{ngämene}}
\entry{ngämral}{\headword{ngämral}\pos{n.}\sensenumber{2}\definition{type of vine that grows in the bush and irritates skin}}
\entry{ngän}{\headword{ngän}\pos{n.}\sensenumber{2}\definition{boundary}\example{Ende tän bo ekaklle ngän}{the boundary of Ende lands}}
\entry{ngäna1}{\headword{ngäna1}\pos{pers. pron.}\sensenumber{2}\definition{I (first person singular pronoun, nominative form)}\example{Ngäna gotar.}{I slept.}\sensenumber{2}\definition{restrictive copular form of ngäna (present form)}\example{Ngäna ngämo känyer ngämo pemliangae ngänawaeben ngämo ma me.}{It is only me with my family alone in my house.}\sensenumber{2}\definition{copular form of ngäna (present form)}\example{Gänyaolle gall zizag a ngänawaenen.}{The owner of this canoe is me.}\sensenumber{2}\definition{ablative-possessive form of ngäna}\example{Bongo ngämene eka de malaem anggan.}{You obey my words.}\sensenumber{2}\definition{dative form of ngäna}\example{Bäne däräng a daden, be ngämlle ddone dan.}{You have a dog, but I don't (lit. a dog exists to you, but not to me).}\sensenumber{2}\definition{possessive form of ngäna}\example{Ngämo zegatt ttängäm a Malläm.}{My birthplace is Malam.}\sensenumber{2}\definition{emphatic form of ngäna}\example{Ngänawa lla de umllang anggan.}{It is I that tells people.}\sensenumber{2}\definition{accusative form of ngäna}\example{Ngänäm kollmäll a mudan.}{Don't follow me.}\allomorph{ngäm}\allomorph{ngän}\subentry{\headword{ngänawaeben}\pos{cop.}\definition{restrictive copular form of ngäna (present form)}}\subentry{\headword{ngänawaenen}\pos{cop.}\definition{copular form of ngäna (present form)}}\subentry{\headword{ngämene}\pos{pers. pron.}\definition{ablative-possessive form of ngäna}}\subentry{\headword{ngämlle}\pos{pers. pron.}\definition{dative form of ngäna}}\subentry{\headword{ngämo}\pos{pers. pron.}\definition{possessive form of ngäna}}\subentry{\headword{ngänawa}\pos{pers. pron.}\definition{emphatic form of ngäna}}\subentry{\headword{ngänäm}\pos{pers. pron.}\definition{accusative form of ngäna}}}
\entry{ngäna2}{\headword{ngäna2}\variant{var. of}{ngänaeka}}
\entry{ngänaeka}{\headword{ngänaeka}\pos{A vt.}\sensenumber{2}\definition{to cry (about, for)}\example{Bongo ewatta ngänaeka alle?}{Why are you crying?}\example{Bom ngänaeka dägagallo.}{They cry for him.}\sensenumber{2}\definition{tear}\sensenumber{2}\definition{crying, in tears}\sensenumber{2}\definition{to burst into tears}\example{Bogo ngänaeka gompllogon.}{He burst into tears.}\sensenumber{2}\definition{nonsingular form of ngänaeka}\example{Tämamae llɨg de papa dägnegnän, ngänaekangänaeka gognän.}{He hit all the boys and they cried.}\etymology{redup. of ngänaeka + =ang}\subentry{\headword{ngänaeka täräm}\pos{n.}\definition{tear}}\subentry{\headword{ngänaekongänaekong}\pos{adv.}\definition{crying, in tears}}\subentry{\headword{ngänaeka pollongg}\pos{S vi.}\definition{to burst into tears}}\subentry{\headword{ngänaekangänaeka}\pos{A vi.}\definition{nonsingular form of ngänaeka}}}
\entry{ngänam}{\headword{ngänam}\pos{S vt.}\sensenumber{1}\definition{to understand, recognize}\example{Ubi mɨnyi eka midd de bängnameyo.}{They will understand the meaning.}\example{Bibi ddone ke ngänaem amalla?}{Don't you all understand?}\example{Bogo gongnamän ada, nagda obom era kuki dägagän.}{He recognized that his friend had deceived him.}\sensenumber{2}\definition{understanding}\sensenumber{2}\definition{unrecognizable, obscure}\example{Ngaskäma Ende eka da mɨnyi ngänameny bogon källkae.}{Maybe the Ende language will become unrecognizable later.}\allomorph{ngnam}\allomorph{ngnaem}\allomorph{ngänaem}\allomorph{ngän}\allomorph{ngn}\etymology{ngänam + =meny}\subentry{\headword{ngänameny}\pos{mod.}\definition{unrecognizable, obscure}}}
\entry{ngänaurur}{\headword{ngänaurur}\pos{A vi.}\sensenumber{2}\definition{to mourn}}
\entry{=ngänäm}{\headword{=ngänäm}\pos{n. cl.}\sensenumber{2}\definition{similative case clitic; like}\example{Ttongo lla de dirom ngänäm zuwoe a mudan.}{Don't shoot another person like they are a cassowary.}}
\entry{ngänngän}{\headword{ngänngän}\pos{S vi.}\sensenumber{2}\definition{to swear}}
\entry{ngänongg}{\headword{ngänongg}\variant{var. of}{ngonongg}}
\entry{ngänttäg}{\headword{ngänttäg}\pos{S vi.}\sensenumber{1}\definition{to arrive, return}\example{Ge lla ngänttäg allan ttongo ttängäm atta kilikiliang.}{This man is returning from another place happily.}\example{Ngämo näkäp e gongttägän.}{It came to my mind.}\example{Mɨnyi ddob sisor eka da ako bälltaemnän.}{Then other new languages will arrive.}\sensenumber{2}\definition{to bring, carry}\example{Llɨg kälsre de ine da do ddob llayaba pate dängttägän.}{The water carried the boy over to some people over there.}\allomorph{ngttäg}\allomorph{llɨtam}\allomorph{llɨtaem}\allomorph{lltam}\allomorph{lltaem}\allomorph{ngttäng}\allomorph{llttaem}\allomorph{ngttägmäll}\allomorph{llätaem}\allomorph{ngänttäg}\allomorph{ngttä}\allomorph{ngänttä}\allomorph{ngättäg}\allomorph{ngttag}}
\entry{ngänglam}{\headword{ngänglam}\pos{S vi.}\sensenumber{2}\definition{to buzz}\example{Ngänglamang dan.}{It's buzzing.}}
\entry{ngänyanyemmeny}{\headword{ngänyanyemmeny}\pos{mod.}\sensenumber{2}\definition{mean, unkind}}
\entry{ngänyngäny}{\headword{ngänyngäny}\pos{S vt.}\sensenumber{2}\definition{to swallow}\example{Amtet de nängänymällneg!}{Take (lit. swallow) a breath!}\allomorph{ngäny}}
\entry{ngäs}{\headword{ngäs}\pos{S vi.}\sensenumber{1}\definition{to return, come back}\example{Käsre llɨg a ngänangänong ma we dängäsmällän.}{Then the boys went home crying.}\example{Bol a angosan.}{The ball came back.}\sensenumber{2}\definition{to return, bring back}\example{Joshua däbe ttägäll käp de deyangosän ma we.}{Joshua brought that money back home.}\sensenumber{2}\definition{repeatedly}\example{Kaptte de kotang de ngäsangngäsang mättnan a mudan.}{Don't wear dirty clothes over and over.}\sensenumber{2}\definition{back, the other way}\example{Ubi komlla gopnaeyo a ngäsengäse dindugiyu.}{Those two turned around and ran back.}\allomorph{ngos}\allomorph{ngis}\allomorph{ngäsmäll}\allomorph{nges}\allomorph{ngismäll}\allomorph{ngosmäll}\allomorph{ngism}\etymology{redup. of ngäs + =ang}\subentry{\headword{ngäsangngäsang}\pos{adv.}\definition{repeatedly}}\subentry{\headword{ngäsengäse}\pos{adv.}\definition{back, the other way}}}
\entry{ngätae}{\headword{ngätae}\pos{S vt.}\sensenumber{2}\definition{to check}\example{Lla da mɨnyi ngätaenen e ballnän.}{The man will be going to check.}}
\entry{ngätt}{\headword{ngätt}\pos{n.}\sensenumber{2}\definition{yard}\example{Däräng a yuwog ngätt me källa bebeyag dan.}{The dog takes craps in many yards.}}
\entry{ngättangätta}{\headword{ngättangätta}\pos{S vt.}\sensenumber{1}\definition{to block, obscure}\example{Bongo ddone ikop dagne, llo da ngänäm dangttawalle.}{You couldn't see me; the tree was obscuring me.}\sensenumber{2}\definition{to occupy, take over}\example{Särem a dangttanän ttängäm de.}{Darkness took over the village.}\allomorph{ngtta}\allomorph{ngtte}}
\entry{ngättangätte}{\headword{ngättangätte}\pos{S vt.}\sensenumber{2}\definition{to count}\example{Ingglis eka walle dangättnealle.}{I was counting in English.}\allomorph{ngttä}\allomorph{nttäng}\allomorph{ngätta}\allomorph{ngätt}}
\entry{ngättäma}{\headword{ngättäma}\pos{n.}\sensenumber{2}\definition{place, spot}\example{tudi ma ngättma}{place for fishing}\example{Däba ngättäma we gotarän.}{In that place, he slept.}\allomorph{ngätt}\allomorph{ngätt ma}}
\entry{ngättma}{\headword{ngättma}\variant{fast speech var. of}{ngättäma}}
\entry{nge}{\headword{nge}\pos{n.}\sensenumber{1}\definition{coconut palm}\example{Ngäna mulldae dan ge nge de bängkäll.}{I can climb this coconut palm.}\sensenumber{2}\definition{coconutWätät ulle dan, nge ine da dädär a lla da ibeny eran. (It's an important food; when the coconut water is dry, people plant it.)}\example{nge däg}{coconut bunch}\example{Ngämi nge de diwenyeya.}{We (excl.) brought over the coconut.}\sensenumber{1}\definition{fully dry coconut}\example{Nge dɨdɨr bo ngasnen a kemibi dag, ada otät sespen a, mubine da, kul ngasnen a ako ddob kemibi ttoen a.}{Dry coconut has many uses: you can boil it, cook it in a cake, making balls, and few other things.}\sensenumber{2}\definition{the eleventh and final stage of coconut growth in which the fruit is completely dry and brown}\sensenumber{2}\definition{coconut cream}\sensenumber{2}\definition{coconut water}\sensenumber{1}\definition{coconut flower}\example{Niki tämamae apte ttang llɨpɨt pa de dägäddnegän nge popo me.}{Niki killed all five birds in the coconut flowers.}\sensenumber{2}\definition{the fourth stage of coconut growth in which flowers appear}\sensenumber{2}\definition{type of snakeNge toko me giddollag gullem da. Tupitupi da. Lla ddäddägmeny da. (It's a snake that lives in coconut palms. It's long. It doesn't bite people.)}\sensenumber{1}\definition{tiny coconut, baby coconut}\example{Ddob lla da gae nge tikop de kaepnen amallo.}{Some people chew the baby coconuts of the gae palm.}\sensenumber{2}\definition{the sixth stage of coconut growth in which the fruits have a diameter of \textbackslashtextasciitilde5 cm}\sensenumber{2}\definition{coconut frond}\subentry{\headword{nge dɨdɨr}\pos{n.}\definition{fully dry coconut}}\subentry{\headword{nge id}\pos{n.}\definition{coconut cream}}\subentry{\headword{nge ine}\pos{n.}\definition{coconut water}}\subentry{\headword{nge popo}\pos{n.}\definition{coconut flower}}\subentry{\headword{nge tärmir}\pos{n.}\definition{type of snakeNge toko me giddollag gullem da. Tupitupi da. Lla ddäddägmeny da. (It's a snake that lives in coconut palms. It's long. It doesn't bite people.)}}\subentry{\headword{nge tikop}\pos{n.}\definition{tiny coconut, baby coconut}}\subentry{\headword{nge ttam}\pos{n.}\definition{coconut frond}}}
\entry{nge ibeny}{\headword{nge ibeny}\pos{n.}\sensenumber{2}\definition{planting a coconut as a gesture of peace}}
\entry{nge pollgo}{\headword{nge pollgo}\pos{n.}\sensenumber{2}\definition{coconut frog, green}}
\entry{Ngeba}{\headword{Ngeba}\pos{pn.}\sensenumber{2}\definition{Ngeba (toponym)}}
\entry{Ngerbab}{\headword{Ngerbab}\pos{pn.}\sensenumber{2}\definition{male personal name}}
\entry{ngetae}{\headword{ngetae}\pos{S vi.}\sensenumber{1}\definition{to touch}\example{Ttongo lla da mälla da walle nägabällallo ddɨg alle angeteallo.}{A man was standing with his wife with their backs touching.}\sensenumber{2}\definition{to get engaged}\example{Lla da medäda wa mägda de wanseg yaran a obo mälla peyang ngetae allan.}{The man leaves his father and mother and gets engaged to his woman.}\sensenumber{3}\definition{to unite, join}\example{Lla da ddone mɨnyi sapasapang beyagän Adi bäne enanae ngetaeatt de.}{Man shall not separate what God has joined together for good.}\sensenumber{4}\definition{to grip, hold onto}\example{Ngäma ttang de deyangeteyo.}{They gripped our (excl.) hands.}\allomorph{ngete}}
\entry{ngetam}{\headword{ngetam}\pos{S vi.}\sensenumber{4}\definition{to come down, descend}\example{Angetam.}{[You] come down.}\example{Baet gongetamän matu we.}{The cuscus came down to the ground.}}
\entry{ngleg}{\headword{ngleg}\pos{S vt.}\sensenumber{4}\definition{to clear the ground before building}\example{Banglegeya.}{Let's clear the ground.}}
\entry{ngllongg}{\headword{ngllongg}\pos{S vt.}\sensenumber{4}\definition{to forget}\example{Ubi gongllomenynegän otät kongkom e.}{They forgot to bring food.}\allomorph{ngllomeny}}
\entry{ngoe}{\headword{ngoe}\variant{sp. var. of}{ngoi2}}
\entry{ngoeang}{\headword{ngoeang}\pos{n.}\sensenumber{4}\definition{traditional Y-shaped house post (used prior to nails)}}
\entry{ngoetrangoetra}{\headword{ngoetrangoetra}\pos{n.}\sensenumber{4}\definition{type of ant that doesn't bite}}
\entry{ngoi1}{\headword{ngoi1}\pos{A vt.}\sensenumber{4}\definition{to crowd, surround}\example{Ngoi dägaebeyo tämamae mänmän de.}{They surrounded all the girls.}}
\entry{ngoi2}{\headword{ngoi2}\pos{n.}\sensenumber{4}\definition{tooth}}
\entry{ngok}{\headword{ngok}\pos{ideo.}\sensenumber{4}\definition{oink, snort}}
\entry{ngokngok}{\headword{ngokngok}\pos{n.}\sensenumber{4}\definition{boobook (Australian, barking [barking owl])}}
\entry{ngolo}{\headword{ngolo}\pos{n.}\sensenumber{4}\definition{type of big yam with a purple interior, thorns, and hairs}}
\entry{ngollot}{\headword{ngollot}\pos{S vt.}\sensenumber{4}\definition{to make burst, make explode}\example{Ddol a nyäng de bungolltän.}{The gas made the bag explode.}\allomorph{ngollt}}
\entry{ngonenngonen duwem}{\headword{ngonenngonen duwem}\pos{A vi. \textbackslash& vt.}\sensenumber{4}\definition{to scarf down, eat very quickly}}
\entry{ngonoe}{\headword{ngonoe}\pos{S vt.}\sensenumber{4}\definition{to ask}\example{Llamda da abo ubim ngonoenen de deyangkamän.}{The old man started to question those two.}\example{Ubi obaoba gongnoemenynegnän ada, "Ge endan?"}{They were asking themselves, "What is this?"}\sensenumber{4}\definition{question}\example{Daballe ttongo ngonoe eka de nangnoeyan.}{After that, she asked another question.}\allomorph{ngnoe}\allomorph{ngonoenen}\subentry{\headword{ngonoe eka}\pos{n.}\definition{question}}}
\entry{ngonongg}{\headword{ngonongg}\pos{S vi.}\sensenumber{1}\definition{to think}\example{Bogo ada gongnomenynän ada ka däräng a pollnenang me erallo.}{He was thinking that the dogs were barking.}\example{Emi ulle abal gongnomenyän.}{Emi was thinking hard.}\sensenumber{2}\definition{to think about, think of, recall, remember}\example{Ngäna bam angnonggan.}{I thought of you.}\example{Eka de tämamae ngäna era ddone ngonomeny allan.}{I can't recall all the words.}\allomorph{ngnongg}\allomorph{ngnomeny}\allomorph{ngonomeny}\allomorph{ngono}\allomorph{ngno}\allomorph{ngnoe}\allomorph{ngonoe}}
\entry{ngongo}{\headword{ngongo}\pos{S vt.}\sensenumber{2}\definition{to smooth, sand (a surface)}\example{Turik de nongowan.}{I smoothed the axe.}\allomorph{ngo}}
\entry{ngongop}{\headword{ngongop}\pos{S vt.}\sensenumber{2}\definition{to hug, embrace}\example{Mäg da llɨg kälsre de nongopan.}{The mother hugged her little boy.}\allomorph{ngop}}
\entry{ngowangowe}{\headword{ngowangowe}\pos{mod.}\sensenumber{2}\definition{narrow}\example{Ge walle da ngowangowe dan gall ibi wi.}{The river is too narrow for the boat to pass.}}
\lettersection{Ny ny}
\entry{nyamäll}{\headword{nyamäll}\pos{S vi.}\sensenumber{2}\definition{to be late}\example{Ubi dänyamällän.}{They were late.}}
\entry{nyamällatt}{\headword{nyamällatt}\pos{kin.}\sensenumber{2}\definition{woman's exchange sibling's child}}
\entry{nyan}{\headword{nyan}\pos{n.}\sensenumber{2}\definition{type of tree with small leaves and bark used for mumu and topping a roof}}
\entry{nyanyem}{\headword{nyanyem}\pos{S vt.}\sensenumber{2}\definition{to respect}\example{Bogo mälla de obo ddone mermerae nyanyem eran.}{He isn't properly respecting his wife.}\example{Ngämo llɨg de mɨnyi banyemeyo.}{They will respect my son.}\allomorph{nyem}}
\entry{nyanyu}{\headword{nyanyu}\pos{n.}\sensenumber{1}\definition{action}\example{Bogo gagäll nyanyu di ngasnges eran mällada pate.}{He's doing bad things to his wife.}\sensenumber{2}\definition{to act, dramatize}\example{Ge lla ekameny dan, be mɨnyi eka de ttang alle mae nanyunegneyo.}{This man is mute, but just his hands will act out the words.}\allomorph{nyu}}
\entry{nyaraman}{\headword{nyaraman}\pos{n.}\sensenumber{2}\definition{fine day}}
\entry{nyägae}{\headword{nyägae}\pos{S vi.}\sensenumber{1}\definition{to flail}\example{Guzi da nyägaenen allan.}{The crayfish was flailing around.}\sensenumber{2}\definition{to stir}\example{Mälla da wätät de nanyägaenen spun alle.}{The woman was stirring the food with a spoon.}\example{Danygaeaebneyo sana peyang.}{They were stirring them together with sago.}\allomorph{nygae}\allomorph{nyga}\allomorph{nägaeya}\allomorph{nyänggae}}
\entry{nyäkiyam}{\headword{nyäkiyam}\pos{n.}\sensenumber{2}\definition{tear}}
\entry{nyäkukub}{\headword{nyäkukub}\pos{n.}\sensenumber{2}\definition{sago washing basket}}
\entry{nyämaenyämae}{\headword{nyämaenyämae}\pos{S vt.}\sensenumber{2}\definition{to surround}\example{Lla da obom danymaeyo.}{People surrounded him.}\allomorph{nymae}}
\entry{nyäng1}{\headword{nyäng1}\pos{n.}\sensenumber{2}\definition{basket; bagZa zarnen ma dan. (It's for holding things.)}\example{Nyäng a mätta alle bodo gogon.}{The bag became filled with yams.}}
\entry{nyäng2}{\headword{nyäng2}\pos{S vi.}\sensenumber{2}\definition{to dance}\example{Dänyängnän obo peyang.}{They were dancing with him.}\allomorph{ngyäng}}
\entry{nyäng tär}{\headword{nyäng tär}\pos{***}\sensenumber{2}\definition{Woven strap/handle of a nyäng (bag)}}
\entry{nyängkallbidd}{\headword{nyängkallbidd}\pos{n.}\sensenumber{2}\definition{type of bagLlo ttoe o ttope mamlla walle inenatt za zarnen e za komnen e. (A thing made from woven bark or reeds for holding and carrying things.)}\example{Mätta komnen e nyängkallbidd a mer dan.}{Nyängkallbidd bags are good for carrying yams.}}
\entry{nyängkallddäb}{\headword{nyängkallddäb}\pos{n.}\sensenumber{2}\definition{type of basket}}
\entry{nyäny}{\headword{nyäny}\pos{S vt.}\sensenumber{1}\definition{to paint}\example{Ubi pale alle nänyallo.}{They painted themselves with white clay.}\sensenumber{2}\definition{to anoint}\example{Ubi mer mollong owel de dällädaebeyo obo pätt nyäny e.}{They bought fragrant oils for anointing his body.}\allomorph{ny}\allomorph{nyänan}}
\entry{nyänye}{\headword{nyänye}\pos{S vt.}\sensenumber{2}\definition{to share, split}\example{Ngämi dänyeya däbe ttägäll käp de.}{We (excl.) shared that money.}\example{Otät de nyänan amallo.}{They're sharing the food.}\allomorph{ny}\allomorph{nye}\allomorph{nyä}}
\entry{nyärab}{\headword{nyärab}\pos{S vi.}\sensenumber{2}\definition{to go up, ascend}\example{Tutu gonyärabeya.}{We ascended the hill.}\allomorph{nyäraeb}}
\entry{nyärmeny}{\headword{nyärmeny}\pos{n.}\sensenumber{2}\definition{little argument between children}}
\entry{nyäroe}{\headword{nyäroe}\pos{S vi.}\sensenumber{2}\definition{to creep}\example{Angde ngäna ikop däga ada ddia dan, ngäna nyäroe de dängkam.}{When I saw the deer, I started to creep up.}}
\entry{nyärpae}{\headword{nyärpae}\pos{S vi.}\sensenumber{2}\definition{to refuse (someone in the dative)}\example{Däbe män a oblle nyärpae ma da ddone mulldae gogon.}{That girl couldn't refuse him.}}
\entry{nyeny}{\headword{nyeny}\pos{n.}\sensenumber{2}\definition{type of large tree that grows along the river and in swamps with white flowers, brown fruit, and white bark used for making mumu and topping a roof}}
\entry{nying}{\headword{nying}\pos{n.}\sensenumber{2}\definition{foot}\example{Baba bom iddob sana täkäll da nazunan nying me.}{Last night, a sago thorn got Dad in the foot.}\sensenumber{2}\definition{sandal}\sensenumber{2}\definition{toenail}\sensenumber{2}\definition{toe}\sensenumber{2}\definition{type of game involving kicking}\sensenumber{2}\definition{shoe}\subentry{\headword{nying kollop}\pos{n.}\definition{sandal}}\subentry{\headword{nying lläbe}\pos{n.}\definition{toenail}}\subentry{\headword{nying lläpät}\pos{n.}\definition{toe}}\subentry{\headword{nying tongoe}\pos{n.}\definition{type of game involving kicking}}\subentry{\headword{nying ttoe}\pos{n.}\definition{shoe}}}
\entry{nyinggulgul}{\headword{nyinggulgul}\pos{n.}\sensenumber{2}\definition{type of birdNyeny pimbyom alle ma gogoag pa kälsre dan. (It's a small bird that builds its nest from nyeny bark.)}}
\entry{nyingnying}{\headword{nyingnying}\pos{n.}\sensenumber{2}\definition{foundation of fence}\etymology{redup. of nying}}
\entry{nylib}{\headword{nylib}\pos{n.}\sensenumber{2}\definition{arrow\textbackslash_type}}
\entry{nyonga}{\headword{nyonga}\pos{n.}\sensenumber{2}\definition{Triton cockatooLlo ik me giddollag pa, pällämpälläm dan. (A bird that lives in trees, it's white.)}}
\entry{nyongkoe}{\headword{nyongkoe}\pos{S vt.}\sensenumber{2}\definition{to pull}\example{Angde tudi gognän, obo tudi di kollba ulle da danykoeyän.}{While fishing, a big fish pulled on her line.}\allomorph{nykoin}\allomorph{nykoe}}
\entry{nyongo}{\headword{nyongo}\pos{n.}\sensenumber{1}\definition{road, path, way}\example{Gänya nyongo dae nalle.}{[You] take this road.}\sensenumber{2}\definition{way, method}\example{Erodias nyongo de deyagnegnän Zon bälle gäz e.}{Herodias was searching for a way to kill John.}}
\entry{nyongo taempägag}{\headword{nyongo taempägag}\pos{n.}\sensenumber{2}\definition{navigator (of a canoe)}\etymology{taempäg + =ang, lit. 'way-shower'}}
\entry{nyukukum}{\headword{nyukukum}\pos{n.}\sensenumber{2}\definition{type of bag, used for squeezing sago, a thing woven from tree bark or reeds to wash sago or to fill with sago.Llo ttoe o ttope mamlle alle inenatt sana kämnan e a sana därnan e. sana teya kängkäm ma dan.}\example{Bäne aoli tot nyukukum dag?}{How many nyukukum bags do you have?}}
\lettersection{O}
\entry{o1}{\headword{o1}\pos{coord.}\sensenumber{2}\definition{or}\example{ik mi o upe me}{inside or outside}}
\entry{o2}{\headword{o2}\pos{mod.}\sensenumber{2}\definition{ripe}\example{Mätta ttam a o agnegan.}{The yam leaves will be ripe.}}
\entry{o3}{\headword{o3}\pos{interj.}\sensenumber{2}\definition{oh}}
\entry{=o}{\headword{=o}\pos{ptcl.}\sensenumber{2}\definition{vocative marker}}
\entry{o klak}{\headword{o klak}\pos{adv.}\sensenumber{2}\definition{o'clock}\example{Tri o klak gogän.}{It's three o'clock.}\etymology{from Englisho'clock}}
\entry{o klok}{\headword{o klok}\variant{sp. var. of}{o klak}}
\entry{oba2}{\headword{oba2}\pos{subr.}\sensenumber{2}\definition{so that}\example{Bongo ine ma nalle oba bongo ine neyatan.}{You, go to the well so you can fetch water.}}
\entry{Obama}{\headword{Obama}\pos{pn.}\sensenumber{2}\definition{Obama (toponym)}}
\entry{obergaban}{\headword{obergaban}\pos{n.}\sensenumber{2}\definition{clear day}}
\entry{Obewa}{\headword{Obewa}\pos{pn.}\sensenumber{2}\definition{male personal name}}
\entry{obi}{\headword{obi}\variant{dial. var. of}{ubi}}
\entry{obo zaga}{\headword{obo zaga}\variant{sp. var. of}{obozaga}}
\entry{oboll}{\headword{oboll}\pos{n.}\sensenumber{2}\definition{type of small tree that grows in the grassland (\textbackslashtextasciitilde2 m) with white flowers and red fruit with a large brown seed; fruit is eaten by birds; wood is used for sturdy posts}}
\entry{obosasa}{\headword{obosasa}\pos{n.}\sensenumber{1}\definition{Australian rufous fantailWalle gutu dae papllägag pa dan. (It's a bird that flies through dry riverbeds.)}\sensenumber{2}\definition{blue jewel-babbler}}
\entry{od}{\headword{od}\pos{n.}\sensenumber{2}\definition{type of fatty freshwater fish}}
\entry{odaode}{\headword{odaode}\variant{dial. var. of}{udaude}}
\entry{odoolo}{\headword{odoolo}\pos{n.}\sensenumber{2}\definition{type of bird}}
\entry{Ogbaperma}{\headword{Ogbaperma}\pos{pn.}\sensenumber{2}\definition{Ogbaperma (camping place)}}
\entry{Ogoa}{\headword{Ogoa}\pos{pn.}\sensenumber{2}\definition{male personal name}}
\entry{ogog}{\headword{ogog}\pos{n.}\sensenumber{2}\definition{grey-crowned babblerOba ma de pall dop alle gonen amallo. (They build their houses with parts of the red palm.)}}
\entry{oi}{\headword{oi}\pos{interj.}\sensenumber{2}\definition{oh}}
\entry{ok}{\headword{ok}\variant{sp. var. of}{oke}}
\entry{oke}{\headword{oke}\pos{interj.}\sensenumber{2}\definition{okay}}
\entry{okokol}{\headword{okokol}\pos{A vt.}\sensenumber{2}\definition{to welcome}\example{Bogo mɨnyi llɨg kälsre de okokol bägagän.}{She will welcome the little boy.}}
\entry{okookol}{\headword{okookol}\variant{sp. var. of}{okokol}}
\entry{ol1}{\headword{ol1}\pos{n.}\sensenumber{2}\definition{hall}\etymology{from Englishhall}}
\entry{ol2}{\headword{ol2}\variant{var. of}{wäl}}
\entry{Olalea}{\headword{Olalea}\pos{pn.}\sensenumber{2}\definition{female personal name}}
\entry{Old Maoto}{\headword{Old Maoto}\pos{pn.}\sensenumber{2}\definition{Old Maoto (toponym)}}
\entry{olib}{\headword{olib}\pos{n.}\sensenumber{2}\definition{olive}\etymology{from Englisholive}}
\entry{olmapänyik}{\headword{olmapänyik}\pos{n.}\sensenumber{2}\definition{armpit}\example{olmapänyik kom}{armpit hair}}
\entry{olmopäga}{\headword{olmopäga}\pos{n.}\sensenumber{2}\definition{type of sagoUlle päddabag sana dan bisel ngänäm. (A sago that grows big like bisel.)}}
\entry{oll}{\headword{oll}\pos{n.}\sensenumber{2}\definition{sugarcaneKaekep ma dan, mokowang dan. (It's for chewing; it's tasty.)}}
\entry{olle}{\headword{olle}\pos{A vi.}\sensenumber{1}\definition{to shout, yell, call out}\example{Olle de gongkamän.}{She started to shout.}\sensenumber{2}\definition{to call (over), summon}\example{Bongo ngänäm olle nagalle gänyaolle.}{You called me over here.}}
\entry{=olle}{\headword{=olle}\pos{n. cl.}\sensenumber{2}\definition{allative case clitic; to, into, towards}\allomorph{balle}\allomorph{alle}}
\entry{ollondd}{\headword{ollondd}\pos{n.}\sensenumber{2}\definition{root}}
\entry{ollong}{\headword{ollong}\pos{n.}\sensenumber{2}\definition{time, occasion}\example{ttongdae ollong, komlla ollong, kumuddäga ollong, tupi ollong, mända ollong}{once, twice, thrice, four times, five times}\example{Tätäm ngäna komlla ollong räba de daspun.}{Yesterday, I threw the eraser twice.}}
\entry{omad}{\headword{omad}\pos{n.}\sensenumber{2}\definition{type of friendMällayaba ttaem. Mälla llädäd täräp me aya gulag era ede ubi era omad dageyo. (When a woman gets married, the person who accompanies her and the woman are omad.)}}
\entry{omawe}{\headword{omawe}\pos{n.}\sensenumber{2}\definition{type of tree}}
\entry{omäg}{\headword{omäg}\pos{n.}\sensenumber{2}\definition{magic}\example{omäg ttoen}{sorcery}\example{Digezänän omäg pawa walle.}{He took it out with magic power.}\sensenumber{2}\definition{magician, sorcerer, fortune-teller}\etymology{omäg + =ang}\subentry{\headword{omägag}\pos{n.}\definition{magician, sorcerer, fortune-teller}}}
\entry{omgälgäl}{\headword{omgälgäl}\pos{n.}\sensenumber{2}\definition{red ants}}
\entry{omog}{\headword{omog}\variant{sp. var. of}{omäg}}
\entry{omom}{\headword{omom}\pos{S vt.}\sensenumber{2}\definition{to sweep}\example{Ngäna ag me ma de noman.}{I swept the house in the morning.}\allomorph{om}}
\entry{one}{\headword{one}\variant{sp. var. of}{wan}}
\entry{Ono}{\headword{Ono}\pos{pn.}\sensenumber{2}\definition{Ono (toponym)}}
\entry{Ongg}{\headword{Ongg}\pos{pn.}\sensenumber{2}\definition{male personal name}}
\entry{onggall}{\headword{onggall}\pos{n.}\sensenumber{2}\definition{frilled lizardTäme ingoll ddäddäg a be llan peyang dan, malla ddäddäg ma dan. (A goanna-like animal, but it has frills and is not edible.)}\example{Iba malla onggall ddägnan a umllang dag.}{We (incl.) do not eat frilled lizard.}}
\entry{ony}{\headword{ony}\pos{S vt.}\sensenumber{2}\definition{to carry; get; bring}\example{sabi onyang lla}{person who carries out the law}\example{Era ebdo me mɨnyi ibi bobollab ttängäm e mätta kongkom e?}{On which day will we (incl.) go to the garden to get the yams?}\example{Ngämi nge de diwenyeya.}{We (excl.) brought over the coconut.}\example{Bibra otät de aya bikomän?}{Who is bringing over your (pl.) food?}\sensenumber{2}\definition{causative-applicative form of ony (to make carry; bring for, take to; take from)}\example{Grace ngänäm gull de dangkomonggän.}{Grace had me bring the net.}\example{Bong obo pope sɨmell midd de dangkomonggän.}{Bong brought the pork for his uncle.}\example{Ttall a obo za de dangkämonggän.}{The wallaby took her thing.}\allomorph{weny}\allomorph{kongkom}\allomorph{kom}\allomorph{wony}\allomorph{ny}\allomorph{ngkom}\etymology{kom- [ony] + -ngg}\subentry{\headword{komongg}\pos{S vd.}\definition{causative-applicative form of ony (to make carry; bring for, take to; take from)}}}
\entry{opa}{\headword{opa}\pos{n.}\sensenumber{2}\definition{type of tree}}
\entry{opa ttangtte}{\headword{opa ttangtte}\pos{n.}\sensenumber{2}\definition{type of tree}}
\entry{opap}{\headword{opap}\pos{S vt.}\sensenumber{2}\definition{to cross over, pass over, move across}\example{Ibi bopäll bem de apte menae e daupeya.}{Let's cross the sea to the other side.}\allomorph{ope}\allomorph{up}\allomorph{opanen}\allomorph{upe}\allomorph{opa}}
\entry{ope}{\headword{ope}\variant{fr. var. of}{upe}}
\entry{Opo}{\headword{Opo}\pos{pn.}\sensenumber{2}\definition{Opo (toponym)}}
\entry{opodo}{\headword{opodo}\pos{n.}\sensenumber{2}\definition{weaving pattern with checks}}
\entry{opop}{\headword{opop}\pos{v.}\sensenumber{1}\definition{tangle}\sensenumber{2}\definition{wind}}
\entry{orbam}{\headword{orbam}\pos{n.}\sensenumber{2}\definition{marshTawa me källäm dan. Yäbäd me kollba we pamnenma dan. (It's a pond within a swamp. It's a fishing spot when sunny.)}}
\entry{Oriomo}{\headword{Oriomo}\pos{pn.}\sensenumber{2}\definition{Oriomo (Wipi-speaking village in Oriomo-Bituri Rural LLG)}}
\entry{Orpmang}{\headword{Orpmang}\pos{pn.}\sensenumber{2}\definition{Wipi language}}
\entry{orwa}{\headword{orwa}\pos{n.}\sensenumber{2}\definition{suffering}\example{Bäne orwa atta agezän.}{Come out from your suffering.}\example{Bogo itrell atta orwa gognän.}{He was suffering from illness.}}
\entry{osne}{\headword{osne}\pos{n.}\sensenumber{2}\definition{type of small taro}}
\entry{ospel}{\headword{ospel}\pos{n.}\sensenumber{2}\definition{hospital; aid post}}
\entry{otal}{\headword{otal}\pos{S vi.}\sensenumber{2}\definition{to perch}\example{Pa da llo me gotalalle.}{The bird perched in the tree.}}
\entry{otät}{\headword{otät}\pos{S vt.}\sensenumber{1}\definition{to eat}\example{Diba toto we, bogo sana dorko de doton.}{That evening, she ate dry sago.}\example{Ubi llo käp otnanang dag.}{They eat tree fruit.}\sensenumber{2}\definition{food}\example{Bogo mer wätät de yu dägagän.}{She cooked nice food.}\sensenumber{3}\definition{hunger}\example{Otät atta ngämo kutt a mängallmeny agmallo.}{My bones became weak from hunger.}\sensenumber{3}\definition{starvation}\sensenumber{3}\definition{kitchen}\sensenumber{3}\definition{edible}\allomorph{ot}\allomorph{wätät}\allomorph{wätat}\allomorph{t}\allomorph{wät}\etymology{lit. 'place for cooking food'}\subentry{\headword{otät ngänaeka ttoen}\pos{n.}\definition{starvation}}\subentry{\headword{otät yu ma}\pos{n.}\definition{kitchen}}\subentry{\headword{otnan ma}\pos{mod.}\definition{edible}}}
\entry{Ouli}{\headword{Ouli}\pos{pn.}\sensenumber{3}\definition{female personal name}}
\entry{owel}{\headword{owel}\pos{n.}\sensenumber{3}\definition{oil}}
\lettersection{P}
\entry{pa}{\headword{pa}\pos{n.}\sensenumber{3}\definition{bird}\example{Bogo pa näddägan ngämäne gäzatt de.}{He ate the bird that I killed.}\sensenumber{3}\definition{feather}\subentry{\headword{pa kom}\pos{n.}\definition{feather}}}
\entry{padiem}{\headword{padiem}\pos{n.}\sensenumber{3}\definition{type of introduced bananaTubutubu pänyanzag dan. Obo käp a tämamae ulleulle tanong dag, o me käp a otänan ma dag, ada binzenen ma dag yu mi. Ako kire da yu ma dag ada umie da mubine ngasnges ma dan nge dädär kolnenatt id alle. (It grows tall. Its fruit are all a little big; when ripe, its fruit are eaten and heated on the fire. Also, when unripe, they are cooked and made into mubine with coconut cream from a dry coconut.)}}
\entry{paeb}{\headword{paeb}\pos{num.}\sensenumber{3}\definition{five (English numeral)}\etymology{from Englishfive}}
\entry{Paeke}{\headword{Paeke}\pos{pn.}\sensenumber{3}\definition{female personal name}}
\entry{Paelet}{\headword{Paelet}\pos{pn.}\sensenumber{3}\definition{Pilate}}
\entry{paengg}{\headword{paengg}\pos{S vt.}\sensenumber{3}\definition{to guess}\example{Ngäna paengg eran ada Jugu mɨnyi sisri bongttägän Wim atta.}{I'm guessing that Jugu will arrive from Wim now.}\allomorph{mpaeg}}
\entry{pag}{\headword{pag}\pos{n.}\sensenumber{3}\definition{salt}}
\entry{paib}{\headword{paib}\variant{sp. var. of}{paeb}}
\entry{Paine}{\headword{Paine}\pos{pn.}\sensenumber{3}\definition{male personal name}}
\entry{paitt}{\headword{paitt}\pos{n.}\sensenumber{3}\definition{fight}\etymology{from Englishfight}}
\entry{pakätt}{\headword{pakätt}\pos{n.}\sensenumber{3}\definition{widow's robe}}
\entry{paklle}{\headword{paklle}\pos{n.}\sensenumber{3}\definition{type of snakeDdäddäg ma gullem da. Sägädsägäd ttoeang da. (It's an edible snake. It has yellow skin.)}}
\entry{Pakllepakllemäll}{\headword{Pakllepakllemäll}\pos{pn.}\sensenumber{3}\definition{Pakllepakllemäll (camping place)}}
\entry{pakos}{\headword{pakos}\pos{n.}\sensenumber{3}\definition{type of spear}\example{Madura pakos de daspunän.}{Madura shot the spear.}}
\entry{palament}{\headword{palament}\pos{n.}\sensenumber{3}\definition{parliament}\etymology{from Englishparliament}}
\entry{pale}{\headword{pale}\pos{n.}\sensenumber{3}\definition{type of white clay used for painting babies}}
\entry{Palsa}{\headword{Palsa}\pos{pn.}\sensenumber{3}\definition{Palsa (toponym)}}
\entry{pall}{\headword{pall}\pos{n.}\sensenumber{3}\definition{varieties of palms with coconuts with a red exocarp}}
\entry{pall kottllam}{\headword{pall kottllam}\pos{n.}\sensenumber{3}\definition{red-bellied short-necked turtle}}
\entry{pall kubull}{\headword{pall kubull}\pos{n.}\sensenumber{3}\definition{red-legged pademelon}}
\entry{pall tawe}{\headword{pall tawe}\pos{n.}\sensenumber{3}\definition{type of large palm that grows in the grassland with white flowers and coconuts with a red exocarp; small sticks are good for houses, and bark is used for walling}}
\entry{pallall}{\headword{pallall}\pos{n.}\sensenumber{1}\definition{direction; area}\example{Ngäna naigae pallall e ibi allan.}{I'm going south.}\example{Adawalle de ddia da gupudrellän iba pallall e.}{From there, the deer spread in our (incl.) direction.}\sensenumber{2}\definition{side}\example{sawe pallall}{left side}\example{Ubi guspullnän ekaklle we obo ngattong pallall alle.}{They were falling to the ground in front of him.}\sensenumber{3}\definition{about}\example{Män a llɨg pate däräng pallall e eka gogon.}{The girl talked to the boy about the dog.}}
\entry{pallam}{\headword{pallam}\pos{S vt.}\sensenumber{3}\definition{to cut open}\example{Obom dapellaemnän.}{He was cutting him open (i.e. for surgery).}\allomorph{pallaem}\allomorph{pellam}\allomorph{pellaem}}
\entry{pallängkmeny}{\headword{pallängkmeny}\pos{S vt.}\sensenumber{1}\definition{to divide}\example{Da ttongo kantri me lla da bompallängkmenynegän obaoba komlla kullum e a bogäddeyo obaoba, däbe kantri da mɨnyi ddone bakällmewän.}{If the people in a country divide themselves into two groups and fight each other, that country will not survive.}\sensenumber{2}\definition{to judge}\example{God bo enanae ttoen pallängkmeny}{God's Final Judgment}\allomorph{mpallängkmeny}}
\entry{pallkeakeya}{\headword{pallkeakeya}\pos{n.}\sensenumber{2}\definition{type of red mushroom that grows in the grasslandBawa me päddabag dan. Sana peyang yu ma dan. (It grows during June and July. It's cooked with sago.)}}
\entry{pallta}{\headword{pallta}\pos{mod.}\sensenumber{2}\definition{flat}}
\entry{pam}{\headword{pam}\pos{n.}\sensenumber{2}\definition{injection, shot, vaccine}\example{Ospel e mängalae gobäll, llɨg di dokomeya pam alle zuwoenen e.}{We quickly went to the hospital, bringing the children to get shots.}}
\entry{pama}{\headword{pama}\pos{n.}\sensenumber{2}\definition{farmer}\etymology{from Englishfarmer}}
\entry{pambu}{\headword{pambu}\pos{n.}\sensenumber{2}\definition{hoe (tool)}}
\entry{pameny}{\headword{pameny}\pos{S vi.}\sensenumber{2}\definition{to scream}\example{Lla da ompamenyan.}{The man screamed.}\allomorph{mpameny}}
\entry{pamker}{\headword{pamker}\pos{n.}\sensenumber{2}\definition{pumpkin}}
\entry{pamli}{\headword{pamli}\variant{sp. var. of}{pemli}}
\entry{pampem}{\headword{pampem}\pos{S vt.}\sensenumber{2}\definition{to fish}\example{Ag me källäm pamnen e gobäll.}{In the morning, we went to fish at the pond.}\allomorph{pam}\allomorph{pamnen}}
\entry{pana}{\headword{pana}\pos{n.}\sensenumber{2}\definition{type of relationshipIttma bin llayaba. (Ceremonial term for someone.)}}
\entry{Panakawa}{\headword{Panakawa}\pos{pn.}\sensenumber{2}\definition{Panakawa (toponym)}}
\entry{panda}{\headword{panda}\pos{n.}\sensenumber{2}\definition{type of tree}}
\entry{pane}{\headword{pane}\pos{n.}\sensenumber{2}\definition{pot}\example{Ddob mälla da pane moepang de ikrol alle gällämnan allo.}{Some women wash dirty pots using ash.}}
\entry{pani}{\headword{pani}\pos{mod.}\sensenumber{2}\definition{funny}\etymology{from Englishfunny}}
\entry{panis}{\headword{panis}\pos{A vt.}\sensenumber{2}\definition{to punish}\etymology{from Englishpunish}}
\entry{panggopanggo}{\headword{panggopanggo}\pos{n.}\sensenumber{2}\definition{type of round yam with a white interior and hairs}}
\entry{panya}{\headword{panya}\pos{n.}\sensenumber{2}\definition{pineapple}}
\entry{panypeny}{\headword{panypeny}\pos{S vt.}\sensenumber{2}\definition{to speak, talk, say}\example{Ngäna Ende eka de panypeny eran.}{I speak Ende.}\example{Ngäna Ingglis eka, Motu eka de ddone bäpanyneg.}{I can't speak English or Hiri Motu.}\example{Bogo ge eka de däpanyän.}{She said this.}\allomorph{pany}\allomorph{panynen}\allomorph{peny}}
\entry{paopao}{\headword{paopao}\pos{S vt.}\sensenumber{2}\definition{to cut around, prune (a plant)}\example{Däbe kapang ulle de näpao!}{[You] prune the big acacia tree!}\allomorph{pao}}
\entry{pap}{\headword{pap}\pos{n.}\sensenumber{2}\definition{type of round mutae yam}}
\entry{papa}{\headword{papa}\pos{A vt.}\sensenumber{2}\definition{to hit, beat}\example{Papa nägagan.}{He hit him.}\allomorph{papa}}
\entry{papälläk}{\headword{papälläk}\pos{S vt.}\sensenumber{2}\definition{to split, chop}\example{Bogo yu di däpellknegän.}{She split the wood.}\allomorph{mpällk}\allomorph{pälläk}\allomorph{pällängk}\allomorph{mpellk}\allomorph{pällk}\allomorph{pallk}}
\entry{pape}{\headword{pape}\pos{S vt.}\sensenumber{1}\definition{to cut (grass)}\example{Kitar pu de däpeyo.}{They cut the kitar grass.}\sensenumber{2}\definition{to smash, crush (something soft, but not a flower)}\example{Ngäna mameat de näpewan.}{I smashed the papaya.}\example{Ubi mätta de bäpeaemneyo.}{They will be smashing yams.}\allomorph{pe}}
\entry{papek}{\headword{papek}\pos{n.}\sensenumber{1}\definition{dam, blockage; wall}\sensenumber{2}\definition{to block; close}\example{Ud a papekatt dan.}{The door has been closed.}\example{Bogo de papek allan ttang alle.}{He blocked him with his hand.}\sensenumber{3}\definition{to wall, make a wall}\example{llo ttoe kädnanatt ma paknen e}{a thing that is taken off trees to make walls}\example{Oba ma da pall put alle papekatt dan.}{Their house is walled with pall bark.}\allomorph{paknen}\allomorph{pak}}
\entry{papiye}{\headword{papiye}\pos{n.}\sensenumber{3}\definition{animal tracks}\example{Ddone ada ddäddäg papiye dag sisri wälläng me.}{There are many animal tracks in the bush now.}}
\entry{paplläg}{\headword{paplläg}\pos{S vi.}\sensenumber{3}\definition{to fly}\example{Ngämi pällgaeb amalla.}{We are flying.}\example{Pa da dapllägän matu we.}{The bird flew down.}\allomorph{plläg}\allomorph{palläg}}
\entry{papllek}{\headword{papllek}\variant{var. of}{papälläk}}
\entry{papoe}{\headword{papoe}\pos{S vt.}\sensenumber{3}\definition{to pierce, make a hole}\example{Bogo obo llan de napoeyan.}{She pierced her ear.}\example{Ngäna nyäng tär ngasnges e nyäng de napoenan waeya käsre alle.}{I used the small wire to make holes in the bag to put the new bag string.}\allomorph{poe}}
\entry{Papon}{\headword{Papon}\pos{pn.}\sensenumber{3}\definition{male personal name}}
\entry{papu}{\headword{papu}\variant{baby talk var. of}{kapu}}
\entry{Papua Niugini}{\headword{Papua Niugini}\pos{pn.}\sensenumber{3}\definition{Papua New Guinea}}
\entry{Papwa Niyu Gini}{\headword{Papwa Niyu Gini}\variant{sp. var. of}{Papua Niugini}}
\entry{Papwa Nu Gini}{\headword{Papwa Nu Gini}\variant{sp. var. of}{Papua Niugini}}
\entry{para}{\headword{para}\pos{n.}\sensenumber{3}\definition{event where harvest and hunting bounty are compared and gifted for bragging rights}}
\entry{Parade}{\headword{Parade}\pos{n.}\sensenumber{3}\definition{Parade (toponym)}}
\entry{Parama}{\headword{Parama}\pos{pn.}\sensenumber{3}\definition{Parama (in Kiwai Rural LLG)}}
\entry{parga}{\headword{parga}\pos{n.}\sensenumber{3}\definition{bridge}\example{Bob a parga de dällekän.}{The flood destroyed the bridge.}}
\entry{paro kottllam}{\headword{paro kottllam}\pos{n.}\sensenumber{3}\definition{type of turtle with red scales on neck}}
\entry{parpar}{\headword{parpar}\pos{n.}\sensenumber{3}\definition{type of sago bundle wrapped in a long cylinder in sago leaves}}
\entry{pasis}{\headword{pasis}\pos{n.}\sensenumber{3}\definition{deep}}
\entry{Paskam}{\headword{Paskam}\pos{pn.}\sensenumber{3}\definition{male personal name}}
\entry{paspas}{\headword{paspas}\pos{n.}\sensenumber{3}\definition{type of game where players try to keep a ball off the ground}\etymology{redup. of Englishpass}}
\entry{pasta}{\headword{pasta}\pos{n.}\sensenumber{3}\definition{pastorAdi bäne eka panypenyang lla sos me. (God's spokesperson in church.)}\example{Ngämi sisri pasta melem de ngasnen eralla.}{Now we are doing pastor work.}\etymology{from Englishpastor}}
\entry{pat1}{\headword{pat1}\pos{n.}\sensenumber{3}\definition{type of taro that is eaten from the suckers}}
\entry{pat2}{\headword{pat2}\pos{n.}\sensenumber{3}\definition{egg yolk}}
\entry{Pata}{\headword{Pata}\pos{pn.}\sensenumber{3}\definition{male personal name}}
\entry{=patalle}{\headword{=patalle}\variant{dial. var. of}{=patatt}}
\entry{patara}{\headword{patara}\pos{n.}\sensenumber{3}\definition{wall}\example{Komlla ebdo me abo ada käg a wa patara melem a dibem deyanges.}{Then, I made the floors and walls in two days.}}
\entry{patarapatara}{\headword{patarapatara}\pos{n.}\sensenumber{3}\definition{type of hand game where players make an L shape with their fingers}\etymology{redup. of patara}}
\entry{=patatt}{\headword{=patatt}\pos{n. cl.}\sensenumber{3}\definition{animate ablative case clitic; from (a person marked with the possessive)}\example{Gagäll anyke da bäne män bo patatt a zäme agezänan.}{The bad spirit has already left your daughter.}}
\entry{=pate}{\headword{=pate}\pos{n. cl.}\sensenumber{3}\definition{animate allative case clitic; to, towards, at (a person optionally marked with the possessive)}\example{Meri dallän Yokon pate.}{Mary went to Yokon.}\example{Llamda da ddone ada mikutt gogon oba pate.}{The old man got very angry at them.}}
\entry{patepate}{\headword{patepate}\pos{n.}\sensenumber{3}\definition{bamboo sticks used for percussionTtongdae täl am alle ngasnges att dan a kälae pop peyang. Komlla dop tubutubu alle papa ma dan. (It's made from one internode of bamboo with a little hole. It's hit with two short sticks.)}}
\entry{Patha}{\headword{Patha}\pos{pn.}\sensenumber{3}\definition{male personal name}}
\entry{patiti}{\headword{patiti}\pos{n.}\sensenumber{3}\definition{type of birdLlo dop alle ma gogowag pa kälsre dan. (It's a small bird that builds its home from sticks.)}}
\entry{patkoll}{\headword{patkoll}\pos{n.}\sensenumber{3}\definition{bundle}\example{Ngäna yu de papllekdae bäga, bongo abo pattkolldae ignig a ada ma we ikom.}{I'll just chop the wood; then you just make the bundles and carry them home.}\example{Ngäna kollba de patkoll däga nyäng peyang.}{I bundled the fish with the bag.}\sensenumber{3}\definition{carrying bundles, with bundles}\allomorph{patko}\subentry{\headword{patkopatkoll}\pos{adv.}\definition{carrying bundles, with bundles}}}
\entry{=patme}{\headword{=patme}\pos{n. cl.}\sensenumber{3}\definition{animate locative case clitic; at someone's place; for (a person marked with the possessive)}\example{Ngämo za da dag do Pauma bo patme.}{My things are there at Pauma's place.}\example{Ngäna oba patme ada ulle gog.}{I was raised by them (lit. I grew up at theirs).}\example{Bogo däpleon iba patme.}{He died for us.}}
\entry{patrol}{\headword{patrol}\pos{n.}\sensenumber{3}\definition{patrol}\etymology{from Englishpatrol}}
\entry{patt1}{\headword{patt1}\pos{n.}\sensenumber{1}\definition{tree trunk (fallen)}\example{Ngäna llo patt toko me godmen.}{I sat on top of a tree trunk.}\sensenumber{2}\definition{drum shell}\sensenumber{3}\definition{coconut husking stickLlo popoatt nge kopnen e. (From sharpened wood, for husking coconuts.)}\example{Gänyme nge kopnen ma patt de aeya dängkänän?}{Who pulled out the coconut husking sticks?}}
\entry{pattkoll}{\headword{pattkoll}\variant{var. of}{patkoll}}
\entry{pattlle}{\headword{pattlle}\pos{n.}\sensenumber{3}\definition{type of small bamboo that grows in the bush and along creeks; used to cook sago and make flutes; may be burned for fertile land for planting}}
\entry{Paul}{\headword{Paul}\variant{sp. var. of}{Pol}}
\entry{Pauma}{\headword{Pauma}\pos{pn.}\sensenumber{3}\definition{female personal name}}
\entry{pauro}{\headword{pauro}\pos{n.}\sensenumber{3}\definition{chickenBädab e ekawang pa dan. (It's a bird that sings at dawn.)}}
\entry{Pawaturi}{\headword{Pawaturi}\pos{pn.}\sensenumber{3}\definition{Pahoturi River}}
\entry{paya}{\headword{paya}\pos{A vt.}\sensenumber{3}\definition{to shoot, fire}\etymology{from Englishfire}}
\entry{pazi}{\headword{pazi}\pos{n.}\sensenumber{3}\definition{year}\example{da pazi me, de pazi me}{last year, next year}\example{kumuddäga pazi imnealle}{three years later}\example{Bäne pazi da auli dan?}{How old are you (lit. how many are your years)?}}
\entry{päd}{\headword{päd}\pos{n.}\sensenumber{3}\definition{scar}\sensenumber{3}\definition{tattoo}\subentry{\headword{pädpäd}\pos{n.}\definition{tattoo}}}
\entry{pädoe}{\headword{pädoe}\pos{S vi.}\sensenumber{1}\definition{to blow}\example{Wel a gopädoeän.}{The wind blew.}\sensenumber{2}\definition{to blow}\example{Mälla da wa lla da utt de dapädoenegeyo.}{The man and women blew conch shells.}\allomorph{pdoe}\allomorph{pädoenen}\allomorph{podoe}}
\entry{pädrall}{\headword{pädrall}\pos{S vi.}\sensenumber{1}\definition{to spread (out), scatter, disperse}\example{Lla da oba mällamällong gupädrellän otät ma.}{The men and their wives spread out to get food.}\example{Oba näkäpngon a tämamae apädrellan.}{Their thoughts are all scattered.}\sensenumber{2}\definition{to spread (out), scatter, sow}\example{däm kutt pädrallag lla}{man sowing seeds}\example{Bogo up di dotnegnän a ttoe dapädrällnegnän.}{He was eating bananas and scattering the peels.}\allomorph{pädrell}\allomorph{pädräll}\allomorph{pdärell}}
\entry{päddab}{\headword{päddab}\pos{S vi.}\sensenumber{1}\definition{to grow}\example{Tomato da däpddabän.}{The tomato plant grew.}\sensenumber{2}\definition{to be cooked, be done}\example{Ngäma wätät a zime opddaenegan.}{Our (excl.) food has finished cooking.}\allomorph{pdda}\allomorph{päddae}\allomorph{pddae}\allomorph{pädd}}
\entry{päddpädd}{\headword{päddpädd}\pos{S vi.}\sensenumber{1}\definition{to grow}\example{ine toko me päddnenang towall}{grass growing on the water's surface}\example{Däm kutt a ekaklle me guspullän a däpäddän.}{The seed fell to the ground and grew.}\sensenumber{2}\definition{to explode}\example{Yu me pattlle da päddpädd allan.}{The bamboo explodes in the fire.}\sensenumber{2}\definition{causative-applicative form of päddpädd}\example{Lla da täl de dämpäddmenyaemallo yu alle.}{People used to blow up bamboo with fire.}\allomorph{pädd}\etymology{pädd- [päddpädd] + -ngg}\subentry{\headword{pänddäg2}\pos{S vt.}\definition{causative-applicative form of päddpädd}}}
\entry{pägamän}{\headword{pägamän}\pos{n.}\sensenumber{2}\definition{type of yam}}
\entry{päk}{\headword{päk}\pos{loc.}\sensenumber{2}\definition{edge}\example{gall päk}{edge of the canoe}}
\entry{pälan}{\headword{pälan}\pos{n.}\sensenumber{2}\definition{plan}\example{Mälla da pälan gognegän.}{The women made a plan.}\etymology{from Englishplan}}
\entry{pälengg}{\headword{pälengg}\pos{S vi.}\sensenumber{2}\definition{to drop down, fall}\example{Angde gogllaeaebne, yogoll ulle da dipliwän.}{While we were paddling, a heavy rain started coming down.}\allomorph{pli}}
\entry{pälkom}{\headword{pälkom}\pos{n.}\sensenumber{2}\definition{body hair}\example{Ngämo pälkom da guirewän.}{My hairs stood.}}
\entry{pälläk}{\headword{pälläk}\pos{n.}\sensenumber{2}\definition{type of introduced bananaTubutubu pänyanzag dan. Obo däg a yuwog dag. Käp a obo o me otänan ma dag, ako o da ine ttänttämang e säpall ma dag, ako kire da yu ma dag. (It grows tall. Its bunches are plentiful. When ripe, its fruit are eaten, and also to put in boiling water; when unripe, they are cooked.)}}
\entry{pälläm}{\headword{pälläm}\pos{mod.}\sensenumber{2}\definition{visible}\example{Ddäg kutt dae pälläm gognän.}{Only the backbone was visible.}}
\entry{pällämpälläm}{\headword{pällämpälläm}\pos{col.}\sensenumber{1}\definition{white, bright}\example{Obo iddpo da pällämpälläm a gombenmällnegnän.}{His clothes were white and shining.}\example{Zime abo ballme da gopnän pällämpälläm gognän ttängäm a.}{Dawn had already broken and the village was becoming bright.}\sensenumber{2}\definition{white, Western}\sensenumber{2}\definition{English}\etymology{redup. ofpälläm}\subentry{\headword{pällämpälläm eka}\pos{pn.}\definition{English}}}
\entry{pällämpälläm yurwe}{\headword{pällämpälläm yurwe}\pos{n.}\sensenumber{2}\definition{type of tree with white flowers and edible white fruit.}}
\entry{pällganen}{\headword{pällganen}\pos{S vt.}\sensenumber{2}\definition{to hang}\example{Ttongo mälla da iddpo de nällpäganegnan.}{A woman was hanging the clothes.}\allomorph{llpäga}\allomorph{plläga}\allomorph{pällga}}
\entry{pällkam}{\headword{pällkam}\pos{S vt.}\sensenumber{2}\definition{to split, break}\example{Ngäna kakoll de napällkaman.}{I broke the plate.}\example{Mälla da täl de dapällkaemaemneyo.}{The women were splitting the bamboo.}\example{Ngäna ngämo ikop glas de kamekame napällkaman.}{I accidentally broke my glasses.}\sensenumber{2}\definition{piece}\example{yu pallkoll}{piece of firewood}\sensenumber{2}\definition{pieces, shards}\allomorph{pellkam}\allomorph{pallk}\allomorph{pällkaem}\allomorph{pällk}\subentry{\headword{pallkoll}\pos{n.}\definition{piece}}\subentry{\headword{pällkapällkae}\pos{n.}\definition{pieces, shards}}}
\entry{pällkom}{\headword{pällkom}\variant{var. of}{pälkom}}
\entry{Pällmang}{\headword{Pällmang}\pos{pn.}\sensenumber{2}\definition{Pällmang (toponym)}}
\entry{pällnampällnam}{\headword{pällnampällnam}\pos{adv.}\sensenumber{2}\definition{squatting, crouching}\example{Bogo käg me pällnampällnam adämenan.}{She squatted down on the floor.}}
\entry{pällttän}{\headword{pällttän}\pos{S vi.}\sensenumber{2}\definition{to set off, start walking}\example{Däpällätt abo do Taolang.}{I started walking to Taolang.}\allomorph{pällätt}\allomorph{päpällätt}\allomorph{pällttän}\allomorph{pllätt}\allomorph{pälltt}\allomorph{päpllätt}}
\entry{pällulle}{\headword{pällulle}\pos{n.}\sensenumber{2}\definition{lung}}
\entry{pämbäll}{\headword{pämbäll}\pos{n.}\sensenumber{2}\definition{poison\textbackslash_root}}
\entry{pänae}{\headword{pänae}\pos{S vi.}\sensenumber{1}\definition{to turn back, turn around}\example{Ubi komlla gopnaeyo a ngäsengäse dindugiyu.}{The two of them turned around and ran the other way.}\sensenumber{2}\definition{to capsize, turn over, flip over}\example{Gopnaeyän.}{It capsized.}\example{Ngämaene gall a ada ka bopänayän, adawatta däbe za da ngämim ge deyangkollmällnän.}{Our (excl.) boat nearly capsized because that thing was following us.}\sensenumber{3}\definition{to turn, become, transform}\example{Obo mikutt a ttänttämang e gopnaeyän.}{His anger made him hot.}\example{Lla daeya Moli, be angde walle we guspunän, bunkuttang e gopnaen.}{Moli was a man, but when he jumped into the water, he turned into a catfish.}\sensenumber{4}\definition{to translate}\example{Ngäna ttoenttoen de näpänaeyan Ende we.}{I translated this story into Ende.}\sensenumber{5}\definition{zenith (position of the sun at high noon)}\allomorph{pnae}\allomorph{mpänae}}
\entry{pänaemeny}{\headword{pänaemeny}\pos{S vt.}\sensenumber{5}\definition{to check}\example{Mälla da mɨnyi bobällnän mätta mɨnyi bampnaemenyaemneyo ada mätta da zäme käp agan abo.}{The women will go and check the yams, whether they are ready yet.}\allomorph{mpänaemeny}\allomorph{mpnaemeny}}
\entry{pänbäll}{\headword{pänbäll}\pos{n.}\sensenumber{5}\definition{poisonous vine or root (used in fishing to stun fish)}}
\entry{pänddäg1}{\headword{pänddäg1}\pos{S vi.}\sensenumber{1}\definition{to start a grassfire}\example{Bogo gompäddägän.}{He started the grassfire.}\sensenumber{2}\definition{to hatch}\example{Pauro käp a gompäddägän.}{The chicken egg started hatching.}\sensenumber{3}\definition{to break water}\sensenumber{4}\definition{to mature (of a plant or a person)}\allomorph{pänddmeny}\allomorph{mpäddmeny}\allomorph{mpäddäg}}
\entry{pänmällang mälla}{\headword{pänmällang mälla}\pos{n.}\sensenumber{4}\definition{type of big taro}}
\entry{pänongg}{\headword{pänongg}\pos{S vi.}\sensenumber{1}\definition{to awake, wake up, get up}\example{Ngäna gomponongg.}{I woke up.}\sensenumber{2}\definition{to wake up, wake}\example{Ngäna obom pänongg eran.}{I am waking him up.}\allomorph{päno}\allomorph{mpono}}
\entry{pänpän}{\headword{pänpän}\pos{n.}\sensenumber{1}\definition{dust}\example{Bibi mɨnyi bina nying pänpän de näpttayaemneyo.}{You (all) will be shaking the dust from your feet.}\sensenumber{2}\definition{calcium hydroxide, lime}}
\entry{pängän}{\headword{pängän}\pos{S vi.}\sensenumber{2}\definition{to disappear}\example{Bogo apngänän.}{He disappeared.}\allomorph{pngän}\allomorph{png}}
\entry{pänggmeny}{\headword{pänggmeny}\pos{S vt.}\sensenumber{2}\definition{to protect, look after, take care of}\example{Ede adawatta ibi mɨnyi iba pätt de mermerae bampägmenyaemnalla.}{This is why we must take care of our bodies.}\allomorph{mpägmeny}\allomorph{mpänggmeny}\allomorph{mpäg}}
\entry{pänyae}{\headword{pänyae}\pos{S vi.}\sensenumber{2}\definition{to hop}\example{Bogo pänyaemällang dan.}{He hops slowly.}}
\entry{pänyik}{\headword{pänyik}\pos{A vi.}\sensenumber{2}\definition{to whisper}}
\entry{päpa}{\headword{päpa}\pos{n.}\sensenumber{2}\definition{line}}
\entry{päräl}{\headword{päräl}\pos{n.}\sensenumber{2}\definition{radjah shelduckLlo patt me a ine me giddollag pa dan. (It lives in tree trunks and in water.)}}
\entry{päre}{\headword{päre}\pos{adv.}\sensenumber{2}\definition{nonsingular form of säre}\example{Llokottang dag päre ngäma ttam a.}{Sadly, our (excl.) lives are difficult.}}
\entry{pärta}{\headword{pärta}\pos{num.}\sensenumber{2}\definition{thirty-six (yam counting numeral; 6\textbackslashtextasciicircum2)}\etymology{from a Yam language; compare Nen prta}}
\entry{päs}{\headword{päs}\pos{n.}\sensenumber{2}\definition{type of sugarcane-like plant with fruit on top}}
\entry{pätangeny}{\headword{pätangeny}\pos{v.}\sensenumber{2}\definition{splash}}
\entry{pätaräll}{\headword{pätaräll}\pos{v.}\sensenumber{2}\definition{splash}}
\entry{pätär}{\headword{pätär}\pos{n.}\sensenumber{2}\definition{white hair}\example{Bogo zäme pätär peyang allan.}{He already has white hair.}}
\entry{pätpät}{\headword{pätpät}\pos{S vt.}\sensenumber{2}\definition{to dryine käpät a pätt me angämae ddone agan a näpätan.}\allomorph{pät}}
\entry{pätt}{\headword{pätt}\pos{n.}\sensenumber{1}\definition{body (of a person or animal)}\example{Obo pätt a ulle daeya.}{Her body was large.}\example{Ngämo pätt a llowamang gogon.}{My body was tired.}\example{Ngäna käza bo pätt de net alle kaen dägne.}{I wrapped the crocodile's body with a net.}\sensenumber{2}\definition{trunk of a plant in the ground; a single plant}\example{Nge pätt a tupi dan.}{The coconut palm trunk is long.}\example{Ubi up wo pätt dowae e gogeyo.}{The two of them got close to the ripe banana tree.}\example{Käp ku da wit pätt atta teti gongesnegän, ddob wit pätt atta siksti.}{Thirty seeds came from the wheat plant; from another wheat plant, sixty.}\example{Bogo kunur pätt de däträpnegalle.}{He reaped the corn plants.}\sensenumber{1}\definition{node (ring or line on bamboo or sugarcane that separates internodes)}\sensenumber{2}\definition{part of a bow}\etymology{pätt + käp}\subentry{\headword{pättkäp}\pos{n.}\definition{node (ring or line on bamboo or sugarcane that separates internodes)}}}
\entry{Pätta}{\headword{Pätta}\pos{pn.}\sensenumber{2}\definition{Pätta (toponym)}}
\entry{pättapätte}{\headword{pättapätte}\pos{S vt.}\sensenumber{1}\definition{to shake off}\example{Bibi mɨnyi bina nying pänpän de näpttayaemneyo.}{You (all) will be shaking the dust from your feet.}\sensenumber{2}\definition{to hit grass with a stick to scare away animals}\allomorph{ptta}\allomorph{ptte}\allomorph{pättanen}\allomorph{pättäpätta}}
\entry{pättäk}{\headword{pättäk}\pos{mod.}\sensenumber{2}\definition{short}\example{Ngäna era mälla tupi dan, bongo era pättäk dan.}{I'm a tall woman, but you're short.}\sensenumber{2}\definition{a bit short}\sensenumber{2}\definition{nonsingular form of pättäk}\example{Ge ma da tämamae pättäkpättäk dag.}{All these houses are short.}\subentry{\headword{pättäkpättäk2}\pos{mod.}\definition{a bit short}}\subentry{\headword{pättäkpättäk1}\pos{mod.}\definition{nonsingular form of pättäk}}}
\entry{pättäl}{\headword{pättäl}\pos{n.}\sensenumber{2}\definition{type of tree that grows near swamps with yellow flowers and bark used for rope to tie firewood}}
\entry{pättol}{\headword{pättol}\pos{S vt.}\sensenumber{2}\definition{to start singing}\example{Bogo bandra de dipttolän.}{He starting singing a song.}\allomorph{pttol}}
\entry{päzäg}{\headword{päzäg}\pos{kin.}\sensenumber{1}\definition{brother-in-law (man's wife's brother or man's sister's husband; reciprocal)Mälla da bo mang de päzäg eka bägagän.}\sensenumber{2}\definition{sister-in-law (man's wife's sister)}\sensenumber{2}\definition{in-laws}\subentry{\headword{päzäpäzäg}\pos{kin.}\definition{in-laws}}}
\entry{päzɨg}{\headword{päzɨg}\variant{sp. var. of}{päzäg}}
\entry{peba}{\headword{peba}\pos{n.}\sensenumber{2}\definition{paper}\etymology{from Englishpaper}}
\entry{pedae}{\headword{pedae}\pos{S vt.}\sensenumber{2}\definition{to sweep}\example{Ma de dapedealle.}{He sweeped the house.}\allomorph{pede}}
\entry{Pedaya}{\headword{Pedaya}\pos{pn.}\sensenumber{2}\definition{Pedaya (toponym)}}
\entry{Pedro}{\headword{Pedro}\pos{pn.}\sensenumber{2}\definition{male personal name}}
\entry{pemli}{\headword{pemli}\pos{n.}\sensenumber{2}\definition{family}\etymology{from Englishfamily}}
\entry{pen}{\headword{pen}\pos{n.}\sensenumber{2}\definition{pen}\etymology{from Englishpen}}
\entry{penangg}{\headword{penangg}\variant{var. of}{penongg}}
\entry{pendäg}{\headword{pendäg}\pos{S vi.}\sensenumber{1}\definition{to trip}\example{Gopendägalle ekaklle we.}{He was tripping and falling to the ground.}\sensenumber{2}\definition{to push to the ground, jostle, trip}\example{Lla gul ulle daeya, Yesu adame obo kollmällang de umllang dägnegän oblle gall kälsäre särämbae e, oba lla da ddone obom beyawändägallo a bimpidägmällnallo.}{The crowd was big, so Jesus told his disciples to prepare a small boat for him so that people would not come crowd him and be pushing him.}\allomorph{mpedäg}\allomorph{mpedämeny}\allomorph{pendämeny}\allomorph{mpedä}}
\entry{penmällpenmäll}{\headword{penmällpenmäll}\pos{mod.}\sensenumber{2}\definition{spotted}}
\entry{penongg}{\headword{penongg}\pos{S vi.}\sensenumber{1}\definition{to burn, set on fire, ignite}\example{Ge towall a mängalae penongg allan.}{This grass is burning fast.}\example{Yäbäd a aempononggan.}{The sun blazed.}\sensenumber{2}\definition{to burn, set on fire}\example{Ngäna towall penameny gongkam.}{I started to ignite the grass.}\example{Yu daempononggeyo ma de.}{They set fire to the house.}\example{Ause da to dänglläbänän, kullkull de daimponomenyän tämamae.}{The old woman got the light and burned the whole garden.}\allomorph{ipänongg}\allomorph{impänonmeny}\allomorph{penomeny}\allomorph{emponomeny}\allomorph{emponongg}\allomorph{empänungg}\allomorph{maemponongg}\allomorph{empono}}
\entry{Pentae}{\headword{Pentae}\pos{pn.}\sensenumber{2}\definition{female personal name}}
\entry{pentae}{\headword{pentae}\pos{S vi.}\sensenumber{1}\definition{to spread, be transmitted}\example{Ge itrell a mɨnyi bompetaeyän.}{This disease will spread.}\example{Ge itrell a mängalae lla yaba pate pentaeang dan.}{This disease spreads to others quickly.}\sensenumber{2}\definition{to transfer, transmit, spread, pass on, pass down}\example{Ende lla da oba omog ttoen tameny de oba llɨg aba pate dampetaeneyo.}{Ende people were passing their sorcery on to their children.}\sensenumber{2}\definition{directly, in person, personally}\example{Ubi upe me dägabällabän adawatta ma ik a gonttämawän lla walle, be eka de pentapentae dandmoeyo ma ik e obo pate.}{They stood outside because the inside of the house was full of people, but they sent word directly to him inside.}\allomorph{mpetae}\allomorph{mpet}\subentry{\headword{pentapentae}\pos{adv.}\definition{directly, in person, personally}}}
\entry{Pentai}{\headword{Pentai}\variant{sp. var. of}{Pentae}}
\entry{pentngeny}{\headword{pentngeny}\pos{S vi.}\sensenumber{2}\definition{to trip}\example{Bogo säremsärem gompetngenyän yu pallkoll me.}{In the darkness, he tripped over a piece of wood.}\allomorph{mpetngeny}}
\entry{Penz}{\headword{Penz}\pos{n.}\sensenumber{2}\definition{name of a river}}
\entry{penganyäm}{\headword{penganyäm}\pos{n.}\sensenumber{2}\definition{type of yam}}
\entry{pengg}{\headword{pengg}\pos{S vt.}\sensenumber{2}\definition{to make the killing shot, deliver the final blow}\example{Ddäg aeya nampegan obo ddäg dan.}{The back [of an animal] belongs to him, the one who delivered the final blow.}\allomorph{mpeg}}
\entry{pepätt}{\headword{pepätt}\pos{n.}\sensenumber{1}\definition{black-faced monarch}\sensenumber{2}\definition{spot-winged monarch}}
\entry{pepeb}{\headword{pepeb}\pos{n.}\sensenumber{2}\definition{folktale, legend}\example{Ngämlle ttongo pepeb de nällät.}{[You] tell me one folktale.}\sensenumber{2}\definition{legend}\subentry{\headword{pepeb eka}\pos{n.}\definition{legend}}}
\entry{pepeb wup}{\headword{pepeb wup}\pos{n.}\sensenumber{2}\definition{type of introduced bananaYuwog pänyanzag dan, obo pätt a ddone ulle dan. Obo däg a ddone yuwog dag, obo käp a o me bänga otät ma dan be ddone mäzi mokowang dan, kutt a dadeg ako tomowang dan. Ge era mäsdae wup abal bo tonton wup dan, gudne wällang wup dan. (It grows all around; its trunk is not big. Its bunches are not plentiful; when ripe, though its fruit is edible, it's not tasty; there are seeds and it's sour. This is an original banana; it's a banana of the old bush.)}}
\entry{pera}{\headword{pera}\pos{n.}\sensenumber{2}\definition{bird perch}\sensenumber{2}\definition{perching}\subentry{\headword{perapera}\pos{adv.}\definition{perching}}}
\entry{peraengg}{\headword{peraengg}\variant{sp. var. of}{peraingg}}
\entry{peraingg}{\headword{peraingg}\pos{S vi.}\sensenumber{1}\definition{to cross, intersect}\example{Tawe da goimpregeyo.}{The tawe palms were crossed.}\example{Ngäna Karama atta yaran, ngämi Winson alle aimpregalla.}{I came back from Karama; Winson and I had intersecting paths (but we didn't see each other).}\sensenumber{2}\definition{to protrude, stick out}\example{Sɨmell ada daeya, ngoi peraingg att.}{The pig, its tusks were sticking out like this.}\allomorph{impreg}}
\entry{perälla}{\headword{perälla}\pos{n.}\sensenumber{2}\definition{type of vineMa bengae ma wälläng mamlla dan. (It's wild rope for house roofing.)}}
\entry{pes}{\headword{pes}\pos{ord. num.}\sensenumber{2}\definition{first}\example{pes täräp}{the first time}\etymology{from Englishfirst}}
\entry{petapeta}{\headword{petapeta}\pos{mod.}\sensenumber{1}\definition{thin (inanimate)}\example{Ge käg a petapeta dan}{The floor is thin.}\sensenumber{2}\definition{shallow}\example{Källäm a petapeta abal gogon.}{The ponds became very shallow.}}
\entry{Petepo}{\headword{Petepo}\pos{pn.}\sensenumber{2}\definition{female personal name}}
\entry{Peter}{\headword{Peter}\variant{sp. var. of}{Pita}}
\entry{Petom}{\headword{Petom}\pos{pn.}\sensenumber{2}\definition{Petom (toponym)}}
\entry{petron}{\headword{petron}\pos{n.}\sensenumber{2}\definition{blood vessel; vein; artery}}
\entry{pett}{\headword{pett}\pos{n.}\sensenumber{2}\definition{hip}}
\entry{pewälewäle}{\headword{pewälewäle}\pos{n.}\sensenumber{2}\definition{green pygmy gooseWalle toko dae ibiyag pa dan. (It's a bird that travels on top of the water.)}}
\entry{Pewe}{\headword{Pewe}\pos{pn.}\sensenumber{2}\definition{male personal name}}
\entry{peyam}{\headword{peyam}\pos{S vi.}\sensenumber{1}\definition{to come out from water, surface}\example{Däbe käza da walle amne me gopeyamän.}{That crocodile surfaced in the middle of the pond.}\sensenumber{2}\definition{to pop up, reappear, resurface}\example{Bongo gänyame opeyamalleǃ}{You're here again!}\allomorph{pey}}
\entry{=peyang}{\headword{=peyang}\pos{n. cl.}\sensenumber{2}\definition{comitative clitic; with (a noun marked with the possessive)}\example{Ngäna Grace bo peyang duwem agan.}{I ate with Grace.}}
\entry{Piasorosoro}{\headword{Piasorosoro}\pos{pn.}\sensenumber{2}\definition{male personal name}}
\entry{pid}{\headword{pid}\pos{n.}\sensenumber{2}\definition{horsefly}}
\entry{Pidgin}{\headword{Pidgin}\variant{sp. var. of}{Pizin}}
\entry{pidor}{\headword{pidor}\pos{n.}\sensenumber{2}\definition{white-bellied sea-eagleKollba ddädagag pa ulle dan. (It's a big bird that hunts fish.)}}
\entry{Pidortama}{\headword{Pidortama}\pos{pn.}\sensenumber{2}\definition{Pidortama (toponym)}}
\entry{pidroll}{\headword{pidroll}\pos{n.}\sensenumber{2}\definition{black palm weevilDdäddäg ma dan. (It's edible.)}}
\entry{pig}{\headword{pig}\pos{n.}\sensenumber{2}\definition{type of cultivated tree with edible fruit}}
\entry{pilat}{\headword{pilat}\variant{var. of}{pilatt}}
\entry{pilatt}{\headword{pilatt}\pos{n.}\sensenumber{2}\definition{plate, dish}\example{Ngäna mɨnyi pilatt de bänggllämneg.}{I will wash the dishes.}\etymology{from Englishplate}}
\entry{pimbyom}{\headword{pimbyom}\pos{n.}\sensenumber{2}\definition{small pieces of bark}}
\entry{pin}{\headword{pin}\pos{n.}\sensenumber{2}\definition{type of tree with white or red flowers and composite fruit that birds eatUtt käkäm de näkepneg a id de nänaneg kumye täräp me, be ddobae tomowang dan. Kakne ttoe de to we kädnan ma dan. (The shoots and young leaves are chewed and the extract is drunk when sick with a cough, but it's very sour. The outer bark is removed for light.)}}
\entry{Pinang}{\headword{Pinang}\pos{pn.}\sensenumber{2}\definition{Pinang (toponym)}}
\entry{pinsäg}{\headword{pinsäg}\pos{S vi.}\sensenumber{2}\definition{to tear}\example{Iddpo da komlla gumpisägeyo tuk alle do matu.}{The cloth tore in two from top to bottom.}\allomorph{mpisäg}}
\entry{pintta}{\headword{pintta}\pos{n.}\sensenumber{2}\definition{palm cockatooLlo ik me giddollag bätbät pa ulle dan. (It's a big black bird that lives in trees.)}}
\entry{Pinzin}{\headword{Pinzin}\variant{dial. var. of}{Pizin}}
\entry{pinzopinzo1}{\headword{pinzopinzo1}\pos{n.}\sensenumber{2}\definition{type of small insect that lives in the ground}}
\entry{pinzopinzo2}{\headword{pinzopinzo2}\pos{n.}\sensenumber{2}\definition{curly hair}}
\entry{ping}{\headword{ping}\pos{n.}\sensenumber{2}\definition{baby pin}}
\entry{Pingam}{\headword{Pingam}\pos{pn.}\sensenumber{2}\definition{female personal name}}
\entry{pinggudd}{\headword{pinggudd}\pos{n.}\sensenumber{2}\definition{skirt}\example{Bogo eraeya pinggudd ulle peyang daeya.}{She was wearing a long skirt.}}
\entry{piny}{\headword{piny}\pos{n.}\sensenumber{2}\definition{kingfisher (azure, little)Ekaklle ik me giddollag dan. (It's a bird that lives in the ground.)}}
\entry{pinya dorollog}{\headword{pinya dorollog}\pos{n.}\sensenumber{2}\definition{sacred kingfisherBawa me tongoeang pa dan. (It's a bird that plays in the rain.)}}
\entry{pip}{\headword{pip}\pos{n.}\sensenumber{2}\definition{red beeLla ddäddägag dan. (It bites people.)}}
\entry{Pipi}{\headword{Pipi}\pos{pn.}\sensenumber{2}\definition{female personal name}}
\entry{pipi}{\headword{pipi}\pos{S vt.}\sensenumber{2}\definition{to shoot, spear}\example{Ngäna ddob mozaya kollba de däpineg.}{I speared some mozaya fish.}\allomorph{pi}}
\entry{Pipiato}{\headword{Pipiato}\pos{pn.}\sensenumber{2}\definition{female personal name}}
\entry{pipllo}{\headword{pipllo}\pos{n.}\sensenumber{2}\definition{lizard}}
\entry{pipllug}{\headword{pipllug}\pos{S vi.}\sensenumber{2}\definition{to fly}\example{Bongo mɨnyi ikop nägneg ddob apapi di bipllugaebnän.}{You will see some butterflies flying.}\allomorph{pullgaeb}\allomorph{pllug}\allomorph{ipllug}}
\entry{pipti}{\headword{pipti}\pos{num.}\sensenumber{2}\definition{fifty}\etymology{from Englishfifty}}
\entry{piptin}{\headword{piptin}\pos{num.}\sensenumber{2}\definition{fifteen (English numeral)}\etymology{from Englishfifteen}}
\entry{pirik}{\headword{pirik}\pos{n.}\sensenumber{2}\definition{baton, stick}\example{Pirik a lla metmäll ma dan.}{A baton is for hitting people.}}
\entry{pirngän}{\headword{pirngän}\pos{S vt.}\sensenumber{2}\definition{to draw a weapon, take out}\example{Ngäna pakos de dapirngän.}{I took out a spear.}}
\entry{piro}{\headword{piro}\pos{n.}\sensenumber{1}\definition{starZa dan iddob me indrang allan ddapall me. (It's a thing that shines in the sky at night.)}\sensenumber{2}\definition{star weaving pattern}\sensenumber{2}\definition{type of game where the first person to see a star in the sky wins}\sensenumber{2}\definition{to become unconscious}\example{Bogo piro nangttängenan.}{He fell down unconcsious.}\etymology{lit. 'to count stars'}\subentry{\headword{piro kanas}\pos{n.}\definition{type of game where the first person to see a star in the sky wins}}\subentry{\headword{piro ttängattänge}\pos{S vt.}\definition{to become unconscious}}}
\entry{pisam}{\headword{pisam}\pos{S vi.}\sensenumber{1}\definition{to tear (of something thin)}\example{Ngämo kaptte da apisaman.}{My garment tore.}\sensenumber{2}\definition{to tear (something thin)}\example{Ngäna nyäg de napisaman.}{I broke the basket.}\example{Ngäna peba de pisam eran.}{I am tearing the paper.}\sensenumber{2}\definition{small pieces}\example{Llaeyabira ako midd pisepise dikomeya ma we.}{We brought small pieces of meat for everyone.}\allomorph{pisaem}\subentry{\headword{pisepise}\pos{n.}\definition{small pieces}}}
\entry{Pisi}{\headword{Pisi}\pos{pn.}\sensenumber{2}\definition{Pisi (in Gogodala Rural LLG)}}
\entry{Piskae}{\headword{Piskae}\pos{pn.}\sensenumber{2}\definition{Piskae (toponym)}}
\entry{pisor}{\headword{pisor}\pos{n.}\sensenumber{2}\definition{piece}}
\entry{pit}{\headword{pit}\pos{n.}\sensenumber{2}\definition{small insect that lives in the swamp and eats taro and sweet potato}}
\entry{Pita}{\headword{Pita}\pos{pn.}\sensenumber{2}\definition{male personal name}}
\entry{pitatep}{\headword{pitatep}\pos{v.}\sensenumber{1}\definition{to lift an animal or person and throw them to the ground, hit them hard against the ground}\sensenumber{2}\definition{to tackle someone, as in rugby}}
\entry{pite1}{\headword{pite1}\pos{n.}\sensenumber{1}\definition{grass skirtLlo ttoe o sana ttam utt kädnanatt alle inenatt ingong me mättmätt e. (A thing made from tree bark or sago leaves to wear when dancing.)}\example{Ingnenang aba pite da mermerae abal beräberäl agnegnan iddob me ingnen täräp me.}{The dancers' grass skirts swung nicely during the night dance.}\sensenumber{2}\definition{tailfeather}\example{Kakayam pa bo pite da mer tutpi dag.}{The tailfeathers of a bird of paradise are nice and long.}\sensenumber{2}\definition{under the skirt}\subentry{\headword{pitepite}\pos{adv.}\definition{under the skirt}}}
\entry{pite2}{\headword{pite2}\pos{n.}\sensenumber{2}\definition{large winged ankom}}
\entry{Pitepo}{\headword{Pitepo}\pos{pn.}\sensenumber{2}\definition{female personal name}}
\entry{pitkae}{\headword{pitkae}\pos{S vt.}\sensenumber{2}\definition{to untie}\example{Ngäna mɨnyi bomällkam a obo nying kollop tär de bäpitkaeneg.}{I will bend down and untie his sandal laces.}\allomorph{pätkae}}
\entry{pitpit}{\headword{pitpit}\pos{n.}\sensenumber{2}\definition{sugarcane}\etymology{from Tok Pisinpitpit}}
\entry{pitratra}{\headword{pitratra}\pos{n.}\sensenumber{2}\definition{masked lapwingKälläm me giddollag pa dan. (It's a bird that lives in ponds.)}}
\entry{pitt}{\headword{pitt}\pos{n.}\sensenumber{1}\definition{arrowhead hafting string}\sensenumber{2}\definition{band}\example{ttang pitt, ttäle pitt}{bracelet, leg band}}
\entry{pittbudar}{\headword{pittbudar}\pos{n.}\sensenumber{2}\definition{type of edible grub found in the bush and grassland}}
\entry{pittpitt}{\headword{pittpitt}\pos{S vt.}\sensenumber{1}\definition{to weave}\example{Sana dor de adingoll däpittaemeya.}{We weaved the sago stalks like this.}\sensenumber{2}\definition{to sew, stitch}\example{Amo umllang dan pittpitt a?}{Who here knows how to sew?}\example{Pittnen de dängkaemnegän.}{She started to stitch (the wounds).}\allomorph{pitt}}
\entry{piya}{\headword{piya}\pos{n.}\sensenumber{2}\definition{type of flowering plant}\sensenumber{2}\definition{planting a piya plant as a gesture of peace}\sensenumber{2}\definition{planting a piya plant as a gesture of peace}\subentry{\headword{piya gany}\pos{n.}\definition{planting a piya plant as a gesture of peace}}\subentry{\headword{piya ibeny}\pos{n.}\definition{planting a piya plant as a gesture of peace}}}
\entry{piye}{\headword{piye}\pos{n.}\sensenumber{1}\definition{pus}\sensenumber{2}\definition{sperm}}
\entry{piyepiye}{\headword{piyepiye}\pos{n.}\sensenumber{2}\definition{blister}\etymology{redup. of piye}}
\entry{piyupiyu}{\headword{piyupiyu}\pos{n.}\sensenumber{2}\definition{large vine that grows from tree to tree in the bush}\sensenumber{2}\definition{type of grubWälläng me ddäddäg ma budar dan. (It's an edible grub found in the bush.)}\subentry{\headword{piyupiyu budar}\pos{n.}\definition{type of grubWälläng me ddäddäg ma budar dan. (It's an edible grub found in the bush.)}}}
\entry{Pizi}{\headword{Pizi}\pos{pn.}\sensenumber{2}\definition{Pizi (toponym)}}
\entry{Pizin}{\headword{Pizin}\pos{pn.}\sensenumber{2}\definition{Tok Pisin}}
\entry{pizin}{\headword{pizin}\variant{sp. var. of}{Pizin}}
\entry{pɨddab}{\headword{pɨddab}\variant{var. of}{päddab}}
\entry{pɨnpɨn}{\headword{pɨnpɨn}\variant{var. of}{pänpän}}
\entry{pɨnyapɨnye}{\headword{pɨnyapɨnye}\pos{n.}\sensenumber{2}\definition{area with burnt grass}\example{Ngäna pɨnyapɨnye koenmäll e ibi allan.}{I'm going to go hunting in the burnt grass area.}}
\entry{plaig}{\headword{plaig}\pos{n.}\sensenumber{2}\definition{flag}}
\entry{plein}{\headword{plein}\variant{sp. var. of}{plen}}
\entry{plen}{\headword{plen}\pos{n.}\sensenumber{2}\definition{airplane}\sensenumber{2}\definition{airstrip}\etymology{from Englishplane}\subentry{\headword{plenngätt}\pos{n.}\definition{airstrip}}}
\entry{plenplen}{\headword{plenplen}\pos{n.}\sensenumber{2}\definition{pinwheel}\etymology{redup. of plen}}
\entry{plengg}{\headword{plengg}\pos{S vi.}\sensenumber{1}\definition{to die}\example{Bogo däplewän.}{He died.}\sensenumber{2}\definition{to cut down, cut off}\example{Ngäna llo de plengg eran.}{I am cutting down the tree.}\allomorph{ple}\allomorph{mpleg}\allomorph{mple}}
\entry{pllaengän}{\headword{pllaengän}\pos{v.}\sensenumber{2}\definition{strike}\allomorph{pllaeng}}
\entry{pllulle}{\headword{pllulle}\variant{var. of}{pällulle}}
\entry{pllulleaga}{\headword{pllulleaga}\pos{n.}\sensenumber{2}\definition{the seventh and final stage of sago growth during which the pith will not yield any starch}}
\entry{pllulli}{\headword{pllulli}\variant{var. of}{pällulle}}
\entry{pllutt}{\headword{pllutt}\pos{n.}\sensenumber{2}\definition{type of vine with fruit that are yellow when ripe; children eat them}}
\entry{po1}{\headword{po1}\pos{n.}\sensenumber{2}\definition{mound of dirt around a plant}\example{Nae po de gagäll a mudan.}{Don't ruin the sweet potato mound.}}
\entry{po2}{\headword{po2}\pos{num.}\sensenumber{2}\definition{four (English numeral; also general)}\etymology{from Englishfour}}
\entry{po3}{\headword{po3}\pos{A vi.}\sensenumber{2}\definition{to block an animal from escaping}\example{Bongo ngattong e po ag.}{You, block in front.}}
\entry{po4}{\headword{po4}\pos{S vt.}\sensenumber{2}\definition{to pierce}\example{Mälläng e dapoeaemeyo.}{They pierced noses.}}
\entry{poapoa}{\headword{poapoa}\pos{mod.}\sensenumber{1}\definition{light (in weight)}\sensenumber{2}\definition{light, bright}\example{Bädab pwapwa agnan.}{Dawn was breaking.}\sensenumber{3}\definition{easy}\example{Poapoa bogon.}{It will be easier.}}
\entry{pob}{\headword{pob}\pos{n.}\sensenumber{3}\definition{savanna}}
\entry{pobllem}{\headword{pobllem}\pos{mod.}\sensenumber{3}\definition{smooth}\example{pobllem ibima}{smooth path}}
\entry{Podar}{\headword{Podar}\variant{var. of}{Podare}}
\entry{Podare}{\headword{Podare}\pos{pn.}\sensenumber{3}\definition{Podare (Wipi-speaking village in Oriomo-Bituri Rural LLG)}}
\entry{poder}{\headword{poder}\pos{n.}\sensenumber{3}\definition{scapula, shoulder blade}}
\entry{podd}{\headword{podd}\pos{n.}\sensenumber{3}\definition{bald head, baldness}\example{Zäme podd a gongesän.}{He is already bald.}}
\entry{poddem}{\headword{poddem}\pos{n.}\sensenumber{3}\definition{young or small mammal (e.g. deer, wallaby)}}
\entry{poddpodd}{\headword{poddpodd}\pos{n.}\sensenumber{3}\definition{plain, field, clearing}}
\entry{Pogo}{\headword{Pogo}\pos{pn.}\sensenumber{3}\definition{Pogo (toponym)}}
\entry{pogpog}{\headword{pogpog}\pos{n.}\sensenumber{3}\definition{cockroach}}
\entry{Pol}{\headword{Pol}\pos{pn.}\sensenumber{3}\definition{male personal name}}
\entry{Polin}{\headword{Polin}\pos{pn.}\sensenumber{3}\definition{female personal name}}
\entry{polis}{\headword{polis}\pos{n.}\sensenumber{3}\definition{police}\example{Polis a namllamallo ge lla de.}{Police seized this man.}\etymology{from Englishpolice}}
\entry{Poll}{\headword{Poll}\pos{pn.}\sensenumber{3}\definition{male personal name}}
\entry{polle}{\headword{polle}\pos{n.}\sensenumber{3}\definition{fenceOtät papek ma ttängäm me ddäddäg lelang att. (To protect food in the garden from scary animals.)}\example{Ubi ttängäm kuddäll me polle de kame däkätteyo.}{They fenced the old garden.}\sensenumber{3}\definition{base of fence}\sensenumber{3}\definition{small fence}\subentry{\headword{polle bo}\pos{n.}\definition{base of fence}}\subentry{\headword{pollepolle}\pos{n.}\definition{small fence}}}
\entry{pollgo}{\headword{pollgo}\pos{n.}\sensenumber{3}\definition{frog}\example{Pollgo da täl ik mi giddollag daeya.}{The frog lived in the bamboo.}}
\entry{pollgo suwe}{\headword{pollgo suwe}\pos{n.}\sensenumber{3}\definition{dirt left on skin after coming out of the water}\example{Bäne pätt a pollgo suwe peyang dan.}{Your body is covered in dirt from the water.}\etymology{lit. 'frog urine'}}
\entry{pollgo ttatta kuttang}{\headword{pollgo ttatta kuttang}\pos{n.}\sensenumber{3}\definition{weaving pattern with concentric diamonds}}
\entry{pollo}{\headword{pollo}\pos{mod.}\sensenumber{3}\definition{young}\example{pollo täräp}{youth}\example{llɨg pollo}{young boy}}
\entry{pollon}{\headword{pollon}\pos{n.}\sensenumber{3}\definition{type of small bush}\example{Bogo gotarnän pollon me.}{He was sleeping in a small bush.}\sensenumber{3}\definition{nonsingular form of pollon}\subentry{\headword{pollonpollon}\pos{n.}\definition{nonsingular form of pollon}}}
\entry{pollon bällabollott}{\headword{pollon bällabollott}\pos{n.}\sensenumber{3}\definition{type of big taro}}
\entry{pollpoll}{\headword{pollpoll}\pos{S vt.}\sensenumber{3}\definition{to bark (at)}\example{Ge däräng a sämell ulle de dupollneyo.}{These dogs were barking at the big pig.}\example{Kwalde bogo ada gongnomenynän ada ka däräng a pollnenang me erallo däbe sämell de.}{Kwalde thought that the dogs were still barking at the pig.}\allomorph{poll}}
\entry{pom}{\headword{pom}\pos{n.}\sensenumber{3}\definition{type of long yam with a white interior, thorns, and hairs}}
\entry{poma}{\headword{poma}\pos{n.}\sensenumber{3}\definition{type of pandanus with leaves that women weave into matsNying kemibiag, wälläng o wallemäg me päddabag dan. (With many leg-shaped roots, it grows in the bush or by creeks.)}}
\entry{pomana}{\headword{pomana}\pos{n.}\sensenumber{3}\definition{fishing style in which men climb trees and shoot with ddangol spears}}
\entry{pomer}{\headword{pomer}\pos{n.}\sensenumber{3}\definition{whistle}\example{Ge lla da pomer alle bandra nällɨtan.}{This man whistled a song.}}
\entry{pomila}{\headword{pomila}\pos{n.}\sensenumber{3}\definition{type of citrus tree with big, edible fruit}}
\entry{pondo}{\headword{pondo}\pos{n.}\sensenumber{3}\definition{type of tree that grows in the swamp and old gardens with yellow flowers and pod-shaped fruit with inedible seeds}}
\entry{Pondollowang}{\headword{Pondollowang}\pos{pn.}\sensenumber{3}\definition{Pondollowang (camping place)}}
\entry{ponong}{\headword{ponong}\pos{n.}\sensenumber{3}\definition{type of tree that grows in grassland with white flowers and small green fruit and wood used for postsTtoe de kängkäm ma dan nane we. (The bark is squeezed and the liquid is drunk.)}}
\entry{Ponongllowang}{\headword{Ponongllowang}\pos{pn.}\sensenumber{3}\definition{Ponongllowang (toponym)}}
\entry{ponor}{\headword{ponor}\pos{S vi.}\sensenumber{3}\definition{to start running}\example{Angde ubi up wo pätt dowae e gogeyo, llamda da oba pate eraeranyang dipnurän.}{When they got close to the ripe banana tree, the old man began to run after them, yelling.}\allomorph{pnur}\allomorph{pänur}}
\entry{ponganem}{\headword{ponganem}\pos{n.}\sensenumber{3}\definition{type of small yam with a white interior, hairs, and thorns}}
\entry{Pongarke}{\headword{Pongarke}\pos{pn.}\sensenumber{3}\definition{Pongariki (Nambo-speaking village in Morehead Rural LLG)}}
\entry{pongoll}{\headword{pongoll}\pos{n.}\sensenumber{3}\definition{type of yam with a white interior and no hairs or thorns}}
\entry{poo}{\headword{poo}\variant{sp. var. of}{po2}}
\entry{pop}{\headword{pop}\pos{n.}\sensenumber{3}\definition{hole}\example{Tupi ttang llɨpɨt de pop me nowanseg.}{Put your long finger in the hole.}\sensenumber{1}\definition{holey, having a hole}\example{Ine kutt da bo lid a popang dan.}{The lid of the water container has a hole.}\sensenumber{2}\definition{rubbish, darn}\etymology{pop + =ang}\subentry{\headword{popang}\pos{mod.}\definition{holey, having a hole}}}
\entry{pope}{\headword{pope}\pos{kin.}\sensenumber{2}\definition{uncle (one's mother's brother)Mäg bo mangmang. (Mother's brothers.)}\example{Lomae obo pope mu de zäme dangesän.}{Lomae already paid her uncle her birth payment.}\sensenumber{2}\definition{uncle payment}\subentry{\headword{pope mu}\pos{n.}\definition{uncle payment}}}
\entry{popel}{\headword{popel}\variant{var. of}{popell}}
\entry{popell}{\headword{popell}\pos{n.}\sensenumber{1}\definition{type of introduced bananaDdone ulle pättang dan, käp a obo pällämpälläm dag, ako o meae otänan ma dag, a kire da ddone yuma dag. Ako pätt a obo täräpänan ma da popel e ankom peyang kaipnen e mer moko dan. (It's not tall, the fruit are white, and when ripe, its fruit is eaten, and when unripe, it's not cooked. Also, its trunk is cut for popel and chewing with ants; it's tasty.)}\sensenumber{2}\definition{dish consisting of ants and various types of bananas (incl. popell)}}
\entry{poper}{\headword{poper}\pos{mod.}\sensenumber{2}\definition{afraid, startled, scared}\example{Pa poper gogon a ma ik atta gogezänän.}{The bird got scared and got out from inside the house.}\example{Lelang eka dändär att me tikop poper agne.}{The heart will be racing after hearing a scary story.}\sensenumber{2}\definition{to scare, startle, frighten}\example{Poper naengkoegan.}{He scared her.}\example{Mänmän de poper inkoegag bongttägän.}{He will arrive suddenly, startling the girls.}\sensenumber{2}\definition{to surprise}\example{Ngäna obom poper nangkwigan.}{I surprised her.}\subentry{\headword{poper ingkoeg}\pos{S vt.}\definition{to scare, startle, frighten}}\subentry{\headword{poper kwingg}\pos{S vt.}\definition{to surprise}}}
\entry{popllem}{\headword{popllem}\pos{S vi.}\sensenumber{2}\definition{to flap}\example{Kaptte da wel me angde popllem anggan, dedam era mamaldae pätnan anggan.}{When the clothes flap in the wind, they dry quickly.}}
\entry{popllo}{\headword{popllo}\pos{v.}\sensenumber{2}\definition{twist}}
\entry{popo1}{\headword{popo1}\pos{S vt.}\sensenumber{1}\definition{to sharpen}\example{Dompa de däpoaemneyo.}{They were sharpening the arrows.}\example{Ge pensil a popoatt dan.}{This pencil is sharpened.}\sensenumber{2}\definition{to carve}\example{Gall a llo popoatt dan.}{The canoe is carved wood.}\example{Bägäl de popo eran.}{He shapes the bow.}\sensenumber{2}\definition{pencil sharpener}\allomorph{po}\allomorph{pu}\subentry{\headword{ponen ma}\pos{n.}\definition{pencil sharpener}}}
\entry{popo2}{\headword{popo2}\pos{n.}\sensenumber{2}\definition{flower}}
\entry{porak}{\headword{porak}\pos{n.}\sensenumber{2}\definition{arrow\textbackslash_type}}
\entry{porma}{\headword{porma}\pos{n.}\sensenumber{2}\definition{traditional dish consisting of meat on top of sago cooked in sago or banana leaves}}
\entry{pos}{\headword{pos}\pos{n.}\sensenumber{2}\definition{post}\etymology{from Englishpost}}
\entry{poses}{\headword{poses}\pos{n.}\sensenumber{1}\definition{martin (fairy, tree)Yogoll me tongoeang pa dan. (It's a bird that plays in the rain.)}\sensenumber{2}\definition{Pacific swift}}
\entry{pot}{\headword{pot}\pos{n.}\sensenumber{1}\definition{tip, end, point; base (of yam)}\sensenumber{2}\definition{talent, gift}\sensenumber{3}\definition{vagina of a mammal}}
\entry{Pot Mosbi}{\headword{Pot Mosbi}\pos{pn.}\sensenumber{3}\definition{Port Moresby (the capital city of Papua New Guinea)}\etymology{from Tok PisinPot Mosbi}}
\entry{poti}{\headword{poti}\pos{num.}\sensenumber{3}\definition{forty}\etymology{from Englishforty}}
\entry{potin}{\headword{potin}\pos{num.}\sensenumber{3}\definition{fourteen (English numeral)}\etymology{from Englishfourteen}}
\entry{potkam}{\headword{potkam}\pos{n.}\sensenumber{3}\definition{type of cultivated tree that also grows in the savanna; treats cough and aches}}
\entry{potne kätt}{\headword{potne kätt}\pos{n.}\sensenumber{3}\definition{bivalve\textbackslash_type}}
\entry{potopoto}{\headword{potopoto}\pos{n.}\sensenumber{3}\definition{type of tree that grows near swamps and creeks with yellow and white flowers and fruit that falls on the ground; animals eat itLlo ulle dan. Obo igi ttoe de wabeb ma dan, kängkäm ma dan nanen e. Itrell sapasapang atta lla da mänyi ai bogon. (It's a big tree. Its bark is beaten, squeezed, and the liquid is drunk. People will heal from various ailments.)}}
\entry{potpot}{\headword{potpot}\pos{S vt.}\sensenumber{3}\definition{to slice open}\example{Bogo sɨmell käm dupotän.}{He sliced open the pig stomach.}\allomorph{pot}}
\entry{Pottängäm}{\headword{Pottängäm}\pos{pn.}\sensenumber{3}\definition{Pottängäm (camp and garden place)}}
\entry{Praede}{\headword{Praede}\variant{sp. var. of}{praedde}}
\entry{praede}{\headword{praede}\variant{sp. var. of}{praedde}}
\entry{praedde}{\headword{praedde}\pos{n.}\sensenumber{3}\definition{Friday}\etymology{from EnglishFriday}}
\entry{praemari skul}{\headword{praemari skul}\pos{n.}\sensenumber{3}\definition{primary school}\etymology{from Englishprimary}}
\entry{praemeri skul}{\headword{praemeri skul}\variant{sp. var. of}{praemari skul}}
\entry{pre}{\headword{pre}\variant{fast speech var. of}{päre}}
\entry{prep}{\headword{prep}\pos{n.}\sensenumber{3}\definition{preparatory school}\etymology{clipping of Englishpreparatory}}
\entry{pri}{\headword{pri}\pos{mod.}\sensenumber{3}\definition{free}\etymology{from Englishfree}}
\entry{priede}{\headword{priede}\variant{sp. var. of}{praedde}}
\entry{Prisila}{\headword{Prisila}\variant{sp. var. of}{Priscilla}}
\entry{Priski}{\headword{Priski}\pos{pn.}\sensenumber{3}\definition{female personal name}}
\entry{Priscilla}{\headword{Priscilla}\pos{pn.}\sensenumber{3}\definition{female personal name}}
\entry{probens}{\headword{probens}\pos{n.}\sensenumber{3}\definition{province}\etymology{from Englishprovince}}
\entry{pu}{\headword{pu}\pos{n.}\sensenumber{1}\definition{floating grass or island in swamp}\example{Kapalla pu me mälla da mondre anggan.}{The woman gardens on the floating kapalla grass.}\sensenumber{2}\definition{swamp gardenKapalla, kitar o ttalme ada toko me ttängäm. (A garden on kapalla, kitar, or ttalme grass.)}\example{Oba pu mi biye de ddone itrel a anggan.}{The disease didn't strike the taro in their swamp garden.}}
\entry{pud}{\headword{pud}\pos{n.}\sensenumber{2}\definition{end (of a long object)}\example{pop peyang pud}{the end with a hole}}
\entry{puder}{\headword{puder}\pos{n.}\sensenumber{2}\definition{type of long grass}}
\entry{pudrell}{\headword{pudrell}\variant{sp. var. of}{pädrall}}
\entry{pudd1}{\headword{pudd1}\pos{n.}\sensenumber{2}\definition{place with flattened grass where wallabies sleep}}
\entry{pudd2}{\headword{pudd2}\pos{n.}\sensenumber{2}\definition{taro shoot}}
\entry{pugupugu}{\headword{pugupugu}\pos{n.}\sensenumber{2}\definition{collared imperial pigeonYäbäd me ekawang. (It's a bird that sings when it's sunny.)}}
\entry{puikmetutu}{\headword{puikmetutu}\pos{n.}\sensenumber{2}\definition{type of gecko}}
\entry{Puinde}{\headword{Puinde}\pos{pn.}\sensenumber{2}\definition{male personal name}}
\entry{pukong}{\headword{pukong}\pos{mod.}\sensenumber{2}\definition{thick}\example{Ge peba da pukong dan.}{This paper is thick.}}
\entry{puku}{\headword{puku}\pos{n.}\sensenumber{2}\definition{type of arrow}}
\entry{puku kubull}{\headword{puku kubull}\pos{n.}\sensenumber{2}\definition{type of bush wallaby with white tail}}
\entry{pullkom}{\headword{pullkom}\pos{n.}\sensenumber{2}\definition{tailfeather}}
\entry{pumi}{\headword{pumi}\pos{mod.}\sensenumber{2}\definition{exhausting, tiring, strenuous}\example{Pumi melem de nangesallo.}{They did exhausting work.}}
\entry{pumkak}{\headword{pumkak}\pos{n.}\sensenumber{2}\definition{type of sago}}
\entry{pungg2}{\headword{pungg2}\pos{S vi.}\sensenumber{2}\definition{to get sick, be in pain}\example{Da bam kungge da naddägän, bongo mɨnyi ttattlle ampug.}{If a spider bites you, you will get very sick.}\allomorph{mpug}}
\entry{pupulläkäm}{\headword{pupulläkäm}\pos{n.}\sensenumber{2}\definition{type of mushroomAp me päddabag, ddäddäg ma dan. (It grows in the grassland; it's edible.)}}
\entry{purta}{\headword{purta}\variant{var. of}{pärta}}
\entry{pus}{\headword{pus}\pos{n.}\sensenumber{2}\definition{cat}}
\entry{put}{\headword{put}\pos{n.}\sensenumber{2}\definition{type of brown barkLlo ttoe kädnanatt ma paknen e. (Tree bark removed for walling the house.)}\example{Oba ma da pall put alle papekatt dan.}{Their house is walled with pall bark walling.}}
\entry{putt}{\headword{putt}\pos{n.}\sensenumber{1}\definition{hoof}\example{Ddia da putt gottkewän.}{The deer folded its hooves.}\sensenumber{2}\definition{six (yam counting numeral; 6\textbackslashtextasciicircum1)}}
\entry{putt kämamkämam}{\headword{putt kämamkämam}\pos{n.}\sensenumber{2}\definition{type of sago bundle wrapped with sago leaves and tied at one end}\etymology{lit. 'folded hoof'}}
\entry{puyem}{\headword{puyem}\pos{n.}\sensenumber{2}\definition{hunting event in which one person hits the ground with a short stick}}
\entry{puzi}{\headword{puzi}\pos{n.}\sensenumber{2}\definition{type of bird with crownBädab e ekawang ddäddäg ma pa ulle dan, ekaklle me ibiag dan. (It's a big, edible bird that sings at dawn; it travels on land.)}}
\entry{pwapwa}{\headword{pwapwa}\variant{sp. var. of}{poapoa}}
\lettersection{R}
\entry{raba}{\headword{raba}\variant{sp. var. of}{räba}}
\entry{Rabaol}{\headword{Rabaol}\variant{sp. var. of}{Rabaul}}
\entry{Rabaul}{\headword{Rabaul}\pos{pn.}\sensenumber{2}\definition{Rabaul (toponym)}}
\entry{rabis}{\headword{rabis}\pos{n.}\sensenumber{2}\definition{rubbish, trash}\etymology{from Englishrubbish}}
\entry{Ranky}{\headword{Ranky}\pos{pn.}\sensenumber{2}\definition{male personal name}}
\entry{rarae}{\headword{rarae}\pos{n.}\sensenumber{1}\definition{hunting technique in which hunters line up to cover ground in the absence of dogs}\sensenumber{2}\definition{fishing technique in which ladies line up with nets}}
\entry{Raroge}{\headword{Raroge}\pos{pn.}\sensenumber{2}\definition{Raroge (toponym)}}
\entry{Rasol}{\headword{Rasol}\pos{pn.}\sensenumber{2}\definition{male personal name}}
\entry{rata}{\headword{rata}\pos{n.}\sensenumber{2}\definition{type of flowering plant with red bracts and nectar that attracts insects}}
\entry{Raynold}{\headword{Raynold}\pos{pn.}\sensenumber{2}\definition{male personal name}}
\entry{räba}{\headword{räba}\pos{n.}\sensenumber{2}\definition{rubber; eraser}\etymology{from Englishrubber}}
\entry{Räba blok}{\headword{Räba blok}\pos{pn.}\sensenumber{2}\definition{Rubber block (toponym)}}
\entry{rädräd}{\headword{rädräd}\pos{n.}\sensenumber{2}\definition{type of grey fish that lives in the swamp}}
\entry{redi}{\headword{redi}\pos{A vt.}\sensenumber{2}\definition{to prepare, make ready}\etymology{from Englishready}}
\entry{Redley}{\headword{Redley}\pos{pn.}\sensenumber{2}\definition{male personal name}}
\entry{Reend}{\headword{Reend}\pos{pn.}\sensenumber{2}\definition{male personal name}}
\entry{Regina}{\headword{Regina}\pos{pn.}\sensenumber{2}\definition{female personal name}}
\entry{Reks}{\headword{Reks}\pos{pn.}\sensenumber{2}\definition{male personal name}}
\entry{Rena}{\headword{Rena}\pos{pn.}\sensenumber{2}\definition{female personal name}}
\entry{retam}{\headword{retam}\pos{n.}\sensenumber{2}\definition{type of big yam}}
\entry{Rex}{\headword{Rex}\variant{sp. var. of}{Reks}}
\entry{Rhoda}{\headword{Rhoda}\pos{pn.}\sensenumber{2}\definition{female personal name}}
\entry{Rhodda}{\headword{Rhodda}\variant{var. of}{Rhoda}}
\entry{Rind}{\headword{Rind}\pos{pn.}\sensenumber{2}\definition{male personal name}}
\entry{rir}{\headword{rir}\pos{n.}\sensenumber{2}\definition{type of medium-sized bamboo}}
\entry{Richard}{\headword{Richard}\pos{pn.}\sensenumber{2}\definition{male personal name}}
\entry{Roaele}{\headword{Roaele}\pos{pn.}\sensenumber{2}\definition{male personal name}}
\entry{Roak}{\headword{Roak}\pos{pn.}\sensenumber{2}\definition{male personal name}}
\entry{Robae}{\headword{Robae}\pos{pn.}\sensenumber{2}\definition{female personal name}}
\entry{Robai}{\headword{Robai}\variant{sp. var. of}{Robae}}
\entry{Roda}{\headword{Roda}\variant{var. of}{Rhoda}}
\entry{rol}{\headword{rol}\pos{n.}\sensenumber{2}\definition{type of hairy caterpillar}}
\entry{rolkutt}{\headword{rolkutt}\pos{n.}\sensenumber{2}\definition{crawling grass}}
\entry{Rom}{\headword{Rom}\pos{pn.}\sensenumber{2}\definition{Rome}}
\entry{rop}{\headword{rop}\pos{n.}\sensenumber{2}\definition{rope}\etymology{from Englishrope}}
\entry{Ros}{\headword{Ros}\variant{sp. var. of}{Rose}}
\entry{Rose}{\headword{Rose}\pos{pn.}\sensenumber{2}\definition{female personal name}}
\entry{Rosela}{\headword{Rosela}\pos{pn.}\sensenumber{2}\definition{female personal name}}
\entry{Rowak}{\headword{Rowak}\pos{pn.}\sensenumber{2}\definition{male personal name}}
\entry{Rual}{\headword{Rual}\pos{pn.}\sensenumber{2}\definition{Rual (toponym)}}
\entry{rubi}{\headword{rubi}\pos{n.}\sensenumber{1}\definition{type of long yam with a white interior, white skin, and no thorns}\sensenumber{2}\definition{type of arrow}}
\entry{rullgoe}{\headword{rullgoe}\pos{S vt.}\sensenumber{2}\definition{to drag}\example{Tim ngänäm dangminggän ngämi däbe käza de tutu wi darullgoeya.}{Tim helped me drag that crocodile onto land.}}
\entry{rupi}{\headword{rupi}\pos{n.}\sensenumber{2}\definition{type of leaf}}
\entry{ruriruri}{\headword{ruriruri}\pos{n.}\sensenumber{2}\definition{earthquake}}
\lettersection{S}
\entry{sabi}{\headword{sabi}\pos{n.}\sensenumber{1}\definition{law, rule}\example{Ngämi bam da batrameya, abo bongo ge sabi de nongkollmäll.}{We will take you, but you must follow these rules.}\sensenumber{2}\definition{taboo}\example{Ngämo mänang a ngänäm de aläm bagneyo ada, ngämo bin ttam a mudan, sabi.}{My in-laws will be warning me that it's taboo to call me by my name.}\etymology{possibly from Tok Pisin save 'habit', ultimately from Portuguese saber 'know'}}
\entry{sabol}{\headword{sabol}\pos{n.}\sensenumber{2}\definition{shovel, spadeTtängäm gllaenen ma dan. (It's for digging the garden.)}\etymology{from Englishshovel}}
\entry{Sadua}{\headword{Sadua}\pos{pn.}\sensenumber{2}\definition{male personal name}}
\entry{Saduwa}{\headword{Saduwa}\variant{var. of}{Sadua}}
\entry{sae}{\headword{sae}\pos{S vt.}\sensenumber{1}\definition{to close, cover}\example{Nga bongo ikop nasǃ}{You, close your eyes!}\example{Däräng a ikop dasän.}{The dog closed his eyes.}\sensenumber{2}\definition{to extinguish, put out}\example{Yu da saemeny dan.}{The fire isn't put out.}\sensenumber{2}\definition{closed}\example{Däräng täräpang dagirnän ikop saeyang.}{The dog stayed there a long time, eyes closed.}\allomorph{s}\etymology{sae + =ang}\subentry{\headword{saeyang}\pos{mod.}\definition{closed}}}
\entry{saem}{\headword{saem}\pos{v.}\sensenumber{2}\definition{to babble}\example{Bogo saem allan.}{He's babbling.}}
\entry{Saemon}{\headword{Saemon}\pos{pn.}\sensenumber{2}\definition{male personal name}}
\entry{saen}{\headword{saen}\pos{n.}\sensenumber{2}\definition{sign}\example{Mälla da saen dangesän.}{The woman made a sign.}\etymology{from Englishsign}}
\entry{saeten}{\headword{saeten}\pos{pn.}\sensenumber{2}\definition{Satan}\etymology{from EnglishSatan}}
\entry{sagwan}{\headword{sagwan}\pos{n.}\sensenumber{2}\definition{woodworking toolGall tɨnen ma dan. (It's for hollowing canoes.)}}
\entry{Saisiato}{\headword{Saisiato}\pos{pn.}\sensenumber{2}\definition{female personal name}}
\entry{sakar}{\headword{sakar}\pos{n.}\sensenumber{2}\definition{type of edible pandanusKäp tupiang ulle mab dan. (A big pandanus with a long fruit.)}}
\entry{Sakoyaratt}{\headword{Sakoyaratt}\pos{pn.}\sensenumber{2}\definition{Sakoyaratt (toponym)}}
\entry{saks}{\headword{saks}\pos{n.}\sensenumber{2}\definition{socks}\etymology{from Englishsocks}}
\entry{Sali}{\headword{Sali}\pos{pn.}\sensenumber{2}\definition{male personal name}}
\entry{Salome}{\headword{Salome}\pos{pn.}\sensenumber{2}\definition{female personal name}}
\entry{Saly}{\headword{Saly}\variant{sp. var. of}{Sali}}
\entry{Sam}{\headword{Sam}\pos{pn.}\sensenumber{2}\definition{male personal name}}
\entry{Samae}{\headword{Samae}\pos{pn.}\sensenumber{2}\definition{male personal name}}
\entry{Samari}{\headword{Samari}\pos{pn.}\sensenumber{2}\definition{Samari (toponym)}}
\entry{Samat}{\headword{Samat}\pos{pn.}\sensenumber{2}\definition{female personal name}}
\entry{sambuag}{\headword{sambuag}\pos{n.}\sensenumber{2}\definition{lobster; prawn}}
\entry{samoa}{\headword{samoa}\pos{n.}\sensenumber{2}\definition{type of introduced bananaUlle pättang dan, obo däg a yuwog dag. Obo käp a sägäsägäd dag, o me otät ma dan, ako kire da yu ma dan. (It has a big trunk; its bunches are plentiful. Its fruit are yellow; when ripe, it's eaten, and when unripe, it's cooked.)}}
\entry{Samson}{\headword{Samson}\pos{pn.}\sensenumber{2}\definition{male personal name}}
\entry{Samuel}{\headword{Samuel}\pos{pn.}\sensenumber{2}\definition{male personal name}}
\entry{sana}{\headword{sana}\pos{n.}\sensenumber{2}\definition{sago}\example{Sana tɨt iran.}{She is pounding the sago.}\example{Sana ttäkoe eran.}{She is chopping the sago tree.}\sensenumber{2}\definition{sago grub (larva of black palm weevil)Sana patt me o sana utt me daden, ddäddäg ma dan sana peyang. (They live in sago trunks and shoots; it's eaten with sago.)}\sensenumber{2}\definition{sago grub (larva of black palm weevil)}\sensenumber{2}\definition{the second stage of sago growth during which the plant is \textbackslashtextasciitilde3 m tall}\sensenumber{2}\definition{type of mushroomSanateya me päddäbag, ddäddäg ma dan. (It grows on sago; it's edible.)}\sensenumber{2}\definition{sago pith}\sensenumber{1}\definition{area where sago grows}\example{Sanawang de yu da dättämän Kibobma me.}{The fire burned the sago growing area in Kibobma.}\sensenumber{2}\definition{parable}\etymology{sana + =ang}\subentry{\headword{sana budar}\pos{n.}\definition{sago grub (larva of black palm weevil)Sana patt me o sana utt me daden, ddäddäg ma dan sana peyang. (They live in sago trunks and shoots; it's eaten with sago.)}}\subentry{\headword{sana marmar}\pos{n.}\definition{sago grub (larva of black palm weevil)}}\subentry{\headword{sana wuttang}\pos{n.}\definition{the second stage of sago growth during which the plant is \textbackslashtextasciitilde3 m tall}}\subentry{\headword{sanalläkäm}\pos{n.}\definition{type of mushroomSanateya me päddäbag, ddäddäg ma dan. (It grows on sago; it's edible.)}}\subentry{\headword{sanateya}\pos{n.}\definition{sago pith}}\subentry{\headword{sanawang}\pos{n.}\definition{area where sago grows}}}
\entry{sana tätäkang}{\headword{sana tätäkang}\pos{n.}\sensenumber{2}\definition{type of spear}}
\entry{sana wutwut}{\headword{sana wutwut}\pos{n.}\sensenumber{2}\definition{sago dish cooked with pieces of banana leaf between the layers}}
\entry{sanasana}{\headword{sanasana}\pos{n.}\sensenumber{2}\definition{type of edible sagoSanawang me päddabag za dan, ttam kloklong.}\etymology{redup. of sana}}
\entry{sandana}{\headword{sandana}\pos{n.}\sensenumber{2}\definition{type of black palm; in this middle stage, the pith is chewed like sugarcane during the dry season for its moisture}}
\entry{sande}{\headword{sande}\pos{n.}\sensenumber{1}\definition{Sunday}\sensenumber{2}\definition{week}\example{Lla da ttongo sande makäp me mɨnyi täre we gonsärbemamalle.}{People will be preparing for the feast for a week.}\etymology{from EnglishSunday}}
\entry{Sandra}{\headword{Sandra}\pos{pn.}\sensenumber{2}\definition{female personal name}}
\entry{Sanford}{\headword{Sanford}\pos{pn.}\sensenumber{2}\definition{male personal name}}
\entry{sanga}{\headword{sanga}\pos{n.}\sensenumber{2}\definition{black-necked storkTukdae papllägag pa dan. (It's a bird that flies through the sky.)}}
\entry{sangapawi}{\headword{sangapawi}\pos{n.}\sensenumber{2}\definition{type of big, round yam with a white interior, white or red skin, hairs, and no thorns}}
\entry{saomasaoma}{\headword{saomasaoma}\pos{n.}\sensenumber{1}\definition{armpit}\sensenumber{2}\definition{dancing bandTtang o ttäle pitt dirom kom peyang inenatt ingong täräp me mättnan e. (Arm or leg bands woven with cassowary feathers to wear during dancing time.)}\example{De lla da gagäll ikopang agan saomasaoma mättmättatt a.}{That man looks good (lit. bad) in his armbands.}}
\entry{sapang}{\headword{sapang}\pos{mod.}\sensenumber{2}\definition{separate, apart, different; own, personal}\example{sapang ngättma we}{to a different place}\example{Ddob oba masamasar a eragaeya, ddob oba sapang eka da dadegaeya.}{Some of their ancestors, some of them had separate languages.}\example{Bogo kollba de sapang dowansegän.}{He put the fish aside.}\example{Ubi sapang dag a ngämi ade sapang dag.}{They are on their own and we (excl.) are also on our own.}\example{Puinde ade obo sapang nyängang dägagän.}{Puinde also had his personal bag.}\sensenumber{2}\definition{various, many different}\example{sapasapang ttängäm att lla}{people from many different places}\example{Ddob melem a sapasapang dadegän.}{There are a variety of other jobs.}\example{Iba eka da gullem sapasapang pallall e dan.}{Our (incl.) story is about various kinds of snakes.}\subentry{\headword{sapasapang}\pos{mod.}\definition{various, many different}}}
\entry{sapangsapang}{\headword{sapangsapang}\variant{var. of}{sapasapang}}
\entry{sapebllabllot}{\headword{sapebllabllot}\pos{n.}\sensenumber{2}\definition{type of taro}}
\entry{saper ine}{\headword{saper ine}\pos{n.}\sensenumber{2}\definition{clean waterKätt ine, yäbäd bang me ddone baddbeddag dan. (Clear water; it doesn't dry up during the dry season.)}}
\entry{sapiri}{\headword{sapiri}\pos{n.}\sensenumber{2}\definition{type of flowering plant}}
\entry{Sapusa}{\headword{Sapusa}\pos{pn.}\sensenumber{2}\definition{female personal name}}
\entry{Sara}{\headword{Sara}\pos{pn.}\sensenumber{2}\definition{female personal name}}
\entry{Sarah}{\headword{Sarah}\variant{sp. var. of}{Sera}}
\entry{Sarbi}{\headword{Sarbi}\pos{pn.}\sensenumber{2}\definition{female personal name}}
\entry{sare}{\headword{sare}\pos{n.}\sensenumber{2}\definition{type of big taro}}
\entry{sari}{\headword{sari}\pos{interj.}\sensenumber{2}\definition{sorry}\etymology{from Englishsorry}}
\entry{Sasa}{\headword{Sasa}\pos{pn.}\sensenumber{2}\definition{female personal name}}
\entry{sasapang}{\headword{sasapang}\variant{fast speech var. of}{sapasapang}}
\entry{Sasi}{\headword{Sasi}\pos{pn.}\sensenumber{2}\definition{female personal name}}
\entry{sat}{\headword{sat}\pos{n.}\sensenumber{2}\definition{widower}}
\entry{satade}{\headword{satade}\pos{n.}\sensenumber{2}\definition{Saturday}\etymology{from EnglishSaturday}}
\entry{saus}{\headword{saus}\pos{n.}\sensenumber{2}\definition{type of yam with a white or purple interior, thorns, and no hairs}}
\entry{Sawa}{\headword{Sawa}\pos{pn.}\sensenumber{2}\definition{male personal name}}
\entry{Sawapo}{\headword{Sawapo}\pos{pn.}\sensenumber{2}\definition{male personal name}}
\entry{sawasap}{\headword{sawasap}\pos{n.}\sensenumber{2}\definition{type of cultivated tree with edible yellow or green fruit}\etymology{from Englishsour sap}}
\entry{sawe}{\headword{sawe}\pos{mod.}\sensenumber{2}\definition{left}\example{Ngämlle sawe alle giri de nanttog.}{Give me the knife with your left hand.}}
\entry{Saweta}{\headword{Saweta}\pos{pn.}\sensenumber{2}\definition{Saweta (toponym)}}
\entry{sawis}{\headword{sawis}\pos{n.}\sensenumber{2}\definition{type of small yam with a long shape and purple interior}}
\entry{sawiya}{\headword{sawiya}\pos{n.}\sensenumber{2}\definition{little egretKollba ttonenang pa pällämpälläm dan. (It's a white bird that collects fish.)}}
\entry{sägädag}{\headword{sägädag}\pos{mod.}\sensenumber{2}\definition{yellow}}
\entry{sägädsägäd}{\headword{sägädsägäd}\variant{sp. var. of}{sägäsägäd}}
\entry{sägäsägäd}{\headword{sägäsägäd}\pos{col.}\sensenumber{2}\definition{yellow}\allomorph{sägädag}}
\entry{sägäsägäd manika}{\headword{sägäsägäd manika}\pos{n.}\sensenumber{2}\definition{yellow cassava}}
\entry{Sägrep}{\headword{Sägrep}\pos{pn.}\sensenumber{2}\definition{male personal name}}
\entry{säkar}{\headword{säkar}\pos{n.}\sensenumber{2}\definition{type of big palm tree that grows near big rivers with white flowers and hanging red fruit that cassowaries eatBägäl e popo ma dan. (It's sharpened into bows.)}}
\entry{säkmällsäkmäll}{\headword{säkmällsäkmäll}\pos{adv.}\sensenumber{2}\definition{limping}\example{Bogo säkmällsäkmäll ibi allan.}{He is walking with a limp.}}
\entry{sämell}{\headword{sämell}\variant{var. of}{sɨmell}}
\entry{sämongg}{\headword{sämongg}\pos{S vi.}\sensenumber{2}\definition{to feel}\example{Ngäna mermerang ansomägan.}{I feel fine.}\example{Obo pätt me gonsomonggän ada obo itrell de dektta llɨtt dägagän.}{He felt in his body that the doctor had stopped his illness.}\allomorph{nsämängg}\allomorph{nsemäg}\allomorph{nsomongg}\allomorph{nsomäg}\allomorph{nsomo}\allomorph{nsomängg}\allomorph{nsmomeny}}
\entry{sänasäne}{\headword{sänasäne}\pos{S vt.}\sensenumber{2}\definition{to take out}\example{Ngämi ddäddäg de mɨnyi bäsnaemeyo säspen att.}{We (excl.) will take out the meat from the pot.}\allomorph{sna}\allomorph{sne}}
\entry{sänd}{\headword{sänd}\pos{n.}\sensenumber{2}\definition{type of big yam with a white interior and few hairs}}
\entry{Sände}{\headword{Sände}\variant{sp. var. of}{sande}}
\entry{sände}{\headword{sände}\variant{sp. var. of}{Sände}}
\entry{säne}{\headword{säne}\pos{n.}\sensenumber{2}\definition{type of yam}}
\entry{sänge}{\headword{sänge}\pos{A vt.}\sensenumber{2}\definition{to betray}\example{Bina makäp me ttongo aeya mɨnyi ngänäm sänge bagän, ttongo aeya duwem allan ngämo peyang.}{One among you will betray me: the one who eats with me.}}
\entry{säpalek}{\headword{säpalek}\pos{n.}\sensenumber{2}\definition{type of bagNge ttam kutt o kito ttam nyäng iatt mälla da eraballe za kemibi bokomän. (Bag woven from coconut fronds or palm leaves with which women carry many things.)}\example{Bibiae mätta de kemibi era säpaleng alle de ikoman.}{Bibiae brought many yams with the spalek bag.}}
\entry{säpaleng}{\headword{säpaleng}\variant{dial. var. of}{säpalek}}
\entry{sära}{\headword{sära}\pos{n.}\sensenumber{1}\definition{tail}\example{Käza da kollba bo sära de däddägän.}{The crocodile ate the fish tail.}\sensenumber{2}\definition{end of a bow}\sensenumber{3}\definition{uncut grass skirt}\sensenumber{3}\definition{sacrum (bone)}\sensenumber{3}\definition{caudal fin}\sensenumber{3}\definition{mesothorax}\sensenumber{1}\definition{tip of tail}\sensenumber{2}\definition{lastborn, youngest}\sensenumber{2}\definition{lastborn, youngest}\subentry{\headword{sära mit}\pos{n.}\definition{sacrum (bone)}}\subentry{\headword{sära pakall}\pos{n.}\definition{caudal fin}}\subentry{\headword{sära pättkäp}\pos{n.}\definition{mesothorax}}\subentry{\headword{sära pot}\pos{n.}\definition{tip of tail}}\subentry{\headword{särapipi}\pos{n.}\definition{lastborn, youngest}}}
\entry{sära pipi}{\headword{sära pipi}\variant{sp. var. of}{särapipi}}
\entry{särämbae}{\headword{särämbae}\pos{S vi.}\sensenumber{1}\definition{to prepare, be prepared, be ready, get ready}\example{Tämamae ansärbemom!}{Everyone, get ready!}\example{Ngäna ada ngonongg allan ada malla bongo särämbaeatt dan.}{I think that you are not prepared.}\sensenumber{2}\definition{to prepare, arrange, get ready}\example{Ubi oba gall me dagwaeya, gull de dansärämbenegneyo.}{The two of them were in their canoe, preparing their nets.}\example{Ngäna obo pätt de nansärämbeyan au we.}{I prepared his body for burial.}\sensenumber{3}\definition{to fix, solve, resolve}\example{Ngämlle mokowang melem dan lla bo buddo särämbae.}{I enjoy the task of solving people's problems.}\allomorph{nsärämbe}\allomorph{nsärbemam}\allomorph{särmbae}\allomorph{nsärbe}\allomorph{nserbe}}
\entry{säre}{\headword{säre}\pos{adv.}\sensenumber{3}\definition{sadly}\example{Ause da säre kullkull de daempononggän tämamae.}{The old woman sadly burned all the grass.}}
\entry{särem}{\headword{särem}\pos{n.}\sensenumber{3}\definition{darkness, dark}\example{ag särem me}{at dawn (lit. in the dark of morning)}\example{Särem a dangttanän ttängäm de.}{Darkness came over the village.}\sensenumber{3}\definition{dark, dim}\example{Iddob me ttängäm a säremang abal gogän.}{At night, the village became very dark.}\example{Llamäg o mällause täräp me iba ikop a mɨnyi säremang bognegän.}{In old age, our (incl.) eyes will go dim.}\sensenumber{3}\definition{cloudy, dim}\sensenumber{3}\definition{in darkness, in the dark}\example{Ngämi säremsärem giddollnenang dagaeya.}{We (excl.) were living in darkness.}\etymology{särem + =ang}\subentry{\headword{säremang}\pos{mod.}\definition{dark, dim}}\subentry{\headword{säresäremang}\pos{mod.}\definition{cloudy, dim}}\subentry{\headword{säremsärem}\pos{adv.}\definition{in darkness, in the dark}}}
\entry{säremang ma}{\headword{säremang ma}\pos{n.}\sensenumber{3}\definition{prison, jail}\etymology{lit. 'dark house'}}
\entry{säroe}{\headword{säroe}\variant{sp. var. of}{soroe}}
\entry{säs}{\headword{säs}\pos{n.}\sensenumber{3}\definition{type of sago wrapped in young leaves}}
\entry{säsäri}{\headword{säsäri}\variant{sp. var. of}{sisri}}
\entry{säsäs}{\headword{säsäs}\pos{S vi.}\sensenumber{3}\definition{to rub}\example{Gosäsneyo obaoba.}{They were rubbing each other.}\allomorph{säs}\allomorph{säsnan}}
\entry{säspen}{\headword{säspen}\pos{n.}\sensenumber{1}\definition{pot, saucepan}\example{Ngämi ddäddäg de mɨnyi bäsnaemeyo säspen att.}{We (excl.) will take out the meat from the pot.}\sensenumber{2}\definition{to boil}\example{Ede ge ddia de däbänya wa dikomya ma we, ngäma sespen dägayaebeya wa däddägaebeya.}{So we (excl.) cut that deer, and brought it home, boiled it, and ate it.}\etymology{from Englishsaucepan}}
\entry{säspull}{\headword{säspull}\variant{dial. var. of}{sispull}}
\entry{säsramsäsram}{\headword{säsramsäsram}\pos{adv.}\sensenumber{2}\definition{shuffling}}
\entry{säsri}{\headword{säsri}\variant{sp. var. of}{sisri}}
\entry{SDA}{\headword{SDA}\pos{n.}\sensenumber{2}\definition{SDA (Seventh-day Adventist Church)}}
\entry{se1}{\headword{se1}\pos{n.}\sensenumber{2}\definition{yes}\example{Se imomdae dan.}{Yes, this story is true.}}
\entry{Sebe}{\headword{Sebe}\pos{pn.}\sensenumber{2}\definition{Sebe (Bine-speaking village in Oriomo-Bituri Rural LLG)}}
\entry{seben}{\headword{seben}\pos{num.}\sensenumber{2}\definition{seven (English numeral; also general)}\etymology{from Englishseven}}
\entry{sebenti}{\headword{sebenti}\pos{num.}\sensenumber{2}\definition{seventy}\etymology{from Englishseventy}}
\entry{sebentin}{\headword{sebentin}\pos{num.}\sensenumber{2}\definition{seventeen (English numeral)}\etymology{from Englishseventeen}}
\entry{sebor}{\headword{sebor}\pos{A vt.}\sensenumber{2}\definition{to greet}\example{Angde bogo ma we bongttägän tämamae lla de sebor bägnegän.}{When she comes to the village, she greets every person.}\sensenumber{2}\definition{to welcome}\example{Mälla da llo de seborsebor bägayaebneyo.}{The women will welcome the men.}\subentry{\headword{seborsebor}\pos{A vt.}\definition{to welcome}}}
\entry{sebosebor}{\headword{sebosebor}\variant{fast speech var. of}{seborsebor}}
\entry{sek}{\headword{sek}\pos{A vt.}\sensenumber{2}\definition{to check}\example{Ngämo nyäng sek nägaeyo.}{[You all] check my bag.}\etymology{from Englishcheck}}
\entry{sel1}{\headword{sel1}\pos{n.}\sensenumber{2}\definition{cell}\example{Polis a namllamallo lla de ngämo gänye nätramallo, sel ma ik i nazänallo.}{The police took my husband and put him in the cell.}\etymology{from Englishcell}}
\entry{sel2}{\headword{sel2}\pos{A vt.}\sensenumber{2}\definition{to sell}\example{Bogo obo otät de sel anggan.}{He's selling his food.}\etymology{from Englishsell}}
\entry{sele}{\headword{sele}\pos{n.}\sensenumber{2}\definition{chiliIne ttänttämang peyang käp de ute ine ttänttämang ma dan. (Clean sores with hot water with the fruit.)}}
\entry{sem1}{\headword{sem1}\pos{n.}\sensenumber{2}\definition{type of tree that grows in the bush; used for making ropeTtoe de kängkäm ma dan nane we itrellang me. (The bark is squeezed and the liquid is drunk when sick.)}}
\entry{sem2}{\headword{sem2}\pos{mod.}\sensenumber{2}\definition{same}\example{sem ekawang lla}{person who speaks the same language}\etymology{from Englishsame}}
\entry{sens}{\headword{sens}\pos{A vi. \textbackslash& vt.}\sensenumber{1}\definition{to change}\example{Bin a sens gogon.}{The name changed.}\example{Eka kutt di sens erallo.}{They are changing the word.}\sensenumber{2}\definition{to exchange (in marriage)}\example{Obom sens dägagän.}{He exchanged her for marriage.}\allomorph{sensens}\etymology{from Englishchange}}
\entry{Senti}{\headword{Senti}\pos{pn.}\sensenumber{2}\definition{male personal name}}
\entry{Sera}{\headword{Sera}\pos{pn.}\sensenumber{2}\definition{female personal name}}
\entry{ses}{\headword{ses}\variant{sp. var. of}{sos}}
\entry{seseyam}{\headword{seseyam}\pos{ideo.}\sensenumber{2}\definition{swish, sound of legs moving}}
\entry{sespen}{\headword{sespen}\variant{sp. var. of}{säspen}}
\entry{seven}{\headword{seven}\variant{sp. var. of}{seben}}
\entry{seventi}{\headword{seventi}\variant{sp. var. of}{sebenti}}
\entry{Shamae}{\headword{Shamae}\variant{sp. var. of}{Samae}}
\entry{Sharon}{\headword{Sharon}\pos{pn.}\sensenumber{2}\definition{female personal name}}
\entry{Shim}{\headword{Shim}\pos{pn.}\sensenumber{2}\definition{male personal name}}
\entry{Sibideri}{\headword{Sibideri}\pos{pn.}\sensenumber{2}\definition{Sibidiri (Idi-speaking village in Morehead Rural LLG)}}
\entry{Sibidir}{\headword{Sibidir}\variant{var. of}{Sibideri}}
\entry{Sibiya}{\headword{Sibiya}\pos{pn.}\sensenumber{2}\definition{male personal name}}
\entry{Sibne}{\headword{Sibne}\pos{pn.}\sensenumber{2}\definition{Sibne (toponym)}}
\entry{sidompa}{\headword{sidompa}\pos{n.}\sensenumber{2}\definition{type of spear}}
\entry{Siga}{\headword{Siga}\pos{pn.}\sensenumber{2}\definition{Siga (toponym)}}
\entry{Sigabaduru}{\headword{Sigabaduru}\pos{pn.}\sensenumber{2}\definition{Sigabaduru (in Kiwai Rural LLG)}}
\entry{sigip}{\headword{sigip}\pos{n.}\sensenumber{1}\definition{type of palm with fruit hanging from long pedicels; people chew the fruit like betelnutGoro wälläng me päddabag dan. (It grows in the jungle.)}\sensenumber{2}\definition{type of big taro}}
\entry{siklakla}{\headword{siklakla}\pos{n.}\sensenumber{1}\definition{golden-headed cisticola}\sensenumber{2}\definition{Australian reed warbler}\sensenumber{3}\definition{singing bush lark}}
\entry{siks}{\headword{siks}\pos{num.}\sensenumber{3}\definition{six (English numeral; also general)}\etymology{from Englishsix}}
\entry{siksti}{\headword{siksti}\pos{num.}\sensenumber{3}\definition{sixty}\etymology{from Englishsixty}}
\entry{sikstin}{\headword{sikstin}\pos{num.}\sensenumber{3}\definition{sixteen (English numeral)}\etymology{from Englishsixteen}}
\entry{Siku}{\headword{Siku}\pos{pn.}\sensenumber{3}\definition{male personal name}}
\entry{sili}{\headword{sili}\variant{var. of}{sele}}
\entry{Simon}{\headword{Simon}\variant{sp. var. of}{Saemon}}
\entry{Sini}{\headword{Sini}\pos{pn.}\sensenumber{3}\definition{female personal name}}
\entry{Sintia}{\headword{Sintia}\pos{pn.}\sensenumber{3}\definition{female personal name}}
\entry{singoll}{\headword{singoll}\pos{A vt.}\sensenumber{3}\definition{to give, provide, share}\example{Ngämlle mɨnyi ngämo nag wätät yuatt singoll bagän.}{My friend will share cooked food with me.}}
\entry{singosingol}{\headword{singosingol}\pos{adv.}\sensenumber{3}\definition{upwind, windward}\example{Bogo wel singosingol dallän.}{He went windward.}}
\entry{sip1}{\headword{sip1}\pos{n.}\sensenumber{3}\definition{sheep}\example{Ibi sisiri Yesu bo sip dag.}{We are now Jesus' sheep.}\etymology{from Englishsheep}}
\entry{sip2}{\headword{sip2}\pos{n.}\sensenumber{3}\definition{chief}\etymology{from Englishchief}}
\entry{sipel}{\headword{sipel}\pos{A vi.}\sensenumber{3}\definition{to rest}\example{Angde pos ganen a gottamänän ngäna sipel gog.}{When the post planting finished, I took a rest.}}
\entry{sipik}{\headword{sipik}\pos{n.}\sensenumber{3}\definition{type of large yam with a white and purple interior}}
\entry{siporo}{\headword{siporo}\pos{n.}\sensenumber{3}\definition{type of cultivated tree with thorns and sour yellow fruitKumye täräp me käp de otnan ma dan. (The fruit is eaten when sick with a cough.)}}
\entry{sir}{\headword{sir}\pos{n.}\sensenumber{3}\definition{type of tree that grows in the bush with white flowers and edible black fruit with one seed inside; cassowary and pigs eat the fruit}}
\entry{sirem}{\headword{sirem}\variant{sp. var. of}{särem}}
\entry{Sirmitang}{\headword{Sirmitang}\pos{pn.}\sensenumber{3}\definition{Sirmitang (toponym)}}
\entry{sis1}{\headword{sis1}\pos{n.}\sensenumber{3}\definition{season when new gardens are cleared and fenced (fourteenth season; corresponds to early November)}}
\entry{sis2}{\headword{sis2}\pos{n.}\sensenumber{3}\definition{type of flying ant that come out from their flooded anthills in November}}
\entry{sisel}{\headword{sisel}\pos{n.}\sensenumber{3}\definition{chiselMa llo tranen ma dan. (It's for carving out logs for houses.)}\etymology{from Englishchisel}}
\entry{sisi}{\headword{sisi}\pos{n.}\sensenumber{1}\definition{type of pandanus with a trunk used for woodTupiae päddabag da, nying kemibiag wallemäg me. (It grows tall, with many leg-shaped roots in the creek.)}\sensenumber{2}\definition{the third stage of coconut growth in which the seedling has formed (after planting)}}
\entry{sisiri}{\headword{sisiri}\variant{sp. var. of}{sisri}}
\entry{sisor}{\headword{sisor}\pos{mod.}\sensenumber{1}\definition{new}\example{Känazbag, ngäna mɨnyi bibra sisor ttoenttoen de bɨllɨt.}{Tomorrow, I will tell you a new story.}\sensenumber{2}\definition{young}\example{sisor lla}{young people}\sensenumber{2}\definition{newborn, infant}\subentry{\headword{sisor llɨg}\pos{n.}\definition{newborn, infant}}}
\entry{sisor pazi}{\headword{sisor pazi}\pos{n.}\sensenumber{2}\definition{season of New Years (sixteenth season; corresponds to December)}}
\entry{sispull}{\headword{sispull}\pos{n.}\sensenumber{2}\definition{maggotZonenang za toko me giddollnenang za pällämpälläm kälekäle, kupoll ingoll dag. (They are small, white, hookworm-like things that live on top of rotting things.)}}
\entry{sisri}{\headword{sisri}\pos{adv.}\sensenumber{1}\definition{now}\example{Mo da sisri ttäkam allan.}{The bridge is breaking now.}\sensenumber{2}\definition{this; current}\sensenumber{2}\definition{this morning}\sensenumber{2}\definition{today}\sensenumber{2}\definition{tonight}\sensenumber{2}\definition{this evening}\subentry{\headword{sisri ag}\pos{n.}\definition{this morning}}\subentry{\headword{sisri ebdo}\pos{n.}\definition{today}}\subentry{\headword{sisri iddob}\pos{n.}\definition{tonight}}\subentry{\headword{sisri toto}\pos{n.}\definition{this evening}}}
\entry{Sisuar}{\headword{Sisuar}\pos{pn.}\sensenumber{2}\definition{female personal name}}
\entry{siti}{\headword{siti}\pos{n.}\sensenumber{2}\definition{city}\etymology{from Englishcity}}
\entry{six}{\headword{six}\variant{sp. var. of}{siks}}
\entry{sixty}{\headword{sixty}\variant{sp. var. of}{siksti}}
\entry{sɨmell}{\headword{sɨmell}\pos{n.}\sensenumber{1}\definition{pig}\example{sɨmell mägda}{sow}\sensenumber{2}\definition{truck}}
\entry{sɨmell källamokott}{\headword{sɨmell källamokott}\pos{n.}\sensenumber{2}\definition{type of spiky tree that grows in the bush with white flowers, yellow fruit, and hardwood used for firewood}\etymology{lit. 'hard pig poop'}}
\entry{sɨmellkom}{\headword{sɨmellkom}\pos{n.}\sensenumber{2}\definition{type of sagoUlle päddab sana dan, ddobae mägag dan.}\etymology{sɨmell + kom, lit. 'pig hair'}}
\entry{sɨmellmak}{\headword{sɨmellmak}\pos{n.}\sensenumber{2}\definition{weaving pattern with arrows pointing right}\etymology{sɨmell + mak, lit. 'pig mark'}}
\entry{sɨrem}{\headword{sɨrem}\variant{sp. var. of}{särem}}
\entry{sɨs1}{\headword{sɨs1}\pos{interj.}\sensenumber{2}\definition{command given to a dog to chase an animal}}
\entry{sɨs2}{\headword{sɨs2}\pos{S vt.}\sensenumber{2}\definition{to extinguish, turn off}\example{Ngäna tos de dɨs.}{I turned off the light.}\allomorph{s}}
\entry{Skola}{\headword{Skola}\pos{pn.}\sensenumber{2}\definition{female personal name}}
\entry{skul}{\headword{skul}\pos{n.}\sensenumber{2}\definition{school; education}\sensenumber{2}\definition{student}\sensenumber{2}\definition{educated}\etymology{from Englishschool}\subentry{\headword{skulang}\pos{n.}\definition{student}}\subentry{\headword{skulatt}\pos{mod.}\definition{educated}}}
\entry{slaslak}{\headword{slaslak}\pos{n.}\sensenumber{2}\definition{red-winged parrotLlo ik me giddollag pa dan. (It's a bird that lives in trees.)}}
\entry{sllollongg}{\headword{sllollongg}\pos{S vi.}\sensenumber{2}\definition{to sit close together}\example{Ubi komllaebmae ansllollongallo duliballe.}{The two of them sat close together on that side.}\allomorph{nsllollong}}
\entry{so1}{\headword{so1}\pos{n.}\sensenumber{2}\definition{type of black palm; in this mature stage (\textbackslashtextasciitilde3 m), the wood is used for house flooring and containers for squeezing sago, and the pith is eatenMa käg tater e ttäkoema dan. (It's to chop and make into house flooring.)}}
\entry{Soba}{\headword{Soba}\pos{pn.}\sensenumber{2}\definition{male personal name}}
\entry{Sobam}{\headword{Sobam}\pos{pn.}\sensenumber{2}\definition{male personal name}}
\entry{Sobeya}{\headword{Sobeya}\pos{pn.}\sensenumber{2}\definition{Sobeya (toponym)}}
\entry{sod}{\headword{sod}\pos{n.}\sensenumber{2}\definition{shirt}\etymology{from Englishshirt}}
\entry{Sogale}{\headword{Sogale}\pos{pn.}\sensenumber{2}\definition{Sogale (Bine-speaking village in Oriomo-Bituri Rural LLG)}}
\entry{Soka}{\headword{Soka}\pos{pn.}\sensenumber{2}\definition{male personal name}}
\entry{soka}{\headword{soka}\pos{n.}\sensenumber{2}\definition{soccer}\etymology{from Englishsoccer}}
\entry{Sokola}{\headword{Sokola}\pos{pn.}\sensenumber{2}\definition{female personal name}}
\entry{sokpa}{\headword{sokpa}\pos{n.}\sensenumber{1}\definition{tobacco}\sensenumber{2}\definition{cigarette}\example{Iba sokpa dag gänyme gall toko me nanen e}{Our (incl.) cigarettes are here on the canoe for smoking.}}
\entry{sokpa kllokllop}{\headword{sokpa kllokllop}\pos{n.}\sensenumber{2}\definition{notebook-sized mat woven of tobacco}}
\entry{sokpa llaweatt}{\headword{sokpa llaweatt}\pos{n.}\sensenumber{2}\definition{tobacco woven like a rope}}
\entry{solt}{\headword{solt}\pos{n.}\sensenumber{2}\definition{salt}\etymology{from Englishsalt}}
\entry{Soma}{\headword{Soma}\pos{pn.}\sensenumber{2}\definition{male personal name}}
\entry{Songno}{\headword{Songno}\pos{pn.}\sensenumber{2}\definition{female personal name}}
\entry{sori}{\headword{sori}\variant{sp. var. of}{sari}}
\entry{soroe}{\headword{soroe}\pos{S vt.}\sensenumber{1}\definition{to try, attempt}\example{Ongg kame ako dasäroeyän.}{Ongg tried it again.}\example{Ubi mɨnyi oba mängall alle bosroenegnän ngämene eka ngänaem e.}{They will be trying hard to understand my words.}\sensenumber{2}\definition{to challenge, try, test; tempt}\example{Ubi ngonoe eka walle obom dasroeyo.}{They challenged him with a question.}\example{Bogo dedme poti ebdo dagirnän a saeten dasroenän.}{He stayed there for forty days and Satan was tempting him.}\sensenumber{3}\definition{test, exam}\example{Amo bun da soroe me mullae gogalle, bogo mɨnyi dallalle tuk skul e.}{He who does well on the test will go to secondary school.}\allomorph{sroe}}
\entry{sorry}{\headword{sorry}\variant{sp. var. of}{sari}}
\entry{sos}{\headword{sos}\pos{n.}\sensenumber{3}\definition{church}\example{Tämamae wiyamom sos e.}{Everyone, come to church.}\etymology{from Englishchurch}}
\entry{sosoga}{\headword{sosoga}\pos{n.}\sensenumber{3}\definition{type of sagoAi mägag sana dan. (It's a sago that produces well.)}}
\entry{Sowa}{\headword{Sowa}\pos{pn.}\sensenumber{3}\definition{male personal name}}
\entry{Sowati}{\headword{Sowati}\pos{pn.}\sensenumber{3}\definition{male personal name}}
\entry{soccer}{\headword{soccer}\variant{sp. var. of}{soka}}
\entry{spalek}{\headword{spalek}\variant{fast speech var. of}{säpalek}}
\entry{spallma}{\headword{spallma}\pos{n.}\sensenumber{3}\definition{two friends who split a twin coconut frondMällayaba ttaem. (Amongst women.)}}
\entry{spun1}{\headword{spun1}\pos{n.}\sensenumber{3}\definition{spoon}\etymology{from Englishspoon}}
\entry{spun2}{\headword{spun2}\pos{S vi.}\sensenumber{1}\definition{to fall; set}\example{Sɨmell a kuddäll a guspunän.}{The pig fell down dead.}\example{Kakoll a obo ttang atta aspunan.}{The plate fell from his hand.}\example{Yäbäd a zäme guspunmällnän.}{The sun was already setting.}\sensenumber{2}\definition{to jump}\example{Däbe llɨg walle we guspunän.}{That boy jumped into the water.}\sensenumber{3}\definition{to throw; shoot}\example{Ngämi tudi daspullaemnalla.}{We (excl.) started throwing fishing lines.}\example{Ako kame kakoll de katkatre toko we naspullan mängall e enanae opallknegan.}{Again, she threw the dishes on top of the table forcefully and they finally broke.}\example{Ibi mɨnyi damärärmae toboll baspunigalla.}{We (incl.) will both shoot arrows at the same time.}\sensenumber{4}\definition{to remove an outer layer}\example{Ttoe de naspunan.}{He removed the skin.}\allomorph{spull}\allomorph{nspull}\allomorph{spu}\allomorph{sampull}\allomorph{sämpall}\allomorph{säpall}\allomorph{spall}}
\entry{Stanis}{\headword{Stanis}\pos{pn.}\sensenumber{4}\definition{male personal name}}
\entry{Stashalyn}{\headword{Stashalyn}\pos{pn.}\sensenumber{4}\definition{female personal name}}
\entry{Steven}{\headword{Steven}\variant{sp. var. of}{Stibin}}
\entry{Stibin}{\headword{Stibin}\pos{pn.}\sensenumber{4}\definition{male personal name}}
\entry{Stiven}{\headword{Stiven}\variant{sp. var. of}{Stibin}}
\entry{stoa}{\headword{stoa}\pos{n.}\sensenumber{4}\definition{store}\example{Gull tupi di stoa me llädnan amallo.}{They bought long nets in the store.}\etymology{from Englishstore}}
\entry{stori}{\headword{stori}\pos{n.}\sensenumber{4}\definition{story}\etymology{from Englishstory}}
\entry{story}{\headword{story}\variant{sp. var. of}{stori}}
\entry{stowa}{\headword{stowa}\variant{sp. var. of}{stoa}}
\entry{su}{\headword{su}\pos{n.}\sensenumber{1}\definition{prey}\sensenumber{2}\definition{secret}\example{Su eka midd a zäme bina pate indrang gognegän.}{The secret meaning has already been revealed to you all.}}
\entry{Suame}{\headword{Suame}\pos{pn.}\sensenumber{2}\definition{Suame (toponym)}}
\entry{suga}{\headword{suga}\pos{n.}\sensenumber{2}\definition{sugar}\etymology{from Englishsugar}}
\entry{suga galbe}{\headword{suga galbe}\pos{n.}\sensenumber{2}\definition{type of large yam with a white interior}}
\entry{Sui}{\headword{Sui}\variant{sp. var. of}{Suwi}}
\entry{Suki}{\headword{Suki}\pos{pn.}\sensenumber{2}\definition{Suki (in Morehead Rural LLG)}}
\entry{sukul}{\headword{sukul}\variant{sp. var. of}{skul}}
\entry{Suliki}{\headword{Suliki}\pos{pn.}\sensenumber{2}\definition{male personal name}}
\entry{sulut}{\headword{sulut}\pos{n.}\sensenumber{2}\definition{type of taro}}
\entry{supun}{\headword{supun}\variant{dial. var. of}{spun2}}
\entry{sur}{\headword{sur}\pos{n.}\sensenumber{2}\definition{pushing toolGall domoe ma llo tupi. (A long piece of wood for pushing a canoe.)}}
\entry{surum}{\headword{surum}\pos{n.}\sensenumber{2}\definition{sand}}
\entry{surusuru}{\headword{surusuru}\pos{n.}\sensenumber{2}\definition{type of tree that grows in the bush with white flowers and strong wood used for house posts and firewood; it was traditionally lit and used as a light at night because it burns slowly}}
\entry{Susan}{\headword{Susan}\pos{pn.}\sensenumber{2}\definition{female personal name}}
\entry{susu}{\headword{susu}\pos{n.}\sensenumber{2}\definition{breastmilk}\etymology{from Tok Pisinsusu, ultimately from Malay susu}}
\entry{suwe}{\headword{suwe}\pos{n.}\sensenumber{1}\definition{urine, peeLla suwe da meresen dan amtetang me, ako ddob itrell me ade. Ako ute de suwe tängg ma dan käkälläp e, ute kädkäd e. (Human urine is medicine for asthma and for other illnesses. Sores are also urinated on to make them sting, to make cool them down.)}\example{Yokon wälläng e dallän suwe ma.}{Yokon went to the bush to pee.}\sensenumber{2}\definition{(euphemistic) testicles}\sensenumber{2}\definition{to urinate on, pee on}\example{Ekaklle de suwe näntägan.}{He peed on the ground.}\subentry{\headword{suwe tängg}\pos{S vt.}\definition{to urinate on, pee on}}}
\entry{suwe gäd}{\headword{suwe gäd}\pos{n.}\sensenumber{2}\definition{body part}}
\entry{Suwede}{\headword{Suwede}\pos{pn.}\sensenumber{2}\definition{male personal name}}
\entry{Suwi}{\headword{Suwi}\pos{pn.}\sensenumber{2}\definition{Sui (in Kiwai Rural LLG)}}
\entry{Sylvien}{\headword{Sylvien}\pos{pn.}\sensenumber{2}\definition{female personal name}}
\lettersection{T}
\entry{tab}{\headword{tab}\pos{n.}\sensenumber{1}\definition{promise, oath; engagement}\example{Tab de obo pate dangesän.}{She made him a promise.}\sensenumber{2}\definition{to promise}\example{Ngäna bibim tab anggan.}{I promise you all.}}
\entry{tabe}{\headword{tabe}\pos{n.}\sensenumber{1}\definition{soft pad or support that placed between one's back and a spalek basket}\sensenumber{2}\definition{to pad your back from the spalek}}
\entry{Tabita}{\headword{Tabita}\pos{pn.}\sensenumber{2}\definition{female personal name}}
\entry{Tabubil}{\headword{Tabubil}\pos{pn.}\sensenumber{2}\definition{Tabubil (toponym)}}
\entry{tada}{\headword{tada}\pos{n.}\sensenumber{2}\definition{fish trapwadär alle täl alle iatt (Woven from cane, from bamboo.)}\example{Ngäna kollba de tada walle iringän eran.}{I am catching the fish with the trap.}}
\entry{tae}{\headword{tae}\pos{n.}\sensenumber{2}\definition{type of tall tree that grows by rivers with white flowers, green fruit, and sturdy wood that can be used as a bridge}\sensenumber{2}\definition{type of edible grub}\sensenumber{2}\definition{type of mushroom}\subentry{\headword{tae budar}\pos{n.}\definition{type of edible grub}}\subentry{\headword{tae lläkäm}\pos{n.}\definition{type of mushroom}}}
\entry{taem}{\headword{taem}\pos{n.}\sensenumber{2}\definition{time}\example{ddob taem me}{sometimes}\etymology{from Englishtime}}
\entry{taemataema}{\headword{taemataema}\pos{n.}\sensenumber{2}\definition{type of tree that grows near the swamp with long yellow flowers and leaves that are used to treat fungal skin infections}}
\entry{Taeme}{\headword{Taeme}\variant{sp. var. of}{Tame}}
\entry{Taeme loanword}{\headword{Taeme loanword}\sensenumber{2}\definition{loanword}}
\entry{taempäg}{\headword{taempäg}\pos{S vt.}\sensenumber{2}\definition{to show, indicate, reveal}\example{Bogo mɨnyi tuk me ulle abal känttatt de yantepägän.}{He will show you all a very large bedroom upstairs.}\example{Adi gontepägän obo täbatäbe de.}{God revealed his plan.}\allomorph{taempmeny}\allomorph{ntepäg}\allomorph{ntepmeny}\allomorph{taemp}\allomorph{ntep}}
\entry{taewa}{\headword{taewa}\pos{n.}\sensenumber{2}\definition{type of bird}}
\entry{Tag}{\headword{Tag}\pos{pn.}\sensenumber{2}\definition{personal name}}
\entry{Tai}{\headword{Tai}\variant{sp. var. of}{Tayi}}
\entry{taim}{\headword{taim}\variant{sp. var. of}{taem}}
\entry{tainäm}{\headword{tainäm}\pos{n.}\sensenumber{2}\definition{mosquito net}}
\entry{Takeya}{\headword{Takeya}\pos{pn.}\sensenumber{2}\definition{female personal name}}
\entry{talapia}{\headword{talapia}\pos{n.}\sensenumber{2}\definition{tilapia}\etymology{from Englishtilapia}}
\entry{talme}{\headword{talme}\pos{A vi. \textbackslash& vt.}\sensenumber{2}\definition{to give birth (to)}\example{Ttongdae ebdo meae bode do Malläm me talme gogän.}{She gave birth in Malläm on the same day.}\example{Obom talme dägagän.}{She gave birth to her.}}
\entry{tall}{\headword{tall}\pos{n.}\sensenumber{2}\definition{type of tree with wood used for posts and easily-peeling bark used for walling; also lit as a torch}}
\entry{Tallabunang}{\headword{Tallabunang}\pos{pn.}\sensenumber{2}\definition{Tallabunang (toponym)}}
\entry{tamatama}{\headword{tamatama}\pos{n.}\sensenumber{2}\definition{bean}}
\entry{Tame}{\headword{Tame}\pos{pn.}\sensenumber{2}\definition{Taeme language (Pahoturi River language spoken in Kinkin alongside Ende)}}
\entry{tame}{\headword{tame}\pos{n.}\sensenumber{2}\definition{wave (of water)}}
\entry{tameny}{\headword{tameny}\pos{S vt.}\sensenumber{1}\definition{to teach}\example{Ngäna bam bantemeny walle gllangglla de.}{I'm going to teach you to swim.}\example{Yesu ubira yuwog abal ttoen de sanawang eka walle däntamenynegnän.}{Jesus was teaching them many things using parables.}\sensenumber{2}\definition{to dicuss, converse}\sensenumber{3}\definition{to tell; converse with, speak to}\sensenumber{3}\definition{teacher}\allomorph{ntameny}\allomorph{ntemeny}\allomorph{ntemenyam}\etymology{tameny + =ang}\subentry{\headword{tamenyang}\pos{n.}\definition{teacher}}}
\entry{tamllägtamlläg}{\headword{tamllägtamlläg}\pos{n.}\sensenumber{3}\definition{type of small caterpillar}}
\entry{tamongle}{\headword{tamongle}\pos{n.}\sensenumber{3}\definition{type of mutae yam shaped like a foot}}
\entry{tan}{\headword{tan}\pos{n.}\sensenumber{1}\definition{type of short plant with white flowers, brown fruit, and branches used as a broom}\sensenumber{2}\definition{broom, can come from different types of palm e.g. käg tan, mäta tanMa omnen ma dan. (It's for sweeping the house.)}}
\entry{Tanisha}{\headword{Tanisha}\pos{pn.}\sensenumber{2}\definition{female personal name}}
\entry{tanong}{\headword{tanong}\pos{adv.}\sensenumber{2}\definition{a little}\example{utale kanong}{a bit far}}
\entry{tanteny}{\headword{tanteny}\pos{n.}\sensenumber{2}\definition{type of treeLlo kälsäre dan. Ttoe a kädkäd ma dan a ddobae kaekepang dan. Wälländd ttoe de kulltoenen ma dan wabnen e a kämnan e. Obo ine da nane ma dan itrell sapasapang llädnan täräp me, be ddobae tomowang dan. (It's a small tree. The bark is removed and is chewed a lot. The skin of the roots is removed to be beaten and squeezed. Its liquid is drunk when one gets sick with various illnesses, but it's very sour.)}}
\entry{tanyäb}{\headword{tanyäb}\variant{dial. var. of}{tanyib}}
\entry{tanyib}{\headword{tanyib}\pos{n.}\sensenumber{2}\definition{radioulna (fused radius and ulna, or arm bone) of a flying fox}\example{Ngäma mäda bi giri alle topoll bo tanyib de dupoaemallo a ngäma mɨllɨng e dapoaemallo.}{Our (excl.) fathers used to sharpen the phalanges of a flying fox with a knife and pierce our noses.}}
\entry{tanyteny}{\headword{tanyteny}\variant{var. of}{tanteny}}
\entry{Tao}{\headword{Tao}\pos{pn.}\sensenumber{2}\definition{male personal name}}
\entry{Taolang}{\headword{Taolang}\pos{pn.}\sensenumber{2}\definition{Taolang (camping place)}}
\entry{taosen}{\headword{taosen}\pos{num.}\sensenumber{2}\definition{thousand}\example{ttongo taosen tukituki}{more than one thousand}\etymology{from Englishthousand}}
\entry{tap1}{\headword{tap1}\pos{A vt.}\sensenumber{2}\definition{to dock, land}\example{Gall de tap dägageya wa ma we deyareya inu we.}{We docked the boat and returned home to sleep.}\sensenumber{2}\definition{dock, wharf}\example{Ubi mängalae gall tapma we dagllaeyo.}{We (incl.) paddled quickly to the canoe dock.}\etymology{tap + =ma}\subentry{\headword{tapma}\pos{n.}\definition{dock, wharf}}}
\entry{tap2}{\headword{tap2}\pos{S vt.}\sensenumber{2}\definition{to harvest}\example{Tapnen de gongkaemallo.}{They started the harvest.}\example{Dätepän.}{He harvested it.}\allomorph{taep}\allomorph{tep}}
\entry{tap3}{\headword{tap3}\pos{n.}\sensenumber{2}\definition{tunnel}}
\entry{tap4}{\headword{tap4}\pos{n.}\sensenumber{2}\definition{type of big yam with a white interior, thorns, and hairs}}
\entry{Tapila}{\headword{Tapila}\pos{pn.}\sensenumber{2}\definition{Tapila (Makayam-speaking village in Gogodala Rural LLG; GPS: -8.414202, 143.016867)}}
\entry{Tapma}{\headword{Tapma}\pos{pn.}\sensenumber{2}\definition{Tapma (toponym)}}
\entry{tarakoll}{\headword{tarakoll}\pos{n.}\sensenumber{2}\definition{protective outer wall built by ancestors}}
\entry{tarasoso}{\headword{tarasoso}\pos{n.}\sensenumber{2}\definition{type of birdYogoll me tongoeang pa bätbät dan. (It's a black bird that plays in the rain.)}}
\entry{tarketarke}{\headword{tarketarke}\pos{mod.}\sensenumber{2}\definition{brittle}\example{Llo ttam a tarketarke dag.}{The leaves are brittle.}}
\entry{tarko}{\headword{tarko}\pos{n.}\sensenumber{2}\definition{female friends who have shared a joined fruit or vegetable}\example{Kathy bo tarko da Mome aenen.}{Kathy's tarko is Mome.}}
\entry{tarme}{\headword{tarme}\pos{n.}\sensenumber{2}\definition{kookaburra (blue-winged; spangled)Bädab e ekawang pa dan. (It's a bird that sings at dawn.)}}
\entry{tarme ballmenyang}{\headword{tarme ballmenyang}\pos{n.}\sensenumber{2}\definition{season when crops are bearing fruit (fifth season; corresponds to March)}}
\entry{tarme koeme}{\headword{tarme koeme}\pos{n.}\sensenumber{2}\definition{type of tree}}
\entry{tarmekälla}{\headword{tarmekälla}\pos{n.}\sensenumber{2}\definition{type of taro}\etymology{tarme + källa, lit. 'kookaburra poop'}}
\entry{taromba}{\headword{taromba}\pos{num.}\sensenumber{2}\definition{216 (yam counting numeral; 6\textbackslashtextasciicircum3)}\etymology{from a Yam language; compare Nen taromba}}
\entry{tata1}{\headword{tata1}\variant{baby talk var. of}{ddäddäg1}}
\entry{tata2}{\headword{tata2}\pos{n.}\sensenumber{2}\definition{junction, intersection}\example{Bogo nyongo tata me dägabällän.}{She stood at the road junction.}}
\entry{tatanggli}{\headword{tatanggli}\pos{n.}\sensenumber{2}\definition{willie wagtailPullkom wanewanen pa kälsre, ap me giddollag dan. (A small bird that shakes its tailfeathers, it lives in the savanna.)}}
\entry{tatarke}{\headword{tatarke}\variant{fast speech var. of}{tarketarke}}
\entry{tatäraem}{\headword{tatäraem}\pos{n.}\sensenumber{2}\definition{noise}\example{Tatraem de gondärän.}{She heard the noise.}\example{De dräm de aeya tatäraem eran?}{Who's making noise in the drum?}}
\entry{tatäräp}{\headword{tatäräp}\variant{var. of}{tätäräp1}}
\entry{tater}{\headword{tater}\pos{n.}\sensenumber{1}\definition{matGällall inenatt dämanen e o yununin i. (Woven from gällall pandanus; to eat or sleep on.)}\example{Gaem ai tater gollaeb de inen anggan.}{Gaem is weaving many mats.}\sensenumber{2}\definition{to spread out}\example{Ubi oba iddpo de däternegeyo dongki bo ddäg toko me.}{They spread out their cloaks over the donkey's back.}\allomorph{ter}}
\entry{tatraem}{\headword{tatraem}\variant{fast speech var. of}{tatäraem}}
\entry{tatruk}{\headword{tatruk}\pos{n.}\sensenumber{2}\definition{type of poisonous ant}}
\entry{tatu}{\headword{tatu}\pos{A vi. \textbackslash& vt.}\sensenumber{2}\definition{to bathe, wash oneself; wash (an animate object)Ine alle gollaebmeny pätt me kot spall e. (Pouring water on one's body to remove dirt.)}\example{Ubi obaoba tatu anggan.}{They are washing each other.}\example{Bogo tatu agan.}{She bathed.}\sensenumber{2}\definition{washing place, outdoor bathing area}\sensenumber{2}\definition{cleanly}\example{tatutatu giddollnen}{living cleanly}\etymology{tatu + =ma}\subentry{\headword{tatuma}\pos{n.}\definition{washing place, outdoor bathing area}}\subentry{\headword{tatutatu}\pos{adv.}\definition{cleanly}}}
\entry{tawa}{\headword{tawa}\pos{n.}\sensenumber{2}\definition{swampWalle me towall dan. (It's grass in the water.)}}
\entry{tawa aeb}{\headword{tawa aeb}\pos{n.}\sensenumber{2}\definition{Australasian swamphen}}
\entry{Tawabo}{\headword{Tawabo}\pos{pn.}\sensenumber{2}\definition{Tawabo (toponym)}}
\entry{tawar}{\headword{tawar}\pos{n.}\sensenumber{2}\definition{totem symbol (thing that represents a clan)}}
\entry{tawe}{\headword{tawe}\pos{n.}\sensenumber{2}\definition{type of slim, tall palm with coconuts that come in red or green varieties, bunches of yellow fruit that birds eat, and fronds used for camp flooringTobäll e ponen ma za dan. (It's for sharpening into spears.)}}
\entry{tawe ttäp}{\headword{tawe ttäp}\pos{n.}\sensenumber{2}\definition{type of spearToboll e tawe pallkoll popoatt. (A piece of tawe wood sharpened into a spear.)}\example{Ngäna tätäm kukullang me ttongo ttall de tawe ttäp alle dazu.}{Yesterday, I shot one wallaby in the grassfire with the tawe ttäp arrow.}}
\entry{tawekutt}{\headword{tawekutt}\pos{n.}\sensenumber{1}\definition{partially dry coconut}\example{Sana peyang nyänggae ma nge da imomdae gänyan tawekutt a.}{The right coconut to mix with sago is this tawekutt.}\sensenumber{2}\definition{the tenth stage of coconut growth during which the fruit begins to dry and brown}}
\entry{Tawemitang}{\headword{Tawemitang}\pos{pn.}\sensenumber{2}\definition{Tawemitang (toponym)}}
\entry{Tayi}{\headword{Tayi}\pos{pn.}\sensenumber{2}\definition{Tai (in Gogodala Rural LLG)}}
\entry{täb}{\headword{täb}\pos{n.}\sensenumber{2}\definition{type of tree}}
\entry{täbab}{\headword{täbab}\pos{S vt.}\sensenumber{2}\definition{to watch over}\allomorph{täb}}
\entry{täbatäbe}{\headword{täbatäbe}\pos{S vi.}\sensenumber{1}\definition{to plan}\example{Ngäna gotäba ngämo nge ibeny e.}{I planned to plant my coconut.}\sensenumber{2}\definition{plan}\example{Ngämi diba toto me ngäma täbatäbe de dangeseya iddob me ddia ma koenmäll e.}{We (excl.) made the plan that evening to chase the deer at night.}\allomorph{toba}\allomorph{täba}\allomorph{täbe}\allomorph{täbanen}\allomorph{tba}\allomorph{täbo}\allomorph{tbe}}
\entry{täbädd}{\headword{täbädd}\pos{n.}\sensenumber{2}\definition{guest, visitor, stranger}\example{Eran ngämo täbädd känttatt a erame?}{Where is my guest room?}}
\entry{täbäll}{\headword{täbäll}\variant{sp. var. of}{tobäll}}
\entry{täbäll pud}{\headword{täbäll pud}\pos{n.}\sensenumber{1}\definition{type of biting bee found in trees}\sensenumber{2}\definition{type of tree}}
\entry{täbe}{\headword{täbe}\pos{n.}\sensenumber{2}\definition{type of big tree that grows in the bush with white flowers and drupes that fall with an edible seed inside}\sensenumber{2}\definition{type of grubWälläng me täbe patt budar, ddäddäg ma dan. (A grub in trunks of täbe trees in the bush; it's edible.)}\sensenumber{2}\definition{type of mushroomWälläng me täbe patt me dan, ddäddäg ma dan. (On the trunks of täbe trees in the bush; it's edible.)}\subentry{\headword{täbe budar}\pos{n.}\definition{type of grubWälläng me täbe patt budar, ddäddäg ma dan. (A grub in trunks of täbe trees in the bush; it's edible.)}}\subentry{\headword{täbe lläkäm}\pos{n.}\definition{type of mushroomWälläng me täbe patt me dan, ddäddäg ma dan. (On the trunks of täbe trees in the bush; it's edible.)}}}
\entry{täbie}{\headword{täbie}\pos{n.}\sensenumber{1}\definition{black sunbirdBätbät pa kälsre, nge popo naneang dan. (A small black bird, it drinks coconut flowers.)}\sensenumber{2}\definition{garden sunbird}}
\entry{täbom}{\headword{täbom}\pos{n.}\sensenumber{2}\definition{type of small yam with a white interior, hairs, and thorns}}
\entry{tägab}{\headword{tägab}\pos{S vt.}\sensenumber{2}\definition{to turn upside down}\example{Ubi obom tutu we dirngäneyo a ingoll dätägabeyo ine gollab e obo bod att.}{They pulled her onto land and turned her upside down so the water would come out of her mouth.}\allomorph{täg}}
\entry{täk}{\headword{täk}\pos{n.}\sensenumber{2}\definition{clitoris}}
\entry{täkäll}{\headword{täkäll}\pos{n.}\sensenumber{1}\definition{fin}\example{kollba täkäll}{fish fin}\sensenumber{2}\definition{horn}\example{ddia täkäll}{deer horn}\sensenumber{3}\definition{thorn}\example{siporo täkäll}{lemon thorns}}
\entry{täkälltäkäll kutt}{\headword{täkälltäkäll kutt}\pos{n.}\sensenumber{3}\definition{spine}}
\entry{täkälluit}{\headword{täkälluit}\pos{n.}\sensenumber{3}\definition{radial cartilage}}
\entry{täkla}{\headword{täkla}\pos{n.}\sensenumber{3}\definition{tree type}}
\entry{täko}{\headword{täko}\pos{n.}\sensenumber{3}\definition{any body part}}
\entry{täl}{\headword{täl}\pos{n.}\sensenumber{3}\definition{type of large bamboo that grows anywhere; used for bows, bow strings, axe handles, spears, and clothing pegs}}
\entry{täl wadär}{\headword{täl wadär}\pos{n.}\sensenumber{3}\definition{bamboo string}}
\entry{täli}{\headword{täli}\pos{S vt.}\sensenumber{3}\definition{to repeat}\example{Bogo dallän ikopse we, dantlinigän däbem eka dae.}{He went to pray, repeating just those words.}\allomorph{ntli}\allomorph{ntäli}}
\entry{tälpe}{\headword{tälpe}\pos{S vi.}\sensenumber{3}\definition{to volunteer}\example{Bogo atälpeyan.}{He volunteered.}}
\entry{tältäl}{\headword{tältäl}\pos{n.}\sensenumber{3}\definition{type of grass}}
\entry{Täm}{\headword{Täm}\pos{pn.}\sensenumber{3}\definition{female personal name}}
\entry{täma}{\headword{täma}\pos{n.}\sensenumber{3}\definition{husk, exocarp/mesocarp (of coconut)}}
\entry{tämallang mälla}{\headword{tämallang mälla}\pos{n.}\sensenumber{3}\definition{type of big taro}}
\entry{tämamae}{\headword{tämamae}\pos{quant.}\sensenumber{1}\definition{all, every}\example{tämamae ngalen, tämamae lla}{everything, everyone}\example{Kollba da tämamae dadrowän.}{All the fish died.}\sensenumber{2}\definition{whole, entire}\example{tämamae bem gallgall}{the entire coastline}}
\entry{tämani}{\headword{tämani}\pos{n.}\sensenumber{2}\definition{type of large yam with a white or white and red interior}}
\entry{tämbameny}{\headword{tämbameny}\pos{S vt.}\sensenumber{2}\definition{to instruct}\example{Ause da bom däntäbemenyän, "Abo bongo goeg de naemponomeny."}{The old woman instructed him, "You must burn the garden."}\allomorph{ntäbemeny}}
\entry{täme}{\headword{täme}\pos{n.}\sensenumber{2}\definition{monitor lizard, goannaKäza ingoll ddäddäg be ap me a wälläng me giddollag dan. (A crocodile-like animal, but it lives in the savannah and the bush.)}}
\entry{täme käp}{\headword{täme käp}\pos{n.}\sensenumber{2}\definition{type of round yam with a white interior, red skin, and hairs}\etymology{lit. 'lizard egg'}}
\entry{täme sära}{\headword{täme sära}\pos{n.}\sensenumber{2}\definition{bush rope}}
\entry{tämpeyam}{\headword{tämpeyam}\pos{S vt.}\sensenumber{2}\definition{to abandon, give up}\example{Bongo ewatta ke ngänäm nantäpeyamalle?}{Why have you abandoned me?}\example{Paelet Yesu bom Rom mäk lla yaba pate däntäpeyamän ddänggaddängge we.}{Pilate gave Jesus up to the Roman soldiers to be crucified.}\allomorph{ntäpey}}
\entry{tän}{\headword{tän}\pos{n.}\sensenumber{1}\definition{tribe, nation, people}\example{Ende tän}{the Ende people}\sensenumber{2}\definition{clan}\example{Ngämi lla tän dag.}{We (excl.) are of the same clan.}\sensenumber{3}\definition{stem}}
\entry{täne}{\headword{täne}\variant{var. of}{tine}}
\entry{tängg}{\headword{tängg}\pos{S vt.}\sensenumber{3}\definition{to urinate on, pee on}\allomorph{ntäg}}
\entry{tänggag}{\headword{tänggag}\pos{S vt.}\sensenumber{1}\definition{to make a dog more sensitive to smells by rubbing lemongrass on their nose}\sensenumber{2}\definition{to steal}\allomorph{tänggameny}\allomorph{ntägag}\allomorph{ntägameny}\allomorph{ntägeg}\allomorph{ntägemeny}}
\entry{täny}{\headword{täny}\pos{n.}\sensenumber{1}\definition{lesser fig-parrotTagom ik me giddollag pa dan. (It's a bird that lives in tagom trees.)}\sensenumber{2}\definition{red-flanked lorikeet}}
\entry{täp}{\headword{täp}\pos{n.}\sensenumber{2}\definition{sago shoot}\example{Sana täp a tupi dan.}{The sago shoot is long.}\sensenumber{2}\definition{the fourth stage of sago growth in which the shoot emerges, indicating the pith is ready to be harvested}\example{Masar tätäm täp gazenatt sana de nagda bälle dantepägän.}{Yesterday, grandfather showed a flowering sago to his friend.}\etymology{lit. 'shoot that has emerged'}\subentry{\headword{täp gazenatt}\pos{n.}\definition{the fourth stage of sago growth in which the shoot emerges, indicating the pith is ready to be harvested}}}
\entry{täpäll}{\headword{täpäll}\pos{n.}\sensenumber{2}\definition{type of pandanus used in a traditional hairstyle and woven together to make a big mat}}
\entry{täpe}{\headword{täpe}\pos{n.}\sensenumber{2}\definition{mud}\sensenumber{2}\definition{muddy}\example{Llɨg a täpetäpe gognegnän yogoll me.}{The kids were getting muddy in the rain.}\subentry{\headword{täpetäpe}\pos{mod.}\definition{muddy}}}
\entry{tär}{\headword{tär}\pos{n.}\sensenumber{1}\definition{string, line}\example{Ngäna obo nying kollop tär de mɨnyi bäpitkaeneg.}{I will untie his sandal laces.}\sensenumber{2}\definition{strap or handle of a bag/basket (nyäng)}}
\entry{tära}{\headword{tära}\pos{A vt.}\sensenumber{2}\definition{to finish}\example{Ematta kollbameny abal angttägalle ma we? Bäne tämamae tära nägnegalle.}{Why did you return home without any fish? You finished them all.}}
\entry{tärabol}{\headword{tärabol}\pos{n.}\sensenumber{2}\definition{trouble}\etymology{from Englishtrouble}}
\entry{tärak}{\headword{tärak}\pos{S vi.}\sensenumber{1}\definition{to go inside}\example{Bogo dindugän do kukiny mama de ikop dägagän dibaeya däbe ik e gotärakän do gotägolän.}{He ran to a pile of tall grass that he saw and went inside and hid there.}\sensenumber{2}\definition{to put in}\example{Ngäna pakos de pop nyongo dae dätrak.}{I put the pakos spear through the hole.}}
\entry{täral}{\headword{täral}\pos{n.}\sensenumber{2}\definition{type of tree that grows in the grassland with white flowers, brown fruit, and wood used for house posts}}
\entry{täral pällämpälläm}{\headword{täral pällämpälläm}\pos{n.}\sensenumber{2}\definition{type of tree}}
\entry{täram1}{\headword{täram1}\variant{var. of}{tɨram}}
\entry{täram2}{\headword{täram2}\pos{S vt.}\sensenumber{2}\definition{to lead, take, carry, collect}\example{Kwakmae obom dätramän walle we.}{Kwakmae led him to the water.}\example{Ine da obom dätramän do Dowan tutu.}{The water carried him to Dowan mountain.}\example{Ubi bibim mɨnyi kwatt e yatraemneyo.}{They will take you all to court.}\example{Polis a obom täram erallo daolle.}{The police are taking him away.}\allomorph{täraem}\allomorph{tr}}
\entry{tärangesa}{\headword{tärangesa}\pos{n.}\sensenumber{1}\definition{pinky, little finger}\example{Ge tärängesa da sɨmell bod ik i gozenän.}{This pinky went in the pig's mouth.}\sensenumber{2}\definition{one (lit. pinky; body counting numeral)}}
\entry{tärangg}{\headword{tärangg}\pos{S vt.}\sensenumber{2}\definition{to stop, hold back}\example{Grace kollba kälsäre de daspunän walle we, be mälla da tämamae angde ikop dägageyo, ubi Grace bom dantärageyo ada, "Kollba de supun a mudan!"}{Grace threw the small fish back into the water, but when all the women saw her, they held her back, saying, "Don't throw back the fish!"}\allomorph{tära}\allomorph{ntärag}\allomorph{tärameny}\allomorph{trameny}\allomorph{ntrag}\allomorph{ntrameny}}
\entry{Tärapang}{\headword{Tärapang}\pos{pn.}\sensenumber{2}\definition{female personal name}}
\entry{täratäre}{\headword{täratäre}\pos{S vt.}\sensenumber{2}\definition{to dig out, hollow out}\allomorph{tära}}
\entry{täräb}{\headword{täräb}\pos{n.}\sensenumber{2}\definition{a dead person's belongings (put outside at a funeral until the feast ends)}\sensenumber{2}\definition{funeral home}\subentry{\headword{täräb ma}\pos{n.}\definition{funeral home}}}
\entry{täräk1}{\headword{täräk1}\pos{n.}\sensenumber{2}\definition{bundle}\example{Bina patme auli sana täräk yuwatt dag?}{How many bundles of cooked sago do you all have?}}
\entry{täräll}{\headword{täräll}\pos{A vt.}\sensenumber{2}\definition{to stick out}\example{Kurupel llamda de dägmar täräll dägnegän.}{Kurupel stuck out his tongue at the old man.}}
\entry{täräm}{\headword{täräm}\pos{n.}\sensenumber{2}\definition{tear}}
\entry{tärämpmeny}{\headword{tärämpmeny}\pos{n.}\sensenumber{2}\definition{sleeping with a dead person's belongings}}
\entry{täränga}{\headword{täränga}\pos{mod.}\sensenumber{2}\definition{low, scant, shallow}\example{Ine kutt me ine da täränga dan.}{The water in the bucket is low.}}
\entry{tärängesa}{\headword{tärängesa}\variant{var. of}{tärangesa}}
\entry{täräp1}{\headword{täräp1}\pos{n.}\sensenumber{2}\definition{time}\example{Ma we ibi täräp.}{It's time to go home.}}
\entry{täräp2}{\headword{täräp2}\pos{A vt.}\sensenumber{2}\definition{to portion, share, split, divide}\example{Sɨmell de täräp dägagän.}{She portioned out the pig.}}
\entry{täräp3}{\headword{täräp3}\pos{n.}\sensenumber{2}\definition{hunting tally (collection of jaw bones often displayed in the front of people's homes)}}
\entry{tärbatt}{\headword{tärbatt}\pos{n.}\sensenumber{2}\definition{orphan}}
\entry{täre}{\headword{täre}\pos{n.}\sensenumber{1}\definition{feast}\example{Lla da ttongo sande makäp me mɨnyi täre we gonsärbemamalle.}{People took one week to prepare for the feast.}\sensenumber{2}\definition{holy, sacred}\sensenumber{2}\definition{Scripture}\sensenumber{2}\definition{temple}\subentry{\headword{täre buk}\pos{n.}\definition{Scripture}}\subentry{\headword{täre ma}\pos{n.}\definition{temple}}}
\entry{tärke käp}{\headword{tärke käp}\pos{n.}\sensenumber{2}\definition{necklace}}
\entry{tärko}{\headword{tärko}\pos{n.}\sensenumber{2}\definition{type of small fish}}
\entry{tärpa}{\headword{tärpa}\pos{A vt.}\sensenumber{1}\definition{to overcook, burn}\sensenumber{2}\definition{yam skin}\sensenumber{3}\definition{leftovers}}
\entry{tärpae}{\headword{tärpae}\pos{n.}\sensenumber{3}\definition{type of yam}}
\entry{tärpam}{\headword{tärpam}\pos{S vt.}\sensenumber{1}\definition{to put across}\example{Ubi yuwog abal llo de datärpaemaemeyo walle me.}{They put many sticks across the water.}\sensenumber{2}\definition{to crucify}\example{Ubi Yesu bom datärpameyo.}{They crucified Jesus.}\sensenumber{2}\definition{cross}\example{llo tärpamatt toko me}{on top of the wooden cross}\allomorph{tärp}\etymology{tärpam + =att}\subentry{\headword{tärpamatt}\pos{n.}\definition{cross}}}
\entry{tärpan}{\headword{tärpan}\pos{v.}\sensenumber{2}\definition{cut}\allomorph{tärpän}}
\entry{tärpi}{\headword{tärpi}\pos{mod.}\sensenumber{2}\definition{slippery, smooth}\example{Nyongo da tärpi dan.}{The road is slippery.}\sensenumber{2}\definition{nonsingular form of tärpi}\example{Ngäna kumuddäga ttägäll käp tärpitärpi de nabllenegan.}{I found three smooth stones.}\subentry{\headword{tärpitärpi}\pos{mod.}\definition{nonsingular form of tärpi}}}
\entry{tärpoll}{\headword{tärpoll}\pos{n.}\sensenumber{2}\definition{piece}\example{llo palltapallta torpoll me}{on a flat piece of wood}}
\entry{tät1}{\headword{tät1}\pos{n.}\sensenumber{1}\definition{stretcher}\example{Ngänäm tät alle datrameyo.}{They carried me with a stretcher.}\sensenumber{2}\definition{ladder}}
\entry{tät2}{\headword{tät2}\pos{n.}\sensenumber{2}\definition{type of insect}}
\entry{tätäb}{\headword{tätäb}\pos{n.}\sensenumber{2}\definition{swelling}}
\entry{tätäk}{\headword{tätäk}\pos{n.}\sensenumber{2}\definition{flea}}
\entry{tätäm}{\headword{tätäm}\pos{adv.}\sensenumber{2}\definition{yesterday}}
\entry{tätämall}{\headword{tätämall}\pos{n.}\sensenumber{2}\definition{type of small insect that glow green}}
\entry{Tätän}{\headword{Tätän}\pos{pn.}\sensenumber{2}\definition{unisex personal name}}
\entry{tätän}{\headword{tätän}\pos{n.}\sensenumber{1}\definition{rib, side, flank}\example{sɨmell tätän}{pork rib}\sensenumber{2}\definition{animal trap}\example{Käza da aittan Martin bo tätän me.}{The crocodile was caught in Martin's trap.}\sensenumber{2}\definition{rib (bone)}\subentry{\headword{tätän kutt}\pos{n.}\definition{rib (bone)}}}
\entry{tätäp}{\headword{tätäp}\pos{n.}\sensenumber{2}\definition{pain}\example{Bogo tätäp de umllang gogän.}{He felt pain.}}
\entry{tätäräk}{\headword{tätäräk}\variant{var. of}{täträk}}
\entry{tätäräp1}{\headword{tätäräp1}\pos{S vt.}\sensenumber{1}\definition{to cut}\example{Ibi goeg tatäräp e.}{Let's go to the garden to cut down trees.}\example{Ngäna ngämo llan de yatärpänan.}{I cut my ears.}\sensenumber{2}\definition{to cut across, take a shortcut}\example{Minong gänyaollemae ditäräpmällnän ngämo pate.}{Minong took a shortcut this way towards me.}\sensenumber{3}\definition{man-made clearing}\allomorph{täräp}\allomorph{träp}\allomorph{tärp}\allomorph{terp}}
\entry{tätäräp2}{\headword{tätäräp2}\pos{n.}\sensenumber{3}\definition{heat}\example{Ngäna yäbäd tätäräp atta gäba we inggol allan.}{I am moving to the shade because of the sun's heat.}\example{Yäbäd a ebdo kälakälae tätäräpang dan, be iddob ag mer dan zanggae e.}{The sun is a bit hot during the day, but the night or morning are good for roaming.}}
\entry{tätärpeyam}{\headword{tätärpeyam}\pos{n.}\sensenumber{3}\definition{type of grass (\textbackslashtextasciitilde1 m) used as a toy}}
\entry{tätkea}{\headword{tätkea}\pos{n.}\sensenumber{3}\definition{type of sagoTubutubu ngällngällang dan. (It grows long.)}}
\entry{tätmar}{\headword{tätmar}\pos{n.}\sensenumber{3}\definition{firefly}}
\entry{täträk}{\headword{täträk}\pos{S vi.}\sensenumber{1}\definition{to go underneath}\example{Käsre dibaballe ada gogon goträkän.}{Then from there, he went underneath.}\sensenumber{2}\definition{to push in, push through}\example{Ngämo giri de käza bo ume ttäp e däträkne.}{I was pushing in my knife into the crocodile's mouth.}\allomorph{träk}\allomorph{täräk}}
\entry{täträp1}{\headword{täträp1}\variant{fast speech var. of}{tätäräp1}}
\entry{täträp2}{\headword{täträp2}\pos{n.}\sensenumber{2}\definition{season of the end of harvesting and the start of hunting (eighth season; corresponds to late May)}}
\entry{teb}{\headword{teb}\pos{n.}\sensenumber{2}\definition{mandible, jawbone}}
\entry{Tebar}{\headword{Tebar}\pos{pn.}\sensenumber{2}\definition{Tebar (toponym)}}
\entry{Teks}{\headword{Teks}\pos{pn.}\sensenumber{2}\definition{male personal name}}
\entry{teks}{\headword{teks}\pos{n.}\sensenumber{2}\definition{tax}\example{teks mani ngällbänanang lla}{tax collector}\etymology{from Englishtax}}
\entry{Teme}{\headword{Teme}\variant{dial. var. of}{Tame}}
\entry{ten}{\headword{ten}\pos{num.}\sensenumber{2}\definition{ten (English numeral, also general)}\etymology{from Englishten}}
\entry{tentatente}{\headword{tentatente}\pos{n.}\sensenumber{2}\definition{category of parasitic plants}}
\entry{tep1}{\headword{tep1}\pos{n.}\sensenumber{2}\definition{tree sap (used in making drums and arrows)Dum, ttäle, ttek a ip llo tep. (Dum, ttäle, ttek, and ip tree sap.)}}
\entry{Tergo}{\headword{Tergo}\pos{pn.}\sensenumber{2}\definition{male personal name}}
\entry{tergony}{\headword{tergony}\pos{S vt.}\sensenumber{2}\definition{to spread, unfold}\example{Lla da tater de mɨnyi bitergonyeyo.}{The people will spread mats.}}
\entry{Terrance}{\headword{Terrance}\pos{pn.}\sensenumber{2}\definition{male personal name}}
\entry{tesde}{\headword{tesde}\pos{n.}\sensenumber{2}\definition{Thursday}\etymology{from EnglishThursday}}
\entry{teste}{\headword{teste}\variant{sp. var. of}{tesde}}
\entry{teti}{\headword{teti}\pos{num.}\sensenumber{2}\definition{thirty}\etymology{from Englishthirty}}
\entry{tetin}{\headword{tetin}\pos{num.}\sensenumber{2}\definition{thirteen (English numeral)}\etymology{from Englishthirteen}}
\entry{Tewa}{\headword{Tewa}\pos{pn.}\sensenumber{2}\definition{male personal name}}
\entry{Tewara}{\headword{Tewara}\pos{pn.}\sensenumber{2}\definition{Tewara (Bitur-speaking village in Oriomo-Bitur Rural LLG)}}
\entry{teweya}{\headword{teweya}\pos{n.}\sensenumber{2}\definition{brush cuckoo}}
\entry{Teyapopo}{\headword{Teyapopo}\pos{pn.}\sensenumber{2}\definition{Teyapopo (toponym)}}
\entry{teyateyar}{\headword{teyateyar}\pos{mod.}\sensenumber{2}\definition{shallow}}
\entry{thirty}{\headword{thirty}\variant{sp. var. of}{teti}}
\entry{Thomas}{\headword{Thomas}\variant{sp. var. of}{Tomas}}
\entry{thousand}{\headword{thousand}\variant{sp. var. of}{taosen}}
\entry{three}{\headword{three}\variant{sp. var. of}{tri}}
\entry{tibekllop}{\headword{tibekllop}\pos{n.}\sensenumber{2}\definition{type of medicineOtnan ma dan källa mam täräp me. (It's taken when stools are bloody.)}}
\entry{tibi}{\headword{tibi}\pos{n.}\sensenumber{2}\definition{tuberculosis}\etymology{from EnglishTB}}
\entry{tibra}{\headword{tibra}\pos{n.}\sensenumber{2}\definition{type of native bananaPätt a ulle dan ako tupi dan, obo käp a o meae otät ma dan, yu ma dan. Be kire da ddone yu ma dan. (The trunk is big and long; only when ripe, it's eaten; it's cooked. But when unripe, it's not cooked.)}}
\entry{tigi}{\headword{tigi}\pos{n.}\sensenumber{2}\definition{type of snakeDdäddäg ma gullem ulle dan. Bätbät dan. Wälläng me giddollag dan. (It's a big, edible snake. It's black. It lives in the bush.)}}
\entry{tikle}{\headword{tikle}\pos{n.}\sensenumber{2}\definition{type of small louse}}
\entry{tikop}{\headword{tikop}\pos{n.}\sensenumber{1}\definition{heart}\sensenumber{2}\definition{beloved, sweetheart, dear}\sensenumber{2}\definition{startled, scared}\example{Lla da ttongo lla de aiyei abal tikopmeny nägagan.}{The man really scared the other man.}\example{Madura tikopmeny gogän.}{Madura was startled.}\etymology{tikop + =meny, lit. 'heartless'}\subentry{\headword{tikopmeny}\pos{mod.}\definition{startled, scared}}}
\entry{tikuku}{\headword{tikuku}\pos{n.}\sensenumber{1}\definition{black-backed bitternTowall me giddollag pullkom tupiang dan pa da. (It's a bird with long tailfeathers that lives in the grass.)}\sensenumber{2}\definition{pheasant coucal}}
\entry{Tim}{\headword{Tim}\pos{pn.}\sensenumber{2}\definition{female personal name}}
\entry{timän}{\headword{timän}\pos{S vt.}\sensenumber{1}\definition{to release, fire, shoot (an arrow)}\example{Ibi mɨnyi toboll de bitimänmällnegalla.}{We (incl.) will be firing arrows.}\sensenumber{2}\definition{end}\example{Ekaklle ulle bo timän a dowae abal agan zime.}{The end of the world is near.}}
\entry{Tina}{\headword{Tina}\pos{pn.}\sensenumber{2}\definition{female personal name}}
\entry{tine}{\headword{tine}\pos{n.}\sensenumber{1}\definition{sago leaf}\sensenumber{2}\definition{type of simple roof consisting of a single post covered in sago leaves}}
\entry{tintɨrmoll}{\headword{tintɨrmoll}\variant{sp. var. of}{tintromol}}
\entry{tintromol}{\headword{tintromol}\pos{n.}\sensenumber{2}\definition{type of black or red ants}}
\entry{tintromoltintromol}{\headword{tintromoltintromol}\pos{n.}\sensenumber{2}\definition{type of rhyming game}\etymology{redup. of tintromol}}
\entry{Tirere}{\headword{Tirere}\pos{pn.}\sensenumber{2}\definition{Tirere/Tire'ere (Waboda-speaking village in Kiwai Rural LLG)}}
\entry{tirit}{\headword{tirit}\pos{n.}\sensenumber{2}\definition{cotton}}
\entry{Titi}{\headword{Titi}\pos{pn.}\sensenumber{2}\definition{Titi (toponym)}}
\entry{titi1}{\headword{titi1}\pos{n.}\sensenumber{1}\definition{brown honeyeaterAwi me papllägag pa kälsre dan. (It's a small bird that flies in the evening.)}\sensenumber{2}\definition{brown-backed honeyeater}\sensenumber{3}\definition{yellow-bellied longbill}\sensenumber{4}\definition{rufous-banded honeyeater}\sensenumber{5}\definition{pygmy longbill}}
\entry{titi2}{\headword{titi2}\pos{n.}\sensenumber{5}\definition{batTopoll ingoll dan a kälsäre dan. Ddobae llo käp otnanang dag iddob me. (It's like a flying fox, but small. It eats a lot of fruit at night.)}\example{Titi da ddobae sawasap otnan kämang dag.}{Bats eat lots of sawasap fruit.}}
\entry{Titus}{\headword{Titus}\pos{pn.}\sensenumber{5}\definition{male personal name}}
\entry{Tizag}{\headword{Tizag}\pos{pn.}\sensenumber{5}\definition{Ende dialect}}
\entry{tizag}{\headword{tizag}\variant{sp. var. of}{Tizag}}
\entry{tɨke}{\headword{tɨke}\pos{n.}\sensenumber{5}\definition{magic\textbackslash_type}}
\entry{tɨn}{\headword{tɨn}\pos{n.}\sensenumber{5}\definition{steam}}
\entry{tɨne}{\headword{tɨne}\variant{var. of}{tine}}
\entry{tɨram}{\headword{tɨram}\pos{S vt.}\sensenumber{5}\definition{to open}\example{Ge ud a täramang daeya, be sisri papekatt dan.}{This door was open, but now it's closed.}\allomorph{tɨraem}}
\entry{tɨrangesa}{\headword{tɨrangesa}\variant{var. of}{tärangesa}}
\entry{tɨt1}{\headword{tɨt1}\pos{S vt.}\sensenumber{1}\definition{to beat sago, pound sago}\example{Ubi sana pätt att de tɨt erallo a kängkäm erallo.}{They pound the pith of the sago palm and squeeze it.}\example{Bogo mɨnyi sana bɨtɨn.}{He will pound sago.}\sensenumber{2}\definition{to hollow out, dig out}\example{Ngäna gall de tɨt iran.}{I am digging out a canoe.}\example{Bäne turik de abo iweny gall tɨt i.}{Bring your axe to dig the canoe.}\allomorph{tɨnen}\allomorph{ɨt}\allomorph{t}\allomorph{tɨ}}
\entry{tɨt2}{\headword{tɨt2}\pos{n.}\sensenumber{2}\definition{type of tree found in the bush}}
\entry{tɨtɨp}{\headword{tɨtɨp}\pos{S vt.}\sensenumber{2}\definition{to braid}\example{Bongo bäne bun de mermerae nätɨpnegalle.}{You braided your hair nicely.}\allomorph{tɨp}}
\entry{tɨtɨrɨk}{\headword{tɨtɨrɨk}\variant{var. of}{täträk}}
\entry{tlläpmälltlläpmäll}{\headword{tlläpmälltlläpmäll}\pos{adv.}\sensenumber{2}\definition{nibbling}}
\entry{to}{\headword{to}\pos{n.}\sensenumber{2}\definition{light}\example{Ge to ulle ngäna ikop däga ereya, ngäna lelang atta dindug.}{I saw this big light and ran out of fright.}}
\entry{toball}{\headword{toball}\variant{sp. var. of}{tobäll}}
\entry{tobäll}{\headword{tobäll}\pos{n.}\sensenumber{2}\definition{long spear shot like an arrowDdäddäg, kollba, pa zuwema dan. (For shooting game, fish, and birds.)}\sensenumber{1}\definition{arrowheadDdäddäg gäz ma dan kuddäll e. (It's for killing animals.)}\sensenumber{2}\definition{bullet}\subentry{\headword{tobäll käp}\pos{n.}\definition{arrowheadDdäddäg gäz ma dan kuddäll e. (It's for killing animals.)}}}
\entry{tobäll abal}{\headword{tobäll abal}\pos{n.}\sensenumber{2}\definition{type of spear}}
\entry{toboll}{\headword{toboll}\variant{dial. var. of}{tobäll}}
\entry{toe}{\headword{toe}\pos{n.}\sensenumber{2}\definition{type of tree that grows in the bush with white flowers, black fruit, small edible nuts that doves and cassowaries eat, and pith that is steamed or extracted}}
\entry{toelet}{\headword{toelet}\pos{n.}\sensenumber{2}\definition{toilet}\etymology{from Englishtoilet}}
\entry{toengg}{\headword{toengg}\pos{n.}\sensenumber{2}\definition{confluence, junction (of a river)}}
\entry{Togllaema}{\headword{Togllaema}\pos{pn.}\sensenumber{2}\definition{Togllaema (toponym)}}
\entry{togol}{\headword{togol}\pos{S vi.}\sensenumber{1}\definition{to hide; go away to have sex secretly}\example{Ngäna ngattong botägol.}{I will hide first.}\example{Bongo zime atägolalle?}{Are you already hiding?}\sensenumber{2}\definition{to hide}\example{Ngämim deyatogolmällneyo ngäma mäg.}{Our (excl.) mothers were hiding us.}\sensenumber{2}\definition{hide-and-seek}\example{Togotogol tongoe de ubi dangesneyo oba mokwang tongoe.}{They played hide-and-seek, their favorite game.}\example{Ibi togotogol botongoenallaǃ}{Let's play hide and seek!}\allomorph{tgol}\allomorph{tägol}\allomorph{täga}\allomorph{tge}\allomorph{togo}\subentry{\headword{togotogol}\pos{n.}\definition{hide-and-seek}}}
\entry{Togowa}{\headword{Togowa}\pos{pn.}\sensenumber{2}\definition{Togowa (toponym)}}
\entry{tok}{\headword{tok}\pos{A vt.}\sensenumber{2}\definition{to wrap}\example{Bogo sɨmell midd di tok digewän llo ttam alle.}{He wraps the pork with leaves.}}
\entry{Tok pisin}{\headword{Tok pisin}\pos{pn.}\sensenumber{2}\definition{Tok Pisin}}
\entry{Tok pizin}{\headword{Tok pizin}\variant{var. of}{Tok pisin}}
\entry{toko}{\headword{toko}\pos{loc.}\sensenumber{2}\definition{top}\example{Bogo llo toko me gotarän.}{She slept in the treetop.}}
\entry{tokom}{\headword{tokom}\pos{n.}\sensenumber{2}\definition{bait}}
\entry{tokong}{\headword{tokong}\pos{n.}\sensenumber{2}\definition{bait}\example{Mälla da kok de nägäddaeballo kollba tokong e.}{The women caught grasshoppers for bait.}}
\entry{tokop}{\headword{tokop}\pos{n.}\sensenumber{1}\definition{type of tree that grows in the bush; used as house sticks and medicine}\sensenumber{2}\definition{lump on skin}}
\entry{toma}{\headword{toma}\pos{n.}\sensenumber{2}\definition{wing}\sensenumber{2}\definition{metathorax}\subentry{\headword{toma mit}\pos{n.}\definition{metathorax}}}
\entry{Tomas}{\headword{Tomas}\pos{pn.}\sensenumber{2}\definition{male personal name}}
\entry{Tomato}{\headword{Tomato}\pos{pn.}\sensenumber{2}\definition{female personal name}}
\entry{tomato}{\headword{tomato}\pos{n.}\sensenumber{2}\definition{tomato}\etymology{from Englishtomato}}
\entry{tomäll}{\headword{tomäll}\pos{n.}\sensenumber{2}\definition{wart; fungal skin infection}}
\entry{tomon}{\headword{tomon}\pos{S vi.}\sensenumber{1}\definition{to wait}\example{Ngäna kälakälae gontämon do yäbäd a angde goklowän dam ngäna ibi de gongkam.}{I waited a little until the sun set; then I started walking.}\sensenumber{2}\definition{to wait for, await}\example{Ngäna mänmän e bantmonneg.}{I'll wait for the girls.}\example{Bogo obom deyantämonän.}{She awaited him.}\allomorph{ntämon}\allomorph{ntomony}\allomorph{tomonen}\allomorph{ntmon}\allomorph{ntomon}}
\entry{tomowang}{\headword{tomowang}\pos{mod.}\sensenumber{2}\definition{sour; bitter}\example{Ge wup bo moko da kälakälae tomowang dan.}{This banana's taste is a little sour.}}
\entry{Tomson}{\headword{Tomson}\pos{pn.}\sensenumber{2}\definition{male personal name}}
\entry{tonang}{\headword{tonang}\pos{mod.}\sensenumber{2}\definition{careful, cautious}\example{Bogo tonangae damllaemnegän.}{She held on carefully.}\example{Abo tonangae agaebne ubira.}{You all must be wary of them.}\sensenumber{2}\definition{carefully, cautiously}\example{Ngäna mɨnyi yu de tonangtonang bikom.}{I'll bring the wood carefully.}\subentry{\headword{tonangtonang}\definition{carefully, cautiously}}}
\entry{Toni}{\headword{Toni}\pos{pn.}\sensenumber{2}\definition{male personal name}}
\entry{tonton}{\headword{tonton}\pos{adv.}\sensenumber{2}\definition{directly}\example{Ewatta ke gänya täräp me lla bombllo da wasnen anggan ttowamang ttoen tonton ikop e ddapall att?}{Why does this generation beg to see miracles directly from heaven?}}
\entry{Tonzah}{\headword{Tonzah}\pos{pn.}\sensenumber{2}\definition{male personal name}}
\entry{tongg}{\headword{tongg}\pos{S vt.}\sensenumber{2}\definition{to point}\example{Liseng ttang dontogän Rose bom.}{Liseng pointed her finger at Rose.}\allomorph{ntog}}
\entry{tonggo}{\headword{tonggo}\pos{n.}\sensenumber{2}\definition{type of small bamboo that grows in the bush along creeks with a red interior; sharp when split and used as a cutting tool}}
\entry{tongle}{\headword{tongle}\pos{n.}\sensenumber{2}\definition{leechUte me ddogollag dan, pätt me mɨnyi boddgollän misdae. (It sticks to sores; it will just stick to the body.)}}
\entry{tongoe}{\headword{tongoe}\pos{S vi.}\sensenumber{1}\definition{to play}\example{Nga ubi dag tongoenen meae anggan.}{They are still playing.}\example{Llɨg kälekäle da ebdo ulle atongoenegnan.}{The children were playing all day.}\example{Ibi togotogol botongoe nalla.}{Let's play hide and seek.}\sensenumber{2}\definition{to laugh (at)}\example{Ubi ttongo lla de tongoenen erallo.}{They laugh at someone.}\example{Lama käsre ngämira llɨtɨt dängkamän tongoetongoeang.}{Lama started to tell us (excl.) the story, laughing.}\sensenumber{3}\definition{game; sport}\example{Ngämo mokowang tongoe a basketbol dan.}{My favorite game is basketball.}\sensenumber{3}\definition{playground}\sensenumber{3}\definition{funny, humorous}\example{tongoeang ttoen}{funny story}\allomorph{tngoe}\allomorph{tongoetongoe}\etymology{tongoe + =ang}\subentry{\headword{tongoe ma ngätt}\pos{n.}\definition{playground}}\subentry{\headword{tongoeang}\pos{mod.}\definition{funny, humorous}}}
\entry{topoll}{\headword{topoll}\pos{n.}\sensenumber{3}\definition{flying foxLlo toko me gul mi giddollnenang, iddob me papllägag dan llo popo nanen e. (They live in groups in trees; they fly at night to drink flower nectar.)}\example{Abo ibi topoll gäddnan e dade angde bobäll.}{We can go kill flying foxes any time.}}
\entry{topotopoll}{\headword{topotopoll}\pos{n.}\sensenumber{3}\definition{type of tree that grows in the bush with white flowers; used as a yam stick}\etymology{redup. of topoll}}
\entry{torep}{\headword{torep}\pos{n.}\sensenumber{3}\definition{brown quailTowall ik me ergodag pa kälsre dan. (It's a small bird that lurks in the grass.)}}
\entry{toro}{\headword{toro}\pos{n.}\sensenumber{3}\definition{a symbol of a slain animal, used to display and inform people of the type of animal killed. It was also a hunter's pride to compete or show other hunter' of his skill. Feathers, offcut tail, or animal fur were displayed on a small stick or pitpit or obäll tree stick or stem. In other instances pandanus leaves were symbols for pig, grass for wallaby.}}
\entry{torok}{\headword{torok}\pos{n.}\sensenumber{3}\definition{type of cane used for building houses, bows, and canoes}}
\entry{Torok mittang}{\headword{Torok mittang}\pos{pn.}\sensenumber{3}\definition{Torok mittang (toponym)}}
\entry{toronggogo}{\headword{toronggogo}\pos{n.}\sensenumber{3}\definition{bar-shouldered doveEkaklle me ergodag pa dan. (It's a bird that lurks on the ground.)}}
\entry{torpoll}{\headword{torpoll}\variant{sp. var. of}{tärpoll}}
\entry{torwam}{\headword{torwam}\pos{S vi.}\sensenumber{1}\definition{to lie down}\example{tät me torwamang lla}{man lying down on a stretcher}\example{Ngäna torwam allan.}{I am lying down.}\sensenumber{2}\definition{to lay down}\example{Ngäna obom buntruwam.}{I will lay him down.}\sensenumber{2}\definition{causative-applicative form of torwam}\example{Ngäna bam dantruwamängg.}{I made you lie down.}\allomorph{ntruwam}\allomorph{truwam}\etymology{torwam + -ngg}\subentry{\headword{torwamängg}\pos{S vt.}\definition{causative-applicative form of torwam}}}
\entry{tos}{\headword{tos}\pos{n.}\sensenumber{2}\definition{(Commonwealth) torch, (US) flashlight}\example{Ngäna tos de dɨs.}{I turned off the flashlight.}\etymology{from Englishtorch}}
\entry{tot1}{\headword{tot1}\pos{n.}\sensenumber{2}\definition{type of tree that gows along creeks with white and blue flowers and bark used to weave bags or sago baskets}}
\entry{tot2}{\headword{tot2}\pos{n.}\sensenumber{2}\definition{rubbish, trash, junk}\example{käm tot}{feces (lit. 'stomach trash')}\example{emaemae tot}{all kinds of junk}}
\entry{tot3}{\headword{tot3}\pos{n.}\sensenumber{2}\definition{piece}\example{Bäne aoli tot nyukukum dag?}{How many nyukukum bags do you have?}}
\entry{totkoll}{\headword{totkoll}\pos{n.}\sensenumber{2}\definition{puddle}}
\entry{toto1}{\headword{toto1}\pos{n.}\sensenumber{2}\definition{afternoon; early evening (approx. 1 PM–5 PM)}\sensenumber{2}\definition{dinner}\example{Ge ade ngämo toto duwem e dan.}{This is for my dinner.}\subentry{\headword{toto duwem}\pos{n.}\definition{dinner}}}
\entry{toto2}{\headword{toto2}\pos{n.}\sensenumber{2}\definition{vertical house post}\sensenumber{2}\definition{horizontal house post}\subentry{\headword{toto botta}\pos{n.}\definition{horizontal house post}}}
\entry{toto nyäknyäk}{\headword{toto nyäknyäk}\pos{n.}\sensenumber{2}\definition{Australian owlet-nightjarDämar dädär ik me giddollag pa dan. (It's a bird that lives in dry dämar palms.)}}
\entry{totoe}{\headword{totoe}\pos{v.}\sensenumber{2}\definition{conceive}}
\entry{towall}{\headword{towall}\pos{n.}\sensenumber{2}\definition{grass}\example{Ttall a towall wätätang dan.}{Wallabies are grass eaters.}\sensenumber{2}\definition{grassy}\example{Ge ttängäm a towallang dan.}{This village is grassy.}\etymology{towall + =ang}\subentry{\headword{towallang}\pos{mod.}\definition{grassy}}}
\entry{towallpipi}{\headword{towallpipi}\pos{n.}\sensenumber{2}\definition{type of venomous snakeTtongo pällämpälläm gullem kälsre da. Lla ddäddägmeny dan. (It's a small white snake. Humans can't eat it.)}}
\entry{Towarwamang}{\headword{Towarwamang}\pos{pn.}\sensenumber{2}\definition{Towarwamang (toponym)}}
\entry{traeb}{\headword{traeb}\pos{n.}\sensenumber{2}\definition{tribe}\etymology{from Englishtribe}}
\entry{traib}{\headword{traib}\variant{sp. var. of}{traeb}}
\entry{trak1}{\headword{trak1}\pos{n.}\sensenumber{2}\definition{truck}\etymology{from Englishtruck}}
\entry{trak2}{\headword{trak2}\variant{fast speech var. of}{tärak}}
\entry{tram}{\headword{tram}\variant{fast speech var. of}{täram2}}
\entry{trangg}{\headword{trangg}\variant{fast speech var. of}{tärangg}}
\entry{tratre}{\headword{tratre}\pos{v.}\sensenumber{2}\definition{hollow}}
\entry{träk1}{\headword{träk1}\variant{fast speech var. of}{täräk1}}
\entry{träk2}{\headword{träk2}\variant{var. of}{trak1}}
\entry{träm}{\headword{träm}\variant{fast speech var. of}{täräm}}
\entry{tri}{\headword{tri}\pos{num.}\sensenumber{2}\definition{three (English numeral)}\etymology{from Englishthree}}
\entry{trik}{\headword{trik}\pos{n.}\sensenumber{2}\definition{trick}\etymology{from Englishtrick}}
\entry{trimpeg}{\headword{trimpeg}\pos{v.}\sensenumber{2}\definition{slide}\allomorph{trimpe}}
\entry{trongg}{\headword{trongg}\pos{S vt.}\sensenumber{2}\definition{to follow}\example{Ngäna däbe ddage dantrog do.}{I followed that stream there.}\allomorph{ntrog}\allomorph{ntromeny}\allomorph{ntro}}
\entry{truamängg}{\headword{truamängg}\variant{var. of}{torwamängg}}
\entry{trungg}{\headword{trungg}\pos{S vt.}\sensenumber{2}\definition{to invite, call over, summon}\example{Ngänäm ngämo nag dantrugän Upiara we wayati kongkom e.}{My friend invited me to Upiara to collect watermelons.}\example{Ngäna bam trungg allan, wiya!}{I am calling you over, come!}\allomorph{ntrug}\allomorph{truminy}\allomorph{trug}\allomorph{ntru}}
\entry{tu}{\headword{tu}\pos{num.}\sensenumber{2}\definition{two (English numeral)}\etymology{from Englishtwo}}
\entry{tuba}{\headword{tuba}\pos{n.}\sensenumber{2}\definition{coconut drink traditionally made to welcome people}}
\entry{Tube}{\headword{Tube}\pos{pn.}\sensenumber{2}\definition{male personal name}}
\entry{tubi}{\headword{tubi}\pos{n.}\sensenumber{2}\definition{type of spear made from sago leaf used to kill birds}}
\entry{Tubu}{\headword{Tubu}\pos{pn.}\sensenumber{2}\definition{male personal name}}
\entry{tubu1}{\headword{tubu1}\pos{mod.}\sensenumber{1}\definition{short}\example{llo pallkoll tubu}{short piece of wood}\sensenumber{2}\definition{end; stump}\example{llo tubu, rop tubu}{tree stump, end of rope}\sensenumber{2}\definition{a bit short}\example{Ngäna tubutubu dan be ngämo män a tupi dan.}{I am shorter than my sister.}\sensenumber{2}\definition{nonsingular form of tubu}\example{Däräng bo llan a tubutubu dag.}{The dog's ears are short.}\subentry{\headword{tubutubu1}\pos{mod.}\definition{a bit short}}\subentry{\headword{tubutubu2}\pos{mod.}\definition{nonsingular form of tubu}}}
\entry{tubu2}{\headword{tubu2}\pos{n.}\sensenumber{2}\definition{knee}\example{Ngämo yae obo tubu da apte gagäll dan.}{One of my mother's knees is bad.}\sensenumber{2}\definition{kneecap}\sensenumber{2}\definition{kneeling, on one's knees, on the ground; worshipping}\example{Bogo tubutubu gogon, obo ingoll alle ttäle dowae me.}{He kneeled, his face by her feet.}\example{Zuu lla da Sabat ebdo me sipel gognegnän a Adi pate tubutubu gognegnän.}{The Jews rested and worshipped God on the Sabbath.}\etymology{tubu + kätt}\subentry{\headword{tubukätt}\pos{n.}\definition{kneecap}}\subentry{\headword{tubutubu3}\pos{mod.}\definition{kneeling, on one's knees, on the ground; worshipping}}}
\entry{tudi}{\headword{tudi}\pos{n.}\sensenumber{1}\definition{fishing}\example{Mälla da tudi ma dallän, ako bogo tudi gognän.}{The woman went to go fishing; then she was fishing.}\sensenumber{2}\definition{fishing rod, fishing line}\example{Obo tudi di kollba ulle da danykoeyän.}{A big fish pulled on his line.}\sensenumber{2}\definition{fishing style in which bait is put on fishing lines that are left in the river and checked later}\sensenumber{2}\definition{fishing hookIttnen ma dan. (It's for catching fish.)}\example{Tudi käp sapasapang alle mɨnyi tudi bognegnän.}{They will fish with various kinds of hooks.}\sensenumber{2}\definition{fishing lineTudi käp bo tär ma dan. (It's the string of the fishing hook.)}\example{Tudi tär me guittän.}{It was caught on the fishing line.}\subentry{\headword{tudi ittaenen}\pos{n.}\definition{fishing style in which bait is put on fishing lines that are left in the river and checked later}}\subentry{\headword{tudi käp}\pos{n.}\definition{fishing hookIttnen ma dan. (It's for catching fish.)}}\subentry{\headword{tudi tär}\pos{n.}\definition{fishing lineTudi käp bo tär ma dan. (It's the string of the fishing hook.)}}}
\entry{tugul}{\headword{tugul}\pos{n.}\sensenumber{2}\definition{type of big tree that grows in the bush with green, leaflike flowers and straight wood used for house sticks}}
\entry{tuk}{\headword{tuk}\pos{n.}\sensenumber{1}\definition{air}\example{Ngäna tuk i pipllugag dan.}{I fly into the air.}\sensenumber{2}\definition{top}\example{Gänya anykeanyke da gänyan tuk mi.}{This picture is on top.}\sensenumber{2}\definition{technical university, university}\sensenumber{1}\definition{uphill}\sensenumber{2}\definition{more than, above, over}\example{ttongo taosen tukituki}{more than one thousand}\etymology{redup. of tuk + =e₁}\subentry{\headword{tuk skul}\pos{n.}\definition{technical university, university}}\subentry{\headword{tukituki}\pos{adv.}\definition{uphill}}}
\entry{tuk sukul}{\headword{tuk sukul}\variant{sp. var. of}{tuk skul}}
\entry{tukpi}{\headword{tukpi}\pos{A vt.}\sensenumber{2}\definition{to heap, pile; gather, collect}\example{Bogo mɨnyi bikomän a ttongdae ngättma me tukpi bägnegän.}{He will bring them over and pile them up in one spot.}}
\entry{tulgoe}{\headword{tulgoe}\pos{S vt.}\sensenumber{2}\definition{to gossip about}\example{Tätäm mälla da ngänäm era daitulgoeneyo duli.}{Yesterday, women were gossiping about me there.}\allomorph{itulgoe}}
\entry{tum1}{\headword{tum1}\pos{mod.}\sensenumber{2}\definition{angry}\example{Pol bälle enda tum gogon.}{Paul got very angry.}}
\entry{tum2}{\headword{tum2}\pos{n.}\sensenumber{2}\definition{heap}\example{Bogo käg me tum dägagän.}{She put it in a heap in the container.}\sensenumber{2}\definition{plenty, many}\example{Ngäna kollba tumang de naittnegan.}{I caught plenty of fish.}\example{Tumang lla da Ende eka de panypeny erallo.}{Many people speak the Ende language.}\sensenumber{2}\definition{to gather}\example{Tämamae ddäddäg de tumtum nägagallo.}{They gathered all the carcasses.}\example{Ngämi tumtum gogmam.}{We (excl.) gathered.}\etymology{tum + =ang}\subentry{\headword{tumang}\pos{quant.}\definition{plenty, many}}\subentry{\headword{tumtum}\pos{A vi. \textbackslash& vt.}\definition{to gather}}}
\entry{tumku}{\headword{tumku}\pos{n.}\sensenumber{2}\definition{back of head}}
\entry{Tungnu}{\headword{Tungnu}\pos{pn.}\sensenumber{2}\definition{Tungnu (toponym)}}
\entry{tupi1}{\headword{tupi1}\pos{mod.}\sensenumber{1}\definition{tall}\example{Da bongo lla tupi di nälläd, bongo mɨnyi llɨg kamebiag ag.}{If you marry the tall man, you will have many children.}\sensenumber{2}\definition{long}\example{Bituri walle da tupi abal dan.}{The Bituri river is very long.}\sensenumber{2}\definition{nonsingular form of tupi}\example{Ngämo llan a tubutubu agnegän, obo da tupitupi.}{My ears are short; his are long.}\allomorph{tu}\allomorph{tpi}\subentry{\headword{tupitupi}\pos{mod.}\definition{nonsingular form of tupi}}}
\entry{tupi2}{\headword{tupi2}\pos{n.}\sensenumber{1}\definition{pointer finger, index finger}\sensenumber{2}\definition{four (lit. pointer finger; body counting numeral)}\example{tupi ollong}{four times}}
\entry{tupol}{\headword{tupol}\pos{n.}\sensenumber{2}\definition{type of spear}}
\entry{turik}{\headword{turik}\pos{n.}\sensenumber{2}\definition{axeLlo ttanen a yu pälläknen ma dan. Melem buddog ngasnen ma za. (It's for chopping down trees and splitting wood. A thing for doing hard work.)}\example{Bäne turik de abo iweny gall tɨt i.}{Bring your axe to hollow out the tree.}}
\entry{turku}{\headword{turku}\pos{n.}\sensenumber{2}\definition{thinner piece used with bore to smoke tobacco}}
\entry{turllo}{\headword{turllo}\pos{n.}\sensenumber{2}\definition{type of lizard}}
\entry{turwe}{\headword{turwe}\pos{n.}\sensenumber{2}\definition{shining bronze cuckooLelmeny pa kälsre dan. (It's a fearless small bird.)}}
\entry{tusde}{\headword{tusde}\pos{n.}\sensenumber{2}\definition{Tuesday}\etymology{from EnglishTuesday}}
\entry{tutu1}{\headword{tutu1}\pos{n.}\sensenumber{1}\definition{mountain, hill}\example{Ngängälätt dädär a mänyi tutu atta bolldaeyän kopek e.}{The round rock will roll from the hill into the valley.}\sensenumber{2}\definition{land}\example{Tutu wi darullgoeya käza de gull peyang.}{We dragged the crocodile onto land with the net.}\sensenumber{2}\definition{steep}\sensenumber{2}\definition{nonsingular form of tutu}\example{Tutututu da dag.}{There are mountains.}\etymology{tutu + =ang}\subentry{\headword{tutuang}\pos{mod.}\definition{steep}}\subentry{\headword{tutututu}\pos{n.}\definition{nonsingular form of tutu}}}
\entry{tutu2}{\headword{tutu2}\variant{baby talk var. of}{susu}}
\entry{tutu3}{\headword{tutu3}\pos{S vt.}\sensenumber{2}\definition{to knock fruit continuously}}
\entry{tutuaram}{\headword{tutuaram}\pos{n.}\sensenumber{2}\definition{type of taro}}
\entry{Tutuli}{\headword{Tutuli}\pos{pn.}\sensenumber{2}\definition{male personal name}}
\entry{tuwenti}{\headword{tuwenti}\variant{sp. var. of}{twenti}}
\entry{tuwetuwe}{\headword{tuwetuwe}\pos{n.}\sensenumber{2}\definition{type of small tree that grows in the bush with white flowers and edible red fruit}}
\entry{tuwi}{\headword{tuwi}\pos{n.}\sensenumber{2}\definition{type of tree with white flowers, yellow fruit, and wood used for posts}}
\entry{tuwok}{\headword{tuwok}\pos{n.}\sensenumber{2}\definition{type of tree}}
\entry{tuyem}{\headword{tuyem}\pos{A vi.}\sensenumber{2}\definition{to make a loud noise}}
\entry{Tuyu}{\headword{Tuyu}\pos{pn.}\sensenumber{2}\definition{male personal name}}
\entry{twelb}{\headword{twelb}\pos{num.}\sensenumber{2}\definition{twelve (English numeral)}\etymology{from Englishtwelve}}
\entry{twelve}{\headword{twelve}\variant{sp. var. of}{twelb}}
\entry{twenti}{\headword{twenti}\pos{num.}\sensenumber{2}\definition{twenty}\etymology{from Englishtwenty}}
\entry{two}{\headword{two}\variant{sp. var. of}{tu}}
\lettersection{Tt tt}
\entry{ttaba}{\headword{ttaba}\pos{n.}\sensenumber{2}\definition{plant\textbackslash_type}}
\entry{Ttae}{\headword{Ttae}\pos{pn.}\sensenumber{2}\definition{male personal name}}
\entry{ttaem}{\headword{ttaem}\pos{S vt.}\sensenumber{2}\definition{to pack}\example{Lama mängalae za de dättaemnegän.}{Lama quickly packed her things.}\allomorph{nttaem}\allomorph{tt}}
\entry{ttaempäg}{\headword{ttaempäg}\pos{S vt.}\sensenumber{2}\definition{to seperate, divorce}\example{Ngämi ttaempägmeny dageyo.}{We two are inseparable.}\allomorph{ttaemp}\allomorph{nttaep}}
\entry{ttaengän}{\headword{ttaengän}\pos{S vt.}\sensenumber{2}\definition{to pull (a plant sucker)}}
\entry{ttaepnenttaepnen ma skul}{\headword{ttaepnenttaepnen ma skul}\pos{n.}\sensenumber{2}\definition{technical school}}
\entry{ttagbeag}{\headword{ttagbeag}\pos{mod.}\sensenumber{2}\definition{disorganized, careless}}
\entry{ttalam}{\headword{ttalam}\pos{S vi.}\sensenumber{2}\definition{to split, crack}\example{Käza bo ttatt a gottalamän.}{The crocodile's jaw cracked open.}\allomorph{ttalaem}\allomorph{ttelam}\allomorph{ttelaem}\allomorph{ttalam}}
\entry{ttalamttalam1}{\headword{ttalamttalam1}\pos{n.}\sensenumber{2}\definition{type of tree}}
\entry{ttalamttalam2}{\headword{ttalamttalam2}\pos{adv.}\sensenumber{2}\definition{walking with one's legs spread far apart}}
\entry{ttalängg}{\headword{ttalängg}\pos{S vi.}\sensenumber{2}\definition{to aim}\example{Kwalde mäse pakos tupi di gonttalägän sɨmell zuwoe e.}{Kwalde tried aiming the long arrow to shoot the pig.}\allomorph{nttaläg}}
\entry{ttalme}{\headword{ttalme}\pos{n.}\sensenumber{2}\definition{type of floating grass that grass, deer, and wallaby eatWalle me päddabag bolwod ingoll dan. (It grows in the water and looks like sugarcane.)}}
\entry{Ttall}{\headword{Ttall}\pos{pn.}\sensenumber{2}\definition{male personal name}}
\entry{ttall}{\headword{ttall}\pos{n.}\sensenumber{2}\definition{agile wallaby, sandy wallaby}}
\entry{ttall ip}{\headword{ttall ip}\pos{n.}\sensenumber{2}\definition{type of tree that grows in the bush, grassland, and along creeks with indigo flowers, bark that is chewed, and liquid used as glue for spears}}
\entry{ttall källa}{\headword{ttall källa}\pos{n.}\sensenumber{2}\definition{brahminy kiteTuk me ngälngälang pa ulle dan. (It's a big bird that circles in the air.)}\etymology{lit. 'wallaby poop'}}
\entry{ttall mätta}{\headword{ttall mätta}\pos{n.}\sensenumber{2}\definition{type of big yam with a white interior and few hairs}}
\entry{ttall nge}{\headword{ttall nge}\pos{n.}\sensenumber{2}\definition{type of palm with yellow leaves and coconuts with a yellow exocarp}}
\entry{ttall ttoe}{\headword{ttall ttoe}\pos{n.}\sensenumber{2}\definition{type of tree}\etymology{lit. 'wallaby skin'}}
\entry{ttallängg}{\headword{ttallängg}\pos{v.}\sensenumber{2}\definition{defend}\allomorph{ttall}}
\entry{ttam1}{\headword{ttam1}\pos{n.}\sensenumber{1}\definition{life}\example{Mälla ause da llokttang ttam me dagirnän.}{The old woman lived a hard life.}\sensenumber{2}\definition{alive}\example{Matthew kili allan adawatta bogo sisri ttam dan.}{Matthew is happy because he is alive now.}\example{Bogo kuddäll atta ttam gogon.}{She survived death.}\sensenumber{3}\definition{to save someone's life}\example{Bäne imomdae ttoen da bam ttam nagan.}{Your faith saved your life.}\sensenumber{3}\definition{life}\example{Do ttam giddoll a mer dan.}{Life there is good.}\subentry{\headword{ttam giddoll}\pos{n.}\definition{life}}}
\entry{ttam2}{\headword{ttam2}\pos{S vt.}\sensenumber{1}\definition{to call, name}\example{Da bongo mäse dowae me o utale me nänttam, ddone mɨnyi bongllaeyän adawatta obo llan a ttomoll dageyo.}{If you tried to call him, from near or far, he would not answer because he was deaf in his ears.}\example{Kaya walle do Upiara ttängämang a ttaem nallo.}{Upiara villagers call me Kaya.}\example{Kumddäga pemli de ngäna amim nättaemnegan, ngämi ubim mɨnyi umllang bägaebeya.}{The three families I named, we (excl.) will tell them.}\sensenumber{2}\definition{to confess}\example{Angde bogo gonttamän, llamda da ddone ada mikutt gogon oba pate.}{When he confessed, the old man got very mad at them.}\allomorph{ntt}\allomorph{nttam}\allomorph{ttaem}\allomorph{tt}}
\entry{ttam3}{\headword{ttam3}\pos{S vi.}\sensenumber{2}\definition{to appear}\example{Yesu ngattong abal e Meri Magdalin pate gottamän.}{Jesus appeared to Mary Magdalene first.}}
\entry{ttam4}{\headword{ttam4}\pos{n.}\sensenumber{2}\definition{leaf}\example{Ngämlle kapalla ttam alle sana yuatt a ddobae moko dan.}{I love sago cooked with kapalla leaves.}}
\entry{ttaman}{\headword{ttaman}\variant{var. of}{ttamän}}
\entry{ttamän}{\headword{ttamän}\pos{S vi.}\sensenumber{1}\definition{to finish, end}\example{Soka tongoe bo ebdo da zäme wottamänan.}{The day of the soccer game is over.}\example{Angde tine pittnen a gottamänän, dibaballe ddäganen de dängkam.}{After the sago weaving finished, I started on the roofing.}\sensenumber{2}\definition{to finish, end, complete}\example{Ngäna greid siks de dättemän.}{I completed grade six.}\example{Ngäna ngämo sana de dot dägadäga dättemän.}{I finished eating all my sago.}\allomorph{ttemän}\allomorph{ttämän}\allomorph{ttemänmäll}\allomorph{ttamänmäll}\allomorph{ttem}\allomorph{ttam}}
\entry{ttambällag}{\headword{ttambällag}\pos{n.}\sensenumber{2}\definition{the second stage of coconut growth in which it is planted}\example{Ngäna ngämo nge ttambällag de tätäm dibeny.}{I planted my ttambällag coconut yesterday.}}
\entry{ttanttem}{\headword{ttanttem}\pos{S vi.}\sensenumber{2}\definition{to be confused}\example{Obo eka me ubi gonttemnegän.}{They were confused by his story.}\allomorph{nttem}}
\entry{ttang}{\headword{ttang}\pos{n.}\sensenumber{1}\definition{hand}\example{Kaemne de ttang alle mällam a mudan.}{Don't hold bees with your hands.}\sensenumber{2}\definition{arm}\example{Kottllam obo ttang a wa ttäle da tubutubu dageyo.}{The turtle's arms and legs are short.}\sensenumber{2}\definition{palm}\sensenumber{1}\definition{elbow}\sensenumber{2}\definition{seven (lit. elbow; body counting numeral)}\sensenumber{2}\definition{on all fours}\sensenumber{2}\definition{talon, claw}\sensenumber{1}\definition{finger}\example{Tupi ttang llɨpɨt de pop me nowanseg.}{Put your long finger in the hole.}\sensenumber{2}\definition{unit of five (one hand)}\example{Obo mälla da apte ttang lläpät da.}{He has five wives.}\example{Ngämo ttongo ttang lläpät a muminy dan.}{I owe five kina.}\example{Ngämo kollba mu da komlla ttang lläpät dag.}{The price of my fish is 10 kina.}\sensenumber{2}\definition{palm}\sensenumber{2}\definition{upper part of foreleg (of a reptile)}\sensenumber{2}\definition{bracelet}\sensenumber{2}\definition{to clap}\example{Bogo ttang pllallem allan.}{She is clapping.}\sensenumber{2}\definition{to shake hands with}\example{Nagnag a ngämim ttang deyanttepmenyeyo.}{Friends shook hands with us (excl.).}\sensenumber{2}\definition{with one's hands full}\example{Ngäna ttangttang agan.}{My hands are full.}\subentry{\headword{ttang koll}\pos{n.}\definition{palm}}\subentry{\headword{ttang kum}\pos{n.}\definition{elbow}}\subentry{\headword{ttangkumttangkum}\pos{adv.}\definition{on all fours}}\subentry{\headword{ttang lläbe}\pos{n.}\definition{talon, claw}}\subentry{\headword{ttang lläpät}\pos{n.}\definition{finger}}\subentry{\headword{ttang päk}\pos{n.}\definition{palm}}\subentry{\headword{ttang pe}\pos{n.}\definition{upper part of foreleg (of a reptile)}}\subentry{\headword{ttang pitt}\pos{n.}\definition{bracelet}}\subentry{\headword{ttang pllallem}\pos{A vi.}\definition{to clap}}\subentry{\headword{ttang ttaempäg}\pos{S vt.}\definition{to shake hands with}}\subentry{\headword{ttangttang2}\pos{adv.}\definition{with one's hands full}}}
\entry{ttang käpän}{\headword{ttang käpän}\pos{v.}\sensenumber{2}\definition{to snap one's fingers}}
\entry{ttang tupiang sod}{\headword{ttang tupiang sod}\pos{n.}\sensenumber{2}\definition{long-sleeved shirt}}
\entry{ttangkuttangkumang}{\headword{ttangkuttangkumang}\pos{n.}\sensenumber{2}\definition{sideways chevron weaving pattern}}
\entry{ttangttang1}{\headword{ttangttang1}\pos{n.}\sensenumber{2}\definition{type of bird}}
\entry{ttape}{\headword{ttape}\pos{n.}\sensenumber{2}\definition{type of small, flat fish}}
\entry{ttapeyam}{\headword{ttapeyam}\pos{S vt.}\sensenumber{2}\definition{to open (something long, e.g. door, book)}\allomorph{ttapeyaem}\allomorph{ttepeyam}\allomorph{ttepeyaem}}
\entry{ttapeyamttapeyam}{\headword{ttapeyamttapeyam}\pos{adv.}\sensenumber{2}\definition{walking with one's legs spread far apart}}
\entry{ttaputtapung}{\headword{ttaputtapung}\pos{mod.}\sensenumber{2}\definition{joined together}}
\entry{ttatt}{\headword{ttatt}\pos{n.}\sensenumber{2}\definition{jaw, chin}\example{Simell ttatt a ulle dan.}{A pig's jaw is big.}\sensenumber{2}\definition{chubby cheeks}\sensenumber{2}\definition{beard}\sensenumber{1}\definition{mandible, jawbone}\sensenumber{2}\definition{drum rimLlo tep alle gazibra ttoe de doddgollallo, dape de kakäne damättallo. (Attach the gazibra snakeskin with tree sap; put the drum head over it.}\subentry{\headword{ttatt käpkäp}\pos{n.}\definition{chubby cheeks}}\subentry{\headword{ttatt kom}\pos{n.}\definition{beard}}\subentry{\headword{ttatt kutt}\pos{n.}\definition{mandible, jawbone}}}
\entry{ttatta1}{\headword{ttatta1}\pos{S vt.}\sensenumber{2}\definition{to chop a tree}\example{Bogo llo de ttanen anggan.}{He's chopping a tree.}\allomorph{tta}}
\entry{ttatta2}{\headword{ttatta2}\pos{n.}\sensenumber{2}\definition{lower back}\example{Obo ttatta pallall e deyazuwän.}{He shot towards his lower back.}}
\entry{ttattang}{\headword{ttattang}\pos{n.}\sensenumber{2}\definition{type of big, round yam with a white interior}}
\entry{ttattel}{\headword{ttattel}\pos{n.}\sensenumber{2}\definition{type of thorny vine}}
\entry{ttattep}{\headword{ttattep}\pos{n.}\sensenumber{2}\definition{mature leaf}}
\entry{ttattle}{\headword{ttattle}\pos{n.}\sensenumber{2}\definition{pain}\example{Bogo ttattle de umllang gogän.}{He felt pain.}\sensenumber{2}\definition{sore, in pain; sick, ill}\etymology{ttattle + =ang}\subentry{\headword{ttattleang}\pos{mod.}\definition{sore, in pain; sick, ill}}}
\entry{ttattlong}{\headword{ttattlong}\variant{fast speech var. of}{ttattleang}}
\entry{ttattlläb}{\headword{ttattlläb}\pos{S vi.}\sensenumber{2}\definition{to open}\example{Wäd a wättallban.}{The door opened.}\allomorph{ttallb}}
\entry{ttattlle}{\headword{ttattlle}\variant{var. of}{ttattle}}
\entry{ttattllong}{\headword{ttattllong}\variant{fast speech var. of}{ttattleang}}
\entry{ttäb}{\headword{ttäb}\pos{n.}\sensenumber{2}\definition{Pinon's imperial pigeonLlo toko me giddollag ddäddägma pa dan. (It's an edible bird that lives in treetops.)}}
\entry{ttäbattäbe}{\headword{ttäbattäbe}\pos{S vt.}\sensenumber{2}\definition{to block}\example{Ttäbanen eran, ede ge nyongo da gagäll allan.}{It's blocked, so this road is no good.}\allomorph{ttäba}\allomorph{ttba}}
\entry{ttäbe}{\headword{ttäbe}\pos{n.}\sensenumber{2}\definition{a strong smelling plant whose bark, called ttäbe kollop, was traditionally worn around the neck to give fragrance and perfume smell. It was sometimes chewed and rubbed around the body and head to stop headache. The orignal purpose was also for protection from evil spirits.}}
\entry{Ttäbe Ttäbe}{\headword{Ttäbe Ttäbe}\pos{pn.}\sensenumber{2}\definition{Ttäbe Ttäbe (toponym)}}
\entry{ttäbottäbo}{\headword{ttäbottäbo}\pos{n.}\sensenumber{2}\definition{rectum}}
\entry{ttägäll}{\headword{ttägäll}\pos{n.}\sensenumber{1}\definition{termite mound, anthill (made by termites; ants may also live inside)}\sensenumber{2}\definition{mumu (oven made in the ground with fire, stones, leaves, and bark; the first mumus were made out of termite mounds)}\example{Bogo sɨmell sisiang de däbäddän ttägäll daogän.}{She killed the tame pig and made an earth oven.}\sensenumber{1}\definition{stone (esp. one used for cooking in mumus)}\example{Dämetteya ttägäll käp de.}{We placed the oven stone.}\example{Ngäna kumuddäga ttägäll käp tärpitärpi de nabllenegan.}{I found three smooth stones.}\sensenumber{2}\definition{money}\example{ttägäll käp mit me gäddgädd}{to fight about money}\subentry{\headword{ttägäll käp}\pos{n.}\definition{stone (esp. one used for cooking in mumus)}}}
\entry{Ttägällag kona}{\headword{Ttägällag kona}\pos{pn.}\sensenumber{2}\definition{Ttägälläg corner}}
\entry{Ttägällag pollon}{\headword{Ttägällag pollon}\pos{pn.}\sensenumber{2}\definition{Ttägällag pollon (toponym)}}
\entry{ttäk}{\headword{ttäk}\pos{n.}\sensenumber{1}\definition{type of tree that grows on floating grass (\textbackslashtextasciitilde3 m) with white or yellow flowers, green fruit, and a big trunk used to by children to float or for carfts}\sensenumber{2}\definition{soft wood}}
\entry{ttäkam}{\headword{ttäkam}\pos{S vt.}\sensenumber{2}\definition{to break, snap}\example{Obo ttɨle di dattkamän.}{She broke her leg.}\example{Ubi ngäma sabi de ttäkam erallo.}{They are breaking our (excl.) law.}\allomorph{ttkaem}\allomorph{ttkam}\allomorph{ttk}\allomorph{ttäk}}
\entry{ttäkattäke}{\headword{ttäkattäke}\pos{S vt.}\sensenumber{2}\definition{to fold}\allomorph{ttke}\allomorph{ttka}}
\entry{ttäkäll}{\headword{ttäkäll}\pos{n.}\sensenumber{2}\definition{portion of yams mixed with coconutMätta nge peyang kul att erem de dom erallo. (Yam smashed with coconut, which is shaped by squeezing one's hands.)}}
\entry{ttäkoe}{\headword{ttäkoe}\pos{S vt.}\sensenumber{2}\definition{to chop, cut down, mow; shave}\example{Ngäna pos de ttokoenen de dängkamneg, tämamae naen dattkoeneg.}{I started to cut the posts; I chopped all nine.}\example{Dade towall de battkoe.}{Maybe I'll mow the grass.}\example{Ngäna obo ttatt kom de nattkoeyan.}{I shaved his beard.}\allomorph{ttkoe}\allomorph{ttko}}
\entry{ttäle1}{\headword{ttäle1}\pos{n.}\sensenumber{1}\definition{leg}\example{Ngämo ttäle da kädkädag agallo.}{My legs got cold.}\sensenumber{2}\definition{tendril}\sensenumber{2}\definition{instep}\sensenumber{2}\definition{heel}\sensenumber{2}\definition{claw}\sensenumber{2}\definition{hind leg, hind limb}\sensenumber{2}\definition{tibia}\sensenumber{2}\definition{hind foot (of a reptile) with four toes}\sensenumber{2}\definition{leg band}\subentry{\headword{ttäle koll}\pos{n.}\definition{instep}}\subentry{\headword{ttäle kum}\pos{n.}\definition{heel}}\subentry{\headword{ttäle lläpät}\pos{n.}\definition{claw}}\subentry{\headword{ttäle pätt}\pos{n.}\definition{hind leg, hind limb}}\subentry{\headword{ttäle pättkäp}\pos{n.}\definition{tibia}}\subentry{\headword{ttäle pe}\pos{n.}\definition{hind foot (of a reptile) with four toes}}\subentry{\headword{ttäle pitt}\pos{n.}\definition{leg band}}}
\entry{ttäle2}{\headword{ttäle2}\pos{n.}\sensenumber{2}\definition{type of tree}}
\entry{Ttäle Bun}{\headword{Ttäle Bun}\pos{pn.}\sensenumber{2}\definition{Ttäle Bun (toponym)}}
\entry{Ttäle mitt}{\headword{Ttäle mitt}\pos{pn.}\sensenumber{2}\definition{place with a well in Limol}}
\entry{Ttälebun}{\headword{Ttälebun}\pos{pn.}\sensenumber{2}\definition{Ttalebun (toponym)}}
\entry{ttällam}{\headword{ttällam}\pos{S vt.}\sensenumber{1}\definition{to pass, hand}\example{Ngäna bäne pate neil de bänttllam.}{I will pass you the nail.}\sensenumber{2}\definition{to extend, stretch out, reach out, put out}\example{Bäne ttang de ttättlemae nänttllam.}{Stretch your arm out straight.}\allomorph{nttllam}\allomorph{nttllaem}\allomorph{ttällaem}}
\entry{ttälläp}{\headword{ttälläp}\pos{n.}\sensenumber{2}\definition{type of venomous snakeDdobae lla koenmällang gullem tupi dan. (It's a long snake that follows people a lot.)}}
\entry{ttälle}{\headword{ttälle}\variant{dial. var. of}{ttäle1}}
\entry{ttällma tränymägäll}{\headword{ttällma tränymägäll}\pos{n.}\sensenumber{2}\definition{tree type}}
\entry{ttäm1}{\headword{ttäm1}\pos{S vt.}\sensenumber{2}\definition{to burn; heat on a fire}\example{Yu da obom dättämän.}{The fire burned him.}\example{Därunggu de yu mi wandawandae dättämalle.}{The bamboo tube is heated rotating on the fire.}}
\entry{ttäm2}{\headword{ttäm2}\variant{fast speech var. of}{ttängäm}}
\entry{ttämattäme}{\headword{ttämattäme}\pos{S vi.}\sensenumber{2}\definition{to be crowded}\example{Ngättma da lla walle gonttämawän.}{The place was crowded with people.}\allomorph{nttäma}}
\entry{ttämbe}{\headword{ttämbe}\pos{n.}\sensenumber{2}\definition{type of big tree that grows in the bush with blue, purple, and white flowers and red fruit}}
\entry{ttämbe role}{\headword{ttämbe role}\pos{n.}\sensenumber{2}\definition{type of tree}}
\entry{ttän}{\headword{ttän}\pos{n.}\sensenumber{2}\definition{type of tree that grows in the grassland (\textbackslashtextasciitilde60 m) with yellow flowers, small green fruit, and wood used for house postsUtt de otnan ma dan a ttoe de kängkäm ma dan nane we källa ine täräp me. (The shoot is eaten and the bark is squeezed to be drunk when with diarrhea.)}}
\entry{ttän maigag}{\headword{ttän maigag}\pos{n.}\sensenumber{2}\definition{type of bandicoot}}
\entry{ttänttäm}{\headword{ttänttäm}\pos{n.}\sensenumber{1}\definition{heat}\example{Däbe lla da obo labalaba de dängkänän ttänttäm atta.}{That man took of his lap-lap because of the heat.}\sensenumber{2}\definition{to burn}\example{Ikrol era yu ttänttämatt me pänpän dan.}{Ash is dust left after a fire burns.}\sensenumber{2}\definition{hot}\example{Ngämo moko da ttänttämang kollba ddäddäg e dan.}{I like to eat hot fish.}\allomorph{nttäm}\subentry{\headword{ttänttämang}\pos{mod.}\definition{hot}}}
\entry{ttängattänge}{\headword{ttängattänge}\pos{S vt.}\sensenumber{1}\definition{to read, recite}\example{Ddob llaeyaba eka walle darbnen att eka da, ngäna mɨnyi bangttängeneg.}{Stories written in other people's languages, I will read them.}\sensenumber{2}\definition{date}\example{Sisri ngasekäma ttängattänge da endan?}{What's the date today?}\allomorph{ttänganen}\allomorph{ngttänge}\allomorph{ttänga}\allomorph{nttänge}\allomorph{ngättangätte}}
\entry{ttängäm}{\headword{ttängäm}\pos{n.}\sensenumber{1}\definition{village}\example{Obo ttängäm de enanae dowansegän.}{He left his village for good.}\sensenumber{2}\definition{garden}\example{Ubi ttängäm kuddäll me polle de kame däkätteyo.}{They built a new fence around the abandoned garden.}\sensenumber{3}\definition{place}\example{Ero bäne zegatt ttängäm a?}{Where is your birthplace?}\sensenumber{3}\definition{small garden}\subentry{\headword{ttängämttängäm}\pos{n.}\definition{small garden}}}
\entry{ttängkag}{\headword{ttängkag}\pos{S vt.}\sensenumber{3}\definition{to break with force}\example{Pakos ulle walle ngäna ddia bo ddäg kutt deyanttäkeg.}{I broke the deer's backbone using a big spear.}\allomorph{nttäkeg}}
\entry{ttängkamäll}{\headword{ttängkamäll}\pos{S vt.}\sensenumber{3}\definition{to meet, reach}\example{Ubi dinduag a duduli Awayang bom danttäkämälleyo.}{Running that way, they reached Awayang.}\allomorph{nttkamäll}\allomorph{nttäkämäll}}
\entry{ttäpen}{\headword{ttäpen}\pos{S vi.}\sensenumber{1}\definition{to snap, break}\example{Kwalde mäse pakos de gonttalägän sämell zuwoe e, be bägäl wadär a gottäpenän.}{Kwalde tried to aim the arrow to shoot the pig, but the bow string snapped.}\sensenumber{2}\definition{to break, tear (esp. something long)}\example{Ngäna ngämo bunkom de nättpenan.}{I broke a strand of my hair.}\example{Obo tudi di kollba da dättpenän.}{The fish broke her fishing line.}\allomorph{ttpen}\allomorph{ttäp}\allomorph{ttpe}\allomorph{ttäpan}}
\entry{ttätt}{\headword{ttätt}\pos{mod.}\sensenumber{2}\definition{right}\example{Ge ngämo ttätt ttang dan.}{This is my right hand.}}
\entry{ttätt käp}{\headword{ttätt käp}\pos{n.}\sensenumber{1}\definition{graceful honeyeaterLlo popo naneang dan, kälsre dan. (It drinks flowers; it's small.)}\sensenumber{2}\definition{puff-backed honeyeater}}
\entry{ttättawe}{\headword{ttättawe}\pos{n.}\sensenumber{2}\definition{type of tree}}
\entry{ttättäle}{\headword{ttättäle}\variant{var. of}{ttättle}}
\entry{ttättälläg}{\headword{ttättälläg}\pos{S vt.}\sensenumber{2}\definition{to tear, tear up}\example{Gudne iddpo ttättllägatt pop de mäzi abal ulle ubemang bägagän.}{It will make the torn holes in the old clothing even wider.}\allomorph{ttlläg}}
\entry{ttättäp}{\headword{ttättäp}\pos{n.}\sensenumber{2}\definition{young leaf}}
\entry{ttättle}{\headword{ttättle}\pos{mod.}\sensenumber{1}\definition{correct, proper}\example{Ddob eka kutt panypeny a sisor llɨg aba ddone ttättle dag.}{Some of the words of young people are not pronounced correctly.}\sensenumber{2}\definition{straight}\example{Malläm me nyongo da ddone ttättle dan.}{The road to Malam is not straight.}\example{Ubi deyareyo ttättlemae do up wo kakab a erame daeya.}{They went straight to where the leftover ripe bananas were.}}
\entry{ttätto}{\headword{ttätto}\variant{sp. var. of}{ttotto}}
\entry{ttek}{\headword{ttek}\pos{n.}\sensenumber{2}\definition{type of tree that grows in the grassland with white flowers and sap used as glue for spears}}
\entry{ttette}{\headword{ttette}\pos{n.}\sensenumber{2}\definition{rafter}}
\entry{ttimattima}{\headword{ttimattima}\pos{n.}\sensenumber{2}\definition{limp}\example{Bogo ttimattimang ibiag daeya.}{He walked with a limp.}\allomorph{ttima}}
\entry{ttɨle}{\headword{ttɨle}\variant{dial. var. of}{ttäle1}}
\entry{ttɨp}{\headword{ttɨp}\pos{n.}\sensenumber{1}\definition{type of sagoUlle sana dan, obo koll a mamam dan. Täkällang dan. (It's a big sago; its core is red. It's thorny.)}\sensenumber{2}\definition{type of yam with a white interior}}
\entry{ttoa}{\headword{ttoa}\variant{sp. var. of}{ttowa}}
\entry{ttoe}{\headword{ttoe}\pos{n.}\sensenumber{1}\definition{skin (of a person or animal)}\example{Däbe käza ttoe de Sowati soltang dägagän.}{Sowati salted that crocodile skin.}\sensenumber{2}\definition{bark}\sensenumber{3}\definition{to cover with skin}\example{Alläp de era gazibra ttoe alle ttoe amallo.}{Drums are covered with gazibra snakeskin.}}
\entry{ttoen}{\headword{ttoen}\pos{n.}\sensenumber{1}\definition{story}\example{Ngämlle mer dan ttoenttoen därmall a.}{I like to listen to stories.}\sensenumber{2}\definition{thing}\example{Oba tämamae ttoen bällam ngasnen a eragwaeya llame dagwaeya.}{They did everything together.}\sensenumber{3}\definition{way, method}\example{Yu ikllo da ttongo eka ttoen ngangema dan.}{Smoke is one way of communicating.}\sensenumber{3}\definition{small story}\example{Känazbag, ngäna mɨnyi bibra sisor ttoenttoen de bɨllɨt.}{Tomorrow, I will tell you all a new short story.}\subentry{\headword{ttoenttoen}\pos{n.}\definition{small story}}}
\entry{ttoengg}{\headword{ttoengg}\pos{S vi.}\sensenumber{3}\definition{to split up, scatter}\example{Ibi naeka peyang ttoengeny e dag.}{We will split up in tears.}\allomorph{ttoengeny}\allomorph{nttoengg}\allomorph{nttoengeny}}
\entry{ttoenglla}{\headword{ttoenglla}\pos{n.}\sensenumber{3}\definition{unrelated to one's clan or tribeMälla sinenang. (Exchanging sisters.)}}
\entry{ttoep1}{\headword{ttoep1}\pos{n.}\sensenumber{3}\definition{type of snakeTtongo bätbät gullem ulle da. Ddäddäg ma da. (It's a big, black snake. It's edible.)}}
\entry{ttoep2}{\headword{ttoep2}\pos{n.}\sensenumber{3}\definition{type of tree}}
\entry{ttoettoe}{\headword{ttoettoe}\pos{n.}\sensenumber{3}\definition{blue-faced honeyeaterGul me papällägag pa dag. (It's a bird that flies in flocks.)}}
\entry{ttogottogo}{\headword{ttogottogo}\pos{A vi.}\sensenumber{3}\definition{to become blunt}}
\entry{ttokoe}{\headword{ttokoe}\variant{dial. var. of}{ttäkoe}}
\entry{ttomoll}{\headword{ttomoll}\pos{mod.}\sensenumber{3}\definition{deaf}\example{Obo llan a ttomoll dageyo.}{His ears are deaf.}}
\entry{ttomttom}{\headword{ttomttom}\pos{n.}\sensenumber{1}\definition{yam heap}\sensenumber{2}\definition{yam cooked whole with skin}}
\entry{ttongda}{\headword{ttongda}\variant{fast speech var. of}{ttongdae}}
\entry{ttongg}{\headword{ttongg}\pos{S vd.}\sensenumber{2}\definition{to give}\example{Ngämlle sawe alle giri de nanttog.}{Give me the knife with your left hand.}\sensenumber{2}\definition{selfish, greedy}\sensenumber{2}\definition{generous, giving}\sensenumber{2}\definition{generous, giving}\allomorph{nttog}\allomorph{nttongg}\allomorph{ntto}\allomorph{tto}\allomorph{si}\allomorph{se}\etymology{ttongg + =meny}\subentry{\headword{ttonggmeny}\pos{mod.}\definition{selfish, greedy}}\subentry{\headword{sinensinen}\pos{mod.}\definition{generous, giving}}\subentry{\headword{ttonggttongg}\pos{mod.}\definition{generous, giving}}}
\entry{ttongo1}{\headword{ttongo1}\pos{num.}\sensenumber{1}\definition{one}\example{Ngämlle ttongo nanttog, a bäne ttongo.}{Give me one, and one for you.}\sensenumber{2}\definition{a/an}\example{Abo bongo ttongo ma sisor de nongo.}{You must build a new house.}\sensenumber{3}\definition{another}\example{Ttongo lla da agbänmällnan a ttongo lla da nindugan obo ingoll me.}{One man was jumping and another man ran in front of him.}\sensenumber{4}\definition{next}\example{Ttongo ebdo me ako llo täräpnan e dalle.}{The next morning, I went to cut trees.}\sensenumber{5}\definition{unique}\example{Bongo imomdae ttongo dan.}{You are truly unique.}\sensenumber{5}\definition{someone; anyone}\example{Ttongo lla da ddone aya bäne peyang ine ma ibi allan?}{Is anyone going with you to get water?}\sensenumber{5}\definition{one (yam counting numeral; also general)}\sensenumber{1}\definition{each, one by one}\example{Tämamae mänmän a llɨg di ttongalle ttongalle ume nɨddɨlaeballo.}{Every girl kissed each boy.}\example{Wayati de dikom llayabira dänye ttongottongo alle.}{I brought the watermelons and gave them out to people, one by one.}\sensenumber{2}\definition{few}\example{Ddob ttongottongowalle lla da dade ami ddob ttängäm att a guddällmamän gänyme.}{There are a few others who arrived here from other villages.}\allomorph{ttong}\etymology{from ttongo + =dae₁}\subentry{\headword{ttongo lla}\pos{pers. pron.}\definition{someone; anyone}}\subentry{\headword{ttongdae}\pos{num.}\definition{one (yam counting numeral; also general)}}\subentry{\headword{ttongottongo alle}\pos{adv.}\definition{each, one by one}}}
\entry{ttongo2}{\headword{ttongo2}\pos{n.}\sensenumber{2}\definition{drum handleLla bo ttang alläp mällam ma. (For a person to carry a drum.)}}
\entry{ttongo iddob}{\headword{ttongo iddob}\pos{n.}\sensenumber{2}\definition{the day after tomorrow}}
\entry{ttongoalle ttongoalle}{\headword{ttongoalle ttongoalle}\variant{var. of}{ttongottongo alle}}
\entry{ttongottongalle}{\headword{ttongottongalle}\variant{fast speech var. of}{ttongottongo alle}}
\entry{ttongottongowalle}{\headword{ttongottongowalle}\variant{sp. var. of}{ttongottongo alle}}
\entry{ttongttong}{\headword{ttongttong}\pos{n.}\sensenumber{2}\definition{type of tree}}
\entry{ttope}{\headword{ttope}\pos{n.}\sensenumber{2}\definition{reedNyäng inen ma za dan, walle me päddabag dan. (It's for weaving bags; it grows in the water.)}}
\entry{ttotto}{\headword{ttotto}\pos{S vt.}\sensenumber{1}\definition{to tie}\example{Ngäna ngämo bägäl de ttotto dägneg toboll peyang.}{I tied my bow and arrows together.}\sensenumber{2}\definition{to collect}\example{Bogo ekaklle me gontmonän up gälbänanatt ttonen e.}{He waited on the ground to collect the fallen bananas.}\allomorph{tto}}
\entry{ttottoem}{\headword{ttottoem}\pos{n.}\sensenumber{2}\definition{type of tree that grows inthe swamp and along creeks with a mango-like, inedible fruit that is yellow when ripe}}
\entry{ttowa}{\headword{ttowa}\pos{n.}\sensenumber{2}\definition{Pacific koel}}
\entry{ttowaemang}{\headword{ttowaemang}\pos{mod.}\sensenumber{2}\definition{miraculous}\sensenumber{2}\definition{miracle}\example{Ewatta ke gänya täräp me lla bombllo da wasnen anggan ttowamang ttoen tonton ikop e ddapall att?}{Why does this generation beg to see miracles directly from heaven?}\subentry{\headword{ttowaemang ttoen}\pos{n.}\definition{miracle}}}
\entry{ttowamang}{\headword{ttowamang}\variant{fast speech var. of}{ttowaemang}}
\entry{ttu}{\headword{ttu}\pos{n.}\sensenumber{2}\definition{deep}\example{Angde obom ttu wi dandämoeyän bogo ddone dängllawän, be gonddäwän.}{When she pushed him into the deep, he did not swim, but drowned.}}
\entry{ttullong}{\headword{ttullong}\pos{n.}\sensenumber{2}\definition{large-tailed nightjarIddob me ekawang pa dan. (It's a bird that sings at night.)}}
\entry{ttupe}{\headword{ttupe}\pos{mod.}\sensenumber{2}\definition{bad (of a coconut)}}
\lettersection{U}
\entry{Uba}{\headword{Uba}\pos{pn.}\sensenumber{2}\definition{female personal name}}
\entry{Ubäd}{\headword{Ubäd}\pos{pn.}\sensenumber{2}\definition{male personal name}}
\entry{ubemang}{\headword{ubemang}\pos{mod.}\sensenumber{2}\definition{wide}\example{Däbe abo obo pallall ngänttäg eka da ulle ubemang gogon.}{That story about his arrival spread wide and far.}}
\entry{ubi}{\headword{ubi}\pos{pers. pron.}\sensenumber{2}\definition{they (third person nonsingular pronoun, nominative form)}\sensenumber{2}\definition{clitic form of ubi}\example{Nensi bi nane alle känaebag mɨnyi tudi ma beyareyo.}{Nancy will go fishing tomorrow with her aunt.}\example{Medäda bi ddone kili dägaebeyo.}{Their fathers were not happy.}\sensenumber{2}\definition{ablative-possessive form of ubi}\example{Llɨg obaene longgo de dandär.}{I heard noise from the children.}\sensenumber{2}\definition{clitic form of obaene}\example{llayabaene tot}{people's trash}\sensenumber{2}\definition{possessive form of ubi}\example{Oba mokwang tongoe a togotogol daeya.}{Their favorite game was hide-and-seek.}\sensenumber{2}\definition{themselves (reflexive form of ubi)}\sensenumber{2}\definition{clitic form of oba}\example{Grace a Kate dag mänmän aba peyang.}{Grace and Kate are with the girls.}\example{Ende llayaba ttoenttoen dan.}{This is the Ende people's story.}\example{Do gabmaeyaba ttängäm me dan.}{She is there, where the white people are.}\example{Ngämo mang ba bin a ada, Eric, Francis, Bewag.}{These are my brothers' names: Eric, Francis, and Bewag.}\sensenumber{1}\definition{each other (reciprocal pronoun)}\example{Ubi obaoba tatu anggan.}{They are washing each other.}\example{Llɨg a wa män a ubi obaoba umllang dagwaeya.}{The boy and the girl knew each other.}\sensenumber{2}\definition{themselves (reflexive form of ubi)}\example{Ubi obaoba tatu anggan.}{They are washing themselves.}\example{Llɨg a obaoba gognegän.}{The boys were by themselves.}\sensenumber{2}\definition{accusative form of ubi}\example{Ubim mɨnyi umllang bägaebeya.}{We will inform them.}\sensenumber{2}\definition{clitic form of ubim}\example{Ngäna ngämo masar bim ddone ikop dägnegne.}{I did not see my grandparents.}\sensenumber{2}\definition{dative form of ubi}\example{Bogo ddäddäg de dokonegän ubira.}{He cut the meat for them.}\sensenumber{2}\definition{clitic form of ubira}\example{Bogo llɨg kälekäle abira bandra llɨtɨt e gongkamalle.}{She started to sing to the little kids.}\subentry{\headword{=bi}\pos{pron. cl.}\definition{clitic form of ubi}}\subentry{\headword{obaene}\pos{pers. pron.}\definition{ablative-possessive form of ubi}}\subentry{\headword{=abaene}\pos{pron. cl.}\definition{clitic form of obaene}}\subentry{\headword{oba1}\pos{pers. pron.}\definition{possessive form of ubi}}\subentry{\headword{obazaga}\pos{pers. pron.}\definition{themselves (reflexive form of ubi)}}\subentry{\headword{=aba}\pos{pron. cl.}\definition{clitic form of oba}}\subentry{\headword{obaoba}\pos{pers. pron.}\definition{each other (reciprocal pronoun)}}\subentry{\headword{ubim}\pos{pers. pron.}\definition{accusative form of ubi}}\subentry{\headword{=bim}\pos{pron. cl.}\definition{clitic form of ubim}}\subentry{\headword{ubira}\pos{pers. pron.}\definition{dative form of ubi}}\subentry{\headword{=abira}\pos{pron. cl.}\definition{clitic form of ubira}}}
\entry{ubony}{\headword{ubony}\pos{n.}\sensenumber{2}\definition{type of black beeLla ddäddägag dan, mɨnyi itrell bogon, ddobae ttällanenang dan. Gul mi ttägäll ingoll ma me giddollnenang dag. (It bites people; one will become ill; it's very painful. They live in groups in big nests.)}}
\entry{Ubrag}{\headword{Ubrag}\pos{pn.}\sensenumber{2}\definition{male personal name}}
\entry{ubrattäka}{\headword{ubrattäka}\pos{n.}\sensenumber{2}\definition{type of yam a red interior}}
\entry{ud}{\headword{ud}\pos{n.}\sensenumber{2}\definition{door; gate}\example{Ud a papekang dan.}{The door is closed.}}
\entry{udab}{\headword{udab}\pos{S vi.}\sensenumber{1}\definition{to disappear, get lost, go missing}\example{Ngäna udab allan.}{I am getting lost.}\example{Angde ubi gongoseyo, ubi ikop dägageyo mamam mätta da Meri erem de dibeyän audaban.}{When they came back, they saw that the red yam that Mary had planted was gone.}\sensenumber{2}\definition{to lose}\example{Ubi obom daudabeyo.}{They lost him.}\allomorph{udaeb}\allomorph{ud}}
\entry{udaude}{\headword{udaude}\pos{S vt.}\sensenumber{2}\definition{to light, start (a fire)}\example{Ubi yu de daudeaemeyo.}{Then started some fires.}\allomorph{ude}\allomorph{ode}\allomorph{uda}\allomorph{ud}}
\entry{udu}{\headword{udu}\pos{n.}\sensenumber{2}\definition{walking stick, cane, staff}\example{Bogo udu peyang ibiag dan.}{She walks with a cane.}}
\entry{ugeuge}{\headword{ugeuge}\pos{n.}\sensenumber{2}\definition{type of tree}}
\entry{ugri}{\headword{ugri}\pos{n.}\sensenumber{2}\definition{fever}}
\entry{ugug}{\headword{ugug}\pos{S vt.}\sensenumber{2}\definition{to make mumu}\example{Ngämi ttägäll daugeya.}{We made mumu.}\allomorph{ug}\allomorph{ugnen}}
\entry{uk}{\headword{uk}\pos{A vi.}\sensenumber{2}\definition{to shout, yell, cry out}\example{Ubi nga wukangae anggan.}{They are still shouting.}\example{Mälla da uk allan tätäp atta.}{The woman is crying out in pain.}}
\entry{ukär}{\headword{ukär}\pos{n.}\sensenumber{1}\definition{glossy-mantled manucode}\sensenumber{2}\definition{trumpet manucode}}
\entry{ulle}{\headword{ulle}\pos{mod.}\sensenumber{1}\definition{big, large, great}\example{Yu ttänttämatt a ai dan ute ulle we bogon.}{Burns can become big sores.}\sensenumber{2}\definition{important, great}\example{Tämamae Adi bo gwell makäp me, eran ke ngattong abal e ulle abal gwell a?}{Within all of God's commandments, which commandment is the most important?}\sensenumber{3}\definition{long; tall}\example{pinggudd ulle}{long skirt}\sensenumber{4}\definition{a lot, abundant, plentiful (for uncountable nouns)}\example{Ine da ulle dan.}{There is plenty of water.}\example{Eso ulle.}{Many thanks.}\sensenumber{5}\definition{entire, whole}\example{ebdo ulle, iddob ulle}{all day, all night}\sensenumber{6}\definition{to become big, grow up}\example{Ngäna gänyme ulle gog.}{I grew up here.}\sensenumber{7}\definition{to raise, rear}\example{Aeya bam ulle dagän?}{Who raised you?}\sensenumber{7}\definition{master, owner, ruler, important person}\example{Da ttongo bina makäp me ulle binang e moko bogon, ede ai dan bogo melemang bogon.}{If one of you wants to be a ruler, that person can be a servant instead.}\sensenumber{7}\definition{nonsingular form of ulle binang}\sensenumber{7}\definition{nonsingular form of ulle}\example{Obo mälläng ik a ulleulle abal dagaya.}{His nostrils were very big.}\example{Ngämi ulleulle agmalla.}{We (excl.) grew up.}\etymology{bin + =ang, lit. 'with an important title'}\subentry{\headword{ulle binang}\pos{n.}\definition{master, owner, ruler, important person}}\subentry{\headword{ulleulle binang}\pos{n.}\definition{nonsingular form of ulle binang}}\subentry{\headword{ulleulle}\definition{nonsingular form of ulle}}}
\entry{ulle kottllam}{\headword{ulle kottllam}\pos{n.}\sensenumber{7}\definition{type of big turtle}}
\entry{ullegäll}{\headword{ullegäll}\pos{n.}\sensenumber{7}\definition{type of tree that grows in the grassland with white flowers, black fruits, and red nuts Amtet itrel täräp me kängkäm ma dan nane we. (When having breathing problems, it's squeezed to be drunk.)}}
\entry{ullowae}{\headword{ullowae}\pos{adv.}\sensenumber{7}\definition{fast, quickly}\example{Däräng ullowae dinduag daeya.}{The dog was running quickly.}}
\entry{ullull}{\headword{ullull}\pos{S vt.}\sensenumber{7}\definition{to cross over}\example{Ubi tätäm ttongo mo sisor de daulliyu.}{Yesterday, they crossed over a new bridge.}\allomorph{ull}}
\entry{umaem}{\headword{umaem}\pos{S vi.}\sensenumber{1}\definition{to gather}\example{Lla da gumaemalle a eka gontemenyalle täre pallall e.}{The people will gather and discuss the feast.}\sensenumber{2}\definition{to gather, collect}\example{Ngäna towall de umaem anggan komlla tum.}{I gathered the grass into two heaps.}\allomorph{maem}\allomorph{m}\allomorph{um}}
\entry{Umbuzag}{\headword{Umbuzag}\pos{pn.}\sensenumber{2}\definition{male personal name (name of the original Agob man)}}
\entry{ume ddäddäl}{\headword{ume ddäddäl}\pos{S vt.}\sensenumber{2}\definition{to kiss}\example{Bina mokowang lla da gänyan ngäna erem ume bäddäl.}{The man you all want is the one that I will kiss.}\allomorph{ddäl}}
\entry{ume ttäp}{\headword{ume ttäp}\pos{n.}\sensenumber{2}\definition{mouth}\example{Ngäna däbe giri tupi da ngämo ereya ada käza bo ume ttäp e däträkne.}{I had my long knife and pushed it into the crocodile's mouth.}}
\entry{umettäp}{\headword{umettäp}\variant{sp. var. of}{ume ttäp}}
\entry{umllang}{\headword{umllang}\pos{n.}\sensenumber{1}\definition{knowledge, knowing, awareness}\example{Bäne umllang dan ada ngämo umllang dan pittpitt a.}{You know that I know how to sew (lit. your knowledge is that sewing is my knowledge).}\sensenumber{2}\definition{to know; come to know, learn}\example{Ddone aya umllang gogon ada erowattäm da ge za da gogezän.}{No one knows from where these things came out.}\example{Ngäna mäse umllang gogne orpmang eka de.}{I tried to learn the Wipi language.}\sensenumber{3}\definition{to tell, inform}\example{Ngäna umllang däga, "Imanuel, bongo gall guwo me, ngäna walle we gäbän allan ddia llädäd e."}{I told him, "Imanuel, you stay in the canoe; I am jumping in the water to grab the deer."}\example{Ngäna balle ngämo nag bom umllang bägag bablle ngämingg e.}{I will go tell my friend to help you.}\sensenumber{3}\definition{summary}\etymology{from umull + =ang}\subentry{\headword{umllang eka}\pos{n.}\definition{summary}}}
\entry{umllang bällanen ma skul}{\headword{umllang bällanen ma skul}\pos{n.}\sensenumber{3}\definition{vocational school}\etymology{lit. 'school for finding knowledge'}}
\entry{umull}{\headword{umull}\pos{n.}\sensenumber{3}\definition{wisdom}}
\entry{unkäm}{\headword{unkäm}\variant{var. of}{wänkäm}}
\entry{up}{\headword{up}\pos{n.}\sensenumber{3}\definition{banana}\example{wup däg}{banana bunch}\example{wup käp}{banana fruit}\example{wup ttam}{banana leaf}\allomorph{up}}
\entry{upe}{\headword{upe}\pos{loc.}\sensenumber{3}\definition{outside, out}\example{Meri uziz allan upe we.}{Mary runs out.}}
\entry{upeupe}{\headword{upeupe}\pos{n.}\sensenumber{3}\definition{type of plant with a single stem and edible, tall fruit near the base of stemObo popo bo ine da llɨg kälekäle aba bod uteute nyänan ma dan. (Its flower liquid is applied to children's mouth scores.)}}
\entry{Upiara}{\headword{Upiara}\pos{pn.}\sensenumber{3}\definition{Upiara (Bitur-speaking village in Oriomo-Bituri Rural LLG; GPS: -8.547170, 142.653008)}}
\entry{upiye}{\headword{upiye}\pos{n.}\sensenumber{3}\definition{type of tree used to make kwib charcoal}}
\entry{upma}{\headword{upma}\pos{n.}\sensenumber{3}\definition{two friends who share a twin banana fruitIttma bin llayaba; mällayaba ttaem. (Ceremonial term for someone; amongst women.)}\etymology{up + =ma}}
\entry{upoupoll}{\headword{upoupoll}\pos{n.}\sensenumber{3}\definition{type of tree}}
\entry{upye}{\headword{upye}\pos{n.}\sensenumber{3}\definition{type of tree with white flowers and black fruit that produces a black pigment}}
\entry{Ur}{\headword{Ur}\pos{pn.}\sensenumber{3}\definition{Ur (toponym)}}
\entry{uriar}{\headword{uriar}\pos{n.}\sensenumber{3}\definition{type of palm with coconuts with a purple exocarp}}
\entry{Urimba}{\headword{Urimba}\pos{pn.}\sensenumber{3}\definition{Urimba (toponym)}}
\entry{Uroe}{\headword{Uroe}\pos{pn.}\sensenumber{3}\definition{Wuroi (in Oriomo-Bituri Rural LLG)}}
\entry{uta}{\headword{uta}\pos{S vi.}\sensenumber{3}\definition{go (imperative, singular form)}\sensenumber{3}\definition{plural form of uta}\example{Bibi utamom ngattong.}{You all, go first.}\subentry{\headword{utamom}\pos{S vi.}\definition{plural form of uta}}}
\entry{utale}{\headword{utale}\pos{loc.}\sensenumber{3}\definition{far}\example{Alla utale agan ma da?}{How far is the village?}\allomorph{wutale}}
\entry{utali}{\headword{utali}\variant{sp. var. of}{utale}}
\entry{ute}{\headword{ute}\pos{n.}\sensenumber{3}\definition{sore, blister; wound}\example{Yu ttänttämatt a ai dan ute ulle we bogon.}{Burns can become big sores.}\sensenumber{3}\definition{healthcare worker}\sensenumber{3}\definition{aid post}\sensenumber{3}\definition{with sores}\etymology{mäla- [mälamäle] + -nen + =ang, lit. 'wound patcher'}\subentry{\headword{ute mälanenang}\pos{n.}\definition{healthcare worker}}\subentry{\headword{uteute ma}\pos{n.}\definition{aid post}}\subentry{\headword{uteute}\pos{mod.}\definition{with sores}}}
\entry{utt1}{\headword{utt1}\pos{n.}\sensenumber{3}\definition{conch shell}\example{Da Llimoll me lla da kuddäll bogon, komlla o kumuddäga lla da mɨnyi utt alle eka de bängaeyo.}{If someone in Limol dies, two or three men will deliver the news using a conch shell.}\sensenumber{3}\definition{small conch shell}\subentry{\headword{uttutt}\pos{n.}\definition{small conch shell}}}
\entry{utt2}{\headword{utt2}\pos{n.}\sensenumber{3}\definition{shoot (of a plant)}\example{Duny wätätatt a nge wutt a gokenyamän.}{After the beetle eats the shoot, it bends.}}
\entry{uttang ttatta}{\headword{uttang ttatta}\pos{n.}\sensenumber{3}\definition{type of sagoTäkällang ulle pättang sana dan. (It's a thorny sago with a big trunk.)}\etymology{utt + =ang}}
\entry{uud}{\headword{uud}\variant{sp. var. of}{ud}}
\entry{uwo}{\headword{uwo}\pos{n.}\sensenumber{3}\definition{magnificent riflebirdWälläng me llo käp ballänggag pa dan. (It's a bird that awaits tree fruit in the bush.)}}
\entry{uwo kottllam}{\headword{uwo kottllam}\pos{n.}\sensenumber{3}\definition{type of turtle}}
\entry{Uzaba}{\headword{Uzaba}\pos{pn.}\sensenumber{3}\definition{male personal name}}
\entry{Uziag}{\headword{Uziag}\pos{pn.}\sensenumber{3}\definition{male personal name}}
\entry{uziz}{\headword{uziz}\pos{S vi.}\sensenumber{3}\definition{to run}\allomorph{izmoll}}
\lettersection{W}
\entry{wa}{\headword{wa}\pos{n.}\sensenumber{3}\definition{penis}\sensenumber{3}\definition{sperm}\subentry{\headword{wa piye}\pos{n.}\definition{sperm}}}
\entry{=wa}{\headword{=wa}\pos{n. cl.}\sensenumber{3}\definition{emphatic clitic}\example{Ngämo baba wa ngänäm dantemenyän.}{My dad, he taught me.}\allomorph{we}}
\entry{wabeb}{\headword{wabeb}\pos{S vt.}\sensenumber{3}\definition{to beat, smash, pound}\example{Mälla da biye de duwebnegän.}{The woman pounded the taro.}\example{Ttängäm mondrog lla da melemang de dällädeyo, gagagäll duwebeyo.}{The gardeners grabbed the servant and beat him badly.}\allomorph{web}}
\entry{wabeyawabeya}{\headword{wabeyawabeya}\pos{n.}\sensenumber{3}\definition{type of tree}}
\entry{wadär}{\headword{wadär}\pos{n.}\sensenumber{1}\definition{type of grass; cane}\example{Lla da da bogäddnän, mulldae dan wadär de bängällbänän, wadär alle bäbäddän.}{If a person is fighting, it is possible to get a cane and beat them with it.}\sensenumber{2}\definition{bowstring}\example{Bägäl wadär a gottäpanän.}{The bowstring snapped.}\sensenumber{2}\definition{extra bowstring}\sensenumber{2}\definition{type of game involving pulling cane}\subentry{\headword{wadär mikel}\pos{n.}\definition{extra bowstring}}\subentry{\headword{wadär nyongkoe}\pos{n.}\definition{type of game involving pulling cane}}}
\entry{wadär gullem}{\headword{wadär gullem}\pos{n.}\sensenumber{2}\definition{Papuan python}}
\entry{wadär käp}{\headword{wadär käp}\pos{n.}\sensenumber{2}\definition{type of big taro}}
\entry{Wadär Mitang}{\headword{Wadär Mitang}\pos{pn.}\sensenumber{2}\definition{Wadär Mitang (toponym)}}
\entry{Wae}{\headword{Wae}\pos{pn.}\sensenumber{2}\definition{female personal name}}
\entry{Waedar}{\headword{Waedar}\pos{pn.}\sensenumber{2}\definition{female personal name}}
\entry{Waenum}{\headword{Waenum}\pos{pn.}\sensenumber{2}\definition{female personal name}}
\entry{waeo}{\headword{waeo}\variant{sp. var. of}{waeyo}}
\entry{waetwaet}{\headword{waetwaet}\pos{n.}\sensenumber{2}\definition{type of tree}}
\entry{waewae}{\headword{waewae}\pos{n.}\sensenumber{2}\definition{song sung while beating sago}}
\entry{waeya}{\headword{waeya}\pos{n.}\sensenumber{2}\definition{wire}\example{Ngäna nyäng tär ngasnges e nyäng de napoenan waeya kälsre alle.}{I used the small wire to make holes in the bag to put the new bag string.}\etymology{from Englishwire}}
\entry{waeyo}{\headword{waeyo}\pos{interj.}\sensenumber{2}\definition{shout made in distress}}
\entry{Wagälla}{\headword{Wagälla}\pos{pn.}\sensenumber{2}\definition{Wagälla (toponym)}}
\entry{Wagiba}{\headword{Wagiba}\pos{pn.}\sensenumber{2}\definition{female personal name}}
\entry{Wagiya}{\headword{Wagiya}\pos{pn.}\sensenumber{2}\definition{personal name}}
\entry{waglla}{\headword{waglla}\pos{n.}\sensenumber{2}\definition{bullroarer}\example{waglla eka}{bullroarer signal}}
\entry{Wai}{\headword{Wai}\pos{pn.}\sensenumber{2}\definition{female personal name}}
\entry{Wainum}{\headword{Wainum}\pos{pn.}\sensenumber{2}\definition{female personal name}}
\entry{wak}{\headword{wak}\pos{n.}\sensenumber{2}\definition{Papuan pitta}}
\entry{Waka Källäm}{\headword{Waka Källäm}\pos{pn.}\sensenumber{2}\definition{Waka Pond (in Limol)}}
\entry{wakata}{\headword{wakata}\pos{n.}\sensenumber{2}\definition{type of introduced bananaMer wup dan, obo käp a kumäddäga ngallen me yu ma dag, ada o me otänan ma dag, ako o da yu mi binzenen ma dag, a kire da yu ma dag. (It's a good banana; its fruit are cooked in three ways; when ripe, they're eaten and heated on the fire, and when unripe, they are cooked.)}}
\entry{Wala}{\headword{Wala}\pos{pn.}\sensenumber{2}\definition{personal name}}
\entry{Waliyama}{\headword{Waliyama}\pos{pn.}\sensenumber{2}\definition{Wariama (in Gogodala Rural LLG)}}
\entry{walläg}{\headword{walläg}\pos{v.}\sensenumber{2}\definition{fan}\allomorph{wallgä}}
\entry{walle}{\headword{walle}\pos{n.}\sensenumber{2}\definition{body of water}\example{Män a kädkäd dag walle we mɨnyi bogbaebän.}{The girl will jump into the cold water.}\sensenumber{2}\definition{river, streamTatu ma dan a ako mer sana ibeny ma dan. (A place to bathe and also a good place to plant sago.)}\sensenumber{2}\definition{dry riverbed}\sensenumber{2}\definition{creek; river sourceTudi spallnen ma dan a bobag me ddobae mängallang dan. (It's a fishing place, and when flooded, it's very strong.)}\sensenumber{2}\definition{bank, shore, water's edge}\subentry{\headword{walle ddage}\pos{n.}\definition{river, streamTatu ma dan a ako mer sana ibeny ma dan. (A place to bathe and also a good place to plant sago.)}}\subentry{\headword{walle gutu}\pos{n.}\definition{dry riverbed}}\subentry{\headword{walle mäg}\pos{n.}\definition{creek; river sourceTudi spallnen ma dan a bobag me ddobae mängallang dan. (It's a fishing place, and when flooded, it's very strong.)}}\subentry{\headword{walle menae}\pos{n.}\definition{bank, shore, water's edge}}}
\entry{wallwall}{\headword{wallwall}\pos{v.}\sensenumber{2}\definition{yawn}}
\entry{Wama}{\headword{Wama}\pos{pn.}\sensenumber{2}\definition{female personal name}}
\entry{wamän}{\headword{wamän}\pos{S vi.}\sensenumber{2}\definition{to go out, dissipate, extinguish}\example{Yu da dowamänän.}{The fire went out.}}
\entry{Wamorong}{\headword{Wamorong}\pos{pn.}\sensenumber{2}\definition{Wamorong (toponym)}}
\entry{wan}{\headword{wan}\pos{num.}\sensenumber{2}\definition{one (English numeral)}\etymology{from Englishone}}
\entry{wan pinga}{\headword{wan pinga}\pos{n.}\sensenumber{2}\definition{metal fish spear}\etymology{from Englishone finger}}
\entry{wana}{\headword{wana}\pos{n.}\sensenumber{2}\definition{type of introduced bananaPätt tubutubu dan, obo däg a yuwog dag. Käp a obo o me otät ma dan. Be kire da yuwatt a ttoe a mänyi bonddegemenyän. Ako pätt a obo popel e kaekep ma dan ankom peyang pagag ma dan. (The trunk is long; its bunches are plentiful. When ripe, its fruit is eaten. But when unripe, the skin will get stuck after being cooked. Also, its stem is for popel; chewing with ants and salt.)}}
\entry{Wanadidi}{\headword{Wanadidi}\pos{pn.}\sensenumber{2}\definition{female personal name}}
\entry{wandana}{\headword{wandana}\pos{n.}\sensenumber{2}\definition{grass in the garden}}
\entry{wandawandae}{\headword{wandawandae}\pos{adv.}\sensenumber{2}\definition{rotating, spinning, rolling}}
\entry{Wane}{\headword{Wane}\pos{pn.}\sensenumber{2}\definition{male personal name}}
\entry{wanpadam}{\headword{wanpadam}\pos{n.}\sensenumber{2}\definition{lap-lap}}
\entry{wanseg}{\headword{wanseg}\pos{S vt.}\sensenumber{1}\definition{to put, place, set aside, leave}\example{Abo ddäddäg de nokomeyo a katrekatre toko me nowattälleyo.}{You all should take the pieces of meat and put them on the table.}\example{Pa obom dowansegän a dallän nagda bom dingällbänän a kottllam pate gongttägeyo.}{The bird left him and went to get his friend and returned to the turtle.}\example{Oba llɨg kälekäle tämamae ma me dowattälleyo.}{They left all their small children at home.}\sensenumber{2}\definition{to be left (in a state)}\example{Lama wa Mana ubi obaoba ulle abal kandärmang me gowensegäneyo.}{Lama and Mana were left feeling very sorry for themselves.}\example{Däbe källäm a yäbäd ulle atta dallän do kälae abal gowensegän ine da.}{Because of the strong sun, that pond kept getting smaller until there was very little water left.}\allomorph{wanseg}\allomorph{wattäll}\allomorph{wettäll}\allomorph{wenseg}\allomorph{ottmäll}\allomorph{wansegän}\allomorph{wensegeny}\allomorph{wanse}}
\entry{wanttawantta}{\headword{wanttawantta}\pos{n.}\sensenumber{2}\definition{type of game like capture the flag but with a stick planted in the middle of a ring instead of a flag}}
\entry{wanwen}{\headword{wanwen}\pos{S vi.}\sensenumber{1}\definition{to shake, swing}\example{Bogo bun gowenän ada, "Ddone."}{His head shook like this, "No."}\sensenumber{2}\definition{to shake, swing}\example{Zämllallang a oba bun di dowanaemneyo mikutt me.}{Passers-by shook their heads in anger.}\allomorph{wan}\allomorph{wen}\allomorph{wanen}}
\entry{wangam}{\headword{wangam}\pos{S vt.}\sensenumber{2}\definition{to forget}\example{Ddone mänyi bam bawengameya.}{We will not forget you.}\allomorph{wengam}\allomorph{wengaem}\allomorph{wangaem}\allomorph{wang}\allomorph{weng}}
\entry{wanyweny}{\headword{wanyweny}\pos{S vt.}\sensenumber{2}\definition{to burn}\example{Kullkull wanyweny e gotäbanegnän.}{They were planning to burn a grass fire.}}
\entry{waowaem}{\headword{waowaem}\variant{sp. var. of}{wawaem}}
\entry{wap}{\headword{wap}\pos{n.}\sensenumber{2}\definition{stick}\example{Ngäna wap alle pa de gäddnan dängkaemneg.}{I started killing birds with sticks.}}
\entry{wapo}{\headword{wapo}\pos{interj.}\sensenumber{2}\definition{oh no}}
\entry{Wapotea}{\headword{Wapotea}\pos{pn.}\sensenumber{2}\definition{Wapotea (toponym)}}
\entry{Wara}{\headword{Wara}\pos{pn.}\sensenumber{2}\definition{Wara (toponym)}}
\entry{Warama}{\headword{Warama}\pos{pn.}\sensenumber{2}\definition{male personal name}}
\entry{waramakae}{\headword{waramakae}\pos{num.}\sensenumber{2}\definition{7776 (yam counting numeral; 6\textbackslashtextasciicircum5)}\etymology{from a Yam language; compare Arammba wermeke}}
\entry{waramawarama}{\headword{waramawarama}\pos{n.}\sensenumber{2}\definition{type of tree that grows in the bush with white flowers and edible yellow fruit}}
\entry{Warani}{\headword{Warani}\pos{pn.}\sensenumber{2}\definition{male personal name}}
\entry{Wareka}{\headword{Wareka}\pos{pn.}\sensenumber{2}\definition{female personal name}}
\entry{Wareya}{\headword{Wareya}\pos{pn.}\sensenumber{2}\definition{male personal name}}
\entry{Wariabodolo}{\headword{Wariabodolo}\pos{pn.}\sensenumber{2}\definition{Variobadoro (in Kiwai Rural LLG)}}
\entry{Warik}{\headword{Warik}\pos{pn.}\sensenumber{2}\definition{male personal name}}
\entry{wariwari}{\headword{wariwari}\pos{n.}\sensenumber{2}\definition{sago shoot}}
\entry{waro}{\headword{waro}\pos{n.}\sensenumber{2}\definition{type of turtle}}
\entry{Warola}{\headword{Warola}\pos{pn.}\sensenumber{2}\definition{Warola (camping place in Limol)}}
\entry{Warolla}{\headword{Warolla}\variant{var. of}{Warola}}
\entry{Wasang}{\headword{Wasang}\pos{pn.}\sensenumber{2}\definition{male personal name}}
\entry{wasangwasang}{\headword{wasangwasang}\pos{adv.}\sensenumber{2}\definition{always}\example{Angde Matthew llɨg kälsre me, dibe llo de wasangwasang dängkälne.}{When Matthiew was a little boy, he was always climbing that tree.}}
\entry{wasar}{\headword{wasar}\pos{n.}\sensenumber{2}\definition{type of edible palm}}
\entry{waso}{\headword{waso}\pos{n.}\sensenumber{2}\definition{eastern cattle egretWalle menae dae zanggaeag pa, kollbamäg ddäddäg dan. (A bird that roams around the riverside, it eats fish.)}}
\entry{Wasua}{\headword{Wasua}\pos{pn.}\sensenumber{2}\definition{Wasua (in Gogodala Rural LLG)}}
\entry{Wasuwa}{\headword{Wasuwa}\variant{sp. var. of}{Wasua}}
\entry{waswes}{\headword{waswes}\pos{S vt.}\sensenumber{1}\definition{to ask, beg}\example{Da ngämi e we bam baweseya, ngäma moko da bongo nangesneg ngämira.}{Whatever we (excl.) ask you for, we want you to do it for us.}\example{Ge ngäna ewe de dawes ikopse alle.}{This is what I had asked for in prayer.}\sensenumber{2}\definition{political groupTäbanenatt ttoen paenenang kullum. (The group that talks about planned topics.)}\allomorph{wes}\allomorph{was}}
\entry{wawa}{\headword{wawa}\pos{n.}\sensenumber{2}\definition{type of tree that grows in the bush and along creeks white flowers and blue fruit}}
\entry{wawaem}{\headword{wawaem}\pos{n.}\sensenumber{1}\definition{hiss}\example{Ngäna wawaem de dandär.}{I heard a hiss.}\sensenumber{2}\definition{current}\example{Bituri mängallang abal ine wawaem de dangalltallo mutt i.}{They paddled upstream against the strong current of the Bituri river.}\sensenumber{3}\definition{to flow}\example{Yogoll mamott me ine da bingkälän a wawaem bognän.}{During rainy times, the water rises and flows.}}
\entry{Wawase}{\headword{Wawase}\pos{pn.}\sensenumber{3}\definition{male personal name}}
\entry{Waweba}{\headword{Waweba}\pos{pn.}\sensenumber{3}\definition{male personal name}}
\entry{Wawera}{\headword{Wawera}\pos{pn.}\sensenumber{3}\definition{male personal name}}
\entry{wawonai}{\headword{wawonai}\pos{n.}\sensenumber{3}\definition{type of long yam with a hooked end, white interior, and hairs}}
\entry{waya}{\headword{waya}\pos{n.}\sensenumber{3}\definition{type of pronged metal spear}}
\entry{waya gullem}{\headword{waya gullem}\pos{n.}\sensenumber{3}\definition{type of venomous snakeGullem käsre abal dan. Obo pätt a era llokottllokott dan. (It's a very small snake. Its body is firm.)}}
\entry{Wayapampe}{\headword{Wayapampe}\pos{pn.}\sensenumber{3}\definition{Wayapampe (toponym)}}
\entry{wayati}{\headword{wayati}\pos{n.}\sensenumber{3}\definition{watermelon}\example{Ngämo wayati kongkom e nalle.}{[You] go carry my watermelons.}}
\entry{wäd}{\headword{wäd}\variant{sp. var. of}{ud}}
\entry{wädɨwädɨg}{\headword{wädɨwädɨg}\pos{n.}\sensenumber{3}\definition{type of tree}}
\entry{wädwäd}{\headword{wädwäd}\pos{n.}\sensenumber{3}\definition{type of tree}}
\entry{wägba}{\headword{wägba}\pos{n.}\sensenumber{3}\definition{type of tree that grows in the bush with white flowers, bark used as medicine, and strong wood used for posts; helps make dogs' noses more sensitive}}
\entry{wäk}{\headword{wäk}\variant{var. of}{uk}}
\entry{wäkär}{\headword{wäkär}\pos{n.}\sensenumber{3}\definition{type of birdLlo tomtom alle ma gogowag pa dan. (It's a bird that builds its nest from piles of sticks.)}}
\entry{wäkɨs}{\headword{wäkɨs}\pos{n.}\sensenumber{3}\definition{type of bird}}
\entry{wäl}{\headword{wäl}\pos{n.}\sensenumber{3}\definition{main surface-level stem of a plant with rhizomes (e.g. sweet potato, lemongrass)}\example{Kak bo nae a wäl peyang gognegän.}{Grandmother's sweet potato plants have stems.}}
\entry{wälep}{\headword{wälep}\pos{n.}\sensenumber{3}\definition{type of tree that grows in the bush with blue flowers and small blue fruit}}
\entry{Wäli}{\headword{Wäli}\pos{pn.}\sensenumber{3}\definition{female personal name}}
\entry{wälsa}{\headword{wälsa}\pos{n.}\sensenumber{3}\definition{type of tree that grows in the bush}}
\entry{wäll}{\headword{wäll}\variant{var. of}{oll}}
\entry{wälländd}{\headword{wälländd}\variant{dial. var. of}{ollondd}}
\entry{wälläng}{\headword{wälläng}\pos{n.}\sensenumber{3}\definition{backwoods, hinterland, (Australia) bush (any rural, undeveloped landscape)}\example{Lla da wälläng e dallän.}{The man goes to the bush.}}
\entry{wälläng ttäp}{\headword{wälläng ttäp}\pos{n.}\sensenumber{3}\definition{Papuan eagleWälläng me giddollag pa ulle dan, ddob ddäddäg de mänyi bägäddnegän. (It's a big bird that lives in the bush; it'll kill other animals.)}}
\entry{wällängakäbu}{\headword{wällängakäbu}\pos{n.}\sensenumber{3}\definition{wompoo fruit doveWälläng me giddollag pa dan. (It's a bird that lives in the bush.)}}
\entry{wälle}{\headword{wälle}\variant{var. of}{olle}}
\entry{wällegäll}{\headword{wällegäll}\pos{n.}\sensenumber{3}\definition{type of tree with fruit that are black and edible when ripe, leaves used to roll cigarettes, and roots used to treat toothache or asthma}}
\entry{wällwäll}{\headword{wällwäll}\pos{n.}\sensenumber{3}\definition{type of tree}}
\entry{Wän}{\headword{Wän}\pos{pn.}\sensenumber{3}\definition{female personal name}}
\entry{wän}{\headword{wän}\pos{n.}\sensenumber{3}\definition{boil (on skin)}}
\entry{wänänang}{\headword{wänänang}\pos{mod.}\sensenumber{3}\definition{provider, bringing back food for one's family}\example{Bogo ddobae wänänang dan.}{She is a provider.}}
\entry{wändäg}{\headword{wändäg}\pos{S vt.}\sensenumber{3}\definition{to crowd}\example{Yesu adame obo kollmällang de umllang dägnegän oblle gall kälsäre särämbae e, oba lla da ddone obom beyawändägallo.}{This is why Jesus told his disciples to prepare a small boat for him, so that people wouldn't come crowd him.}}
\entry{wänkäm}{\headword{wänkäm}\pos{n.}\sensenumber{1}\definition{belly button, navel}\sensenumber{2}\definition{anus}\sensenumber{2}\definition{umbilical cord}\subentry{\headword{wänkäm tärpan}\pos{n.}\definition{umbilical cord}}}
\entry{wänkäm molle}{\headword{wänkäm molle}\pos{n.}\sensenumber{2}\definition{soft part of a shoot or sucker (e.g. of taro, banana, or sago)}}
\entry{wäno}{\headword{wäno}\pos{n.}\sensenumber{2}\definition{type of tree that grows in the grassland and along creeks with white flowers, small brown fruit, and bark used on rooves}}
\entry{wängän}{\headword{wängän}\pos{n.}\sensenumber{2}\definition{type of tree}}
\entry{wäny}{\headword{wäny}\variant{sp. var. of}{ony}}
\entry{wäp}{\headword{wäp}\variant{fr. var. of}{up}}
\entry{wärenzbag}{\headword{wärenzbag}\pos{n.}\sensenumber{2}\definition{type of taro}}
\entry{wärpir}{\headword{wärpir}\pos{n.}\sensenumber{2}\definition{slippery mud found by the river}}
\entry{wätaote}{\headword{wätaote}\pos{n.}\sensenumber{1}\definition{type of large vine that grows in bush; used to tie fence posts together}\sensenumber{2}\definition{type of taro}}
\entry{wätät}{\headword{wätät}\variant{var. of}{otät}}
\entry{wäte}{\headword{wäte}\variant{fr. var. of}{ute}}
\entry{Wätt bo ma}{\headword{Wätt bo ma}\pos{pn.}\sensenumber{2}\definition{Wätt bo ma (toponym)}}
\entry{Wäziag}{\headword{Wäziag}\pos{pn.}\sensenumber{2}\definition{personal name}}
\entry{Wed}{\headword{Wed}\pos{pn.}\sensenumber{1}\definition{Wed (toponym)}\sensenumber{2}\definition{female personal name}}
\entry{Wedereamo}{\headword{Wedereamo}\pos{pn.}\sensenumber{2}\definition{Wederehiamo (on the south side of the mouth of the Fly River)}}
\entry{wel}{\headword{wel}\pos{n.}\sensenumber{2}\definition{wind}\example{Känazbag mänyi wel peyang bogon.}{Tomorrow, it will be windy.}\sensenumber{2}\definition{window}\etymology{lit. 'wind door'}\subentry{\headword{wel ud}\pos{n.}\definition{window}}}
\entry{welwel}{\headword{welwel}\pos{n.}\sensenumber{2}\definition{type of birdObo ma da llo tomtom alle gogoag dan. (It builds its nest from piles of sticks.)}}
\entry{welwele}{\headword{welwele}\pos{n.}\sensenumber{2}\definition{dove}}
\entry{Wendi}{\headword{Wendi}\pos{n.}\sensenumber{2}\definition{female personal name}}
\entry{Wendy}{\headword{Wendy}\variant{sp. var. of}{Wendi}}
\entry{wensäg}{\headword{wensäg}\variant{sp. var. of}{wanseg}}
\entry{wer}{\headword{wer}\pos{n.}\sensenumber{2}\definition{type of tree with edible black fruit with one seed}}
\entry{Wesley}{\headword{Wesley}\variant{sp. var. of}{Wesli}}
\entry{Wesli}{\headword{Wesli}\pos{pn.}\sensenumber{2}\definition{male personal name}}
\entry{wi1}{\headword{wi1}\pos{S vi.}\sensenumber{2}\definition{to settle}\example{Särem a deyawinän.}{Darkness settled.}\example{Ttängäm a zäme deyawinän toto me.}{The village had already settled down in the early evening.}}
\entry{wi2}{\headword{wi2}\pos{interj.}\sensenumber{2}\definition{oh}}
\entry{wia}{\headword{wia}\variant{sp. var. of}{wiya}}
\entry{wiasara}{\headword{wiasara}\variant{sp. var. of}{wiyasara}}
\entry{wib}{\headword{wib}\pos{n.}\sensenumber{2}\definition{type of tree}}
\entry{wibell}{\headword{wibell}\pos{n.}\sensenumber{2}\definition{type of tree}}
\entry{Wiben}{\headword{Wiben}\pos{pn.}\sensenumber{2}\definition{male personal name}}
\entry{Widama}{\headword{Widama}\pos{pn.}\sensenumber{2}\definition{Widama (toponym)}}
\entry{widere}{\headword{widere}\pos{n.}\sensenumber{2}\definition{paddle, oarLlo popoatt gall gällae ma za dan. (It's a thing made of wood to paddle a canoe.)}\example{Ngäna ngämlle ttongo widere sisor de popo e dan.}{I want to make a new paddle.}}
\entry{widwid}{\headword{widwid}\pos{n.}\sensenumber{2}\definition{type of plant with big leaves}}
\entry{Wik}{\headword{Wik}\pos{pn.}\sensenumber{2}\definition{male personal name}}
\entry{wik}{\headword{wik}\pos{n.}\sensenumber{2}\definition{week}\etymology{from Englishweek}}
\entry{Wilma}{\headword{Wilma}\pos{pn.}\sensenumber{2}\definition{female personal name}}
\entry{wilwil}{\headword{wilwil}\pos{n.}\sensenumber{2}\definition{type of tree that grows in the bush}}
\entry{Willie}{\headword{Willie}\pos{pn.}\sensenumber{2}\definition{male personal name}}
\entry{Wim}{\headword{Wim}\pos{pn.}\sensenumber{2}\definition{Wim (Kawam-speaking village in Oriomo-Bituri Rural LLG; GPS: 8.762144, 142.770164)}}
\entry{win}{\headword{win}\pos{n.}\sensenumber{2}\definition{win}\etymology{from Englishwin}}
\entry{Wini}{\headword{Wini}\variant{sp. var. of}{Winny}}
\entry{winisde}{\headword{winisde}\pos{n.}\sensenumber{2}\definition{Wednesday}\etymology{from EnglishWednesday}}
\entry{Winny}{\headword{Winny}\pos{pn.}\sensenumber{2}\definition{female personal name}}
\entry{Winsen}{\headword{Winsen}\variant{sp. var. of}{Winson}}
\entry{Winson}{\headword{Winson}\pos{pn.}\sensenumber{2}\definition{male personal name}}
\entry{Wingäm}{\headword{Wingäm}\pos{pn.}\sensenumber{2}\definition{female personal name}}
\entry{Winy}{\headword{Winy}\pos{pn.}\sensenumber{2}\definition{Winy (toponym)}}
\entry{winy}{\headword{winy}\pos{n.}\sensenumber{2}\definition{honeyMokowang abal ine dan, llo ik o ttägäll ik me dan wälläng me o ap me. (It's very sweet water; it's inside trees or termite mounds in the bush or grassland.)}\example{Däbe tätämatt winy a ka tomowangtomowang moko allan.}{That honey from yesterday tastes a little bitter.}\sensenumber{2}\definition{honeycombGazibra ttoe me winyteya de komllaebekomllaebe dongkop ddäganenatt. (Put honeycomp on gazibra snakeskin}\subentry{\headword{winyteya}\pos{n.}\definition{honeycombGazibra ttoe me winyteya de komllaebekomllaebe dongkop ddäganenatt. (Put honeycomp on gazibra snakeskin}}}
\entry{wiowa}{\headword{wiowa}\variant{dial. var. of}{wiyo}}
\entry{wipell}{\headword{wipell}\pos{n.}\sensenumber{2}\definition{type of tall palm that grows along the riversideMa katre tater e ttäkoe ma dan. (Chopped down for house floorboards.)}}
\entry{wipellgallagallab}{\headword{wipellgallagallab}\pos{n.}\sensenumber{2}\definition{chevron weaving pattern}}
\entry{Wipi}{\headword{Wipi}\pos{pn.}\sensenumber{2}\definition{Wipi language}}
\entry{Wipim}{\headword{Wipim}\pos{pn.}\sensenumber{2}\definition{Wipim (Wipi- and Kawam-speaking village in Oriomo-Bituri Rural LLG; GPS: -8.786616, 142.872201)}}
\entry{wirog}{\headword{wirog}\pos{n.}\sensenumber{2}\definition{type of native bananaTupi dan, obo käp a ulleulle dag. Käp a obo o me otät ma da ako kire da yu ma da. Obo pätt ine da ako tatu wi mäkamäke ma dan itrel lelang att. (It's long; its bunches are big. When ripe, its fruit are eaten, and when unripe, it's cooked. The liquid from the stem is used to bathe to ward off disease.)}}
\entry{wiswis}{\headword{wiswis}\pos{n.}\sensenumber{2}\definition{type of tree that grows in the bush with white flowers and edible orange fruit}}
\entry{wit}{\headword{wit}\pos{n.}\sensenumber{2}\definition{wheat}\etymology{from Englishwheat}}
\entry{witara}{\headword{witara}\pos{n.}\sensenumber{2}\definition{type of gardenIne baddnenatt me walle menae me ttängäm. (A garden planted by the river after the water dries up.)}\example{Witara da karama me däm ibeny ma ttängäm dan.}{They are planting watermelon, pumpkin, taro, and aibika in the river garden.}\example{Witara me wayati da, pamker a, biye da wa mompel a däbem de ibeb erallo.}{They are planting watermelon, pumpkin, taro, and aibika in the river garden.}\example{Lla da witara me mondre anggan.}{[You] come over here!}}
\entry{wiya}{\headword{wiya}\pos{S vi.}\sensenumber{2}\definition{come (imperative, singular form)}\example{Wiya gänyerimae!}{[You] come over here!}\sensenumber{2}\definition{plural form of wiya}\example{Tämamae wiyamom!}{Come, everyone!}\subentry{\headword{wiyamom}\pos{S vi.}\definition{plural form of wiya}}}
\entry{wiyasara}{\headword{wiyasara}\pos{n.}\sensenumber{2}\definition{silver gull}}
\entry{wiyo}{\headword{wiyo}\pos{interj.}\sensenumber{2}\definition{wow}\allomorph{wiya}}
\entry{wiyowae}{\headword{wiyowae}\pos{interj.}\sensenumber{2}\definition{wow; phew}\allomorph{wiyowa}}
\entry{wiyowe}{\headword{wiyowe}\pos{n.}\sensenumber{2}\definition{type of large palm that grows in the bush or along creeks with flowers that start from the top and spread downwards and white fruit that hangs like coconut}}
\entry{wizarab}{\headword{wizarab}\pos{n.}\sensenumber{2}\definition{type of pandanus with red fruitDu mab, walle mäg me päddabag dan, ngattong masamasar a dotaemäneyo. (Wild pandanus growing by creeks; our ancestors ate it.)}}
\entry{Wizing}{\headword{Wizing}\pos{pn.}\sensenumber{2}\definition{male personal name}}
\entry{wɨndwɨnd}{\headword{wɨndwɨnd}\pos{S vt.}\sensenumber{2}\definition{to cover, bury}\example{Pätt de ekaklle alle dawɨndeya.}{We covered the body with earth.}\allomorph{wund}\allomorph{wɨnd}}
\entry{Wɨr}{\headword{Wɨr}\pos{pn.}\sensenumber{2}\definition{Wɨr (toponym)}}
\entry{Wɨrbun}{\headword{Wɨrbun}\pos{pn.}\sensenumber{2}\definition{Wɨrbun (toponym)}}
\entry{wɨtwɨt}{\headword{wɨtwɨt}\pos{n.}\sensenumber{2}\definition{type of sago}}
\entry{wo}{\headword{wo}\variant{sp. var. of}{o2}}
\entry{woboll}{\headword{woboll}\pos{n.}\sensenumber{2}\definition{type of plantPollon me päddabag dan, turik do ma dan. (It grows in bushes; it's used for axe handles.)}}
\entry{wod}{\headword{wod}\pos{n.}\sensenumber{1}\definition{type of fatty fish}\sensenumber{2}\definition{type of long yam with a white interior and few hairs}}
\entry{wodd memba}{\headword{wodd memba}\pos{n.}\sensenumber{2}\definition{ward memberLlayabaene moko eka de aya komnen agan. (The one who takes what the people want to say.)}\etymology{from Englishward member}}
\entry{woddowoddo}{\headword{woddowoddo}\pos{n.}\sensenumber{1}\definition{rusty pitohuiWälläng pa dan. (It's a bird of the bush.)}\sensenumber{2}\definition{Amboyna cuckoo-dove}}
\entry{wolläng}{\headword{wolläng}\variant{dial. var. of}{wälläng}}
\entry{wolle}{\headword{wolle}\variant{var. of}{olle}}
\entry{wollong}{\headword{wollong}\variant{dial. var. of}{wälläng}}
\entry{Wonie}{\headword{Wonie}\pos{pn.}\sensenumber{2}\definition{Wonie (Wipi-speaking village in Oriomo-Bituri Rural LLG; GPS: -8.838939, 142.975007)}}
\entry{worbam}{\headword{worbam}\variant{sp. var. of}{orbam}}
\entry{Wot}{\headword{Wot}\pos{pn.}\sensenumber{2}\definition{Wot (toponym)}}
\entry{wowo}{\headword{wowo}\pos{S vt.}\sensenumber{2}\definition{to clear floating grass by pushing through with a canoe}\example{Tätäm ibi tawa de duwonalla.}{Yesterday we cleared the floating grass.}\allomorph{wo}\allomorph{wonen}}
\entry{wud}{\headword{wud}\variant{sp. var. of}{ud}}
\entry{wudu}{\headword{wudu}\variant{dial. var. of}{udu}}
\entry{wuk}{\headword{wuk}\variant{sp. var. of}{uk}}
\entry{Wun}{\headword{Wun}\pos{pn.}\sensenumber{2}\definition{male personal name}}
\entry{wunkäm}{\headword{wunkäm}\variant{var. of}{wänkäm}}
\entry{wup}{\headword{wup}\variant{fr. var. of}{up}}
\entry{wup ttämbällag}{\headword{wup ttämbällag}\pos{n.}\sensenumber{2}\definition{type of spear}}
\entry{Wur}{\headword{Wur}\variant{sp. var. of}{Ur}}
\entry{Wurlaimäll}{\headword{Wurlaimäll}\pos{pn.}\sensenumber{2}\definition{Wurlaimäll (toponym)}}
\entry{Wuroi}{\headword{Wuroi}\variant{sp. var. of}{Uroe}}
\entry{wute}{\headword{wute}\variant{sp. var. of}{ute}}
\entry{wutt}{\headword{wutt}\variant{sp. var. of}{utt1}}
\lettersection{Y}
\entry{ya}{\headword{ya}\pos{interj.}\sensenumber{2}\definition{scram, go away}}
\entry{yaber}{\headword{yaber}\pos{n.}\sensenumber{2}\definition{type of tree that grows in the bush with white flowers and poisonous bark used to catch fish}}
\entry{yad}{\headword{yad}\pos{n.}\sensenumber{2}\definition{yard}\etymology{from Englishyard}}
\entry{yae}{\headword{yae}\pos{kin.}\sensenumber{2}\definition{motherZaze me baba bälle ngäminggag dan}}
\entry{yaedidib}{\headword{yaedidib}\pos{n.}\sensenumber{2}\definition{type of long, narrow grass (\textbackslashtextasciitilde0.3 m)}}
\entry{yagäl}{\headword{yagäl}\pos{n.}\sensenumber{1}\definition{type of tree that grows in the grassland with leaves used to sand bows and spears}\sensenumber{2}\definition{to smooth with the yagäl leaf}}
\entry{yagyag}{\headword{yagyag}\variant{fr. var. of}{yagyeg}}
\entry{yagyeg}{\headword{yagyeg}\pos{S vt.}\sensenumber{2}\definition{to search, look for}\example{Män nagda bom yagnen dängkamän.}{‎‎The girl started to look for her friend.}\example{Erodias nyongo de deyagnegnän Zon bälle gäz e.}{Herodias was looking for ways to kill John.}\allomorph{yag}\allomorph{yangg}\allomorph{ya}}
\entry{yaiya}{\headword{yaiya}\variant{var. of}{yaya}}
\entry{yal}{\headword{yal}\pos{n.}\sensenumber{2}\definition{yellow-billed kingfisherKwälläb ik me giddollag pa dan. (It's a bird that lives in big termite mounds.)}}
\entry{Yam}{\headword{Yam}\pos{pn.}\sensenumber{2}\definition{Yam (toponym)}}
\entry{Yamayama}{\headword{Yamayama}\pos{pn.}\sensenumber{2}\definition{Yamayama (toponym)}}
\entry{Yamega}{\headword{Yamega}\pos{pn.}\sensenumber{2}\definition{Iamega (Wipi-speaking village in Oriomo-Bituri Rural LLG)}}
\entry{Yamkong}{\headword{Yamkong}\pos{n.}\sensenumber{2}\definition{name of clan}}
\entry{yante}{\headword{yante}\pos{n.}\sensenumber{2}\definition{type of large tree that grows in the grassland with white flowers and wood used for house sticks}}
\entry{Yao}{\headword{Yao}\pos{pn.}\sensenumber{2}\definition{Taeme language}\etymology{from Taemeyao 'no'}}
\entry{Yarbab}{\headword{Yarbab}\pos{pn.}\sensenumber{2}\definition{male personal name}}
\entry{Yarte}{\headword{Yarte}\pos{pn.}\sensenumber{2}\definition{Yarte (camping place)}}
\entry{yarte}{\headword{yarte}\pos{n.}\sensenumber{2}\definition{type of tree with young wood used for house sticksLlo ulle dan. Obo igi ttoe a wabeb ma dan a kängkäm ma dan. Kumye itrel täräp me nane ma dan. (It's a big tree. Its inner bark is beaten and squeezed. When sick with cough, the liquid is drunk.)}}
\entry{yaru}{\headword{yaru}\pos{n.}\sensenumber{2}\definition{type of yam with a purple interior, hairs, and thorns}}
\entry{yaryem}{\headword{yaryem}\pos{n.}\sensenumber{2}\definition{type of big yam with a white interior, hairs, and thorns}}
\entry{yatän}{\headword{yatän}\pos{S vt.}\sensenumber{2}\definition{to fetch (water)}\example{Oblle ine neyatän.}{[You] fetch water for him.}\allomorph{yat}}
\entry{yattän}{\headword{yattän}\pos{S vi.}\sensenumber{2}\definition{to disembark, get off, get out}\example{Gall atta goyattänän.}{She got out of the canoe.}}
\entry{Yau}{\headword{Yau}\variant{sp. var. of}{Yow}}
\entry{yaul}{\headword{yaul}\pos{n.}\sensenumber{2}\definition{type of long yam with a white interior and few hairs}}
\entry{Yawani}{\headword{Yawani}\pos{pn.}\sensenumber{2}\definition{male personal name}}
\entry{Yawen}{\headword{Yawen}\pos{pn.}\sensenumber{2}\definition{female personal name}}
\entry{Yawin}{\headword{Yawin}\variant{var. of}{Yawen}}
\entry{yaya}{\headword{yaya}\pos{kin.}\sensenumber{2}\definition{father}}
\entry{yäbäd}{\headword{yäbäd}\pos{n.}\sensenumber{1}\definition{sunDdapall me za dan indrang allan yäbdo me. (It's a thing in the sky that shines during the day.)}\example{Yäbäd de ddapall käkan a dakonewän.}{The clouds covered the sun.}\sensenumber{2}\definition{season when the dry season starts and people go camping in the bush (eleventh season; corresponds to August)}\sensenumber{3}\definition{weather}\example{Mer abal yäbäd daeya.}{It was very good weather.}\sensenumber{1}\definition{dry, hot season characterized by burning grass (twelfth season; corresponds to September)}\sensenumber{2}\definition{drought}\sensenumber{2}\definition{really hot}\sensenumber{2}\definition{solar noon (when the sun reaches its zenith)}\sensenumber{2}\definition{lunch}\sensenumber{2}\definition{to dawn}\subentry{\headword{yäbäd bäng}\pos{n.}\definition{dry, hot season characterized by burning grass (twelfth season; corresponds to September)}}\subentry{\headword{yäbäd kuttakuttang}\pos{mod.}\definition{really hot}}\subentry{\headword{yäbäd tuktuk}\pos{n.}\definition{solar noon (when the sun reaches its zenith)}}\subentry{\headword{yäbäd tuktuk duwem}\pos{n.}\definition{lunch}}\subentry{\headword{yäbäd gazen}\pos{S vi.}\definition{to dawn}}}
\entry{yäbäd källa}{\headword{yäbäd källa}\pos{n.}\sensenumber{2}\definition{type of big taro}\etymology{lit. 'sun poop'}}
\entry{yäbäd ttänttämang}{\headword{yäbäd ttänttämang}\pos{n.}\sensenumber{2}\definition{hot season when new gardens are burnt (thirteenth month; corresponds to October)}}
\entry{yäbäyäbäd}{\headword{yäbäyäbäd}\pos{n.}\sensenumber{1}\definition{type of tree that grows in the bush with white flowers and red fruit}\sensenumber{2}\definition{type of grass}\etymology{redup. ofyäbäd}}
\entry{yäbäyäbäl}{\headword{yäbäyäbäl}\pos{n.}\sensenumber{2}\definition{type of tree}}
\entry{yäbdo}{\headword{yäbdo}\variant{fr. var. of}{ebdo}}
\entry{yäbik}{\headword{yäbik}\pos{n.}\sensenumber{2}\definition{sharp gardening stick}}
\entry{yäkäl}{\headword{yäkäl}\pos{kin.}\sensenumber{2}\definition{cousin}\sensenumber{2}\definition{uncle (one's mother's sister's husband)}\etymology{probably from yäkäl + mäda}\subentry{\headword{yäkälnda}\pos{kin.}\definition{uncle (one's mother's sister's husband)}}}
\entry{yäko}{\headword{yäko}\variant{var. of}{yoko}}
\entry{yämak}{\headword{yämak}\pos{n.}\sensenumber{2}\definition{type of big tree found in the bush and by the river}}
\entry{yämän}{\headword{yämän}\pos{n.}\sensenumber{2}\definition{type of big tuber with a reddish interior (not a yam)}}
\entry{yämbäg}{\headword{yämbäg}\pos{S vt.}\sensenumber{2}\definition{to disown, repudiate}\example{Ngäna mɨnyi ddone abal bam baembäg.}{I will never disown you.}\allomorph{embäg}}
\entry{yänddäna}{\headword{yänddäna}\variant{var. of}{enddäna}}
\entry{yänkllollang}{\headword{yänkllollang}\pos{mod.}\sensenumber{2}\definition{forgetful}}
\entry{yärany}{\headword{yärany}\variant{var. of}{erany}}
\entry{yärmuyärmu}{\headword{yärmuyärmu}\pos{n.}\sensenumber{2}\definition{type of tree}}
\entry{yäru}{\headword{yäru}\pos{n.}\sensenumber{2}\definition{type of small tree with thorns}}
\entry{yätt}{\headword{yätt}\pos{n.}\sensenumber{2}\definition{forehead}}
\entry{yebdo}{\headword{yebdo}\variant{sp. var. of}{ebdo}}
\entry{yerngän}{\headword{yerngän}\variant{var. of}{irängän}}
\entry{Yesu}{\headword{Yesu}\pos{pn.}\sensenumber{2}\definition{Jesus}}
\entry{yewede}{\headword{yewede}\variant{fr. var. of}{ewede}}
\entry{yid}{\headword{yid}\pos{n.}\sensenumber{2}\definition{liquid extracted from a plant}\example{nge id}{coconut cream}\sensenumber{2}\definition{the seventh and final stage of sago growth during which the pith will not yield any starch}\example{Tätäm ubi yɨdmeny sana de däkämeyo.}{Yesterday, they squeezed a dry sago palm.}\etymology{yɨd + =meny}\subentry{\headword{yɨdmeny}\pos{n.}\definition{the seventh and final stage of sago growth during which the pith will not yield any starch}}}
\entry{yimne}{\headword{yimne}\variant{sp. var. of}{imne}}
\entry{Yina}{\headword{Yina}\pos{pn.}\sensenumber{2}\definition{female personal name}}
\entry{yinbo}{\headword{yinbo}\variant{sp. var. of}{inbo}}
\entry{yindrang}{\headword{yindrang}\variant{sp. var. of}{indrang}}
\entry{yinu}{\headword{yinu}\variant{sp. var. of}{inu}}
\entry{yirɨm}{\headword{yirɨm}\variant{sp. var. of}{Iräm}}
\entry{yɨb}{\headword{yɨb}\pos{n.}\sensenumber{2}\definition{type of yam}}
\entry{yɨd}{\headword{yɨd}\variant{var. of}{yid}}
\entry{yo}{\headword{yo}\pos{n.}\sensenumber{2}\definition{liver}}
\entry{yobeg}{\headword{yobeg}\pos{n.}\sensenumber{2}\definition{type of cultivated shrub with white and yellow flowers and long leaves used to tie yam shoots to yam sticks}}
\entry{yogoll}{\headword{yogoll}\pos{n.}\sensenumber{2}\definition{rainDdapall att ine. (Water from the sky.)}\example{Sisri yogoll allan.}{It is raining now.}\sensenumber{2}\definition{black cloud}\sensenumber{2}\definition{dark rain cloud}\subentry{\headword{yogoll kutt}\pos{n.}\definition{black cloud}}\subentry{\headword{yogoll särem}\pos{n.}\definition{dark rain cloud}}}
\entry{Yokas}{\headword{Yokas}\pos{pn.}\sensenumber{2}\definition{Yokas (camping place in Limol)}}
\entry{yoko}{\headword{yoko}\pos{n.}\sensenumber{2}\definition{type of cane used for building houses, bows, and canoes}}
\entry{Yokon}{\headword{Yokon}\pos{pn.}\sensenumber{2}\definition{personal name}}
\entry{yon}{\headword{yon}\pos{n.}\sensenumber{2}\definition{dream}\example{Ngäna käza yon de wätaran.}{I dreamed about crocodiles.}}
\entry{Yop}{\headword{Yop}\pos{pn.}\sensenumber{2}\definition{Yop (toponym)}}
\entry{yorko}{\headword{yorko}\pos{n.}\sensenumber{2}\definition{type of large cane found in the bush}}
\entry{yorkoll}{\headword{yorkoll}\pos{n.}\sensenumber{2}\definition{dirt}\sensenumber{2}\definition{dirty}\example{Bongo ddone yorkollang alle.}{You are not dirty.}\subentry{\headword{yorkollang}\pos{mod.}\definition{dirty}}}
\entry{Yoteang}{\headword{Yoteang}\pos{pn.}\sensenumber{2}\definition{Yoteang (toponym)}}
\entry{yoto}{\headword{yoto}\pos{n.}\sensenumber{2}\definition{type of biting bee found in trees}}
\entry{Yow}{\headword{Yow}\pos{pn.}\sensenumber{2}\definition{Yau (in Gogodala Rural LLG)}}
\entry{yowa}{\headword{yowa}\pos{n.}\sensenumber{2}\definition{vagina}\example{Eiz itrell a lla bo wa piye me a mälla bo yowa käpät me dadeg.}{The HIV virus exists in men's sperm and in women's vaginal fluid.}}
\entry{yowede}{\headword{yowede}\variant{var. of}{ewede}}
\entry{yu}{\headword{yu}\pos{n.}\sensenumber{1}\definition{fire}\example{Sali yu di mermerae lläklläk eran.}{Sali is spreading the fire nicely.}\sensenumber{2}\definition{firewood}\sensenumber{3}\definition{to cook over fire}\example{Obo wätät de yu dägagän, duwem gogon.}{He cooked his food on the fire and ate it.}\example{Pa de yu dägaebneyo.}{They cooked the birds on the fire.}\example{Kollba de yu deagän.}{He cooked his two fish on the fire.}\sensenumber{3}\definition{gun, firearmTobol käp zan ma dan. (It's where you put in bullets.)}\sensenumber{3}\definition{flame}\sensenumber{3}\definition{firewood}\sensenumber{3}\definition{small fire}\sensenumber{3}\definition{charcoal}\sensenumber{3}\definition{hell}\sensenumber{3}\definition{burning wood, burnt wood}\etymology{lit. 'fire bow'}\subentry{\headword{yu bägäl}\pos{n.}\definition{gun, firearmTobol käp zan ma dan. (It's where you put in bullets.)}}\subentry{\headword{yu dumbrel}\pos{n.}\definition{flame}}\subentry{\headword{yu kire}\pos{n.}\definition{firewood}}\subentry{\headword{yu ngongom}\pos{n.}\definition{small fire}}\subentry{\headword{yu torkomoll}\pos{n.}\definition{charcoal}}\subentry{\headword{yu ttängäm}\pos{n.}\definition{hell}}\subentry{\headword{yu ttätta}\pos{n.}\definition{burning wood, burnt wood}}}
\entry{yu bäng}{\headword{yu bäng}\pos{n.}\sensenumber{3}\definition{firestick (to start a fire)}\example{Auma watta ako ngäna yu bäng de beyangäs.}{I'll bring the firestick back from the grave.}}
\entry{yubud}{\headword{yubud}\pos{n.}\sensenumber{3}\definition{type of tree}}
\entry{yuddädda}{\headword{yuddädda}\pos{n.}\sensenumber{3}\definition{type of palm with branches used for armbandsObo utt a wätat ma dan, ma ttängäm me dag o wälläng me. (Its shoots are edible; they are in the bush and in the village.)}}
\entry{Yuga}{\headword{Yuga}\pos{pn.}\sensenumber{3}\definition{male personal name}}
\entry{Yugui}{\headword{Yugui}\pos{pn.}\sensenumber{3}\definition{female personal name}}
\entry{Yun}{\headword{Yun}\pos{pn.}\sensenumber{3}\definition{female personal name}}
\entry{yunipom}{\headword{yunipom}\pos{n.}\sensenumber{3}\definition{uniform}\etymology{from Englishuniform}}
\entry{yunu}{\headword{yunu}\variant{var. of}{inu}}
\entry{yure}{\headword{yure}\pos{n.}\sensenumber{3}\definition{type of sagoAi mägag sana dan. (It's a sago that produces well.)}}
\entry{yuru}{\headword{yuru}\pos{n.}\sensenumber{3}\definition{type of pandanus}}
\entry{yurwe}{\headword{yurwe}\pos{n.}\sensenumber{3}\definition{type of tree}}
\entry{yus}{\headword{yus}\pos{A vt.}\sensenumber{3}\definition{to use}\example{Kumuddäga eka de ngäna yus anggan.}{I use three languages.}\etymology{from Englishuse}}
\entry{yuwet}{\headword{yuwet}\pos{n.}\sensenumber{3}\definition{short period of time}\example{Ngäna bam mɨnyi yuwet e pen de banttog.}{I will lend a pen to you (lit. give you a pen for a short time).}\sensenumber{3}\definition{temporarily, briefly}\example{Ubi siti we yuwetyuwet gobällnän.}{They went into the city briefly.}\subentry{\headword{yuwetyuwet}\pos{adv.}\definition{temporarily, briefly}}}
\entry{yuwog}{\headword{yuwog}\pos{quant.}\sensenumber{1}\definition{many, a lot of (for countable nouns)}\example{yuwog abal mani}{a lot of money}\example{Ngämo nagnag a yuwog abal dagän.}{I have many friends.}\example{Yuwog abal kollba de ngämi daittaemnalla.}{We (excl.) were catching many fish.}\sensenumber{2}\definition{for no reason, on a whim}\example{Ngäna yuwog dae ibi allan Bisuaka we.}{I'm going to Bisuaka on a whim.}\example{Ngäna yuwog dae giddoll allan.}{I am living without any food.}\sensenumber{1}\definition{needlessly, excessively}\example{Lla da o mälla da obaoba ddone yuwoyuwog bulignegnän.}{Men and women should not be having sex with each other needlessly.}\sensenumber{2}\definition{not completely, not properly}\example{Sisor llɨg a ubi yuwoyuwog panypeny erallo Ende eka de.}{Young kids aren't speaking Ende properly.}\subentry{\headword{yuwoyuwog}\pos{adv.}\definition{needlessly, excessively}}}
\lettersection{Z}
\entry{za}{\headword{za}\pos{n.}\sensenumber{2}\definition{thing}\example{Oba za da gänyag.}{Here are their things.}}
\entry{zaa}{\headword{zaa}\variant{sp. var. of}{za}}
\entry{zaga}{\headword{zaga}\pos{pers. pron.}\sensenumber{2}\definition{self (forms reflexive pronouns)}\example{Ngäna ngämo zaga butruwam.}{I will lay myself down.}}
\entry{zagu}{\headword{zagu}\pos{n.}\sensenumber{2}\definition{type of sugarcane-like plant that grows in the swamp}}
\entry{Zakae}{\headword{Zakae}\pos{pn.}\sensenumber{2}\definition{male personal name}}
\entry{Zakai}{\headword{Zakai}\variant{sp. var. of}{Zakae}}
\entry{zan}{\headword{zan}\pos{S vi.}\sensenumber{1}\definition{to enter, go in, go into}\example{lla zazer ma märäll me}{sized so that a person could enter}\example{Ngäna mɨnyi bozen.}{I will enter.}\example{Pollgo da täl dɨdɨr ik e gozenän.}{The frog went into the dry bamboo.}\example{Bibi ngämo ma ik e zarnen amalla.}{You all are going in my house.}\sensenumber{2}\definition{to put in}\example{Polis a obom säremang ma ik i däzaneyo.}{The police put him in prison.}\example{Bogo up wo de nyäng e dazernän.}{He put the ripe bananas in the bag.}\allomorph{zen}\allomorph{zazer}\allomorph{zer}\allomorph{zar}\allomorph{ze}\allomorph{za}}
\entry{Zanor}{\headword{Zanor}\pos{pn.}\sensenumber{2}\definition{Zanor (in Gogodala Rural LLG; GPS: -8.459817, 142.688025)}}
\entry{zanzi}{\headword{zanzi}\pos{S vi.}\sensenumber{2}\definition{to settle}\example{Ge llɨg a dedme gunziwän.}{This boy settled there.}\allomorph{nzi}\allomorph{zae}}
\entry{Zanger}{\headword{Zanger}\pos{pn.}\sensenumber{2}\definition{female personal name}}
\entry{zanggae}{\headword{zanggae}\pos{S vi.}\sensenumber{2}\definition{to roam, go around}\example{Ngäna däba ngata me gonzagae.}{I wandered around in that area.}\allomorph{zanggaenen}\allomorph{nzagae}\allomorph{nzag}}
\entry{Zarma}{\headword{Zarma}\pos{pn.}\sensenumber{2}\definition{Zarma (toponym)}}
\entry{zarmeny}{\headword{zarmeny}\pos{n.}\sensenumber{2}\definition{type of long yam with a white interior, hairs, and no thorns}}
\entry{zawatt}{\headword{zawatt}\pos{n.}\sensenumber{2}\definition{vagina}}
\entry{zazaba}{\headword{zazaba}\pos{n.}\sensenumber{2}\definition{type of bagSana ttam alle iatt sana nyäng (A sago bag woven from sago leaves.)}\example{Mänmän a sana de zazaba we dändäraebeyo.}{The girls filled their sago bags with sago.}}
\entry{zaze}{\headword{zaze}\pos{S vi.}\sensenumber{1}\definition{to give birth, lay}\example{Ngäna käp tumang zazeag dan.}{I lay many eggs.}\sensenumber{2}\definition{generation}\example{ngämo baba bakmall zaze}{my father's siblings (lit. those in the generation with my father)}}
\entry{zäbo}{\headword{zäbo}\pos{n.}\sensenumber{2}\definition{yellow-streaked loryYäbäd me ekawang pa dan. (It's a bird that sings when it's sunny.)}}
\entry{zäm}{\headword{zäm}\pos{S vi.}\sensenumber{2}\definition{to pass through}\example{Ubi tätäm gänyagaeya dinzämän.}{They passed through here yesterday.}\allomorph{nzäm}}
\entry{zämae}{\headword{zämae}\pos{S vt.}\sensenumber{2}\definition{to pour, put, transfer}\example{Ubi ine de zämaenen erallo.}{They are pouring water.}\example{Kollba dängälläbeya, nyäng e danzämaeya.}{We bought fish and put them in a bag.}\allomorph{nzämae}\allomorph{nzäm}}
\entry{zäme}{\headword{zäme}\variant{dial. var. of}{zɨme}}
\entry{zämllall}{\headword{zämllall}\pos{A vi.}\sensenumber{2}\definition{to pass by}\example{Bogo zämllall allan.}{He is passing by.}\sensenumber{2}\definition{passerby}\sensenumber{2}\definition{nonstop}\example{Bogo zämllazämllall ibi allan.}{He walks without stopping.}\etymology{zämllall + =ang}\subentry{\headword{zämllallang}\pos{n.}\definition{passerby}}\subentry{\headword{zämllazämllall}\pos{adv.}\definition{nonstop}}}
\entry{Zedem}{\headword{Zedem}\pos{pn.}\sensenumber{2}\definition{male personal name}}
\entry{zeg}{\headword{zeg}\pos{S vi.}\sensenumber{2}\definition{to be born}\example{Bongo sisri ebdo ag me ozegalle.}{You were born this morning.}\example{Ngäna do gozeg, Kibobma me.}{I was born there, in Kibobma.}\allomorph{zae}\allomorph{zig}}
\entry{Zegma}{\headword{Zegma}\pos{pn.}\sensenumber{2}\definition{Zegma (camping place)}}
\entry{Zeims}{\headword{Zeims}\pos{pn.}\sensenumber{2}\definition{male personal name}}
\entry{Zekraeya}{\headword{Zekraeya}\pos{pn.}\sensenumber{2}\definition{male personal name}}
\entry{zel}{\headword{zel}\pos{n.}\sensenumber{2}\definition{jail}\example{Obom zel ma ik i näzanallo.}{They put him in jail.}\etymology{from Englishjail}}
\entry{Zelma}{\headword{Zelma}\pos{pn.}\sensenumber{2}\definition{female personal name}}
\entry{zem}{\headword{zem}\pos{n.}\sensenumber{2}\definition{germ}\etymology{from Englishgerm}}
\entry{Zemila}{\headword{Zemila}\variant{sp. var. of}{Jemila}}
\entry{Zen}{\headword{Zen}\pos{pn.}\sensenumber{2}\definition{male personal name}}
\entry{Zeri}{\headword{Zeri}\variant{sp. var. of}{Jerry}}
\entry{Zeriko}{\headword{Zeriko}\pos{pn.}\sensenumber{2}\definition{Jericho}}
\entry{zib}{\headword{zib}\pos{n.}\sensenumber{2}\definition{type of big tree that grows in the bush}}
\entry{zib mäka}{\headword{zib mäka}\pos{n.}\sensenumber{2}\definition{type of introduced bananaTtongo mer wup dan, obo pätt a ulle dan. Däg a obo yuwog dag. Ako käp a obo o me otät ma dan, ako yu mi binzenen ma dag, a kire da yu ma dag. (It's a good banana; its trunk is big. Its bunches are plentiful. When ripe, its fruit are eaten and heated on the fire, and when unripe, they are cooked.)}}
\entry{zigae}{\headword{zigae}\pos{S vt.}\sensenumber{2}\definition{to wrap}\example{Ngäna mänyän de ttam alle zigae eran.}{I am wrapping the mänyän fish with leaves.}\example{Ngämi sana mängalae dazgiaebeya.}{We (excl.) quickly wrapped the sago.}\allomorph{zgi}}
\entry{Zina}{\headword{Zina}\pos{pn.}\sensenumber{2}\definition{female personal name}}
\entry{zire}{\headword{zire}\pos{n.}\sensenumber{2}\definition{barramundi}}
\entry{ziz}{\headword{ziz}\pos{n.}\sensenumber{2}\definition{insect}}
\entry{zizag}{\headword{zizag}\pos{n.}\sensenumber{2}\definition{owner, master, lord}\example{Zizag Adi}{the Lord God}\example{ttängäm zizag}{landowner}\example{Gänyaolle gall zizag a ngänawaenen.}{The owner of this canoe is me.}}
\entry{zizi}{\headword{zizi}\pos{S vt.}\sensenumber{2}\definition{to uncover, lift}\example{Angde ngämi ttägäll de däziya, duwem gogmam.}{When we (excl.) uncovered the mumu, we ate.}\example{Wel gullbe da gongttägän a mägda bo pite de däzinän.}{A strong wind came and lifted his mother's skirt.}\allomorph{zinen}\allomorph{zi}\allomorph{z}\allomorph{ziz}}
\entry{zɨme}{\headword{zɨme}\pos{adv.}\sensenumber{2}\definition{already}\example{Ngäna bam zime ikop nagan.}{I already saw you.}\example{Llɨg a zime ngänaeka gognegnän kallkäll atta.}{The children were already crying from cold.}}
\entry{Zo}{\headword{Zo}\pos{pn.}\sensenumber{2}\definition{male personal name}}
\entry{zo}{\headword{zo}\pos{n.}\sensenumber{2}\definition{fawn-breasted bowerbirdPällon me giddollag pa, llo tomtom alle ma gogowag dan ekaklle me. (A bird that lives in bushes; it builds its nest on the ground from piles of sticks.)}}
\entry{zobo ik}{\headword{zobo ik}\pos{n.}\sensenumber{2}\definition{walling (bark on house between sago walls and roof)}\etymology{lit. 'bowerbird's inside'}}
\entry{zogam}{\headword{zogam}\pos{n.}\sensenumber{2}\definition{rat}}
\entry{Zon}{\headword{Zon}\variant{sp. var. of}{John}}
\entry{Zonas}{\headword{Zonas}\pos{pn.}\sensenumber{2}\definition{male personal name}}
\entry{Zoni}{\headword{Zoni}\variant{sp. var. of}{Johnny}}
\entry{zora}{\headword{zora}\pos{n.}\sensenumber{2}\definition{sharp stick for peeling sago before beating it}}
\entry{Zosep}{\headword{Zosep}\pos{pn.}\sensenumber{2}\definition{male personal name}}
\entry{zowag}{\headword{zowag}\pos{mod.}\sensenumber{2}\definition{hoarse}\example{Pitapo bo inkɨm a zowag dan.}{Pitapo's voice is hoarse.}}
\entry{zozo}{\headword{zozo}\pos{S vi.}\sensenumber{2}\definition{to rot, go bad}\example{Kumddäga ebdo ddɨgatt alle up a mɨnyi bozowän.}{After three days, the banana will rot.}\example{Lläkäm a mamall zozoang dan.}{The mushroom goes bad quickly.}\allomorph{zo}}
\entry{Zudas}{\headword{Zudas}\pos{pn.}\sensenumber{2}\definition{male personal name}}
\entry{Zudiya}{\headword{Zudiya}\pos{pn.}\sensenumber{2}\definition{Judea}}
\entry{Zugu}{\headword{Zugu}\pos{pn.}\sensenumber{2}\definition{male personal name}}
\entry{Zuli}{\headword{Zuli}\pos{pn.}\sensenumber{2}\definition{female personal name}}
\entry{Zurusalem}{\headword{Zurusalem}\pos{pn.}\sensenumber{2}\definition{Jerusalem}}
\entry{zuwoe}{\headword{zuwoe}\pos{S vt.}\sensenumber{1}\definition{to shoot}\example{Bodog käbama dallän a sɨmell gullbe de dazuwän.}{Bodog went hunting and shot a boar.}\sensenumber{2}\definition{to pierce; to inject, administer a shot}\example{Baba bom sana täkäll da nazunan nying me.}{The sago thorn pierced Dad in the leg.}\example{Ngänäm doktor da pam alle nazuwan.}{The doctor gave me an injection.}\allomorph{zu}\allomorph{zuwae}}
\lettersection{C}
\entry{Caso}{\headword{Caso}\pos{pn.}\sensenumber{2}\definition{male personal name}}
\entry{Cathy}{\headword{Cathy}\pos{pn.}\sensenumber{2}\definition{female personal name}}
\entry{Charles}{\headword{Charles}\pos{pn.}\sensenumber{2}\definition{male personal name}}
\entry{Christina}{\headword{Christina}\pos{pn.}\sensenumber{2}\definition{female personal name}}
\entry{Cynthia}{\headword{Cynthia}\variant{sp. var. of}{Sintia}}
\lettersection{F}
\entry{faeb}{\headword{faeb}\variant{sp. var. of}{paeb}}
\entry{fama}{\headword{fama}\variant{sp. var. of}{pama}}
\entry{family}{\headword{family}\variant{sp. var. of}{pemli}}
\entry{Feliks}{\headword{Feliks}\variant{sp. var. of}{Felix}}
\entry{Felix}{\headword{Felix}\pos{pn.}\sensenumber{2}\definition{male personal name}}
\entry{femli}{\headword{femli}\variant{sp. var. of}{pemli}}
\entry{fifti}{\headword{fifti}\variant{sp. var. of}{pipti}}
\entry{fiftin}{\headword{fiftin}\variant{sp. var. of}{piptin}}
\entry{five}{\headword{five}\variant{sp. var. of}{paeb}}
\entry{Flora}{\headword{Flora}\pos{pn.}\sensenumber{2}\definition{female personal name}}
\entry{fo}{\headword{fo}\variant{sp. var. of}{po2}}
\entry{forti}{\headword{forti}\variant{sp. var. of}{poti}}
\entry{forty}{\headword{forty}\variant{sp. var. of}{poti}}
\entry{foti}{\headword{foti}\variant{sp. var. of}{poti}}
\entry{four}{\headword{four}\variant{sp. var. of}{po2}}
\entry{Frank}{\headword{Frank}\pos{pn.}\sensenumber{2}\definition{male personal name}}
\entry{Francis}{\headword{Francis}\pos{pn.}\sensenumber{2}\definition{male personal name}}
\lettersection{J}
\entry{Jae}{\headword{Jae}\pos{pn.}\sensenumber{2}\definition{male personal name}}
\entry{James}{\headword{James}\variant{sp. var. of}{Zeims}}
\entry{Jamila}{\headword{Jamila}\variant{sp. var. of}{Jamilah}}
\entry{Jamilah}{\headword{Jamilah}\pos{pn.}\sensenumber{2}\definition{female personal name}}
\entry{Jane}{\headword{Jane}\pos{pn.}\sensenumber{2}\definition{female personal name}}
\entry{Jack}{\headword{Jack}\pos{pn.}\sensenumber{2}\definition{male personal name}}
\entry{Jackae}{\headword{Jackae}\pos{pn.}\sensenumber{2}\definition{male personal name}}
\entry{Jacklin}{\headword{Jacklin}\pos{pn.}\sensenumber{2}\definition{female personal name}}
\entry{Jeiks}{\headword{Jeiks}\pos{pn.}\sensenumber{2}\definition{male personal name}}
\entry{Jeimi}{\headword{Jeimi}\pos{pn.}\sensenumber{2}\definition{male personal name}}
\entry{Jemila}{\headword{Jemila}\pos{pn.}\sensenumber{2}\definition{female personal name}}
\entry{Jen}{\headword{Jen}\pos{pn.}\sensenumber{2}\definition{male personal name}}
\entry{Jeped}{\headword{Jeped}\pos{pn.}\sensenumber{2}\definition{male personal name}}
\entry{Jerry}{\headword{Jerry}\pos{pn.}\sensenumber{2}\definition{male personal name}}
\entry{Jeff}{\headword{Jeff}\pos{pn.}\sensenumber{2}\definition{male personal name}}
\entry{Jo}{\headword{Jo}\pos{pn.}\sensenumber{2}\definition{male personal name}}
\entry{Joanna}{\headword{Joanna}\pos{pn.}\sensenumber{2}\definition{female personal name}}
\entry{Joden}{\headword{Joden}\pos{pn.}\sensenumber{2}\definition{male personal name}}
\entry{Joe}{\headword{Joe}\variant{sp. var. of}{Zo}}
\entry{Joe-noh}{\headword{Joe-noh}\pos{pn.}\sensenumber{2}\definition{male personal name}}
\entry{Joebert}{\headword{Joebert}\pos{pn.}\sensenumber{2}\definition{male personal name}}
\entry{John}{\headword{John}\pos{pn.}\sensenumber{2}\definition{male personal name}}
\entry{Johnny}{\headword{Johnny}\pos{pn.}\sensenumber{2}\definition{male personal name}}
\entry{Jonas}{\headword{Jonas}\variant{sp. var. of}{Zonas}}
\entry{Jonathan}{\headword{Jonathan}\pos{pn.}\sensenumber{2}\definition{male personal name}}
\entry{Joni}{\headword{Joni}\variant{sp. var. of}{Johnny}}
\entry{Jordan}{\headword{Jordan}\pos{pn.}\sensenumber{2}\definition{male personal name}}
\entry{Joseph}{\headword{Joseph}\variant{sp. var. of}{Zosep}}
\entry{Joshua}{\headword{Joshua}\pos{pn.}\sensenumber{2}\definition{male personal name}}
\entry{Jowanang}{\headword{Jowanang}\pos{pn.}\sensenumber{2}\definition{male personal name}}
\entry{Joy-Lin}{\headword{Joy-Lin}\pos{pn.}\sensenumber{2}\definition{female personal name}}
\entry{Joys}{\headword{Joys}\pos{pn.}\sensenumber{2}\definition{female personal name}}
\entry{Jubli}{\headword{Jubli}\pos{pn.}\sensenumber{2}\definition{male personal name}}
\entry{Judas}{\headword{Judas}\variant{sp. var. of}{Zudas}}
\entry{Jugu}{\headword{Jugu}\pos{pn.}\sensenumber{2}\definition{male personal name}}
\entry{Julia}{\headword{Julia}\pos{pn.}\sensenumber{2}\definition{female personal name}}
\entry{Julie}{\headword{Julie}\variant{sp. var. of}{Zuli}}
\entry{Julienne}{\headword{Julienne}\pos{pn.}\sensenumber{2}\definition{female personal name}}
\entry{Junior}{\headword{Junior}\pos{pn.}\sensenumber{2}\definition{male personal name}}
\lettersection{Q}
\entry{Queenie}{\headword{Queenie}\pos{pn.}\sensenumber{2}\definition{female personal name}}
\entry{Quin}{\headword{Quin}\pos{pn.}\sensenumber{2}\definition{male personal name}}
\entry{Quinten}{\headword{Quinten}\pos{pn.}\sensenumber{2}\definition{male personal name}}
\entry{Quinteth}{\headword{Quinteth}\pos{pn.}\sensenumber{2}\definition{female personal name}}
\entry{Quinton}{\headword{Quinton}\pos{pn.}\sensenumber{2}\definition{male personal name}}
\lettersection{V}
\entry{Vanessa}{\headword{Vanessa}\pos{pn.}\sensenumber{2}\definition{female personal name}}
\entry{Vincent}{\headword{Vincent}\pos{pn.}\sensenumber{2}\definition{male personal name}}
\entry{Victoria}{\headword{Victoria}\pos{pn.}\sensenumber{2}\definition{female personal name}}
\end{document}