
\section{A vi.}
\begin{enumerate}
\item \entry{A vi.}{\headword{aläm} \definition{1. to be wary}}
\item \entry{A vi.}{\headword{amamärang} \definition{1. to echo}}
\item \entry{A vi.}{\headword{ambag} \definition{1. to cause trouble}}
\item \entry{A vi.}{\headword{amtet} \definition{1. to breathe}}
\item \entry{A vi.}{\headword{ansi} \definition{1. to sneeze}}
\item \entry{A vi.}{\headword{ballbell} \definition{1. to cook improperly}}
\item \entry{A vi.}{\headword{baottbaott} \definition{1. to walk around aimlessly}}
\item \entry{A vi.}{\headword{bib1} \definition{1. to break ground, surface}}
\item \entry{A vi.}{\headword{daem} \definition{1. to blink}}
\item \entry{A vi.}{\headword{daudau} \definition{1. to nod}}
\item \entry{A vi.}{\headword{där} \definition{1. to match, balance, agree}}
\item \entry{A vi.}{\headword{gänggllam} \definition{1. to snore}}
\item \entry{A vi.}{\headword{kanyekanye} \definition{1. to move around, travel}}
\item \entry{A vi.}{\headword{kanyekanye} \definition{1. to go hunting}}
\item \entry{A vi.}{\headword{kawa} \definition{1. to preach}}
\item \entry{A vi.}{\headword{käza wirwir2} \definition{1. to call a crocodile out from the water}}
\item \entry{A vi.}{\headword{kuddäll} \definition{1. to die}}
\item \entry{A vi.}{\headword{lus} \definition{1. lose}}
\item \entry{A vi.}{\headword{llätt} \definition{1. to stop, end, finish}}
\item \entry{A vi.}{\headword{mam} \definition{1. to bleed}}
\item \entry{A vi.}{\headword{mämrem} \definition{1. to growl}}
\item \entry{A vi.}{\headword{mängall} \definition{1. to struggle}}
\item \entry{A vi.}{\headword{märal} \definition{1. to curse, swear}}
\item \entry{A vi.}{\headword{ngal} \definition{1. to waddle, shuffle}}
\item \entry{A vi.}{\headword{ngänaurur} \definition{1. to mourn}}
\item \entry{A vi.}{\headword{olle} \definition{1. to shout, yell, call out}}
\item \entry{A vi.}{\headword{pänyik} \definition{1. to whisper}}
\item \entry{A vi.}{\headword{po3} \definition{1. to block an animal from escaping}}
\item \entry{A vi.}{\headword{sipel} \definition{1. to rest}}
\item \entry{A vi.}{\headword{tuyem} \definition{1. to make a loud noise}}
\item \entry{A vi.}{\headword{ttänttäm} \definition{1. to burn}}
\item \entry{A vi.}{\headword{ttogottogo} \definition{1. to become blunt}}
\item \entry{A vi.}{\headword{uk} \definition{1. to shout, yell, cry out}}
\item \entry{A vi.}{\headword{ulle} \definition{1. to become big, grow up}}
\item \entry{A vi.}{\headword{umllang} \definition{1. to know; come to know, learn}}
\item \entry{A vi.}{\headword{wawaem} \definition{1. to flow}}
\item \entry{A vi.}{\headword{zämllall} \definition{1. to pass by}}
\item \entry{A vi.}{\headword{zämllall} \definition{1. passerby}}
\item \entry{A vi.}{\headword{zämllall} \definition{1. nonstop}}
\end{enumerate}

\section{A vi. & vt.}
\begin{enumerate}
\item \entry{A vi. & vt.}{\headword{allka} \definition{1. to shout (at)}}
\item \entry{A vi. & vt.}{\headword{bombllo} \definition{1. to increase, proliferate, multiply}}
\item \entry{A vi. & vt.}{\headword{duwem} \definition{1. to eat}}
\item \entry{A vi. & vt.}{\headword{ddoddollem} \definition{1. to make noise}}
\item \entry{A vi. & vt.}{\headword{eka} \definition{1. to say, speak, talk, tell}}
\item \entry{A vi. & vt.}{\headword{erany} \definition{1. to scream, shout, yell (at)}}
\item \entry{A vi. & vt.}{\headword{ikop} \definition{1. to look}}
\item \entry{A vi. & vt.}{\headword{mondremondre} \definition{1. to move}}
\item \entry{A vi. & vt.}{\headword{ngonenngonen duwem} \definition{1. to scarf down, eat very quickly}}
\item \entry{A vi. & vt.}{\headword{sens} \definition{1. to change}}
\item \entry{A vi. & vt.}{\headword{talme} \definition{1. to give birth (to)}}
\item \entry{A vi. & vt.}{\headword{tatu} \definition{1. to bathe, wash oneself; wash (an animate object)}}
\item \entry{A vi. & vt.}{\headword{tatu} \definition{1. washing place, outdoor bathing area}}
\item \entry{A vi. & vt.}{\headword{tatu} \definition{1. cleanly}}
\end{enumerate}

\section{A vt.}
\begin{enumerate}
\item \entry{A vt.}{\headword{aläm} \definition{1. to warn, caution}}
\item \entry{A vt.}{\headword{aoao} \definition{1. to chase away (e.g. humans, dogs, chickens, but not wild animals)}}
\item \entry{A vt.}{\headword{aräram} \definition{1. to sand or smooth a canoe (last step before burning)}}
\item \entry{A vt.}{\headword{au} \definition{1. to bury}}
\item \entry{A vt.}{\headword{beräberäl} \definition{1. to swing}}
\item \entry{A vt.}{\headword{bir} \definition{1. to roast}}
\item \entry{A vt.}{\headword{bittang} \definition{1. to litter, dirty}}
\item \entry{A vt.}{\headword{buddo} \definition{1. to trouble, bother}}
\item \entry{A vt.}{\headword{dom} \definition{1. to clench (one's fist)}}
\item \entry{A vt.}{\headword{druwem} \definition{1. to knock on}}
\item \entry{A vt.}{\headword{gämäll} \definition{1. to steal}}
\item \entry{A vt.}{\headword{gämäll} \definition{1. to steal}}
\item \entry{A vt.}{\headword{gul} \definition{1. to accompany}}
\item \entry{A vt.}{\headword{ikop} \definition{1. to see}}
\item \entry{A vt.}{\headword{imomdae} \definition{1. to believe}}
\item \entry{A vt.}{\headword{indrang} \definition{1. to show, reveal, illuminate, bring to light}}
\item \entry{A vt.}{\headword{kaen} \definition{1. to wrap up}}
\item \entry{A vt.}{\headword{kapu} \definition{1. to carry}}
\item \entry{A vt.}{\headword{ketrop} \definition{1. to stop the rain}}
\item \entry{A vt.}{\headword{kili} \definition{1. to praise}}
\item \entry{A vt.}{\headword{kili} \definition{1. to insult}}
\item \entry{A vt.}{\headword{kuimang} \definition{1. to knock on}}
\item \entry{A vt.}{\headword{kuki} \definition{1. to deceive, trick, lie to}}
\item \entry{A vt.}{\headword{kul} \definition{1. to smash food}}
\item \entry{A vt.}{\headword{lel} \definition{1. to fear, be afraid of}}
\item \entry{A vt.}{\headword{lel} \definition{1. to be ashamed (of)}}
\item \entry{A vt.}{\headword{lok} \definition{1. lock}}
\item \entry{A vt.}{\headword{mak} \definition{1. to choose}}
\item \entry{A vt.}{\headword{manglle} \definition{1. to lure}}
\item \entry{A vt.}{\headword{manglle} \definition{1. friendly, appealing, sociable}}
\item \entry{A vt.}{\headword{matta} \definition{1. to shoulder, carry on one's shoulder}}
\item \entry{A vt.}{\headword{mäkäp} \definition{1. to knot}}
\item \entry{A vt.}{\headword{märal} \definition{1. to curse}}
\item \entry{A vt.}{\headword{mer1} \definition{1. to bless}}
\item \entry{A vt.}{\headword{met} \definition{1. to sharpen}}
\item \entry{A vt.}{\headword{miks} \definition{1. to mix}}
\item \entry{A vt.}{\headword{mikutt} \definition{1. to hate}}
\item \entry{A vt.}{\headword{mipdab} \definition{1. to accommodate; offer food to a visitor}}
\item \entry{A vt.}{\headword{mitmit1} \definition{1. to miss, long for}}
\item \entry{A vt.}{\headword{mur} \definition{1. to clean, tidy}}
\item \entry{A vt.}{\headword{ngänaeka} \definition{1. to cry (about, for)}}
\item \entry{A vt.}{\headword{ngänaeka} \definition{1. tear}}
\item \entry{A vt.}{\headword{ngänaeka} \definition{1. crying, in tears}}
\item \entry{A vt.}{\headword{ngänaeka} \definition{1. to burst into tears}}
\item \entry{A vt.}{\headword{ngänaeka} \definition{1. nonsingular form of ngänaeka}}
\item \entry{A vt.}{\headword{ngoi1} \definition{1. to crowd, surround}}
\item \entry{A vt.}{\headword{okokol} \definition{1. to welcome}}
\item \entry{A vt.}{\headword{olle} \definition{1. to call (over), summon}}
\item \entry{A vt.}{\headword{panis} \definition{1. to punish}}
\item \entry{A vt.}{\headword{papa} \definition{1. to hit, beat}}
\item \entry{A vt.}{\headword{paya} \definition{1. to shoot, fire}}
\item \entry{A vt.}{\headword{redi} \definition{1. to prepare, make ready}}
\item \entry{A vt.}{\headword{sänge} \definition{1. to betray}}
\item \entry{A vt.}{\headword{säspen} \definition{1. to boil}}
\item \entry{A vt.}{\headword{sebor} \definition{1. to greet}}
\item \entry{A vt.}{\headword{sebor} \definition{1. to welcome}}
\item \entry{A vt.}{\headword{sek} \definition{1. to check}}
\item \entry{A vt.}{\headword{sel2} \definition{1. to sell}}
\item \entry{A vt.}{\headword{sens} \definition{1. to exchange (in marriage)}}
\item \entry{A vt.}{\headword{singoll} \definition{1. to give, provide, share}}
\item \entry{A vt.}{\headword{tab} \definition{1. to promise}}
\item \entry{A vt.}{\headword{tabe} \definition{1. to pad your back from the spalek}}
\item \entry{A vt.}{\headword{tap1} \definition{1. to dock, land}}
\item \entry{A vt.}{\headword{tap1} \definition{1. dock, wharf}}
\item \entry{A vt.}{\headword{tära} \definition{1. to finish}}
\item \entry{A vt.}{\headword{täräll} \definition{1. to stick out}}
\item \entry{A vt.}{\headword{täräp2} \definition{1. to portion, share, split, divide}}
\item \entry{A vt.}{\headword{tärpa} \definition{1. to overcook, burn}}
\item \entry{A vt.}{\headword{tok} \definition{1. to wrap}}
\item \entry{A vt.}{\headword{tukpi} \definition{1. to heap, pile; gather, collect}}
\item \entry{A vt.}{\headword{ttam1} \definition{1. to save someone's life}}
\item \entry{A vt.}{\headword{ttoe} \definition{1. to cover with skin}}
\item \entry{A vt.}{\headword{ulle} \definition{1. to raise, rear}}
\item \entry{A vt.}{\headword{umllang} \definition{1. to tell, inform}}
\item \entry{A vt.}{\headword{yagäl} \definition{1. to smooth with the yagäl leaf}}
\item \entry{A vt.}{\headword{yu} \definition{1. to cook over fire}}
\item \entry{A vt.}{\headword{yus} \definition{1. to use}}
\end{enumerate}

\section{S vd.}
\begin{enumerate}
\item \entry{S vd.}{\headword{dämoe} \definition{1. to send}}
\item \entry{S vd.}{\headword{ttongg} \definition{1. to give}}
\item \entry{S vd.}{\headword{ttongg} \definition{1. selfish, greedy}}
\item \entry{S vd.}{\headword{ttongg} \definition{1. generous, giving}}
\item \entry{S vd.}{\headword{ttongg} \definition{1. generous, giving}}
\end{enumerate}

\section{S vi.}
\begin{enumerate}
\item \entry{S vi.}{\headword{baddbedd} \definition{1. to dry up}}
\item \entry{S vi.}{\headword{bädab} \definition{1. to shine brightly (of the moon)}}
\item \entry{S vi.}{\headword{bädab} \definition{1. to dawn, break}}
\item \entry{S vi.}{\headword{bällängg} \definition{1. to split up, separate}}
\item \entry{S vi.}{\headword{bällge} \definition{1. to spread, scatter, move away}}
\item \entry{S vi.}{\headword{bebe2} \definition{1. to leak, excrete}}
\item \entry{S vi.}{\headword{benmäll} \definition{1. to shine, flash}}
\item \entry{S vi.}{\headword{blab} \definition{1. to mature, reach puberty}}
\item \entry{S vi.}{\headword{bobllem} \definition{1. to flap, shake (of skin)}}
\item \entry{S vi.}{\headword{bombllo} \definition{1. to increase, proliferate, multiply}}
\item \entry{S vi.}{\headword{dadäräd} \definition{1. to clear (e.g. of the mind, sky)}}
\item \entry{S vi.}{\headword{dagal} \definition{1. to board, get on}}
\item \entry{S vi.}{\headword{dangg} \definition{1. to burn}}
\item \entry{S vi.}{\headword{dangkam} \definition{1. to lean}}
\item \entry{S vi.}{\headword{dangkam} \definition{1. to rely on}}
\item \entry{S vi.}{\headword{däbae} \definition{1. to drizzle}}
\item \entry{S vi.}{\headword{däbae} \definition{1. to knock fruit continuously}}
\item \entry{S vi.}{\headword{dämbaemeny} \definition{1. to doze off}}
\item \entry{S vi.}{\headword{dämen} \definition{1. to sit}}
\item \entry{S vi.}{\headword{dämen} \definition{1. causative-applicative form of dämen}}
\item \entry{S vi.}{\headword{dämen} \definition{1. applicative form of dämen}}
\item \entry{S vi.}{\headword{dändär1} \definition{1. listen}}
\item \entry{S vi.}{\headword{dedre} \definition{1. to descend, go down}}
\item \entry{S vi.}{\headword{dindu} \definition{1. to run, flee, escape}}
\item \entry{S vi.}{\headword{dodro1} \definition{1. to slip}}
\item \entry{S vi.}{\headword{dodro1} \definition{1. to die}}
\item \entry{S vi.}{\headword{dongkal} \definition{1. to stop, end}}
\item \entry{S vi.}{\headword{ddäddäg4} \definition{1. to look, watch}}
\item \entry{S vi.}{\headword{ddägaddäge} \definition{1. to branch out}}
\item \entry{S vi.}{\headword{ddäll2} \definition{1. to arrive}}
\item \entry{S vi.}{\headword{ddälläb} \definition{1. to fall over (of a tree)}}
\item \entry{S vi.}{\headword{ddänddäl} \definition{1. to climb}}
\item \entry{S vi.}{\headword{ddänddäm} \definition{1. to drown}}
\item \entry{S vi.}{\headword{ddänddäm} \definition{1. to set (of the moon)}}
\item \entry{S vi.}{\headword{ddänddäm} \definition{1. sink}}
\item \entry{S vi.}{\headword{ddogoll} \definition{1. to stick}}
\item \entry{S vi.}{\headword{ddogoll} \definition{1. to join}}
\item \entry{S vi.}{\headword{ddogoll} \definition{1. to lean}}
\item \entry{S vi.}{\headword{ddonddo} \definition{1. to boast, brag}}
\item \entry{S vi.}{\headword{eka} \definition{1. to discuss, converse}}
\item \entry{S vi.}{\headword{ergod} \definition{1. to crawl}}
\item \entry{S vi.}{\headword{ergod} \definition{1. toddler}}
\item \entry{S vi.}{\headword{gambän} \definition{1. to stand}}
\item \entry{S vi.}{\headword{gambän} \definition{1. causative-applicative form of gambän}}
\item \entry{S vi.}{\headword{gazen} \definition{1. to come out, get out, exit, escape}}
\item \entry{S vi.}{\headword{gazen} \definition{1. to rise, come out; shine}}
\item \entry{S vi.}{\headword{gäbaeb} \definition{1. to jump}}
\item \entry{S vi.}{\headword{gäbän} \definition{1. to jump, hop}}
\item \entry{S vi.}{\headword{gägabäll} \definition{1. to stand}}
\item \entry{S vi.}{\headword{gällbän} \definition{1. to break}}
\item \entry{S vi.}{\headword{giddoll} \definition{1. to live, reside}}
\item \entry{S vi.}{\headword{giddoll} \definition{1. to stay, remain}}
\item \entry{S vi.}{\headword{gllae1} \definition{1. to float, drift; paddle}}
\item \entry{S vi.}{\headword{gllae2} \definition{1. to shine}}
\item \entry{S vi.}{\headword{gllangglla} \definition{1. to swim}}
\item \entry{S vi.}{\headword{gollab} \definition{1. flow}}
\item \entry{S vi.}{\headword{gongg} \definition{1. to disturb a bees nest and then to feel the bites}}
\item \entry{S vi.}{\headword{gugi} \definition{1. to stand}}
\item \entry{S vi.}{\headword{ibi2} \definition{1. to go}}
\item \entry{S vi.}{\headword{ibi2} \definition{1. to walk}}
\item \entry{S vi.}{\headword{ikop} \definition{1. to look, watch}}
\item \entry{S vi.}{\headword{inttemängg} \definition{1. to part ways, leave}}
\item \entry{S vi.}{\headword{inu} \definition{1. to sleep}}
\item \entry{S vi.}{\headword{inggoemeny} \definition{1. to joke}}
\item \entry{S vi.}{\headword{inggoemeny} \definition{1. to blaspheme; insult, mock}}
\item \entry{S vi.}{\headword{inggol} \definition{1. to move}}
\item \entry{S vi.}{\headword{ingong} \definition{1. to dance}}
\item \entry{S vi.}{\headword{irängän} \definition{1. to come out}}
\item \entry{S vi.}{\headword{itrell} \definition{1. to get hurt}}
\item \entry{S vi.}{\headword{ittall} \definition{1. to follow closely, stalk (+ peyang)}}
\item \entry{S vi.}{\headword{kakal} \definition{1. to enter, board, go in}}
\item \entry{S vi.}{\headword{kalla} \definition{1. to lie down}}
\item \entry{S vi.}{\headword{kam1} \definition{1. to start, begin}}
\item \entry{S vi.}{\headword{kambäg} \definition{1. to escape, flee}}
\item \entry{S vi.}{\headword{kanyam} \definition{1. bend}}
\item \entry{S vi.}{\headword{käbab} \definition{1. to stop}}
\item \entry{S vi.}{\headword{käklläp} \definition{1. to sting, be painful}}
\item \entry{S vi.}{\headword{käklläp} \definition{1. to be amused}}
\item \entry{S vi.}{\headword{kälakäle} \definition{1. to set (of the sun)}}
\item \entry{S vi.}{\headword{källmakällme} \definition{1. to survive}}
\item \entry{S vi.}{\headword{kämbäg} \definition{1. to dive}}
\item \entry{S vi.}{\headword{kän2} \definition{1. to withdraw, come out}}
\item \entry{S vi.}{\headword{känz} \definition{1. to go aside, go off course}}
\item \entry{S vi.}{\headword{kängkäl} \definition{1. to ascend, climb, go up, rise}}
\item \entry{S vi.}{\headword{kätam} \definition{1. to splash (of an aquatic animal)}}
\item \entry{S vi.}{\headword{kätam} \definition{1. to flip, do a cartwheel}}
\item \entry{S vi.}{\headword{klloklloe} \definition{1. to gather, join, come together}}
\item \entry{S vi.}{\headword{koen} \definition{1. to turn back}}
\item \entry{S vi.}{\headword{kunen} \definition{1. to flee, scatter, run away}}
\item \entry{S vi.}{\headword{laem} \definition{1. to argue}}
\item \entry{S vi.}{\headword{läpon} \definition{1. to be amazed, be in awe}}
\item \entry{S vi.}{\headword{liglig} \definition{1. to have sex, copulate}}
\item \entry{S vi.}{\headword{liglig} \definition{1. having a lot of sex}}
\item \entry{S vi.}{\headword{liglig} \definition{1. sexual relations}}
\item \entry{S vi.}{\headword{lɨklɨk} \definition{1. to melt}}
\item \entry{S vi.}{\headword{lläb} \definition{1. to go under}}
\item \entry{S vi.}{\headword{llädäd} \definition{1. to come, arrive (figuratively)}}
\item \entry{S vi.}{\headword{llɨtt1} \definition{1. to get worse, worsen}}
\item \entry{S vi.}{\headword{llollom} \definition{1. to break, be damaged}}
\item \entry{S vi.}{\headword{mame} \definition{1. to fall, rain}}
\item \entry{S vi.}{\headword{mällkae} \definition{1. to bend}}
\item \entry{S vi.}{\headword{mällkam} \definition{1. to bend}}
\item \entry{S vi.}{\headword{mällkam} \definition{1. bending over}}
\item \entry{S vi.}{\headword{mänddmändd} \definition{1. to drown, struggle in water}}
\item \entry{S vi.}{\headword{mänddmändd} \definition{1. sink}}
\item \entry{S vi.}{\headword{mättmätt} \definition{1. to wear, dress oneself, put on}}
\item \entry{S vi.}{\headword{menttäg} \definition{1. to shoulder, carry one one's shoulders or head}}
\item \entry{S vi.}{\headword{nganae} \definition{1. to spin, rotate}}
\item \entry{S vi.}{\headword{ngasnges} \definition{1. to happen}}
\item \entry{S vi.}{\headword{ngädngäd} \definition{1. to roll up, curl}}
\item \entry{S vi.}{\headword{ngällae} \definition{1. to look back}}
\item \entry{S vi.}{\headword{ngällbän} \definition{1. to rise, arise, come up}}
\item \entry{S vi.}{\headword{ngällbän} \definition{1. to awake, wake up, get up}}
\item \entry{S vi.}{\headword{ngämae} \definition{1. to go around, take the long way}}
\item \entry{S vi.}{\headword{ngämingg} \definition{1. to answer}}
\item \entry{S vi.}{\headword{ngänngän} \definition{1. to swear}}
\item \entry{S vi.}{\headword{ngänttäg} \definition{1. to arrive, return}}
\item \entry{S vi.}{\headword{ngänglam} \definition{1. to buzz}}
\item \entry{S vi.}{\headword{ngäs} \definition{1. to return, come back}}
\item \entry{S vi.}{\headword{ngetae} \definition{1. to touch}}
\item \entry{S vi.}{\headword{ngetae} \definition{1. to get engaged}}
\item \entry{S vi.}{\headword{ngetam} \definition{1. to come down, descend}}
\item \entry{S vi.}{\headword{ngonongg} \definition{1. to think}}
\item \entry{S vi.}{\headword{nyamäll} \definition{1. to be late}}
\item \entry{S vi.}{\headword{nyägae} \definition{1. to flail}}
\item \entry{S vi.}{\headword{nyäng2} \definition{1. to dance}}
\item \entry{S vi.}{\headword{nyärab} \definition{1. to go up, ascend}}
\item \entry{S vi.}{\headword{nyäroe} \definition{1. to creep}}
\item \entry{S vi.}{\headword{nyärpae} \definition{1. to refuse (someone in the dative)}}
\item \entry{S vi.}{\headword{otal} \definition{1. to perch}}
\item \entry{S vi.}{\headword{pameny} \definition{1. to scream}}
\item \entry{S vi.}{\headword{paplläg} \definition{1. to fly}}
\item \entry{S vi.}{\headword{pädoe} \definition{1. to blow}}
\item \entry{S vi.}{\headword{pädrall} \definition{1. to spread (out), scatter, disperse}}
\item \entry{S vi.}{\headword{päddab} \definition{1. to grow}}
\item \entry{S vi.}{\headword{päddab} \definition{1. to be cooked, be done}}
\item \entry{S vi.}{\headword{päddpädd} \definition{1. to grow}}
\item \entry{S vi.}{\headword{päddpädd} \definition{1. to explode}}
\item \entry{S vi.}{\headword{päddpädd} \definition{1. causative-applicative form of päddpädd}}
\item \entry{S vi.}{\headword{pälengg} \definition{1. to drop down, fall}}
\item \entry{S vi.}{\headword{pällttän} \definition{1. to set off, start walking}}
\item \entry{S vi.}{\headword{pänae} \definition{1. to turn back, turn around}}
\item \entry{S vi.}{\headword{pänae} \definition{1. to capsize, turn over, flip over}}
\item \entry{S vi.}{\headword{pänae} \definition{1. to turn, become, transform}}
\item \entry{S vi.}{\headword{pänddäg1} \definition{1. to start a grassfire}}
\item \entry{S vi.}{\headword{pänddäg1} \definition{1. to hatch}}
\item \entry{S vi.}{\headword{pänddäg1} \definition{1. to break water}}
\item \entry{S vi.}{\headword{pänddäg1} \definition{1. to mature (of a plant or a person)}}
\item \entry{S vi.}{\headword{pänongg} \definition{1. to awake, wake up, get up}}
\item \entry{S vi.}{\headword{pängän} \definition{1. to disappear}}
\item \entry{S vi.}{\headword{pänyae} \definition{1. to hop}}
\item \entry{S vi.}{\headword{pendäg} \definition{1. to trip}}
\item \entry{S vi.}{\headword{penongg} \definition{1. to burn, set on fire, ignite}}
\item \entry{S vi.}{\headword{pentae} \definition{1. to spread, be transmitted}}
\item \entry{S vi.}{\headword{pentngeny} \definition{1. to trip}}
\item \entry{S vi.}{\headword{peraingg} \definition{1. to cross, intersect}}
\item \entry{S vi.}{\headword{peraingg} \definition{1. to protrude, stick out}}
\item \entry{S vi.}{\headword{peyam} \definition{1. to come out from water, surface}}
\item \entry{S vi.}{\headword{peyam} \definition{1. to pop up, reappear, resurface}}
\item \entry{S vi.}{\headword{pinsäg} \definition{1. to tear}}
\item \entry{S vi.}{\headword{pipllug} \definition{1. to fly}}
\item \entry{S vi.}{\headword{pisam} \definition{1. to tear (of something thin)}}
\item \entry{S vi.}{\headword{plengg} \definition{1. to die}}
\item \entry{S vi.}{\headword{ponor} \definition{1. to start running}}
\item \entry{S vi.}{\headword{popllem} \definition{1. to flap}}
\item \entry{S vi.}{\headword{pungg2} \definition{1. to get sick, be in pain}}
\item \entry{S vi.}{\headword{sämongg} \definition{1. to feel}}
\item \entry{S vi.}{\headword{särämbae} \definition{1. to prepare, be prepared, be ready, get ready}}
\item \entry{S vi.}{\headword{säsäs} \definition{1. to rub}}
\item \entry{S vi.}{\headword{sllollongg} \definition{1. to sit close together}}
\item \entry{S vi.}{\headword{spun2} \definition{1. to fall; set}}
\item \entry{S vi.}{\headword{spun2} \definition{1. to jump}}
\item \entry{S vi.}{\headword{tameny} \definition{1. to dicuss, converse}}
\item \entry{S vi.}{\headword{täbatäbe} \definition{1. to plan}}
\item \entry{S vi.}{\headword{tälpe} \definition{1. to volunteer}}
\item \entry{S vi.}{\headword{tärak} \definition{1. to go inside}}
\item \entry{S vi.}{\headword{tätäräp1} \definition{1. to cut across, take a shortcut}}
\item \entry{S vi.}{\headword{täträk} \definition{1. to go underneath}}
\item \entry{S vi.}{\headword{togol} \definition{1. to hide; go away to have sex secretly}}
\item \entry{S vi.}{\headword{tomon} \definition{1. to wait}}
\item \entry{S vi.}{\headword{tongoe} \definition{1. to play}}
\item \entry{S vi.}{\headword{torwam} \definition{1. to lie down}}
\item \entry{S vi.}{\headword{ttalam} \definition{1. to split, crack}}
\item \entry{S vi.}{\headword{ttalängg} \definition{1. to aim}}
\item \entry{S vi.}{\headword{ttam2} \definition{1. to confess}}
\item \entry{S vi.}{\headword{ttam3} \definition{1. to appear}}
\item \entry{S vi.}{\headword{ttamän} \definition{1. to finish, end}}
\item \entry{S vi.}{\headword{ttanttem} \definition{1. to be confused}}
\item \entry{S vi.}{\headword{ttattlläb} \definition{1. to open}}
\item \entry{S vi.}{\headword{ttämattäme} \definition{1. to be crowded}}
\item \entry{S vi.}{\headword{ttäpen} \definition{1. to snap, break}}
\item \entry{S vi.}{\headword{ttoengg} \definition{1. to split up, scatter}}
\item \entry{S vi.}{\headword{udab} \definition{1. to disappear, get lost, go missing}}
\item \entry{S vi.}{\headword{umaem} \definition{1. to gather}}
\item \entry{S vi.}{\headword{uta} \definition{1. go (imperative, singular form)}}
\item \entry{S vi.}{\headword{uta} \definition{1. plural form of uta}}
\item \entry{S vi.}{\headword{uziz} \definition{1. to run}}
\item \entry{S vi.}{\headword{wamän} \definition{1. to go out, dissipate, extinguish}}
\item \entry{S vi.}{\headword{wanseg} \definition{1. to be left (in a state)}}
\item \entry{S vi.}{\headword{wanwen} \definition{1. to shake, swing}}
\item \entry{S vi.}{\headword{wi1} \definition{1. to settle}}
\item \entry{S vi.}{\headword{wiya} \definition{1. come (imperative, singular form)}}
\item \entry{S vi.}{\headword{wiya} \definition{1. plural form of wiya}}
\item \entry{S vi.}{\headword{yattän} \definition{1. to disembark, get off, get out}}
\item \entry{S vi.}{\headword{zan} \definition{1. to enter, go in, go into}}
\item \entry{S vi.}{\headword{zanzi} \definition{1. to settle}}
\item \entry{S vi.}{\headword{zanggae} \definition{1. to roam, go around}}
\item \entry{S vi.}{\headword{zaze} \definition{1. to give birth, lay}}
\item \entry{S vi.}{\headword{zäm} \definition{1. to pass through}}
\item \entry{S vi.}{\headword{zeg} \definition{1. to be born}}
\item \entry{S vi.}{\headword{zozo} \definition{1. to rot, go bad}}
\end{enumerate}

\section{S vt.}
\begin{enumerate}
\item \entry{S vt.}{\headword{ballɨngg} \definition{1. to welcome, greet}}
\item \entry{S vt.}{\headword{ballɨngg} \definition{1. to predict}}
\item \entry{S vt.}{\headword{bäbälläd} \definition{1. to drop without warning}}
\item \entry{S vt.}{\headword{bälämbäl} \definition{1. to miss, long for}}
\item \entry{S vt.}{\headword{bälämbäl} \definition{1. to remember, think of}}
\item \entry{S vt.}{\headword{bällabälle} \definition{1. to find}}
\item \entry{S vt.}{\headword{bällgab} \definition{1. to open (e.g. eyes, flowers)}}
\item \entry{S vt.}{\headword{bänamb} \definition{1. to open (something folded, e.g. book, mouth)}}
\item \entry{S vt.}{\headword{bändaeg} \definition{1. to pull an all-nighter (stay awake)}}
\item \entry{S vt.}{\headword{bänybäny} \definition{1. to cut, slice (flesh)}}
\item \entry{S vt.}{\headword{bendoe} \definition{1. to confuse, mix up}}
\item \entry{S vt.}{\headword{bengae} \definition{1. to roof, cover}}
\item \entry{S vt.}{\headword{beyambäg} \definition{1. to chase}}
\item \entry{S vt.}{\headword{binzeg} \definition{1. to heat, warm}}
\item \entry{S vt.}{\headword{blläg} \definition{1. to serve}}
\item \entry{S vt.}{\headword{bollboll} \definition{1. to open sago}}
\item \entry{S vt.}{\headword{bungg} \definition{1. to ambush}}
\item \entry{S vt.}{\headword{dadäräb} \definition{1. to write}}
\item \entry{S vt.}{\headword{dadäräb} \definition{1. to dress}}
\item \entry{S vt.}{\headword{dadäräb} \definition{1. to decorate}}
\item \entry{S vt.}{\headword{dagal} \definition{1. to board, load, put on}}
\item \entry{S vt.}{\headword{dalab} \definition{1. to open, pierce, make a hole}}
\item \entry{S vt.}{\headword{däba3} \definition{1. to place, put}}
\item \entry{S vt.}{\headword{däbäll} \definition{1. to touch}}
\item \entry{S vt.}{\headword{dädäräb} \definition{1. to cut grass}}
\item \entry{S vt.}{\headword{dämoe} \definition{1. to push}}
\item \entry{S vt.}{\headword{dämoe} \definition{1. causative-applicative form of dämoe}}
\item \entry{S vt.}{\headword{dändär1} \definition{1. to hear, listen}}
\item \entry{S vt.}{\headword{dändär1} \definition{1. to sense, feel}}
\item \entry{S vt.}{\headword{dändär2} \definition{1. to stuff}}
\item \entry{S vt.}{\headword{dändär2} \definition{1. bagful of sago}}
\item \entry{S vt.}{\headword{dändärek} \definition{1. to control, influence, rule, govern}}
\item \entry{S vt.}{\headword{därängg} \definition{1. to lead}}
\item \entry{S vt.}{\headword{derägmäll} \definition{1. to rebuke, scold}}
\item \entry{S vt.}{\headword{dodro2} \definition{1. to clean}}
\item \entry{S vt.}{\headword{dradre2} \definition{1. to dress}}
\item \entry{S vt.}{\headword{duab} \definition{1. to knock over; blow down}}
\item \entry{S vt.}{\headword{ddaddällɨg} \definition{1. to destroy}}
\item \entry{S vt.}{\headword{ddaddu} \definition{1. to remove a shoot}}
\item \entry{S vt.}{\headword{ddaebän} \definition{1. to divorce}}
\item \entry{S vt.}{\headword{ddaebän} \definition{1. to adopt}}
\item \entry{S vt.}{\headword{ddangoe} \definition{1. to force to do}}
\item \entry{S vt.}{\headword{ddäddäg1} \definition{1. to eat meat; bite}}
\item \entry{S vt.}{\headword{ddäddäg2} \definition{1. to peel, remove}}
\item \entry{S vt.}{\headword{ddäddäg3} \definition{1. to pain, ache, hurt}}
\item \entry{S vt.}{\headword{ddäddäg4} \definition{1. to look, watch}}
\item \entry{S vt.}{\headword{ddäddäg5} \definition{1. to bind}}
\item \entry{S vt.}{\headword{ddäddäl} \definition{1. to shove}}
\item \entry{S vt.}{\headword{ddägaddäge} \definition{1. to write}}
\item \entry{S vt.}{\headword{ddällgoe} \definition{1. to go through thick bush in search of animals}}
\item \entry{S vt.}{\headword{ddällgoe} \definition{1. to disturb a beehive}}
\item \entry{S vt.}{\headword{ddällombog} \definition{1. to miss}}
\item \entry{S vt.}{\headword{ddän} \definition{1. to pick, gather, harvest}}
\item \entry{S vt.}{\headword{ddänddäm} \definition{1. to drown, sink}}
\item \entry{S vt.}{\headword{ddänddäm} \definition{1. to worry}}
\item \entry{S vt.}{\headword{ddänmäll} \definition{1. to struggle}}
\item \entry{S vt.}{\headword{ddänggab} \definition{1. to hold, grab (with one's teeth)}}
\item \entry{S vt.}{\headword{ddänggaddängge} \definition{1. to crucify}}
\item \entry{S vt.}{\headword{ddänggaddängge} \definition{1. to catch, trap}}
\item \entry{S vt.}{\headword{ddel} \definition{1. to explain}}
\item \entry{S vt.}{\headword{ddogoll} \definition{1. to put together}}
\item \entry{S vt.}{\headword{ddollombog} \definition{1. to misspeak, speak with mistakes}}
\item \entry{S vt.}{\headword{ddungg} \definition{1. to decapitate}}
\item \entry{S vt.}{\headword{eka} \definition{1. to tell; converse with, speak to}}
\item \entry{S vt.}{\headword{erängg} \definition{1. to test, try; taste}}
\item \entry{S vt.}{\headword{erär} \definition{1. to name; pass down a name}}
\item \entry{S vt.}{\headword{erär} \definition{1. to measure}}
\item \entry{S vt.}{\headword{gallab} \definition{1. to open (one's mouth)}}
\item \entry{S vt.}{\headword{gany} \definition{1. to plant, place in the ground, put in}}
\item \entry{S vt.}{\headword{gany} \definition{1. to focus, tune}}
\item \entry{S vt.}{\headword{gazen} \definition{1. to take out}}
\item \entry{S vt.}{\headword{gädagäde} \definition{1. to beat sago, pound sago}}
\item \entry{S vt.}{\headword{gädagäde} \definition{1. to sharpen}}
\item \entry{S vt.}{\headword{gädagäde} \definition{1. to throw a tantrum}}
\item \entry{S vt.}{\headword{gäddgädd} \definition{1. to fight}}
\item \entry{S vt.}{\headword{gäddgädd} \definition{1. to catch}}
\item \entry{S vt.}{\headword{gäddgädd} \definition{1. to cut (leaves)}}
\item \entry{S vt.}{\headword{gäglib} \definition{1. to chase, scare away/off (wild animals)}}
\item \entry{S vt.}{\headword{gäglib} \definition{1. to pluck}}
\item \entry{S vt.}{\headword{gäglläb} \definition{1. to weed for the first time (hard work to remove the large plants)}}
\item \entry{S vt.}{\headword{gälbän} \definition{1. to knock (a fruit)}}
\item \entry{S vt.}{\headword{gälbän} \definition{1. to defeather, depilate, remove hair; remove teeth}}
\item \entry{S vt.}{\headword{gämoe} \definition{1. to miss}}
\item \entry{S vt.}{\headword{gämoe} \definition{1. to miss, feel longing for}}
\item \entry{S vt.}{\headword{gänggälläm} \definition{1. to wash (a body part or inanimate object)}}
\item \entry{S vt.}{\headword{gängglläd} \definition{1. to shove, push}}
\item \entry{S vt.}{\headword{gängglläd} \definition{1. to extend (e.g. a garden)}}
\item \entry{S vt.}{\headword{gäz} \definition{1. to kill}}
\item \entry{S vt.}{\headword{gäz} \definition{1. to hit, beat}}
\item \entry{S vt.}{\headword{gɨg} \definition{1. to collect ants}}
\item \entry{S vt.}{\headword{gɨngg} \definition{1. to ambush, gang up on, attack in a large group}}
\item \entry{S vt.}{\headword{gleb} \definition{1. to take, steal}}
\item \entry{S vt.}{\headword{gllae1} \definition{1. to paddle; pedal}}
\item \entry{S vt.}{\headword{gllae1} \definition{1. to dig, spade}}
\item \entry{S vt.}{\headword{gllaglle} \definition{1. to skin, remove skin}}
\item \entry{S vt.}{\headword{gllo} \definition{1. to take out, remove}}
\item \entry{S vt.}{\headword{gogo1} \definition{1. to build}}
\item \entry{S vt.}{\headword{gollab} \definition{1. to pour}}
\item \entry{S vt.}{\headword{gollab} \definition{1. applicative form of gollab}}
\item \entry{S vt.}{\headword{gomoe} \definition{1. to make a mistake}}
\item \entry{S vt.}{\headword{gonagone} \definition{1. to cook}}
\item \entry{S vt.}{\headword{gonagone} \definition{1. to burn}}
\item \entry{S vt.}{\headword{gonagone} \definition{1. to heat}}
\item \entry{S vt.}{\headword{gungg} \definition{1. to marry a widow}}
\item \entry{S vt.}{\headword{i} \definition{1. to weave; interlock}}
\item \entry{S vt.}{\headword{i} \definition{1. plain weaving pattern}}
\item \entry{S vt.}{\headword{ibeny} \definition{1. to plant}}
\item \entry{S vt.}{\headword{idän} \definition{1. to pick, harvest; dig up, uproot}}
\item \entry{S vt.}{\headword{ikop} \definition{1. to look, watch}}
\item \entry{S vt.}{\headword{imullgoe} \definition{1. to drop, push, make fall}}
\item \entry{S vt.}{\headword{inam} \definition{1. to cover}}
\item \entry{S vt.}{\headword{inam} \definition{1. to weigh down, press down}}
\item \entry{S vt.}{\headword{indugoeg} \definition{1. to command}}
\item \entry{S vt.}{\headword{inmoll} \definition{1. to cover}}
\item \entry{S vt.}{\headword{inmoll} \definition{1. to step on; vanquish}}
\item \entry{S vt.}{\headword{inttemängg} \definition{1. to leave, see off, release, set free}}
\item \entry{S vt.}{\headword{inungoe} \definition{1. to shake (when dancing)}}
\item \entry{S vt.}{\headword{irängän} \definition{1. to get out, lift out}}
\item \entry{S vt.}{\headword{ittɨtt} \definition{1. to catch (an aquatic animal)}}
\item \entry{S vt.}{\headword{kaekep} \definition{1. to chew}}
\item \entry{S vt.}{\headword{kaekep} \definition{1. to struggle to cut (e.g. with a dull knife)}}
\item \entry{S vt.}{\headword{kalläntäg} \definition{1. to split}}
\item \entry{S vt.}{\headword{kallɨngg} \definition{1. to kill}}
\item \entry{S vt.}{\headword{kallkell} \definition{1. to deleaf a plant to reveal the shoot, or to take the skin off}}
\item \entry{S vt.}{\headword{kalltam} \definition{1. to split}}
\item \entry{S vt.}{\headword{kam1} \definition{1. to start}}
\item \entry{S vt.}{\headword{kam2} \definition{1. to cut (e.g. meat, skin)}}
\item \entry{S vt.}{\headword{kangkäg} \definition{1. to carry, bear (a load)}}
\item \entry{S vt.}{\headword{käbab} \definition{1. to stop}}
\item \entry{S vt.}{\headword{kädaeb} \definition{1. to break, split}}
\item \entry{S vt.}{\headword{kädaeb} \definition{1. to share, split, portion}}
\item \entry{S vt.}{\headword{kädaeb} \definition{1. pieces}}
\item \entry{S vt.}{\headword{kädbae} \definition{1. to test, try}}
\item \entry{S vt.}{\headword{kädkäd2} \definition{1. to remove bark, debark}}
\item \entry{S vt.}{\headword{käklläp} \definition{1. to weed for the second time (easy work to remove the remaining plants)}}
\item \entry{S vt.}{\headword{käkllätt} \definition{1. to weed roughly (leaving bits behind)}}
\item \entry{S vt.}{\headword{käkllätt} \definition{1. to fight, argue}}
\item \entry{S vt.}{\headword{kälbae} \definition{1. to singe (use brief heat to remove hair or down)}}
\item \entry{S vt.}{\headword{kälkäl} \definition{1. to lie, fabricate}}
\item \entry{S vt.}{\headword{källakälle} \definition{1. to poison the river}}
\item \entry{S vt.}{\headword{källakälle} \definition{1. to scrape food off the fire, e.g. banana, taro, yam}}
\item \entry{S vt.}{\headword{källttakälltte} \definition{1. to cast away, expel}}
\item \entry{S vt.}{\headword{käm4} \definition{1. to heal}}
\item \entry{S vt.}{\headword{kämbäg} \definition{1. to baptize}}
\item \entry{S vt.}{\headword{kän2} \definition{1. to remove, take out, take off, undo}}
\item \entry{S vt.}{\headword{kängkäl} \definition{1. to ascend, climb, go up}}
\item \entry{S vt.}{\headword{kängkäm} \definition{1. to squeeze, press}}
\item \entry{S vt.}{\headword{kättkätt1} \definition{1. to start a new weaving pattern}}
\item \entry{S vt.}{\headword{kättkätt2} \definition{1. to fence, wall, build a fence or wall}}
\item \entry{S vt.}{\headword{kɨllakɨlle} \definition{1. to scrape}}
\item \entry{S vt.}{\headword{kɨllkɨll} \definition{1. to dig}}
\item \entry{S vt.}{\headword{klloklloe} \definition{1. to gather, bring together, mix}}
\item \entry{S vt.}{\headword{koenmäll} \definition{1. to chase}}
\item \entry{S vt.}{\headword{kokllo} \definition{1. to scratch, scrape}}
\item \entry{S vt.}{\headword{koko1} \definition{1. to cut (flesh or meat), butcher}}
\item \entry{S vt.}{\headword{kokop} \definition{1. to peel, skin, husk}}
\item \entry{S vt.}{\headword{kokto} \definition{1. to bail (water)}}
\item \entry{S vt.}{\headword{kollam} \definition{1. to stand up}}
\item \entry{S vt.}{\headword{kolldän} \definition{1. to shoot; stab}}
\item \entry{S vt.}{\headword{kollmäll} \definition{1. to follow}}
\item \entry{S vt.}{\headword{kollmäll} \definition{1. follower, disciple}}
\item \entry{S vt.}{\headword{kollwany} \definition{1. to hang}}
\item \entry{S vt.}{\headword{konakone} \definition{1. to cover}}
\item \entry{S vt.}{\headword{kutt snameny} \definition{1. a funeral tradition, where you take the body to an isolated place and one person will stay with the body and a spirit will come to tell them how they died.}}
\item \entry{S vt.}{\headword{laem} \definition{1. to roll, wrap}}
\item \entry{S vt.}{\headword{lulu} \definition{1. to gossip about, tell rumors about}}
\item \entry{S vt.}{\headword{llakllek} \definition{1. to destroy}}
\item \entry{S vt.}{\headword{llanded} \definition{1. to clear}}
\item \entry{S vt.}{\headword{llanded} \definition{1. to clarify, make clear, explain}}
\item \entry{S vt.}{\headword{llatat1} \definition{1. to trace, track, follow the blood (of an animal to kill it)}}
\item \entry{S vt.}{\headword{llatat2} \definition{1. to twist, wrap}}
\item \entry{S vt.}{\headword{llädae} \definition{1. to roll}}
\item \entry{S vt.}{\headword{llädäd} \definition{1. to grab, get, catch}}
\item \entry{S vt.}{\headword{llädäd} \definition{1. to buy}}
\item \entry{S vt.}{\headword{llädäd} \definition{1. to marry}}
\item \entry{S vt.}{\headword{lläklläk} \definition{1. to spread fire}}
\item \entry{S vt.}{\headword{lläntäg} \definition{1. to tell}}
\item \entry{S vt.}{\headword{lläpän} \definition{1. to dig, harvest, unearth (a tuber or corm)}}
\item \entry{S vt.}{\headword{llätät2} \definition{1. to get rid of oven stones}}
\item \entry{S vt.}{\headword{llätmäll} \definition{1. to choose, select}}
\item \entry{S vt.}{\headword{llɨtɨt1} \definition{1. to tell, report, say}}
\item \entry{S vt.}{\headword{llɨtɨt1} \definition{1. to sing}}
\item \entry{S vt.}{\headword{llɨtɨt2} \definition{1. to butcher, cut}}
\item \entry{S vt.}{\headword{llollom} \definition{1. to break}}
\item \entry{S vt.}{\headword{lluwam} \definition{1. to incapacitate, make unable}}
\item \entry{S vt.}{\headword{malam} \definition{1. to obey, follow}}
\item \entry{S vt.}{\headword{malam} \definition{1. to accept}}
\item \entry{S vt.}{\headword{malmal} \definition{1. to mark land (for planting or settlement)}}
\item \entry{S vt.}{\headword{malmal} \definition{1. to accuse, blame}}
\item \entry{S vt.}{\headword{mamoe} \definition{1. to hunt, go hunting}}
\item \entry{S vt.}{\headword{mamon} \definition{1. to string (e.g. a bow, sago beater)}}
\item \entry{S vt.}{\headword{mamon} \definition{1. to fashion, shape, make}}
\item \entry{S vt.}{\headword{mattgal} \definition{1. to put in fire}}
\item \entry{S vt.}{\headword{mattmett1} \definition{1. to put in oven}}
\item \entry{S vt.}{\headword{mäkamäke} \definition{1. to use}}
\item \entry{S vt.}{\headword{mälamäle} \definition{1. to patch}}
\item \entry{S vt.}{\headword{mälamäle} \definition{1. to dress (a wound)}}
\item \entry{S vt.}{\headword{mälmäl} \definition{1. to squeeze}}
\item \entry{S vt.}{\headword{mälmäl} \definition{1. accuse}}
\item \entry{S vt.}{\headword{mällam2} \definition{1. to hold; get, grab, catch}}
\item \entry{S vt.}{\headword{mällamälla} \definition{1. to tie}}
\item \entry{S vt.}{\headword{mändmänd} \definition{1. to raise, rear}}
\item \entry{S vt.}{\headword{mändmänd} \definition{1. to feed}}
\item \entry{S vt.}{\headword{mänddmändd} \definition{1. to drown, force underwater}}
\item \entry{S vt.}{\headword{mättae} \definition{1. to threaten, raise fists}}
\item \entry{S vt.}{\headword{mättmätt} \definition{1. to dress}}
\item \entry{S vt.}{\headword{mena} \definition{1. to scorch}}
\item \entry{S vt.}{\headword{metmäll} \definition{1. to beat, flog, hit}}
\item \entry{S vt.}{\headword{mokon} \definition{1. to anoint}}
\item \entry{S vt.}{\headword{nane2} \definition{1. to drink}}
\item \entry{S vt.}{\headword{ngallmeny} \definition{1. to advise}}
\item \entry{S vt.}{\headword{nganae} \definition{1. to coil, go around}}
\item \entry{S vt.}{\headword{nganzig} \definition{1. to pass, overtake}}
\item \entry{S vt.}{\headword{ngange} \definition{1. to communicate, deliver a message}}
\item \entry{S vt.}{\headword{ngange} \definition{1. messenger}}
\item \entry{S vt.}{\headword{ngangem} \definition{1. to adopt}}
\item \entry{S vt.}{\headword{ngangleb} \definition{1. to look for, search for}}
\item \entry{S vt.}{\headword{ngarängg} \definition{1. to encounter, meet, run into}}
\item \entry{S vt.}{\headword{ngasnges} \definition{1. to do}}
\item \entry{S vt.}{\headword{ngasnges} \definition{1. to make}}
\item \entry{S vt.}{\headword{ngädngäd} \definition{1. to roll up, curl}}
\item \entry{S vt.}{\headword{ngädngäd} \definition{1. to fold}}
\item \entry{S vt.}{\headword{ngällbän} \definition{1. to lift}}
\item \entry{S vt.}{\headword{ngällbän} \definition{1. to get, take}}
\item \entry{S vt.}{\headword{ngällbän} \definition{1. to wake up, to wake}}
\item \entry{S vt.}{\headword{ngällngäll} \definition{1. to produce, yield, bear (fruit)}}
\item \entry{S vt.}{\headword{ngämar} \definition{1. to haul}}
\item \entry{S vt.}{\headword{ngämen} \definition{1. to reach, catch up to}}
\item \entry{S vt.}{\headword{ngämingg} \definition{1. to help}}
\item \entry{S vt.}{\headword{ngänam} \definition{1. to understand, recognize}}
\item \entry{S vt.}{\headword{ngänttäg} \definition{1. to bring, carry}}
\item \entry{S vt.}{\headword{ngänyngäny} \definition{1. to swallow}}
\item \entry{S vt.}{\headword{ngäs} \definition{1. to return, bring back}}
\item \entry{S vt.}{\headword{ngätae} \definition{1. to check}}
\item \entry{S vt.}{\headword{ngättangätta} \definition{1. to block, obscure}}
\item \entry{S vt.}{\headword{ngättangätta} \definition{1. to occupy, take over}}
\item \entry{S vt.}{\headword{ngättangätte} \definition{1. to count}}
\item \entry{S vt.}{\headword{ngetae} \definition{1. to unite, join}}
\item \entry{S vt.}{\headword{ngetae} \definition{1. to grip, hold onto}}
\item \entry{S vt.}{\headword{ngleg} \definition{1. to clear the ground before building}}
\item \entry{S vt.}{\headword{ngllongg} \definition{1. to forget}}
\item \entry{S vt.}{\headword{ngollot} \definition{1. to make burst, make explode}}
\item \entry{S vt.}{\headword{ngonoe} \definition{1. to ask}}
\item \entry{S vt.}{\headword{ngonoe} \definition{1. question}}
\item \entry{S vt.}{\headword{ngonongg} \definition{1. to think about, think of, recall, remember}}
\item \entry{S vt.}{\headword{ngongo} \definition{1. to smooth, sand (a surface)}}
\item \entry{S vt.}{\headword{ngongop} \definition{1. to hug, embrace}}
\item \entry{S vt.}{\headword{nyanyem} \definition{1. to respect}}
\item \entry{S vt.}{\headword{nyanyu} \definition{1. to act, dramatize}}
\item \entry{S vt.}{\headword{nyägae} \definition{1. to stir}}
\item \entry{S vt.}{\headword{nyämaenyämae} \definition{1. to surround}}
\item \entry{S vt.}{\headword{nyäny} \definition{1. to paint}}
\item \entry{S vt.}{\headword{nyäny} \definition{1. to anoint}}
\item \entry{S vt.}{\headword{nyänye} \definition{1. to share, split}}
\item \entry{S vt.}{\headword{nyongkoe} \definition{1. to pull}}
\item \entry{S vt.}{\headword{omom} \definition{1. to sweep}}
\item \entry{S vt.}{\headword{ony} \definition{1. to carry; get; bring}}
\item \entry{S vt.}{\headword{ony} \definition{1. causative-applicative form of ony (to make carry; bring for, take to; take from)}}
\item \entry{S vt.}{\headword{opap} \definition{1. to cross over, pass over, move across}}
\item \entry{S vt.}{\headword{otät} \definition{1. to eat}}
\item \entry{S vt.}{\headword{paengg} \definition{1. to guess}}
\item \entry{S vt.}{\headword{pallam} \definition{1. to cut open}}
\item \entry{S vt.}{\headword{pallängkmeny} \definition{1. to divide}}
\item \entry{S vt.}{\headword{pallängkmeny} \definition{1. to judge}}
\item \entry{S vt.}{\headword{pampem} \definition{1. to fish}}
\item \entry{S vt.}{\headword{panypeny} \definition{1. to speak, talk, say}}
\item \entry{S vt.}{\headword{paopao} \definition{1. to cut around, prune (a plant)}}
\item \entry{S vt.}{\headword{papälläk} \definition{1. to split, chop}}
\item \entry{S vt.}{\headword{pape} \definition{1. to cut (grass)}}
\item \entry{S vt.}{\headword{pape} \definition{1. to smash, crush (something soft, but not a flower)}}
\item \entry{S vt.}{\headword{papek} \definition{1. to block; close}}
\item \entry{S vt.}{\headword{papek} \definition{1. to wall, make a wall}}
\item \entry{S vt.}{\headword{papoe} \definition{1. to pierce, make a hole}}
\item \entry{S vt.}{\headword{pädoe} \definition{1. to blow}}
\item \entry{S vt.}{\headword{pädrall} \definition{1. to spread (out), scatter, sow}}
\item \entry{S vt.}{\headword{pällganen} \definition{1. to hang}}
\item \entry{S vt.}{\headword{pällkam} \definition{1. to split, break}}
\item \entry{S vt.}{\headword{pällkam} \definition{1. piece}}
\item \entry{S vt.}{\headword{pällkam} \definition{1. pieces, shards}}
\item \entry{S vt.}{\headword{pänae} \definition{1. to translate}}
\item \entry{S vt.}{\headword{pänaemeny} \definition{1. to check}}
\item \entry{S vt.}{\headword{pänongg} \definition{1. to wake up, wake}}
\item \entry{S vt.}{\headword{pänggmeny} \definition{1. to protect, look after, take care of}}
\item \entry{S vt.}{\headword{pätpät} \definition{1. to dry}}
\item \entry{S vt.}{\headword{pättapätte} \definition{1. to shake off}}
\item \entry{S vt.}{\headword{pättapätte} \definition{1. to hit grass with a stick to scare away animals}}
\item \entry{S vt.}{\headword{pättol} \definition{1. to start singing}}
\item \entry{S vt.}{\headword{pedae} \definition{1. to sweep}}
\item \entry{S vt.}{\headword{pendäg} \definition{1. to push to the ground, jostle, trip}}
\item \entry{S vt.}{\headword{penongg} \definition{1. to burn, set on fire}}
\item \entry{S vt.}{\headword{pentae} \definition{1. to transfer, transmit, spread, pass on, pass down}}
\item \entry{S vt.}{\headword{pengg} \definition{1. to make the killing shot, deliver the final blow}}
\item \entry{S vt.}{\headword{pipi} \definition{1. to shoot, spear}}
\item \entry{S vt.}{\headword{pirngän} \definition{1. to draw a weapon, take out}}
\item \entry{S vt.}{\headword{pisam} \definition{1. to tear (something thin)}}
\item \entry{S vt.}{\headword{pitkae} \definition{1. to untie}}
\item \entry{S vt.}{\headword{pittpitt} \definition{1. to weave}}
\item \entry{S vt.}{\headword{pittpitt} \definition{1. to sew, stitch}}
\item \entry{S vt.}{\headword{plengg} \definition{1. to cut down, cut off}}
\item \entry{S vt.}{\headword{po4} \definition{1. to pierce}}
\item \entry{S vt.}{\headword{pollpoll} \definition{1. to bark (at)}}
\item \entry{S vt.}{\headword{popo1} \definition{1. to sharpen}}
\item \entry{S vt.}{\headword{popo1} \definition{1. to carve}}
\item \entry{S vt.}{\headword{popo1} \definition{1. pencil sharpener}}
\item \entry{S vt.}{\headword{potpot} \definition{1. to slice open}}
\item \entry{S vt.}{\headword{rullgoe} \definition{1. to drag}}
\item \entry{S vt.}{\headword{sae} \definition{1. to close, cover}}
\item \entry{S vt.}{\headword{sae} \definition{1. to extinguish, put out}}
\item \entry{S vt.}{\headword{sae} \definition{1. closed}}
\item \entry{S vt.}{\headword{sänasäne} \definition{1. to take out}}
\item \entry{S vt.}{\headword{särämbae} \definition{1. to prepare, arrange, get ready}}
\item \entry{S vt.}{\headword{särämbae} \definition{1. to fix, solve, resolve}}
\item \entry{S vt.}{\headword{sɨs2} \definition{1. to extinguish, turn off}}
\item \entry{S vt.}{\headword{soroe} \definition{1. to try, attempt}}
\item \entry{S vt.}{\headword{soroe} \definition{1. to challenge, try, test; tempt}}
\item \entry{S vt.}{\headword{spun2} \definition{1. to throw; shoot}}
\item \entry{S vt.}{\headword{spun2} \definition{1. to remove an outer layer}}
\item \entry{S vt.}{\headword{taempäg} \definition{1. to show, indicate, reveal}}
\item \entry{S vt.}{\headword{tameny} \definition{1. to teach}}
\item \entry{S vt.}{\headword{tameny} \definition{1. to tell; converse with, speak to}}
\item \entry{S vt.}{\headword{tap2} \definition{1. to harvest}}
\item \entry{S vt.}{\headword{tater} \definition{1. to spread out}}
\item \entry{S vt.}{\headword{täbab} \definition{1. to watch over}}
\item \entry{S vt.}{\headword{tägab} \definition{1. to turn upside down}}
\item \entry{S vt.}{\headword{täli} \definition{1. to repeat}}
\item \entry{S vt.}{\headword{tämbameny} \definition{1. to instruct}}
\item \entry{S vt.}{\headword{tämpeyam} \definition{1. to abandon, give up}}
\item \entry{S vt.}{\headword{tängg} \definition{1. to urinate on, pee on}}
\item \entry{S vt.}{\headword{tänggag} \definition{1. to make a dog more sensitive to smells by rubbing lemongrass on their nose}}
\item \entry{S vt.}{\headword{tänggag} \definition{1. to steal}}
\item \entry{S vt.}{\headword{tärak} \definition{1. to put in}}
\item \entry{S vt.}{\headword{täram2} \definition{1. to lead, take, carry, collect}}
\item \entry{S vt.}{\headword{tärangg} \definition{1. to stop, hold back}}
\item \entry{S vt.}{\headword{täratäre} \definition{1. to dig out, hollow out}}
\item \entry{S vt.}{\headword{tärpam} \definition{1. to put across}}
\item \entry{S vt.}{\headword{tärpam} \definition{1. to crucify}}
\item \entry{S vt.}{\headword{tärpam} \definition{1. cross}}
\item \entry{S vt.}{\headword{tätäräp1} \definition{1. to cut}}
\item \entry{S vt.}{\headword{täträk} \definition{1. to push in, push through}}
\item \entry{S vt.}{\headword{tergony} \definition{1. to spread, unfold}}
\item \entry{S vt.}{\headword{timän} \definition{1. to release, fire, shoot (an arrow)}}
\item \entry{S vt.}{\headword{tɨram} \definition{1. to open}}
\item \entry{S vt.}{\headword{tɨt1} \definition{1. to beat sago, pound sago}}
\item \entry{S vt.}{\headword{tɨt1} \definition{1. to hollow out, dig out}}
\item \entry{S vt.}{\headword{tɨtɨp} \definition{1. to braid}}
\item \entry{S vt.}{\headword{togol} \definition{1. to hide}}
\item \entry{S vt.}{\headword{tomon} \definition{1. to wait for, await}}
\item \entry{S vt.}{\headword{tongg} \definition{1. to point}}
\item \entry{S vt.}{\headword{tongoe} \definition{1. to laugh (at)}}
\item \entry{S vt.}{\headword{torwam} \definition{1. to lay down}}
\item \entry{S vt.}{\headword{torwam} \definition{1. causative-applicative form of torwam}}
\item \entry{S vt.}{\headword{trongg} \definition{1. to follow}}
\item \entry{S vt.}{\headword{trungg} \definition{1. to invite, call over, summon}}
\item \entry{S vt.}{\headword{tulgoe} \definition{1. to gossip about}}
\item \entry{S vt.}{\headword{tutu3} \definition{1. to knock fruit continuously}}
\item \entry{S vt.}{\headword{ttaem} \definition{1. to pack}}
\item \entry{S vt.}{\headword{ttaempäg} \definition{1. to seperate, divorce}}
\item \entry{S vt.}{\headword{ttaengän} \definition{1. to pull (a plant sucker)}}
\item \entry{S vt.}{\headword{ttam2} \definition{1. to call, name}}
\item \entry{S vt.}{\headword{ttamän} \definition{1. to finish, end, complete}}
\item \entry{S vt.}{\headword{ttapeyam} \definition{1. to open (something long, e.g. door, book)}}
\item \entry{S vt.}{\headword{ttatta1} \definition{1. to chop a tree}}
\item \entry{S vt.}{\headword{ttäbattäbe} \definition{1. to block}}
\item \entry{S vt.}{\headword{ttäkam} \definition{1. to break, snap}}
\item \entry{S vt.}{\headword{ttäkattäke} \definition{1. to fold}}
\item \entry{S vt.}{\headword{ttäkoe} \definition{1. to chop, cut down, mow; shave}}
\item \entry{S vt.}{\headword{ttällam} \definition{1. to pass, hand}}
\item \entry{S vt.}{\headword{ttällam} \definition{1. to extend, stretch out, reach out, put out}}
\item \entry{S vt.}{\headword{ttäm1} \definition{1. to burn; heat on a fire}}
\item \entry{S vt.}{\headword{ttängattänge} \definition{1. to read, recite}}
\item \entry{S vt.}{\headword{ttängkag} \definition{1. to break with force}}
\item \entry{S vt.}{\headword{ttängkamäll} \definition{1. to meet, reach}}
\item \entry{S vt.}{\headword{ttäpen} \definition{1. to break, tear (esp. something long)}}
\item \entry{S vt.}{\headword{ttättälläg} \definition{1. to tear, tear up}}
\item \entry{S vt.}{\headword{ttotto} \definition{1. to tie}}
\item \entry{S vt.}{\headword{ttotto} \definition{1. to collect}}
\item \entry{S vt.}{\headword{udab} \definition{1. to lose}}
\item \entry{S vt.}{\headword{udaude} \definition{1. to light, start (a fire)}}
\item \entry{S vt.}{\headword{ugug} \definition{1. to make mumu}}
\item \entry{S vt.}{\headword{ullull} \definition{1. to cross over}}
\item \entry{S vt.}{\headword{umaem} \definition{1. to gather, collect}}
\item \entry{S vt.}{\headword{ume ddäddäl} \definition{1. to kiss}}
\item \entry{S vt.}{\headword{wabeb} \definition{1. to beat, smash, pound}}
\item \entry{S vt.}{\headword{wanseg} \definition{1. to put, place, set aside, leave}}
\item \entry{S vt.}{\headword{wanwen} \definition{1. to shake, swing}}
\item \entry{S vt.}{\headword{wangam} \definition{1. to forget}}
\item \entry{S vt.}{\headword{wanyweny} \definition{1. to burn}}
\item \entry{S vt.}{\headword{waswes} \definition{1. to ask, beg}}
\item \entry{S vt.}{\headword{wändäg} \definition{1. to crowd}}
\item \entry{S vt.}{\headword{wɨndwɨnd} \definition{1. to cover, bury}}
\item \entry{S vt.}{\headword{wowo} \definition{1. to clear floating grass by pushing through with a canoe}}
\item \entry{S vt.}{\headword{yagyeg} \definition{1. to search, look for}}
\item \entry{S vt.}{\headword{yatän} \definition{1. to fetch (water)}}
\item \entry{S vt.}{\headword{yämbäg} \definition{1. to disown, repudiate}}
\item \entry{S vt.}{\headword{zan} \definition{1. to put in}}
\item \entry{S vt.}{\headword{zämae} \definition{1. to pour, put, transfer}}
\item \entry{S vt.}{\headword{zigae} \definition{1. to wrap}}
\item \entry{S vt.}{\headword{zizi} \definition{1. to uncover, lift}}
\item \entry{S vt.}{\headword{zuwoe} \definition{1. to shoot}}
\item \entry{S vt.}{\headword{zuwoe} \definition{1. to pierce; to inject, administer a shot}}
\end{enumerate}

\section{TAM ptcl.}
\begin{enumerate}
\item \entry{TAM ptcl.}{\headword{abo} \definition{1. must, necessative mood}}
\item \entry{TAM ptcl.}{\headword{ada} \definition{1. ~ ako}}
\item \entry{TAM ptcl.}{\headword{bänga} \definition{1. should}}
\item \entry{TAM ptcl.}{\headword{da1} \definition{1. might, may, could (marks potential mood)}}
\item \entry{TAM ptcl.}{\headword{ka1} \definition{1. counterfactual particle (often preceded by ada)}}
\item \entry{TAM ptcl.}{\headword{ke} \definition{1. counterfactual}}
\item \entry{TAM ptcl.}{\headword{koo} \definition{1. until}}
\item \entry{TAM ptcl.}{\headword{mäse} \definition{1. imminent particle (indicates that something about to take place)}}
\item \entry{TAM ptcl.}{\headword{mäse} \definition{1. conative particle (indicates an attempt; to try to do something)}}
\item \entry{TAM ptcl.}{\headword{mɨnyi} \definition{1. future tense particle}}
\item \entry{TAM ptcl.}{\headword{nga} \definition{1. marks immediate or near future}}
\item \entry{TAM ptcl.}{\headword{ngase} \definition{1. hortative/optative particle}}
\item \entry{TAM ptcl.}{\headword{ngasekäma} \definition{1. potential marker; may, could, might}}
\end{enumerate}

\section{adv.}
\begin{enumerate}
\item \entry{adv.}{\headword{abal} \definition{1. very}}
\item \entry{adv.}{\headword{abal} \definition{1. exactly, just}}
\item \entry{adv.}{\headword{abo} \definition{1. then, afterwards}}
\item \entry{adv.}{\headword{ada} \definition{1. like this, thus, so}}
\item \entry{adv.}{\headword{ada} \definition{1. ~ ada}}
\item \entry{adv.}{\headword{ada} \definition{1. ~ ada}}
\item \entry{adv.}{\headword{ada} \definition{1. this is why, therefore}}
\item \entry{adv.}{\headword{ada} \definition{1. at this moment}}
\item \entry{adv.}{\headword{ada} \definition{1. so that}}
\item \entry{adv.}{\headword{ada} \definition{1. like this}}
\item \entry{adv.}{\headword{ada} \definition{1. the same way}}
\item \entry{adv.}{\headword{ada} \definition{1. then, so}}
\item \entry{adv.}{\headword{ada} \definition{1. why, therefore (marks a result or consequence)}}
\item \entry{adv.}{\headword{ade} \definition{1. also}}
\item \entry{adv.}{\headword{ai1} \definition{1. very}}
\item \entry{adv.}{\headword{ako} \definition{1. also, too}}
\item \entry{adv.}{\headword{ako} \definition{1. again}}
\item \entry{adv.}{\headword{ako} \definition{1. then}}
\item \entry{adv.}{\headword{alla} \definition{1. how, what}}
\item \entry{adv.}{\headword{alla} \definition{1. how}}
\item \entry{adv.}{\headword{amiyeamiye} \definition{1. upwind, against the wind}}
\item \entry{adv.}{\headword{amtet} \definition{1. nonstop, without breathing}}
\item \entry{adv.}{\headword{angäm} \definition{1. quickly}}
\item \entry{adv.}{\headword{angde} \definition{1. when, while, as}}
\item \entry{adv.}{\headword{any1} \definition{1. like this, thus, so}}
\item \entry{adv.}{\headword{aoli} \definition{1. almost, nearly}}
\item \entry{adv.}{\headword{binbäddbädd} \definition{1. fully, completely}}
\item \entry{adv.}{\headword{dade3} \definition{1. maybe}}
\item \entry{adv.}{\headword{dam} \definition{1. then}}
\item \entry{adv.}{\headword{damärärmae} \definition{1. simultaneously}}
\item \entry{adv.}{\headword{daramdaram} \definition{1. shining brightly}}
\item \entry{adv.}{\headword{dägadäga} \definition{1. completely}}
\item \entry{adv.}{\headword{där} \definition{1. in pairs}}
\item \entry{adv.}{\headword{duli} \definition{1. away (from a place towards another direction)}}
\item \entry{adv.}{\headword{duli} \definition{1. that way}}
\item \entry{adv.}{\headword{duli} \definition{1. ablative form of duli}}
\item \entry{adv.}{\headword{duli} \definition{1. ablative form of duli}}
\item \entry{adv.}{\headword{ddänddängeny} \definition{1. immediately}}
\item \entry{adv.}{\headword{ddobae} \definition{1. very}}
\item \entry{adv.}{\headword{ddobae} \definition{1. very, extremely}}
\item \entry{adv.}{\headword{ddugwemddugwem} \definition{1. stomping}}
\item \entry{adv.}{\headword{ede} \definition{1. then}}
\item \entry{adv.}{\headword{enanae} \definition{1. directly, straight}}
\item \entry{adv.}{\headword{enanae} \definition{1. forever, for good}}
\item \entry{adv.}{\headword{gagäll} \definition{1. badly, poorly}}
\item \entry{adv.}{\headword{gee} \definition{1. intensifier}}
\item \entry{adv.}{\headword{gudne} \definition{1. long ago}}
\item \entry{adv.}{\headword{imne} \definition{1. afterwards, after, later}}
\item \entry{adv.}{\headword{indrang} \definition{1. clearly}}
\item \entry{adv.}{\headword{ingong} \definition{1. dancing around}}
\item \entry{adv.}{\headword{kame1} \definition{1. unknowingly, mistakenly, wrongly}}
\item \entry{adv.}{\headword{kame2} \definition{1. again}}
\item \entry{adv.}{\headword{kälae} \definition{1. a little}}
\item \entry{adv.}{\headword{kälae} \definition{1. into pieces}}
\item \entry{adv.}{\headword{kälae} \definition{1. nonsingular form of kälae}}
\item \entry{adv.}{\headword{kälepalle} \definition{1. slowly}}
\item \entry{adv.}{\headword{källkae} \definition{1. later, in the future}}
\item \entry{adv.}{\headword{känazbag} \definition{1. tomorrow}}
\item \entry{adv.}{\headword{käsre} \definition{1. then}}
\item \entry{adv.}{\headword{kllomokllomoll} \definition{1. downwind, leeward}}
\item \entry{adv.}{\headword{llame} \definition{1. together}}
\item \entry{adv.}{\headword{llätt} \definition{1. nonstop}}
\item \entry{adv.}{\headword{mamall} \definition{1. quickly}}
\item \entry{adv.}{\headword{mamall} \definition{1. immediately, right away}}
\item \entry{adv.}{\headword{mamall} \definition{1. quickly}}
\item \entry{adv.}{\headword{mäse} \definition{1. naked}}
\item \entry{adv.}{\headword{mäse} \definition{1. empty-handed}}
\item \entry{adv.}{\headword{mer1} \definition{1. properly, nicely, well}}
\item \entry{adv.}{\headword{misdae} \definition{1. just, simply}}
\item \entry{adv.}{\headword{mizi} \definition{1. usually}}
\item \entry{adv.}{\headword{nälan} \definition{1. accidentally}}
\item \entry{adv.}{\headword{ngalbongalboe} \definition{1. on the side}}
\item \entry{adv.}{\headword{ngasekäma} \definition{1. maybe}}
\item \entry{adv.}{\headword{ngattong} \definition{1. first, at first, initially, previously, before}}
\item \entry{adv.}{\headword{ngäs} \definition{1. repeatedly}}
\item \entry{adv.}{\headword{ngäs} \definition{1. back, the other way}}
\item \entry{adv.}{\headword{o klak} \definition{1. o'clock}}
\item \entry{adv.}{\headword{pällnampällnam} \definition{1. squatting, crouching}}
\item \entry{adv.}{\headword{päre} \definition{1. nonsingular form of säre}}
\item \entry{adv.}{\headword{pentae} \definition{1. directly, in person, personally}}
\item \entry{adv.}{\headword{säkmällsäkmäll} \definition{1. limping}}
\item \entry{adv.}{\headword{säre} \definition{1. sadly}}
\item \entry{adv.}{\headword{säsramsäsram} \definition{1. shuffling}}
\item \entry{adv.}{\headword{singosingol} \definition{1. upwind, windward}}
\item \entry{adv.}{\headword{sisri} \definition{1. now}}
\item \entry{adv.}{\headword{tanong} \definition{1. a little}}
\item \entry{adv.}{\headword{tätäm} \definition{1. yesterday}}
\item \entry{adv.}{\headword{tlläpmälltlläpmäll} \definition{1. nibbling}}
\item \entry{adv.}{\headword{tonton} \definition{1. directly}}
\item \entry{adv.}{\headword{ttalamttalam2} \definition{1. walking with one's legs spread far apart}}
\item \entry{adv.}{\headword{ttapeyamttapeyam} \definition{1. walking with one's legs spread far apart}}
\item \entry{adv.}{\headword{ttongo1} \definition{1. each, one by one}}
\item \entry{adv.}{\headword{ullowae} \definition{1. fast, quickly}}
\item \entry{adv.}{\headword{wandawandae} \definition{1. rotating, spinning, rolling}}
\item \entry{adv.}{\headword{yuwog} \definition{1. for no reason, on a whim}}
\item \entry{adv.}{\headword{yuwog} \definition{1. needlessly, excessively}}
\item \entry{adv.}{\headword{yuwog} \definition{1. not completely, not properly}}
\item \entry{adv.}{\headword{zɨme} \definition{1. already}}
\end{enumerate}

\section{adv. dem.}
\begin{enumerate}
\item \entry{adv. dem.}{\headword{da3} \definition{1. ablative form of da; after that}}
\item \entry{adv. dem.}{\headword{da3} \definition{1. ablative form of da; therefore}}
\item \entry{adv. dem.}{\headword{da3} \definition{1. allative form of da}}
\item \entry{adv. dem.}{\headword{da3} \definition{1. allative form of da with perlative clitic}}
\item \entry{adv. dem.}{\headword{de1} \definition{1. there (mesial)}}
\item \entry{adv. dem.}{\headword{didri} \definition{1. there}}
\item \entry{adv. dem.}{\headword{didri} \definition{1. ablative form of didri}}
\item \entry{adv. dem.}{\headword{do1} \definition{1. over there (distal)}}
\item \entry{adv. dem.}{\headword{do1} \definition{1. there}}
\item \entry{adv. dem.}{\headword{do1} \definition{1. ablative form of do}}
\item \entry{adv. dem.}{\headword{duli} \definition{1. over there (distal)}}
\item \entry{adv. dem.}{\headword{gänya} \definition{1. here (proximal)}}
\item \entry{adv. dem.}{\headword{gänya} \definition{1. here}}
\item \entry{adv. dem.}{\headword{gänya} \definition{1. restrictive copular form of gänya (present singular form)}}
\item \entry{adv. dem.}{\headword{gänya} \definition{1. copular form of gänya (present singular form)}}
\item \entry{adv. dem.}{\headword{gänya} \definition{1. past singular form of gänyan}}
\item \entry{adv. dem.}{\headword{gänya} \definition{1. present plural form of gänyan}}
\item \entry{adv. dem.}{\headword{gänya} \definition{1. past plural form of gänyan}}
\item \entry{adv. dem.}{\headword{gänya} \definition{1. present dual form of gänyan}}
\item \entry{adv. dem.}{\headword{gänya} \definition{1. past dual form of gänyan}}
\item \entry{adv. dem.}{\headword{gänya} \definition{1. this way}}
\item \entry{adv. dem.}{\headword{gänya} \definition{1. gänyeri with perlative clitic}}
\item \entry{adv. dem.}{\headword{gänya} \definition{1. allative form of gänyeri}}
\item \entry{adv. dem.}{\headword{gänya} \definition{1. ablative form of gänya}}
\item \entry{adv. dem.}{\headword{gänya} \definition{1. ablative form of gänya}}
\item \entry{adv. dem.}{\headword{gänya} \definition{1. allative form of gänya}}
\item \entry{adv. dem.}{\headword{gänya} \definition{1. allative form of gänya with perlative clitic}}
\end{enumerate}

\section{cl.}
\begin{enumerate}
\item \entry{cl.}{\headword{=ae1} \definition{1. restrictive clitic; only}}
\item \entry{cl.}{\headword{=ae2} \definition{1. adverbializing clitic}}
\item \entry{cl.}{\headword{=ang} \definition{1. multipurpose attributive suffix that functions as an imperfective or resultative participle, agentive nominalizer, and generic adjective-forming suffix}}
\item \entry{cl.}{\headword{=ma} \definition{1. characteristic case clitic (indicates purpose, source, or reason)}}
\end{enumerate}

\section{col.}
\begin{enumerate}
\item \entry{col.}{\headword{kätt} \definition{1. blue (lit. 'shell water')}}
\item \entry{col.}{\headword{kätt} \definition{1. type of game played with shells}}
\item \entry{col.}{\headword{mam} \definition{1. red; pink}}
\item \entry{col.}{\headword{pällämpälläm} \definition{1. white, bright}}
\item \entry{col.}{\headword{sägäsägäd} \definition{1. yellow}}
\end{enumerate}

\section{comp.}
\begin{enumerate}
\item \entry{comp.}{\headword{ada} \definition{1. that}}
\end{enumerate}

\section{coord.}
\begin{enumerate}
\item \entry{coord.}{\headword{a} \definition{1. and}}
\item \entry{coord.}{\headword{abe} \definition{1. but}}
\item \entry{coord.}{\headword{be} \definition{1. but}}
\item \entry{coord.}{\headword{o1} \definition{1. or}}
\end{enumerate}

\section{cop.}
\begin{enumerate}
\item \entry{cop.}{\headword{amig} \definition{1. contraction of ami dag}}
\item \entry{cop.}{\headword{aya} \definition{1. copular form of aya (present singular form)}}
\item \entry{cop.}{\headword{aya} \definition{1. past singular form of aenen}}
\item \entry{cop.}{\headword{aya} \definition{1. instrumental-comitative form of aya}}
\item \entry{cop.}{\headword{daden} \definition{1. to exist, have, there is (present singular form)}}
\item \entry{cop.}{\headword{daden} \definition{1. past singular form of daden}}
\item \entry{cop.}{\headword{daden} \definition{1. present plural form of daden}}
\item \entry{cop.}{\headword{daden} \definition{1. past plural form of daden}}
\item \entry{cop.}{\headword{daden} \definition{1. present dual form of daden}}
\item \entry{cop.}{\headword{daden} \definition{1. past dual form of daden}}
\item \entry{cop.}{\headword{dan} \definition{1. to be, exist, is, there is (present singular form)}}
\item \entry{cop.}{\headword{dan} \definition{1. past singular form of dan}}
\item \entry{cop.}{\headword{dan} \definition{1. present plural form of dan}}
\item \entry{cop.}{\headword{dan} \definition{1. past plural form of dan}}
\item \entry{cop.}{\headword{dan} \definition{1. present dual form of dan}}
\item \entry{cop.}{\headword{dan} \definition{1. past dual form of dan}}
\item \entry{cop.}{\headword{dibaeya} \definition{1. cop.that.pst.sgS}}
\item \entry{cop.}{\headword{mudan} \definition{1. prohibitive copula (present singular form)}}
\item \entry{cop.}{\headword{mudan} \definition{1. present plural form of mudan}}
\item \entry{cop.}{\headword{mudan} \definition{1. present dual form of mudan}}
\end{enumerate}

\section{dem.}
\begin{enumerate}
\item \entry{dem.}{\headword{da3} \definition{1. that (mesial determiner and pronoun)}}
\item \entry{dem.}{\headword{däba2} \definition{1. that (mesial determiner)}}
\item \entry{dem.}{\headword{däba2} \definition{1. allative form of däba}}
\item \entry{dem.}{\headword{däba2} \definition{1. copular form of däba (present singular form)}}
\item \entry{dem.}{\headword{däba2} \definition{1. past singular form of däban}}
\item \entry{dem.}{\headword{däba2} \definition{1. present plural form of däban}}
\item \entry{dem.}{\headword{däba2} \definition{1. past plural form of däban}}
\item \entry{dem.}{\headword{däba2} \definition{1. past dual form of däban}}
\item \entry{dem.}{\headword{däba2} \definition{1. ablative form of däba}}
\item \entry{dem.}{\headword{däba2} \definition{1. accusative form of däba}}
\item \entry{dem.}{\headword{de1} \definition{1. that (mesial determiner)}}
\item \entry{dem.}{\headword{de1} \definition{1. there}}
\item \entry{dem.}{\headword{de1} \definition{1. there}}
\item \entry{dem.}{\headword{de1} \definition{1. then, at that time}}
\item \entry{dem.}{\headword{de1} \definition{1. there}}
\item \entry{dem.}{\headword{diba} \definition{1. that (mesial determiner)}}
\item \entry{dem.}{\headword{diba} \definition{1. copular form of diba (present singular form)}}
\item \entry{dem.}{\headword{diba} \definition{1. plural present form of diban}}
\item \entry{dem.}{\headword{diba} \definition{1. past plural form of diban}}
\item \entry{dem.}{\headword{diba} \definition{1. present dual form of diban}}
\item \entry{dem.}{\headword{diba} \definition{1. past dual form of diban}}
\item \entry{dem.}{\headword{diba} \definition{1. ablative form of diba; then, thereupon}}
\item \entry{dem.}{\headword{diba} \definition{1. ablative form of diba; because}}
\item \entry{dem.}{\headword{diba} \definition{1. allative form of diba}}
\item \entry{dem.}{\headword{diba} \definition{1. accusative form of diba}}
\item \entry{dem.}{\headword{diban1} \definition{1. there}}
\item \entry{dem.}{\headword{diban1} \definition{1. that one}}
\item \entry{dem.}{\headword{gänya} \definition{1. this (proximal determiner)}}
\item \entry{dem.}{\headword{ge1} \definition{1. this (proximal determiner and pronoun)}}
\item \entry{dem.}{\headword{ge1} \definition{1. accusative form of ge}}
\end{enumerate}

\section{det.}
\begin{enumerate}
\item \entry{det.}{\headword{da3} \definition{1. last}}
\item \entry{det.}{\headword{de1} \definition{1. next}}
\item \entry{det.}{\headword{ddob} \definition{1. other}}
\item \entry{det.}{\headword{ttongo1} \definition{1. another}}
\item \entry{det.}{\headword{ttongo1} \definition{1. next}}
\end{enumerate}

\section{disc. ptcl.}
\begin{enumerate}
\item \entry{disc. ptcl.}{\headword{=a} \definition{1. vocative particle}}
\item \entry{disc. ptcl.}{\headword{=e2} \definition{1. vocative}}
\item \entry{disc. ptcl.}{\headword{ka1} \definition{1. emphatic particle}}
\end{enumerate}

\section{ideo.}
\begin{enumerate}
\item \entry{ideo.}{\headword{ansi} \definition{1. achoo}}
\item \entry{ideo.}{\headword{bäbrem} \definition{1. sound made by cassowaries}}
\item \entry{ideo.}{\headword{dun} \definition{1. sound of a drum}}
\item \entry{ideo.}{\headword{gllu} \definition{1. splash}}
\item \entry{ideo.}{\headword{käntrokäntrom nane} \definition{1. gulp}}
\item \entry{ideo.}{\headword{keam} \definition{1. sound made by deer}}
\item \entry{ideo.}{\headword{kunglle} \definition{1. sound made by flying foxes}}
\item \entry{ideo.}{\headword{kwa} \definition{1. sound made by a dog in pain}}
\item \entry{ideo.}{\headword{llällam} \definition{1. rustling (of plants); roaring (of water)}}
\item \entry{ideo.}{\headword{llällam} \definition{1. noisy}}
\item \entry{ideo.}{\headword{ngok} \definition{1. oink, snort}}
\item \entry{ideo.}{\headword{seseyam} \definition{1. swish, sound of legs moving}}
\end{enumerate}

\section{indefinite article}
\begin{enumerate}
\item \entry{indefinite article}{\headword{ttongo1} \definition{1. a/an}}
\end{enumerate}

\section{int. pron.}
\begin{enumerate}
\item \entry{int. pron.}{\headword{ami} \definition{1. who (nonsingular interrogative pronoun, nominative form)}}
\item \entry{int. pron.}{\headword{ami} \definition{1. possessive form of ami}}
\item \entry{int. pron.}{\headword{ami} \definition{1. comitative of ami}}
\item \entry{int. pron.}{\headword{ami} \definition{1. comitative form of ami}}
\item \entry{int. pron.}{\headword{ami} \definition{1. accusative form of ami}}
\item \entry{int. pron.}{\headword{ami} \definition{1. dative form of ami}}
\item \entry{int. pron.}{\headword{aoli} \definition{1. how many, how much}}
\item \entry{int. pron.}{\headword{aya} \definition{1. who (singular interrogative pronoun, nominative form)}}
\item \entry{int. pron.}{\headword{aya} \definition{1. ablative-possessive form of aya}}
\item \entry{int. pron.}{\headword{aya} \definition{1. dative form of aya}}
\item \entry{int. pron.}{\headword{aya} \definition{1. possessive form of aya}}
\item \entry{int. pron.}{\headword{aya} \definition{1. accusative form of aya}}
\item \entry{int. pron.}{\headword{e2} \definition{1. which, what}}
\item \entry{int. pron.}{\headword{e2} \definition{1. what}}
\item \entry{int. pron.}{\headword{e2} \definition{1. copular form of enda (present singular form)}}
\item \entry{int. pron.}{\headword{e2} \definition{1. past singular form of endan}}
\item \entry{int. pron.}{\headword{e2} \definition{1. past plural form of endan}}
\item \entry{int. pron.}{\headword{e2} \definition{1. present plural form of endan}}
\item \entry{int. pron.}{\headword{e2} \definition{1. past plural form of endan}}
\item \entry{int. pron.}{\headword{e2} \definition{1. past singular form of endan}}
\item \entry{int. pron.}{\headword{e2} \definition{1. accusative form of enda}}
\item \entry{int. pron.}{\headword{e2} \definition{1. which, what}}
\item \entry{int. pron.}{\headword{e2} \definition{1. copular form of era (present singular form)}}
\item \entry{int. pron.}{\headword{e2} \definition{1. past plural form of eran}}
\item \entry{int. pron.}{\headword{e2} \definition{1. present plural form of eran}}
\item \entry{int. pron.}{\headword{e2} \definition{1. past plural form of eran}}
\item \entry{int. pron.}{\headword{e2} \definition{1. past dual form of eran}}
\item \entry{int. pron.}{\headword{e2} \definition{1. instrumental form of era}}
\item \entry{int. pron.}{\headword{e2} \definition{1. where, in which (locative form of era)}}
\item \entry{int. pron.}{\headword{e2} \definition{1. accusative form of era}}
\item \entry{int. pron.}{\headword{e2} \definition{1. where}}
\item \entry{int. pron.}{\headword{e2} \definition{1. ablative form of ero}}
\item \entry{int. pron.}{\headword{e2} \definition{1. allative form of ero}}
\item \entry{int. pron.}{\headword{e2} \definition{1. why}}
\item \entry{int. pron.}{\headword{e2} \definition{1. accusative form of e}}
\item \entry{int. pron.}{\headword{e2} \definition{1. ablative form of e; why (relative)}}
\item \entry{int. pron.}{\headword{e2} \definition{1. ablative form of e; why (interrogative)}}
\item \entry{int. pron.}{\headword{ili} \definition{1. where}}
\end{enumerate}

\section{int. ptcl.}
\begin{enumerate}
\item \entry{int. ptcl.}{\headword{ka1} \definition{1. question particle}}
\item \entry{int. ptcl.}{\headword{ke} \definition{1. question particle}}
\end{enumerate}

\section{interj.}
\begin{enumerate}
\item \entry{interj.}{\headword{aba} \definition{1. go (command given to animals)}}
\item \entry{interj.}{\headword{ae} \definition{1. ah}}
\item \entry{interj.}{\headword{ai1} \definition{1. alright, okay}}
\item \entry{interj.}{\headword{ai1} \definition{1. properly, well}}
\item \entry{interj.}{\headword{ai2} \definition{1. ah}}
\item \entry{interj.}{\headword{ai2} \definition{1. hey}}
\item \entry{interj.}{\headword{ao} \definition{1. yes}}
\item \entry{interj.}{\headword{ddone} \definition{1. no}}
\item \entry{interj.}{\headword{e1} \definition{1. whoa}}
\item \entry{interj.}{\headword{e1} \definition{1. let's go}}
\item \entry{interj.}{\headword{ei} \definition{1. hey}}
\item \entry{interj.}{\headword{es} \definition{1. come (command given to animals)}}
\item \entry{interj.}{\headword{eso} \definition{1. thank you}}
\item \entry{interj.}{\headword{ibi2} \definition{1. let's go}}
\item \entry{interj.}{\headword{ka2} \definition{1. no}}
\item \entry{interj.}{\headword{kandärmang1} \definition{1. sorry}}
\item \entry{interj.}{\headword{o3} \definition{1. oh}}
\item \entry{interj.}{\headword{oi} \definition{1. oh}}
\item \entry{interj.}{\headword{oke} \definition{1. okay}}
\item \entry{interj.}{\headword{pop} \definition{1. rubbish, darn}}
\item \entry{interj.}{\headword{sari} \definition{1. sorry}}
\item \entry{interj.}{\headword{sɨs1} \definition{1. command given to a dog to chase an animal}}
\item \entry{interj.}{\headword{waeyo} \definition{1. shout made in distress}}
\item \entry{interj.}{\headword{wapo} \definition{1. oh no}}
\item \entry{interj.}{\headword{wi2} \definition{1. oh}}
\item \entry{interj.}{\headword{wiyo} \definition{1. wow}}
\item \entry{interj.}{\headword{wiyowae} \definition{1. wow; phew}}
\item \entry{interj.}{\headword{ya} \definition{1. scram, go away}}
\end{enumerate}

\section{kin.}
\begin{enumerate}
\item \entry{kin.}{\headword{baba} \definition{1. father}}
\item \entry{kin.}{\headword{dada} \definition{1. older sibling of the same sex (man's older brother or woman's older sister)}}
\item \entry{kin.}{\headword{dedi} \definition{1. daddy}}
\item \entry{kin.}{\headword{erang} \definition{1. exchange sibling; exchange brother (man's wife's brother who marries the man's sister; reciprocal); exchange sister (woman's husband's sister who marries the woman's brother; reciprocal)}}
\item \entry{kin.}{\headword{erang} \definition{1. exchange cousin (one's parent's exchange sibling's child)}}
\item \entry{kin.}{\headword{erang} \definition{1. exchange uncle (one's parent's exchange brother)}}
\item \entry{kin.}{\headword{erang} \definition{1. exchange aunt (one's parent's exchange sister)}}
\item \entry{kin.}{\headword{gullbe} \definition{1. husband}}
\item \entry{kin.}{\headword{inbo} \definition{1. brother-in-law (woman's husband's brother)}}
\item \entry{kin.}{\headword{inbo} \definition{1. sister-in-law (man's elder brother's wife or woman's husband's younger sister; reciprocal)}}
\item \entry{kin.}{\headword{izig} \definition{1. co-sister-in-law (woman's husband's brother's wife)}}
\item \entry{kin.}{\headword{izig} \definition{1. co-wife (another woman married to the same husband)}}
\item \entry{kin.}{\headword{izig} \definition{1. polygynous marriage}}
\item \entry{kin.}{\headword{kak} \definition{1. grandmother (one's parent's mother; reciprocal)}}
\item \entry{kin.}{\headword{kak} \definition{1. grandchild (woman's child's child; reciprocal)}}
\item \entry{kin.}{\headword{kak} \definition{1. mother-in-law (woman's husband's mother; reciprocal)}}
\item \entry{kin.}{\headword{kak} \definition{1. daughter-in-law (woman's son's wife; reciprocal)}}
\item \entry{kin.}{\headword{kak} \definition{1. nonsingular form of kak}}
\item \entry{kin.}{\headword{kobeam} \definition{1. co-brother-in-law (man's wife's sister's husband)}}
\item \entry{kin.}{\headword{kok1} \definition{1. grandchild (one's child's child)}}
\item \entry{kin.}{\headword{kok1} \definition{1. daughter-in-law (one's son's wife)}}
\item \entry{kin.}{\headword{kok1} \definition{1. nonsingular form of kok}}
\item \entry{kin.}{\headword{lla} \definition{1. husband}}
\item \entry{kin.}{\headword{llɨg} \definition{1. son}}
\item \entry{kin.}{\headword{llɨg} \definition{1. child}}
\item \entry{kin.}{\headword{mama2} \definition{1. mother}}
\item \entry{kin.}{\headword{mami} \definition{1. mommy, mummy}}
\item \entry{kin.}{\headword{mang} \definition{1. brother (of a woman)}}
\item \entry{kin.}{\headword{mang} \definition{1. nonsingular form of mang}}
\item \entry{kin.}{\headword{masar} \definition{1. grandfather (one's parent's father; reciprocal); ancestor}}
\item \entry{kin.}{\headword{masar} \definition{1. grandchild (man's child's child; reciprocal)}}
\item \entry{kin.}{\headword{masar} \definition{1. father-in-law (woman's husband's father; reciprocal)}}
\item \entry{kin.}{\headword{masar} \definition{1. daughter-in-law (man's son's wife; reciprocal)}}
\item \entry{kin.}{\headword{masar} \definition{1. uncle-in-law (woman's husband's mother's younger brother; reciprocal)}}
\item \entry{kin.}{\headword{masar} \definition{1. niece-in-law (man's elder sister's son's wife; reciprocal)}}
\item \entry{kin.}{\headword{masar} \definition{1. grandparents}}
\item \entry{kin.}{\headword{masar} \definition{1. forefathers, ancestors}}
\item \entry{kin.}{\headword{mäda} \definition{1. father}}
\item \entry{kin.}{\headword{mäda} \definition{1. aunt (one's father's elder sister)}}
\item \entry{kin.}{\headword{mäda} \definition{1. uncle (one's father's elder brother)}}
\item \entry{kin.}{\headword{mäda} \definition{1. brother-in-law (woman's husband's elder brother)}}
\item \entry{kin.}{\headword{mäda} \definition{1. sister-in-law (woman's husband's elder sister or woman's younger brother's wife; reciprocal)}}
\item \entry{kin.}{\headword{mäda} \definition{1. fatherless}}
\item \entry{kin.}{\headword{mäg} \definition{1. mother}}
\item \entry{kin.}{\headword{mäg} \definition{1. aunt (one's father's elder brother's wife)}}
\item \entry{kin.}{\headword{mäg} \definition{1. motherless}}
\item \entry{kin.}{\headword{mälla} \definition{1. wife}}
\item \entry{kin.}{\headword{mälla yae} \definition{1. aunt (one's mother's elder sister)}}
\item \entry{kin.}{\headword{mällpa} \definition{1. aunt (one's mother's sister)}}
\item \entry{kin.}{\headword{män3} \definition{1. sister}}
\item \entry{kin.}{\headword{män3} \definition{1. She's my daughter.}}
\item \entry{kin.}{\headword{mänang} \definition{1. father-in-law (man's wife's father; reciprocal)}}
\item \entry{kin.}{\headword{mänang} \definition{1. mother-in-law (man's wife's mother; reciprocal)}}
\item \entry{kin.}{\headword{mänang} \definition{1. son-in-law (one's daughter's husband; reciprocal)}}
\item \entry{kin.}{\headword{mänyan} \definition{1. younger sibling of the same-sex (man's younger brother or woman's younger sister)}}
\item \entry{kin.}{\headword{mänyan} \definition{1. co-sibling-in-law (man's wife's younger sister's husband or woman's husband's younger brother's wife)}}
\item \entry{kin.}{\headword{mänyan} \definition{1. younger siblings}}
\item \entry{kin.}{\headword{meyang} \definition{1. uncle (one's father's younger brother)}}
\item \entry{kin.}{\headword{meyang} \definition{1. brother-in-law (woman's husband's younger brother)}}
\item \entry{kin.}{\headword{mosen} \definition{1. older sibling of the same-sex (man's older brother or woman's older sister)}}
\item \entry{kin.}{\headword{mosen} \definition{1. older siblings}}
\item \entry{kin.}{\headword{nag} \definition{1. friend}}
\item \entry{kin.}{\headword{nag} \definition{1. friendship}}
\item \entry{kin.}{\headword{nag} \definition{1. nonsingular form of nag}}
\item \entry{kin.}{\headword{nane1} \definition{1. aunt (one's parent's younger sister)}}
\item \entry{kin.}{\headword{nopmäg} \definition{1. one's mother's exchange sister}}
\item \entry{kin.}{\headword{noponda} \definition{1. one's father's exchange brother (still owes him)}}
\item \entry{kin.}{\headword{nyamällatt} \definition{1. woman's exchange sibling's child}}
\item \entry{kin.}{\headword{päzäg} \definition{1. brother-in-law (man's wife's brother or man's sister's husband; reciprocal)}}
\item \entry{kin.}{\headword{päzäg} \definition{1. sister-in-law (man's wife's sister)}}
\item \entry{kin.}{\headword{päzäg} \definition{1. in-laws}}
\item \entry{kin.}{\headword{pope} \definition{1. uncle (one's mother's brother)}}
\item \entry{kin.}{\headword{pope} \definition{1. uncle payment}}
\item \entry{kin.}{\headword{yae} \definition{1. mother}}
\item \entry{kin.}{\headword{yaya} \definition{1. father}}
\item \entry{kin.}{\headword{yäkäl} \definition{1. cousin}}
\item \entry{kin.}{\headword{yäkäl} \definition{1. uncle (one's mother's sister's husband)}}
\end{enumerate}

\section{loc.}
\begin{enumerate}
\item \entry{loc.}{\headword{amne} \definition{1. center, middle}}
\item \entry{loc.}{\headword{dowae} \definition{1. vicinity, proximity}}
\item \entry{loc.}{\headword{dowae} \definition{1. close together, next to each other, neighboring, adjacent}}
\item \entry{loc.}{\headword{dowae} \definition{1. straight}}
\item \entry{loc.}{\headword{ddäg} \definition{1. outside}}
\item \entry{loc.}{\headword{ekaklle} \definition{1. low}}
\item \entry{loc.}{\headword{eroe} \definition{1. side}}
\item \entry{loc.}{\headword{golloll} \definition{1. back, behind}}
\item \entry{loc.}{\headword{guwo} \definition{1. inside}}
\item \entry{loc.}{\headword{igi} \definition{1. bottom}}
\item \entry{loc.}{\headword{igi} \definition{1. underneath}}
\item \entry{loc.}{\headword{igi} \definition{1. underwear}}
\item \entry{loc.}{\headword{igi} \definition{1. under, beneath}}
\item \entry{loc.}{\headword{ik} \definition{1. inside}}
\item \entry{loc.}{\headword{imne} \definition{1. rear, behind, back}}
\item \entry{loc.}{\headword{ingoll} \definition{1. front}}
\item \entry{loc.}{\headword{käm2} \definition{1. underneath, under, beneath}}
\item \entry{loc.}{\headword{koll} \definition{1. inner part}}
\item \entry{loc.}{\headword{ku1} \definition{1. center, core, middle}}
\item \entry{loc.}{\headword{ku1} \definition{1. the ninth stage of coconut growth in which the endosperm of the fruit is fully formed and coconut water remains}}
\item \entry{loc.}{\headword{kum} \definition{1. butt, base, bottom, lower or back part of something}}
\item \entry{loc.}{\headword{makäp} \definition{1. inside, in, within, among}}
\item \entry{loc.}{\headword{matu} \definition{1. lower part}}
\item \entry{loc.}{\headword{menae} \definition{1. side, edge; vicinity}}
\item \entry{loc.}{\headword{ngattong} \definition{1. front}}
\item \entry{loc.}{\headword{pallall} \definition{1. side}}
\item \entry{loc.}{\headword{päk} \definition{1. edge}}
\item \entry{loc.}{\headword{toko} \definition{1. top}}
\item \entry{loc.}{\headword{tuk} \definition{1. top}}
\item \entry{loc.}{\headword{upe} \definition{1. outside, out}}
\item \entry{loc.}{\headword{utale} \definition{1. far}}
\end{enumerate}

\section{mod.}
\begin{enumerate}
\item \entry{mod.}{\headword{abade} \definition{1. future, later, upcoming, impending}}
\item \entry{mod.}{\headword{ai1} \definition{1. good}}
\item \entry{mod.}{\headword{ai1} \definition{1. modal adjective (expresses permission, possibility, obligation)}}
\item \entry{mod.}{\headword{ako} \definition{1. other}}
\item \entry{mod.}{\headword{ause} \definition{1. old (of a woman)}}
\item \entry{mod.}{\headword{ause} \definition{1. nonsingular form of ause}}
\item \entry{mod.}{\headword{awi} \definition{1. early}}
\item \entry{mod.}{\headword{batt1} \definition{1. adult, mature}}
\item \entry{mod.}{\headword{bɨt} \definition{1. dark}}
\item \entry{mod.}{\headword{bɨt} \definition{1. black}}
\item \entry{mod.}{\headword{bɨt} \definition{1. purple (color of sawis, purple yam)}}
\item \entry{mod.}{\headword{bodo} \definition{1. full}}
\item \entry{mod.}{\headword{bodo} \definition{1. nonsingular form of bodo}}
\item \entry{mod.}{\headword{buddo} \definition{1. heavy}}
\item \entry{mod.}{\headword{buddo} \definition{1. deep (of a voice)}}
\item \entry{mod.}{\headword{damong1} \definition{1. healthy, well}}
\item \entry{mod.}{\headword{dädär1} \definition{1. dry}}
\item \entry{mod.}{\headword{dädär1} \definition{1. hard}}
\item \entry{mod.}{\headword{dämädämäll} \definition{1. numb, paralyzed}}
\item \entry{mod.}{\headword{däroledärole} \definition{1. dry}}
\item \entry{mod.}{\headword{dɨl} \definition{1. bitter; sour}}
\item \entry{mod.}{\headword{dorko} \definition{1. dry}}
\item \entry{mod.}{\headword{du} \definition{1. wild (of plants)}}
\item \entry{mod.}{\headword{duwar} \definition{1. young}}
\item \entry{mod.}{\headword{ddage} \definition{1. with many branches}}
\item \entry{mod.}{\headword{ddäddäbeabag} \definition{1. independent, resourceful}}
\item \entry{mod.}{\headword{ddägnan ma} \definition{1. edible}}
\item \entry{mod.}{\headword{ddokddok2} \definition{1. blunt}}
\item \entry{mod.}{\headword{ddonddo} \definition{1. proud}}
\item \entry{mod.}{\headword{emaemae} \definition{1. various, many different kinds}}
\item \entry{mod.}{\headword{enanae} \definition{1. final; finally, at last}}
\item \entry{mod.}{\headword{enanae} \definition{1. eternal}}
\item \entry{mod.}{\headword{gabma} \definition{1. white, Western; foreign}}
\item \entry{mod.}{\headword{gagäll} \definition{1. bad, rotten}}
\item \entry{mod.}{\headword{gagäll} \definition{1. dead}}
\item \entry{mod.}{\headword{giddoll} \definition{1. menstruating, on one's period}}
\item \entry{mod.}{\headword{giddoll} \definition{1. in menopause}}
\item \entry{mod.}{\headword{goro} \definition{1. lush, overgrown, wild}}
\item \entry{mod.}{\headword{goro} \definition{1. jungle}}
\item \entry{mod.}{\headword{gudae1} \definition{1. early morning}}
\item \entry{mod.}{\headword{gudae1} \definition{1. earlier, previously, in the past, before}}
\item \entry{mod.}{\headword{gudne} \definition{1. old}}
\item \entry{mod.}{\headword{gullabgullab} \definition{1. fat}}
\item \entry{mod.}{\headword{gullbe} \definition{1. male animal}}
\item \entry{mod.}{\headword{gullbe} \definition{1. huge, big}}
\item \entry{mod.}{\headword{gullbe} \definition{1. married (of a female)}}
\item \entry{mod.}{\headword{gullbe} \definition{1. married (of a female)}}
\item \entry{mod.}{\headword{gwell} \definition{1. secret}}
\item \entry{mod.}{\headword{imne} \definition{1. after}}
\item \entry{mod.}{\headword{imne} \definition{1. from behind}}
\item \entry{mod.}{\headword{imne} \definition{1. behind, late; later, after}}
\item \entry{mod.}{\headword{imne} \definition{1. after, later}}
\item \entry{mod.}{\headword{imomdae} \definition{1. true, real, actual}}
\item \entry{mod.}{\headword{imomdae} \definition{1. correct, right}}
\item \entry{mod.}{\headword{imomdae} \definition{1. honest}}
\item \entry{mod.}{\headword{imomdae} \definition{1. honest}}
\item \entry{mod.}{\headword{imomdae} \definition{1. faith, religion}}
\item \entry{mod.}{\headword{imomdae} \definition{1. believer, faithful}}
\item \entry{mod.}{\headword{indrang} \definition{1. luminous, bright}}
\item \entry{mod.}{\headword{inngoeinngoe} \definition{1. shaky}}
\item \entry{mod.}{\headword{inu} \definition{1. asleep, sleeping}}
\item \entry{mod.}{\headword{inu} \definition{1. fast asleep, dead asleep}}
\item \entry{mod.}{\headword{inu} \definition{1. to guard in one's sleep, sleep with}}
\item \entry{mod.}{\headword{inu} \definition{1. sleepy}}
\item \entry{mod.}{\headword{iräpang} \definition{1. dirty}}
\item \entry{mod.}{\headword{itrell} \definition{1. ill, sick}}
\item \entry{mod.}{\headword{kakab} \definition{1. leftover, remaining}}
\item \entry{mod.}{\headword{kakakän} \definition{1. loose, floating, unbound}}
\item \entry{mod.}{\headword{kakakän} \definition{1. floating, loose}}
\item \entry{mod.}{\headword{kalmoe} \definition{1. pliable, bendable}}
\item \entry{mod.}{\headword{kalmoe} \definition{1. welcoming}}
\item \entry{mod.}{\headword{kalokalo} \definition{1. pliable, flexible, bendable}}
\item \entry{mod.}{\headword{kame1} \definition{1. unknown, unfamiliar}}
\item \entry{mod.}{\headword{kamekong} \definition{1. busy}}
\item \entry{mod.}{\headword{kandärmang1} \definition{1. sorry, apologetic, regretful}}
\item \entry{mod.}{\headword{kandärmang1} \definition{1. sorry, pitiful, unfortunate, sad}}
\item \entry{mod.}{\headword{katre} \definition{1. raised, elevated}}
\item \entry{mod.}{\headword{kädkäd1} \definition{1. cold}}
\item \entry{mod.}{\headword{kälae} \definition{1. small, little}}
\item \entry{mod.}{\headword{kälsäre} \definition{1. small, little}}
\item \entry{mod.}{\headword{källakällae} \definition{1. hospitable}}
\item \entry{mod.}{\headword{kämag} \definition{1. west, western}}
\item \entry{mod.}{\headword{känyär} \definition{1. quiet}}
\item \entry{mod.}{\headword{känyär} \definition{1. secret, secretly}}
\item \entry{mod.}{\headword{känyär} \definition{1. alone, by oneself (follows a genitive noun)}}
\item \entry{mod.}{\headword{känyär} \definition{1. secret}}
\item \entry{mod.}{\headword{känyär} \definition{1. secret}}
\item \entry{mod.}{\headword{känyär} \definition{1. silent, quiet}}
\item \entry{mod.}{\headword{känyär} \definition{1. to soothe}}
\item \entry{mod.}{\headword{känyär} \definition{1. secretly}}
\item \entry{mod.}{\headword{käp1} \definition{1. bearing fruit}}
\item \entry{mod.}{\headword{kili} \definition{1. happy}}
\item \entry{mod.}{\headword{kili} \definition{1. happy}}
\item \entry{mod.}{\headword{kili} \definition{1. unhappy, sad, annoyed}}
\item \entry{mod.}{\headword{kili} \definition{1. to greet}}
\item \entry{mod.}{\headword{kili} \definition{1. to rejoice, celebrate}}
\item \entry{mod.}{\headword{kili} \definition{1. happily, joyfully}}
\item \entry{mod.}{\headword{kinekineang} \definition{1. smart}}
\item \entry{mod.}{\headword{kire} \definition{1. unripe; raw, fresh}}
\item \entry{mod.}{\headword{kire} \definition{1. green}}
\item \entry{mod.}{\headword{koepang} \definition{1. sour}}
\item \entry{mod.}{\headword{koko2} \definition{1. sprouted}}
\item \entry{mod.}{\headword{konkon} \definition{1. crazy, mad, insane, mentally ill}}
\item \entry{mod.}{\headword{konkon} \definition{1. stupid, ignorant, foolish}}
\item \entry{mod.}{\headword{konkon} \definition{1. intoxicated, intoxicating, consciousness-altering, drunk}}
\item \entry{mod.}{\headword{kristen} \definition{1. Christian}}
\item \entry{mod.}{\headword{kuddäkuddäll} \definition{1. soft}}
\item \entry{mod.}{\headword{kuddäkuddäll} \definition{1. easy}}
\item \entry{mod.}{\headword{kuddäll} \definition{1. dead}}
\item \entry{mod.}{\headword{kuddäll} \definition{1. abandoned}}
\item \entry{mod.}{\headword{kuki} \definition{1. false, deceptive}}
\item \entry{mod.}{\headword{kukoll} \definition{1. healthy, fertile, vibrant}}
\item \entry{mod.}{\headword{kukoll} \definition{1. green}}
\item \entry{mod.}{\headword{kutt} \definition{1. thin, bony}}
\item \entry{mod.}{\headword{kutt} \definition{1. to shiver}}
\item \entry{mod.}{\headword{kwaratang} \definition{1. thin, skinny, slim (animate)}}
\item \entry{mod.}{\headword{lel} \definition{1. afraid, scared}}
\item \entry{mod.}{\headword{lel} \definition{1. shameful}}
\item \entry{mod.}{\headword{lel} \definition{1. brave, fearless, bold}}
\item \entry{mod.}{\headword{llama2} \definition{1. unsatisfied}}
\item \entry{mod.}{\headword{llama2} \definition{1. hesitant, reluctant}}
\item \entry{mod.}{\headword{lläng} \definition{1. sharp}}
\item \entry{mod.}{\headword{lläng} \definition{1. dull, blunt}}
\item \entry{mod.}{\headword{llokott} \definition{1. hard, firm}}
\item \entry{mod.}{\headword{llokott} \definition{1. difficult, hard}}
\item \entry{mod.}{\headword{llokott} \definition{1. strong}}
\item \entry{mod.}{\headword{llokott} \definition{1. hard}}
\item \entry{mod.}{\headword{llokott} \definition{1. difficult}}
\item \entry{mod.}{\headword{llokott} \definition{1. stubborn}}
\item \entry{mod.}{\headword{llokott} \definition{1. firmly, tightly, strongly}}
\item \entry{mod.}{\headword{llune} \definition{1. wild}}
\item \entry{mod.}{\headword{ma} \definition{1. sacred}}
\item \entry{mod.}{\headword{mam} \definition{1. bloody}}
\item \entry{mod.}{\headword{mam} \definition{1. red}}
\item \entry{mod.}{\headword{mam} \definition{1. red}}
\item \entry{mod.}{\headword{mamamemett} \definition{1. fast beat}}
\item \entry{mod.}{\headword{masar} \definition{1. ancestral, old (of one's grandparents' time)}}
\item \entry{mod.}{\headword{mängal} \definition{1. quick}}
\item \entry{mod.}{\headword{mängal} \definition{1. quick}}
\item \entry{mod.}{\headword{mängallang} \definition{1. strong}}
\item \entry{mod.}{\headword{mäse} \definition{1. not yet full, unsatisfied}}
\item \entry{mod.}{\headword{mätaru} \definition{1. calm, peaceful, quiet}}
\item \entry{mod.}{\headword{mätaru} \definition{1. peace officer}}
\item \entry{mod.}{\headword{melem} \definition{1. hardworking}}
\item \entry{mod.}{\headword{mer1} \definition{1. good}}
\item \entry{mod.}{\headword{mer1} \definition{1. holy}}
\item \entry{mod.}{\headword{mikutt} \definition{1. angry, mad}}
\item \entry{mod.}{\headword{mikutt} \definition{1. angry, short-tempered}}
\item \entry{mod.}{\headword{mikutt} \definition{1. calm}}
\item \entry{mod.}{\headword{mu} \definition{1. valuable}}
\item \entry{mod.}{\headword{mu} \definition{1. in debt}}
\item \entry{mod.}{\headword{mullae} \definition{1. enough}}
\item \entry{mod.}{\headword{mullae} \definition{1. able, can, be allowed}}
\item \entry{mod.}{\headword{mullae} \definition{1. nonsingular form of mullae}}
\item \entry{mod.}{\headword{muttbul} \definition{1. quiet}}
\item \entry{mod.}{\headword{naigae} \definition{1. south}}
\item \entry{mod.}{\headword{nongonongor} \definition{1. itchy}}
\item \entry{mod.}{\headword{nganae} \definition{1. wrapping around}}
\item \entry{mod.}{\headword{ngälngäl} \definition{1. round}}
\item \entry{mod.}{\headword{ngämingg} \definition{1. helpful, supportive}}
\item \entry{mod.}{\headword{ngänam} \definition{1. unrecognizable, obscure}}
\item \entry{mod.}{\headword{ngänyanyemmeny} \definition{1. mean, unkind}}
\item \entry{mod.}{\headword{ngowangowe} \definition{1. narrow}}
\item \entry{mod.}{\headword{o2} \definition{1. ripe}}
\item \entry{mod.}{\headword{pallta} \definition{1. flat}}
\item \entry{mod.}{\headword{pani} \definition{1. funny}}
\item \entry{mod.}{\headword{pälläm} \definition{1. visible}}
\item \entry{mod.}{\headword{pällämpälläm} \definition{1. white, Western}}
\item \entry{mod.}{\headword{pättäk} \definition{1. short}}
\item \entry{mod.}{\headword{pättäk} \definition{1. a bit short}}
\item \entry{mod.}{\headword{pättäk} \definition{1. nonsingular form of pättäk}}
\item \entry{mod.}{\headword{penmällpenmäll} \definition{1. spotted}}
\item \entry{mod.}{\headword{petapeta} \definition{1. thin (inanimate)}}
\item \entry{mod.}{\headword{petapeta} \definition{1. shallow}}
\item \entry{mod.}{\headword{poapoa} \definition{1. light (in weight)}}
\item \entry{mod.}{\headword{poapoa} \definition{1. light, bright}}
\item \entry{mod.}{\headword{poapoa} \definition{1. easy}}
\item \entry{mod.}{\headword{pobllem} \definition{1. smooth}}
\item \entry{mod.}{\headword{pollo} \definition{1. young}}
\item \entry{mod.}{\headword{pop} \definition{1. holey, having a hole}}
\item \entry{mod.}{\headword{poper} \definition{1. afraid, startled, scared}}
\item \entry{mod.}{\headword{poper} \definition{1. to scare, startle, frighten}}
\item \entry{mod.}{\headword{poper} \definition{1. to surprise}}
\item \entry{mod.}{\headword{pri} \definition{1. free}}
\item \entry{mod.}{\headword{pukong} \definition{1. thick}}
\item \entry{mod.}{\headword{pumi} \definition{1. exhausting, tiring, strenuous}}
\item \entry{mod.}{\headword{sapang} \definition{1. separate, apart, different; own, personal}}
\item \entry{mod.}{\headword{sapang} \definition{1. various, many different}}
\item \entry{mod.}{\headword{sawe} \definition{1. left}}
\item \entry{mod.}{\headword{sägädag} \definition{1. yellow}}
\item \entry{mod.}{\headword{sem2} \definition{1. same}}
\item \entry{mod.}{\headword{sisor} \definition{1. new}}
\item \entry{mod.}{\headword{sisor} \definition{1. young}}
\item \entry{mod.}{\headword{sisor} \definition{1. newborn, infant}}
\item \entry{mod.}{\headword{sisri} \definition{1. this; current}}
\item \entry{mod.}{\headword{su} \definition{1. secret}}
\item \entry{mod.}{\headword{tarketarke} \definition{1. brittle}}
\item \entry{mod.}{\headword{tämamae} \definition{1. whole, entire}}
\item \entry{mod.}{\headword{täränga} \definition{1. low, scant, shallow}}
\item \entry{mod.}{\headword{täre} \definition{1. holy, sacred}}
\item \entry{mod.}{\headword{tärpi} \definition{1. slippery, smooth}}
\item \entry{mod.}{\headword{tärpi} \definition{1. nonsingular form of tärpi}}
\item \entry{mod.}{\headword{teyateyar} \definition{1. shallow}}
\item \entry{mod.}{\headword{tomowang} \definition{1. sour; bitter}}
\item \entry{mod.}{\headword{tonang} \definition{1. careful, cautious}}
\item \entry{mod.}{\headword{tonang} \definition{1. carefully, cautiously}}
\item \entry{mod.}{\headword{tubu1} \definition{1. short}}
\item \entry{mod.}{\headword{tubu1} \definition{1. a bit short}}
\item \entry{mod.}{\headword{tubu1} \definition{1. nonsingular form of tubu}}
\item \entry{mod.}{\headword{tum1} \definition{1. angry}}
\item \entry{mod.}{\headword{tupi1} \definition{1. tall}}
\item \entry{mod.}{\headword{tupi1} \definition{1. long}}
\item \entry{mod.}{\headword{tupi1} \definition{1. nonsingular form of tupi}}
\item \entry{mod.}{\headword{ttagbeag} \definition{1. disorganized, careless}}
\item \entry{mod.}{\headword{ttam1} \definition{1. alive}}
\item \entry{mod.}{\headword{ttaputtapung} \definition{1. joined together}}
\item \entry{mod.}{\headword{ttänttäm} \definition{1. hot}}
\item \entry{mod.}{\headword{ttätt} \definition{1. right}}
\item \entry{mod.}{\headword{ttättle} \definition{1. correct, proper}}
\item \entry{mod.}{\headword{ttättle} \definition{1. straight}}
\item \entry{mod.}{\headword{ttomoll} \definition{1. deaf}}
\item \entry{mod.}{\headword{ttongo1} \definition{1. unique}}
\item \entry{mod.}{\headword{ttowaemang} \definition{1. miraculous}}
\item \entry{mod.}{\headword{ttowaemang} \definition{1. miracle}}
\item \entry{mod.}{\headword{ttupe} \definition{1. bad (of a coconut)}}
\item \entry{mod.}{\headword{ubemang} \definition{1. wide}}
\item \entry{mod.}{\headword{ulle} \definition{1. big, large, great}}
\item \entry{mod.}{\headword{ulle} \definition{1. important, great}}
\item \entry{mod.}{\headword{ulle} \definition{1. long; tall}}
\item \entry{mod.}{\headword{ulle} \definition{1. entire, whole}}
\item \entry{mod.}{\headword{wänänang} \definition{1. provider, bringing back food for one's family}}
\item \entry{mod.}{\headword{yänkllollang} \definition{1. forgetful}}
\item \entry{mod.}{\headword{zowag} \definition{1. hoarse}}
\end{enumerate}

\section{n.}
\begin{enumerate}
\item \entry{n.}{\headword{abor} \definition{1. sago beater}}
\item \entry{n.}{\headword{adräl} \definition{1. type of tree}}
\item \entry{n.}{\headword{aduwi} \definition{1. type of tree}}
\item \entry{n.}{\headword{aeb1} \definition{1. black-billed/yellow-legged brushturkey}}
\item \entry{n.}{\headword{aebaeb} \definition{1. type of tree}}
\item \entry{n.}{\headword{aengap} \definition{1. type of big yam with a white interior and no thorns}}
\item \entry{n.}{\headword{aetruaetru} \definition{1. type of tree with edible blue fruits and white flowers, found in the bush and along big creeks}}
\item \entry{n.}{\headword{ag} \definition{1. morning (approx. 5 AM–11 AM)}}
\item \entry{n.}{\headword{ag} \definition{1. breakfast}}
\item \entry{n.}{\headword{ai skul} \definition{1. secondary school, high school}}
\item \entry{n.}{\headword{aitar} \definition{1. type of palm tree that grows along creeks with pools; shoots and soft trunk are edible; similar to dumar}}
\item \entry{n.}{\headword{ali} \definition{1. conch shell}}
\item \entry{n.}{\headword{alläp} \definition{1. kundu drum}}
\item \entry{n.}{\headword{alläp} \definition{1. guitar}}
\item \entry{n.}{\headword{allko} \definition{1. fly (insect)}}
\item \entry{n.}{\headword{allko kukut} \definition{1. blue fly}}
\item \entry{n.}{\headword{allko wallägnewallägnen} \definition{1. type of sago}}
\item \entry{n.}{\headword{am} \definition{1. internode (section of bamboo or sugarcane, separated by nodes)}}
\item \entry{n.}{\headword{ama2} \definition{1. hammer}}
\item \entry{n.}{\headword{amamär} \definition{1. woven rope}}
\item \entry{n.}{\headword{amäramär} \definition{1. braid}}
\item \entry{n.}{\headword{amba} \definition{1. type of tree found in the grassland}}
\item \entry{n.}{\headword{amtet} \definition{1. breath}}
\item \entry{n.}{\headword{ankom} \definition{1. ant}}
\item \entry{n.}{\headword{anggog} \definition{1. thigh}}
\item \entry{n.}{\headword{angkäpäll} \definition{1. type of big tree found in the bush with white flowers and fruit, which cassowary eat when they fall}}
\item \entry{n.}{\headword{anyke} \definition{1. spirit}}
\item \entry{n.}{\headword{anyke} \definition{1. to be in shock}}
\item \entry{n.}{\headword{anyke} \definition{1. picture, image, reflection}}
\item \entry{n.}{\headword{ap} \definition{1. grassland, savannah}}
\item \entry{n.}{\headword{apapi} \definition{1. butterfly}}
\item \entry{n.}{\headword{apapi bärät} \definition{1. type of yam}}
\item \entry{n.}{\headword{apapun} \definition{1. type of short grass that disperses seeds by attaching to animal fur}}
\item \entry{n.}{\headword{apgllu} \definition{1. small plant with a root that makes a maroon pigment for dyeing grass skirts when mixed with ash (e.g. Acacia ash or coconut ash). When not mixed with ash, it makes a yellow pigment.}}
\item \entry{n.}{\headword{arabuni} \definition{1. red backed buttonquail}}
\item \entry{n.}{\headword{arle} \definition{1. scream}}
\item \entry{n.}{\headword{arup} \definition{1. clothing type}}
\item \entry{n.}{\headword{as} \definition{1. type of introduced banana}}
\item \entry{n.}{\headword{asa bume} \definition{1. type of bird}}
\item \entry{n.}{\headword{asiasi} \definition{1. type of large tree that grows in the bush and along big creeks, with white flowers and big red fruits eaten by cassowary and children; the trunk is used to make canoes}}
\item \entry{n.}{\headword{asip} \definition{1. type of introduced banana}}
\item \entry{n.}{\headword{atata kottllam} \definition{1. type of turtle}}
\item \entry{n.}{\headword{atrepo} \definition{1. taro type}}
\item \entry{n.}{\headword{au} \definition{1. burial}}
\item \entry{n.}{\headword{au} \definition{1. grave}}
\item \entry{n.}{\headword{aulämän} \definition{1. conception}}
\item \entry{n.}{\headword{auri} \definition{1. metal}}
\item \entry{n.}{\headword{ause} \definition{1. old woman}}
\item \entry{n.}{\headword{ause} \definition{1. nonsingular form of ause}}
\item \entry{n.}{\headword{awäll} \definition{1. type of tree that grows in the bush (where yam gardens are made) with white and blue flowers and dark, edible fruit}}
\item \entry{n.}{\headword{awe1} \definition{1. savannah}}
\item \entry{n.}{\headword{awe2} \definition{1. cassowary (used when hunting)}}
\item \entry{n.}{\headword{awi} \definition{1. evening (approx. 5 PM till dark)}}
\item \entry{n.}{\headword{bab} \definition{1. type of small yam with a white interior}}
\item \entry{n.}{\headword{babaem} \definition{1. season characterized by wind and going hunting (fourth season; corresponds to late February)}}
\item \entry{n.}{\headword{babdu} \definition{1. type of taro}}
\item \entry{n.}{\headword{badar} \definition{1. type of tree that grows in the grassland with white flowers}}
\item \entry{n.}{\headword{baebol} \definition{1. Bible}}
\item \entry{n.}{\headword{baet} \definition{1. cuscus}}
\item \entry{n.}{\headword{bagama} \definition{1. collar}}
\item \entry{n.}{\headword{bagen} \definition{1. type of big taro}}
\item \entry{n.}{\headword{ballma} \definition{1. type of biting bee found in trees}}
\item \entry{n.}{\headword{ballme} \definition{1. dawn, daybreak}}
\item \entry{n.}{\headword{ballo bällabällott} \definition{1. type of big taro}}
\item \entry{n.}{\headword{bamearoro} \definition{1. type of mushroom}}
\item \entry{n.}{\headword{band} \definition{1. type of tree}}
\item \entry{n.}{\headword{bandra} \definition{1. song}}
\item \entry{n.}{\headword{banggo} \definition{1. type of long yam with a white interior, thorns, and no hair}}
\item \entry{n.}{\headword{banggu} \definition{1. headdress}}
\item \entry{n.}{\headword{baob} \definition{1. water lily}}
\item \entry{n.}{\headword{bargae} \definition{1. type of fish}}
\item \entry{n.}{\headword{batri} \definition{1. battery}}
\item \entry{n.}{\headword{batri} \definition{1. battery}}
\item \entry{n.}{\headword{batt2} \definition{1. central lateral beam of a house}}
\item \entry{n.}{\headword{baur} \definition{1. type of spear}}
\item \entry{n.}{\headword{bawa} \definition{1. season characterized by hunting and fishing in heavy rain (ninth season; corresponds to June)}}
\item \entry{n.}{\headword{bawa} \definition{1. rain shower}}
\item \entry{n.}{\headword{bawa} \definition{1. season characterized by hunting and fishing in light rain (tenth season; corresponds to July)}}
\item \entry{n.}{\headword{bawa} \definition{1. wave}}
\item \entry{n.}{\headword{bawa} \definition{1. drizzle}}
\item \entry{n.}{\headword{bazere} \definition{1. type of purple yam with hairs and no thorns}}
\item \entry{n.}{\headword{bäbnge} \definition{1. type of palm with coconuts with a light yellow and green exocarp}}
\item \entry{n.}{\headword{bäd} \definition{1. type of large tree that grows in the grassland with wood used for firewood and bark used for building walls}}
\item \entry{n.}{\headword{bädab} \definition{1. dawn}}
\item \entry{n.}{\headword{bädab} \definition{1. morning star}}
\item \entry{n.}{\headword{bädma} \definition{1. type of medicinal plant}}
\item \entry{n.}{\headword{bädma} \definition{1. planting a bädma plant as a gesture of peace}}
\item \entry{n.}{\headword{bädmaol} \definition{1. small sago flower}}
\item \entry{n.}{\headword{bädde1} \definition{1. type of large tree found by rivers}}
\item \entry{n.}{\headword{bädde2} \definition{1. type of big taro}}
\item \entry{n.}{\headword{bägallem} \definition{1. type of tree that grows in the grassland with wood used for firewood and yellow fruit}}
\item \entry{n.}{\headword{bägäbägäl} \definition{1. Achilles/calcaneal tendon (on the back of ankle)}}
\item \entry{n.}{\headword{bägäl} \definition{1. bow}}
\item \entry{n.}{\headword{bägäl} \definition{1. gunfire}}
\item \entry{n.}{\headword{bägäl} \definition{1. part of bow}}
\item \entry{n.}{\headword{bägäl} \definition{1. place for bows and spears}}
\item \entry{n.}{\headword{bägäl} \definition{1. extra bowstring}}
\item \entry{n.}{\headword{bägäl} \definition{1. small bow}}
\item \entry{n.}{\headword{bägäm} \definition{1. type of tree found by creeks with bark used for weaving bags or strong rope}}
\item \entry{n.}{\headword{bägem} \definition{1. type of tree with pink flowers and fist-sized fruit with a pit}}
\item \entry{n.}{\headword{bäkän} \definition{1. type of cultivated plant with leaves eaten with sago}}
\item \entry{n.}{\headword{bäll} \definition{1. thigh}}
\item \entry{n.}{\headword{bäll} \definition{1. femur}}
\item \entry{n.}{\headword{bälläg1} \definition{1. type of introduced banana}}
\item \entry{n.}{\headword{bälläg2} \definition{1. Areca palm}}
\item \entry{n.}{\headword{bällämbäll} \definition{1. thought}}
\item \entry{n.}{\headword{bällämbäll} \definition{1. thoughtless, apathetic, uncaring}}
\item \entry{n.}{\headword{bällkäp} \definition{1. "ant egg (large)" - ant pupae. Edible.}}
\item \entry{n.}{\headword{bällma} \definition{1. spit, saliva}}
\item \entry{n.}{\headword{bällma} \definition{1. spit, saliva}}
\item \entry{n.}{\headword{bällma} \definition{1. to spit at}}
\item \entry{n.}{\headword{bälltoe} \definition{1. type of tree}}
\item \entry{n.}{\headword{bämäng} \definition{1. type of tree}}
\item \entry{n.}{\headword{bänäm} \definition{1. type of very small insect that lives on bandicoots}}
\item \entry{n.}{\headword{bänäm} \definition{1. thorns on sago leaves}}
\item \entry{n.}{\headword{bändam} \definition{1. type of tree}}
\item \entry{n.}{\headword{bänz1} \definition{1. mosquito}}
\item \entry{n.}{\headword{bänz2} \definition{1. type of biting bee found in trees}}
\item \entry{n.}{\headword{bänzibänzi} \definition{1. type of sago}}
\item \entry{n.}{\headword{bäng} \definition{1. firestick (to start a fire)}}
\item \entry{n.}{\headword{bäräbäräl} \definition{1. type of bird}}
\item \entry{n.}{\headword{bärät} \definition{1. type of small yam}}
\item \entry{n.}{\headword{bärke} \definition{1. Papuan eclectus}}
\item \entry{n.}{\headword{bärkebärke} \definition{1. type of algae that can be red or green like a parrot}}
\item \entry{n.}{\headword{bät1} \definition{1. type of tree}}
\item \entry{n.}{\headword{bätäny} \definition{1. type of tree}}
\item \entry{n.}{\headword{bätte} \definition{1. type of snake}}
\item \entry{n.}{\headword{beatururang} \definition{1. season characterized by thunderstorms and flooding (second season; corresponds to early February)}}
\item \entry{n.}{\headword{bebe1} \definition{1. type of pandanus with long fruit (~2 feet)}}
\item \entry{n.}{\headword{bebi} \definition{1. baby}}
\item \entry{n.}{\headword{begere} \definition{1. type of long purple yam}}
\item \entry{n.}{\headword{bel} \definition{1. bell}}
\item \entry{n.}{\headword{bem} \definition{1. sea, ocean}}
\item \entry{n.}{\headword{benanbenan} \definition{1. type of spear}}
\item \entry{n.}{\headword{benmäll} \definition{1. shine, flash}}
\item \entry{n.}{\headword{bengae} \definition{1. roof, roofing}}
\item \entry{n.}{\headword{bette} \definition{1. crimson finch}}
\item \entry{n.}{\headword{beyat} \definition{1. type of tree that grows in the bush with wood used for house sticks}}
\item \entry{n.}{\headword{bib2} \definition{1. spring water}}
\item \entry{n.}{\headword{bible} \definition{1. type of big taro}}
\item \entry{n.}{\headword{bibol} \definition{1. type of bird}}
\item \entry{n.}{\headword{biboz} \definition{1. fairywren (emperor, white-shouldered)}}
\item \entry{n.}{\headword{big} \definition{1. type of very large tree that grows in the bush with wood used for firewood, especially when camping.}}
\item \entry{n.}{\headword{big} \definition{1. type of grub}}
\item \entry{n.}{\headword{bigma} \definition{1. enclosure, pen, sty}}
\item \entry{n.}{\headword{bikme} \definition{1. type of palm tree with hanging, poisonous yellow and green fruits that can be eaten after being buried by the creek for up to 2 years and then cooked on the fire}}
\item \entry{n.}{\headword{bikme} \definition{1. type of bikme palm with very chewy fruit}}
\item \entry{n.}{\headword{bikme} \definition{1. type of bikme palm with yellow fruit that is very hard}}
\item \entry{n.}{\headword{bikme} \definition{1. type of bikme palm with fruit that is soft like sago}}
\item \entry{n.}{\headword{bikme} \definition{1. type of bikme palm with fruit that is medium-firm}}
\item \entry{n.}{\headword{bikme tutu} \definition{1. type of string game}}
\item \entry{n.}{\headword{bikwem} \definition{1. fireplace}}
\item \entry{n.}{\headword{bile} \definition{1. salt}}
\item \entry{n.}{\headword{bilod} \definition{1. type of spear}}
\item \entry{n.}{\headword{bin} \definition{1. name}}
\item \entry{n.}{\headword{bin} \definition{1. namesake}}
\item \entry{n.}{\headword{bin} \definition{1. serious}}
\item \entry{n.}{\headword{bir} \definition{1. spit, skewer}}
\item \entry{n.}{\headword{biro} \definition{1. pen (writing implement)}}
\item \entry{n.}{\headword{bisbis} \definition{1. type of stingless bee}}
\item \entry{n.}{\headword{bisel} \definition{1. type of sago that grows tall and wide}}
\item \entry{n.}{\headword{bisnis} \definition{1. business}}
\item \entry{n.}{\headword{bitän} \definition{1. type of animal}}
\item \entry{n.}{\headword{bittott} \definition{1. grey squirrel}}
\item \entry{n.}{\headword{biwiz} \definition{1. type of large tree that grows in the bush with purple flowers and wood used for kundu drums and canoes}}
\item \entry{n.}{\headword{biye} \definition{1. taro}}
\item \entry{n.}{\headword{bɨd} \definition{1. gum tree}}
\item \entry{n.}{\headword{bɨk} \definition{1. poisoned creek}}
\item \entry{n.}{\headword{blengud} \definition{1. blanket}}
\item \entry{n.}{\headword{bllablla} \definition{1. type of cordyline with big leaves, red fruit, and leaves that are used to fan fire and tied around the waist and chest for dancing}}
\item \entry{n.}{\headword{bllolla} \definition{1. type of tree}}
\item \entry{n.}{\headword{bob1} \definition{1. flood}}
\item \entry{n.}{\headword{bob1} \definition{1. flooded}}
\item \entry{n.}{\headword{bob2} \definition{1. type of tree}}
\item \entry{n.}{\headword{bob2} \definition{1. type of mushroom}}
\item \entry{n.}{\headword{bobngätt} \definition{1. place name}}
\item \entry{n.}{\headword{bod} \definition{1. mouth}}
\item \entry{n.}{\headword{bod} \definition{1. beak}}
\item \entry{n.}{\headword{bod} \definition{1. lip of bag (braid that goes along the rim of the bag to finish it)}}
\item \entry{n.}{\headword{bod} \definition{1. lip}}
\item \entry{n.}{\headword{bodobodom} \definition{1. type of biting ant}}
\item \entry{n.}{\headword{boddo} \definition{1. type of large tree with a big trunk, fruit that deer eat, and aerial prop roots}}
\item \entry{n.}{\headword{boe} \definition{1. type of cultivated tree with edible indigo fruit and white flowers that attract birds and butterflies}}
\item \entry{n.}{\headword{bog} \definition{1. type of taro}}
\item \entry{n.}{\headword{boga} \definition{1. type of bird}}
\item \entry{n.}{\headword{boge} \definition{1. mudfish}}
\item \entry{n.}{\headword{bogel} \definition{1. seaweed}}
\item \entry{n.}{\headword{bogobogo} \definition{1. type of small yam with a pure white interior}}
\item \entry{n.}{\headword{boko} \definition{1. type of lizard}}
\item \entry{n.}{\headword{bol} \definition{1. ball}}
\item \entry{n.}{\headword{bolod} \definition{1. sugarcane}}
\item \entry{n.}{\headword{bollga} \definition{1. type of sago}}
\item \entry{n.}{\headword{boma} \definition{1. stump}}
\item \entry{n.}{\headword{bomall} \definition{1. type of tree that grows in the bush with bark used to make sago baskets}}
\item \entry{n.}{\headword{bombom} \definition{1. type of tree with big leaves and soft wood that floats and is carved by children}}
\item \entry{n.}{\headword{bomo} \definition{1. aerial root (e.g. of pandanus)}}
\item \entry{n.}{\headword{bonzro} \definition{1. dancing on the side playfully}}
\item \entry{n.}{\headword{bonydre} \definition{1. goshawk (grey-headed, brown); collared sparrowhawk}}
\item \entry{n.}{\headword{bor} \definition{1. scar}}
\item \entry{n.}{\headword{borale} \definition{1. traditional bamboo flute used to scare wallabies}}
\item \entry{n.}{\headword{bore} \definition{1. traditional bamboo pipe for smoking tobacco}}
\item \entry{n.}{\headword{bormop} \definition{1. type of spear}}
\item \entry{n.}{\headword{boser} \definition{1. rock found in creek}}
\item \entry{n.}{\headword{bott} \definition{1. boat}}
\item \entry{n.}{\headword{botta} \definition{1. lateral beam placed directly on the house post}}
\item \entry{n.}{\headword{buata} \definition{1. betel nut, areca nut (fruit of Areca catechu)}}
\item \entry{n.}{\headword{budar} \definition{1. grub, larva}}
\item \entry{n.}{\headword{budombudom} \definition{1. red ants}}
\item \entry{n.}{\headword{buddo} \definition{1. problem, issue}}
\item \entry{n.}{\headword{buddo} \definition{1. weight}}
\item \entry{n.}{\headword{buddo} \definition{1. carrying a load}}
\item \entry{n.}{\headword{buddo} \definition{1. nonsingular form of buddo}}
\item \entry{n.}{\headword{bugu} \definition{1. sheath, base, midrib (of a palm leaf)}}
\item \entry{n.}{\headword{buidde} \definition{1. club (weapon)}}
\item \entry{n.}{\headword{buitu} \definition{1. stick with a round base}}
\item \entry{n.}{\headword{buk} \definition{1. book}}
\item \entry{n.}{\headword{bulwem} \definition{1. type of big yam with a white and light purple interior and no hairs}}
\item \entry{n.}{\headword{bullalla} \definition{1. type of flower}}
\item \entry{n.}{\headword{bullull} \definition{1. Papuan frogmouth}}
\item \entry{n.}{\headword{bullull} \definition{1. rufous owl}}
\item \entry{n.}{\headword{bumo} \definition{1. tucker bag}}
\item \entry{n.}{\headword{bumrel} \definition{1. beetle}}
\item \entry{n.}{\headword{bun} \definition{1. head}}
\item \entry{n.}{\headword{bun} \definition{1. part of a bow}}
\item \entry{n.}{\headword{bun} \definition{1. mouth (of a river)}}
\item \entry{n.}{\headword{bun} \definition{1. owner}}
\item \entry{n.}{\headword{bun} \definition{1. dandruff}}
\item \entry{n.}{\headword{bun} \definition{1. head}}
\item \entry{n.}{\headword{bun} \definition{1. hair (on head)}}
\item \entry{n.}{\headword{bun} \definition{1. skull}}
\item \entry{n.}{\headword{bun} \definition{1. instructor, leader}}
\item \entry{n.}{\headword{bun} \definition{1. smart}}
\item \entry{n.}{\headword{bunbun} \definition{1. type of plant}}
\item \entry{n.}{\headword{bunkälle bunkälle} \definition{1. type of game where the players tie hair and hide}}
\item \entry{n.}{\headword{bunkom tätäp} \definition{1. hair tied with string}}
\item \entry{n.}{\headword{bunkombunkom} \definition{1. type of tall, big tree that grows along creek with wood used for canoes}}
\item \entry{n.}{\headword{bunkuttang} \definition{1. catfish}}
\item \entry{n.}{\headword{bunmat} \definition{1. center of a garden}}
\item \entry{n.}{\headword{bur} \definition{1. type of bird}}
\item \entry{n.}{\headword{burag} \definition{1. bride price (given to the bride's family by the groom)}}
\item \entry{n.}{\headword{burara} \definition{1. water lily}}
\item \entry{n.}{\headword{buwo} \definition{1. type of native banana}}
\item \entry{n.}{\headword{buz} \definition{1. type of cultivated tree with fist-sized, edible green fruit and light purple flowers}}
\item \entry{n.}{\headword{dabit} \definition{1. palm spear}}
\item \entry{n.}{\headword{dadargu} \definition{1. disturbance}}
\item \entry{n.}{\headword{dadär} \definition{1. type of net suspended in the water by sticks}}
\item \entry{n.}{\headword{dade1} \definition{1. yam stick}}
\item \entry{n.}{\headword{dadel} \definition{1. harvest}}
\item \entry{n.}{\headword{dadel} \definition{1. season of harvesting young gardens (seventh season; corresponds to early May)}}
\item \entry{n.}{\headword{daendae} \definition{1. flowering plant that is said to be ancestral to the area, with many types; flowers used as adornment when dancing}}
\item \entry{n.}{\headword{daga} \definition{1. betel}}
\item \entry{n.}{\headword{daindaim} \definition{1. drizzle}}
\item \entry{n.}{\headword{dale} \definition{1. ash}}
\item \entry{n.}{\headword{dang} \definition{1. length (of a house)}}
\item \entry{n.}{\headword{dang} \definition{1. corner post}}
\item \entry{n.}{\headword{dangkälmang} \definition{1. type of medium-sized, long yam with a white interior and thorns}}
\item \entry{n.}{\headword{dangne} \definition{1. crawling vine that grows in the grassland with purple flowers; used to make rope}}
\item \entry{n.}{\headword{dangne} \definition{1. type of grub}}
\item \entry{n.}{\headword{dape} \definition{1. drum head}}
\item \entry{n.}{\headword{dara1} \definition{1. type of traditional medicine}}
\item \entry{n.}{\headword{dara2} \definition{1. vine_type}}
\item \entry{n.}{\headword{daradara} \definition{1. type of vine with yellow fruit that are red when ripe}}
\item \entry{n.}{\headword{darkukiny} \definition{1. type of grass}}
\item \entry{n.}{\headword{darombe} \definition{1. mouth harp. quarter-moon-shaped bamboo flute with honey inside; takes a day to make}}
\item \entry{n.}{\headword{dauma} \definition{1. type of introduced banana}}
\item \entry{n.}{\headword{däba1} \definition{1. type of tree that grows in the grassland with leaves used to wrap sago and durable wood used for kundu drums, house posts, and formerly, bridges}}
\item \entry{n.}{\headword{däbi} \definition{1. green-backed honeyeater}}
\item \entry{n.}{\headword{dädäk} \definition{1. wall}}
\item \entry{n.}{\headword{dädär1} \definition{1. stone, rock}}
\item \entry{n.}{\headword{dädär1} \definition{1. stone}}
\item \entry{n.}{\headword{dädär2} \definition{1. type of big taro}}
\item \entry{n.}{\headword{däen} \definition{1. type of snake}}
\item \entry{n.}{\headword{däg} \definition{1. hand, group, bunch, set}}
\item \entry{n.}{\headword{dägäldägäl} \definition{1. intestines}}
\item \entry{n.}{\headword{dägmar} \definition{1. tongue}}
\item \entry{n.}{\headword{dägmar} \definition{1. spoon}}
\item \entry{n.}{\headword{däkna} \definition{1. small black termite mound that burns for a long time; after a woman gives birth, it is heated in the fire, wrapped in bark and cloth, placed under a mat, and used to warm the woman's stomach}}
\item \entry{n.}{\headword{däm} \definition{1. plant}}
\item \entry{n.}{\headword{däm ibenen} \definition{1. season when crops are planted (fifteenth season; corresponds to late November)}}
\item \entry{n.}{\headword{dämar} \definition{1. type of palm tree (~2 m) that used to be cooked and eaten; also fed to pigs}}
\item \entry{n.}{\headword{dämbag} \definition{1. lazy person, weak person}}
\item \entry{n.}{\headword{dänäk} \definition{1. type of small bush with reddish fruit that is black when ripe}}
\item \entry{n.}{\headword{dändak} \definition{1. type of purple yam with purple skin and no thorns or hairs}}
\item \entry{n.}{\headword{dändäräm} \definition{1. type of small tree with white and purple flowers and many hard, marble-sized, green fruit that children play with}}
\item \entry{n.}{\headword{dändäräm} \definition{1. type of marble game}}
\item \entry{n.}{\headword{dänräp} \definition{1. fish scale}}
\item \entry{n.}{\headword{dänräp} \definition{1. scab}}
\item \entry{n.}{\headword{dängam} \definition{1. Blyth's hornbil}}
\item \entry{n.}{\headword{dänyäk} \definition{1. small plant that grows in the grassland with purple and white flowers and blue fruit that children like to eat}}
\item \entry{n.}{\headword{där} \definition{1. pair}}
\item \entry{n.}{\headword{däräng} \definition{1. dog}}
\item \entry{n.}{\headword{däräng olleolle} \definition{1. type of possum-like animal with spotted skin}}
\item \entry{n.}{\headword{därängbun} \definition{1. type of pandanus with a curved fruit shaped like a dog's head}}
\item \entry{n.}{\headword{därängge} \definition{1. small orchid with blue, yellow, white, purple flowers}}
\item \entry{n.}{\headword{däränggedärängge} \definition{1. large wild orchid}}
\item \entry{n.}{\headword{därba} \definition{1. type of snake}}
\item \entry{n.}{\headword{därmir1} \definition{1. type of introduced banana}}
\item \entry{n.}{\headword{därmir2} \definition{1. type of tree that is used to treat sores}}
\item \entry{n.}{\headword{därollog} \definition{1. brolga}}
\item \entry{n.}{\headword{därunggu} \definition{1. bamboo tube}}
\item \entry{n.}{\headword{dektta} \definition{1. doctor}}
\item \entry{n.}{\headword{del} \definition{1. coconut lorikeet}}
\item \entry{n.}{\headword{dem} \definition{1. type of sago used for paints}}
\item \entry{n.}{\headword{deodeo} \definition{1. termite}}
\item \entry{n.}{\headword{diaba} \definition{1. type of spear}}
\item \entry{n.}{\headword{dibie} \definition{1. spectacled longbill}}
\item \entry{n.}{\headword{diboz} \definition{1. type of bird}}
\item \entry{n.}{\headword{digodigol} \definition{1. type of tree}}
\item \entry{n.}{\headword{digol} \definition{1. type of big, tall tree that grows near gardens in the bush with white flowers, green fruits, and special, valuable red wood that is used for ax handles}}
\item \entry{n.}{\headword{dikun} \definition{1. deacon}}
\item \entry{n.}{\headword{dimes} \definition{1. type of cultivated tree with sour, mango-like fruit}}
\item \entry{n.}{\headword{dindu} \definition{1. race}}
\item \entry{n.}{\headword{dini} \definition{1. type of small tree that grows in the bush with white flowers and red fruit that cassowaries like to eat}}
\item \entry{n.}{\headword{dinidini} \definition{1. type of small tree that grows in the bush with red fruit}}
\item \entry{n.}{\headword{dinggel} \definition{1. sugar glider}}
\item \entry{n.}{\headword{dinggi} \definition{1. dinghy}}
\item \entry{n.}{\headword{dinggoll} \definition{1. opossum}}
\item \entry{n.}{\headword{dirindi1} \definition{1. type of large yam with a white interior and thorns; with or without hairs}}
\item \entry{n.}{\headword{dirindi2} \definition{1. type of tree}}
\item \entry{n.}{\headword{dirom} \definition{1. southern cassowary}}
\item \entry{n.}{\headword{dirom} \definition{1. cassowary egg}}
\item \entry{n.}{\headword{dirom} \definition{1. to menstruate for the first time}}
\item \entry{n.}{\headword{dirom käp2} \definition{1. type of small taro}}
\item \entry{n.}{\headword{dirom mas} \definition{1. Cassowary fibula, the smaller leg bone that is next to the larger leg bone.}}
\item \entry{n.}{\headword{diromdirom} \definition{1. type of small tree with round, flat, edible fruit that are green and red when ripe}}
\item \entry{n.}{\headword{distrik} \definition{1. district}}
\item \entry{n.}{\headword{dit} \definition{1. type of cane used for building houses, bows, and canoes}}
\item \entry{n.}{\headword{do2} \definition{1. handle}}
\item \entry{n.}{\headword{do2} \definition{1. femur}}
\item \entry{n.}{\headword{dogma} \definition{1. type of tree that grows in the bush with wood that smells like matches and is good for house posts}}
\item \entry{n.}{\headword{domäll} \definition{1. type of pandanus}}
\item \entry{n.}{\headword{domäll} \definition{1. old-style sewn mat made of domäll pandanus}}
\item \entry{n.}{\headword{dompa} \definition{1. type of blunt arrow}}
\item \entry{n.}{\headword{dompa} \definition{1. penis (slang)}}
\item \entry{n.}{\headword{dompadompa} \definition{1. type of spear}}
\item \entry{n.}{\headword{dompak} \definition{1. eel}}
\item \entry{n.}{\headword{dongkäral} \definition{1. type of lizard}}
\item \entry{n.}{\headword{dongki} \definition{1. donkey}}
\item \entry{n.}{\headword{dor} \definition{1. stalk}}
\item \entry{n.}{\headword{dorllog} \definition{1. rufous-bellied kookaburra}}
\item \entry{n.}{\headword{doros} \definition{1. pants}}
\item \entry{n.}{\headword{dowa} \definition{1. type of tree that grows in the grassland along creeks with wood used for firewood}}
\item \entry{n.}{\headword{dradre1} \definition{1. type of tree that grows in swamp with edible, currant-sized blue fruit}}
\item \entry{n.}{\headword{du kyakya} \definition{1. hook-billed kingfisher}}
\item \entry{n.}{\headword{dubllodubllom} \definition{1. type of tree that grows in the bush with yellow fruit that have a hard seed, in which there is an edible nut}}
\item \entry{n.}{\headword{dugo} \definition{1. type of bird}}
\item \entry{n.}{\headword{dum1} \definition{1. type of big tree that grows in bush with strong wood used for making canoes and paddles, sap used to paint bowstrings or to patch holes, yellow flowers, and green fruit; planted near house for shade}}
\item \entry{n.}{\headword{dum2} \definition{1. placenta}}
\item \entry{n.}{\headword{dum3} \definition{1. width (of a house)}}
\item \entry{n.}{\headword{dum3} \definition{1. to surround}}
\item \entry{n.}{\headword{dumbi} \definition{1. type of red tree}}
\item \entry{n.}{\headword{dundu kllamen} \definition{1. type of game involving a race}}
\item \entry{n.}{\headword{duny} \definition{1. beetle}}
\item \entry{n.}{\headword{dupi} \definition{1. stomach}}
\item \entry{n.}{\headword{dur} \definition{1. type of medium-sized bamboo that grows along creeks; used for dancing; young plants used for cooking sago}}
\item \entry{n.}{\headword{durgu} \definition{1. cliff}}
\item \entry{n.}{\headword{duwel} \definition{1. type of tall tree that grows in the bush near yam gardens}}
\item \entry{n.}{\headword{duwel sära} \definition{1. type of sago bundle wrapped in sago leaves}}
\item \entry{n.}{\headword{duwel sära} \definition{1. the third stage of sago growth in which the leaves are shorter and the pith is almost ready to be harvested}}
\item \entry{n.}{\headword{duwem} \definition{1. food, meal}}
\item \entry{n.}{\headword{duwem} \definition{1. feast, fellowship meal}}
\item \entry{n.}{\headword{duwie ku} \definition{1. type of big purple yam}}
\item \entry{n.}{\headword{ddadd} \definition{1. type of large tree with white flowers and blue fruit; wood used for making canoes}}
\item \entry{n.}{\headword{ddage} \definition{1. branch}}
\item \entry{n.}{\headword{ddage} \definition{1. stream, tributary}}
\item \entry{n.}{\headword{ddage} \definition{1. river mouth}}
\item \entry{n.}{\headword{ddage} \definition{1. river source}}
\item \entry{n.}{\headword{ddage} \definition{1. the fifth stage of sago growth in which the inflorescence has emerged}}
\item \entry{n.}{\headword{ddallwe} \definition{1. type of tree}}
\item \entry{n.}{\headword{ddamba} \definition{1. wing}}
\item \entry{n.}{\headword{ddamba} \definition{1. pectoral fin}}
\item \entry{n.}{\headword{ddamba} \definition{1. metathorax}}
\item \entry{n.}{\headword{ddangol} \definition{1. type of spear}}
\item \entry{n.}{\headword{ddapall} \definition{1. sky}}
\item \entry{n.}{\headword{ddapall} \definition{1. heaven}}
\item \entry{n.}{\headword{ddapall} \definition{1. cloud}}
\item \entry{n.}{\headword{ddapall} \definition{1. cloud}}
\item \entry{n.}{\headword{ddapall} \definition{1. heaven}}
\item \entry{n.}{\headword{ddäb} \definition{1. anus}}
\item \entry{n.}{\headword{ddäddäg1} \definition{1. edible animal, game, meat}}
\item \entry{n.}{\headword{ddäddäg1} \definition{1. bite (of an animal)}}
\item \entry{n.}{\headword{ddäddäg1} \definition{1. hunger for meat}}
\item \entry{n.}{\headword{ddäddäg1} \definition{1. leather, hide, animal skin}}
\item \entry{n.}{\headword{ddäddäll} \definition{1. thunder}}
\item \entry{n.}{\headword{ddäg} \definition{1. back}}
\item \entry{n.}{\headword{ddäg} \definition{1. backbone, spine}}
\item \entry{n.}{\headword{ddäg} \definition{1. later, after}}
\item \entry{n.}{\headword{ddäg} \definition{1. from behind}}
\item \entry{n.}{\headword{ddäll1} \definition{1. chest}}
\item \entry{n.}{\headword{ddäll1} \definition{1. part of the sago trunk closest to the leaves before the shoot}}
\item \entry{n.}{\headword{ddäll1} \definition{1. chest hair}}
\item \entry{n.}{\headword{ddäll1} \definition{1. sternum, breastbone}}
\item \entry{n.}{\headword{ddällpoyampoyam} \definition{1. type of mushroom}}
\item \entry{n.}{\headword{ddäma} \definition{1. basket}}
\item \entry{n.}{\headword{ddäma} \definition{1. pouch of a marsupial}}
\item \entry{n.}{\headword{ddäma} \definition{1. uterus}}
\item \entry{n.}{\headword{ddäma} \definition{1. birth payment made to the maternal uncle}}
\item \entry{n.}{\headword{ddämoemkäp kuibiag} \definition{1. type of python}}
\item \entry{n.}{\headword{ddängall} \definition{1. type of stinging bee found in trees}}
\item \entry{n.}{\headword{ddia} \definition{1. deer}}
\item \entry{n.}{\headword{ddoga} \definition{1. type of tree}}
\item \entry{n.}{\headword{ddogollop} \definition{1. reptile scale}}
\item \entry{n.}{\headword{ddokddok1} \definition{1. type of spear}}
\item \entry{n.}{\headword{ddokop} \definition{1. kidney}}
\item \entry{n.}{\headword{ddol} \definition{1. foam, bubbles, gas}}
\item \entry{n.}{\headword{ddol} \definition{1. foamy, bubbly, gassy}}
\item \entry{n.}{\headword{ddongddong} \definition{1. thick cluster of short grass}}
\item \entry{n.}{\headword{ddumbi} \definition{1. type of spear topped with the claw of a cassowary}}
\item \entry{n.}{\headword{ebagal} \definition{1. type of spear}}
\item \entry{n.}{\headword{ebdo} \definition{1. day}}
\item \entry{n.}{\headword{ebdo} \definition{1. noon (approx. 11 AM –1 PM)}}
\item \entry{n.}{\headword{ebdo} \definition{1. lunch}}
\item \entry{n.}{\headword{eddom} \definition{1. the day before yesterday}}
\item \entry{n.}{\headword{eiz} \definition{1. HIV; AIDS}}
\item \entry{n.}{\headword{eka} \definition{1. language}}
\item \entry{n.}{\headword{eka} \definition{1. word, message, news}}
\item \entry{n.}{\headword{eka} \definition{1. story}}
\item \entry{n.}{\headword{eka} \definition{1. sound, song, call}}
\item \entry{n.}{\headword{eka} \definition{1. word (single unit)}}
\item \entry{n.}{\headword{eka} \definition{1. messenger, prophet}}
\item \entry{n.}{\headword{eka} \definition{1. meaning}}
\item \entry{n.}{\headword{eka} \definition{1. answer, reply}}
\item \entry{n.}{\headword{eka} \definition{1. disagreement, debate, argument}}
\item \entry{n.}{\headword{eka} \definition{1. throat}}
\item \entry{n.}{\headword{eka} \definition{1. to argue}}
\item \entry{n.}{\headword{eka} \definition{1. to speak}}
\item \entry{n.}{\headword{eka} \definition{1. speaker}}
\item \entry{n.}{\headword{eka} \definition{1. spokesperson}}
\item \entry{n.}{\headword{eka} \definition{1. to make noise}}
\item \entry{n.}{\headword{ekaklle} \definition{1. land}}
\item \entry{n.}{\headword{ekaklle} \definition{1. ground}}
\item \entry{n.}{\headword{ekaklle} \definition{1. Earth}}
\item \entry{n.}{\headword{ekaklle} \definition{1. world}}
\item \entry{n.}{\headword{elementri skul} \definition{1. elementary school}}
\item \entry{n.}{\headword{Em} \definition{1. Em language (Pahoturi River language spoken in Kurunti, Kibuli, Beyambod)}}
\item \entry{n.}{\headword{enddäna} \definition{1. clearing}}
\item \entry{n.}{\headword{enddäna} \definition{1. in the open, openly, freely}}
\item \entry{n.}{\headword{enzul} \definition{1. angel}}
\item \entry{n.}{\headword{erany} \definition{1. scream}}
\item \entry{n.}{\headword{eria} \definition{1. area}}
\item \entry{n.}{\headword{erkoll} \definition{1. dirty water}}
\item \entry{n.}{\headword{esam} \definition{1. lemongrass}}
\item \entry{n.}{\headword{ewembe} \definition{1. type of very big yam with a white interior, thorns, and hairs}}
\item \entry{n.}{\headword{ezi} \definition{1. sharp edge}}
\item \entry{n.}{\headword{gabana} \definition{1. governor}}
\item \entry{n.}{\headword{gabän1} \definition{1. wrist}}
\item \entry{n.}{\headword{gablle} \definition{1. hip; waist}}
\item \entry{n.}{\headword{gabma} \definition{1. white person}}
\item \entry{n.}{\headword{gabma} \definition{1. gun}}
\item \entry{n.}{\headword{gabmantt} \definition{1. government}}
\item \entry{n.}{\headword{gae nge} \definition{1. type of palm with coconuts with a red or green exocarp; the husk is chewed and the young coconut water is drunk}}
\item \entry{n.}{\headword{gagäll} \definition{1. sin}}
\item \entry{n.}{\headword{gagäll ine} \definition{1. beer}}
\item \entry{n.}{\headword{gageb} \definition{1. hind leg}}
\item \entry{n.}{\headword{gaguma} \definition{1. yamhouse}}
\item \entry{n.}{\headword{gaimbi} \definition{1. type of fruit tree}}
\item \entry{n.}{\headword{gal} \definition{1. food offering}}
\item \entry{n.}{\headword{galbe} \definition{1. purple/greater yam}}
\item \entry{n.}{\headword{galib} \definition{1. type of spear}}
\item \entry{n.}{\headword{galigali} \definition{1. type of small taro}}
\item \entry{n.}{\headword{gall} \definition{1. canoe, boat}}
\item \entry{n.}{\headword{gall} \definition{1. canoe outrigger}}
\item \entry{n.}{\headword{gall} \definition{1. navigator (of a canoe)}}
\item \entry{n.}{\headword{gall} \definition{1. operator (of a canoe)}}
\item \entry{n.}{\headword{gallgall} \definition{1. bank, coast, shore}}
\item \entry{n.}{\headword{gamo1} \definition{1. type of large sea turtle}}
\item \entry{n.}{\headword{gamo2} \definition{1. plant_type}}
\item \entry{n.}{\headword{gamu} \definition{1. type of ginger with flat leaves; used as medicine for centipede bites and as bait for catching flying foxes; chew it first and the flying fox will eat it and become lethargic}}
\item \entry{n.}{\headword{gangan} \definition{1. type of tree}}
\item \entry{n.}{\headword{gaopi} \definition{1. Australian pelican}}
\item \entry{n.}{\headword{gaora} \definition{1. type of sago}}
\item \entry{n.}{\headword{gara} \definition{1. aquatic leech}}
\item \entry{n.}{\headword{gastol} \definition{1. type of fish}}
\item \entry{n.}{\headword{gaugau} \definition{1. type of big tree that grows in the bush along creeks with wood used for canoes}}
\item \entry{n.}{\headword{gazibra} \definition{1. type of edible water snake with skin like sandpaper}}
\item \entry{n.}{\headword{gäba} \definition{1. shade}}
\item \entry{n.}{\headword{gäba} \definition{1. shady}}
\item \entry{n.}{\headword{gäbgäb} \definition{1. type of tree}}
\item \entry{n.}{\headword{gäboll} \definition{1. magpie-lark}}
\item \entry{n.}{\headword{gägäb ine} \definition{1. dew}}
\item \entry{n.}{\headword{gäl} \definition{1. type of tree that grows in the bush with white flowers, brown seeds, and fruit with a yellow pericarp and green exocarp}}
\item \entry{n.}{\headword{gälas} \definition{1. weaving pattern with concentric squares}}
\item \entry{n.}{\headword{gäleb} \definition{1. type of tree}}
\item \entry{n.}{\headword{gällall} \definition{1. type of pandanus}}
\item \entry{n.}{\headword{gällatater} \definition{1. type of tree}}
\item \entry{n.}{\headword{gälle} \definition{1. type of big tree that grows in the bush near big creeks with wood used for canoes, white and blue flowers, and edible red fruit}}
\item \entry{n.}{\headword{gän} \definition{1. gun}}
\item \entry{n.}{\headword{gärep} \definition{1. grape}}
\item \entry{n.}{\headword{geagell} \definition{1. Lewin's rail}}
\item \entry{n.}{\headword{geawe} \definition{1. type of big tree that grows along creeks with bright purple flowers and wood used for canoes}}
\item \entry{n.}{\headword{ger} \definition{1. type of big tree that grows in the bush}}
\item \entry{n.}{\headword{gi} \definition{1. grease, fat}}
\item \entry{n.}{\headword{gidre} \definition{1. enemy}}
\item \entry{n.}{\headword{giddoll} \definition{1. life}}
\item \entry{n.}{\headword{giegier} \definition{1. white-browed crake}}
\item \entry{n.}{\headword{gilib} \definition{1. type of bird}}
\item \entry{n.}{\headword{girag} \definition{1. long-nosed echymipera}}
\item \entry{n.}{\headword{girag dirindi} \definition{1. type of yam}}
\item \entry{n.}{\headword{giragirag} \definition{1. type of tree}}
\item \entry{n.}{\headword{giri} \definition{1. knife}}
\item \entry{n.}{\headword{giri} \definition{1. sword}}
\item \entry{n.}{\headword{giritai} \definition{1. type of long yam with a white interior and hairs}}
\item \entry{n.}{\headword{giriwak} \definition{1. type of tool}}
\item \entry{n.}{\headword{giro} \definition{1. chicken pox}}
\item \entry{n.}{\headword{gita} \definition{1. guitar}}
\item \entry{n.}{\headword{giwi} \definition{1. fruit dove (coroneted, orange-bellied, pink-spotted, orange-fronted)}}
\item \entry{n.}{\headword{glas} \definition{1. glass}}
\item \entry{n.}{\headword{glle} \definition{1. type of tree with small edible fruit and wood used to make canoes}}
\item \entry{n.}{\headword{gllogllo} \definition{1. marbled frogmouth}}
\item \entry{n.}{\headword{go1} \definition{1. drain}}
\item \entry{n.}{\headword{god} \definition{1. type of cultivated fruit tree similar to dimes tree}}
\item \entry{n.}{\headword{goeg} \definition{1. old garden that is ready to be cleared again}}
\item \entry{n.}{\headword{gogäle} \definition{1. noise}}
\item \entry{n.}{\headword{gogo2} \definition{1. varieties of palms with coconuts with a dark green exocarp}}
\item \entry{n.}{\headword{gogo kottllam} \definition{1. type of turtle with yellow scales on neck}}
\item \entry{n.}{\headword{gogodd} \definition{1. spleen}}
\item \entry{n.}{\headword{gogodd} \definition{1. bladder}}
\item \entry{n.}{\headword{gogodd} \definition{1. water breaking}}
\item \entry{n.}{\headword{gol} \definition{1. goal}}
\item \entry{n.}{\headword{golgol} \definition{1. type of tree that grows in the bush with a straight trunk and wood used for house sticks}}
\item \entry{n.}{\headword{gollob} \definition{1. outer layer, hull, shell (e.g. of a turtle, egg)}}
\item \entry{n.}{\headword{gollolla} \definition{1. type of palm}}
\item \entry{n.}{\headword{gonz} \definition{1. reed}}
\item \entry{n.}{\headword{gonzagonzar} \definition{1. type of tree}}
\item \entry{n.}{\headword{gonzagonzar} \definition{1. type of grub}}
\item \entry{n.}{\headword{gongglem} \definition{1. immature coconut that is partially solid inside}}
\item \entry{n.}{\headword{gongglem} \definition{1. the eighth stage of coconut growth during which the endosperm of the fruit is solidifying}}
\item \entry{n.}{\headword{gonggo} \definition{1. piping bellbird/crested pitohui}}
\item \entry{n.}{\headword{gora} \definition{1. rattle}}
\item \entry{n.}{\headword{goral} \definition{1. type of tree}}
\item \entry{n.}{\headword{grawa} \definition{1. Australasian darter}}
\item \entry{n.}{\headword{grawa} \definition{1. little corella}}
\item \entry{n.}{\headword{greid} \definition{1. grade}}
\item \entry{n.}{\headword{gubare} \definition{1. crossbeam}}
\item \entry{n.}{\headword{guboll} \definition{1. New Guinean magpie}}
\item \entry{n.}{\headword{gudae1} \definition{1. (the) past, before}}
\item \entry{n.}{\headword{guem} \definition{1. channel; deep part of river}}
\item \entry{n.}{\headword{gugall} \definition{1. type of plant with red, yellow, and white flowers and fruit with small, round seeds; children use the fruit in bamboo blow guns}}
\item \entry{n.}{\headword{gugu1} \definition{1. type of big taro}}
\item \entry{n.}{\headword{gugu2} \definition{1. row of leaves going up the roof}}
\item \entry{n.}{\headword{gul} \definition{1. crowd, group, mob; school (of fish)}}
\item \entry{n.}{\headword{gul} \definition{1. crowd}}
\item \entry{n.}{\headword{guli} \definition{1. type of tree}}
\item \entry{n.}{\headword{gulin} \definition{1. crab}}
\item \entry{n.}{\headword{gull} \definition{1. net}}
\item \entry{n.}{\headword{gullba} \definition{1. bundle of sago}}
\item \entry{n.}{\headword{gullem} \definition{1. snake}}
\item \entry{n.}{\headword{gullem suwetar} \definition{1. type of tree that is used as medicine for snake bites and to repel snakes}}
\item \entry{n.}{\headword{gullme} \definition{1. type of flowering tree that grows on the riverside in swamps with wood used for firewood and long, hanging red flowers that smell nice}}
\item \entry{n.}{\headword{gullme käpang} \definition{1. type of spear}}
\item \entry{n.}{\headword{guwaba} \definition{1. guava tree; water steeped with its leaves is used to wash sores}}
\item \entry{n.}{\headword{guwo} \definition{1. heart}}
\item \entry{n.}{\headword{guzi} \definition{1. yabby (type of crayfish)}}
\item \entry{n.}{\headword{guziguzi} \definition{1. type of tree}}
\item \entry{n.}{\headword{gwaga} \definition{1. type of big tree that grows in the bush with big, edible fruit that are yellow outside, red inside, and stain lips brown; when ripe, it will open and drop the heart-shaped seed}}
\item \entry{n.}{\headword{gwaga} \definition{1. the sixth stage of sago growth in which the fruits have matured and the pith is drier}}
\item \entry{n.}{\headword{gwaga} \definition{1. fruit of sago palm}}
\item \entry{n.}{\headword{gwara} \definition{1. lightning}}
\item \entry{n.}{\headword{gwazi} \definition{1. type of big taro}}
\item \entry{n.}{\headword{gwälläd} \definition{1. type of tree}}
\item \entry{n.}{\headword{gwängäm} \definition{1. sacrifice}}
\item \entry{n.}{\headword{gwell} \definition{1. rule, law}}
\item \entry{n.}{\headword{hedmasta} \definition{1. headmaster}}
\item \entry{n.}{\headword{ibe} \definition{1. mist; fog}}
\item \entry{n.}{\headword{ibi2} \definition{1. footprint}}
\item \entry{n.}{\headword{ibik} \definition{1. digging stick}}
\item \entry{n.}{\headword{Ibru} \definition{1. Hebrew}}
\item \entry{n.}{\headword{idaida} \definition{1. type of game played outside in open space}}
\item \entry{n.}{\headword{idoidog} \definition{1. harpoon}}
\item \entry{n.}{\headword{idd} \definition{1. ghost}}
\item \entry{n.}{\headword{idd} \definition{1. afterlife}}
\item \entry{n.}{\headword{iddi} \definition{1. soup}}
\item \entry{n.}{\headword{iddnge} \definition{1. type of coconut palm}}
\item \entry{n.}{\headword{iddob} \definition{1. night}}
\item \entry{n.}{\headword{iddob} \definition{1. midnight}}
\item \entry{n.}{\headword{iddoiddob} \definition{1. type of tree}}
\item \entry{n.}{\headword{iddpo} \definition{1. clothing, clothes}}
\item \entry{n.}{\headword{ikikib} \definition{1. dizziness}}
\item \entry{n.}{\headword{ikllo} \definition{1. smoke}}
\item \entry{n.}{\headword{ikllo} \definition{1. grey; the color of smoke}}
\item \entry{n.}{\headword{ikoll} \definition{1. incident, problem, trouble}}
\item \entry{n.}{\headword{ikop} \definition{1. eye}}
\item \entry{n.}{\headword{ikop} \definition{1. eyeglasses, spectacles; goggles}}
\item \entry{n.}{\headword{ikop} \definition{1. eyeball}}
\item \entry{n.}{\headword{ikop} \definition{1. eyelash}}
\item \entry{n.}{\headword{ikop} \definition{1. pupil}}
\item \entry{n.}{\headword{ikop} \definition{1. fainting, losing consciousness}}
\item \entry{n.}{\headword{ikop} \definition{1. blind}}
\item \entry{n.}{\headword{ikop} \definition{1. to watch, look after, patrol}}
\item \entry{n.}{\headword{ikop} \definition{1. eyelid}}
\item \entry{n.}{\headword{ikop} \definition{1. unseen, invisible}}
\item \entry{n.}{\headword{ikop} \definition{1. wink; eyebrow raise}}
\item \entry{n.}{\headword{ikop} \definition{1. to peek}}
\item \entry{n.}{\headword{ikop} \definition{1. to watch}}
\item \entry{n.}{\headword{ikop} \definition{1. to watch}}
\item \entry{n.}{\headword{ikopse} \definition{1. prayer}}
\item \entry{n.}{\headword{ikopse} \definition{1. church, temple}}
\item \entry{n.}{\headword{ikrol} \definition{1. ash}}
\item \entry{n.}{\headword{imomdae} \definition{1. truth}}
\item \entry{n.}{\headword{imonzimonz} \definition{1. tag (game)}}
\item \entry{n.}{\headword{inbunatt} \definition{1. mallet fish (big, lives in creek, lean)}}
\item \entry{n.}{\headword{indrang} \definition{1. light}}
\item \entry{n.}{\headword{indre} \definition{1. type of tree that grows in the grassland with edible brown-yellow fruit and good, long-burning firewood}}
\item \entry{n.}{\headword{indre} \definition{1. type of snake}}
\item \entry{n.}{\headword{ine} \definition{1. water; liquid}}
\item \entry{n.}{\headword{ine} \definition{1. alcoholic beverage}}
\item \entry{n.}{\headword{ine} \definition{1. spring (water source)}}
\item \entry{n.}{\headword{ine} \definition{1. mud, clay}}
\item \entry{n.}{\headword{ine} \definition{1. well (water source)}}
\item \entry{n.}{\headword{ine} \definition{1. bucket, water container}}
\item \entry{n.}{\headword{ine} \definition{1. well (water source)}}
\item \entry{n.}{\headword{ine} \definition{1. seasick}}
\item \entry{n.}{\headword{ine konkonang} \definition{1. beer}}
\item \entry{n.}{\headword{ine mallmell} \definition{1. type of snake}}
\item \entry{n.}{\headword{inkäm} \definition{1. voice}}
\item \entry{n.}{\headword{inkäm} \definition{1. neck}}
\item \entry{n.}{\headword{inkäm} \definition{1. voice}}
\item \entry{n.}{\headword{inkäm} \definition{1. throat}}
\item \entry{n.}{\headword{inkätt} \definition{1. neck; throat}}
\item \entry{n.}{\headword{inkätt} \definition{1. hollow of drum}}
\item \entry{n.}{\headword{inmol} \definition{1. type of small tree that grows in the bush along creeks; used to weave grass skirts after being pounded}}
\item \entry{n.}{\headword{inpiak} \definition{1. whistling kite}}
\item \entry{n.}{\headword{intot} \definition{1. brow bone}}
\item \entry{n.}{\headword{intot} \definition{1. eyebrow}}
\item \entry{n.}{\headword{inu} \definition{1. sleep}}
\item \entry{n.}{\headword{inu} \definition{1. night}}
\item \entry{n.}{\headword{inuinu2} \definition{1. white-faced robin}}
\item \entry{n.}{\headword{ingoll} \definition{1. face}}
\item \entry{n.}{\headword{ingong} \definition{1. dance}}
\item \entry{n.}{\headword{ingong} \definition{1. sing-sing}}
\item \entry{n.}{\headword{ioläm} \definition{1. type of bird}}
\item \entry{n.}{\headword{ip} \definition{1. type of tree that grows in the grassland with bark that is chewed and sap used as an adhesive or poured on a spear to strengthen it}}
\item \entry{n.}{\headword{irwe} \definition{1. type of cultivated tree with white flowers and juicy, red and white fruit with two seeds; used to treat cough}}
\item \entry{n.}{\headword{irwe} \definition{1. two female friends who share a twin fruit from the irwe tree}}
\item \entry{n.}{\headword{ist} \definition{1. yeast}}
\item \entry{n.}{\headword{ita} \definition{1. type of sedge}}
\item \entry{n.}{\headword{itaita} \definition{1. type of big tree that grows in the bush with red fruit and a straight trunk used for timber}}
\item \entry{n.}{\headword{itbonmäll} \definition{1. dirt}}
\item \entry{n.}{\headword{itrell} \definition{1. disease, illness, sickness}}
\item \entry{n.}{\headword{ittma} \definition{1. husband's house}}
\item \entry{n.}{\headword{iwae} \definition{1. type of weapon}}
\item \entry{n.}{\headword{iya} \definition{1. Australian masked owl}}
\item \entry{n.}{\headword{iyeiyem} \definition{1. common emerald dove}}
\item \entry{n.}{\headword{Izag} \definition{1. name of clan}}
\item \entry{n.}{\headword{iziz} \definition{1. bribery}}
\item \entry{n.}{\headword{kab} \definition{1. string, rope, fiber}}
\item \entry{n.}{\headword{kabadu} \definition{1. traditional dish consisting of coconut cream, sago, meat, and tulip greens}}
\item \entry{n.}{\headword{kabag} \definition{1. type of thick grass that grows in swamps}}
\item \entry{n.}{\headword{kabär} \definition{1. type of big taro}}
\item \entry{n.}{\headword{kabkab} \definition{1. type of vine-like plant}}
\item \entry{n.}{\headword{kae} \definition{1. a type of plant traditionally chewed and used to welcome people}}
\item \entry{n.}{\headword{kae ine} \definition{1. wine}}
\item \entry{n.}{\headword{kaeg} \definition{1. close male friend of the same age who went through initiation at the same time}}
\item \entry{n.}{\headword{kaekae} \definition{1. type of small plant with blue and purple flowers and black fruit that cassowaries eat}}
\item \entry{n.}{\headword{kaembre} \definition{1. type of tree}}
\item \entry{n.}{\headword{kaemne} \definition{1. bee}}
\item \entry{n.}{\headword{kaepse} \definition{1. type of tree that grows in the bush with white flowers and big, yellow fruit that cassowaries and deer eat}}
\item \entry{n.}{\headword{kakab} \definition{1. leftovers, remainder, remnant}}
\item \entry{n.}{\headword{kakakän} \definition{1. (outer) space}}
\item \entry{n.}{\headword{kakayam} \definition{1. greater bird-of-paradise}}
\item \entry{n.}{\headword{kakep} \definition{1. pain}}
\item \entry{n.}{\headword{kakoll} \definition{1. dish}}
\item \entry{n.}{\headword{kakoll} \definition{1. cup}}
\item \entry{n.}{\headword{kakoll} \definition{1. endocarp of coconut}}
\item \entry{n.}{\headword{kakud} \definition{1. type of thin, curved, and long yam without thorns}}
\item \entry{n.}{\headword{kala} \definition{1. color, pigment, dye}}
\item \entry{n.}{\headword{kalemtoe} \definition{1. type of tree that grows in the bush with small, long, thin edible nuts that must be broken with stone; nuts are edible when steamed and are also eaten by cassowaries and doves; flesh is used for cream; similar to toe tree}}
\item \entry{n.}{\headword{kalkmo} \definition{1. joints}}
\item \entry{n.}{\headword{kaltakaltamang} \definition{1. type of spear}}
\item \entry{n.}{\headword{kallakalla} \definition{1. arrow_type}}
\item \entry{n.}{\headword{kallamatt} \definition{1. Oriental dollarbird}}
\item \entry{n.}{\headword{kalläg} \definition{1. type of edible, fatty fish with big scales; found in the swamp, similar to barramundi}}
\item \entry{n.}{\headword{kallekalle} \definition{1. type of yam with a white interior, red skin, and hairs}}
\item \entry{n.}{\headword{kallkäll} \definition{1. cold}}
\item \entry{n.}{\headword{kallkäll} \definition{1. long-sleeve shirt; coat}}
\item \entry{n.}{\headword{kallkäll} \definition{1. to warm up, get warm, warm oneself}}
\item \entry{n.}{\headword{kallmo} \definition{1. butcherbird (black-backed, hooded)}}
\item \entry{n.}{\headword{kamäkamät} \definition{1. type of big tree that grows in the bush with yellow flowers and big yellow fruit that are collected}}
\item \entry{n.}{\headword{kame1} \definition{1. ignorance, non-knowing; incomprehension, non-understanding}}
\item \entry{n.}{\headword{kameny} \definition{1. absence}}
\item \entry{n.}{\headword{kamo} \definition{1. reciprocal term for the young man and the older man that takes him through initiation}}
\item \entry{n.}{\headword{kampani} \definition{1. company}}
\item \entry{n.}{\headword{kamuka} \definition{1. type of cultivated thorny citrus tree with small white flowers and softball-size fruit with thick green skin}}
\item \entry{n.}{\headword{kanas} \definition{1. type of basic arrow}}
\item \entry{n.}{\headword{kanas ma} \definition{1. type of pointed roof not found in Limol}}
\item \entry{n.}{\headword{kandärmang1} \definition{1. pity, sympathy}}
\item \entry{n.}{\headword{kandärmang1} \definition{1. apology}}
\item \entry{n.}{\headword{kandärmang2} \definition{1. present, gift}}
\item \entry{n.}{\headword{kanken} \definition{1. type of mushroom}}
\item \entry{n.}{\headword{kannas} \definition{1. type of bow made out of pitpit}}
\item \entry{n.}{\headword{kanoe} \definition{1. type of tree with fruits that cassowary, pigs, and deer eat and bark that is put in the water to kill fish}}
\item \entry{n.}{\headword{kansel} \definition{1. counsel}}
\item \entry{n.}{\headword{kantärpie} \definition{1. log stuck in the water}}
\item \entry{n.}{\headword{kang} \definition{1. sucker (additional unwanted shoot that grows by the base of a tree)}}
\item \entry{n.}{\headword{kaonggall} \definition{1. yellow-faced myna}}
\item \entry{n.}{\headword{kap} \definition{1. cup}}
\item \entry{n.}{\headword{kapa} \definition{1. knife_type}}
\item \entry{n.}{\headword{kapalla} \definition{1. floating grass}}
\item \entry{n.}{\headword{kapang} \definition{1. Acacia}}
\item \entry{n.}{\headword{kapang} \definition{1. type of grub}}
\item \entry{n.}{\headword{kapang bile} \definition{1. type of medicine}}
\item \entry{n.}{\headword{kapangmändär} \definition{1. type of mushroom}}
\item \entry{n.}{\headword{kapän} \definition{1. wrist}}
\item \entry{n.}{\headword{kapera} \definition{1. male friend or partner who comes from out of town}}
\item \entry{n.}{\headword{kapkap} \definition{1. mudskipper}}
\item \entry{n.}{\headword{kapräl} \definition{1. type of tree}}
\item \entry{n.}{\headword{kaptte} \definition{1. cloth}}
\item \entry{n.}{\headword{kaptte} \definition{1. clothing, clothes; piece of clothing, garment}}
\item \entry{n.}{\headword{kaptte} \definition{1. washing board}}
\item \entry{n.}{\headword{kaptte} \definition{1. clothes line}}
\item \entry{n.}{\headword{kaptte} \definition{1. short trousers}}
\item \entry{n.}{\headword{kaptte} \definition{1. long trousers}}
\item \entry{n.}{\headword{kaptte} \definition{1. laundry, washing clothes}}
\item \entry{n.}{\headword{karado} \definition{1. long spear for fishing}}
\item \entry{n.}{\headword{karama} \definition{1. swamp}}
\item \entry{n.}{\headword{kargeam} \definition{1. type of big taro}}
\item \entry{n.}{\headword{karita} \definition{1. type of introduced banana}}
\item \entry{n.}{\headword{karpo} \definition{1. jar}}
\item \entry{n.}{\headword{kastom} \definition{1. custom}}
\item \entry{n.}{\headword{katre} \definition{1. board; flooring}}
\item \entry{n.}{\headword{katre} \definition{1. raised house, stilt house}}
\item \entry{n.}{\headword{katre} \definition{1. table; desk}}
\item \entry{n.}{\headword{katre} \definition{1. shelf}}
\item \entry{n.}{\headword{katt1} \definition{1. Meyer's friarbird}}
\item \entry{n.}{\headword{katt2} \definition{1. type of medium-sized rodent that lives in the bush}}
\item \entry{n.}{\headword{kau} \definition{1. wrestling clothes}}
\item \entry{n.}{\headword{kawa} \definition{1. announcement, notice; plea}}
\item \entry{n.}{\headword{käba} \definition{1. solo hunt early in the morning}}
\item \entry{n.}{\headword{käban} \definition{1. louse}}
\item \entry{n.}{\headword{käban} \definition{1. nit, louse egg}}
\item \entry{n.}{\headword{käbädral} \definition{1. type of tree that grows in the bush with sturdy wood}}
\item \entry{n.}{\headword{käbäll} \definition{1. type of tree that grows in the bush with soft wood used for canoe paddles}}
\item \entry{n.}{\headword{kädebällag mälla} \definition{1. type of big taro}}
\item \entry{n.}{\headword{kädgal} \definition{1. type of tree that grows in the bush}}
\item \entry{n.}{\headword{kädkäd3} \definition{1. type of initiation that one must do before you get something from someone}}
\item \entry{n.}{\headword{käg1} \definition{1. type of palm with wood used for flooring}}
\item \entry{n.}{\headword{käg1} \definition{1. floor}}
\item \entry{n.}{\headword{käg1} \definition{1. horizontal boards on which the floor goes}}
\item \entry{n.}{\headword{käg1} \definition{1. flooring, floor mat}}
\item \entry{n.}{\headword{käg bänbänang} \definition{1. type of spear}}
\item \entry{n.}{\headword{käk} \definition{1. bubble}}
\item \entry{n.}{\headword{käkan} \definition{1. tide}}
\item \entry{n.}{\headword{käkäm} \definition{1. young leaf}}
\item \entry{n.}{\headword{käkäp} \definition{1. half}}
\item \entry{n.}{\headword{käkäpyo} \definition{1. type of tree that grows in the grassland and savannah with flowers that wallaby and deer eat}}
\item \entry{n.}{\headword{käkoll} \definition{1. baby mat}}
\item \entry{n.}{\headword{käkpäl1} \definition{1. type of tree that grows in the bush; burned to fertilize the ground}}
\item \entry{n.}{\headword{käkpäl2} \definition{1. sago cooked directly over the fire}}
\item \entry{n.}{\headword{kälaepot} \definition{1. tiptoes}}
\item \entry{n.}{\headword{käll} \definition{1. spleen}}
\item \entry{n.}{\headword{källa} \definition{1. feces, poop, waste}}
\item \entry{n.}{\headword{källa} \definition{1. intestines, bowels, guts, innards (of an animal)}}
\item \entry{n.}{\headword{källa} \definition{1. toilet (lit. hole for pooping)}}
\item \entry{n.}{\headword{källa} \definition{1. while pooping; constantly needing to poop}}
\item \entry{n.}{\headword{källa} \definition{1. traveling urgently, arduously, or nonstop as if to a toilet}}
\item \entry{n.}{\headword{källa mit} \definition{1. bottom of fence}}
\item \entry{n.}{\headword{källatolma} \definition{1. middle finger}}
\item \entry{n.}{\headword{källayoyo} \definition{1. type of tree that grows in the bush with leaves used as toilet paper}}
\item \entry{n.}{\headword{källäm} \definition{1. pond; lagoon}}
\item \entry{n.}{\headword{källän} \definition{1. belt}}
\item \entry{n.}{\headword{källän} \definition{1. waist}}
\item \entry{n.}{\headword{källkae} \definition{1. future}}
\item \entry{n.}{\headword{käm1} \definition{1. stomach}}
\item \entry{n.}{\headword{käm1} \definition{1. womb}}
\item \entry{n.}{\headword{käm1} \definition{1. pregnant}}
\item \entry{n.}{\headword{käm3} \definition{1. love, enjoyment}}
\item \entry{n.}{\headword{kämag} \definition{1. round dance with singers and kundu drum in the middle}}
\item \entry{n.}{\headword{kämag} \definition{1. west wind; windy storm from the west}}
\item \entry{n.}{\headword{kämag} \definition{1. season characterized by windy storms from the west (first season; corresponds to January)}}
\item \entry{n.}{\headword{kämag} \definition{1. Big winds in January, a big wind from the west that blows trees down}}
\item \entry{n.}{\headword{kämag} \definition{1. west}}
\item \entry{n.}{\headword{käman} \definition{1. traditional type of cassava}}
\item \entry{n.}{\headword{kämany} \definition{1. type of friendship}}
\item \entry{n.}{\headword{kämätt} \definition{1. testicle}}
\item \entry{n.}{\headword{kämgag} \definition{1. type of lizard}}
\item \entry{n.}{\headword{kämlla} \definition{1. short-beaked echidna}}
\item \entry{n.}{\headword{kämoe} \definition{1. famine}}
\item \entry{n.}{\headword{kämser käpang} \definition{1. type of arrow}}
\item \entry{n.}{\headword{kämsir} \definition{1. type of tree the grows in the bush with fruit that is black outside and green and red inside and is eaten by cassowaries; similar to sirtree fruit}}
\item \entry{n.}{\headword{kämsir} \definition{1. type of grub}}
\item \entry{n.}{\headword{kämtupi} \definition{1. type of ginger}}
\item \entry{n.}{\headword{kän1} \definition{1. type of big, round yam with a white interior and thorns}}
\item \entry{n.}{\headword{känär} \definition{1. type of edible grub found in the bush}}
\item \entry{n.}{\headword{känttatt} \definition{1. bedroom; room, chamber}}
\item \entry{n.}{\headword{käp1} \definition{1. fruit}}
\item \entry{n.}{\headword{käp1} \definition{1. egg}}
\item \entry{n.}{\headword{käp1} \definition{1. nit}}
\item \entry{n.}{\headword{käpät} \definition{1. moisture}}
\item \entry{n.}{\headword{käpät} \definition{1. wet}}
\item \entry{n.}{\headword{käpät} \definition{1. wet, soaked}}
\item \entry{n.}{\headword{käpkumett} \definition{1. type of tree}}
\item \entry{n.}{\headword{käpom} \definition{1. type of big tree that grows in the bush with white flowers and edible white fruit; used as medicine for cough}}
\item \entry{n.}{\headword{käpre} \definition{1. type of big yam with a white interior, thorns, and few hairs}}
\item \entry{n.}{\headword{kär pipiem} \definition{1. purple-tailed imperial pigeon}}
\item \entry{n.}{\headword{kätäräl} \definition{1. color}}
\item \entry{n.}{\headword{kätäräl} \definition{1. multicolored}}
\item \entry{n.}{\headword{kätt} \definition{1. bivalve; shell of a mollusc}}
\item \entry{n.}{\headword{kätt} \definition{1. shell blade}}
\item \entry{n.}{\headword{kätt} \definition{1. (slang) vagina, vulva}}
\item \entry{n.}{\headword{kättapun} \definition{1. type of reed}}
\item \entry{n.}{\headword{kättekätte} \definition{1. red-cheeked parrot}}
\item \entry{n.}{\headword{kättkätt1} \definition{1. weaving pattern with alternating cross and chevron}}
\item \entry{n.}{\headword{kättlla} \definition{1. type of tree that grows in the bush with white flowers}}
\item \entry{n.}{\headword{käza} \definition{1. Hall's New Guinea crocodile}}
\item \entry{n.}{\headword{käza} \definition{1. iguana}}
\item \entry{n.}{\headword{käza allko} \definition{1. type of fly that antagonizes other flies}}
\item \entry{n.}{\headword{käza bädma} \definition{1. type of plant that only the crocodile clan wears when going hunting for crocodiles; also a traditional medicine}}
\item \entry{n.}{\headword{käza burala} \definition{1. water lily}}
\item \entry{n.}{\headword{käza wirwir1} \definition{1. frilled monarch}}
\item \entry{n.}{\headword{käzabun} \definition{1. large sago flower}}
\item \entry{n.}{\headword{käzapig} \definition{1. type of tree with big, sour, black fruit and white and blue flowers}}
\item \entry{n.}{\headword{kek} \definition{1. orange-footed scrubfowl}}
\item \entry{n.}{\headword{kemol} \definition{1. camel}}
\item \entry{n.}{\headword{kemp} \definition{1. camp}}
\item \entry{n.}{\headword{kep} \definition{1. hip}}
\item \entry{n.}{\headword{kep} \definition{1. hip bone}}
\item \entry{n.}{\headword{keräma pudd} \definition{1. type of small yam with a white interior}}
\item \entry{n.}{\headword{kerema} \definition{1. type of taro}}
\item \entry{n.}{\headword{keta} \definition{1. roof post}}
\item \entry{n.}{\headword{ketmar} \definition{1. type of tree that grows in old gardens; after being skinned and soaked, it is weaved into skirts}}
\item \entry{n.}{\headword{ketmar} \definition{1. two yams hanging from a stick in the center of a yam counting pile}}
\item \entry{n.}{\headword{ketol} \definition{1. kettle}}
\item \entry{n.}{\headword{ketrop} \definition{1. season when rain clouds form but are blown away by the wind}}
\item \entry{n.}{\headword{keyadaola} \definition{1. type of bird}}
\item \entry{n.}{\headword{kidwe} \definition{1. millipede}}
\item \entry{n.}{\headword{kikiem} \definition{1. type of bird}}
\item \entry{n.}{\headword{kiklem} \definition{1. type of small edible rodent}}
\item \entry{n.}{\headword{kili} \definition{1. happiness, joy}}
\item \entry{n.}{\headword{kina} \definition{1. kina (PGK, the currency of Papua New Guinea)}}
\item \entry{n.}{\headword{kinpop} \definition{1. type of tree that grows in the bush with white flowers and wood used for house sticks and rafters}}
\item \entry{n.}{\headword{kip} \definition{1. top of a plant}}
\item \entry{n.}{\headword{kip papa} \definition{1. top of fence}}
\item \entry{n.}{\headword{kisin} \definition{1. kitchen}}
\item \entry{n.}{\headword{kitapatt} \definition{1. type of bandicoot-like animal}}
\item \entry{n.}{\headword{kitar} \definition{1. floating grass}}
\item \entry{n.}{\headword{kito} \definition{1. type of black palm; in this immature stage, the shoots eaten as medicine and used to make baskets Used to build house and mat in the bush}}
\item \entry{n.}{\headword{kiyaddadda} \definition{1. paradise kingfisher (common, buff-breasted, little)}}
\item \entry{n.}{\headword{kɨllɨll} \definition{1. type of snake}}
\item \entry{n.}{\headword{klak1} \definition{1. court clerk}}
\item \entry{n.}{\headword{klak2} \definition{1. type of harpoon for fishing}}
\item \entry{n.}{\headword{klas} \definition{1. class}}
\item \entry{n.}{\headword{klasrum} \definition{1. classroom}}
\item \entry{n.}{\headword{kobädd} \definition{1. type of tree}}
\item \entry{n.}{\headword{kobe} \definition{1. type of tree that grows along creeks (~ 3 m) with white and red or blue and purple flowers and edible red fruit with black or white stripes and 4-6 seeds inside}}
\item \entry{n.}{\headword{kodor} \definition{1. piece, lump}}
\item \entry{n.}{\headword{kodowa} \definition{1. dish consisting of sago cooked in leaves on the fire}}
\item \entry{n.}{\headword{koeme} \definition{1. type of tree that grows along creeks with edible, round red fruit}}
\item \entry{n.}{\headword{koemekoeme} \definition{1. type of tree that grows along creeks with inedible, small red fruit}}
\item \entry{n.}{\headword{koenbäll} \definition{1. type of tree that grows in the bush (especially in old gardens) with hanging green fruit and liquid used to treat sores}}
\item \entry{n.}{\headword{kok2} \definition{1. moon}}
\item \entry{n.}{\headword{kok2} \definition{1. month}}
\item \entry{n.}{\headword{kok2} \definition{1. half moon}}
\item \entry{n.}{\headword{kok2} \definition{1. moonlight}}
\item \entry{n.}{\headword{kok2} \definition{1. crescent moon}}
\item \entry{n.}{\headword{kok2} \definition{1. new moon}}
\item \entry{n.}{\headword{kok2} \definition{1. first quarter moon}}
\item \entry{n.}{\headword{kok2} \definition{1. full moon}}
\item \entry{n.}{\headword{kok2} \definition{1. first light (the end of a new moon)}}
\item \entry{n.}{\headword{kok2} \definition{1. orgy}}
\item \entry{n.}{\headword{kok2} \definition{1. moonlight}}
\item \entry{n.}{\headword{kok2} \definition{1. to go a month without menstruating}}
\item \entry{n.}{\headword{kok2} \definition{1. fast}}
\item \entry{n.}{\headword{kok3} \definition{1. grasshopper}}
\item \entry{n.}{\headword{kok patar} \definition{1. necklace}}
\item \entry{n.}{\headword{kokall} \definition{1. type of palm with fruit that are smaller than coconuts}}
\item \entry{n.}{\headword{kokall} \definition{1. two friends who share a twin fruit from the kokall tree (palm tree in the bush)}}
\item \entry{n.}{\headword{kokallkokall} \definition{1. type of tree}}
\item \entry{n.}{\headword{kokäl} \definition{1. small mudcrab}}
\item \entry{n.}{\headword{kokeya moleg} \definition{1. lost sacred stone with markings on it; represents a girl that belongs to the Dumollang clan}}
\item \entry{n.}{\headword{kokmer} \definition{1. when a hunter calls out his sister's name after shooting an animal (she will get the back of the animal if it's killed)}}
\item \entry{n.}{\headword{kokne} \definition{1. type of tree that grows in the grassland with blue flowers and edible blue fruit}}
\item \entry{n.}{\headword{kokngal} \definition{1. hunting in the rain}}
\item \entry{n.}{\headword{kokngal} \definition{1. season characterized by heavy rain, hunting, and fishing with lines and nets (third season; corresponds to mid-February)}}
\item \entry{n.}{\headword{koko2} \definition{1. shoot (of a plant)}}
\item \entry{n.}{\headword{koko2} \definition{1. the first stage of coconut growth in which the shoot has just emerged (before planting)}}
\item \entry{n.}{\headword{kokol} \definition{1. type of introduced banana}}
\item \entry{n.}{\headword{kokopasi} \definition{1. shrikethrush (Arafura, rufous)}}
\item \entry{n.}{\headword{kokpe} \definition{1. type of tree with yellow flowers and leaves that are used to wrap food for cooking on the fire}}
\item \entry{n.}{\headword{koktakokta} \definition{1. when dead trees or tree branches become dry and shine in the night}}
\item \entry{n.}{\headword{kol} \definition{1. sago pith}}
\item \entry{n.}{\headword{kolem} \definition{1. type of palm with coconuts that come in red or green varieties}}
\item \entry{n.}{\headword{kolos} \definition{1. constable uniform}}
\item \entry{n.}{\headword{kolwa} \definition{1. type of tree that grows in grassland with a trunk that has a diameter of ~9 cm, but very strong and used for spears and weapons}}
\item \entry{n.}{\headword{koll} \definition{1. part of a bow}}
\item \entry{n.}{\headword{kollba} \definition{1. fish}}
\item \entry{n.}{\headword{kollba} \definition{1. fish fin}}
\item \entry{n.}{\headword{kollko} \definition{1. breadfruit}}
\item \entry{n.}{\headword{kollkokollko} \definition{1. type of tree that grows in bush with edible red fruit with an edible, big round nut in the seed}}
\item \entry{n.}{\headword{kollmos ttam} \definition{1. soul}}
\item \entry{n.}{\headword{kollong} \definition{1. type of tree that grows in the bush; used for grass skirts}}
\item \entry{n.}{\headword{kom} \definition{1. hair; hair-like thing}}
\item \entry{n.}{\headword{kom} \definition{1. feather}}
\item \entry{n.}{\headword{komälle} \definition{1. type of game played with string}}
\item \entry{n.}{\headword{komene} \definition{1. postpartum period during which a woman warms herself by the fire}}
\item \entry{n.}{\headword{kominiti} \definition{1. community}}
\item \entry{n.}{\headword{komiti} \definition{1. committee}}
\item \entry{n.}{\headword{komllazegatt} \definition{1. twin}}
\item \entry{n.}{\headword{komlle} \definition{1. type of game played with string}}
\item \entry{n.}{\headword{komo1} \definition{1. centipede}}
\item \entry{n.}{\headword{komo1} \definition{1. scorpion}}
\item \entry{n.}{\headword{komo1} \definition{1. type of black scorpion}}
\item \entry{n.}{\headword{komo1} \definition{1. type of scorpion}}
\item \entry{n.}{\headword{komo2} \definition{1. ginger}}
\item \entry{n.}{\headword{komony} \definition{1. tongs}}
\item \entry{n.}{\headword{komotupi molle molle} \definition{1. type of dancing game that involves ginger}}
\item \entry{n.}{\headword{kona} \definition{1. corner}}
\item \entry{n.}{\headword{kona} \definition{1. district, section, area (of a settlement)}}
\item \entry{n.}{\headword{konakone} \definition{1. cover, sheet, blanket}}
\item \entry{n.}{\headword{konskak} \definition{1. type of big taro}}
\item \entry{n.}{\headword{konzar} \definition{1. type of small yam with a white interior, hairs, and small thorns}}
\item \entry{n.}{\headword{konykony} \definition{1. type of stinging insect that lives in the ground}}
\item \entry{n.}{\headword{konymad} \definition{1. man who steals a woman's things during her initiation ceremony}}
\item \entry{n.}{\headword{kopae därängmeny} \definition{1. to go to another village to trade}}
\item \entry{n.}{\headword{kopek} \definition{1. valley}}
\item \entry{n.}{\headword{kopek} \definition{1. pit, hole}}
\item \entry{n.}{\headword{kopllalle} \definition{1. oriole (brown, olive-backed, green)}}
\item \entry{n.}{\headword{koplle} \definition{1. type of big tree that grows in the bush with fruit that cassowaries eat}}
\item \entry{n.}{\headword{koplle} \definition{1. type of game involving throwing koplle fruit}}
\item \entry{n.}{\headword{kormas} \definition{1. type of bird}}
\item \entry{n.}{\headword{korolläm} \definition{1. type of vine that grows beside creeks. Leaves are used to wrap things.}}
\item \entry{n.}{\headword{kos} \definition{1. course}}
\item \entry{n.}{\headword{kosrom} \definition{1. type of large mushroom that grows in the winter}}
\item \entry{n.}{\headword{kot1} \definition{1. dirt}}
\item \entry{n.}{\headword{kot1} \definition{1. dirty, unclean}}
\item \entry{n.}{\headword{kot1} \definition{1. clean, pure}}
\item \entry{n.}{\headword{kot1} \definition{1. dirty, unclean}}
\item \entry{n.}{\headword{kote} \definition{1. nape}}
\item \entry{n.}{\headword{kote} \definition{1. cervical vertebrae}}
\item \entry{n.}{\headword{kote} \definition{1. to hang one's head}}
\item \entry{n.}{\headword{kotol} \definition{1. traditional practice of sterilizing women after having 6–7 kids; the placenta is buried and a coconut is planted on top}}
\item \entry{n.}{\headword{kotol} \definition{1. traditional practice of stopping smoking}}
\item \entry{n.}{\headword{kotom} \definition{1. head covering, crown}}
\item \entry{n.}{\headword{kott} \definition{1. testicles}}
\item \entry{n.}{\headword{kottllam} \definition{1. turtle, tortoise}}
\item \entry{n.}{\headword{kowa} \definition{1. upper back}}
\item \entry{n.}{\headword{krismas} \definition{1. Christmas}}
\item \entry{n.}{\headword{kube} \definition{1. bucket}}
\item \entry{n.}{\headword{kubllu} \definition{1. type of tree that grows in the bush and savannah with bark used for string; similar to kapang}}
\item \entry{n.}{\headword{kubull} \definition{1. bush wallaby, dusky pademelon}}
\item \entry{n.}{\headword{kud} \definition{1. type of pandanus with fat triangular fruit}}
\item \entry{n.}{\headword{kud dämadämar} \definition{1. dragonfly}}
\item \entry{n.}{\headword{kudädäri} \definition{1. Zoe's imperial pigeon}}
\item \entry{n.}{\headword{kudu} \definition{1. corner}}
\item \entry{n.}{\headword{kuddäb} \definition{1. raft}}
\item \entry{n.}{\headword{kuddäll} \definition{1. death}}
\item \entry{n.}{\headword{kui} \definition{1. island}}
\item \entry{n.}{\headword{kuib} \definition{1. type of drum}}
\item \entry{n.}{\headword{kuibiag} \definition{1. Papuan black snake}}
\item \entry{n.}{\headword{kuki} \definition{1. lie}}
\item \entry{n.}{\headword{kukiny} \definition{1. type of long-leaf grass}}
\item \entry{n.}{\headword{kukiny} \definition{1. type of spear made from kukiny grass that is used to hunt birds}}
\item \entry{n.}{\headword{kukpi} \definition{1. type of tree that grows in the bush with big fruit that children play with}}
\item \entry{n.}{\headword{kuku} \definition{1. type of grass}}
\item \entry{n.}{\headword{kulläb} \definition{1. large black termite mound}}
\item \entry{n.}{\headword{kullkull} \definition{1. grassfire; burnt grass}}
\item \entry{n.}{\headword{kullkull} \definition{1. burnt (of an area)}}
\item \entry{n.}{\headword{kullum} \definition{1. group}}
\item \entry{n.}{\headword{kum} \definition{1. buttocks, butt}}
\item \entry{n.}{\headword{kum} \definition{1. abdomen of an insect}}
\item \entry{n.}{\headword{kum} \definition{1. stinger (of an insect)}}
\item \entry{n.}{\headword{kum} \definition{1. back}}
\item \entry{n.}{\headword{kumi} \definition{1. central roof beam}}
\item \entry{n.}{\headword{kumi} \definition{1. bark on top of roof}}
\item \entry{n.}{\headword{kumiye} \definition{1. cough}}
\item \entry{n.}{\headword{kumiye} \definition{1. phlegm}}
\item \entry{n.}{\headword{kumiye} \definition{1. part of a fish}}
\item \entry{n.}{\headword{kumiye} \definition{1. bark on roof ridge that prevents rain from coming in}}
\item \entry{n.}{\headword{kunob} \definition{1. type of tree that grows around creeks with light wood used for house sticks}}
\item \entry{n.}{\headword{kunu} \definition{1. type of short tree that grows in the grassland with poisonous bark for stunning fish}}
\item \entry{n.}{\headword{kunun} \definition{1. season when crops are ready to be harvested (sixth season; corresponds to April)}}
\item \entry{n.}{\headword{kunur} \definition{1. corn, maize}}
\item \entry{n.}{\headword{kunur} \definition{1. corn kernel}}
\item \entry{n.}{\headword{kunuwälläb} \definition{1. type of big taro}}
\item \entry{n.}{\headword{kungge1} \definition{1. spider}}
\item \entry{n.}{\headword{kungge2} \definition{1. type of tree}}
\item \entry{n.}{\headword{kup} \definition{1. hole, pit}}
\item \entry{n.}{\headword{kup} \definition{1. well}}
\item \entry{n.}{\headword{kup} \definition{1. valley}}
\item \entry{n.}{\headword{kup} \definition{1. small hole, pothole}}
\item \entry{n.}{\headword{kupi käp} \definition{1. type of fruit (~4 cm in diameter) that children shoot at birds}}
\item \entry{n.}{\headword{kupull} \definition{1. earthworm, worm}}
\item \entry{n.}{\headword{kuram} \definition{1. type of long yam with a white interior, hairs, and thorns}}
\item \entry{n.}{\headword{kurikuri} \definition{1. game involving spinning tree fruit on one's hand}}
\item \entry{n.}{\headword{kurkur} \definition{1. type of bird}}
\item \entry{n.}{\headword{kurmirang} \definition{1. type of spear}}
\item \entry{n.}{\headword{kus} \definition{1. type of bird}}
\item \entry{n.}{\headword{kut} \definition{1. raincloud}}
\item \entry{n.}{\headword{kutae} \definition{1. type of small yam}}
\item \entry{n.}{\headword{kutkut} \definition{1. name of clan}}
\item \entry{n.}{\headword{kutt} \definition{1. bone}}
\item \entry{n.}{\headword{kutt} \definition{1. seed, core}}
\item \entry{n.}{\headword{kutt gugu} \definition{1. type of peace restoration}}
\item \entry{n.}{\headword{kutt llo} \definition{1. type of tree that grows in the bush with bark that is scraped and rubbed on sores}}
\item \entry{n.}{\headword{kuyu} \definition{1. type of tree that grows in the bush or along creeks with soft wood that rots easily}}
\item \entry{n.}{\headword{kwae} \definition{1. type of yam with a white or purple interior}}
\item \entry{n.}{\headword{kwaena kutt} \definition{1. hip bone}}
\item \entry{n.}{\headword{kwakall} \definition{1. type of tree}}
\item \entry{n.}{\headword{kwakasru} \definition{1. type of snake}}
\item \entry{n.}{\headword{kwallang} \definition{1. type of bush used for posts}}
\item \entry{n.}{\headword{kwantta} \definition{1. type of tree that grows in the grassland (~9 m) with green and yellow leaves used as bow pigment and hard fruit used to play a hockey-like game; also used for posts; liquid is extracted from the bark and given to dogs when they show signs of sickness}}
\item \entry{n.}{\headword{kwangka} \definition{1. Torresian crow}}
\item \entry{n.}{\headword{kwangka lläkäm} \definition{1. type of inedible mushroom}}
\item \entry{n.}{\headword{kwarakwara} \definition{1. eastern hooded pitta}}
\item \entry{n.}{\headword{kwas} \definition{1. type of taro kongkong}}
\item \entry{n.}{\headword{kwata} \definition{1. type of tree that grows in the bush; used for house sticks}}
\item \entry{n.}{\headword{kwatt} \definition{1. court}}
\item \entry{n.}{\headword{kwätäs} \definition{1. type of tree}}
\item \entry{n.}{\headword{kwib} \definition{1. charcoal made from a particular tree called upiye, used for painting during dance and initiation ceremony}}
\item \entry{n.}{\headword{labalaba} \definition{1. lap-lap}}
\item \entry{n.}{\headword{labelabet} \definition{1. type of game involving planting a stick in the middle of a circle}}
\item \entry{n.}{\headword{laen} \definition{1. line}}
\item \entry{n.}{\headword{läläm} \definition{1. muddy spot}}
\item \entry{n.}{\headword{lel} \definition{1. fear}}
\item \entry{n.}{\headword{lel} \definition{1. shame}}
\item \entry{n.}{\headword{lepade} \definition{1. type of cultivated tree with purple and white flowers and edible black fruit that kids like to eat}}
\item \entry{n.}{\headword{lepresi} \definition{1. leprosy}}
\item \entry{n.}{\headword{lesna} \definition{1. type of tree}}
\item \entry{n.}{\headword{lid} \definition{1. lid}}
\item \entry{n.}{\headword{lida} \definition{1. leader}}
\item \entry{n.}{\headword{Limoll} \definition{1. Limol villager, person from Limol}}
\item \entry{n.}{\headword{linge} \definition{1. type of palm tree with blue flowers}}
\item \entry{n.}{\headword{lizom} \definition{1. type of mushroom}}
\item \entry{n.}{\headword{lo} \definition{1. law}}
\item \entry{n.}{\headword{loli} \definition{1. candy}}
\item \entry{n.}{\headword{lomenyang} \definition{1. someone who complains, complainer}}
\item \entry{n.}{\headword{lonsis} \definition{1. lemon plant}}
\item \entry{n.}{\headword{longgo} \definition{1. noise}}
\item \entry{n.}{\headword{lla} \definition{1. man, male}}
\item \entry{n.}{\headword{lla} \definition{1. person, human being}}
\item \entry{n.}{\headword{lla} \definition{1. generation}}
\item \entry{n.}{\headword{lla} \definition{1. crowd}}
\item \entry{n.}{\headword{lla} \definition{1. audience}}
\item \entry{n.}{\headword{lla} \definition{1. boy}}
\item \entry{n.}{\headword{lla} \definition{1. relative, kinsman, clansman}}
\item \entry{n.}{\headword{lla} \definition{1. other people's, of others}}
\item \entry{n.}{\headword{lla} \definition{1. old man}}
\item \entry{n.}{\headword{lla} \definition{1. old man}}
\item \entry{n.}{\headword{lla diben} \definition{1. type of snake}}
\item \entry{n.}{\headword{lla gugu} \definition{1. restoring peace after a murder by trading a young girl in the victim's place}}
\item \entry{n.}{\headword{lla up} \definition{1. type of introduced banana}}
\item \entry{n.}{\headword{lla winy} \definition{1. type of biting bee that lives in trees}}
\item \entry{n.}{\headword{llakällakätt} \definition{1. type of tree that grows in the bush; used for house sticks}}
\item \entry{n.}{\headword{llan} \definition{1. ear}}
\item \entry{n.}{\headword{llan} \definition{1. eardrum}}
\item \entry{n.}{\headword{llan} \definition{1. earwax}}
\item \entry{n.}{\headword{llan} \definition{1. gill}}
\item \entry{n.}{\headword{llan} \definition{1. ear hair}}
\item \entry{n.}{\headword{llan} \definition{1. operculum (gill cover)}}
\item \entry{n.}{\headword{llan} \definition{1. to listen}}
\item \entry{n.}{\headword{llan} \definition{1. to turn one's ear, listen intently}}
\item \entry{n.}{\headword{llapu} \definition{1. type of big tree that grows in the bush with red fruit and white flowers}}
\item \entry{n.}{\headword{llapuyurwe} \definition{1. type of tree with seedless, edible red fruit}}
\item \entry{n.}{\headword{llatata} \definition{1. Food (such as sago, ripe bananas, and coconut cream, or yams and coconut cream) wrapped in a woven cococnut leaf (with a banana leaf within it) and cooked in a mumu}}
\item \entry{n.}{\headword{lläbe} \definition{1. nail, claw (of a finger or toe)}}
\item \entry{n.}{\headword{lläkäm} \definition{1. mushroom}}
\item \entry{n.}{\headword{llälläp} \definition{1. type of small snake that catches frogs}}
\item \entry{n.}{\headword{lläpät} \definition{1. digit}}
\item \entry{n.}{\headword{llätt} \definition{1. end}}
\item \entry{n.}{\headword{llimba} \definition{1. fingernail}}
\item \entry{n.}{\headword{llɨg} \definition{1. boy}}
\item \entry{n.}{\headword{llɨg} \definition{1. womb, uterus}}
\item \entry{n.}{\headword{llɨg} \definition{1. placenta}}
\item \entry{n.}{\headword{llɨg} \definition{1. infant}}
\item \entry{n.}{\headword{llɨg} \definition{1. childless}}
\item \entry{n.}{\headword{llɨg} \definition{1. nonsingular form of llɨg}}
\item \entry{n.}{\headword{llo} \definition{1. tree}}
\item \entry{n.}{\headword{llo} \definition{1. wood}}
\item \entry{n.}{\headword{llo} \definition{1. stick}}
\item \entry{n.}{\headword{llo} \definition{1. wooden stick}}
\item \entry{n.}{\headword{llo} \definition{1. tree branch}}
\item \entry{n.}{\headword{llo} \definition{1. forked tree branch}}
\item \entry{n.}{\headword{llo} \definition{1. fallen tree}}
\item \entry{n.}{\headword{llo} \definition{1. tree trunk}}
\item \entry{n.}{\headword{llo} \definition{1. tree stump}}
\item \entry{n.}{\headword{llo} \definition{1. tree leaf}}
\item \entry{n.}{\headword{llo} \definition{1. tree bark}}
\item \entry{n.}{\headword{llowam} \definition{1. fatigue, tiredness}}
\item \entry{n.}{\headword{llowam} \definition{1. disdain, hate}}
\item \entry{n.}{\headword{llowam} \definition{1. tired}}
\item \entry{n.}{\headword{llowam} \definition{1. unpreferable, unpleasant, tiresome (to someone in the dative)}}
\item \entry{n.}{\headword{llowam} \definition{1. upset, annoyed}}
\item \entry{n.}{\headword{llowawi} \definition{1. type of tree}}
\item \entry{n.}{\headword{llupi} \definition{1. branch}}
\item \entry{n.}{\headword{llupi ttäganen ttägnen} \definition{1. type of game involving hiding a tree branch in the water}}
\item \entry{n.}{\headword{ma} \definition{1. house, home}}
\item \entry{n.}{\headword{ma} \definition{1. place, location}}
\item \entry{n.}{\headword{ma} \definition{1. community}}
\item \entry{n.}{\headword{ma} \definition{1. clan totem (the sacred symbol of a clan group; usually a plant, animal, or body part)}}
\item \entry{n.}{\headword{ma} \definition{1. space under a house}}
\item \entry{n.}{\headword{ma kisin} \definition{1. traditional house built directly on the ground}}
\item \entry{n.}{\headword{ma ttängäm} \definition{1. type of tree}}
\item \entry{n.}{\headword{mab} \definition{1. pandanus}}
\item \entry{n.}{\headword{madik} \definition{1. type of tool}}
\item \entry{n.}{\headword{madmed} \definition{1. type of big tree that grows in the bush}}
\item \entry{n.}{\headword{maduma} \definition{1. village}}
\item \entry{n.}{\headword{madura} \definition{1. type of introduced banana}}
\item \entry{n.}{\headword{maebo} \definition{1. type of spear made from sago}}
\item \entry{n.}{\headword{maemae} \definition{1. type of big tree that grows in places where yam gardens are planted; used for making canoes and paddles}}
\item \entry{n.}{\headword{maer} \definition{1. myrrh}}
\item \entry{n.}{\headword{mai} \definition{1. type of sago}}
\item \entry{n.}{\headword{maigag1} \definition{1. northern brown bandicoot}}
\item \entry{n.}{\headword{maigag2} \definition{1. type of introduced banana}}
\item \entry{n.}{\headword{maiwa} \definition{1. type of pandanus with a long, smooth fruit cooked in mumu}}
\item \entry{n.}{\headword{maiya} \definition{1. bulbil (fruit of yam that is planted)}}
\item \entry{n.}{\headword{mak} \definition{1. mark}}
\item \entry{n.}{\headword{makäp} \definition{1. duration}}
\item \entry{n.}{\headword{maket} \definition{1. market}}
\item \entry{n.}{\headword{malonäbe} \definition{1. type of reed that is too soft for weaving}}
\item \entry{n.}{\headword{mam} \definition{1. blood}}
\item \entry{n.}{\headword{mama1} \definition{1. grass pile}}
\item \entry{n.}{\headword{mama1} \definition{1. small_house}}
\item \entry{n.}{\headword{mama3} \definition{1. food (baby talk word)}}
\item \entry{n.}{\headword{mambag} \definition{1. type of goanna}}
\item \entry{n.}{\headword{mameat} \definition{1. papaya, pawpaw}}
\item \entry{n.}{\headword{mamkiel} \definition{1. type of native banana}}
\item \entry{n.}{\headword{mamlla} \definition{1. rope, string}}
\item \entry{n.}{\headword{mamoe} \definition{1. hunt}}
\item \entry{n.}{\headword{mamos1} \definition{1. comb-crested jacana}}
\item \entry{n.}{\headword{mamos2} \definition{1. village constable}}
\item \entry{n.}{\headword{mandri} \definition{1. cultivated lemon tree}}
\item \entry{n.}{\headword{mandde} \definition{1. Monday}}
\item \entry{n.}{\headword{mani} \definition{1. money}}
\item \entry{n.}{\headword{mani} \definition{1. kina (PGK, the currency of Papua New Guinea)}}
\item \entry{n.}{\headword{mani} \definition{1. money}}
\item \entry{n.}{\headword{manika} \definition{1. cassava}}
\item \entry{n.}{\headword{mankäp} \definition{1. calf (back of leg)}}
\item \entry{n.}{\headword{mantär} \definition{1. type of flat woven rope made from tree bark.}}
\item \entry{n.}{\headword{manggo} \definition{1. mango tree}}
\item \entry{n.}{\headword{mangkimangki} \definition{1. type of game involving chasing}}
\item \entry{n.}{\headword{marat} \definition{1. weaving pattern with a circular bottom and rectangular sides}}
\item \entry{n.}{\headword{mare} \definition{1. type of pandanus}}
\item \entry{n.}{\headword{maret} \definition{1. marriage}}
\item \entry{n.}{\headword{maribärät} \definition{1. type of long yam with a white interior and hairs}}
\item \entry{n.}{\headword{markae} \definition{1. white person}}
\item \entry{n.}{\headword{markae} \definition{1. white, Western; foreign}}
\item \entry{n.}{\headword{markae} \definition{1. gun}}
\item \entry{n.}{\headword{markae} \definition{1. English}}
\item \entry{n.}{\headword{markaebärät} \definition{1. type of medium-sized yam with a white interior, thorns, and no hairs}}
\item \entry{n.}{\headword{mas} \definition{1. small bone found in cassowaries}}
\item \entry{n.}{\headword{masaka} \definition{1. single stem plant with inedible fruit}}
\item \entry{n.}{\headword{matamata} \definition{1. type of tree that grows along swamps with white and blue flowers and edible fruit (black outside, red inside) that ripen during January and February}}
\item \entry{n.}{\headword{matu} \definition{1. ground}}
\item \entry{n.}{\headword{matta} \definition{1. shoulder}}
\item \entry{n.}{\headword{mattmett2} \definition{1. plain, flat area}}
\item \entry{n.}{\headword{mawa} \definition{1. magic}}
\item \entry{n.}{\headword{maza} \definition{1. reef}}
\item \entry{n.}{\headword{mäd kubull} \definition{1. black bush wallaby}}
\item \entry{n.}{\headword{mäda} \definition{1. owner (of an animal)}}
\item \entry{n.}{\headword{mäg} \definition{1. source}}
\item \entry{n.}{\headword{mäga} \definition{1. type of sago}}
\item \entry{n.}{\headword{mägäll} \definition{1. type of tree with bark used to tie spears, white flowers, edible leaves, and finger-sized, edible red fruit with edible seeds}}
\item \entry{n.}{\headword{mägäll} \definition{1. strength}}
\item \entry{n.}{\headword{mägäll} \definition{1. friends who share a twin fruit from the mägäll tree}}
\item \entry{n.}{\headword{mägda} \definition{1. coconut body (as opposed to the shoot)}}
\item \entry{n.}{\headword{mäk} \definition{1. war}}
\item \entry{n.}{\headword{mäk} \definition{1. soldier}}
\item \entry{n.}{\headword{mäka} \definition{1. plantar wart (on the soles of feet; caused by a virus)}}
\item \entry{n.}{\headword{mäkan} \definition{1. desire (of someone)}}
\item \entry{n.}{\headword{mäkat} \definition{1. rat}}
\item \entry{n.}{\headword{mäkämäkäp} \definition{1. dreadlocks}}
\item \entry{n.}{\headword{mäkäp} \definition{1. knot}}
\item \entry{n.}{\headword{mälla} \definition{1. woman, female}}
\item \entry{n.}{\headword{mälla} \definition{1. old woman}}
\item \entry{n.}{\headword{mälla} \definition{1. married (of a male)}}
\item \entry{n.}{\headword{mälla} \definition{1. adultery (of a man)}}
\item \entry{n.}{\headword{mällakutang} \definition{1. type of tree that grows in the bush with white flowers, black bark, and wood used for house sticks}}
\item \entry{n.}{\headword{mällam1} \definition{1. type of plant}}
\item \entry{n.}{\headword{mälläng} \definition{1. nose}}
\item \entry{n.}{\headword{mälläng} \definition{1. nostril}}
\item \entry{n.}{\headword{mälläng} \definition{1. nose hair}}
\item \entry{n.}{\headword{mälläng} \definition{1. antenna}}
\item \entry{n.}{\headword{mälläng} \definition{1. snout}}
\item \entry{n.}{\headword{mällänggäbe} \definition{1. type of plant that grows along the riverside}}
\item \entry{n.}{\headword{mällät} \definition{1. type of tree that grows in the grassland with fruit used to ignite fires and yellow flowers with edible nectar}}
\item \entry{n.}{\headword{mällät} \definition{1. type of grub}}
\item \entry{n.}{\headword{mällät käp1} \definition{1. second stage of sago growth}}
\item \entry{n.}{\headword{mällät käp2} \definition{1. D'Albertis python}}
\item \entry{n.}{\headword{mällätgugu} \definition{1. type of taro}}
\item \entry{n.}{\headword{mällkakallamatt} \definition{1. type of venomous snake}}
\item \entry{n.}{\headword{mällpe} \definition{1. mucus}}
\item \entry{n.}{\headword{mämbär} \definition{1. type of big, round yam with a white interior, hairs, and no thorns}}
\item \entry{n.}{\headword{män1} \definition{1. type of big tree cultivated in the bush with white flowers and blue and purple fruit that cassowary eat; tobacco is planted in the soil near this tree after it is burned}}
\item \entry{n.}{\headword{män3} \definition{1. girl}}
\item \entry{n.}{\headword{män3} \definition{1. young girl}}
\item \entry{n.}{\headword{män3} \definition{1. nonsingular form of män duwar}}
\item \entry{n.}{\headword{män3} \definition{1. nonsingular form of män}}
\item \entry{n.}{\headword{mänang} \definition{1. brother-in-law (one's younger sister's husband)}}
\item \entry{n.}{\headword{mända} \definition{1. thumb; big toe}}
\item \entry{n.}{\headword{mänkot} \definition{1. uninitiated person}}
\item \entry{n.}{\headword{mänkot} \definition{1. non-believer}}
\item \entry{n.}{\headword{mänmänpitepite} \definition{1. type of plant that grows by the river; eaten by wallabies}}
\item \entry{n.}{\headword{mängall} \definition{1. strength, power}}
\item \entry{n.}{\headword{mängall} \definition{1. poison}}
\item \entry{n.}{\headword{mängall} \definition{1. encouragement, cheer}}
\item \entry{n.}{\headword{mängall} \definition{1. weak}}
\item \entry{n.}{\headword{mänyän} \definition{1. type of fish}}
\item \entry{n.}{\headword{märäl1} \definition{1. size}}
\item \entry{n.}{\headword{märäl2} \definition{1. age-mate, someone of the same age}}
\item \entry{n.}{\headword{mäta} \definition{1. type of tree with a red stem, young green leaves, and bark used as rope when old; big trees are used to light fire; liquid is extracted and drunk to treat cough}}
\item \entry{n.}{\headword{mäta} \definition{1. rope made from mäta bark (used to secure fencing)}}
\item \entry{n.}{\headword{mätar onyang2} \definition{1. type of tree}}
\item \entry{n.}{\headword{mätämätär} \definition{1. type of tree}}
\item \entry{n.}{\headword{mätemäte} \definition{1. type of introduced banana}}
\item \entry{n.}{\headword{mätka} \definition{1. type of tall grass with a single stem and bunches of up to 12 small, edible fruit that grow near the base}}
\item \entry{n.}{\headword{mätkakallamatt} \definition{1. type of snake}}
\item \entry{n.}{\headword{mätkin} \definition{1. ring finger}}
\item \entry{n.}{\headword{mätta} \definition{1. lesser yam}}
\item \entry{n.}{\headword{mätta pirpir} \definition{1. rainbow bee-eater}}
\item \entry{n.}{\headword{medol} \definition{1. medal}}
\item \entry{n.}{\headword{mekae} \definition{1. type of tree with white flowers and edible nuts that children eat}}
\item \entry{n.}{\headword{mekewa} \definition{1. type of introduced banana}}
\item \entry{n.}{\headword{melem} \definition{1. work}}
\item \entry{n.}{\headword{melem} \definition{1. servant, laborer}}
\item \entry{n.}{\headword{memba} \definition{1. member}}
\item \entry{n.}{\headword{memram} \definition{1. sweat}}
\item \entry{n.}{\headword{mend} \definition{1. type of bird}}
\item \entry{n.}{\headword{menizment} \definition{1. management}}
\item \entry{n.}{\headword{mer2} \definition{1. type of spear}}
\item \entry{n.}{\headword{meresin} \definition{1. medicine}}
\item \entry{n.}{\headword{metar} \definition{1. type of long yam with a white interior, white skin, and hairs}}
\item \entry{n.}{\headword{midi} \definition{1. magic_type}}
\item \entry{n.}{\headword{midd} \definition{1. meat}}
\item \entry{n.}{\headword{midd} \definition{1. meaning; core, essence}}
\item \entry{n.}{\headword{mik} \definition{1. widow, widower}}
\item \entry{n.}{\headword{mik} \definition{1. nonsingular form of mik}}
\item \entry{n.}{\headword{miks} \definition{1. mix, mixture}}
\item \entry{n.}{\headword{mimi} \definition{1. pig (hunting word)}}
\item \entry{n.}{\headword{mimidämäll} \definition{1. type of snake}}
\item \entry{n.}{\headword{ministri} \definition{1. ministry}}
\item \entry{n.}{\headword{mintor} \definition{1. type of tree with yellow flowers that bloom in June and July and a root is used as a kind of hockey stick}}
\item \entry{n.}{\headword{minggore manggo} \definition{1. type of tree}}
\item \entry{n.}{\headword{miny} \definition{1. ant egg/larvae (small)}}
\item \entry{n.}{\headword{miriwa} \definition{1. type of pandanus}}
\item \entry{n.}{\headword{miroli} \definition{1. black-capped lory}}
\item \entry{n.}{\headword{mise} \definition{1. common cicadabird}}
\item \entry{n.}{\headword{misituryam} \definition{1. type of long yam with a white or purple interior}}
\item \entry{n.}{\headword{mislok} \definition{1. type of introduced banana}}
\item \entry{n.}{\headword{mismis} \definition{1. type of tree used for firewood}}
\item \entry{n.}{\headword{mista} \definition{1. mister}}
\item \entry{n.}{\headword{mit} \definition{1. base (of a plant)}}
\item \entry{n.}{\headword{mit} \definition{1. reason, sake}}
\item \entry{n.}{\headword{mit} \definition{1. origin, source}}
\item \entry{n.}{\headword{mit} \definition{1. guilt}}
\item \entry{n.}{\headword{mit} \definition{1. to blame}}
\item \entry{n.}{\headword{miting} \definition{1. meeting}}
\item \entry{n.}{\headword{mitmit2} \definition{1. blunt axe}}
\item \entry{n.}{\headword{mitoem} \definition{1. type of mushroom}}
\item \entry{n.}{\headword{mittapa} \definition{1. extremities of face (i.e. temples, cheek, chin)}}
\item \entry{n.}{\headword{miwiwi} \definition{1. dabbling duck (pacific black duck, grey teal)}}
\item \entry{n.}{\headword{miwiwi} \definition{1. spotted whistling duck}}
\item \entry{n.}{\headword{mɨka} \definition{1. type of introduced banana}}
\item \entry{n.}{\headword{mo} \definition{1. step, stair(s)}}
\item \entry{n.}{\headword{mo} \definition{1. bridge}}
\item \entry{n.}{\headword{mo} \definition{1. staircase}}
\item \entry{n.}{\headword{mobera} \definition{1. outrigger}}
\item \entry{n.}{\headword{moep} \definition{1. charcoal dust}}
\item \entry{n.}{\headword{moep} \definition{1. matured}}
\item \entry{n.}{\headword{moep} \definition{1. charcoal-stained}}
\item \entry{n.}{\headword{moepo} \definition{1. type of tree that grows in the bush with white flowers, inedible red fruit, and poisonous seeds and bark; an indicator that the soil is fertile and good for making a garden}}
\item \entry{n.}{\headword{moepotatae} \definition{1. type of tree}}
\item \entry{n.}{\headword{mok} \definition{1. friarbird (noisy, little, helmeted)}}
\item \entry{n.}{\headword{moko} \definition{1. desire, want; love}}
\item \entry{n.}{\headword{moko} \definition{1. taste, flavor}}
\item \entry{n.}{\headword{moko} \definition{1. desirable, enjoyable; preferred, favorite}}
\item \entry{n.}{\headword{moko} \definition{1. tasty, delicious, flavorful; sweet}}
\item \entry{n.}{\headword{mokoll} \definition{1. type of tree that grows in the bush with thick bark, white flowers, and small green fruit}}
\item \entry{n.}{\headword{moksir} \definition{1. type of tree}}
\item \entry{n.}{\headword{molemoleg} \definition{1. type of grub}}
\item \entry{n.}{\headword{moll} \definition{1. type of tree that grows around Malam with white flowers}}
\item \entry{n.}{\headword{molle} \definition{1. scent, odor, smell}}
\item \entry{n.}{\headword{molle} \definition{1. smelling}}
\item \entry{n.}{\headword{molle} \definition{1. scented}}
\item \entry{n.}{\headword{molle} \definition{1. odorless}}
\item \entry{n.}{\headword{molle} \definition{1. to smell, perceive a smell}}
\item \entry{n.}{\headword{molle} \definition{1. to smell, take a whiff}}
\item \entry{n.}{\headword{molle} \definition{1. to sniff, smell}}
\item \entry{n.}{\headword{mollok} \definition{1. type of introduced banana}}
\item \entry{n.}{\headword{momana ma} \definition{1. hideout made of leaves for hunting cassowaries}}
\item \entry{n.}{\headword{momea gäl} \definition{1. type of tree}}
\item \entry{n.}{\headword{momolltätän} \definition{1. type of introduced banana}}
\item \entry{n.}{\headword{mompara} \definition{1. type of tree that grows near swamps and creeks with white flowers and yellow and brown fruit that ripen in April and May and are eaten by deer}}
\item \entry{n.}{\headword{mompel} \definition{1. aibika}}
\item \entry{n.}{\headword{mondo} \definition{1. type of tree that grows in the bush with green or purple fruits that animals eat}}
\item \entry{n.}{\headword{mondre} \definition{1. gardening, garden work}}
\item \entry{n.}{\headword{mondre} \definition{1. gardener, having a green thumb}}
\item \entry{n.}{\headword{mopmop} \definition{1. type of tree that grows in the grassland (~3 m) with white flowers and red fruit that are eaten to treat cough}}
\item \entry{n.}{\headword{mormor} \definition{1. type of small herb with white flowers and ovate leaves}}
\item \entry{n.}{\headword{mos} \definition{1. type of goanna}}
\item \entry{n.}{\headword{mosen} \definition{1. eldest, firstborn}}
\item \entry{n.}{\headword{mosen} \definition{1. elders, seniors}}
\item \entry{n.}{\headword{motom} \definition{1. type of small taro}}
\item \entry{n.}{\headword{Moyäm} \definition{1. Moyäm (toponym)}}
\item \entry{n.}{\headword{moza} \definition{1. type of venomous snake}}
\item \entry{n.}{\headword{mozaya} \definition{1. type of large, edible fish found in the swamp}}
\item \entry{n.}{\headword{mu} \definition{1. payment, price, value}}
\item \entry{n.}{\headword{mu} \definition{1. response, reply, answer; repayment, revenge}}
\item \entry{n.}{\headword{mubine} \definition{1. dish consisting of food (such as banana, yam, or sweet potato) with coconut cream. up mubine, mätta mubine, nai mubine}}
\item \entry{n.}{\headword{mugbusu} \definition{1. type of taro}}
\item \entry{n.}{\headword{mulkul} \definition{1. brain}}
\item \entry{n.}{\headword{mulmul} \definition{1. rite of passage}}
\item \entry{n.}{\headword{mullamulla} \definition{1. medicine}}
\item \entry{n.}{\headword{mupni} \definition{1. type of cultivated mango tree with fruit with yellow skin and a white interior that is juiced}}
\item \entry{n.}{\headword{muro} \definition{1. magic}}
\item \entry{n.}{\headword{mutae} \definition{1. type of yam with a yellow interior, no thorns, and a vine that grows clockwise}}
\item \entry{n.}{\headword{mutt} \definition{1. river source}}
\item \entry{n.}{\headword{muttmutt} \definition{1. type of grub that lives in the ground and eats crops}}
\item \entry{n.}{\headword{nadum} \definition{1. namesake}}
\item \entry{n.}{\headword{nae} \definition{1. sweet potato}}
\item \entry{n.}{\headword{nallib} \definition{1. type of spear}}
\item \entry{n.}{\headword{näkäp} \definition{1. mind, mindset, consciousness; thoughts}}
\item \entry{n.}{\headword{näkäp} \definition{1. thought, knowledge}}
\item \entry{n.}{\headword{näkäp} \definition{1. smart}}
\item \entry{n.}{\headword{näkäp} \definition{1. foolish}}
\item \entry{n.}{\headword{näkäp} \definition{1. to change one's mind}}
\item \entry{n.}{\headword{näkäp} \definition{1. to worry}}
\item \entry{n.}{\headword{nänäm} \definition{1. diagonal checkerboard weaving pattern}}
\item \entry{n.}{\headword{nängga} \definition{1. type of pandanus with bunches of fruit with strong shells; they fall one at a time}}
\item \entry{n.}{\headword{nätnät} \definition{1. insects}}
\item \entry{n.}{\headword{nes} \definition{1. nurse}}
\item \entry{n.}{\headword{net} \definition{1. net}}
\item \entry{n.}{\headword{nidol} \definition{1. needle}}
\item \entry{n.}{\headword{nikbin} \definition{1. nickname}}
\item \entry{n.}{\headword{nil} \definition{1. nail (made of metal)}}
\item \entry{n.}{\headword{nineyem} \definition{1. whisper}}
\item \entry{n.}{\headword{nongg} \definition{1. type of gecko}}
\item \entry{n.}{\headword{nop mu} \definition{1. inter-tribe payment}}
\item \entry{n.}{\headword{nora} \definition{1. type of big introduced tree with red flowers that attract birds}}
\item \entry{n.}{\headword{nurae} \definition{1. type of large bird}}
\item \entry{n.}{\headword{ngalen} \definition{1. way, habit, manner, custom}}
\item \entry{n.}{\headword{ngalen} \definition{1. thing}}
\item \entry{n.}{\headword{ngallngall} \definition{1. catbird (spotted, ochre-breasted)}}
\item \entry{n.}{\headword{ngam} \definition{1. breast}}
\item \entry{n.}{\headword{ngam} \definition{1. nipple, teat}}
\item \entry{n.}{\headword{ngam} \definition{1. breast}}
\item \entry{n.}{\headword{ngamtep} \definition{1. type of mushroom}}
\item \entry{n.}{\headword{ngata} \definition{1. spot}}
\item \entry{n.}{\headword{ngata} \definition{1. around, about, approximately}}
\item \entry{n.}{\headword{ngata} \definition{1. small space}}
\item \entry{n.}{\headword{ngatengate} \definition{1. gliding possum, sugar glider}}
\item \entry{n.}{\headword{ngatt} \definition{1. type of tree}}
\item \entry{n.}{\headword{ngattong} \definition{1. beginning; past}}
\item \entry{n.}{\headword{ngattong} \definition{1. the very beginning}}
\item \entry{n.}{\headword{ngattong} \definition{1. very beginning, very first}}
\item \entry{n.}{\headword{ngattong} \definition{1. ahead, in front}}
\item \entry{n.}{\headword{ngämengäme} \definition{1. type of vine with sweet, edible fruit that are yellow when ripe}}
\item \entry{n.}{\headword{ngämral} \definition{1. type of vine that grows in the bush and irritates skin}}
\item \entry{n.}{\headword{ngän} \definition{1. boundary}}
\item \entry{n.}{\headword{ngänam} \definition{1. understanding}}
\item \entry{n.}{\headword{ngätt} \definition{1. yard}}
\item \entry{n.}{\headword{ngättäma} \definition{1. place, spot}}
\item \entry{n.}{\headword{nge} \definition{1. coconut palm}}
\item \entry{n.}{\headword{nge} \definition{1. coconut}}
\item \entry{n.}{\headword{nge} \definition{1. fully dry coconut}}
\item \entry{n.}{\headword{nge} \definition{1. the eleventh and final stage of coconut growth in which the fruit is completely dry and brown}}
\item \entry{n.}{\headword{nge} \definition{1. coconut cream}}
\item \entry{n.}{\headword{nge} \definition{1. coconut water}}
\item \entry{n.}{\headword{nge} \definition{1. coconut flower}}
\item \entry{n.}{\headword{nge} \definition{1. the fourth stage of coconut growth in which flowers appear}}
\item \entry{n.}{\headword{nge} \definition{1. type of snake}}
\item \entry{n.}{\headword{nge} \definition{1. tiny coconut, baby coconut}}
\item \entry{n.}{\headword{nge} \definition{1. the sixth stage of coconut growth in which the fruits have a diameter of ~5 cm}}
\item \entry{n.}{\headword{nge} \definition{1. coconut frond}}
\item \entry{n.}{\headword{nge ibeny} \definition{1. planting a coconut as a gesture of peace}}
\item \entry{n.}{\headword{nge pollgo} \definition{1. coconut frog, green}}
\item \entry{n.}{\headword{ngoeang} \definition{1. traditional Y-shaped house post (used prior to nails)}}
\item \entry{n.}{\headword{ngoetrangoetra} \definition{1. type of ant that doesn't bite}}
\item \entry{n.}{\headword{ngoi2} \definition{1. tooth}}
\item \entry{n.}{\headword{ngokngok} \definition{1. boobook (Australian, barking [barking owl])}}
\item \entry{n.}{\headword{ngolo} \definition{1. type of big yam with a purple interior, thorns, and hairs}}
\item \entry{n.}{\headword{nyan} \definition{1. type of tree with small leaves and bark used for mumu and topping a roof}}
\item \entry{n.}{\headword{nyanyu} \definition{1. action}}
\item \entry{n.}{\headword{nyaraman} \definition{1. fine day}}
\item \entry{n.}{\headword{nyäkiyam} \definition{1. tear}}
\item \entry{n.}{\headword{nyäkukub} \definition{1. sago washing basket}}
\item \entry{n.}{\headword{nyäng1} \definition{1. basket; bag}}
\item \entry{n.}{\headword{nyäng1} \definition{1. woven strap or handle of a nyäng}}
\item \entry{n.}{\headword{nyäng1} \definition{1. type of bag}}
\item \entry{n.}{\headword{nyäng1} \definition{1. type of basket}}
\item \entry{n.}{\headword{nyärmeny} \definition{1. little argument between children}}
\item \entry{n.}{\headword{nyeny} \definition{1. type of large tree that grows along the river and in swamps with white flowers, brown fruit, and white bark used for making mumu and topping a roof}}
\item \entry{n.}{\headword{nying} \definition{1. foot}}
\item \entry{n.}{\headword{nying} \definition{1. sandal}}
\item \entry{n.}{\headword{nying} \definition{1. toenail}}
\item \entry{n.}{\headword{nying} \definition{1. toe}}
\item \entry{n.}{\headword{nying} \definition{1. type of game involving kicking}}
\item \entry{n.}{\headword{nying} \definition{1. shoe}}
\item \entry{n.}{\headword{nyinggulgul} \definition{1. type of bird}}
\item \entry{n.}{\headword{nyingnying} \definition{1. foundation of fence}}
\item \entry{n.}{\headword{nyonga} \definition{1. Triton cockatoo}}
\item \entry{n.}{\headword{nyongo} \definition{1. road, path, way}}
\item \entry{n.}{\headword{nyongo} \definition{1. way, method}}
\item \entry{n.}{\headword{nyongo taempägag} \definition{1. navigator (of a canoe)}}
\item \entry{n.}{\headword{nyukukum} \definition{1. type of bag, used for squeezing sago, a thing woven from tree bark or reeds to wash sago or to fill with sago.}}
\item \entry{n.}{\headword{obergaban} \definition{1. clear day}}
\item \entry{n.}{\headword{oboll} \definition{1. type of small tree that grows in the grassland (~2 m) with white flowers and red fruit with a large brown seed; fruit is eaten by birds; wood is used for sturdy posts}}
\item \entry{n.}{\headword{obosasa} \definition{1. Australian rufous fantail}}
\item \entry{n.}{\headword{obosasa} \definition{1. blue jewel-babbler}}
\item \entry{n.}{\headword{od} \definition{1. type of fatty freshwater fish}}
\item \entry{n.}{\headword{odoolo} \definition{1. type of bird}}
\item \entry{n.}{\headword{ogog} \definition{1. grey-crowned babbler}}
\item \entry{n.}{\headword{ol1} \definition{1. hall}}
\item \entry{n.}{\headword{olib} \definition{1. olive}}
\item \entry{n.}{\headword{olmapänyik} \definition{1. armpit}}
\item \entry{n.}{\headword{olmopäga} \definition{1. type of sago}}
\item \entry{n.}{\headword{oll} \definition{1. sugarcane}}
\item \entry{n.}{\headword{ollondd} \definition{1. root}}
\item \entry{n.}{\headword{ollong} \definition{1. time, occasion}}
\item \entry{n.}{\headword{omad} \definition{1. type of friend}}
\item \entry{n.}{\headword{omawe} \definition{1. type of tree}}
\item \entry{n.}{\headword{omäg} \definition{1. magic}}
\item \entry{n.}{\headword{omäg} \definition{1. magician, sorcerer, fortune-teller}}
\item \entry{n.}{\headword{omgälgäl} \definition{1. red ants}}
\item \entry{n.}{\headword{onggall} \definition{1. frilled lizard}}
\item \entry{n.}{\headword{opa} \definition{1. type of tree}}
\item \entry{n.}{\headword{opa ttangtte} \definition{1. type of tree}}
\item \entry{n.}{\headword{opodo} \definition{1. weaving pattern with checks}}
\item \entry{n.}{\headword{orbam} \definition{1. marsh}}
\item \entry{n.}{\headword{orwa} \definition{1. suffering}}
\item \entry{n.}{\headword{osne} \definition{1. type of small taro}}
\item \entry{n.}{\headword{ospel} \definition{1. hospital; aid post}}
\item \entry{n.}{\headword{otät} \definition{1. food}}
\item \entry{n.}{\headword{otät} \definition{1. hunger}}
\item \entry{n.}{\headword{otät} \definition{1. starvation}}
\item \entry{n.}{\headword{otät} \definition{1. kitchen}}
\item \entry{n.}{\headword{otät} \definition{1. edible}}
\item \entry{n.}{\headword{owel} \definition{1. oil}}
\item \entry{n.}{\headword{pa} \definition{1. bird}}
\item \entry{n.}{\headword{pa} \definition{1. feather}}
\item \entry{n.}{\headword{padiem} \definition{1. type of introduced banana}}
\item \entry{n.}{\headword{pag} \definition{1. salt}}
\item \entry{n.}{\headword{paitt} \definition{1. fight}}
\item \entry{n.}{\headword{pakätt} \definition{1. widow's robe}}
\item \entry{n.}{\headword{paklle} \definition{1. type of snake}}
\item \entry{n.}{\headword{pakos} \definition{1. type of spear}}
\item \entry{n.}{\headword{palament} \definition{1. parliament}}
\item \entry{n.}{\headword{pale} \definition{1. type of white clay used for painting babies}}
\item \entry{n.}{\headword{pall} \definition{1. varieties of palms with coconuts with a red exocarp}}
\item \entry{n.}{\headword{pall kottllam} \definition{1. red-bellied short-necked turtle}}
\item \entry{n.}{\headword{pall kubull} \definition{1. red-legged pademelon}}
\item \entry{n.}{\headword{pall tawe} \definition{1. type of large palm that grows in the grassland with white flowers and coconuts with a red exocarp; small sticks are good for houses, and bark is used for walling}}
\item \entry{n.}{\headword{pallall} \definition{1. direction; area}}
\item \entry{n.}{\headword{pallkeakeya} \definition{1. type of red mushroom that grows in the grassland}}
\item \entry{n.}{\headword{pam} \definition{1. injection, shot, vaccine}}
\item \entry{n.}{\headword{pama} \definition{1. farmer}}
\item \entry{n.}{\headword{pambu} \definition{1. hoe (tool)}}
\item \entry{n.}{\headword{pamker} \definition{1. pumpkin}}
\item \entry{n.}{\headword{pana} \definition{1. type of relationship}}
\item \entry{n.}{\headword{panda} \definition{1. type of tree}}
\item \entry{n.}{\headword{pane} \definition{1. pot}}
\item \entry{n.}{\headword{panggopanggo} \definition{1. type of round yam with a white interior and hairs}}
\item \entry{n.}{\headword{panya} \definition{1. pineapple}}
\item \entry{n.}{\headword{pap} \definition{1. type of round mutae yam}}
\item \entry{n.}{\headword{papek} \definition{1. dam, blockage; wall}}
\item \entry{n.}{\headword{papiye} \definition{1. animal tracks}}
\item \entry{n.}{\headword{para} \definition{1. event where harvest and hunting bounty are compared and gifted for bragging rights}}
\item \entry{n.}{\headword{Parade} \definition{1. Parade (toponym)}}
\item \entry{n.}{\headword{parga} \definition{1. bridge}}
\item \entry{n.}{\headword{paro kottllam} \definition{1. type of turtle with red scales on neck}}
\item \entry{n.}{\headword{parpar} \definition{1. type of sago bundle wrapped in a long cylinder in sago leaves}}
\item \entry{n.}{\headword{pasis} \definition{1. deep}}
\item \entry{n.}{\headword{paspas} \definition{1. type of game where players try to keep a ball off the ground}}
\item \entry{n.}{\headword{pasta} \definition{1. pastor}}
\item \entry{n.}{\headword{pat1} \definition{1. type of taro that is eaten from the suckers}}
\item \entry{n.}{\headword{pat2} \definition{1. egg yolk}}
\item \entry{n.}{\headword{patara} \definition{1. wall}}
\item \entry{n.}{\headword{patarapatara} \definition{1. type of hand game where players make an L shape with their fingers}}
\item \entry{n.}{\headword{patepate} \definition{1. bamboo sticks used for percussion}}
\item \entry{n.}{\headword{patiti} \definition{1. type of bird}}
\item \entry{n.}{\headword{patkoll} \definition{1. bundle}}
\item \entry{n.}{\headword{patkoll} \definition{1. carrying bundles, with bundles}}
\item \entry{n.}{\headword{patrol} \definition{1. patrol}}
\item \entry{n.}{\headword{patt1} \definition{1. tree trunk (fallen)}}
\item \entry{n.}{\headword{patt1} \definition{1. drum shell}}
\item \entry{n.}{\headword{patt1} \definition{1. coconut husking stick}}
\item \entry{n.}{\headword{pattlle} \definition{1. type of small bamboo that grows in the bush and along creeks; used to cook sago and make flutes; may be burned for fertile land for planting}}
\item \entry{n.}{\headword{pauro} \definition{1. chicken}}
\item \entry{n.}{\headword{pazi} \definition{1. year}}
\item \entry{n.}{\headword{päd} \definition{1. scar}}
\item \entry{n.}{\headword{päd} \definition{1. tattoo}}
\item \entry{n.}{\headword{pägamän} \definition{1. type of yam}}
\item \entry{n.}{\headword{pälan} \definition{1. plan}}
\item \entry{n.}{\headword{pälkom} \definition{1. body hair}}
\item \entry{n.}{\headword{pälläk} \definition{1. type of introduced banana}}
\item \entry{n.}{\headword{pällämpälläm yurwe} \definition{1. type of tree with white flowers and edible white fruit.}}
\item \entry{n.}{\headword{pällulle} \definition{1. lung}}
\item \entry{n.}{\headword{pämbäll} \definition{1. poison_root}}
\item \entry{n.}{\headword{pänae} \definition{1. zenith (position of the sun at high noon)}}
\item \entry{n.}{\headword{pänbäll} \definition{1. poisonous vine or root (used in fishing to stun fish)}}
\item \entry{n.}{\headword{pänmällang mälla} \definition{1. type of big taro}}
\item \entry{n.}{\headword{pänpän} \definition{1. dust}}
\item \entry{n.}{\headword{pänpän} \definition{1. calcium hydroxide, lime}}
\item \entry{n.}{\headword{päpa} \definition{1. line}}
\item \entry{n.}{\headword{päräl} \definition{1. radjah shelduck}}
\item \entry{n.}{\headword{päs} \definition{1. type of sugarcane-like plant with fruit on top}}
\item \entry{n.}{\headword{pätär} \definition{1. white hair}}
\item \entry{n.}{\headword{pätt} \definition{1. body (of a person or animal)}}
\item \entry{n.}{\headword{pätt} \definition{1. trunk of a plant in the ground; a single plant}}
\item \entry{n.}{\headword{pätt} \definition{1. node (ring or line on bamboo or sugarcane that separates internodes)}}
\item \entry{n.}{\headword{pätt} \definition{1. part of a bow}}
\item \entry{n.}{\headword{pättäl} \definition{1. type of tree that grows near swamps with yellow flowers and bark used for rope to tie firewood}}
\item \entry{n.}{\headword{peba} \definition{1. paper}}
\item \entry{n.}{\headword{pemli} \definition{1. family}}
\item \entry{n.}{\headword{pen} \definition{1. pen}}
\item \entry{n.}{\headword{Penz} \definition{1. name of a river}}
\item \entry{n.}{\headword{penganyäm} \definition{1. type of yam}}
\item \entry{n.}{\headword{pepätt} \definition{1. black-faced monarch}}
\item \entry{n.}{\headword{pepätt} \definition{1. spot-winged monarch}}
\item \entry{n.}{\headword{pepeb} \definition{1. folktale, legend}}
\item \entry{n.}{\headword{pepeb} \definition{1. legend}}
\item \entry{n.}{\headword{pepeb wup} \definition{1. type of introduced banana}}
\item \entry{n.}{\headword{pera} \definition{1. bird perch}}
\item \entry{n.}{\headword{pera} \definition{1. perching}}
\item \entry{n.}{\headword{perälla} \definition{1. type of vine}}
\item \entry{n.}{\headword{petron} \definition{1. blood vessel; vein; artery}}
\item \entry{n.}{\headword{pett} \definition{1. hip}}
\item \entry{n.}{\headword{pewälewäle} \definition{1. green pygmy goose}}
\item \entry{n.}{\headword{pid} \definition{1. horsefly}}
\item \entry{n.}{\headword{pidor} \definition{1. white-bellied sea-eagle}}
\item \entry{n.}{\headword{pidroll} \definition{1. black palm weevil}}
\item \entry{n.}{\headword{pig} \definition{1. type of cultivated tree with edible fruit}}
\item \entry{n.}{\headword{pilatt} \definition{1. plate, dish}}
\item \entry{n.}{\headword{pimbyom} \definition{1. small pieces of bark}}
\item \entry{n.}{\headword{pin} \definition{1. type of tree with white or red flowers and composite fruit that birds eat}}
\item \entry{n.}{\headword{pintta} \definition{1. palm cockatoo}}
\item \entry{n.}{\headword{pinzopinzo1} \definition{1. type of small insect that lives in the ground}}
\item \entry{n.}{\headword{pinzopinzo2} \definition{1. curly hair}}
\item \entry{n.}{\headword{ping} \definition{1. baby pin}}
\item \entry{n.}{\headword{pinggudd} \definition{1. skirt}}
\item \entry{n.}{\headword{piny} \definition{1. kingfisher (azure, little)}}
\item \entry{n.}{\headword{pinya dorollog} \definition{1. sacred kingfisher}}
\item \entry{n.}{\headword{pip} \definition{1. red bee}}
\item \entry{n.}{\headword{pipllo} \definition{1. lizard}}
\item \entry{n.}{\headword{pirik} \definition{1. baton, stick}}
\item \entry{n.}{\headword{piro} \definition{1. star}}
\item \entry{n.}{\headword{piro} \definition{1. star weaving pattern}}
\item \entry{n.}{\headword{piro} \definition{1. type of game where the first person to see a star in the sky wins}}
\item \entry{n.}{\headword{piro} \definition{1. to become unconscious}}
\item \entry{n.}{\headword{pisam} \definition{1. small pieces}}
\item \entry{n.}{\headword{pisor} \definition{1. piece}}
\item \entry{n.}{\headword{pit} \definition{1. small insect that lives in the swamp and eats taro and sweet potato}}
\item \entry{n.}{\headword{pite1} \definition{1. grass skirt}}
\item \entry{n.}{\headword{pite1} \definition{1. tailfeather}}
\item \entry{n.}{\headword{pite1} \definition{1. under the skirt}}
\item \entry{n.}{\headword{pite2} \definition{1. large winged ankom}}
\item \entry{n.}{\headword{pitpit} \definition{1. sugarcane}}
\item \entry{n.}{\headword{pitratra} \definition{1. masked lapwing}}
\item \entry{n.}{\headword{pitt} \definition{1. arrowhead hafting string}}
\item \entry{n.}{\headword{pitt} \definition{1. band}}
\item \entry{n.}{\headword{pittbudar} \definition{1. type of edible grub found in the bush and grassland}}
\item \entry{n.}{\headword{piya} \definition{1. type of flowering plant}}
\item \entry{n.}{\headword{piya} \definition{1. planting a piya plant as a gesture of peace}}
\item \entry{n.}{\headword{piya} \definition{1. planting a piya plant as a gesture of peace}}
\item \entry{n.}{\headword{piye} \definition{1. pus}}
\item \entry{n.}{\headword{piye} \definition{1. sperm}}
\item \entry{n.}{\headword{piyepiye} \definition{1. blister}}
\item \entry{n.}{\headword{piyupiyu} \definition{1. large vine that grows from tree to tree in the bush}}
\item \entry{n.}{\headword{piyupiyu} \definition{1. type of grub}}
\item \entry{n.}{\headword{pɨnyapɨnye} \definition{1. area with burnt grass}}
\item \entry{n.}{\headword{plaig} \definition{1. flag}}
\item \entry{n.}{\headword{plen} \definition{1. airplane}}
\item \entry{n.}{\headword{plen} \definition{1. airstrip}}
\item \entry{n.}{\headword{plenplen} \definition{1. pinwheel}}
\item \entry{n.}{\headword{pllulleaga} \definition{1. the seventh and final stage of sago growth during which the pith will not yield any starch}}
\item \entry{n.}{\headword{pllutt} \definition{1. type of vine with fruit that are yellow when ripe; children eat them}}
\item \entry{n.}{\headword{po1} \definition{1. mound of dirt around a plant}}
\item \entry{n.}{\headword{pob} \definition{1. savanna}}
\item \entry{n.}{\headword{poder} \definition{1. scapula, shoulder blade}}
\item \entry{n.}{\headword{podd} \definition{1. bald head, baldness}}
\item \entry{n.}{\headword{poddem} \definition{1. young or small mammal (e.g. deer, wallaby)}}
\item \entry{n.}{\headword{poddpodd} \definition{1. plain, field, clearing}}
\item \entry{n.}{\headword{pogpog} \definition{1. cockroach}}
\item \entry{n.}{\headword{polis} \definition{1. police}}
\item \entry{n.}{\headword{polle} \definition{1. fence}}
\item \entry{n.}{\headword{polle} \definition{1. base of fence}}
\item \entry{n.}{\headword{polle} \definition{1. small fence}}
\item \entry{n.}{\headword{pollgo} \definition{1. frog}}
\item \entry{n.}{\headword{pollgo suwe} \definition{1. dirt left on skin after coming out of the water}}
\item \entry{n.}{\headword{pollgo ttatta kuttang} \definition{1. weaving pattern with concentric diamonds}}
\item \entry{n.}{\headword{pollon} \definition{1. type of small bush}}
\item \entry{n.}{\headword{pollon} \definition{1. nonsingular form of pollon}}
\item \entry{n.}{\headword{pollon bällabollott} \definition{1. type of big taro}}
\item \entry{n.}{\headword{pom} \definition{1. type of long yam with a white interior, thorns, and hairs}}
\item \entry{n.}{\headword{poma} \definition{1. type of pandanus with leaves that women weave into mats}}
\item \entry{n.}{\headword{pomana} \definition{1. fishing style in which men climb trees and shoot with ddangol spears}}
\item \entry{n.}{\headword{pomer} \definition{1. whistle}}
\item \entry{n.}{\headword{pomila} \definition{1. type of citrus tree with big, edible fruit}}
\item \entry{n.}{\headword{pondo} \definition{1. type of tree that grows in the swamp and old gardens with yellow flowers and pod-shaped fruit with inedible seeds}}
\item \entry{n.}{\headword{ponong} \definition{1. type of tree that grows in grassland with white flowers and small green fruit and wood used for posts}}
\item \entry{n.}{\headword{ponganem} \definition{1. type of small yam with a white interior, hairs, and thorns}}
\item \entry{n.}{\headword{pongoll} \definition{1. type of yam with a white interior and no hairs or thorns}}
\item \entry{n.}{\headword{pop} \definition{1. hole}}
\item \entry{n.}{\headword{popell} \definition{1. type of introduced banana}}
\item \entry{n.}{\headword{popell} \definition{1. dish consisting of ants and various types of bananas (incl. popell)}}
\item \entry{n.}{\headword{popo2} \definition{1. flower}}
\item \entry{n.}{\headword{porak} \definition{1. type of spear}}
\item \entry{n.}{\headword{porma} \definition{1. traditional dish consisting of meat on top of sago cooked in sago or banana leaves}}
\item \entry{n.}{\headword{pos} \definition{1. post}}
\item \entry{n.}{\headword{poses} \definition{1. martin (fairy, tree)}}
\item \entry{n.}{\headword{poses} \definition{1. Pacific swift}}
\item \entry{n.}{\headword{pot} \definition{1. tip, end, point; base (of yam)}}
\item \entry{n.}{\headword{pot} \definition{1. talent, gift}}
\item \entry{n.}{\headword{pot} \definition{1. vagina of a mammal}}
\item \entry{n.}{\headword{potkam} \definition{1. type of cultivated tree that also grows in the savanna; treats cough and aches}}
\item \entry{n.}{\headword{potne kätt} \definition{1. bivalve_type}}
\item \entry{n.}{\headword{potopoto} \definition{1. type of tree that grows near swamps and creeks with yellow and white flowers and fruit that falls on the ground; animals eat it}}
\item \entry{n.}{\headword{praedde} \definition{1. Friday}}
\item \entry{n.}{\headword{praemari skul} \definition{1. primary school}}
\item \entry{n.}{\headword{prep} \definition{1. preparatory school}}
\item \entry{n.}{\headword{probens} \definition{1. province}}
\item \entry{n.}{\headword{pu} \definition{1. floating grass or island in swamp}}
\item \entry{n.}{\headword{pu} \definition{1. swamp garden}}
\item \entry{n.}{\headword{pud} \definition{1. end (of a long object)}}
\item \entry{n.}{\headword{puder} \definition{1. type of long grass}}
\item \entry{n.}{\headword{pudd1} \definition{1. place with flattened grass where wallabies sleep}}
\item \entry{n.}{\headword{pudd2} \definition{1. taro shoot}}
\item \entry{n.}{\headword{pugupugu} \definition{1. collared imperial pigeon}}
\item \entry{n.}{\headword{puikmetutu} \definition{1. type of gecko}}
\item \entry{n.}{\headword{puku} \definition{1. type of arrow}}
\item \entry{n.}{\headword{puku kubull} \definition{1. type of bush wallaby with white tail}}
\item \entry{n.}{\headword{pullkom} \definition{1. tailfeather}}
\item \entry{n.}{\headword{pumkak} \definition{1. type of sago}}
\item \entry{n.}{\headword{pupulläkäm} \definition{1. type of mushroom}}
\item \entry{n.}{\headword{pus} \definition{1. cat}}
\item \entry{n.}{\headword{put} \definition{1. type of brown bark}}
\item \entry{n.}{\headword{putt} \definition{1. hoof}}
\item \entry{n.}{\headword{putt kämamkämam} \definition{1. type of sago bundle wrapped with sago leaves and tied at one end}}
\item \entry{n.}{\headword{puyem} \definition{1. hunting event in which one person hits the ground with a short stick}}
\item \entry{n.}{\headword{puzi} \definition{1. type of bird with crown}}
\item \entry{n.}{\headword{rabis} \definition{1. rubbish, trash}}
\item \entry{n.}{\headword{rarae} \definition{1. hunting technique in which hunters line up to cover ground in the absence of dogs}}
\item \entry{n.}{\headword{rarae} \definition{1. fishing technique in which ladies line up with nets}}
\item \entry{n.}{\headword{rata} \definition{1. type of flowering plant with red bracts and nectar that attracts insects}}
\item \entry{n.}{\headword{räba} \definition{1. rubber; eraser}}
\item \entry{n.}{\headword{rädräd} \definition{1. type of grey fish that lives in the swamp}}
\item \entry{n.}{\headword{retam} \definition{1. type of big yam}}
\item \entry{n.}{\headword{rir} \definition{1. type of medium-sized bamboo}}
\item \entry{n.}{\headword{rol} \definition{1. type of hairy caterpillar}}
\item \entry{n.}{\headword{rolkutt} \definition{1. crawling grass}}
\item \entry{n.}{\headword{rop} \definition{1. rope}}
\item \entry{n.}{\headword{rubi} \definition{1. type of long yam with a white interior, white skin, and no thorns}}
\item \entry{n.}{\headword{rubi} \definition{1. type of arrow}}
\item \entry{n.}{\headword{rupi} \definition{1. type of leaf}}
\item \entry{n.}{\headword{ruriruri} \definition{1. earthquake}}
\item \entry{n.}{\headword{sabi} \definition{1. law, rule}}
\item \entry{n.}{\headword{sabi} \definition{1. taboo}}
\item \entry{n.}{\headword{sabol} \definition{1. shovel, spade}}
\item \entry{n.}{\headword{saen} \definition{1. sign}}
\item \entry{n.}{\headword{sagwan} \definition{1. woodworking tool}}
\item \entry{n.}{\headword{sakar} \definition{1. type of edible pandanus}}
\item \entry{n.}{\headword{saks} \definition{1. socks}}
\item \entry{n.}{\headword{sambuag} \definition{1. lobster; prawn}}
\item \entry{n.}{\headword{samoa} \definition{1. type of introduced banana}}
\item \entry{n.}{\headword{sana} \definition{1. sago}}
\item \entry{n.}{\headword{sana} \definition{1. sago grub (larva of black palm weevil)}}
\item \entry{n.}{\headword{sana} \definition{1. sago grub (larva of black palm weevil)}}
\item \entry{n.}{\headword{sana} \definition{1. the second stage of sago growth during which the plant is ~3 m tall}}
\item \entry{n.}{\headword{sana} \definition{1. type of mushroom}}
\item \entry{n.}{\headword{sana} \definition{1. sago pith}}
\item \entry{n.}{\headword{sana} \definition{1. area where sago grows}}
\item \entry{n.}{\headword{sana} \definition{1. parable}}
\item \entry{n.}{\headword{sana tätäkang} \definition{1. type of spear}}
\item \entry{n.}{\headword{sana wutwut} \definition{1. sago dish cooked with pieces of banana leaf between the layers}}
\item \entry{n.}{\headword{sanasana} \definition{1. type of edible sago}}
\item \entry{n.}{\headword{sandana} \definition{1. type of black palm; in this middle stage, the pith is chewed like sugarcane during the dry season for its moisture}}
\item \entry{n.}{\headword{sande} \definition{1. Sunday}}
\item \entry{n.}{\headword{sande} \definition{1. week}}
\item \entry{n.}{\headword{sanga} \definition{1. black-necked stork}}
\item \entry{n.}{\headword{sangapawi} \definition{1. type of big, round yam with a white interior, white or red skin, hairs, and no thorns}}
\item \entry{n.}{\headword{saomasaoma} \definition{1. armpit}}
\item \entry{n.}{\headword{saomasaoma} \definition{1. dancing band}}
\item \entry{n.}{\headword{sapebllabllot} \definition{1. type of taro}}
\item \entry{n.}{\headword{saper ine} \definition{1. clean water}}
\item \entry{n.}{\headword{sapiri} \definition{1. type of flowering plant}}
\item \entry{n.}{\headword{sare} \definition{1. type of big taro}}
\item \entry{n.}{\headword{sat} \definition{1. widower}}
\item \entry{n.}{\headword{satade} \definition{1. Saturday}}
\item \entry{n.}{\headword{saus} \definition{1. type of yam with a white or purple interior, thorns, and no hairs}}
\item \entry{n.}{\headword{sawasap} \definition{1. type of cultivated tree with edible yellow or green fruit}}
\item \entry{n.}{\headword{sawis} \definition{1. type of small yam with a long shape and purple interior}}
\item \entry{n.}{\headword{sawiya} \definition{1. little egret}}
\item \entry{n.}{\headword{sägäsägäd manika} \definition{1. yellow cassava}}
\item \entry{n.}{\headword{säkar} \definition{1. type of big palm tree that grows near big rivers with white flowers and hanging red fruit that cassowaries eat}}
\item \entry{n.}{\headword{sänd} \definition{1. type of big yam with a white interior and few hairs}}
\item \entry{n.}{\headword{säne} \definition{1. type of yam}}
\item \entry{n.}{\headword{säpalek} \definition{1. type of bag}}
\item \entry{n.}{\headword{sära} \definition{1. tail}}
\item \entry{n.}{\headword{sära} \definition{1. end of a bow}}
\item \entry{n.}{\headword{sära} \definition{1. uncut grass skirt}}
\item \entry{n.}{\headword{sära} \definition{1. sacrum (bone)}}
\item \entry{n.}{\headword{sära} \definition{1. caudal fin}}
\item \entry{n.}{\headword{sära} \definition{1. mesothorax}}
\item \entry{n.}{\headword{sära} \definition{1. tip of tail}}
\item \entry{n.}{\headword{sära} \definition{1. lastborn, youngest}}
\item \entry{n.}{\headword{sära} \definition{1. lastborn, youngest}}
\item \entry{n.}{\headword{särem} \definition{1. darkness, dark}}
\item \entry{n.}{\headword{särem} \definition{1. dark, dim}}
\item \entry{n.}{\headword{särem} \definition{1. cloudy, dim}}
\item \entry{n.}{\headword{särem} \definition{1. in darkness, in the dark}}
\item \entry{n.}{\headword{säremang ma} \definition{1. prison, jail}}
\item \entry{n.}{\headword{säs} \definition{1. type of sago wrapped in young leaves}}
\item \entry{n.}{\headword{säspen} \definition{1. pot, saucepan}}
\item \entry{n.}{\headword{SDA} \definition{1. SDA (Seventh-day Adventist Church)}}
\item \entry{n.}{\headword{se1} \definition{1. yes}}
\item \entry{n.}{\headword{sel1} \definition{1. cell}}
\item \entry{n.}{\headword{sele} \definition{1. chili}}
\item \entry{n.}{\headword{sem1} \definition{1. type of tree that grows in the bush; used for making rope}}
\item \entry{n.}{\headword{sidompa} \definition{1. type of spear}}
\item \entry{n.}{\headword{sigip} \definition{1. type of palm with fruit hanging from long pedicels; people chew the fruit like betelnut}}
\item \entry{n.}{\headword{sigip} \definition{1. type of big taro}}
\item \entry{n.}{\headword{siklakla} \definition{1. golden-headed cisticola}}
\item \entry{n.}{\headword{siklakla} \definition{1. Australian reed warbler}}
\item \entry{n.}{\headword{siklakla} \definition{1. singing bush lark}}
\item \entry{n.}{\headword{sip1} \definition{1. sheep}}
\item \entry{n.}{\headword{sip2} \definition{1. chief}}
\item \entry{n.}{\headword{sipik} \definition{1. type of large yam with a white and purple interior}}
\item \entry{n.}{\headword{siporo} \definition{1. type of cultivated tree with thorns and sour yellow fruit}}
\item \entry{n.}{\headword{sir} \definition{1. type of tree that grows in the bush with white flowers and edible black fruit with one seed inside; cassowary and pigs eat the fruit}}
\item \entry{n.}{\headword{sis1} \definition{1. season when new gardens are cleared and fenced (fourteenth season; corresponds to early November)}}
\item \entry{n.}{\headword{sis2} \definition{1. type of flying ant that come out from their flooded anthills in November}}
\item \entry{n.}{\headword{sisel} \definition{1. chisel}}
\item \entry{n.}{\headword{sisi} \definition{1. type of pandanus with a trunk used for wood}}
\item \entry{n.}{\headword{sisi} \definition{1. the third stage of coconut growth in which the seedling has formed (after planting)}}
\item \entry{n.}{\headword{sisor pazi} \definition{1. season of New Years (sixteenth season; corresponds to December)}}
\item \entry{n.}{\headword{sispull} \definition{1. maggot}}
\item \entry{n.}{\headword{sisri} \definition{1. this morning}}
\item \entry{n.}{\headword{sisri} \definition{1. today}}
\item \entry{n.}{\headword{sisri} \definition{1. tonight}}
\item \entry{n.}{\headword{sisri} \definition{1. this evening}}
\item \entry{n.}{\headword{siti} \definition{1. city}}
\item \entry{n.}{\headword{sɨmell} \definition{1. pig}}
\item \entry{n.}{\headword{sɨmell} \definition{1. truck}}
\item \entry{n.}{\headword{sɨmell källamokott} \definition{1. type of spiky tree that grows in the bush with white flowers, yellow fruit, and hardwood used for firewood}}
\item \entry{n.}{\headword{sɨmellkom} \definition{1. type of sago}}
\item \entry{n.}{\headword{sɨmellmak} \definition{1. weaving pattern with arrows pointing right}}
\item \entry{n.}{\headword{skul} \definition{1. school; education}}
\item \entry{n.}{\headword{skul} \definition{1. student}}
\item \entry{n.}{\headword{skul} \definition{1. educated}}
\item \entry{n.}{\headword{slaslak} \definition{1. red-winged parrot}}
\item \entry{n.}{\headword{so1} \definition{1. type of black palm; in this mature stage (~3 m), the wood is used for house flooring and containers for squeezing sago, and the pith is eaten}}
\item \entry{n.}{\headword{sod} \definition{1. shirt}}
\item \entry{n.}{\headword{soka} \definition{1. soccer}}
\item \entry{n.}{\headword{sokpa} \definition{1. tobacco}}
\item \entry{n.}{\headword{sokpa} \definition{1. cigarette}}
\item \entry{n.}{\headword{sokpa kllokllop} \definition{1. notebook-sized mat woven of tobacco}}
\item \entry{n.}{\headword{sokpa llaweatt} \definition{1. tobacco woven like a rope}}
\item \entry{n.}{\headword{solt} \definition{1. salt}}
\item \entry{n.}{\headword{soroe} \definition{1. test, exam}}
\item \entry{n.}{\headword{sos} \definition{1. church}}
\item \entry{n.}{\headword{sosoga} \definition{1. type of sago}}
\item \entry{n.}{\headword{spallma} \definition{1. two friends who split a twin coconut frond}}
\item \entry{n.}{\headword{spun1} \definition{1. spoon}}
\item \entry{n.}{\headword{stoa} \definition{1. store}}
\item \entry{n.}{\headword{stori} \definition{1. story}}
\item \entry{n.}{\headword{su} \definition{1. prey}}
\item \entry{n.}{\headword{suga} \definition{1. sugar}}
\item \entry{n.}{\headword{suga galbe} \definition{1. type of large yam with a white interior}}
\item \entry{n.}{\headword{sulut} \definition{1. type of taro}}
\item \entry{n.}{\headword{sur} \definition{1. pushing tool}}
\item \entry{n.}{\headword{surum} \definition{1. sand}}
\item \entry{n.}{\headword{surusuru} \definition{1. type of tree that grows in the bush with white flowers and strong wood used for house posts and firewood; it was traditionally lit and used as a light at night because it burns slowly}}
\item \entry{n.}{\headword{susu} \definition{1. breastmilk}}
\item \entry{n.}{\headword{suwe} \definition{1. urine, pee}}
\item \entry{n.}{\headword{suwe} \definition{1. (euphemistic) testicles}}
\item \entry{n.}{\headword{suwe} \definition{1. to urinate on, pee on}}
\item \entry{n.}{\headword{suwe gäd} \definition{1. body part}}
\item \entry{n.}{\headword{tab} \definition{1. promise, oath; engagement}}
\item \entry{n.}{\headword{tabe} \definition{1. soft pad or support that placed between one's back and a spalek basket}}
\item \entry{n.}{\headword{tada} \definition{1. fish trap}}
\item \entry{n.}{\headword{tae} \definition{1. type of tall tree that grows by rivers with white flowers, green fruit, and sturdy wood that can be used as a bridge}}
\item \entry{n.}{\headword{tae} \definition{1. type of edible grub}}
\item \entry{n.}{\headword{tae} \definition{1. type of mushroom}}
\item \entry{n.}{\headword{taem} \definition{1. time}}
\item \entry{n.}{\headword{taemataema} \definition{1. type of tree that grows near the swamp with long yellow flowers and leaves that are used to treat fungal skin infections}}
\item \entry{n.}{\headword{taewa} \definition{1. type of bird}}
\item \entry{n.}{\headword{tainäm} \definition{1. mosquito net}}
\item \entry{n.}{\headword{talapia} \definition{1. tilapia}}
\item \entry{n.}{\headword{tall} \definition{1. type of tree with wood used for posts and easily-peeling bark used for walling; also lit as a torch}}
\item \entry{n.}{\headword{tamatama} \definition{1. bean}}
\item \entry{n.}{\headword{tame} \definition{1. wave (of water)}}
\item \entry{n.}{\headword{tameny} \definition{1. teacher}}
\item \entry{n.}{\headword{tamllägtamlläg} \definition{1. type of small caterpillar}}
\item \entry{n.}{\headword{tamongle} \definition{1. type of mutae yam shaped like a foot}}
\item \entry{n.}{\headword{tan} \definition{1. type of short plant with white flowers, brown fruit, and branches used as a broom}}
\item \entry{n.}{\headword{tan} \definition{1. broom, can come from different types of palm e.g. käg tan, mäta tan}}
\item \entry{n.}{\headword{tanteny} \definition{1. type of tree}}
\item \entry{n.}{\headword{tanyib} \definition{1. radioulna (fused radius and ulna, or arm bone) of a flying fox}}
\item \entry{n.}{\headword{tap3} \definition{1. tunnel}}
\item \entry{n.}{\headword{tap4} \definition{1. type of big yam with a white interior, thorns, and hairs}}
\item \entry{n.}{\headword{tarakoll} \definition{1. protective outer wall built by ancestors}}
\item \entry{n.}{\headword{tarasoso} \definition{1. type of bird}}
\item \entry{n.}{\headword{tarko} \definition{1. female friends who have shared a joined fruit or vegetable}}
\item \entry{n.}{\headword{tarme} \definition{1. kookaburra (blue-winged; spangled)}}
\item \entry{n.}{\headword{tarme ballmenyang} \definition{1. season when crops are bearing fruit (fifth season; corresponds to March)}}
\item \entry{n.}{\headword{tarme koeme} \definition{1. type of tree}}
\item \entry{n.}{\headword{tarmekälla} \definition{1. type of taro}}
\item \entry{n.}{\headword{tata2} \definition{1. junction, intersection}}
\item \entry{n.}{\headword{tatanggli} \definition{1. willie wagtail}}
\item \entry{n.}{\headword{tatäraem} \definition{1. noise}}
\item \entry{n.}{\headword{tater} \definition{1. mat}}
\item \entry{n.}{\headword{tatruk} \definition{1. type of poisonous ant}}
\item \entry{n.}{\headword{tawa} \definition{1. swamp}}
\item \entry{n.}{\headword{tawa aeb} \definition{1. Australasian swamphen}}
\item \entry{n.}{\headword{tawar} \definition{1. totem symbol (thing that represents a clan)}}
\item \entry{n.}{\headword{tawe} \definition{1. type of slim, tall palm with coconuts that come in red or green varieties, bunches of yellow fruit that birds eat, and fronds used for camp flooring}}
\item \entry{n.}{\headword{tawe ttäp} \definition{1. type of spear}}
\item \entry{n.}{\headword{tawekutt} \definition{1. partially dry coconut}}
\item \entry{n.}{\headword{tawekutt} \definition{1. the tenth stage of coconut growth during which the fruit begins to dry and brown}}
\item \entry{n.}{\headword{täb} \definition{1. type of tree}}
\item \entry{n.}{\headword{täbatäbe} \definition{1. plan}}
\item \entry{n.}{\headword{täbädd} \definition{1. guest, visitor, stranger}}
\item \entry{n.}{\headword{täbäll pud} \definition{1. type of biting bee found in trees}}
\item \entry{n.}{\headword{täbäll pud} \definition{1. type of tree}}
\item \entry{n.}{\headword{täbe} \definition{1. type of big tree that grows in the bush with white flowers and drupes that fall with an edible seed inside}}
\item \entry{n.}{\headword{täbe} \definition{1. type of grub}}
\item \entry{n.}{\headword{täbe} \definition{1. type of mushroom}}
\item \entry{n.}{\headword{täbie} \definition{1. black sunbird}}
\item \entry{n.}{\headword{täbie} \definition{1. garden sunbird}}
\item \entry{n.}{\headword{täbom} \definition{1. type of small yam with a white interior, hairs, and thorns}}
\item \entry{n.}{\headword{täk} \definition{1. clitoris}}
\item \entry{n.}{\headword{täkäll} \definition{1. fin}}
\item \entry{n.}{\headword{täkäll} \definition{1. horn}}
\item \entry{n.}{\headword{täkäll} \definition{1. thorn}}
\item \entry{n.}{\headword{täkälltäkäll kutt} \definition{1. spine}}
\item \entry{n.}{\headword{täkälluit} \definition{1. radial cartilage}}
\item \entry{n.}{\headword{täkla} \definition{1. tree type}}
\item \entry{n.}{\headword{täko} \definition{1. any body part}}
\item \entry{n.}{\headword{täl} \definition{1. type of large bamboo that grows anywhere; used for bows, bow strings, axe handles, spears, and clothing pegs}}
\item \entry{n.}{\headword{täl wadär} \definition{1. bamboo string}}
\item \entry{n.}{\headword{tältäl} \definition{1. type of grass}}
\item \entry{n.}{\headword{täma} \definition{1. husk, exocarp/mesocarp (of coconut)}}
\item \entry{n.}{\headword{tämallang mälla} \definition{1. type of big taro}}
\item \entry{n.}{\headword{tämani} \definition{1. type of large yam with a white or white and red interior}}
\item \entry{n.}{\headword{täme} \definition{1. monitor lizard, goanna}}
\item \entry{n.}{\headword{täme käp} \definition{1. type of round yam with a white interior, red skin, and hairs}}
\item \entry{n.}{\headword{täme sära} \definition{1. bush rope}}
\item \entry{n.}{\headword{tän} \definition{1. tribe, nation, people}}
\item \entry{n.}{\headword{tän} \definition{1. clan}}
\item \entry{n.}{\headword{tän} \definition{1. stem}}
\item \entry{n.}{\headword{täny} \definition{1. lesser fig-parrot}}
\item \entry{n.}{\headword{täny} \definition{1. red-flanked lorikeet}}
\item \entry{n.}{\headword{täp1} \definition{1. sago shoot}}
\item \entry{n.}{\headword{täp1} \definition{1. the fourth stage of sago growth in which the shoot emerges, indicating the pith is ready to be harvested}}
\item \entry{n.}{\headword{täpäll} \definition{1. type of pandanus used in a traditional hairstyle and woven together to make a big mat}}
\item \entry{n.}{\headword{täpe} \definition{1. mud}}
\item \entry{n.}{\headword{täpe} \definition{1. muddy}}
\item \entry{n.}{\headword{tär} \definition{1. string, line}}
\item \entry{n.}{\headword{tär} \definition{1. strap or handle of a bag/basket (nyäng)}}
\item \entry{n.}{\headword{tärabol} \definition{1. trouble}}
\item \entry{n.}{\headword{täral} \definition{1. type of tree that grows in the grassland with white flowers, brown fruit, and wood used for house posts}}
\item \entry{n.}{\headword{täral pällämpälläm} \definition{1. type of tree}}
\item \entry{n.}{\headword{tärangesa} \definition{1. pinky, little finger}}
\item \entry{n.}{\headword{täräb} \definition{1. a dead person's belongings (put outside at a funeral until the feast ends)}}
\item \entry{n.}{\headword{täräb} \definition{1. funeral home}}
\item \entry{n.}{\headword{täräk1} \definition{1. bundle}}
\item \entry{n.}{\headword{täräm} \definition{1. tear}}
\item \entry{n.}{\headword{tärämpmeny} \definition{1. sleeping with a dead person's belongings}}
\item \entry{n.}{\headword{täräp1} \definition{1. time}}
\item \entry{n.}{\headword{täräp3} \definition{1. hunting tally (collection of jaw bones often displayed in the front of people's homes)}}
\item \entry{n.}{\headword{tärbatt} \definition{1. orphan}}
\item \entry{n.}{\headword{täre} \definition{1. feast}}
\item \entry{n.}{\headword{täre} \definition{1. Scripture}}
\item \entry{n.}{\headword{täre} \definition{1. temple}}
\item \entry{n.}{\headword{tärke käp} \definition{1. necklace}}
\item \entry{n.}{\headword{tärko} \definition{1. type of small fish}}
\item \entry{n.}{\headword{tärpa} \definition{1. yam skin}}
\item \entry{n.}{\headword{tärpa} \definition{1. leftovers}}
\item \entry{n.}{\headword{tärpae} \definition{1. type of yam}}
\item \entry{n.}{\headword{tärpoll} \definition{1. piece}}
\item \entry{n.}{\headword{tät1} \definition{1. stretcher}}
\item \entry{n.}{\headword{tät1} \definition{1. ladder}}
\item \entry{n.}{\headword{tät2} \definition{1. type of insect}}
\item \entry{n.}{\headword{tätäb} \definition{1. swelling}}
\item \entry{n.}{\headword{tätäk} \definition{1. flea}}
\item \entry{n.}{\headword{tätämall} \definition{1. type of small insect that glow green}}
\item \entry{n.}{\headword{tätän} \definition{1. rib, side, flank}}
\item \entry{n.}{\headword{tätän} \definition{1. animal trap}}
\item \entry{n.}{\headword{tätän} \definition{1. rib (bone)}}
\item \entry{n.}{\headword{tätäp} \definition{1. pain}}
\item \entry{n.}{\headword{tätäräp1} \definition{1. man-made clearing}}
\item \entry{n.}{\headword{tätäräp2} \definition{1. heat}}
\item \entry{n.}{\headword{tätärpeyam} \definition{1. type of grass (~1 m) used as a toy}}
\item \entry{n.}{\headword{tätkea} \definition{1. type of sago}}
\item \entry{n.}{\headword{tätmar} \definition{1. firefly}}
\item \entry{n.}{\headword{täträp2} \definition{1. season of the end of harvesting and the start of hunting (eighth season; corresponds to late May)}}
\item \entry{n.}{\headword{teb} \definition{1. mandible, jawbone}}
\item \entry{n.}{\headword{teks} \definition{1. tax}}
\item \entry{n.}{\headword{tentatente} \definition{1. category of parasitic plants}}
\item \entry{n.}{\headword{tep1} \definition{1. tree sap (used in making drums and arrows)}}
\item \entry{n.}{\headword{tesde} \definition{1. Thursday}}
\item \entry{n.}{\headword{teweya} \definition{1. brush cuckoo}}
\item \entry{n.}{\headword{tibekllop} \definition{1. type of medicine}}
\item \entry{n.}{\headword{tibi} \definition{1. tuberculosis}}
\item \entry{n.}{\headword{tibra} \definition{1. type of native banana}}
\item \entry{n.}{\headword{tigi} \definition{1. type of snake}}
\item \entry{n.}{\headword{tikle} \definition{1. type of small louse}}
\item \entry{n.}{\headword{tikop} \definition{1. heart}}
\item \entry{n.}{\headword{tikop} \definition{1. beloved, sweetheart, dear}}
\item \entry{n.}{\headword{tikop} \definition{1. startled, scared}}
\item \entry{n.}{\headword{tikuku} \definition{1. black-backed bittern}}
\item \entry{n.}{\headword{tikuku} \definition{1. pheasant coucal}}
\item \entry{n.}{\headword{timän} \definition{1. end}}
\item \entry{n.}{\headword{tine} \definition{1. sago leaf}}
\item \entry{n.}{\headword{tine} \definition{1. type of simple roof consisting of a single post covered in sago leaves}}
\item \entry{n.}{\headword{tintromol} \definition{1. type of black or red ants}}
\item \entry{n.}{\headword{tintromoltintromol} \definition{1. type of rhyming game}}
\item \entry{n.}{\headword{tirit} \definition{1. cotton}}
\item \entry{n.}{\headword{titi1} \definition{1. brown honeyeater}}
\item \entry{n.}{\headword{titi1} \definition{1. brown-backed honeyeater}}
\item \entry{n.}{\headword{titi1} \definition{1. yellow-bellied longbill}}
\item \entry{n.}{\headword{titi1} \definition{1. rufous-banded honeyeater}}
\item \entry{n.}{\headword{titi1} \definition{1. pygmy longbill}}
\item \entry{n.}{\headword{titi2} \definition{1. bat}}
\item \entry{n.}{\headword{tɨke} \definition{1. magic_type}}
\item \entry{n.}{\headword{tɨn} \definition{1. steam}}
\item \entry{n.}{\headword{tɨt2} \definition{1. type of tree found in the bush}}
\item \entry{n.}{\headword{to} \definition{1. light}}
\item \entry{n.}{\headword{tobäll} \definition{1. long spear shot like an arrow}}
\item \entry{n.}{\headword{tobäll} \definition{1. arrowhead}}
\item \entry{n.}{\headword{tobäll} \definition{1. bullet}}
\item \entry{n.}{\headword{tobäll abal} \definition{1. type of spear}}
\item \entry{n.}{\headword{toe} \definition{1. type of tree that grows in the bush with white flowers, black fruit, small edible nuts that doves and cassowaries eat, and pith that is steamed or extracted}}
\item \entry{n.}{\headword{toelet} \definition{1. toilet}}
\item \entry{n.}{\headword{toengg} \definition{1. confluence, junction (of a river)}}
\item \entry{n.}{\headword{togol} \definition{1. hide-and-seek}}
\item \entry{n.}{\headword{tokom} \definition{1. bait}}
\item \entry{n.}{\headword{tokong} \definition{1. bait}}
\item \entry{n.}{\headword{tokop} \definition{1. type of tree that grows in the bush; used as house sticks and medicine}}
\item \entry{n.}{\headword{tokop} \definition{1. lump on skin}}
\item \entry{n.}{\headword{toma} \definition{1. wing}}
\item \entry{n.}{\headword{toma} \definition{1. metathorax}}
\item \entry{n.}{\headword{tomato} \definition{1. tomato}}
\item \entry{n.}{\headword{tomäll} \definition{1. wart; fungal skin infection}}
\item \entry{n.}{\headword{tonggo} \definition{1. type of small bamboo that grows in the bush along creeks with a red interior; sharp when split and used as a cutting tool}}
\item \entry{n.}{\headword{tongle} \definition{1. leech}}
\item \entry{n.}{\headword{tongoe} \definition{1. game; sport}}
\item \entry{n.}{\headword{tongoe} \definition{1. playground}}
\item \entry{n.}{\headword{tongoe} \definition{1. funny, humorous}}
\item \entry{n.}{\headword{topoll} \definition{1. flying fox}}
\item \entry{n.}{\headword{topotopoll} \definition{1. type of tree that grows in the bush with white flowers; used as a yam stick}}
\item \entry{n.}{\headword{torep} \definition{1. brown quail}}
\item \entry{n.}{\headword{toro} \definition{1. a symbol of a slain animal, used to display and inform people of the type of animal killed. It was also a hunter's pride to compete or show other hunter' of his skill. Feathers, offcut tail, or animal fur were displayed on a small stick or pitpit or obäll tree stick or stem. In other instances pandanus leaves were symbols for pig, grass for wallaby.}}
\item \entry{n.}{\headword{torok} \definition{1. type of cane used for building houses, bows, and canoes}}
\item \entry{n.}{\headword{toronggogo} \definition{1. bar-shouldered dove}}
\item \entry{n.}{\headword{tos} \definition{1. (Commonwealth) torch, (US) flashlight}}
\item \entry{n.}{\headword{tot1} \definition{1. type of tree that gows along creeks with white and blue flowers and bark used to weave bags or sago baskets}}
\item \entry{n.}{\headword{tot2} \definition{1. rubbish, trash, junk}}
\item \entry{n.}{\headword{tot3} \definition{1. piece}}
\item \entry{n.}{\headword{totkoll} \definition{1. puddle}}
\item \entry{n.}{\headword{toto1} \definition{1. afternoon; early evening (approx. 1 PM–5 PM)}}
\item \entry{n.}{\headword{toto1} \definition{1. dinner}}
\item \entry{n.}{\headword{toto2} \definition{1. vertical house post}}
\item \entry{n.}{\headword{toto2} \definition{1. horizontal house post}}
\item \entry{n.}{\headword{toto nyäknyäk} \definition{1. Australian owlet-nightjar}}
\item \entry{n.}{\headword{towall} \definition{1. grass}}
\item \entry{n.}{\headword{towall} \definition{1. grassy}}
\item \entry{n.}{\headword{towallpipi} \definition{1. type of venomous snake}}
\item \entry{n.}{\headword{traeb} \definition{1. tribe}}
\item \entry{n.}{\headword{trak1} \definition{1. truck}}
\item \entry{n.}{\headword{trik} \definition{1. trick}}
\item \entry{n.}{\headword{tuba} \definition{1. coconut drink traditionally made to welcome people}}
\item \entry{n.}{\headword{tubi} \definition{1. type of spear made from sago leaf used to kill birds}}
\item \entry{n.}{\headword{tubu1} \definition{1. end; stump}}
\item \entry{n.}{\headword{tubu2} \definition{1. knee}}
\item \entry{n.}{\headword{tubu2} \definition{1. kneecap}}
\item \entry{n.}{\headword{tubu2} \definition{1. kneeling, on one's knees, on the ground; worshipping}}
\item \entry{n.}{\headword{tudi} \definition{1. fishing}}
\item \entry{n.}{\headword{tudi} \definition{1. fishing rod, fishing line}}
\item \entry{n.}{\headword{tudi} \definition{1. fishing style in which bait is put on fishing lines that are left in the river and checked later}}
\item \entry{n.}{\headword{tudi} \definition{1. fishing hook}}
\item \entry{n.}{\headword{tudi} \definition{1. fishing line}}
\item \entry{n.}{\headword{tugul} \definition{1. type of big tree that grows in the bush with green, leaflike flowers and straight wood used for house sticks}}
\item \entry{n.}{\headword{tuk} \definition{1. air}}
\item \entry{n.}{\headword{tuk} \definition{1. technical university, university}}
\item \entry{n.}{\headword{tuk} \definition{1. uphill}}
\item \entry{n.}{\headword{tuk} \definition{1. more than, above, over}}
\item \entry{n.}{\headword{tum2} \definition{1. heap}}
\item \entry{n.}{\headword{tum2} \definition{1. plenty, many}}
\item \entry{n.}{\headword{tum2} \definition{1. to gather}}
\item \entry{n.}{\headword{tumku} \definition{1. back of head}}
\item \entry{n.}{\headword{tupi2} \definition{1. pointer finger, index finger}}
\item \entry{n.}{\headword{tupol} \definition{1. type of spear}}
\item \entry{n.}{\headword{turik} \definition{1. axe}}
\item \entry{n.}{\headword{turku} \definition{1. thinner piece used withboreto smoke tobacco}}
\item \entry{n.}{\headword{turllo} \definition{1. type of lizard}}
\item \entry{n.}{\headword{turwe} \definition{1. shining bronze cuckoo}}
\item \entry{n.}{\headword{tusde} \definition{1. Tuesday}}
\item \entry{n.}{\headword{tutu1} \definition{1. mountain, hill}}
\item \entry{n.}{\headword{tutu1} \definition{1. land}}
\item \entry{n.}{\headword{tutu1} \definition{1. steep}}
\item \entry{n.}{\headword{tutu1} \definition{1. nonsingular form of tutu}}
\item \entry{n.}{\headword{tutuaram} \definition{1. type of taro}}
\item \entry{n.}{\headword{tuwetuwe} \definition{1. type of small tree that grows in the bush with white flowers and edible red fruit}}
\item \entry{n.}{\headword{tuwi} \definition{1. type of tree with white flowers, yellow fruit, and wood used for posts}}
\item \entry{n.}{\headword{tuwok} \definition{1. type of tree}}
\item \entry{n.}{\headword{ttaba} \definition{1. plant_type}}
\item \entry{n.}{\headword{ttaepnenttaepnen ma skul} \definition{1. technical school}}
\item \entry{n.}{\headword{ttalamttalam1} \definition{1. type of tree}}
\item \entry{n.}{\headword{ttalme} \definition{1. type of floating grass that grass, deer, and wallaby eat}}
\item \entry{n.}{\headword{ttall} \definition{1. agile wallaby, sandy wallaby}}
\item \entry{n.}{\headword{ttall ip} \definition{1. type of tree that grows in the bush, grassland, and along creeks with indigo flowers, bark that is chewed, and liquid used as glue for spears}}
\item \entry{n.}{\headword{ttall källa} \definition{1. brahminy kite}}
\item \entry{n.}{\headword{ttall mätta} \definition{1. type of big yam with a white interior and few hairs}}
\item \entry{n.}{\headword{ttall nge} \definition{1. type of palm with yellow leaves and coconuts with a yellow exocarp}}
\item \entry{n.}{\headword{ttall ttoe} \definition{1. type of tree}}
\item \entry{n.}{\headword{ttam1} \definition{1. life}}
\item \entry{n.}{\headword{ttam1} \definition{1. life}}
\item \entry{n.}{\headword{ttam4} \definition{1. leaf}}
\item \entry{n.}{\headword{ttambällag} \definition{1. the second stage of coconut growth in which it is planted}}
\item \entry{n.}{\headword{ttang} \definition{1. hand}}
\item \entry{n.}{\headword{ttang} \definition{1. arm}}
\item \entry{n.}{\headword{ttang} \definition{1. palm}}
\item \entry{n.}{\headword{ttang} \definition{1. elbow}}
\item \entry{n.}{\headword{ttang} \definition{1. on all fours}}
\item \entry{n.}{\headword{ttang} \definition{1. talon, claw}}
\item \entry{n.}{\headword{ttang} \definition{1. finger}}
\item \entry{n.}{\headword{ttang} \definition{1. palm}}
\item \entry{n.}{\headword{ttang} \definition{1. upper part of foreleg (of a reptile)}}
\item \entry{n.}{\headword{ttang} \definition{1. bracelet}}
\item \entry{n.}{\headword{ttang} \definition{1. to clap}}
\item \entry{n.}{\headword{ttang} \definition{1. to shake hands with}}
\item \entry{n.}{\headword{ttang} \definition{1. with one's hands full}}
\item \entry{n.}{\headword{ttang tupiang sod} \definition{1. long-sleeved shirt}}
\item \entry{n.}{\headword{ttangkuttangkumang} \definition{1. sideways chevron weaving pattern}}
\item \entry{n.}{\headword{ttangttang1} \definition{1. type of bird}}
\item \entry{n.}{\headword{ttape} \definition{1. type of small, flat fish}}
\item \entry{n.}{\headword{ttatt} \definition{1. jaw, chin}}
\item \entry{n.}{\headword{ttatt} \definition{1. chubby cheeks}}
\item \entry{n.}{\headword{ttatt} \definition{1. beard}}
\item \entry{n.}{\headword{ttatt} \definition{1. mandible, jawbone}}
\item \entry{n.}{\headword{ttatt} \definition{1. drum rim}}
\item \entry{n.}{\headword{ttatta2} \definition{1. lower back}}
\item \entry{n.}{\headword{ttattang} \definition{1. type of big, round yam with a white interior}}
\item \entry{n.}{\headword{ttattel} \definition{1. type of thorny vine}}
\item \entry{n.}{\headword{ttattep} \definition{1. mature leaf}}
\item \entry{n.}{\headword{ttattle} \definition{1. pain}}
\item \entry{n.}{\headword{ttattle} \definition{1. sore, in pain; sick, ill}}
\item \entry{n.}{\headword{ttäb} \definition{1. Pinon's imperial pigeon}}
\item \entry{n.}{\headword{ttäbe} \definition{1. a strong smelling plant whose bark, called ttäbe kollop, was traditionally worn around the neck to give fragrance and perfume smell. It was sometimes chewed and rubbed around the body and head to stop headache. The orignal purpose was also for protection from evil spirits.}}
\item \entry{n.}{\headword{ttäbottäbo} \definition{1. rectum}}
\item \entry{n.}{\headword{ttägäll} \definition{1. termite mound, anthill (made by termites; ants may also live inside)}}
\item \entry{n.}{\headword{ttägäll} \definition{1. mumu (oven made in the ground with fire, stones, leaves, and bark; the first mumus were made out of termite mounds)}}
\item \entry{n.}{\headword{ttägäll} \definition{1. stone (esp. one used for cooking in mumus)}}
\item \entry{n.}{\headword{ttägäll} \definition{1. money}}
\item \entry{n.}{\headword{ttäk} \definition{1. type of tree that grows on floating grass (~3 m) with white or yellow flowers, green fruit, and a big trunk used to by children to float or for carfts}}
\item \entry{n.}{\headword{ttäk} \definition{1. soft wood}}
\item \entry{n.}{\headword{ttäkäll} \definition{1. portion of yams mixed with coconut}}
\item \entry{n.}{\headword{ttäle1} \definition{1. leg}}
\item \entry{n.}{\headword{ttäle1} \definition{1. tendril}}
\item \entry{n.}{\headword{ttäle1} \definition{1. instep}}
\item \entry{n.}{\headword{ttäle1} \definition{1. heel}}
\item \entry{n.}{\headword{ttäle1} \definition{1. claw}}
\item \entry{n.}{\headword{ttäle1} \definition{1. hind leg, hind limb}}
\item \entry{n.}{\headword{ttäle1} \definition{1. tibia}}
\item \entry{n.}{\headword{ttäle1} \definition{1. hind foot (of a reptile) with four toes}}
\item \entry{n.}{\headword{ttäle1} \definition{1. leg band}}
\item \entry{n.}{\headword{ttäle2} \definition{1. type of tree}}
\item \entry{n.}{\headword{ttälläp} \definition{1. type of venomous snake}}
\item \entry{n.}{\headword{ttällma tränymägäll} \definition{1. tree type}}
\item \entry{n.}{\headword{ttämbe} \definition{1. type of big tree that grows in the bush with blue, purple, and white flowers and red fruit}}
\item \entry{n.}{\headword{ttämbe role} \definition{1. type of tree}}
\item \entry{n.}{\headword{ttän} \definition{1. type of tree that grows in the grassland (~60 m) with yellow flowers, small green fruit, and wood used for house posts}}
\item \entry{n.}{\headword{ttän maigag} \definition{1. type of bandicoot}}
\item \entry{n.}{\headword{ttänttäm} \definition{1. heat}}
\item \entry{n.}{\headword{ttängattänge} \definition{1. date}}
\item \entry{n.}{\headword{ttängäm} \definition{1. village}}
\item \entry{n.}{\headword{ttängäm} \definition{1. garden}}
\item \entry{n.}{\headword{ttängäm} \definition{1. place}}
\item \entry{n.}{\headword{ttängäm} \definition{1. small garden}}
\item \entry{n.}{\headword{ttätt käp} \definition{1. graceful honeyeater}}
\item \entry{n.}{\headword{ttätt käp} \definition{1. puff-backed honeyeater}}
\item \entry{n.}{\headword{ttättawe} \definition{1. type of tree}}
\item \entry{n.}{\headword{ttättäp} \definition{1. young leaf}}
\item \entry{n.}{\headword{ttek} \definition{1. type of tree that grows in the grassland with white flowers and sap used as glue for spears}}
\item \entry{n.}{\headword{ttette} \definition{1. rafter}}
\item \entry{n.}{\headword{ttimattima} \definition{1. limp}}
\item \entry{n.}{\headword{ttɨp} \definition{1. type of sago}}
\item \entry{n.}{\headword{ttɨp} \definition{1. type of yam with a white interior}}
\item \entry{n.}{\headword{ttoe} \definition{1. skin (of a person or animal)}}
\item \entry{n.}{\headword{ttoe} \definition{1. bark}}
\item \entry{n.}{\headword{ttoen} \definition{1. story}}
\item \entry{n.}{\headword{ttoen} \definition{1. thing}}
\item \entry{n.}{\headword{ttoen} \definition{1. way, method}}
\item \entry{n.}{\headword{ttoen} \definition{1. small story}}
\item \entry{n.}{\headword{ttoenglla} \definition{1. unrelated to one's clan or tribe}}
\item \entry{n.}{\headword{ttoep1} \definition{1. type of snake}}
\item \entry{n.}{\headword{ttoep2} \definition{1. type of tree}}
\item \entry{n.}{\headword{ttoettoe} \definition{1. blue-faced honeyeater}}
\item \entry{n.}{\headword{ttomttom} \definition{1. yam heap}}
\item \entry{n.}{\headword{ttomttom} \definition{1. yam cooked whole with skin}}
\item \entry{n.}{\headword{ttongo2} \definition{1. drum handle}}
\item \entry{n.}{\headword{ttongo iddob} \definition{1. the day after tomorrow}}
\item \entry{n.}{\headword{ttongttong} \definition{1. type of tree}}
\item \entry{n.}{\headword{ttope} \definition{1. reed}}
\item \entry{n.}{\headword{ttottoem} \definition{1. type of tree that grows inthe swamp and along creeks with a mango-like, inedible fruit that is yellow when ripe}}
\item \entry{n.}{\headword{ttowa} \definition{1. Pacific koel}}
\item \entry{n.}{\headword{ttu} \definition{1. deep}}
\item \entry{n.}{\headword{ttullong} \definition{1. large-tailed nightjar}}
\item \entry{n.}{\headword{ubony} \definition{1. type of black bee}}
\item \entry{n.}{\headword{ubrattäka} \definition{1. type of yam a red interior}}
\item \entry{n.}{\headword{ud} \definition{1. door; gate}}
\item \entry{n.}{\headword{udu} \definition{1. walking stick, cane, staff}}
\item \entry{n.}{\headword{ugeuge} \definition{1. type of tree}}
\item \entry{n.}{\headword{ugri} \definition{1. fever}}
\item \entry{n.}{\headword{ukär} \definition{1. glossy-mantled manucode}}
\item \entry{n.}{\headword{ukär} \definition{1. trumpet manucode}}
\item \entry{n.}{\headword{ulle} \definition{1. master, owner, ruler, important person}}
\item \entry{n.}{\headword{ulle} \definition{1. nonsingular form of ulle binang}}
\item \entry{n.}{\headword{ulle} \definition{1. nonsingular form of ulle}}
\item \entry{n.}{\headword{ulle kottllam} \definition{1. type of big turtle}}
\item \entry{n.}{\headword{ullegäll} \definition{1. type of tree that grows in the grassland with white flowers, black fruits, and red nuts}}
\item \entry{n.}{\headword{ume} \definition{1. mouth}}
\item \entry{n.}{\headword{ume ttäp} \definition{1. mouth}}
\item \entry{n.}{\headword{umllang} \definition{1. knowledge, knowing, awareness}}
\item \entry{n.}{\headword{umllang} \definition{1. summary}}
\item \entry{n.}{\headword{umllang bällanen ma skul} \definition{1. vocational school}}
\item \entry{n.}{\headword{umull} \definition{1. wisdom}}
\item \entry{n.}{\headword{up} \definition{1. banana}}
\item \entry{n.}{\headword{upeupe} \definition{1. type of plant with a single stem and edible, tall fruit near the base of stem}}
\item \entry{n.}{\headword{upiye} \definition{1. type of tree used to make kwib charcoal}}
\item \entry{n.}{\headword{upma} \definition{1. two friends who share a twin banana fruit}}
\item \entry{n.}{\headword{upoupoll} \definition{1. type of tree}}
\item \entry{n.}{\headword{upye} \definition{1. type of tree with white flowers and black fruit that produces a black pigment}}
\item \entry{n.}{\headword{uriar} \definition{1. type of palm with coconuts with a purple exocarp}}
\item \entry{n.}{\headword{ute} \definition{1. sore, blister; wound}}
\item \entry{n.}{\headword{ute} \definition{1. healthcare worker}}
\item \entry{n.}{\headword{ute} \definition{1. aid post}}
\item \entry{n.}{\headword{ute} \definition{1. with sores}}
\item \entry{n.}{\headword{utt1} \definition{1. conch shell}}
\item \entry{n.}{\headword{utt1} \definition{1. small conch shell}}
\item \entry{n.}{\headword{utt2} \definition{1. shoot (of a plant)}}
\item \entry{n.}{\headword{uttang ttatta} \definition{1. type of sago}}
\item \entry{n.}{\headword{uwo} \definition{1. magnificent riflebird}}
\item \entry{n.}{\headword{uwo kottllam} \definition{1. type of turtle}}
\item \entry{n.}{\headword{wa} \definition{1. penis}}
\item \entry{n.}{\headword{wa} \definition{1. sperm}}
\item \entry{n.}{\headword{wabeyawabeya} \definition{1. type of tree}}
\item \entry{n.}{\headword{wadär} \definition{1. type of grass; cane}}
\item \entry{n.}{\headword{wadär} \definition{1. bowstring}}
\item \entry{n.}{\headword{wadär} \definition{1. extra bowstring}}
\item \entry{n.}{\headword{wadär} \definition{1. type of game involving pulling cane}}
\item \entry{n.}{\headword{wadär gullem} \definition{1. Papuan python}}
\item \entry{n.}{\headword{wadär käp} \definition{1. type of big taro}}
\item \entry{n.}{\headword{waetwaet} \definition{1. type of tree}}
\item \entry{n.}{\headword{waewae} \definition{1. song sung while beating sago}}
\item \entry{n.}{\headword{waeya} \definition{1. wire}}
\item \entry{n.}{\headword{waglla} \definition{1. bullroarer}}
\item \entry{n.}{\headword{wak} \definition{1. Papuan pitta}}
\item \entry{n.}{\headword{wakata} \definition{1. type of introduced banana}}
\item \entry{n.}{\headword{walle} \definition{1. body of water}}
\item \entry{n.}{\headword{walle} \definition{1. river, stream}}
\item \entry{n.}{\headword{walle} \definition{1. dry riverbed}}
\item \entry{n.}{\headword{walle} \definition{1. creek; river source}}
\item \entry{n.}{\headword{walle} \definition{1. bank, shore, water's edge}}
\item \entry{n.}{\headword{wan pinga} \definition{1. metal fish spear}}
\item \entry{n.}{\headword{wana} \definition{1. type of introduced banana}}
\item \entry{n.}{\headword{wandana} \definition{1. grass in the garden}}
\item \entry{n.}{\headword{wanpadam} \definition{1. lap-lap}}
\item \entry{n.}{\headword{wanttawantta} \definition{1. type of game like capture the flag but with a stick planted in the middle of a ring instead of a flag}}
\item \entry{n.}{\headword{wap} \definition{1. stick}}
\item \entry{n.}{\headword{waramawarama} \definition{1. type of tree that grows in the bush with white flowers and edible yellow fruit}}
\item \entry{n.}{\headword{wariwari} \definition{1. sago shoot}}
\item \entry{n.}{\headword{waro} \definition{1. type of turtle}}
\item \entry{n.}{\headword{wasar} \definition{1. type of edible palm}}
\item \entry{n.}{\headword{waso} \definition{1. eastern cattle egret}}
\item \entry{n.}{\headword{waswes} \definition{1. political group}}
\item \entry{n.}{\headword{wawa} \definition{1. type of tree that grows in the bush and along creeks white flowers and blue fruit}}
\item \entry{n.}{\headword{wawaem} \definition{1. hiss}}
\item \entry{n.}{\headword{wawaem} \definition{1. current}}
\item \entry{n.}{\headword{wawonai} \definition{1. type of long yam with a hooked end, white interior, and hairs}}
\item \entry{n.}{\headword{waya} \definition{1. type of pronged metal spear}}
\item \entry{n.}{\headword{waya gullem} \definition{1. type of venomous snake}}
\item \entry{n.}{\headword{wayati} \definition{1. watermelon}}
\item \entry{n.}{\headword{wädɨwädɨg} \definition{1. type of tree}}
\item \entry{n.}{\headword{wädwäd} \definition{1. type of tree}}
\item \entry{n.}{\headword{wägba} \definition{1. type of tree that grows in the bush with white flowers, bark used as medicine, and strong wood used for posts; helps make dogs' noses more sensitive}}
\item \entry{n.}{\headword{wäkär} \definition{1. type of bird}}
\item \entry{n.}{\headword{wäkɨs} \definition{1. type of bird}}
\item \entry{n.}{\headword{wäl} \definition{1. main surface-level stem of a plant with rhizomes (e.g. sweet potato, lemongrass)}}
\item \entry{n.}{\headword{wälep} \definition{1. type of tree that grows in the bush with blue flowers and small blue fruit}}
\item \entry{n.}{\headword{wälsa} \definition{1. type of tree that grows in the bush}}
\item \entry{n.}{\headword{wälläng} \definition{1. backwoods, hinterland, (Australia) bush (any rural, undeveloped landscape)}}
\item \entry{n.}{\headword{wälläng ttäp} \definition{1. Papuan eagle}}
\item \entry{n.}{\headword{wällängakäbu} \definition{1. wompoo fruit dove}}
\item \entry{n.}{\headword{wällegäll} \definition{1. type of tree with fruit that are black and edible when ripe, leaves used to roll cigarettes, and roots used to treat toothache or asthma}}
\item \entry{n.}{\headword{wällwäll} \definition{1. type of tree}}
\item \entry{n.}{\headword{wän} \definition{1. boil (on skin)}}
\item \entry{n.}{\headword{wänkäm} \definition{1. belly button, navel}}
\item \entry{n.}{\headword{wänkäm} \definition{1. anus}}
\item \entry{n.}{\headword{wänkäm} \definition{1. umbilical cord}}
\item \entry{n.}{\headword{wänkäm molle} \definition{1. soft part of a shoot or sucker (e.g. of taro, banana, or sago)}}
\item \entry{n.}{\headword{wäno} \definition{1. type of tree that grows in the grassland and along creeks with white flowers, small brown fruit, and bark used on rooves}}
\item \entry{n.}{\headword{wängän} \definition{1. type of tree}}
\item \entry{n.}{\headword{wärenzbag} \definition{1. type of taro}}
\item \entry{n.}{\headword{wärpir} \definition{1. slippery mud found by the river}}
\item \entry{n.}{\headword{wätaote} \definition{1. type of large vine that grows in bush; used to tie fence posts together}}
\item \entry{n.}{\headword{wätaote} \definition{1. type of taro}}
\item \entry{n.}{\headword{wel} \definition{1. wind}}
\item \entry{n.}{\headword{wel} \definition{1. window}}
\item \entry{n.}{\headword{welwel} \definition{1. type of bird}}
\item \entry{n.}{\headword{welwele} \definition{1. dove}}
\item \entry{n.}{\headword{Wendi} \definition{1. female personal name}}
\item \entry{n.}{\headword{wer} \definition{1. type of tree with edible black fruit with one seed}}
\item \entry{n.}{\headword{wib} \definition{1. type of tree}}
\item \entry{n.}{\headword{wibell} \definition{1. type of tree}}
\item \entry{n.}{\headword{widere} \definition{1. paddle, oar}}
\item \entry{n.}{\headword{widwid} \definition{1. type of plant with big leaves}}
\item \entry{n.}{\headword{wik} \definition{1. week}}
\item \entry{n.}{\headword{wilwil} \definition{1. type of tree that grows in the bush}}
\item \entry{n.}{\headword{win} \definition{1. win}}
\item \entry{n.}{\headword{winisde} \definition{1. Wednesday}}
\item \entry{n.}{\headword{winy} \definition{1. honey}}
\item \entry{n.}{\headword{winy} \definition{1. honeycomb}}
\item \entry{n.}{\headword{wipell} \definition{1. type of tall palm that grows along the riverside}}
\item \entry{n.}{\headword{wipellgallagallab} \definition{1. chevron weaving pattern}}
\item \entry{n.}{\headword{wirog} \definition{1. type of native banana}}
\item \entry{n.}{\headword{wiswis} \definition{1. type of tree that grows in the bush with white flowers and edible orange fruit}}
\item \entry{n.}{\headword{wit} \definition{1. wheat}}
\item \entry{n.}{\headword{witara} \definition{1. type of garden}}
\item \entry{n.}{\headword{wiyasara} \definition{1. silver gull}}
\item \entry{n.}{\headword{wiyowe} \definition{1. type of large palm that grows in the bush or along creeks with flowers that start from the top and spread downwards and white fruit that hangs like coconut}}
\item \entry{n.}{\headword{wizarab} \definition{1. type of pandanus with red fruit}}
\item \entry{n.}{\headword{wɨtwɨt} \definition{1. type of sago}}
\item \entry{n.}{\headword{woboll} \definition{1. type of plant}}
\item \entry{n.}{\headword{wod} \definition{1. type of fatty fish}}
\item \entry{n.}{\headword{wod} \definition{1. type of long yam with a white interior and few hairs}}
\item \entry{n.}{\headword{wodd memba} \definition{1. ward member}}
\item \entry{n.}{\headword{woddowoddo} \definition{1. rusty pitohui}}
\item \entry{n.}{\headword{woddowoddo} \definition{1. Amboyna cuckoo-dove}}
\item \entry{n.}{\headword{wup ttämbällag} \definition{1. type of spear}}
\item \entry{n.}{\headword{yaber} \definition{1. type of tree that grows in the bush with white flowers and poisonous bark used to catch fish}}
\item \entry{n.}{\headword{yad} \definition{1. yard}}
\item \entry{n.}{\headword{yaedidib} \definition{1. type of long, narrow grass (~0.3 m)}}
\item \entry{n.}{\headword{yagäl} \definition{1. type of tree that grows in the grassland with leaves used to sand bows and spears}}
\item \entry{n.}{\headword{yal} \definition{1. yellow-billed kingfisher}}
\item \entry{n.}{\headword{Yamkong} \definition{1. name of clan}}
\item \entry{n.}{\headword{yante} \definition{1. type of large tree that grows in the grassland with white flowers and wood used for house sticks}}
\item \entry{n.}{\headword{yarte} \definition{1. type of tree with young wood used for house sticks}}
\item \entry{n.}{\headword{yaru} \definition{1. type of yam with a purple interior, hairs, and thorns}}
\item \entry{n.}{\headword{yaryem} \definition{1. type of big yam with a white interior, hairs, and thorns}}
\item \entry{n.}{\headword{yaul} \definition{1. type of long yam with a white interior and few hairs}}
\item \entry{n.}{\headword{yäbäd} \definition{1. sun}}
\item \entry{n.}{\headword{yäbäd} \definition{1. season when the dry season starts and people go camping in the bush (eleventh season; corresponds to August)}}
\item \entry{n.}{\headword{yäbäd} \definition{1. weather}}
\item \entry{n.}{\headword{yäbäd} \definition{1. dry, hot season characterized by burning grass (twelfth season; corresponds to September)}}
\item \entry{n.}{\headword{yäbäd} \definition{1. drought}}
\item \entry{n.}{\headword{yäbäd} \definition{1. really hot}}
\item \entry{n.}{\headword{yäbäd} \definition{1. solar noon (when the sun reaches its zenith)}}
\item \entry{n.}{\headword{yäbäd} \definition{1. lunch}}
\item \entry{n.}{\headword{yäbäd} \definition{1. to dawn}}
\item \entry{n.}{\headword{yäbäd källa} \definition{1. type of big taro}}
\item \entry{n.}{\headword{yäbäd ttänttämang} \definition{1. hot season when new gardens are burnt (thirteenth month; corresponds to October)}}
\item \entry{n.}{\headword{yäbäyäbäd} \definition{1. type of tree that grows in the bush with white flowers and red fruit}}
\item \entry{n.}{\headword{yäbäyäbäd} \definition{1. type of grass}}
\item \entry{n.}{\headword{yäbäyäbäl} \definition{1. type of tree}}
\item \entry{n.}{\headword{yäbik} \definition{1. sharp gardening stick}}
\item \entry{n.}{\headword{yämak} \definition{1. type of big tree found in the bush and by the river}}
\item \entry{n.}{\headword{yämän} \definition{1. type of big tuber with a reddish interior (not a yam)}}
\item \entry{n.}{\headword{yärmuyärmu} \definition{1. type of tree}}
\item \entry{n.}{\headword{yäru} \definition{1. type of small tree with thorns}}
\item \entry{n.}{\headword{yätt} \definition{1. forehead}}
\item \entry{n.}{\headword{yid} \definition{1. liquid extracted from a plant}}
\item \entry{n.}{\headword{yid} \definition{1. the seventh and final stage of sago growth during which the pith will not yield any starch}}
\item \entry{n.}{\headword{yɨb} \definition{1. type of yam}}
\item \entry{n.}{\headword{yo} \definition{1. liver}}
\item \entry{n.}{\headword{yobeg} \definition{1. type of cultivated shrub with white and yellow flowers and long leaves used to tie yam shoots to yam sticks}}
\item \entry{n.}{\headword{yogoll} \definition{1. rain}}
\item \entry{n.}{\headword{yogoll} \definition{1. black cloud}}
\item \entry{n.}{\headword{yogoll} \definition{1. dark rain cloud}}
\item \entry{n.}{\headword{yoko} \definition{1. type of cane used for building houses, bows, and canoes}}
\item \entry{n.}{\headword{yon} \definition{1. dream}}
\item \entry{n.}{\headword{yorko} \definition{1. type of large cane found in the bush}}
\item \entry{n.}{\headword{yorkoll} \definition{1. dirt}}
\item \entry{n.}{\headword{yorkoll} \definition{1. dirty}}
\item \entry{n.}{\headword{yoto} \definition{1. type of biting bee found in trees}}
\item \entry{n.}{\headword{yowa} \definition{1. vagina}}
\item \entry{n.}{\headword{yu} \definition{1. fire}}
\item \entry{n.}{\headword{yu} \definition{1. firewood}}
\item \entry{n.}{\headword{yu} \definition{1. gun, firearm}}
\item \entry{n.}{\headword{yu} \definition{1. flame}}
\item \entry{n.}{\headword{yu} \definition{1. firewood}}
\item \entry{n.}{\headword{yu} \definition{1. small fire}}
\item \entry{n.}{\headword{yu} \definition{1. charcoal}}
\item \entry{n.}{\headword{yu} \definition{1. hell}}
\item \entry{n.}{\headword{yu} \definition{1. burning wood, burnt wood}}
\item \entry{n.}{\headword{yu bäng} \definition{1. firestick (to start a fire)}}
\item \entry{n.}{\headword{yubud} \definition{1. type of tree}}
\item \entry{n.}{\headword{yuddädda} \definition{1. type of palm with branches used for armbands}}
\item \entry{n.}{\headword{yunipom} \definition{1. uniform}}
\item \entry{n.}{\headword{yure} \definition{1. type of sago}}
\item \entry{n.}{\headword{yuru} \definition{1. type of pandanus}}
\item \entry{n.}{\headword{yurwe} \definition{1. type of tree}}
\item \entry{n.}{\headword{yuwet} \definition{1. short period of time}}
\item \entry{n.}{\headword{yuwet} \definition{1. temporarily, briefly}}
\item \entry{n.}{\headword{za} \definition{1. thing}}
\item \entry{n.}{\headword{zagu} \definition{1. type of sugarcane-like plant that grows in the swamp}}
\item \entry{n.}{\headword{zarmeny} \definition{1. type of long yam with a white interior, hairs, and no thorns}}
\item \entry{n.}{\headword{zawatt} \definition{1. vagina}}
\item \entry{n.}{\headword{zazaba} \definition{1. type of bag}}
\item \entry{n.}{\headword{zaze} \definition{1. generation}}
\item \entry{n.}{\headword{zäbo} \definition{1. yellow-streaked lory}}
\item \entry{n.}{\headword{zel} \definition{1. jail}}
\item \entry{n.}{\headword{zem} \definition{1. germ}}
\item \entry{n.}{\headword{zib} \definition{1. type of big tree that grows in the bush}}
\item \entry{n.}{\headword{zib mäka} \definition{1. type of introduced banana}}
\item \entry{n.}{\headword{zire} \definition{1. barramundi}}
\item \entry{n.}{\headword{ziz} \definition{1. insect}}
\item \entry{n.}{\headword{zizag} \definition{1. owner, master, lord}}
\item \entry{n.}{\headword{zo} \definition{1. fawn-breasted bowerbird}}
\item \entry{n.}{\headword{zobo ik} \definition{1. walling (bark on house between sago walls and roof)}}
\item \entry{n.}{\headword{zogam} \definition{1. rat}}
\item \entry{n.}{\headword{zora} \definition{1. sharp stick for peeling sago before beating it}}
\end{enumerate}

\section{n. cl.}
\begin{enumerate}
\item \entry{n. cl.}{\headword{=ae3} \definition{1. distal vocative clitic}}
\item \entry{n. cl.}{\headword{=aebe} \definition{1. restrictive clitic; only}}
\item \entry{n. cl.}{\headword{=alle1} \definition{1. instrumental case clitic; with (using)}}
\item \entry{n. cl.}{\headword{=alle1} \definition{1. comitative case clitic; with (together)}}
\item \entry{n. cl.}{\headword{=alle2} \definition{1. ablative case clitic; from}}
\item \entry{n. cl.}{\headword{=att} \definition{1. ablative case clitic; from (used for location, time, source, or cause)}}
\item \entry{n. cl.}{\headword{=bakmall} \definition{1. comitative case clitic; with (only used after oba, ama, and nouns followed by =aba)}}
\item \entry{n. cl.}{\headword{=da1} \definition{1. nominative clitic (marks the subject or agent of the verb)}}
\item \entry{n. cl.}{\headword{=da1} \definition{1. accusative clitic on conjoined objects (marks the object of the verb in conjoined noun phrases)}}
\item \entry{n. cl.}{\headword{=da2} \definition{1. close possessive clitic}}
\item \entry{n. cl.}{\headword{=da2} \definition{1. close possessive kinship clitic (marks third person possession on the nominal phrase; can only attach to phrases with kinship nominal heads)}}
\item \entry{n. cl.}{\headword{=dae2} \definition{1. perlative case clitic; along, through}}
\item \entry{n. cl.}{\headword{=daebe} \definition{1. restrictive clitic; only}}
\item \entry{n. cl.}{\headword{=daebe} \definition{1. copular form of daebe (present singular form)}}
\item \entry{n. cl.}{\headword{=daebe} \definition{1. present plural form of daeben}}
\item \entry{n. cl.}{\headword{=daebe} \definition{1. past plural form of daeben}}
\item \entry{n. cl.}{\headword{=daebe} \definition{1. present dual form of daeben}}
\item \entry{n. cl.}{\headword{=daebe} \definition{1. past singular form of daeben}}
\item \entry{n. cl.}{\headword{=de} \definition{1. accusative clitic}}
\item \entry{n. cl.}{\headword{=de} \definition{1. argument focus marker}}
\item \entry{n. cl.}{\headword{e2} \definition{1. focus marker}}
\item \entry{n. cl.}{\headword{=e1} \definition{1. allative case clitic; to, into, towards}}
\item \entry{n. cl.}{\headword{=e1} \definition{1. for, to (purposive use of the allative case; also follows nonfinite verbs to express desire, intention, or the start of an action)}}
\item \entry{n. cl.}{\headword{=ingoll} \definition{1. similative clitic; like}}
\item \entry{n. cl.}{\headword{=kämall} \definition{1. comitative case clitic; with (only used after oba, ama, and nouns followed by =aba)}}
\item \entry{n. cl.}{\headword{=mae1} \definition{1. restrictive clitic; only}}
\item \entry{n. cl.}{\headword{=mae2} \definition{1. perlative case clitic; through}}
\item \entry{n. cl.}{\headword{=mattäm} \definition{1. ablative case clitic; from}}
\item \entry{n. cl.}{\headword{=me} \definition{1. locative case clitic; in, at, on}}
\item \entry{n. cl.}{\headword{=meny} \definition{1. privative clitic; without}}
\item \entry{n. cl.}{\headword{=ngänäm} \definition{1. similative case clitic; like}}
\item \entry{n. cl.}{\headword{=olle} \definition{1. allative case clitic; to, into, towards}}
\item \entry{n. cl.}{\headword{=patatt} \definition{1. animate ablative case clitic; from (a person marked with the possessive)}}
\item \entry{n. cl.}{\headword{=pate} \definition{1. animate allative case clitic; to, towards, at (a person optionally marked with the possessive)}}
\item \entry{n. cl.}{\headword{=patme} \definition{1. animate locative case clitic; at someone's place; for (a person marked with the possessive)}}
\item \entry{n. cl.}{\headword{=peyang} \definition{1. comitative clitic; with (a noun marked with the possessive)}}
\item \entry{n. cl.}{\headword{=wa} \definition{1. emphatic clitic}}
\end{enumerate}

\section{neg. ptcl.}
\begin{enumerate}
\item \entry{neg. ptcl.}{\headword{ddone} \definition{1. not}}
\item \entry{neg. ptcl.}{\headword{ddone} \definition{1. a_lot}}
\item \entry{neg. ptcl.}{\headword{malla} \definition{1. not}}
\end{enumerate}

\section{nom.}
\begin{enumerate}
\item \entry{nom.}{\headword{lla} \definition{1. male}}
\item \entry{nom.}{\headword{ttäpen} \definition{1. split}}
\end{enumerate}

\section{num.}
\begin{enumerate}
\item \entry{num.}{\headword{andred1} \definition{1. hundred}}
\item \entry{num.}{\headword{apte gabän} \definition{1. fourteen (body counting numeral)}}
\item \entry{num.}{\headword{apte kllatolma} \definition{1. seventeen (body counting numeral)}}
\item \entry{num.}{\headword{apte matta} \definition{1. twelve (body counting numeral)}}
\item \entry{num.}{\headword{apte mända} \definition{1. fifteen (body counting numeral)}}
\item \entry{num.}{\headword{apte mätkin} \definition{1. eighteen (body counting numeral)}}
\item \entry{num.}{\headword{apte ngam} \definition{1. eleven (body counting numeral)}}
\item \entry{num.}{\headword{apte tärangesa} \definition{1. nineteen (body counting numeral)}}
\item \entry{num.}{\headword{apte tupi} \definition{1. sixteen (body counting numeral)}}
\item \entry{num.}{\headword{apte ttang kum} \definition{1. thirteen (body counting numeral)}}
\item \entry{num.}{\headword{damona} \definition{1. 1296 (yam counting numeral; 6^4)}}
\item \entry{num.}{\headword{ddäll1} \definition{1. ten (lit. chest; body counting numeral)}}
\item \entry{num.}{\headword{eit} \definition{1. eight (English numeral; also general)}}
\item \entry{num.}{\headword{eiti} \definition{1. eighty}}
\item \entry{num.}{\headword{eitin} \definition{1. eighteen (English numeral)}}
\item \entry{num.}{\headword{eleben} \definition{1. eleven (English numeral)}}
\item \entry{num.}{\headword{gabän1} \definition{1. six (lit. wrist; body counting numeral)}}
\item \entry{num.}{\headword{källatolma} \definition{1. three (lit. middle finger; body counting numeral)}}
\item \entry{num.}{\headword{komlla} \definition{1. two (yam counting numeral; also general)}}
\item \entry{num.}{\headword{komlla} \definition{1. second}}
\item \entry{num.}{\headword{komlla} \definition{1. exactly two; both}}
\item \entry{num.}{\headword{komlla} \definition{1. both}}
\item \entry{num.}{\headword{komlla komlla} \definition{1. four (yam counting numeral; 2+2)}}
\item \entry{num.}{\headword{komlla komlla} \definition{1. exactly four (2+2)}}
\item \entry{num.}{\headword{komlla komlla ttongo dduma} \definition{1. five (yam counting numeral; 2+2+1)}}
\item \entry{num.}{\headword{komlla putt} \definition{1. twelve (yam counting numeral; 2×6)}}
\item \entry{num.}{\headword{kumuddäga} \definition{1. three (yam counting numeral; also general)}}
\item \entry{num.}{\headword{kumuddäga} \definition{1. third}}
\item \entry{num.}{\headword{kumuddäga} \definition{1. exactly three}}
\item \entry{num.}{\headword{kumuddägakumuddäga} \definition{1. six (3+3)}}
\item \entry{num.}{\headword{matta} \definition{1. eight (lit. shoulder; body counting numeral)}}
\item \entry{num.}{\headword{mända} \definition{1. five (lit. thumb; body counting numeral)}}
\item \entry{num.}{\headword{mätkin} \definition{1. two (lit. ring finger; body counting numeral)}}
\item \entry{num.}{\headword{naen} \definition{1. nine (English numeral; also general)}}
\item \entry{num.}{\headword{naenti} \definition{1. ninety}}
\item \entry{num.}{\headword{naentin} \definition{1. nineteen (English numeral)}}
\item \entry{num.}{\headword{ngam} \definition{1. nine (lit. breast; body counting numeral)}}
\item \entry{num.}{\headword{paeb} \definition{1. five (English numeral)}}
\item \entry{num.}{\headword{pärta} \definition{1. thirty-six (yam counting numeral; 6^2)}}
\item \entry{num.}{\headword{pipti} \definition{1. fifty}}
\item \entry{num.}{\headword{piptin} \definition{1. fifteen (English numeral)}}
\item \entry{num.}{\headword{po2} \definition{1. four (English numeral; also general)}}
\item \entry{num.}{\headword{poti} \definition{1. forty}}
\item \entry{num.}{\headword{potin} \definition{1. fourteen (English numeral)}}
\item \entry{num.}{\headword{putt} \definition{1. six (yam counting numeral; 6^1)}}
\item \entry{num.}{\headword{seben} \definition{1. seven (English numeral; also general)}}
\item \entry{num.}{\headword{sebenti} \definition{1. seventy}}
\item \entry{num.}{\headword{sebentin} \definition{1. seventeen (English numeral)}}
\item \entry{num.}{\headword{siks} \definition{1. six (English numeral; also general)}}
\item \entry{num.}{\headword{siksti} \definition{1. sixty}}
\item \entry{num.}{\headword{sikstin} \definition{1. sixteen (English numeral)}}
\item \entry{num.}{\headword{taosen} \definition{1. thousand}}
\item \entry{num.}{\headword{taromba} \definition{1. 216 (yam counting numeral; 6^3)}}
\item \entry{num.}{\headword{tärangesa} \definition{1. one (lit. pinky; body counting numeral)}}
\item \entry{num.}{\headword{ten} \definition{1. ten (English numeral, also general)}}
\item \entry{num.}{\headword{teti} \definition{1. thirty}}
\item \entry{num.}{\headword{tetin} \definition{1. thirteen (English numeral)}}
\item \entry{num.}{\headword{tri} \definition{1. three (English numeral)}}
\item \entry{num.}{\headword{tu} \definition{1. two (English numeral)}}
\item \entry{num.}{\headword{tupi2} \definition{1. four (lit. pointer finger; body counting numeral)}}
\item \entry{num.}{\headword{twelb} \definition{1. twelve (English numeral)}}
\item \entry{num.}{\headword{twenti} \definition{1. twenty}}
\item \entry{num.}{\headword{ttang} \definition{1. seven (lit. elbow; body counting numeral)}}
\item \entry{num.}{\headword{ttang} \definition{1. unit of five (one hand)}}
\item \entry{num.}{\headword{ttongo1} \definition{1. one}}
\item \entry{num.}{\headword{wan} \definition{1. one (English numeral)}}
\item \entry{num.}{\headword{waramakae} \definition{1. 7776 (yam counting numeral; 6^5)}}
\end{enumerate}

\section{ord. num.}
\begin{enumerate}
\item \entry{ord. num.}{\headword{ngattong} \definition{1. first}}
\item \entry{ord. num.}{\headword{pes} \definition{1. first}}
\end{enumerate}

\section{p.}
\begin{enumerate}
\item \entry{p.}{\headword{mer ag} \definition{1. good morning}}
\item \entry{p.}{\headword{mer awi} \definition{1. good evening}}
\item \entry{p.}{\headword{mer iddob} \definition{1. good night}}
\item \entry{p.}{\headword{mer toto} \definition{1. good afternoon}}
\end{enumerate}

\section{pers. pron.}
\begin{enumerate}
\item \entry{pers. pron.}{\headword{beyawa} \definition{1. he, she (emphatic third person singular pronoun)}}
\item \entry{pers. pron.}{\headword{beyawa} \definition{1. resctrictive form of beyawa}}
\item \entry{pers. pron.}{\headword{beyawa} \definition{1. copular form of beyawaebe}}
\item \entry{pers. pron.}{\headword{beyawa} \definition{1. copular form of beyawa (present form)}}
\item \entry{pers. pron.}{\headword{beyawa} \definition{1. past form of beyawaenen}}
\item \entry{pers. pron.}{\headword{beyawa} \definition{1. present copular form of beyawa}}
\item \entry{pers. pron.}{\headword{bibi} \definition{1. you all, you (second person nonsingular pronoun, nominative form)}}
\item \entry{pers. pron.}{\headword{bibi} \definition{1. accusative form of bibi}}
\item \entry{pers. pron.}{\headword{bibi} \definition{1. dative form of bibi}}
\item \entry{pers. pron.}{\headword{bibi} \definition{1. possessive form of bibi}}
\item \entry{pers. pron.}{\headword{bibi} \definition{1. ablative-possessive form of bibi}}
\item \entry{pers. pron.}{\headword{bogo} \definition{1. he, she (third person singular animate pronoun, nominative form)}}
\item \entry{pers. pron.}{\headword{bogo} \definition{1. additive form of bogo}}
\item \entry{pers. pron.}{\headword{bogo} \definition{1. instrumental-comitative form of bogo}}
\item \entry{pers. pron.}{\headword{bogo} \definition{1. ablative-possessive form of bogo}}
\item \entry{pers. pron.}{\headword{bogo} \definition{1. clitic form of obene}}
\item \entry{pers. pron.}{\headword{bogo} \definition{1. dative form of bogo}}
\item \entry{pers. pron.}{\headword{bogo} \definition{1. clitic form of oblle}}
\item \entry{pers. pron.}{\headword{bogo} \definition{1. possessive form of bogo}}
\item \entry{pers. pron.}{\headword{bogo} \definition{1. oneself, himself, herself (reflexive form of bogo)}}
\item \entry{pers. pron.}{\headword{bogo} \definition{1. clitic form of obo}}
\item \entry{pers. pron.}{\headword{bogo} \definition{1. oneself, himself, herself (reflexive form of bogo)}}
\item \entry{pers. pron.}{\headword{bogo} \definition{1. accusative form of bogo}}
\item \entry{pers. pron.}{\headword{bogo} \definition{1. clitic form of obom}}
\item \entry{pers. pron.}{\headword{bongo} \definition{1. you (second person singular pronoun, nominative form)}}
\item \entry{pers. pron.}{\headword{bongo} \definition{1. additive form of bongo}}
\item \entry{pers. pron.}{\headword{bongo} \definition{1. additive form of bongo}}
\item \entry{pers. pron.}{\headword{bongo} \definition{1. dative form of bongo}}
\item \entry{pers. pron.}{\headword{bongo} \definition{1. accusative form of bongo}}
\item \entry{pers. pron.}{\headword{bongo} \definition{1. possessive form of bongo}}
\item \entry{pers. pron.}{\headword{bongo} \definition{1. yourself (reflexive form of bongo)}}
\item \entry{pers. pron.}{\headword{bongo} \definition{1. ablative-possessive form of bongo}}
\item \entry{pers. pron.}{\headword{da3} \definition{1. he, she, it (third person singular animate/inanimate pronoun, nominative only)}}
\item \entry{pers. pron.}{\headword{ibi1} \definition{1. we (first person nonsingular inclusive pronoun, nominative form)}}
\item \entry{pers. pron.}{\headword{ibi1} \definition{1. genitive form of ibi}}
\item \entry{pers. pron.}{\headword{ibi1} \definition{1. accusative form of ibi}}
\item \entry{pers. pron.}{\headword{ibi1} \definition{1. dative form of ibi}}
\item \entry{pers. pron.}{\headword{ngämangäma} \definition{1. 1nsg.excl.refl}}
\item \entry{pers. pron.}{\headword{ngämi} \definition{1. we (first person nonsingular exclusive pronoun, nominative form)}}
\item \entry{pers. pron.}{\headword{ngämi} \definition{1. possessive form of ngämi}}
\item \entry{pers. pron.}{\headword{ngämi} \definition{1. ablative-possessive form of ngämi}}
\item \entry{pers. pron.}{\headword{ngämi} \definition{1. accusative form of ngämi}}
\item \entry{pers. pron.}{\headword{ngämi} \definition{1. dative form of ngämi}}
\item \entry{pers. pron.}{\headword{ngäna1} \definition{1. I (first person singular pronoun, nominative form)}}
\item \entry{pers. pron.}{\headword{ngäna1} \definition{1. restrictive copular form of ngäna (present form)}}
\item \entry{pers. pron.}{\headword{ngäna1} \definition{1. copular form of ngäna (present form)}}
\item \entry{pers. pron.}{\headword{ngäna1} \definition{1. ablative-possessive form of ngäna}}
\item \entry{pers. pron.}{\headword{ngäna1} \definition{1. dative form of ngäna}}
\item \entry{pers. pron.}{\headword{ngäna1} \definition{1. possessive form of ngäna}}
\item \entry{pers. pron.}{\headword{ngäna1} \definition{1. emphatic form of ngäna}}
\item \entry{pers. pron.}{\headword{ngäna1} \definition{1. accusative form of ngäna}}
\item \entry{pers. pron.}{\headword{ttongo1} \definition{1. someone; anyone}}
\item \entry{pers. pron.}{\headword{ttongo1} \definition{1. one (yam counting numeral; also general)}}
\item \entry{pers. pron.}{\headword{ubi} \definition{1. they (third person nonsingular pronoun, nominative form)}}
\item \entry{pers. pron.}{\headword{ubi} \definition{1. clitic form of ubi}}
\item \entry{pers. pron.}{\headword{ubi} \definition{1. ablative-possessive form of ubi}}
\item \entry{pers. pron.}{\headword{ubi} \definition{1. clitic form of obaene}}
\item \entry{pers. pron.}{\headword{ubi} \definition{1. possessive form of ubi}}
\item \entry{pers. pron.}{\headword{ubi} \definition{1. themselves (reflexive form of ubi)}}
\item \entry{pers. pron.}{\headword{ubi} \definition{1. clitic form of oba}}
\item \entry{pers. pron.}{\headword{ubi} \definition{1. each other (reciprocal pronoun)}}
\item \entry{pers. pron.}{\headword{ubi} \definition{1. themselves (reflexive form of ubi)}}
\item \entry{pers. pron.}{\headword{ubi} \definition{1. accusative form of ubi}}
\item \entry{pers. pron.}{\headword{ubi} \definition{1. clitic form of ubim}}
\item \entry{pers. pron.}{\headword{ubi} \definition{1. dative form of ubi}}
\item \entry{pers. pron.}{\headword{ubi} \definition{1. clitic form of ubira}}
\item \entry{pers. pron.}{\headword{zaga} \definition{1. self (forms reflexive pronouns)}}
\end{enumerate}

\section{pn.}
\begin{enumerate}
\item \entry{pn.}{\headword{Abagigima} \definition{1. personal name}}
\item \entry{pn.}{\headword{Abam} \definition{1. Abam (Wipi-speaking village in Oriomo-Bituri LLG; GPS: -8.926607 143.190246)}}
\item \entry{pn.}{\headword{Abeam} \definition{1. male personal name}}
\item \entry{pn.}{\headword{Abere} \definition{1. female personal name}}
\item \entry{pn.}{\headword{Aberegerem} \definition{1. Aberagerema (in Kiwai Rural LLG)}}
\item \entry{pn.}{\headword{Abigail} \definition{1. female personal name}}
\item \entry{pn.}{\headword{Abom} \definition{1. Abom (toponym)}}
\item \entry{pn.}{\headword{Adam} \definition{1. PN}}
\item \entry{pn.}{\headword{Adasha} \definition{1. male personal name}}
\item \entry{pn.}{\headword{Adi} \definition{1. God}}
\item \entry{pn.}{\headword{Adu} \definition{1. female personal name}}
\item \entry{pn.}{\headword{Agan} \definition{1. Agan (toponym)}}
\item \entry{pn.}{\headword{Agäb} \definition{1. Agob language (Pahoturi River language)}}
\item \entry{pn.}{\headword{Ailin} \definition{1. female personal name}}
\item \entry{pn.}{\headword{Ainor} \definition{1. Ainor (toponym)}}
\item \entry{pn.}{\headword{Ainor} \definition{1. dog name}}
\item \entry{pn.}{\headword{Aituru} \definition{1. female personal name}}
\item \entry{pn.}{\headword{Aketa} \definition{1. Aketa (in Gogodala Rural LLG)}}
\item \entry{pn.}{\headword{Al} \definition{1. female personal name}}
\item \entry{pn.}{\headword{Alex} \definition{1. male personal name}}
\item \entry{pn.}{\headword{Ali} \definition{1. place name}}
\item \entry{pn.}{\headword{Alopa} \definition{1. female personal name}}
\item \entry{pn.}{\headword{Alphones} \definition{1. male personal name}}
\item \entry{pn.}{\headword{Allambun} \definition{1. Allambun (camping place)}}
\item \entry{pn.}{\headword{Alläpma} \definition{1. Alläpma (toponym)}}
\item \entry{pn.}{\headword{Amadu} \definition{1. male personal name}}
\item \entry{pn.}{\headword{Amanda} \definition{1. female personal name}}
\item \entry{pn.}{\headword{Amerika} \definition{1. United States, America}}
\item \entry{pn.}{\headword{Amne kona} \definition{1. Central corner (in Limol)}}
\item \entry{pn.}{\headword{Ana} \definition{1. female personal name}}
\item \entry{pn.}{\headword{Andrew} \definition{1. male personal name}}
\item \entry{pn.}{\headword{Anna} \definition{1. female personal name}}
\item \entry{pn.}{\headword{Ansel} \definition{1. male personal name}}
\item \entry{pn.}{\headword{Anton} \definition{1. male personal name}}
\item \entry{pn.}{\headword{Apang} \definition{1. language name}}
\item \entry{pn.}{\headword{Apdo} \definition{1. female personal name}}
\item \entry{pn.}{\headword{Apodo} \definition{1. female personal name}}
\item \entry{pn.}{\headword{Arägapetkae} \definition{1. Arägapetkae (toponym)}}
\item \entry{pn.}{\headword{Arua} \definition{1. male personal name}}
\item \entry{pn.}{\headword{Arupi} \definition{1. Arupi (toponym)}}
\item \entry{pn.}{\headword{Aruwa} \definition{1. male personal name}}
\item \entry{pn.}{\headword{Asika} \definition{1. female personal name}}
\item \entry{pn.}{\headword{Australia} \definition{1. Australia}}
\item \entry{pn.}{\headword{Awaba} \definition{1. Awaba (toponym)}}
\item \entry{pn.}{\headword{Awayang} \definition{1. male personal name}}
\item \entry{pn.}{\headword{Awi} \definition{1. Awi (toponym)}}
\item \entry{pn.}{\headword{Azaya} \definition{1. male personal name}}
\item \entry{pn.}{\headword{Babaze} \definition{1. male personal name}}
\item \entry{pn.}{\headword{Bablela} \definition{1. male personal name}}
\item \entry{pn.}{\headword{Babra} \definition{1. female personal name}}
\item \entry{pn.}{\headword{Babu} \definition{1. male personal name}}
\item \entry{pn.}{\headword{Badu} \definition{1. male personal name}}
\item \entry{pn.}{\headword{Baewa} \definition{1. male personal name}}
\item \entry{pn.}{\headword{Baiduwa} \definition{1. Baiduwa (toponym)}}
\item \entry{pn.}{\headword{Baim} \definition{1. Baim (toponym)}}
\item \entry{pn.}{\headword{Balimo} \definition{1. Balimo (toponym)}}
\item \entry{pn.}{\headword{Bana} \definition{1. female personal name}}
\item \entry{pn.}{\headword{Baogab} \definition{1. Baogab (an island in Karama swamp used for camping; has coconuts and bananas)}}
\item \entry{pn.}{\headword{Barekam} \definition{1. male personal name}}
\item \entry{pn.}{\headword{Basido} \definition{1. Basido (toponym)}}
\item \entry{pn.}{\headword{Basido kona} \definition{1. Pastor's corner (in Limol)}}
\item \entry{pn.}{\headword{Bati} \definition{1. female personal name}}
\item \entry{pn.}{\headword{Bädämalloang} \definition{1. Bädämalloang (sago place along the road from Limol to the canoe place where a large tree has fallen halfway over the path (AZ95))}}
\item \entry{pn.}{\headword{Bäglle} \definition{1. male personal name}}
\item \entry{pn.}{\headword{Bämäg} \definition{1. Bämäg (toponym)}}
\item \entry{pn.}{\headword{Bebelin} \definition{1. female personal name}}
\item \entry{pn.}{\headword{Ben} \definition{1. male personal name}}
\item \entry{pn.}{\headword{Benaeya} \definition{1. male personal name}}
\item \entry{pn.}{\headword{Bensi} \definition{1. female personal name}}
\item \entry{pn.}{\headword{Benson} \definition{1. male personal name}}
\item \entry{pn.}{\headword{Benta} \definition{1. female personal name}}
\item \entry{pn.}{\headword{Ber} \definition{1. Ber (toponym)}}
\item \entry{pn.}{\headword{Beradi} \definition{1. Beradi (toponym)}}
\item \entry{pn.}{\headword{Bes} \definition{1. female personal name}}
\item \entry{pn.}{\headword{Bessie} \definition{1. female personal name}}
\item \entry{pn.}{\headword{Betliem} \definition{1. Bethlehem}}
\item \entry{pn.}{\headword{Bewag} \definition{1. male personal name}}
\item \entry{pn.}{\headword{Bibiae} \definition{1. female personal name}}
\item \entry{pn.}{\headword{Bidog} \definition{1. male personal name}}
\item \entry{pn.}{\headword{Big} \definition{1. Big (toponym)}}
\item \entry{pn.}{\headword{Bigag} \definition{1. male personal name}}
\item \entry{pn.}{\headword{Bigiya} \definition{1. female personal name}}
\item \entry{pn.}{\headword{Bigjay} \definition{1. male personal name}}
\item \entry{pn.}{\headword{Biks} \definition{1. personal name}}
\item \entry{pn.}{\headword{Biku} \definition{1. male personal name}}
\item \entry{pn.}{\headword{Bimadbn} \definition{1. Bimadbn (toponym)}}
\item \entry{pn.}{\headword{Bine} \definition{1. Bine language (spoken to the east)}}
\item \entry{pn.}{\headword{Binyomoll Källäm} \definition{1. Binyomoll Pond (on the road to Kinkin)}}
\item \entry{pn.}{\headword{Bipi} \definition{1. Bipi (toponym)}}
\item \entry{pn.}{\headword{Birigi} \definition{1. male personal name}}
\item \entry{pn.}{\headword{Bisenmo} \definition{1. Bisenmo (toponym)}}
\item \entry{pn.}{\headword{Bisuaka} \definition{1. Bisuaka (Bituri-speaking village in Oriomo-Bituri Rural LLG; has a primary school but no aid post)}}
\item \entry{pn.}{\headword{Bitur} \definition{1. Bitur language (spoken to the north)}}
\item \entry{pn.}{\headword{Biyewolatt} \definition{1. Biyewolatt (toponym)}}
\item \entry{pn.}{\headword{Bobe} \definition{1. Bobe (toponym)}}
\item \entry{pn.}{\headword{Bobngätt} \definition{1. Bobngätt (toponym)}}
\item \entry{pn.}{\headword{Bobzag} \definition{1. personal name}}
\item \entry{pn.}{\headword{Bobzag} \definition{1. dog name}}
\item \entry{pn.}{\headword{Bobze} \definition{1. Bobze (toponym)}}
\item \entry{pn.}{\headword{Bodog} \definition{1. male personal name}}
\item \entry{pn.}{\headword{Bok} \definition{1. Buk (Taeme-speaking village in Morehead Rural LLG; from Limol, one must pass through Kinkin and Kondobol)}}
\item \entry{pn.}{\headword{Bollga} \definition{1. female personal name}}
\item \entry{pn.}{\headword{Bolloll} \definition{1. Bolloll (toponym, on the southward road to Malam)}}
\item \entry{pn.}{\headword{Bomso} \definition{1. male personal name}}
\item \entry{pn.}{\headword{Bonibi} \definition{1. female personal name}}
\item \entry{pn.}{\headword{Bong} \definition{1. male personal name}}
\item \entry{pn.}{\headword{Bonybony} \definition{1. Bonybony (camping, garden, and sago place (AX94))}}
\item \entry{pn.}{\headword{Boze} \definition{1. Boze (Agob- and Bine-speaking village in Oriomo-Bituri Rural LLG; from Limol, one must pass through Malam, Kurunti, and Kibuli)}}
\item \entry{pn.}{\headword{Bozorob} \definition{1. Bozorob (camping place on the road to Kurunti; from Limol, one must pass through Malam)}}
\item \entry{pn.}{\headword{Breton} \definition{1. male personal name}}
\item \entry{pn.}{\headword{Buib} \definition{1. Buib (toponym)}}
\item \entry{pn.}{\headword{Buiddobuiddog} \definition{1. Buiddobuiddog (sago place and creek; also a road (Top L AZ96))}}
\item \entry{pn.}{\headword{Bundae} \definition{1. male personal name}}
\item \entry{pn.}{\headword{Bunkuttangmälla} \definition{1. female personal name}}
\item \entry{pn.}{\headword{Buyubun} \definition{1. Buyubun (toponym)}}
\item \entry{pn.}{\headword{Buzi} \definition{1. Buzi (in Kiwai Rural LLG)}}
\item \entry{pn.}{\headword{Dabe} \definition{1. Dabe (toponym)}}
\item \entry{pn.}{\headword{Dabi} \definition{1. female personal name}}
\item \entry{pn.}{\headword{Dadi} \definition{1. male personal name}}
\item \entry{pn.}{\headword{Daena} \definition{1. female personal name}}
\item \entry{pn.}{\headword{Daeyagmälla} \definition{1. female personal name}}
\item \entry{pn.}{\headword{Daeyna} \definition{1. female personal name}}
\item \entry{pn.}{\headword{Daiba} \definition{1. garden and sago place of Jerry Dareda (along the road to Bisuaka)}}
\item \entry{pn.}{\headword{Daniel} \definition{1. male personal name}}
\item \entry{pn.}{\headword{Danipa} \definition{1. male personal name}}
\item \entry{pn.}{\headword{Dara} \definition{1. male personal name}}
\item \entry{pn.}{\headword{Dareda} \definition{1. male personal name}}
\item \entry{pn.}{\headword{Darren} \definition{1. male personal name}}
\item \entry{pn.}{\headword{Daru} \definition{1. Daru (capital city of Western Province, located on Daru Island)}}
\item \entry{pn.}{\headword{Daru} \definition{1. Daru Island}}
\item \entry{pn.}{\headword{Därall} \definition{1. Därall (toponym)}}
\item \entry{pn.}{\headword{Därängge} \definition{1. male personal name}}
\item \entry{pn.}{\headword{Deboa} \definition{1. male personal name}}
\item \entry{pn.}{\headword{Deibid} \definition{1. male personal name}}
\item \entry{pn.}{\headword{Delema} \definition{1. Delema (toponym)}}
\item \entry{pn.}{\headword{Derideri} \definition{1. Derideri (Nambo-speaking village in Morehead Rural LLG; east of Morehead)}}
\item \entry{pn.}{\headword{Dettall} \definition{1. Dettall (toponym)}}
\item \entry{pn.}{\headword{Dewara} \definition{1. Dewara (Were/Kiunum-speaking village in Gogodala Rural LLG, along the Fly River; from Limol, one must pass through Upiara and Kondobol)}}
\item \entry{pn.}{\headword{Diandra} \definition{1. female personal name}}
\item \entry{pn.}{\headword{Dibllag} \definition{1. personal name}}
\item \entry{pn.}{\headword{Dibllagmälla} \definition{1. female personal name}}
\item \entry{pn.}{\headword{Dibor} \definition{1. female personal name}}
\item \entry{pn.}{\headword{Didroe} \definition{1. male personal name}}
\item \entry{pn.}{\headword{Dieb} \definition{1. male personal name}}
\item \entry{pn.}{\headword{Diendra} \definition{1. female personal name}}
\item \entry{pn.}{\headword{Digabo Källäm} \definition{1. Digabo Pond (canoe and fishing place in Taolang; road to Taolang is in AX95)}}
\item \entry{pn.}{\headword{Dikae} \definition{1. male personal name}}
\item \entry{pn.}{\headword{Dikiboe} \definition{1. personal name}}
\item \entry{pn.}{\headword{Dikullowang} \definition{1. Dikullowang (small island and hunting place in Taolang)}}
\item \entry{pn.}{\headword{Dimgi} \definition{1. Dimgi (toponym)}}
\item \entry{pn.}{\headword{Dimiri} \definition{1. Dimiri/Demeri (Idi-speaking village in Morehead Rural LLG; from Limol, one must pass through Kuiwang)}}
\item \entry{pn.}{\headword{Dimisisi} \definition{1. Dimisisi (Idi-speaking village in Morehead Rural LLG; from Limol, one must pass through Kinkin and Bok)}}
\item \entry{pn.}{\headword{Dimoe} \definition{1. female personal name}}
\item \entry{pn.}{\headword{Dimson} \definition{1. male personal name}}
\item \entry{pn.}{\headword{Dipa} \definition{1. male personal name}}
\item \entry{pn.}{\headword{Diwa} \definition{1. male personal name}}
\item \entry{pn.}{\headword{Dobola} \definition{1. male personal name}}
\item \entry{pn.}{\headword{Dokoe} \definition{1. male personal name}}
\item \entry{pn.}{\headword{Doli} \definition{1. male personal name}}
\item \entry{pn.}{\headword{Donae} \definition{1. female personal name}}
\item \entry{pn.}{\headword{Donsi} \definition{1. female personal name}}
\item \entry{pn.}{\headword{Dore} \definition{1. female personal name}}
\item \entry{pn.}{\headword{Dorin} \definition{1. female personal name}}
\item \entry{pn.}{\headword{Doumori} \definition{1. Doumori (in Kiwai Rural LLG)}}
\item \entry{pn.}{\headword{Dowabunang} \definition{1. camping, sago, hunting place, and garden of Kaoga Dobola (on the road to Kinkin AZ94)}}
\item \entry{pn.}{\headword{Dowan} \definition{1. name of a mountain}}
\item \entry{pn.}{\headword{Duaba} \definition{1. Duaba (Gogodala-speaking village in Gogodala Rural LLG)}}
\item \entry{pn.}{\headword{Dubolläplläp} \definition{1. Dubolläplläp (big hill and camping place on the road to Karama swamp; had houses on top in 2015)}}
\item \entry{pn.}{\headword{Dugal} \definition{1. male personal name}}
\item \entry{pn.}{\headword{Dugi} \definition{1. male personal name}}
\item \entry{pn.}{\headword{Duiya} \definition{1. male personal name}}
\item \entry{pn.}{\headword{Duks} \definition{1. male personal name}}
\item \entry{pn.}{\headword{Dukumiang} \definition{1. Dukumiang (fishing, sago, hunting place, and garden; camping place of Bewag Bewag; R side of AY96, take southward road; northward road goes to Kapal)}}
\item \entry{pn.}{\headword{Dum} \definition{1. Dum (big hill on the road to Malam; Bamboo and Zarma creek)}}
\item \entry{pn.}{\headword{Dum Tutu} \definition{1. Dum Mountain}}
\item \entry{pn.}{\headword{Dumoll} \definition{1. Dumoll (Taolang side garden and camping place of Gidu Jerry, Kols Baewa, and Wareya Giniya)}}
\item \entry{pn.}{\headword{Duwaba} \definition{1. Duaba (in Gogodala Rural LLG)}}
\item \entry{pn.}{\headword{Ddämir} \definition{1. Ddamir (toponym)}}
\item \entry{pn.}{\headword{Ddele} \definition{1. Ddele (toponym)}}
\item \entry{pn.}{\headword{Ddele} \definition{1. Ende dialect spoken in Ddele}}
\item \entry{pn.}{\headword{Ddelema} \definition{1. male personal name}}
\item \entry{pn.}{\headword{Edeb} \definition{1. garden, camping, and hunting place of Tewa (on the other side of Karama swamp)}}
\item \entry{pn.}{\headword{Edeb} \definition{1. personal name}}
\item \entry{pn.}{\headword{Edna} \definition{1. female personal name}}
\item \entry{pn.}{\headword{Edward} \definition{1. male personal name}}
\item \entry{pn.}{\headword{Egapo} \definition{1. camping place and large community garden (in Limol; road to garden is visible on L of AZ96)}}
\item \entry{pn.}{\headword{Elisa} \definition{1. female personal name}}
\item \entry{pn.}{\headword{Elizabeth} \definition{1. female personal name}}
\item \entry{pn.}{\headword{Elsie} \definition{1. female personal name}}
\item \entry{pn.}{\headword{Emi} \definition{1. female personal name}}
\item \entry{pn.}{\headword{Ende} \definition{1. Ende language (Pahoturi River language spoken in Limol, Malam, and Kinkin)}}
\item \entry{pn.}{\headword{Endo} \definition{1. female personal name}}
\item \entry{pn.}{\headword{Enza} \definition{1. Enza (Bitur-speaking village in Oriomo-Bituri Rural LLG; from Limol, one must pass through Bisuaka)}}
\item \entry{pn.}{\headword{Erabal} \definition{1. female personal name}}
\item \entry{pn.}{\headword{Eramang} \definition{1. Eramang (the main canoe place in Limol; for camping, fishing, hunting)}}
\item \entry{pn.}{\headword{Erga} \definition{1. female personal name}}
\item \entry{pn.}{\headword{Eric} \definition{1. male personal name}}
\item \entry{pn.}{\headword{Erme} \definition{1. male personal name}}
\item \entry{pn.}{\headword{Erodias} \definition{1. PN}}
\item \entry{pn.}{\headword{Ese} \definition{1. male personal name}}
\item \entry{pn.}{\headword{Essie} \definition{1. female personal name}}
\item \entry{pn.}{\headword{Esta} \definition{1. female personal name}}
\item \entry{pn.}{\headword{Ezra} \definition{1. male personal name}}
\item \entry{pn.}{\headword{Evelyn} \definition{1. female personal name}}
\item \entry{pn.}{\headword{Gabadi} \definition{1. female personal name}}
\item \entry{pn.}{\headword{Gaem} \definition{1. female personal name}}
\item \entry{pn.}{\headword{Gaima} \definition{1. Gaima (toponym)}}
\item \entry{pn.}{\headword{Galibma} \definition{1. sacred place of Biku (Madura) Kangge (on the road to Bisuaka, after Limol ma kuddäll; bush area, creek)}}
\item \entry{pn.}{\headword{Galo} \definition{1. male personal name}}
\item \entry{pn.}{\headword{Galwe} \definition{1. male personal name}}
\item \entry{pn.}{\headword{Gamaewe} \definition{1. Gamaewe (toponym)}}
\item \entry{pn.}{\headword{Gao} \definition{1. male personal name}}
\item \entry{pn.}{\headword{Garayi} \definition{1. male personal name}}
\item \entry{pn.}{\headword{Garaz} \definition{1. male personal name}}
\item \entry{pn.}{\headword{Gäbag} \definition{1. male personal name}}
\item \entry{pn.}{\headword{Gälabi} \definition{1. Gälabi (Wipi-speaking village in Oriomo-Bituri Rural LLG; from Limol, one must pass through Wipim)}}
\item \entry{pn.}{\headword{Gäläb} \definition{1. Gälab (sago place and Geoff Rowak's camping place in Limol)}}
\item \entry{pn.}{\headword{Geleli} \definition{1. Galilee}}
\item \entry{pn.}{\headword{Gene} \definition{1. male personal name}}
\item \entry{pn.}{\headword{Georgina} \definition{1. female personal name}}
\item \entry{pn.}{\headword{Geoff} \definition{1. male personal name}}
\item \entry{pn.}{\headword{Geser} \definition{1. male personal name}}
\item \entry{pn.}{\headword{Gibson} \definition{1. male personal name}}
\item \entry{pn.}{\headword{Gidra} \definition{1. (possibly derogatory) Wipi language}}
\item \entry{pn.}{\headword{Gidu} \definition{1. male personal name}}
\item \entry{pn.}{\headword{Gilbet} \definition{1. male personal name}}
\item \entry{pn.}{\headword{Gimaga} \definition{1. personal name}}
\item \entry{pn.}{\headword{Ginarang} \definition{1. male personal name (the original Ende man [see SE_PN022])}}
\item \entry{pn.}{\headword{Gini} \definition{1. female personal name}}
\item \entry{pn.}{\headword{Ginia} \definition{1. male personal name}}
\item \entry{pn.}{\headword{Giniya} \definition{1. male personal name}}
\item \entry{pn.}{\headword{Giwo} \definition{1. male personal name}}
\item \entry{pn.}{\headword{Gladis} \definition{1. female personal name}}
\item \entry{pn.}{\headword{Glan} \definition{1. female personal name}}
\item \entry{pn.}{\headword{Glendis} \definition{1. female personal name}}
\item \entry{pn.}{\headword{Gloria} \definition{1. female personal name}}
\item \entry{pn.}{\headword{Gllu} \definition{1. Gllu (toponym)}}
\item \entry{pn.}{\headword{Goballwang} \definition{1. Goballwang (camping, hunting, and garden place in Karama swamp between Upiara and Limol)}}
\item \entry{pn.}{\headword{Godd} \definition{1. God}}
\item \entry{pn.}{\headword{Goeg wälläng} \definition{1. garden place in the bush on the west side of the road to Taolang}}
\item \entry{pn.}{\headword{Goge} \definition{1. male personal name}}
\item \entry{pn.}{\headword{Gogodala} \definition{1. Gogodala language}}
\item \entry{pn.}{\headword{Gonzer} \definition{1. female personal name}}
\item \entry{pn.}{\headword{Grace} \definition{1. female personal name}}
\item \entry{pn.}{\headword{Guar} \definition{1. male personal name}}
\item \entry{pn.}{\headword{Gubam} \definition{1. Gubam (in Morehead Rural LLG)}}
\item \entry{pn.}{\headword{Gugu} \definition{1. Gugu (toponym)}}
\item \entry{pn.}{\headword{Gugu Gel} \definition{1. female personal name}}
\item \entry{pn.}{\headword{Guim} \definition{1. personal name}}
\item \entry{pn.}{\headword{Gullbe Bikme Auma} \definition{1. sacred place of Dareda (near Karama swamp)}}
\item \entry{pn.}{\headword{Gullbe bo llädayatt} \definition{1. sacred place of Dareda (large hill)}}
\item \entry{pn.}{\headword{Gullbe bo makollamatt} \definition{1. Dareda's sacred place (hill on the road to Kinkin)}}
\item \entry{pn.}{\headword{Gullem suwe} \definition{1. gardening place; Galo's sacred place (AZ97)}}
\item \entry{pn.}{\headword{Gurel} \definition{1. male personal name}}
\item \entry{pn.}{\headword{Gwen} \definition{1. female personal name}}
\item \entry{pn.}{\headword{Hannah} \definition{1. female personal name}}
\item \entry{pn.}{\headword{Helen} \definition{1. female personal name}}
\item \entry{pn.}{\headword{Hiden} \definition{1. female personal name}}
\item \entry{pn.}{\headword{Ib} \definition{1. female personal name}}
\item \entry{pn.}{\headword{Ibetty} \definition{1. female personal name}}
\item \entry{pn.}{\headword{Ibikang} \definition{1. Ibikang (Kawam-speaking settlement of Wim; not far from Limol (AY96))}}
\item \entry{pn.}{\headword{Idan} \definition{1. male personal name}}
\item \entry{pn.}{\headword{Iden} \definition{1. Eden}}
\item \entry{pn.}{\headword{Idi} \definition{1. Idi language (Pahoturi River language spoken in Dimsisi, Sibidiri, Dimiri, Iblamand, and Biram)}}
\item \entry{pn.}{\headword{Idugoe} \definition{1. male personal name}}
\item \entry{pn.}{\headword{Ilaeza} \definition{1. Elijah}}
\item \entry{pn.}{\headword{Imanuel} \definition{1. Imanuel}}
\item \entry{pn.}{\headword{Ina} \definition{1. female personal name}}
\item \entry{pn.}{\headword{Inapa} \definition{1. male personal name}}
\item \entry{pn.}{\headword{Inawa} \definition{1. male personal name}}
\item \entry{pn.}{\headword{Indonesia} \definition{1. Indonesia}}
\item \entry{pn.}{\headword{Inpiakma} \definition{1. Inpiakma (toponym)}}
\item \entry{pn.}{\headword{Inpir} \definition{1. Makayam/Tirio language}}
\item \entry{pn.}{\headword{Ingglis} \definition{1. English language}}
\item \entry{pn.}{\headword{Ipott} \definition{1. Ipott (on the road to Bisuaka near Limol ma kuddäll)}}
\item \entry{pn.}{\headword{Iräm} \definition{1. creek and washing place (in Limol, AZ95)}}
\item \entry{pn.}{\headword{Iräm} \definition{1. Iräm creek}}
\item \entry{pn.}{\headword{Iräm} \definition{1. Iräm washing place}}
\item \entry{pn.}{\headword{Isago} \definition{1. Isago (big island, perhaps in Fly River)}}
\item \entry{pn.}{\headword{Ivan} \definition{1. male personal name}}
\item \entry{pn.}{\headword{Kadawa} \definition{1. Kadawa (in Kiwai Rural LLG)}}
\item \entry{pn.}{\headword{Kagär} \definition{1. female personal name}}
\item \entry{pn.}{\headword{Kago} \definition{1. male personal name}}
\item \entry{pn.}{\headword{Kakaya} \definition{1. Kakaya (toponym)}}
\item \entry{pn.}{\headword{Kakayam} \definition{1. female personal name}}
\item \entry{pn.}{\headword{Kakeya} \definition{1. Kakeya (bush, camping and sago place of Baba Zi from Upiara; on the road to Bisuaka)}}
\item \entry{pn.}{\headword{Kakos} \definition{1. female personal name}}
\item \entry{pn.}{\headword{Kalamato} \definition{1. female personal name}}
\item \entry{pn.}{\headword{Kaldon} \definition{1. male personal name}}
\item \entry{pn.}{\headword{Kanengga} \definition{1. male personal name}}
\item \entry{pn.}{\headword{Kange} \definition{1. male personal name}}
\item \entry{pn.}{\headword{Kangge} \definition{1. male personal name}}
\item \entry{pn.}{\headword{Kaoga} \definition{1. male personal name}}
\item \entry{pn.}{\headword{Kapal} \definition{1. Kapal (Wipi- and Kawam-speaking village in Oriomo-Bituri Rural LLG; has airstrip, aid post, primary school; from Limol, one must pass through Bisuaka)}}
\item \entry{pn.}{\headword{Kapangang bun} \definition{1. Kapangang bun (sago place on the way to Egapo)}}
\item \entry{pn.}{\headword{Kaparnaom} \definition{1. Capernaum}}
\item \entry{pn.}{\headword{Karama} \definition{1. Karama (swamp and canoe place in Limol)}}
\item \entry{pn.}{\headword{Karamapopo} \definition{1. female personal name}}
\item \entry{pn.}{\headword{Karao} \definition{1. male personal name}}
\item \entry{pn.}{\headword{Karea} \definition{1. male personal name}}
\item \entry{pn.}{\headword{Karen} \definition{1. female personal name}}
\item \entry{pn.}{\headword{Kares} \definition{1. female personal name}}
\item \entry{pn.}{\headword{Karis} \definition{1. female personal name}}
\item \entry{pn.}{\headword{Kas} \definition{1. male personal name}}
\item \entry{pn.}{\headword{Kasakmai} \definition{1. Kasakmai (toponym)}}
\item \entry{pn.}{\headword{Kasakmai} \definition{1. name of a person}}
\item \entry{pn.}{\headword{Kasimap} \definition{1. Kasimap (Abom-speaking village in Gogodala Rural LLG, near Zanor; has an elementary school)}}
\item \entry{pn.}{\headword{Kasir} \definition{1. male personal name}}
\item \entry{pn.}{\headword{Kaso} \definition{1. male personal name}}
\item \entry{pn.}{\headword{Katama} \definition{1. male personal name}}
\item \entry{pn.}{\headword{Kate} \definition{1. female personal name}}
\item \entry{pn.}{\headword{Katherine} \definition{1. female personal name}}
\item \entry{pn.}{\headword{Kauga} \definition{1. male personal name}}
\item \entry{pn.}{\headword{Kawa} \definition{1. male personal name}}
\item \entry{pn.}{\headword{Kawam} \definition{1. Kawam language (Pahoturi River language spoken in Wim)}}
\item \entry{pn.}{\headword{Kawiapo} \definition{1. Kaviapu (village in Gogodala Rural LLG, near Tapila)}}
\item \entry{pn.}{\headword{Kawito} \definition{1. Kawito (station in Gogodala Rural LLG)}}
\item \entry{pn.}{\headword{Kaya} \definition{1. male personal name}}
\item \entry{pn.}{\headword{Kaysy} \definition{1. female personal name}}
\item \entry{pn.}{\headword{Käball} \definition{1. Kaball (toponym)}}
\item \entry{pn.}{\headword{Käball} \definition{1. Ende dialect spoken in Käball}}
\item \entry{pn.}{\headword{Käball} \definition{1. Ende clan with the dog totem}}
\item \entry{pn.}{\headword{Kälnyam} \definition{1. male personal name}}
\item \entry{pn.}{\headword{Källäk} \definition{1. Kallak (toponym)}}
\item \entry{pn.}{\headword{Källängmäll} \definition{1. sago place near Old Kibobma}}
\item \entry{pn.}{\headword{Källid} \definition{1. female personal name}}
\item \entry{pn.}{\headword{Källnyam} \definition{1. male personal name}}
\item \entry{pn.}{\headword{Källtae} \definition{1. female personal name}}
\item \entry{pn.}{\headword{Käm} \definition{1. female personal name}}
\item \entry{pn.}{\headword{Kättpälläk bällämang} \definition{1. Jerry Dareda's sacred place (near ttälebun, on the road to Kinkin, near Binyomoll)}}
\item \entry{pn.}{\headword{Käza kup ine ma} \definition{1. well and sago place of Kwakmae in Limol (behind aid post)}}
\item \entry{pn.}{\headword{Keisi} \definition{1. female personal name}}
\item \entry{pn.}{\headword{Keith} \definition{1. male personal name}}
\item \entry{pn.}{\headword{Keke} \definition{1. female personal name}}
\item \entry{pn.}{\headword{Keks} \definition{1. personal name}}
\item \entry{pn.}{\headword{Kemu} \definition{1. male personal name}}
\item \entry{pn.}{\headword{Kename} \definition{1. Kename (village in Gogodala Rural LLG; on an island in the Fly River)}}
\item \entry{pn.}{\headword{Kenny} \definition{1. male personal name}}
\item \entry{pn.}{\headword{Keren} \definition{1. female personal name}}
\item \entry{pn.}{\headword{Kergowa} \definition{1. Kergowa (in Gogodala Rural LLG; near Balimo)}}
\item \entry{pn.}{\headword{Keriso} \definition{1. Christ}}
\item \entry{pn.}{\headword{Kesa} \definition{1. male personal name}}
\item \entry{pn.}{\headword{Kesama} \definition{1. male personal name}}
\item \entry{pn.}{\headword{Keti} \definition{1. Kurupel Täräp (Limol village), which was moved from Old Limol to Old Man Kurupel's camping place approximately four generations before 2015}}
\item \entry{pn.}{\headword{Ketrin} \definition{1. female personal name}}
\item \entry{pn.}{\headword{Kewameyato} \definition{1. female personal name}}
\item \entry{pn.}{\headword{Kevelyn} \definition{1. female personal name}}
\item \entry{pn.}{\headword{Kiata} \definition{1. male personal name}}
\item \entry{pn.}{\headword{Kibobma} \definition{1. Kibobma (previous settlement of Limol village; on the road to Kinkin, near the creeks)}}
\item \entry{pn.}{\headword{Kibuli} \definition{1. Kibuli (Em-speaking village in Oriomo-Bituri Rural LLG; near Kurunti)}}
\item \entry{pn.}{\headword{Kidarga} \definition{1. male personal name}}
\item \entry{pn.}{\headword{Kikori} \definition{1. Kikori (town in Kikori District, located on the Kikori Delta)}}
\item \entry{pn.}{\headword{Kikori} \definition{1. Kikori (river that flows into the Gulf of Papua)}}
\item \entry{pn.}{\headword{Kila} \definition{1. personal name}}
\item \entry{pn.}{\headword{Kini} \definition{1. Kini (in Gogodala Rural LLG; near Balimo and Awaba)}}
\item \entry{pn.}{\headword{Kinkin} \definition{1. Kinkin (Ende- and Taeme-speaking village in Oriomo-Bituri Rural LLG, near Limol)}}
\item \entry{pn.}{\headword{Kingsli} \definition{1. male personal name}}
\item \entry{pn.}{\headword{Kiongga} \definition{1. Kiongga (toponym)}}
\item \entry{pn.}{\headword{Kiplin} \definition{1. male personal name}}
\item \entry{pn.}{\headword{Kipling} \definition{1. male personal name}}
\item \entry{pn.}{\headword{Kiwai} \definition{1. Kiwai language (offical language of the region, native language of Daru; children's school songs are sometimes in this language)}}
\item \entry{pn.}{\headword{Kobam} \definition{1. male personal name}}
\item \entry{pn.}{\headword{Kobddag} \definition{1. Kobddag (toponym)}}
\item \entry{pn.}{\headword{Kobe} \definition{1. female personal name}}
\item \entry{pn.}{\headword{Kobemitang} \definition{1. Kobemitang (toponym)}}
\item \entry{pn.}{\headword{Koboddag} \definition{1. Koboddag (toponym)}}
\item \entry{pn.}{\headword{Koe} \definition{1. male personal name}}
\item \entry{pn.}{\headword{Koebänang} \definition{1. Koebänang (sago and hunting place; old settlement near Buddobuddog)}}
\item \entry{pn.}{\headword{Koenbäll kutt} \definition{1. Koenbäll kutt (sago and washing place of Paine and Warama Kurupel)}}
\item \entry{pn.}{\headword{Koke} \definition{1. Koke (toponym)}}
\item \entry{pn.}{\headword{Kokma} \definition{1. Kokma (toponym)}}
\item \entry{pn.}{\headword{Kolmet} \definition{1. female personal name}}
\item \entry{pn.}{\headword{Koloam} \definition{1. male personal name}}
\item \entry{pn.}{\headword{Kols} \definition{1. male personal name}}
\item \entry{pn.}{\headword{Kollwam} \definition{1. male personal name}}
\item \entry{pn.}{\headword{Kombosie} \definition{1. male personal name}}
\item \entry{pn.}{\headword{Kondobol} \definition{1. Kondobol (Taeme-speaking village in Morehead Rural LLG; from Limol, one must pass through Kinkin)}}
\item \entry{pn.}{\headword{Kondobu} \definition{1. Konedobu (in Gogodala Rural LLG)}}
\item \entry{pn.}{\headword{Koreya} \definition{1. Korea}}
\item \entry{pn.}{\headword{Kral} \definition{1. female personal name}}
\item \entry{pn.}{\headword{Kristina} \definition{1. female personal name}}
\item \entry{pn.}{\headword{Kudurwe} \definition{1. female personal name}}
\item \entry{pn.}{\headword{kuddäll} \definition{1. Passover}}
\item \entry{pn.}{\headword{kuddäll} \definition{1. life-or-death, to death, as if one may die}}
\item \entry{pn.}{\headword{Kui} \definition{1. Kui (toponym)}}
\item \entry{pn.}{\headword{Kuiwang} \definition{1. Kuiwang (Taeme-speaking village in Morehead Rural LLG; from Limol, one must pass through Malam)}}
\item \entry{pn.}{\headword{Kukpikukpi} \definition{1. Kukpikukpi (toponym)}}
\item \entry{pn.}{\headword{Kuks} \definition{1. male personal name}}
\item \entry{pn.}{\headword{Kukua} \definition{1. male personal name}}
\item \entry{pn.}{\headword{Kukumi} \definition{1. male personal name}}
\item \entry{pn.}{\headword{Kullme} \definition{1. Kullme (garden place near Egapo; filled with abandoned rubber trees)}}
\item \entry{pn.}{\headword{Kullopang} \definition{1. Kullopang (sago and garden place of Kaoga Dobola; on the shortcut road to Kinkin)}}
\item \entry{pn.}{\headword{Kullwam} \definition{1. male personal name}}
\item \entry{pn.}{\headword{Kumull} \definition{1. Kumull (toponym)}}
\item \entry{pn.}{\headword{Kuna} \definition{1. male personal name}}
\item \entry{pn.}{\headword{Kunyemäll} \definition{1. Kunyemäll (on the road to Malam near Zarma; filled with black palms that were cut for the school)}}
\item \entry{pn.}{\headword{Kur} \definition{1. Kur (Wipi-speaking village in Oriomo-Bituri Rural LLG; on the road to Oriomo)}}
\item \entry{pn.}{\headword{Kurunti} \definition{1. Kurunti (Em-speaking village in Oriomo-Bituri Rural LLG; from Limol, one must pass through Malam)}}
\item \entry{pn.}{\headword{Kurupel} \definition{1. male personal name}}
\item \entry{pn.}{\headword{Kutpi Käp} \definition{1. Kutpi Käp (toponym)}}
\item \entry{pn.}{\headword{Kuyu} \definition{1. garden place of Matthew Bodog and Kaoga Dobola in Limol}}
\item \entry{pn.}{\headword{Kwakmae} \definition{1. female personal name}}
\item \entry{pn.}{\headword{Kwalde} \definition{1. male personal name}}
\item \entry{pn.}{\headword{Kwale} \definition{1. female personal name}}
\item \entry{pn.}{\headword{Kwallangkäbäll} \definition{1. community garden place in Limol}}
\item \entry{pn.}{\headword{Kwangkangatt} \definition{1. sacred place of Dobola (on the road to Malam)}}
\item \entry{pn.}{\headword{Kwara} \definition{1. female personal name}}
\item \entry{pn.}{\headword{Kwe} \definition{1. male personal name}}
\item \entry{pn.}{\headword{Lama} \definition{1. female personal name}}
\item \entry{pn.}{\headword{Lamlam} \definition{1. name of a female ancestor (sister of Moli)}}
\item \entry{pn.}{\headword{Lauren} \definition{1. female personal name}}
\item \entry{pn.}{\headword{Lei} \definition{1. Lei (toponym)}}
\item \entry{pn.}{\headword{Lek Märi} \definition{1. Lake Murray (in Lake Murray Rural LLG)}}
\item \entry{pn.}{\headword{Letai} \definition{1. female personal name}}
\item \entry{pn.}{\headword{Lewada} \definition{1. Lewada (Makayam-speaking village in Gogodala Rural LLG, on the Fly River; GPS: 8.327787, 142.785487)}}
\item \entry{pn.}{\headword{Lidiya} \definition{1. female personal name}}
\item \entry{pn.}{\headword{Lili} \definition{1. female personal name}}
\item \entry{pn.}{\headword{Lilian} \definition{1. female personal name}}
\item \entry{pn.}{\headword{Limoll} \definition{1. Limol (Ende-speaking village in Morehead Rural LLG; GPS: -8.641783, 142.682533)}}
\item \entry{pn.}{\headword{Limoll} \definition{1. Ende dialect spoken in Limol}}
\item \entry{pn.}{\headword{Linda} \definition{1. female personal name}}
\item \entry{pn.}{\headword{Linette} \definition{1. female personal name}}
\item \entry{pn.}{\headword{Liseng} \definition{1. PN}}
\item \entry{pn.}{\headword{Lois} \definition{1. female personal name}}
\item \entry{pn.}{\headword{Lomae} \definition{1. female personal name}}
\item \entry{pn.}{\headword{Loni} \definition{1. female personal name}}
\item \entry{pn.}{\headword{Lovelyn} \definition{1. female personal name}}
\item \entry{pn.}{\headword{Ludwina} \definition{1. female personal name}}
\item \entry{pn.}{\headword{Luke} \definition{1. male personal name}}
\item \entry{pn.}{\headword{Lulu} \definition{1. female personal name}}
\item \entry{pn.}{\headword{Lydia} \definition{1. female personal name}}
\item \entry{pn.}{\headword{Lyneth} \definition{1. female personal name}}
\item \entry{pn.}{\headword{Mabudawan} \definition{1. Mabudawan/Mabaduan (Agob-speaking village in Kiwai Rural LLG near Saibai Island)}}
\item \entry{pn.}{\headword{Madima} \definition{1. female personal name}}
\item \entry{pn.}{\headword{Madlin} \definition{1. female personal name}}
\item \entry{pn.}{\headword{Mado} \definition{1. male personal name}}
\item \entry{pn.}{\headword{Madura} \definition{1. male personal name}}
\item \entry{pn.}{\headword{Mak} \definition{1. male personal name}}
\item \entry{pn.}{\headword{Makaka} \definition{1. female personal name}}
\item \entry{pn.}{\headword{Maki} \definition{1. female personal name}}
\item \entry{pn.}{\headword{Malläm} \definition{1. Malam (Ende-speaking village in Morehead Rural LLG; a two-hour walk 9.3km) from Limol; GPS: -8.712716, 142.656097)}}
\item \entry{pn.}{\headword{Mame} \definition{1. female personal name}}
\item \entry{pn.}{\headword{Mamen} \definition{1. Mamen (toponym)}}
\item \entry{pn.}{\headword{Mana} \definition{1. female personal name}}
\item \entry{pn.}{\headword{Manaleato} \definition{1. female personal name}}
\item \entry{pn.}{\headword{Manang} \definition{1. male personal name}}
\item \entry{pn.}{\headword{Maneya} \definition{1. name of a person}}
\item \entry{pn.}{\headword{Mang} \definition{1. Mang (toponym)}}
\item \entry{pn.}{\headword{Mangel} \definition{1. Mangel (toponym)}}
\item \entry{pn.}{\headword{Manggeya} \definition{1. female personal name}}
\item \entry{pn.}{\headword{Mangkol} \definition{1. female personal name}}
\item \entry{pn.}{\headword{Mareas} \definition{1. personal name}}
\item \entry{pn.}{\headword{Marega} \definition{1. male personal name}}
\item \entry{pn.}{\headword{Mareyas} \definition{1. Mareyas (toponym)}}
\item \entry{pn.}{\headword{Maria} \definition{1. female personal name}}
\item \entry{pn.}{\headword{Marian} \definition{1. female personal name}}
\item \entry{pn.}{\headword{Marias} \definition{1. male personal name}}
\item \entry{pn.}{\headword{Marion} \definition{1. female personal name}}
\item \entry{pn.}{\headword{Martha} \definition{1. female personal name}}
\item \entry{pn.}{\headword{Mas} \definition{1. male personal name}}
\item \entry{pn.}{\headword{Masam} \definition{1. Bitur language}}
\item \entry{pn.}{\headword{Masingara} \definition{1. Masingara (Bine-speaking village in Oriomo-Bituri Rural LLG)}}
\item \entry{pn.}{\headword{Masta} \definition{1. female personal name}}
\item \entry{pn.}{\headword{Mata} \definition{1. Mata (in Morehead Rural LLG)}}
\item \entry{pn.}{\headword{Mataru} \definition{1. personal name}}
\item \entry{pn.}{\headword{Matär} \definition{1. male personal name}}
\item \entry{pn.}{\headword{Maten} \definition{1. male personal name}}
\item \entry{pn.}{\headword{Mathilda} \definition{1. female personal name}}
\item \entry{pn.}{\headword{Matias} \definition{1. male personal name}}
\item \entry{pn.}{\headword{Mavis} \definition{1. female personal name}}
\item \entry{pn.}{\headword{Mäkayam} \definition{1. Makayam/Tirio language}}
\item \entry{pn.}{\headword{Mätär} \definition{1. Mätär (toponym)}}
\item \entry{pn.}{\headword{Medang} \definition{1. Medang (toponym)}}
\item \entry{pn.}{\headword{Megam} \definition{1. male personal name}}
\item \entry{pn.}{\headword{Megi} \definition{1. female personal name}}
\item \entry{pn.}{\headword{Meklin} \definition{1. female personal name}}
\item \entry{pn.}{\headword{Meks} \definition{1. male personal name}}
\item \entry{pn.}{\headword{Meliye} \definition{1. Meliye (toponym)}}
\item \entry{pn.}{\headword{Melvin} \definition{1. male personal name}}
\item \entry{pn.}{\headword{Meragag} \definition{1. Meragag (toponym)}}
\item \entry{pn.}{\headword{Meramerall} \definition{1. Meramerall (toponym)}}
\item \entry{pn.}{\headword{Meri} \definition{1. female personal name}}
\item \entry{pn.}{\headword{Merian} \definition{1. female personal name}}
\item \entry{pn.}{\headword{Meroka} \definition{1. female personal name}}
\item \entry{pn.}{\headword{Merol} \definition{1. female personal name}}
\item \entry{pn.}{\headword{Mesa} \definition{1. male personal name}}
\item \entry{pn.}{\headword{Metyu} \definition{1. male personal name}}
\item \entry{pn.}{\headword{Mewato} \definition{1. female personal name}}
\item \entry{pn.}{\headword{Mecklyn} \definition{1. female personal name}}
\item \entry{pn.}{\headword{Migul} \definition{1. male personal name}}
\item \entry{pn.}{\headword{Minkäm} \definition{1. Minkam (toponym)}}
\item \entry{pn.}{\headword{Minkomminkomang} \definition{1. Minkomminkomang (toponym)}}
\item \entry{pn.}{\headword{Minong} \definition{1. male personal name}}
\item \entry{pn.}{\headword{Mingkällbun} \definition{1. Mingkällbun (toponym)}}
\item \entry{pn.}{\headword{Miriang} \definition{1. male personal name}}
\item \entry{pn.}{\headword{Misseilene} \definition{1. female personal name}}
\item \entry{pn.}{\headword{Michael} \definition{1. male personal name}}
\item \entry{pn.}{\headword{Michaelyn} \definition{1. female personal name}}
\item \entry{pn.}{\headword{Michelle} \definition{1. female personal name}}
\item \entry{pn.}{\headword{Moed} \definition{1. Moed (toponym)}}
\item \entry{pn.}{\headword{Moem} \definition{1. Moem (toponym)}}
\item \entry{pn.}{\headword{Moli} \definition{1. name of a male ancestor (brother of Lamlam)}}
\item \entry{pn.}{\headword{Mome} \definition{1. female personal name}}
\item \entry{pn.}{\headword{Momeya} \definition{1. Momeya (toponym)}}
\item \entry{pn.}{\headword{Mompelang} \definition{1. Mompelang (toponym)}}
\item \entry{pn.}{\headword{Moses} \definition{1. male personal name}}
\item \entry{pn.}{\headword{Motu} \definition{1. Hiri Motu language}}
\item \entry{pn.}{\headword{Moyabag} \definition{1. male personal name}}
\item \entry{pn.}{\headword{Mugi} \definition{1. male personal name}}
\item \entry{pn.}{\headword{Muidebag} \definition{1. Muidebag (toponym)}}
\item \entry{pn.}{\headword{Mul} \definition{1. Mul (toponym)}}
\item \entry{pn.}{\headword{Mull} \definition{1. Mull (toponym)}}
\item \entry{pn.}{\headword{Munu} \definition{1. male personal name}}
\item \entry{pn.}{\headword{Mur} \definition{1. Mur (toponym)}}
\item \entry{pn.}{\headword{Musato} \definition{1. female personal name}}
\item \entry{pn.}{\headword{Muyabag} \definition{1. male personal name}}
\item \entry{pn.}{\headword{Naemäll} \definition{1. female personal name}}
\item \entry{pn.}{\headword{Nagab} \definition{1. male personal name}}
\item \entry{pn.}{\headword{Nagat} \definition{1. male personal name}}
\item \entry{pn.}{\headword{Nageg} \definition{1. male personal name}}
\item \entry{pn.}{\headword{Nakaku} \definition{1. Nakaku (toponym)}}
\item \entry{pn.}{\headword{Naklae} \definition{1. male personal name}}
\item \entry{pn.}{\headword{Nakuri} \definition{1. male personal name}}
\item \entry{pn.}{\headword{Nalon} \definition{1. male personal name}}
\item \entry{pn.}{\headword{Nama} \definition{1. male personal name}}
\item \entry{pn.}{\headword{Namaya} \definition{1. female personal name}}
\item \entry{pn.}{\headword{Nanggon} \definition{1. male personal name}}
\item \entry{pn.}{\headword{Naomi} \definition{1. female personal name}}
\item \entry{pn.}{\headword{Narma} \definition{1. male personal name}}
\item \entry{pn.}{\headword{Nasma} \definition{1. male personal name}}
\item \entry{pn.}{\headword{Nazaret} \definition{1. Nazareth}}
\item \entry{pn.}{\headword{Nägäm} \definition{1. male personal name}}
\item \entry{pn.}{\headword{Nänga} \definition{1. female personal name}}
\item \entry{pn.}{\headword{Nedlyn} \definition{1. female personal name}}
\item \entry{pn.}{\headword{Nensi} \definition{1. female personal name}}
\item \entry{pn.}{\headword{Niki} \definition{1. male personal name}}
\item \entry{pn.}{\headword{Nikol} \definition{1. female personal name}}
\item \entry{pn.}{\headword{Niniab} \definition{1. male personal name}}
\item \entry{pn.}{\headword{Nixon} \definition{1. male personal name}}
\item \entry{pn.}{\headword{Noar} \definition{1. female personal name}}
\item \entry{pn.}{\headword{Nogat} \definition{1. male personal name}}
\item \entry{pn.}{\headword{Nolin} \definition{1. female personal name}}
\item \entry{pn.}{\headword{Nope} \definition{1. male personal name}}
\item \entry{pn.}{\headword{Norma} \definition{1. female personal name}}
\item \entry{pn.}{\headword{Nuam} \definition{1. female personal name}}
\item \entry{pn.}{\headword{Nugini} \definition{1. New Guinea}}
\item \entry{pn.}{\headword{Nuopin} \definition{1. female personal name}}
\item \entry{pn.}{\headword{Ngao} \definition{1. Ngao (toponym)}}
\item \entry{pn.}{\headword{Ngeba} \definition{1. Ngeba (toponym)}}
\item \entry{pn.}{\headword{Ngerbab} \definition{1. male personal name}}
\item \entry{pn.}{\headword{Obama} \definition{1. Obama (toponym)}}
\item \entry{pn.}{\headword{Obewa} \definition{1. male personal name}}
\item \entry{pn.}{\headword{Ogbaperma} \definition{1. Ogbaperma (camping place)}}
\item \entry{pn.}{\headword{Ogoa} \definition{1. male personal name}}
\item \entry{pn.}{\headword{Olalea} \definition{1. female personal name}}
\item \entry{pn.}{\headword{Old Maoto} \definition{1. Old Maoto (toponym)}}
\item \entry{pn.}{\headword{Ono} \definition{1. Ono (toponym)}}
\item \entry{pn.}{\headword{Ongg} \definition{1. male personal name}}
\item \entry{pn.}{\headword{Opo} \definition{1. Opo (toponym)}}
\item \entry{pn.}{\headword{Oriomo} \definition{1. Oriomo (Wipi-speaking village in Oriomo-Bituri Rural LLG)}}
\item \entry{pn.}{\headword{Orpmang} \definition{1. Wipi language}}
\item \entry{pn.}{\headword{Ouli} \definition{1. female personal name}}
\item \entry{pn.}{\headword{Paeke} \definition{1. female personal name}}
\item \entry{pn.}{\headword{Paelet} \definition{1. Pilate}}
\item \entry{pn.}{\headword{Paine} \definition{1. male personal name}}
\item \entry{pn.}{\headword{Pakllepakllemäll} \definition{1. Pakllepakllemäll (camping place)}}
\item \entry{pn.}{\headword{Palsa} \definition{1. Palsa (toponym)}}
\item \entry{pn.}{\headword{Panakawa} \definition{1. Panakawa (toponym)}}
\item \entry{pn.}{\headword{Papon} \definition{1. male personal name}}
\item \entry{pn.}{\headword{Papua Niugini} \definition{1. Papua New Guinea}}
\item \entry{pn.}{\headword{Parama} \definition{1. Parama (in Kiwai Rural LLG)}}
\item \entry{pn.}{\headword{Paskam} \definition{1. male personal name}}
\item \entry{pn.}{\headword{Pata} \definition{1. male personal name}}
\item \entry{pn.}{\headword{Patha} \definition{1. male personal name}}
\item \entry{pn.}{\headword{Pauma} \definition{1. female personal name}}
\item \entry{pn.}{\headword{Pawaturi} \definition{1. Pahoturi River}}
\item \entry{pn.}{\headword{pällämpälläm} \definition{1. English}}
\item \entry{pn.}{\headword{Pällmang} \definition{1. Pällmang (toponym)}}
\item \entry{pn.}{\headword{Pätta} \definition{1. Pätta (toponym)}}
\item \entry{pn.}{\headword{Pedaya} \definition{1. Pedaya (toponym)}}
\item \entry{pn.}{\headword{Pedro} \definition{1. male personal name}}
\item \entry{pn.}{\headword{Pentae} \definition{1. female personal name}}
\item \entry{pn.}{\headword{Petepo} \definition{1. female personal name}}
\item \entry{pn.}{\headword{Petom} \definition{1. Petom (toponym)}}
\item \entry{pn.}{\headword{Pewe} \definition{1. male personal name}}
\item \entry{pn.}{\headword{Piasorosoro} \definition{1. male personal name}}
\item \entry{pn.}{\headword{Pidortama} \definition{1. Pidortama (toponym)}}
\item \entry{pn.}{\headword{Pinang} \definition{1. Pinang (toponym)}}
\item \entry{pn.}{\headword{Pingam} \definition{1. female personal name}}
\item \entry{pn.}{\headword{Pipi} \definition{1. female personal name}}
\item \entry{pn.}{\headword{Pipiato} \definition{1. female personal name}}
\item \entry{pn.}{\headword{Pisi} \definition{1. Pisi (in Gogodala Rural LLG)}}
\item \entry{pn.}{\headword{Piskae} \definition{1. Piskae (toponym)}}
\item \entry{pn.}{\headword{Pita} \definition{1. male personal name}}
\item \entry{pn.}{\headword{Pitepo} \definition{1. female personal name}}
\item \entry{pn.}{\headword{Pizi} \definition{1. Pizi (toponym)}}
\item \entry{pn.}{\headword{Pizin} \definition{1. Tok Pisin}}
\item \entry{pn.}{\headword{Podare} \definition{1. Podare (Wipi-speaking village in Oriomo-Bituri Rural LLG)}}
\item \entry{pn.}{\headword{Pogo} \definition{1. Pogo (toponym)}}
\item \entry{pn.}{\headword{Pol} \definition{1. male personal name}}
\item \entry{pn.}{\headword{Polin} \definition{1. female personal name}}
\item \entry{pn.}{\headword{Poll} \definition{1. male personal name}}
\item \entry{pn.}{\headword{Pondollowang} \definition{1. Pondollowang (camping place)}}
\item \entry{pn.}{\headword{Ponongllowang} \definition{1. Ponongllowang (toponym)}}
\item \entry{pn.}{\headword{Pongarke} \definition{1. Pongariki (Nambo-speaking village in Morehead Rural LLG)}}
\item \entry{pn.}{\headword{Pot Mosbi} \definition{1. Port Moresby (the capital city of Papua New Guinea)}}
\item \entry{pn.}{\headword{Pottängäm} \definition{1. Pottängäm (camp and garden place)}}
\item \entry{pn.}{\headword{Priski} \definition{1. female personal name}}
\item \entry{pn.}{\headword{Priscilla} \definition{1. female personal name}}
\item \entry{pn.}{\headword{Puinde} \definition{1. male personal name}}
\item \entry{pn.}{\headword{Rabaul} \definition{1. Rabaul (toponym)}}
\item \entry{pn.}{\headword{Ranky} \definition{1. male personal name}}
\item \entry{pn.}{\headword{Raroge} \definition{1. Raroge (toponym)}}
\item \entry{pn.}{\headword{Rasol} \definition{1. male personal name}}
\item \entry{pn.}{\headword{Raynold} \definition{1. male personal name}}
\item \entry{pn.}{\headword{Räba blok} \definition{1. Rubber block (toponym)}}
\item \entry{pn.}{\headword{Redley} \definition{1. male personal name}}
\item \entry{pn.}{\headword{Reend} \definition{1. male personal name}}
\item \entry{pn.}{\headword{Regina} \definition{1. female personal name}}
\item \entry{pn.}{\headword{Reks} \definition{1. male personal name}}
\item \entry{pn.}{\headword{Rena} \definition{1. female personal name}}
\item \entry{pn.}{\headword{Rhoda} \definition{1. female personal name}}
\item \entry{pn.}{\headword{Rind} \definition{1. male personal name}}
\item \entry{pn.}{\headword{Richard} \definition{1. male personal name}}
\item \entry{pn.}{\headword{Roaele} \definition{1. male personal name}}
\item \entry{pn.}{\headword{Roak} \definition{1. male personal name}}
\item \entry{pn.}{\headword{Robae} \definition{1. female personal name}}
\item \entry{pn.}{\headword{Rom} \definition{1. Rome}}
\item \entry{pn.}{\headword{Rose} \definition{1. female personal name}}
\item \entry{pn.}{\headword{Rosela} \definition{1. female personal name}}
\item \entry{pn.}{\headword{Rowak} \definition{1. male personal name}}
\item \entry{pn.}{\headword{Rual} \definition{1. Rual (toponym)}}
\item \entry{pn.}{\headword{Sadua} \definition{1. male personal name}}
\item \entry{pn.}{\headword{Saemon} \definition{1. male personal name}}
\item \entry{pn.}{\headword{saeten} \definition{1. Satan}}
\item \entry{pn.}{\headword{Saisiato} \definition{1. female personal name}}
\item \entry{pn.}{\headword{Sakoyaratt} \definition{1. Sakoyaratt (toponym)}}
\item \entry{pn.}{\headword{Sali} \definition{1. male personal name}}
\item \entry{pn.}{\headword{Salome} \definition{1. female personal name}}
\item \entry{pn.}{\headword{Sam} \definition{1. male personal name}}
\item \entry{pn.}{\headword{Samae} \definition{1. male personal name}}
\item \entry{pn.}{\headword{Samari} \definition{1. Samari (toponym)}}
\item \entry{pn.}{\headword{Samat} \definition{1. female personal name}}
\item \entry{pn.}{\headword{Samson} \definition{1. male personal name}}
\item \entry{pn.}{\headword{Samuel} \definition{1. male personal name}}
\item \entry{pn.}{\headword{Sandra} \definition{1. female personal name}}
\item \entry{pn.}{\headword{Sanford} \definition{1. male personal name}}
\item \entry{pn.}{\headword{Sapusa} \definition{1. female personal name}}
\item \entry{pn.}{\headword{Sara} \definition{1. female personal name}}
\item \entry{pn.}{\headword{Sarbi} \definition{1. female personal name}}
\item \entry{pn.}{\headword{Sasa} \definition{1. female personal name}}
\item \entry{pn.}{\headword{Sasi} \definition{1. female personal name}}
\item \entry{pn.}{\headword{Sawa} \definition{1. male personal name}}
\item \entry{pn.}{\headword{Sawapo} \definition{1. male personal name}}
\item \entry{pn.}{\headword{Saweta} \definition{1. Saweta (toponym)}}
\item \entry{pn.}{\headword{Sägrep} \definition{1. male personal name}}
\item \entry{pn.}{\headword{Sebe} \definition{1. Sebe (Bine-speaking village in Oriomo-Bituri Rural LLG)}}
\item \entry{pn.}{\headword{Senti} \definition{1. male personal name}}
\item \entry{pn.}{\headword{Sera} \definition{1. female personal name}}
\item \entry{pn.}{\headword{Sharon} \definition{1. female personal name}}
\item \entry{pn.}{\headword{Shim} \definition{1. male personal name}}
\item \entry{pn.}{\headword{Sibideri} \definition{1. Sibidiri (Idi-speaking village in Morehead Rural LLG)}}
\item \entry{pn.}{\headword{Sibiya} \definition{1. male personal name}}
\item \entry{pn.}{\headword{Sibne} \definition{1. Sibne (toponym)}}
\item \entry{pn.}{\headword{Siga} \definition{1. Siga (toponym)}}
\item \entry{pn.}{\headword{Sigabaduru} \definition{1. Sigabaduru (in Kiwai Rural LLG)}}
\item \entry{pn.}{\headword{Siku} \definition{1. male personal name}}
\item \entry{pn.}{\headword{Sini} \definition{1. female personal name}}
\item \entry{pn.}{\headword{Sintia} \definition{1. female personal name}}
\item \entry{pn.}{\headword{Sirmitang} \definition{1. Sirmitang (toponym)}}
\item \entry{pn.}{\headword{Sisuar} \definition{1. female personal name}}
\item \entry{pn.}{\headword{Skola} \definition{1. female personal name}}
\item \entry{pn.}{\headword{Soba} \definition{1. male personal name}}
\item \entry{pn.}{\headword{Sobam} \definition{1. male personal name}}
\item \entry{pn.}{\headword{Sobeya} \definition{1. Sobeya (toponym)}}
\item \entry{pn.}{\headword{Sogale} \definition{1. Sogale (Bine-speaking village in Oriomo-Bituri Rural LLG)}}
\item \entry{pn.}{\headword{Soka} \definition{1. male personal name}}
\item \entry{pn.}{\headword{Sokola} \definition{1. female personal name}}
\item \entry{pn.}{\headword{Soma} \definition{1. male personal name}}
\item \entry{pn.}{\headword{Songno} \definition{1. female personal name}}
\item \entry{pn.}{\headword{Sowa} \definition{1. male personal name}}
\item \entry{pn.}{\headword{Sowati} \definition{1. male personal name}}
\item \entry{pn.}{\headword{Stanis} \definition{1. male personal name}}
\item \entry{pn.}{\headword{Stashalyn} \definition{1. female personal name}}
\item \entry{pn.}{\headword{Stibin} \definition{1. male personal name}}
\item \entry{pn.}{\headword{Suame} \definition{1. Suame (toponym)}}
\item \entry{pn.}{\headword{Suki} \definition{1. Suki (in Morehead Rural LLG)}}
\item \entry{pn.}{\headword{Suliki} \definition{1. male personal name}}
\item \entry{pn.}{\headword{Susan} \definition{1. female personal name}}
\item \entry{pn.}{\headword{Suwede} \definition{1. male personal name}}
\item \entry{pn.}{\headword{Suwi} \definition{1. Sui (in Kiwai Rural LLG)}}
\item \entry{pn.}{\headword{Sylvien} \definition{1. female personal name}}
\item \entry{pn.}{\headword{Tabita} \definition{1. female personal name}}
\item \entry{pn.}{\headword{Tabubil} \definition{1. Tabubil (toponym)}}
\item \entry{pn.}{\headword{Tag} \definition{1. personal name}}
\item \entry{pn.}{\headword{Takeya} \definition{1. female personal name}}
\item \entry{pn.}{\headword{Tallabunang} \definition{1. Tallabunang (toponym)}}
\item \entry{pn.}{\headword{Tame} \definition{1. Taeme language (Pahoturi River language spoken in Kinkin alongside Ende)}}
\item \entry{pn.}{\headword{Tanisha} \definition{1. female personal name}}
\item \entry{pn.}{\headword{Tao} \definition{1. male personal name}}
\item \entry{pn.}{\headword{Taolang} \definition{1. Taolang (camping place)}}
\item \entry{pn.}{\headword{Tapila} \definition{1. Tapila (Makayam-speaking village in Gogodala Rural LLG; GPS: -8.414202, 143.016867)}}
\item \entry{pn.}{\headword{Tapma} \definition{1. Tapma (toponym)}}
\item \entry{pn.}{\headword{Tawabo} \definition{1. Tawabo (toponym)}}
\item \entry{pn.}{\headword{Tawemitang} \definition{1. Tawemitang (toponym)}}
\item \entry{pn.}{\headword{Tayi} \definition{1. Tai (in Gogodala Rural LLG)}}
\item \entry{pn.}{\headword{Täm} \definition{1. female personal name}}
\item \entry{pn.}{\headword{Tärapang} \definition{1. female personal name}}
\item \entry{pn.}{\headword{Tätän} \definition{1. unisex personal name}}
\item \entry{pn.}{\headword{Tebar} \definition{1. Tebar (toponym)}}
\item \entry{pn.}{\headword{Teks} \definition{1. male personal name}}
\item \entry{pn.}{\headword{Tergo} \definition{1. male personal name}}
\item \entry{pn.}{\headword{Terrance} \definition{1. male personal name}}
\item \entry{pn.}{\headword{Tewa} \definition{1. male personal name}}
\item \entry{pn.}{\headword{Tewara} \definition{1. Tewara (Bitur-speaking village in Oriomo-Bitur Rural LLG)}}
\item \entry{pn.}{\headword{Teyapopo} \definition{1. Teyapopo (toponym)}}
\item \entry{pn.}{\headword{Tim} \definition{1. female personal name}}
\item \entry{pn.}{\headword{Tina} \definition{1. female personal name}}
\item \entry{pn.}{\headword{Tirere} \definition{1. Tirere/Tire'ere (Waboda-speaking village in Kiwai Rural LLG)}}
\item \entry{pn.}{\headword{Titi} \definition{1. Titi (toponym)}}
\item \entry{pn.}{\headword{Titus} \definition{1. male personal name}}
\item \entry{pn.}{\headword{Tizag} \definition{1. Ende dialect}}
\item \entry{pn.}{\headword{Togllaema} \definition{1. Togllaema (toponym)}}
\item \entry{pn.}{\headword{Togowa} \definition{1. Togowa (toponym)}}
\item \entry{pn.}{\headword{Tok pisin} \definition{1. Tok Pisin}}
\item \entry{pn.}{\headword{Tomas} \definition{1. male personal name}}
\item \entry{pn.}{\headword{Tomato} \definition{1. female personal name}}
\item \entry{pn.}{\headword{Tomson} \definition{1. male personal name}}
\item \entry{pn.}{\headword{Toni} \definition{1. male personal name}}
\item \entry{pn.}{\headword{Tonzah} \definition{1. male personal name}}
\item \entry{pn.}{\headword{Torok mittang} \definition{1. Torok mittang (toponym)}}
\item \entry{pn.}{\headword{Towarwamang} \definition{1. Towarwamang (toponym)}}
\item \entry{pn.}{\headword{Tube} \definition{1. male personal name}}
\item \entry{pn.}{\headword{Tubu} \definition{1. male personal name}}
\item \entry{pn.}{\headword{Tungnu} \definition{1. Tungnu (toponym)}}
\item \entry{pn.}{\headword{Tutuli} \definition{1. male personal name}}
\item \entry{pn.}{\headword{Tuyu} \definition{1. male personal name}}
\item \entry{pn.}{\headword{Ttae} \definition{1. male personal name}}
\item \entry{pn.}{\headword{Ttall} \definition{1. male personal name}}
\item \entry{pn.}{\headword{Ttäbe Ttäbe} \definition{1. Ttäbe Ttäbe (toponym)}}
\item \entry{pn.}{\headword{Ttägällag kona} \definition{1. Ttägälläg corner}}
\item \entry{pn.}{\headword{Ttägällag pollon} \definition{1. Ttägällag pollon (toponym)}}
\item \entry{pn.}{\headword{Ttäle Bun} \definition{1. Ttäle Bun (toponym)}}
\item \entry{pn.}{\headword{Ttäle mitt} \definition{1. place with a well in Limol}}
\item \entry{pn.}{\headword{Ttälebun} \definition{1. Ttalebun (toponym)}}
\item \entry{pn.}{\headword{Uba} \definition{1. female personal name}}
\item \entry{pn.}{\headword{Ubäd} \definition{1. male personal name}}
\item \entry{pn.}{\headword{Ubrag} \definition{1. male personal name}}
\item \entry{pn.}{\headword{Umbuzag} \definition{1. male personal name (name of the original Agob man)}}
\item \entry{pn.}{\headword{Upiara} \definition{1. Upiara (Bitur-speaking village in Oriomo-Bituri Rural LLG; GPS: -8.547170, 142.653008)}}
\item \entry{pn.}{\headword{Ur} \definition{1. Ur (toponym)}}
\item \entry{pn.}{\headword{Urimba} \definition{1. Urimba (toponym)}}
\item \entry{pn.}{\headword{Uroe} \definition{1. Wuroi (in Oriomo-Bituri Rural LLG)}}
\item \entry{pn.}{\headword{Uzaba} \definition{1. male personal name}}
\item \entry{pn.}{\headword{Uziag} \definition{1. male personal name}}
\item \entry{pn.}{\headword{Wadär Mitang} \definition{1. Wadär Mitang (toponym)}}
\item \entry{pn.}{\headword{Wae} \definition{1. female personal name}}
\item \entry{pn.}{\headword{Waedar} \definition{1. female personal name}}
\item \entry{pn.}{\headword{Waenum} \definition{1. female personal name}}
\item \entry{pn.}{\headword{Wagälla} \definition{1. Wagälla (toponym)}}
\item \entry{pn.}{\headword{Wagiba} \definition{1. female personal name}}
\item \entry{pn.}{\headword{Wagiya} \definition{1. personal name}}
\item \entry{pn.}{\headword{Wai} \definition{1. female personal name}}
\item \entry{pn.}{\headword{Wainum} \definition{1. female personal name}}
\item \entry{pn.}{\headword{Waka Källäm} \definition{1. Waka Pond (in Limol)}}
\item \entry{pn.}{\headword{Wala} \definition{1. personal name}}
\item \entry{pn.}{\headword{Waliyama} \definition{1. Wariama (in Gogodala Rural LLG)}}
\item \entry{pn.}{\headword{Wama} \definition{1. female personal name}}
\item \entry{pn.}{\headword{Wamorong} \definition{1. Wamorong (toponym)}}
\item \entry{pn.}{\headword{Wanadidi} \definition{1. female personal name}}
\item \entry{pn.}{\headword{Wane} \definition{1. male personal name}}
\item \entry{pn.}{\headword{Wapotea} \definition{1. Wapotea (toponym)}}
\item \entry{pn.}{\headword{Wara} \definition{1. Wara (toponym)}}
\item \entry{pn.}{\headword{Warama} \definition{1. male personal name}}
\item \entry{pn.}{\headword{Warani} \definition{1. male personal name}}
\item \entry{pn.}{\headword{Wareka} \definition{1. female personal name}}
\item \entry{pn.}{\headword{Wareya} \definition{1. male personal name}}
\item \entry{pn.}{\headword{Wariabodolo} \definition{1. Variobadoro (in Kiwai Rural LLG)}}
\item \entry{pn.}{\headword{Warik} \definition{1. male personal name}}
\item \entry{pn.}{\headword{Warola} \definition{1. Warola (camping place in Limol)}}
\item \entry{pn.}{\headword{Wasang} \definition{1. male personal name}}
\item \entry{pn.}{\headword{Wasua} \definition{1. Wasua (in Gogodala Rural LLG)}}
\item \entry{pn.}{\headword{Wawase} \definition{1. male personal name}}
\item \entry{pn.}{\headword{Waweba} \definition{1. male personal name}}
\item \entry{pn.}{\headword{Wawera} \definition{1. male personal name}}
\item \entry{pn.}{\headword{Wayapampe} \definition{1. Wayapampe (toponym)}}
\item \entry{pn.}{\headword{Wäli} \definition{1. female personal name}}
\item \entry{pn.}{\headword{Wän} \definition{1. female personal name}}
\item \entry{pn.}{\headword{Wätt bo ma} \definition{1. Wätt bo ma (toponym)}}
\item \entry{pn.}{\headword{Wäziag} \definition{1. personal name}}
\item \entry{pn.}{\headword{Wed} \definition{1. Wed (toponym)}}
\item \entry{pn.}{\headword{Wed} \definition{1. female personal name}}
\item \entry{pn.}{\headword{Wedereamo} \definition{1. Wederehiamo (on the south side of the mouth of the Fly River)}}
\item \entry{pn.}{\headword{Wesli} \definition{1. male personal name}}
\item \entry{pn.}{\headword{Wiben} \definition{1. male personal name}}
\item \entry{pn.}{\headword{Widama} \definition{1. Widama (toponym)}}
\item \entry{pn.}{\headword{Wik} \definition{1. male personal name}}
\item \entry{pn.}{\headword{Wilma} \definition{1. female personal name}}
\item \entry{pn.}{\headword{Willie} \definition{1. male personal name}}
\item \entry{pn.}{\headword{Wim} \definition{1. Wim (Kawam-speaking village in Oriomo-Bituri Rural LLG; GPS: 8.762144, 142.770164)}}
\item \entry{pn.}{\headword{Winny} \definition{1. female personal name}}
\item \entry{pn.}{\headword{Winson} \definition{1. male personal name}}
\item \entry{pn.}{\headword{Wingäm} \definition{1. female personal name}}
\item \entry{pn.}{\headword{Winy} \definition{1. Winy (toponym)}}
\item \entry{pn.}{\headword{Wipi} \definition{1. Wipi language}}
\item \entry{pn.}{\headword{Wipim} \definition{1. Wipim (Wipi- and Kawam-speaking village in Oriomo-Bituri Rural LLG; GPS: -8.786616, 142.872201)}}
\item \entry{pn.}{\headword{Wizing} \definition{1. male personal name}}
\item \entry{pn.}{\headword{Wɨr} \definition{1. Wɨr (toponym)}}
\item \entry{pn.}{\headword{Wɨrbun} \definition{1. Wɨrbun (toponym)}}
\item \entry{pn.}{\headword{Wonie} \definition{1. Wonie (Wipi-speaking village in Oriomo-Bituri Rural LLG; GPS: -8.838939, 142.975007)}}
\item \entry{pn.}{\headword{Wot} \definition{1. Wot (toponym)}}
\item \entry{pn.}{\headword{Wun} \definition{1. male personal name}}
\item \entry{pn.}{\headword{Wurlaimäll} \definition{1. Wurlaimäll (toponym)}}
\item \entry{pn.}{\headword{Yam} \definition{1. Yam (toponym)}}
\item \entry{pn.}{\headword{Yamayama} \definition{1. Yamayama (toponym)}}
\item \entry{pn.}{\headword{Yamega} \definition{1. Iamega (Wipi-speaking village in Oriomo-Bituri Rural LLG)}}
\item \entry{pn.}{\headword{Yao} \definition{1. Taeme language}}
\item \entry{pn.}{\headword{Yarbab} \definition{1. male personal name}}
\item \entry{pn.}{\headword{Yarte} \definition{1. Yarte (camping place)}}
\item \entry{pn.}{\headword{Yawani} \definition{1. male personal name}}
\item \entry{pn.}{\headword{Yawen} \definition{1. female personal name}}
\item \entry{pn.}{\headword{Yesu} \definition{1. Jesus}}
\item \entry{pn.}{\headword{Yina} \definition{1. female personal name}}
\item \entry{pn.}{\headword{Yokas} \definition{1. Yokas (camping place in Limol)}}
\item \entry{pn.}{\headword{Yokon} \definition{1. personal name}}
\item \entry{pn.}{\headword{Yop} \definition{1. Yop (toponym)}}
\item \entry{pn.}{\headword{Yoteang} \definition{1. Yoteang (toponym)}}
\item \entry{pn.}{\headword{Yow} \definition{1. Yau (in Gogodala Rural LLG)}}
\item \entry{pn.}{\headword{Yuga} \definition{1. male personal name}}
\item \entry{pn.}{\headword{Yugui} \definition{1. female personal name}}
\item \entry{pn.}{\headword{Yun} \definition{1. female personal name}}
\item \entry{pn.}{\headword{Zakae} \definition{1. male personal name}}
\item \entry{pn.}{\headword{Zanor} \definition{1. Zanor (in Gogodala Rural LLG; GPS: -8.459817, 142.688025)}}
\item \entry{pn.}{\headword{Zanger} \definition{1. female personal name}}
\item \entry{pn.}{\headword{Zarma} \definition{1. Zarma (toponym)}}
\item \entry{pn.}{\headword{Zedem} \definition{1. male personal name}}
\item \entry{pn.}{\headword{Zegma} \definition{1. Zegma (camping place)}}
\item \entry{pn.}{\headword{Zeims} \definition{1. male personal name}}
\item \entry{pn.}{\headword{Zekraeya} \definition{1. male personal name}}
\item \entry{pn.}{\headword{Zelma} \definition{1. female personal name}}
\item \entry{pn.}{\headword{Zen} \definition{1. male personal name}}
\item \entry{pn.}{\headword{Zeriko} \definition{1. Jericho}}
\item \entry{pn.}{\headword{Zina} \definition{1. female personal name}}
\item \entry{pn.}{\headword{Zo} \definition{1. male personal name}}
\item \entry{pn.}{\headword{Zonas} \definition{1. male personal name}}
\item \entry{pn.}{\headword{Zosep} \definition{1. male personal name}}
\item \entry{pn.}{\headword{Zudas} \definition{1. male personal name}}
\item \entry{pn.}{\headword{Zudiya} \definition{1. Judea}}
\item \entry{pn.}{\headword{Zugu} \definition{1. male personal name}}
\item \entry{pn.}{\headword{Zuli} \definition{1. female personal name}}
\item \entry{pn.}{\headword{Zurusalem} \definition{1. Jerusalem}}
\item \entry{pn.}{\headword{Caso} \definition{1. male personal name}}
\item \entry{pn.}{\headword{Cathy} \definition{1. female personal name}}
\item \entry{pn.}{\headword{Charles} \definition{1. male personal name}}
\item \entry{pn.}{\headword{Christina} \definition{1. female personal name}}
\item \entry{pn.}{\headword{Felix} \definition{1. male personal name}}
\item \entry{pn.}{\headword{Flora} \definition{1. female personal name}}
\item \entry{pn.}{\headword{Frank} \definition{1. male personal name}}
\item \entry{pn.}{\headword{Francis} \definition{1. male personal name}}
\item \entry{pn.}{\headword{Jae} \definition{1. male personal name}}
\item \entry{pn.}{\headword{Jamilah} \definition{1. female personal name}}
\item \entry{pn.}{\headword{Jane} \definition{1. female personal name}}
\item \entry{pn.}{\headword{Jack} \definition{1. male personal name}}
\item \entry{pn.}{\headword{Jackae} \definition{1. male personal name}}
\item \entry{pn.}{\headword{Jacklin} \definition{1. female personal name}}
\item \entry{pn.}{\headword{Jeiks} \definition{1. male personal name}}
\item \entry{pn.}{\headword{Jeimi} \definition{1. male personal name}}
\item \entry{pn.}{\headword{Jemila} \definition{1. female personal name}}
\item \entry{pn.}{\headword{Jen} \definition{1. male personal name}}
\item \entry{pn.}{\headword{Jeped} \definition{1. male personal name}}
\item \entry{pn.}{\headword{Jerry} \definition{1. male personal name}}
\item \entry{pn.}{\headword{Jeff} \definition{1. male personal name}}
\item \entry{pn.}{\headword{Jo} \definition{1. male personal name}}
\item \entry{pn.}{\headword{Joanna} \definition{1. female personal name}}
\item \entry{pn.}{\headword{Joden} \definition{1. male personal name}}
\item \entry{pn.}{\headword{Joe-noh} \definition{1. male personal name}}
\item \entry{pn.}{\headword{Joebert} \definition{1. male personal name}}
\item \entry{pn.}{\headword{John} \definition{1. male personal name}}
\item \entry{pn.}{\headword{Johnny} \definition{1. male personal name}}
\item \entry{pn.}{\headword{Jonathan} \definition{1. male personal name}}
\item \entry{pn.}{\headword{Jordan} \definition{1. male personal name}}
\item \entry{pn.}{\headword{Joshua} \definition{1. male personal name}}
\item \entry{pn.}{\headword{Jowanang} \definition{1. male personal name}}
\item \entry{pn.}{\headword{Joy-Lin} \definition{1. female personal name}}
\item \entry{pn.}{\headword{Joys} \definition{1. female personal name}}
\item \entry{pn.}{\headword{Jubli} \definition{1. male personal name}}
\item \entry{pn.}{\headword{Jugu} \definition{1. male personal name}}
\item \entry{pn.}{\headword{Julia} \definition{1. female personal name}}
\item \entry{pn.}{\headword{Julienne} \definition{1. female personal name}}
\item \entry{pn.}{\headword{Junior} \definition{1. male personal name}}
\item \entry{pn.}{\headword{Queenie} \definition{1. female personal name}}
\item \entry{pn.}{\headword{Quin} \definition{1. male personal name}}
\item \entry{pn.}{\headword{Quinten} \definition{1. male personal name}}
\item \entry{pn.}{\headword{Quinteth} \definition{1. female personal name}}
\item \entry{pn.}{\headword{Quinton} \definition{1. male personal name}}
\item \entry{pn.}{\headword{Vanessa} \definition{1. female personal name}}
\item \entry{pn.}{\headword{Vincent} \definition{1. male personal name}}
\item \entry{pn.}{\headword{Victoria} \definition{1. female personal name}}
\end{enumerate}

\section{post.}
\begin{enumerate}
\item \entry{post.}{\headword{kusi} \definition{1. through}}
\item \entry{post.}{\headword{pallall} \definition{1. about}}
\end{enumerate}

\section{prep.}
\begin{enumerate}
\item \entry{prep.}{\headword{do1} \definition{1. to, until}}
\end{enumerate}

\section{pron.}
\begin{enumerate}
\item \entry{pron.}{\headword{ami} \definition{1. whoever, some people (nonsingular existential pronoun, nominative form)}}
\item \entry{pron.}{\headword{ami} \definition{1. possessive form of ami}}
\item \entry{pron.}{\headword{ami} \definition{1. comitative form of ami}}
\item \entry{pron.}{\headword{ami} \definition{1. comitative form of ami}}
\item \entry{pron.}{\headword{ami} \definition{1. accusative form of ami}}
\item \entry{pron.}{\headword{ami} \definition{1. dative form of ami}}
\item \entry{pron.}{\headword{any1} \definition{1. something}}
\item \entry{pron.}{\headword{aya} \definition{1. whoever, someone (singular existential pronoun, nominative form)}}
\item \entry{pron.}{\headword{aya} \definition{1. ablative-possessive form of aya}}
\item \entry{pron.}{\headword{aya} \definition{1. dative form of aya}}
\item \entry{pron.}{\headword{aya} \definition{1. possessive form of aya}}
\item \entry{pron.}{\headword{aya} \definition{1. accusative form of aya}}
\item \entry{pron.}{\headword{ddob} \definition{1. some, others}}
\item \entry{pron.}{\headword{ddone} \definition{1. nothing}}
\item \entry{pron.}{\headword{ddone} \definition{1. no one, anyone (nominative form)}}
\item \entry{pron.}{\headword{ddone} \definition{1. accusative form of ddone aya}}
\item \entry{pron.}{\headword{ddone} \definition{1. very, a lot (antiphrasis)}}
\item \entry{pron.}{\headword{ili} \definition{1. ablative form of ili}}
\end{enumerate}

\section{property n.}
\begin{enumerate}
\item \entry{property n.}{\headword{käg2} \definition{1. vague container-like thing}}
\item \entry{property n.}{\headword{käg2} \definition{1. container}}
\item \entry{property n.}{\headword{käg2} \definition{1. container for squeezing sago}}
\item \entry{property n.}{\headword{käkan} \definition{1. vague amorphous and/or liquid thing}}
\item \entry{property n.}{\headword{källa} \definition{1. vague amorphous and/or soft thing, lump (e.g. cloud, fish tail, lump of earwax)}}
\item \entry{property n.}{\headword{käp1} \definition{1. vague relatively small and round thing}}
\item \entry{property n.}{\headword{kätt} \definition{1. vague hard and thin thing (e.g. a blade)}}
\item \entry{property n.}{\headword{kutt} \definition{1. vague hard thing}}
\end{enumerate}

\section{ptcl.}
\begin{enumerate}
\item \entry{ptcl.}{\headword{ada} \definition{1. quotative particle (for quoted/direct speech)}}
\item \entry{ptcl.}{\headword{dade3} \definition{1. ever}}
\item \entry{ptcl.}{\headword{=o} \definition{1. vocative marker}}
\end{enumerate}

\section{ptcp.}
\begin{enumerate}
\item \entry{ptcp.}{\headword{=att} \definition{1. past participle clitic}}
\end{enumerate}

\section{quant.}
\begin{enumerate}
\item \entry{quant.}{\headword{ai1} \definition{1. plenty, abundant}}
\item \entry{quant.}{\headword{aoli} \definition{1. multiple, several (i.e. more than one but not necessarily many); few (esp. with the restrictive clitic)}}
\item \entry{quant.}{\headword{aoli} \definition{1. few}}
\item \entry{quant.}{\headword{apte} \definition{1. one side, one end, one half, one (of a pair)}}
\item \entry{quant.}{\headword{bällam} \definition{1. every}}
\item \entry{quant.}{\headword{diban1} \definition{1. another, young people now say däba.}}
\item \entry{quant.}{\headword{ddob} \definition{1. some}}
\item \entry{quant.}{\headword{gollaeb} \definition{1. many}}
\item \entry{quant.}{\headword{gudae2} \definition{1. plenty}}
\item \entry{quant.}{\headword{kälae} \definition{1. little, few}}
\item \entry{quant.}{\headword{kemibi} \definition{1. many}}
\item \entry{quant.}{\headword{mullaemullae2} \definition{1. every}}
\item \entry{quant.}{\headword{tämamae} \definition{1. all, every}}
\item \entry{quant.}{\headword{ttongo1} \definition{1. few}}
\item \entry{quant.}{\headword{ulle} \definition{1. a lot, abundant, plentiful (for uncountable nouns)}}
\item \entry{quant.}{\headword{yuwog} \definition{1. many, a lot of (for countable nouns)}}
\end{enumerate}

\section{rel. pron.}
\begin{enumerate}
\item \entry{rel. pron.}{\headword{ami} \definition{1. who (nonsingular relative pronoun, nominative form)}}
\item \entry{rel. pron.}{\headword{ami} \definition{1. possessive form of ami}}
\item \entry{rel. pron.}{\headword{ami} \definition{1. comitative of ami}}
\item \entry{rel. pron.}{\headword{ami} \definition{1. comitative form of ami}}
\item \entry{rel. pron.}{\headword{ami} \definition{1. accusative form of ami}}
\item \entry{rel. pron.}{\headword{ami} \definition{1. dative form of ami}}
\item \entry{rel. pron.}{\headword{aya} \definition{1. who (singular relative pronoun, nominative form)}}
\item \entry{rel. pron.}{\headword{aya} \definition{1. ablative-possessive form of aya}}
\item \entry{rel. pron.}{\headword{aya} \definition{1. dative form of aya}}
\item \entry{rel. pron.}{\headword{aya} \definition{1. possessive form of aya}}
\item \entry{rel. pron.}{\headword{aya} \definition{1. accusative form of aya}}
\item \entry{rel. pron.}{\headword{e2} \definition{1. which, what, that}}
\item \entry{rel. pron.}{\headword{e2} \definition{1. which, that, who}}
\item \entry{rel. pron.}{\headword{e2} \definition{1. locative form of era; where, in which}}
\item \entry{rel. pron.}{\headword{e2} \definition{1. accusative form of era}}
\item \entry{rel. pron.}{\headword{e2} \definition{1. where}}
\item \entry{rel. pron.}{\headword{e2} \definition{1. ablative form of ero}}
\item \entry{rel. pron.}{\headword{e2} \definition{1. allative form of ero}}
\item \entry{rel. pron.}{\headword{e2} \definition{1. why}}
\item \entry{rel. pron.}{\headword{ili} \definition{1. where}}
\end{enumerate}

\section{subr.}
\begin{enumerate}
\item \entry{subr.}{\headword{ada} \definition{1. because (introduces a cause or reason)}}
\item \entry{subr.}{\headword{bänga} \definition{1. though}}
\item \entry{subr.}{\headword{da1} \definition{1. if, when (introduces a condition)}}
\item \entry{subr.}{\headword{ede} \definition{1. so, therefore}}
\item \entry{subr.}{\headword{oba2} \definition{1. so that}}
\end{enumerate}

\section{v.}
\begin{enumerate}
\item \entry{v.}{\headword{anu} \definition{1. sleep (command given to babies)}}
\item \entry{v.}{\headword{bäblem} \definition{1. shiver}}
\item \entry{v.}{\headword{bällamb} \definition{1. ambush}}
\item \entry{v.}{\headword{daram} \definition{1. shine_brightly}}
\item \entry{v.}{\headword{eka malam} \definition{1. obedient}}
\item \entry{v.}{\headword{erongg} \definition{1. to start weaving (crossing strings)}}
\item \entry{v.}{\headword{ittall} \definition{1. to hang}}
\item \entry{v.}{\headword{kakakän} \definition{1. float}}
\item \entry{v.}{\headword{kolldab} \definition{1. stab}}
\item \entry{v.}{\headword{mänymäny} \definition{1. to vomit}}
\item \entry{v.}{\headword{ngälladab} \definition{1. conceive}}
\item \entry{v.}{\headword{opop} \definition{1. tangle}}
\item \entry{v.}{\headword{opop} \definition{1. wind}}
\item \entry{v.}{\headword{pätangeny} \definition{1. splash}}
\item \entry{v.}{\headword{pätaräll} \definition{1. splash}}
\item \entry{v.}{\headword{pitatep} \definition{1. to lift an animal or person and throw them to the ground, hit them hard against the ground}}
\item \entry{v.}{\headword{pitatep} \definition{1. to tackle someone, as in rugby}}
\item \entry{v.}{\headword{pllaengän} \definition{1. strike}}
\item \entry{v.}{\headword{popllo} \definition{1. twist}}
\item \entry{v.}{\headword{saem} \definition{1. to babble}}
\item \entry{v.}{\headword{täp2} \definition{1. interrupt}}
\item \entry{v.}{\headword{tärpan} \definition{1. cut}}
\item \entry{v.}{\headword{totoe} \definition{1. conceive}}
\item \entry{v.}{\headword{tratre} \definition{1. hollow}}
\item \entry{v.}{\headword{trimpeg} \definition{1. slide}}
\item \entry{v.}{\headword{ttallängg} \definition{1. defend}}
\item \entry{v.}{\headword{ttang käpän} \definition{1. to snap one's fingers}}
\item \entry{v.}{\headword{walläg} \definition{1. fan}}
\item \entry{v.}{\headword{wallwall} \definition{1. yawn}}
\end{enumerate}

\section{v. cl.}
\begin{enumerate}
\item \entry{v. cl.}{\headword{=ma} \definition{1. nominalizer}}
\item \entry{v. cl.}{\headword{=me} \definition{1. during, while}}
\end{enumerate}

